\input /home/harthong/tex/formats/twelvea4.tex
\input /home/harthong/tex/formats/epsf.tex

\auteurcourant={\sl J. Harthong: probabilit\'es et statistique}
\titrecourant={\sl Bibliographie g\'en\'erale}


\pageno=465 
\null\vskip10mm 
 
\centerline{\tit BIBLIOGRAPHIE G\'EN\'ERALE} 
\vskip12mm 
\centerline{\tit 1.} 
\medskip 
\line{\hskip5pt Ouvrages de r\'ef\'erence sur le Calcul 
des probabilit\'es et la Statistique: \hfill }   
\vskip7mm 
\vbox{\hbox{{\bf William Feller}}\hbox{\hskip30pt\it An  
Introduction to Probability Theory and its Applications.} 
\smallskip 
\hbox{\hskip42pt John Wiley \& Sons, New York\hskip10pt $1^{\rm 
st}$ ed. {\oldstyle1950}.} } 
\bigskip\filbreak 
 
\vbox{\hbox{{\bf Maurice G. Kendall, Alan Stuart}}\hbox{\hskip30pt{\it 
The Advanced Theory of Statistics.} \hskip7pt (3 vol.)} 
\smallskip 
\hbox{\hskip42pt Charles Griffin \& co ltd., London\hskip10pt  
  {\oldstyle1976}.  }   } 
\bigskip\filbreak 
 
\vbox{\hbox{{\bf Alfred R\'enyi }}\hbox{\hskip30pt\it Calcul  
des probabilit\'es.} 
\smallskip 
\hbox{\hskip42pt Dunod, Paris\hskip10pt $1^{\rm  
re}$ \'ed.  {\oldstyle1966}; r\'e\'ed. J. Gabay  {\oldstyle1992}.  } } 
\bigskip\filbreak 
 
\vbox{\hbox{{\bf B. L. van der Waerden }}\hbox{\hskip30pt\it  
Mathematische Statistik.} 
\smallskip 
\hbox{\hskip42pt Springer-Verlag, 
Berlin\hskip10pt $1^{\rm st}$ ed. {\oldstyle1957}.} } 
\bigskip\filbreak 
 
\vskip12mm plus5mm minus4mm 
\centerline{\tit 2.} 
\medskip 
\line{\hskip10pt  Autres ouvrages de r\'ef\'erence sur les 
diff\'erents th\`emes abord\'es:\hfill }  
\vskip7mm 
 
\vbox{\hbox{{\bf Joseph Bertrand}} 
\smallskip 
\hbox{\hskip30pt\it Calcul des probabilit\'es.}  
\smallskip 
\hbox{\hskip42pt Gauthier-Villars, Paris \hskip10pt $1^{\rm re}$ \'ed.  
{\oldstyle 1888}} }  
\bigskip\filbreak 
 
\vbox{\hbox{{\bf \'Emile Borel}}
\smallskip 
\hbox{\hskip30pt\it Le hasard} 
\smallskip 
\hbox{\hskip42pt Librairie F\'elix Alcan, Paris \hskip10pt $1^{\rm re}$  
\'ed. {\oldstyle 1914}} } 
\bigskip\filbreak 
 
\vbox{\hbox{{\bf Jorge Luis Borges}}
\smallskip 
\hbox{\hskip30pt\it Fictions.} 
\hbox{\hskip55pt(trad. fran\c caise de P. Verdevoye, Ibarra, et R. 
Caillois)}  
\smallskip 
\hbox{\hskip42pt coll. folio, Gallimard, Paris \hskip10pt {\oldstyle 1989} 
($1^{\rm  re}$ \'ed.  {\oldstyle1957})} }  
\bigskip\filbreak 
 
\vbox{\hbox{{\bf Jean-Paul Delahaye}}
\smallskip 
\hbox{\hskip30pt\it Information,
complexit\'e, et hasard.} 
\smallskip 
\hbox{\hskip42pt \'Editions Herm\`es, Paris \hskip10pt {\oldstyle 1994} } }  
\bigskip\filbreak 
 
\vbox{\hbox{{\bf Richard P. Feynman}} 
\smallskip 
\hbox{{\hskip30pt\it Lectures  on  Physics}\hskip15pt tome 3 
{\hskip5pt\it M\'ecanique quantique.}} 
\hbox{\hskip55pt(trad. fran\c caise de B. Equer et P. Fleury)}  
\smallskip 
\hbox{\hskip42pt Bilingua Addison-Wesley (version bilingue)\hskip10pt  
$1^{\rm st}$ ed. {\oldstyle 1970}.} 
\hbox{\hskip42pt Inter\'editions, Paris (version fran\c caise seulement) 
\hskip10pt  {\oldstyle 1979}.} } 
\bigskip\filbreak 
 
\vbox{\hbox{{\bf David Hume}}
\smallskip 
\hbox{\hskip30pt\it Enqu\^ete sur l'entendement humain.}  
\hbox{\hskip55pt(trad. fran\c caise d'Andr\'e Leroy)} 
\smallskip 
\hbox{\hskip42pt Flammarion, Paris {\oldstyle 1983}\hskip10pt  
($1^{\rm re}$ \'ed. Aubier-Montaigne {\oldstyle 1947}).} } 
\bigskip\filbreak 
 
\vbox{\hbox{{\bf Donald Knuth}} 
\smallskip 
\hbox{\hskip30pt\it The Art of Computer Programming.}  
\hbox{\hskip55pt Vol 2, {\it Seminumerical Algorithms}} 
\smallskip 
\hbox{\hskip42pt Addison-Wesley {\oldstyle 1969}.} } 
\bigskip\filbreak 
 
\vbox{\hbox{{\bf G. Lenoir}}
\smallskip 
\hbox{\hskip30pt\it La mucoviscidose.}  
\smallskip 
\hbox{\hskip42pt Doin, Paris, 1935} } 
\bigskip\filbreak 
 
\vbox{\hbox{{\bf Jean Marc Levy-Leblond, Fran\c coise Balibar}}  
\smallskip 
\hbox{\hskip30pt\it Quantique, rudiments. }  
\smallskip 
\hbox{\hskip42pt Inter\'editions, 
Paris\hskip10pt $1^{\rm re}$ \'ed. {\oldstyle1984}.} } 
\bigskip\filbreak 
 
\vbox{\hbox{{\bf Eug\`ene Lukacs}}
\smallskip 
\hbox{\hskip30pt\it Fonctions caract\'eristiques.}  
\hbox{\hskip55pt(trad. fran\c caise de A. Rosengard)} 
\smallskip 
\hbox{\hskip42pt Dunod, Paris \hskip10pt $1^{\rm re}$ \'ed.  
{\oldstyle1964}} }  
\bigskip\filbreak 
 
\vbox{\hbox{{\bf Fran\c cois Lur\c cat}}
\smallskip 
\hbox{\hskip30pt\it Niels Bohr et la Physique quantique.} 
\smallskip 
\hbox{\hskip42pt \'Ed. du Seuil,  Paris,  \hskip10pt   
{\oldstyle 2001} \hskip10pt  ($2^{\rm re}$ \'ed.)} 
\medskip 
\hbox{\hskip30pt$1^{\rm re}$ \'ed. : {\it Niels Bohr},  
\'Ed. Crit\'erion, Paris, \hskip10pt   {\oldstyle 1990}} } 
\bigskip\filbreak 
 
\vbox{\hbox{{\bf Gustave Mal\'ecot}}
\smallskip 
\hbox{\hskip30pt\it Les math\'ematiques de l'h\'er\'edit\'e.}  
\smallskip 
\hbox{\hskip42pt Masson, Paris \hskip10pt $1^{\rm re}$ \'ed. {\oldstyle 
1948}} }  
\bigskip\filbreak 
 
\vbox{\hbox{{\bf Edward Nelson}}
\smallskip 
\hbox{\hskip30pt\it Radically Elementary Probability Theory.}  
\smallskip 
\hbox{\hskip42pt Princeton University Press, Princeton \hskip10pt 
$1^{\rm st}$ ed. {\oldstyle1989}} }  
\bigskip\filbreak 
 
\vbox{\hbox{{\bf Platon}}
\smallskip 
\hbox{\hskip30pt\it La R\'epublique.}  
\hbox{\hskip55pt(trad. fran\c caise de \'Emile Chambry)} 
\smallskip 
\hbox{\hskip42pt Les Belles Lettres, Paris \hskip10pt {\oldstyle1989} 
($1^{\rm  re}$ \'ed.  {\oldstyle1933}).} } 
\bigskip\filbreak 
 
\vbox{\hbox{{\bf Henri Poincar\'e}}
\smallskip 
\hbox{\hskip30pt\it Calcul des probabilit\'es.}  
\smallskip 
\hbox{\hskip42pt Gauthier-Villars, Paris \hskip10pt $2^{\rm re}$ \'ed.  
{\oldstyle 1912} ($1^{\rm re}$ \'ed. {\oldstyle 1903})}  
\smallskip 
\hbox{\hskip42pt R\'e\'edition Jacques Gabay, Paris, {\oldstyle 1987})}  
\medskip
\hbox{\hskip30pt\it La valeur de la science.}  
\smallskip 
\hbox{\hskip42pt Ernest Flammarion, Paris \hskip10pt $1^{\rm re}$ \'ed.  
{\oldstyle 1910}} }  
\bigskip\filbreak 
 
\vbox{\hbox{{\bf Frederick Reif}} 
\smallskip 
\hbox{{\hskip30pt\it Cours de Physique de Berkeley.} \hskip5pt tome 5  
Physique statistique. }   
\hbox{\hskip55pt(trad. fran\c caise de Pierre Turon)} 
\smallskip 
\hbox{\hskip42pt Armand Colin, 
Paris\hskip10pt $1^{\rm re}$ \'ed. {\oldstyle 1972}.} } 
\bigskip\filbreak 
 
\vskip12mm plus5mm minus4mm 
\centerline{\tit 3.} 
\medskip 
\line{\hskip53pt  Articles originaux cit\'es dans le pr\'esent 
ouvrage\hfill }   
\line{\hskip53pt (sauf chapitre {\bf XIV} qui a sa propre bibliographie):
\hfill }   
\vskip7mm 
 
\vbox{\hbox{{\bf J. S. Bell}} 
\hbox{\hskip30pt\it On the Einstein Podolski Rosen Paradox.}   
\smallskip   
\hbox{{\hskip42pt\sl Physics},  vol. {\bf 1} ({\oldstyle 1964}) pages 
195 --  200.} } 
\bigskip\filbreak 
 
\vbox{\hbox{{\bf Niels Bohr}} 
\hbox{\hskip30pt\it Das Quantenpostulat und die neuere Entwicklung der 
Atomistik.}    
\smallskip   
\hbox{{\hskip42pt\sl Naturwissenschaften},  vol. {\bf 16} ({\oldstyle 
1928}) pages  245 -- 271.} }  
\bigskip\filbreak 
 
\vbox{\hbox{{\bf Niels Bohr}} 
\hbox{\hskip30pt\it Physique atomique et connaissance humaine.}    
\hbox{\hskip30pt(recueil d'articles)}    
\smallskip   
\hbox{\hskip42pt \'Ed. Gauthier-Villars,  Paris,  {\oldstyle 
1972} } }  
\bigskip\filbreak 
 
\vbox{\hbox{{\bf \'Emile Borel}} 
\hbox{\hskip30pt\it Les probabilit\'es d\'enombrables et leurs  
applications arithm\'etiques.}   
\smallskip   
\hbox{\hskip42pt\sl Rendiconti del Circolo Matematico di Palermo}  
\hbox{\hskip42pt vol. {\bf 27} ({\oldstyle 1909}) pages  247 -- 270.} }  
\bigskip\filbreak 
 
\vbox{\hbox{{\bf \'Emile Borel}} 
\hbox{\hskip30pt\it Emploi du th\'eor\`eme de Bernoulli pour le calcul 
d'une infinit\'e de } 
\hbox{\hskip30pt\it coefficients. Application au probl\`eme de l'attente 
\`a un guichet.}    
\smallskip   
\hbox{\hskip42pt{\sl Comptes-rendus de l'Acad\'emie des Sciences}, 
Paris, mars {\oldstyle 1942}}   } 
\bigskip\filbreak 

\vbox{\hbox{{\bf John F. Clauser and Abner Shimony}} 
\hbox{\hskip30pt\it Bell's Theorem: experimental tests and implications.}   
\smallskip   
\hbox{\hskip42pt\sl Reports on Progress in Physics, vol. {\bf 41}, 
{\oldstyle 1978} pages 1181 -- 1927.} }  
\bigskip\filbreak 
 
\vbox{\hbox{{\bf A. Einstein, B. Podolski, N.Rosen}} 
\hbox{\hskip30pt\it 
Can Quantum-Mechanical Description of Physical Reality Be} 
\hbox{\hskip30pt\it Considered Complete?}   
\smallskip   
\hbox{{\hskip42pt\sl Physical Review}, vol. {\bf 47} ({\oldstyle 1935})  
pages   777 -- 780.} }  
\bigskip\filbreak 
 
\vbox{\hbox{{\bf Werner Heisenberg}} 
\hbox{\hskip30pt\it \"Uber den anschaulichen Inhalt der quantentheoretischen} 
\hbox{\hskip30pt\it Kinematik und Mechanik.}   
\smallskip   
\hbox{{\hskip42pt\sl Zeitschrift f\" ur Physik},  vol. {\bf 43}  
({\oldstyle 1927}) 
pages  172 -- 198.} }  
\bigskip\filbreak 
 
\vbox{\hbox{{\bf Max Planck}}\hbox{\hskip30pt\it \"Uber eine 
Verbesserung der Wienschen Spectralgleichung.}  
\smallskip   
\hbox{\hskip42pt\sl Verhandlungen der Deutschen Physikalischen Gesellschaft} 
\hbox{\hskip42pt Band {\bf 2} ({\oldstyle 1900}) pages  237 -- 245.} } 
\bigskip\filbreak 
 
\vbox{\hbox{{\bf Max Planck}}\hbox{\hskip30pt\it \"Uber das 
Gesetz der  Energieverteilung im  Normalspectrum.}  
\smallskip   
\hbox{\hskip42pt\sl Annalen der Physik} 
\hbox{\hskip42pt vol. {\bf 4} ({\oldstyle 1901}) pages  553 -- 563.} } 
\vfill 
 
 
 
\end 
 
