\input /home/harthong/tex/formats/twelvea4.tex
\input /home/harthong/tex/formats/epsf.tex



\auteurcourant={\sl J. Harthong: probabilit\'es et statistique}
\titrecourant={\sl Processus en cascade}

\pageno=177

\def\hfq{\hfill\quad}
\def\cc#1{\hfill#1\quad\hfill}
\def\tv{\vrule height 30pt depth 6pt width0.4pt}
\def\punkt{\vrule height0.4pt depth0pt width0.4pt}
\def\L{{\bb L}} 

\def\ara{\hskip-1.5pt}
\def\asa{\hskip-2pt}
\def\ata{\hskip-2.5pt}
\def\aua{\hskip-1pt}
\def\arb{\hskip1.5pt}
\def\asb{\hskip2pt}
\def\atb{\hskip2.5pt}
\def\aub{\hskip1pt}
\def\xsp{\hskip3pt plus3pt minus2pt} 

\null\vskip 10mm plus1mm minus1mm

\centerline {\tit VIII. LES PROCESSUS EN CASCADE.} 
\medskip 
\centerline {\tit (Une application des fonctions g\'en\'eratrices.)} 

\vskip10mm plus1mm minus1mm

{\narrower\eightpoint\sl Ce chapitre est plus technique que les autres 
et convient plut\^ot aux math\'e\-ma\-ti\-ciens (et \`a condition qu'ils ne
soient pas trop press\'es). On pourra le
consid\'erer comme une suite deprobl\`emes corrig\'es. C'est 
en quelque sorte un chapitre de travaux dirig\'es. Une lecture purement
passive est d\'econseill\'ee: le lecteur n'entirera vraiment profit que s'il  se munit d'un crayon et d'un papier, et 
effectue lui-m\^eme les
calculs pr\'esent\'es ici. 
\medskip 
Ce chapitre peut \^etre omis en premi\`ere lecture, mais on conseille
cependant les probl\`emes {\bf 1} et {\bf 2} du d\'ebut, qui sont
\'el\'ementaires,  et beaucoup plus faciles que ce qui suit. \par}

\vskip7mm plus1mm minus1mm

{\bf Probl\`eme 1.}  Deux joueurs ${\cal A}$ et ${\cal B}$ jouent au jeu 
de {\it pile ou face} de la mani\`ere suivante:  ${\cal A}$ donne $1$ euro
\`a ${\cal B}$ chaque fois que {\it face} sort et ${\cal  B}$ donne $1$
euro \`a ${\cal A}$ chaque fois que {\it pile} sort.  La pi\`ece est 
lanc\'ee $n$ fois de suite.
Trouver la loi de probabilit\'e du gain de ${\cal A}$.
\medskip
Ce probl\`eme a d\'ej\`a \'et\'e trait\'e sous forme d\'eguis\'ee au 
chapitre {\bf  III} (marches al\'eatoires), d\'ebut du {\bf \S 2}. Le gain
peut prendre les valeurs $-n,\, -n + 2,\, -n + 4,\, -n + 6,\; \ldots \;
n - 2,\,
n$, et les probabilit\'es correspondantes sont 
$${\cal P}\, (\hbox{gain} = -n + 2k) = 2^{-n}\; {n\choose k} \simeq
{1\over\sdown{15}\sqrt{\sdown{9} \pi n}}\;
\e^{-{(2k-n)^2\strup{2.5}\over\sdown{5} 2n}}$$ 
Nous allons le retrouver en utilisant les fonctions g\'en\'eratrices.
\medskip
L'id\'ee est simple: nous avons vu que la fonction g\'en\'eratrice de la 
somme de plusieurs variables al\'eatoires ind\'ependantes est le produit
des fonctions g\'en\'eratrices. Nous allons donc d\'ecomposer le gain en 
une somme de variables al\'eatoires ind\'ependantes.
\medskip
Il est \'evident par d\'efinition m\^eme que le gain global sur les $n$ 
lancers est la somme alg\'ebrique des gains obtenus pour chacun des $n$
lancers.  Or pour chaque lancer ce gain est \'egal (en euros) \`a $+1$
ou
$-1$,  selon que {\it face} ou {\it pile} est sorti.  Le gain au$j^{\rm e}$
lancer est donc une variable al\'eatoire $X_j$ dont la loi est
$$X_j = \cases{ +1\quad \bigl( {1\over 2} \bigr) \cr
                             \noalign{\medskip}
                         -1\quad \bigl( {1\over 2} \bigr) \cr }$$
Le gain global est alors la somme $S = X_1 + X_2 +  \cdots  + X_n$.
L'ind\'ependance stochastique des variables al\'eatoires $X_j$ est le 
reflet de l'ind\'ependance causale entre les lancers: dire que $X_j$ est
ind\'ependante de $X_k$ signifie que les probabilit\'es de gagner ou 
perdre $1$ euro au $k^{\rm e}$ lancer ne d\'ependent pas du r\'esultat
obtenu au $j^{\rm e}$ lancer.
\medskip
{\eightpoint {\bf Remarque.} L'ind\'ependance stochastique entre les 
$X_j$ est d\'eduite directement des circonstances physiques du
probl\`eme; mais bien entendu, ces circonstances se refl\`etent aussi 
dans la structure de l'espace des \'epreuves $\Omega$, qui, comme cela
a \'et\'e expliqu\'e au chapitre {\bf IV}, doit \^etre un produit. $\Omega$
est ici l'ensemble des mots de $n$ lettres qu'on peut \'ecrire avec
l'alphabet  $\{ P, F \}$, et s'identifie au produit cart\'esien $\{ P, F \}
\times   \{ P, F\} \times \cdots  \times \{ P, F \}$ ($n$ fois). En tant
qu'{\it application} de $\Omega$ dans \R\ (ce qui est, rappelons-le, la
d\'efinition math\'ematique d'une variable al\'eatoire), $X_j$ est
l'application qui, \`a une \'epreuve $\omega$ qui est un mot de $n$ 
lettres, fait correspondre $+1$ si la $j^{\rm e}$ lettre de $\omega$
est $F$, et $-1$ si la $j^{\rm e}$ lettre de $\omega$ est $P$. Lorsque
l'espace $\Omega$ est consid\'er\'e comme le produit cart\'esien de $n$
fois $\{ P, F \}$, $X_j$ ne d\'epend donc que de la $j^{\rm e}$
coordonn\'ee, ce qui est exactement l'expression math\'ematique de
l'ind\'ependance stochastique.}
\medskip
La fonction g\'en\'eratrice commune des $X_j$ est donc
$$G_X (z) = {1\over 2} \Bigl( z + {1\over\ldown{z}} \Bigr)$$
et puisque les $X_j$ sont ind\'ependantes, leur somme $S$ a pour
fonction g\'en\'eratrice
$$G_S (z) = \Bigl[ {1\over 2} \Bigl( z + {1\over\ldown{z}} 
\Bigr) \Bigr]^n = \sum_{k=0}^{k=n}\; {n \choose k}\, z^{n-2k}$$
On constate dans le d\'eveloppement en puissances de $z$ que les
exposants sont $n-2k$, qui sont les valeurs prises par $S$; quant aux
coefficients correspondants, ils co{\"\i}ncident bien avec le r\'esultat
d\'ej\`a connu.
\medskip
Nous voyons ici sur un exemple tr\`es simple comment fonctionne la 
{\it m\'ethode des fonctions g\'en\'eratrices}:  on commence par 
calculer la fonction g\'en\'eratrice,  qui est un polyn\^ome ou \`a la
limite une fonction analytique de la variable complexe $z$.  Puis un
d\'eveloppement en puissances de $z$ met en \'evidence les exposants
de $z$ qui sont les valeurs de la variable,  ainsi que les coefficients 
qui sont les probabilit\'es correspondantes.
\bigskip
{\bf Probl\`eme 2.}  Les deux joueurs ${\cal A}$ et ${\cal B}$ jouent
maintenant au jeu de {\it pile ou face} d'une mani\`ere diff\'erente: 
comme d'habitude ${\cal A}$ donne $1$ euro \`a ${\cal B}$ chaque fois
que {\it face} sort et ${\cal  B}$ donne $1$ euro \`a ${\cal A}$ chaque
fois que {\it pile} sort;  mais au  lieu de  s'arr\^eter apr\`es un nombre 
$n$ de lancers convenu \`a l'avance,  les joueurs lancent en outre un d\'e 
en m\^eme temps que leur pi\`ece, et d\'ecident d'arr\^eter le jeu d\`es
que le d\'e donne six. 
\smallskip
Trouver la loi de probabilit\'e du gain.
\medskip
Lorsque nous avons r\'esolu le probl\`eme pr\'ec\'edent  (loi de 
probabilit\'e du gain pour un nombre fix\'e $n$ de lancers), nous devions
trouver la loi de  probabilit\'e d'une somme $S = X_1 + X_2 + \cdots +
X_n$ de variables al\'eatoires ind\'ependantes. Cette fois nous avons
encore affaire \`a une somme de variables al\'eatoires ind\'ependantes,
mais le nombre de termes  de la somme est lui-m\^eme al\'eatoire.
\smallskip
On peut donc reformuler le probl\`eme sous la forme abstraite que voici:
soient $X_1\, , \; X_2\, , \; X_3 \ldots$ des variables al\'eatoires
ind\'ependantes, {\it toutes de m\^eme loi}, et $Y$ une variable 
al\'eatoire prenant les valeurs $1\, , \; 2\, , \; 3 \ldots$, et {\it
ind\'ependante} des  $X_1\, , \; X_2\, , \; X_3 \ldots$ 
Trouver la loi de la variable 
$$S = X_1 + X_2 + \cdots + X_Y$$
\medskip
{\bf Solution.} D'apr\`es la formule des probabilit\'es conditionnelles
$(IV.4.)$ on peut d\'ecom\-poser ${\cal P}\, (S = k)$ ainsi:
$${\cal P}\, (S = k) = \sum_{n} {\cal P}\, (S = k \mid Y = n) {\cal P}\, 
(Y = n) \eqno (VIII. 1.)$$
Or l'\'ev\'enement $A = \{ X_1 + X_2 + \cdots + X_Y = k \;\hbox{et}\; Y = 
n  \}$  est \'egal  \`a l'\'ev\'enement $\{ X_1 + X_2 + \cdots + X_n = k
\;\hbox{et}\;  Y = n \}$.  En effet toute \'epreuve $\omega$ appartenant
\`a $A$ est une \'epreuve  pour laquelle $Y(\omega ) = n$, donc 
$S(\omega ) = X_1(\omega ) + X_2(\omega )  + \cdots + X_{Y(\omega )}
(\omega ) = S(\omega ) = X_1(\omega ) +  X_2(\omega ) + \cdots + X_n
(\omega )$.  Ces deux \'ev\'enements \'etant \'egaux, leurs probabilit\'es
sont \'egales. En  outre, $Y$ \'etant stochastiquement ind\'ependante des
$X_j$, on peut dire aussi que 
$${\cal P}\, (X_1 + \cdots + X_n = k \;\hbox{et}\; Y =  n)  = {\cal
P}\, (X_1 + \cdots + X_n = k) \times {\cal P}\, (Y = n)$$ 
Mais {\bf on ne peut pas} dire que
$${\cal P}\, (X_1 + \cdots + X_Y = k \;\hbox{et}\; Y = n) =
{\cal P}\, (X_1 + \cdots + X_Y = k) \times {\cal P}\, (Y = n)$$
car dans le premier cas les variables al\'eatoires $X_1 + X_2 + \cdots +
X_n$ et $Y$ sont ind\'ependantes, tandis que dans le second, les 
variables  al\'eatoires $X_1 + X_2 + \cdots + X_Y$ et $Y$ ne sont pas 
ind\'ependantes. 
\medskip
{\eightpoint {\bf Remarque.} Pour bien comprendre cela, il faut
se repr\'esenter (comme toujours en cas d'ind\'ependance stochastique; 
voir \S {\bf V\ara .\aub 3.}) l'espace $\Omega$ comme un tableau ou une
matrice form\'e  de lignes et de colonnes, de sorte que $X_1(\omega ) +
X_2(\omega ) + \cdots + X_n(\omega )$ ne d\'epend que de la ligne o\`u
se trouve $\omega$ et $Y(\omega )$ que de la colonne o\`u se trouve
$\omega$. Les \'ev\'enements $B = \{ X_1 + X_2 + \cdots + X_n = k \}$ et
$C = \{ Y = n \}$ sont respectivement des ensembles de lignes
compl\`etes et des ensembles de colonnes compl\`etes; leur intersection
est donc un rectangle (une sous-matrice) du tableau, et le cardinal de ce
rectangle est bien le produit du nombre de lignes par le nombre de
colonnes. Mais l'\'ev\'enement $E = \{ X_1 + X_2 + \cdots + X_Y   = k \}$
n'est pas une sous-matrice car $ \{ X_1(\omega ) + X_2(\omega ) + \cdots
+ X_{Y(\omega )} (\omega ) = k \}$ d\'epend {\it \`a la fois} de la ligne et
de la colonne o\`u se trouve l'\'el\'ement $\omega$, de sorte que
l'\'ev\'enement $E$ n'est pas  form\'e de lignes compl\`etes; son
intersection avec $C$ n'est donc pas l'intersection d'un ensemble de
lignes  compl\`etes avec un ensemble de colonnes compl\`etes, mais
est pourtant un rectangle, car les morceaux de lignes qui manquent sont 
tous \`a l'ext\'erieur de l'ensemble $C$.} 
\medskip 
On peut donc \'ecrire 
$$\eqalign{ 
{\cal P}\, (S = k &\mid Y = n) = {{\cal P}\, (S = k \;\hbox{et}\; Y = n) 
\over  {\cal P}\, (Y = n)} \cr
&= {{\cal P}\, (X_1 + X_2 + \cdots + X_n = k \;\hbox{et}\; Y = n) \over 
{\cal P}\, (Y = n)} \cr
&= {{\cal P}\, (X_1 + X_2 + \cdots + X_n = k) \times {\cal P}\, (Y = n)
\over   {\cal P}\, (Y = n)} \cr
&= {\cal P}\, (X_1 + X_2 + \cdots + X_n = k) \cr } \eqno (VIII. 2.)$$
Ainsi la probabilit\'e conditionnelle ${\cal P}\, (S = k \mid Y = n)$ est
\'egale \`a la probabilit\'e ${\cal P}\, (X_1 + X_2 + \cdots + X_n = k)$.
En combinant $(VIII.1)$ et $(VIII.2)$, il vient
$${\cal P}\, (S = k) = \sum_n {\cal P}\, (X_1 + X_2 + \cdots + X_n = k)
\times {\cal P}\, (Y = n) \eqno (VIII. 3.)$$
\medskip
Appelons comme dans le probl\`eme pr\'ec\'edent $G_X(z)$ la fonction
g\'en\'e\-ra\-trice commune des $X_j$ et $G_Y(z)$ celle de $Y$. Par 
d\'efinition: 
$$G_S(z) = \sum_{k} {\cal P}\, (S = k) \; z^k$$
En rempla\c{c}ant ${\cal P}\, (S = k)$ par l'expression $(VIII.3)$:
$$G_S(z) = \sum_{k} \sum_n {\cal P}\, (X_1 + X_2 + \cdots + X_n = k)
\times {\cal P}\, (Y = n) \; z^k$$
Puis en inversant l'ordre de sommation entre $k$ et $n$:
$$G_S(z) = \sum_{n} \Bigl[ \sum_k {\cal P}\, (X_1 + X_2 + \cdots + X_n =
k)\; z^k \Bigr] \times {\cal P}\, (Y = n)$$
Mais l'expression entre crochets n'est autre que $G_X(z)^n$, de sorte que
finalement
$$G_S(z) = \sum_{n} \bigl[ G_X(z) \bigr]^n  {\cal P}\, (Y = n)
= G_Y\bigl( G_X(z) \bigr)$$
ou encore
$$G_S = G_Y \circ G_X \eqno (VIII. 4.)$$
\medskip
Pour appliquer cette relation \`a notre probl\`eme, il nous faut
encore conna{\^\i}tre la fonction $G_Y$,  c'est-\`a-dire la loi de $Y$; 
mais c'est imm\'ediat,  puisqu'il s'agit d'un simple lancer de d\'e: 
l'\'ev\'enement $\{ Y = n \}$ est identique \`a l'\'ev\'enement ``le d\'e 
donne $6$ la $n^{\rm e}$ fois,  et $1$, $2$, $3$, $4$, ou $5$ les $n-1$
fois
 pr\'ec\'edentes.  Pour calculer la probabilit\'e de cet \'ev\'enement, 
l'espace $\Omega_n$ correspondant \`a $n$ lancers successifs du d\'e
suffit:  il est form\'e des mots de $n$ lettres \'ecrits avec l'alphabet
$\{
1,2,3,4,5,6 \}$,  donc $\#\Omega = 6^n$.  Les \'el\'ements de$\{ Y = n \}$
sont,  parmi ces mots,  ceux dont la derni\`ere lettreest $6$ et dont les
$n-1$ premi\`eres sont prises dans l'alphabet$\{ 1,2,3,4,5 \}$,  donc
$\#\{ Y = n \} = 5^{n-1}$.  Ainsi
$${\cal P}\, (Y = n) = {5^{n-1} \over 6^n}$$
et 
$$G_Y(z) = \sum_{n \geq 1} {5^{n-1} \over 6^n}\; z^n$$

Si on calcule cette somme de $n = 1$ \`a $n=N$, on obtient 
$$\sum_{n \geq 1}^{n = N} {5^{n-1} \over 6^n}\; z^n =
{z \over 6 - 5z} \cdot \Bigl[ 1 - \Bigl( {5z \over 6}^N\Bigr) \Bigr]$$
Pour pouvoir faire tendre $N$ vers l'infini, il faudrait avoir construit un
espace $\Omega$ lui-m\^eme infini. Mais il est bien clair que si $n$ est
grand, la probabilit\'e pour que le d\'e ne donne jamais $6$ avant la 
$n^{\rm e}$ fois est infinit\'esimale: par exemple pour $n=1000$ cette
probabilit\'e serait $5^{999} / 6^{1000} \simeq 1.318 \cdot 10^{-80}$.
Pour qu'il y ait un sens \`a distinguer cela de z\'ero, il faudrait que
l'\'equiprobabilit\'e des six faces du d\'e soit v\'erifi\'ee \`a $10^{-80}$
pr\`es. Comme il ne peut en \^etre  ainsi, il est absurde de pr\'etendre
pousser la sommation aussi loin. On va donc consid\'erer que la
probabilit\'e pour que $Y$ soit sup\'erieur \`a $1000$ est nulle, et que 
$$G_Y(z) \simeq {z \over 6 - 5z}$$
Par ailleurs $G_X(z)$, la fonction g\'en\'eratrice des $X_j$, est \'egale
\`a ${1\over 2}\bigl( z + {1\over\textdown z}\bigr)$, donc 
d'apr\`es $(VIII.4)$:
$$G_S(z) \simeq  {{1\over 2}\bigl( z + {1\over\textdown z}\bigr) 
\strup{7}\over\sdown{13} 6 - {5\over 2}\bigl( z + {1\over
\textdown z}\bigr)} = {z^2 + 1 \over -5z^2 + 12z - 5} 
\eqno (VIII.5.)$$ 
Si \`a partir de ce r\'esultat on veut retrouver les probabilit\'es
${\cal P}\, (S = k)$, il faut d\'evelopper cette expression en s\'erie
enti\`ere.  Les valeurs de $S$ sont n\'ecessairement enti\`eres, mais
peuvent \^etre aussi bien positives que n\'egatives; il faut donc
rechercher un d\'eveloppement de Laurent (voir cours de math\'ematique).
Pour cela on commence par d\'ecomposer l'expression obtenue en
\'el\'ements  simples. Les racines du d\'enominateur $-5z^2 + 12z - 5$
sont $$ z_1 = {6 - \sqrt{11} \over 5}\quad ; \qquad 
z_2 = {6 + \sqrt{11} \over 5}\quad ; $$
On remarquera que les deux sont r\'eelles positives, et qu'en outre 
$z_1 < 1$ et $z_2 > 1$.
$$\openup 2\jot\eqalignno{
{z^2 + 1 \over -5z^2 + 12z - 5} &= -{1 \over 5} + {\strup{6} {12\over 5}\,
z \over -5z^2 + 12z - 5} \cr
&= -{1 \over 5} + {A \over \sdown{17} {\displaystyle 1 - {z_1 \over
\ldown {z} }  } }\; {z_1 \over z} + {B \over \sdown{17} {\displaystyle 1 -
{z \over  \ldown {z_2}} }  } \cr }$$ 
On obtient sans difficult\'e les nombres $A$ et $B$: 
$$A = B = {6 \over \sdown{14}5\sqrt{\sdown{9.6}11}}$$
Le d\'eveloppement de Laurent de la fonction $(z^2 + 1) / (-5z^2 + 12z -
5)$ r\'esulte alors des s\'eries g\'eom\'etriques de 
$${1 \over \sdown{17} {\displaystyle 1 - {z_1 \over \ldown {z}} }  }
\; {z_1 \over z}\; 
= \;\sum_{k \geq 1}\; \bigg( {z_1 \over \ldown {z}}\bigg)^k$$
et
$${1 \over \sdown{17} {\displaystyle 1 - {z \over \ldown {z_2}} }  }\; 
= \;\sum_{k \geq 0}\; \bigg( {z \over \ldown {z_2}}\bigg)^k$$
c'est-\`a-dire
$$G_S(z) = {1 \over 5}\; \Bigg( {6 \over \sdown{14}\sqrt{\sdown{9.6}11}} - 1 
\Bigg) +\sum_{k \geq 1} {6 \over \sdown{14}5\sqrt{11}} \Bigg( {6 - 
\sqrt{\sdown{9.6}11}  \over 5} \Bigg)^k\; \bigg( z^k + {1 \over z^k}
\bigg)$$ 
On en d\'eduit que
$${\cal P}\, (S = 0) = {1 \over 5}\; \Bigg( {6 \over \sdown{14}
\sqrt{\sdown{9.6}11}} - 1 \Bigg)$$
et pour $k = 1,\, 2,\, 3, \ldots$:
$${\cal P}\, (S = \pm k) = {6 \over 5\sdown{14}\sqrt{\sdown{9.6}11}}\;
\Bigg( {6 - \sqrt{\sdown{9.6}11}  \over 5} \Bigg)^k$$
\bigskip
Le calcul num\'erique donne les valeurs suivantes:
\bigskip
$$\matrix{
&k\qquad &0 &1 &2 &3  \cr
\noalign{\medskip}
&{\cal P}\, (S = \pm k)\quad &0.1618 &0.1942 &0.1042 &0.0559\cr
\noalign{\vskip16pt}
&k\qquad &4 &5 &6 &7 \cr
\noalign{\medskip}
&{\cal P}\, (S = \pm k)\quad &0.0300 &0.01161 &0.0086 &0.0046\cr }$$
\bigskip
Plus g\'en\'eralement, si au lieu d'un d\'e on avait employ\'e un 
proc\'ed\'e quelconque donnant $q$ r\'esultats possibles au lieu de $6$,
de sorte que la loi de $Y$ e\^ut \'et\'e ${\cal P}\, (Y = n) = {1\over
\textdown{q}} \bigl( 1-{1\over\textdown{q}}\bigr)^{n-1}$ au 
lieu de  ${\cal P}\, (Y = n) = {1\over 6} \bigl( {5\over 6}\bigr)^{n-1}$,
alors on aurait obtenu pour la loi de $S$: 
\bigskip
--- si $k \neq 0$:
$${\cal P}\, (S = \pm k) = {q \over \sdown{14} (q-1)\sqrt{\sdown{9.6}2q-1}} 
\Bigg( {q -  \sqrt{\sdown{9.6}2q-1}  \over q-1} \Bigg)^k$$ 
\medskip
--- et si $k = 0$:
$${\cal P}\, (S = 0) = {1 \over (q-1)}\;\Bigg(
{q\over \sdown{14} \sqrt{\sdown{9.6}2q-1}} - 1\Bigg)$$ 

On constate que la r\'epartition des probabilit\'es
entre les diff\'erentes valeurs de la variable $S$ est tr\`es diff\'erente
de ce  qu'on obtenait pour  la somme $\Sigma = X_1 + X_2 + \cdots +
X_n$ d'un nombre {\it fix\'e} de variables al\'eatoires ind\'ependantes:
pour $n$ grand, $\Sigma$  avait une r\'epartition en  $e^{-x^2}$, ce qui
n'est plus du tout le cas pour $S$. Bien entendu, cela se comprend
ais\'ement: la loi en $e^{-x^2}$ provient  de ce que $\Sigma$ est la 
somme d'un {\it grand} nombre de $X_j$; par contre $S$ peut
al\'eatoirement \^etre la somme d'un grand nombre ou  d'un petit nombre 
de $X_j$. Si $q$ est petit (par exemple $q=6$ comme dans le cas du 
d\'e), il est de toute fa\c{c}on peu probable que $Y$ puisse prendre de
grandes valeurs; mais m\^eme si $q$ est tr\`es grand, il y a une
probabilit\'e non n\'egligeable pour que la somme $S$ soit la somme d'un
nombre trop petit de $X_j$, en tous cas trop petit pour donner une loi de
la forme $e^{-x^2}$. 
\bigskip
{\bf Probl\`eme 3: r\'eaction en cha{\^\i}ne.} Dans la fission nucl\'eaire, 
un neutron casse un noyau d'$U^{235}$ en lib\'erant deux ou trois
nouveaux neutrons, qui \`a leur tour reproduiront chacun l'exploit du
premier. 
\smallskip
La loi de probabilit\'e du nombre de neutrons issu d'une fission \'etant
donn\'ee, trouver la probabilit\'e pour que la r\'eaction ne s'\'eteigne 
pas. 
\medskip
Dans un tel processus interviennent aussi les produits de fission,
qui sont les d\'ebris hautement radioactifs des noyaux d'$U^{235}$; 
mais leur existence,  bien que mena\c{c}ant l'avenir de l'humanit\'e, ne
joue qu'un r\^ole n\'egligeable dans le maintien ou l'extinction de la
r\'eaction en cha{\^\i}ne.  D'autre part,  on peut appliquer aux diff\'erents
neutrons  une statistique tout \`a fait classique,  car les \'ev\'enements 
dont ils sont issus (la fission d'un noyau) sont d\'epourvus de 
coh\'erence collective,  de sorte qu'il n'y a pas interf\'erence
des amplitudes quantiques,  mais superposition incoh\'erente. 
\medskip
Dans un tel processus chaque neutron (particule-m\`ere) donne apr\`es 
une fission un nombre {\it al\'eatoire} de nouveaux neutrons
(particules-filles). Bien que ce soit purement th\'eorique, on 
admettra que tout le processus a \'et\'e initi\'e par un seul premier 
neutron: cela revient \`a ne consid\'erer que la descendance d'un seul
neutron. 
\smallskip
{\eightpoint {\bf Remarque pour les physiciens:} 
Pour chaque neutron de la r\'eaction en cha{\^\i}ne, le nombre 
de g\'en\'erations depuis le tout premier neutron est parfaitement 
d\'etermin\'e: en effet, d'apr\`es la relation d'incertitude
phase -- nombre $\Delta\varphi\times\Delta N \sim \hbar$, le nombre
$N$ de particules serait ind\'etermin\'e si par exemple les particules
\'etaient en phase, comme les photons d'un laser; or les neutrons de
fission sont en g\'en\'eral incoh\'erents, c'est-\`a-dire que l'incertitude
sur la phase est infinie, donc l'incertitude sur le nombre est nulle
(il s'agit de l'incertitude de Heisenberg, non de l'incertitude subjective
provenant de la difficult\'e pratique de mesurer ce nombre).
C'est pour cette raison qu'on peut appliquer une statistique
classique. Il y a donc un sens \`a parler de l'ensemble des neutrons de 
la $n^{\rm e}$ g\'en\'eration. } 
\medskip
Par cons\'equent on peut introduire les variables al\'eatoires 
$$Z_n = \hbox{nombre de neutrons de la $n^{\rm e}$ g\'en\'eration}$$
Par ailleurs, il semble absurde que la fission d'un noyau puisse \^etre
influenc\'ee par les r\'esultats des autres fissions\ftn{1}{Cela ``semble
absurde'', mais la M\'ecanique quantique nous a montr\'e que de telles
absurdit\'es peuvent exister et ne sont donc pas absurdes. C'est pourquoi
il faut consid\'erer le postulat d'ind\'ependance que nous faisons ici 
comme une simple hypoth\`ese susceptible d'\^etre r\'efut\'ee par
l'exp\'erience, ou m\^eme dans certains cas par la th\'eorie elle-m\^eme. 
La question est de savoir si les diff\'erentes fissions nucl\'eaires d'une
m\^eme r\'eaction en cha{\^\i}ne sont {\it s\'eparables}, au sens de la
s\'eparabilit\'e quantique.}; cela implique que la loi de probabilit\'e du
nombre de neutrons issus de chaque fission est toujours la m\^eme, \`a
savoir la loi de $Z_1$, et que ce nombre est stochastiquement
ind\'ependant pour chaque fission. 
\medskip
Consid\'erons alors l'ensemble des neutrons de fission de la $n^{\rm e}$
g\'en\'eration, et num\'erotons les par la pens\'ee de $j=1$ \`a $j=Z_n$. 
Le $j^{\rm e}$ de ces neutrons donne un nombre al\'eatoire $X_j^{(n)}$ 
de particules-filles, et d'apr\`es les hypoth\`eses ci-dessus les
variables al\'eatoires $X_j^{(n)}$ sont stochastiquement 
ind\'ependantes entre elles et ont toutes la m\^eme loi que $Z_1$.
\medskip
On en d\'eduit que le nombre de neutrons de la $(n+1)^{\rm e}$ 
g\'en\'eration est 
$$Z_{n+1} = \sum_{j=1}^{j=Z_n}\; X_j^{(n)}$$
Autrement dit, le nombre de neutrons d'une g\'en\'eration est la 
somme d'un nombre lui-m\^eme al\'eatoire de variables al\'eatoires 
ind\'ependantes et de m\^eme loi, c'est-\`a-dire que nous sommes
ramen\'es au probl\`eme $N^\circ 2$. En appliquant simplement ce que
nous avons trouv\'e alors,  on voit que, si on appelle $G_n(z)$ la 
fonction g\'en\'eratrice de $Z_n$: 
$$G_{n+1}(z) = G_n\bigl( G_1(z) \bigr)\quad ;$$
cela montre par r\'ecurrence que $G_n$ est simplement la $n^{\rm
e}$ it\'er\'ee de $G_1$:
$$G_n = G_1 \circ G_1 \circ G_1 \circ \cdots \circ G_1 \qquad
(n\;\hbox{fois}) \eqno (VIII.6.)$$
\bigskip
{\bf \'Etude d'un cas particulier}
\medskip
Il est parfois possible de calculer explicitement $G_n(z)$. Nous allons
traiter un tel cas en d\'etail, puis nous ferons une \'etude qualitative 
pour le cas g\'en\'eral en nous inspirant de ce cas particulier.
\medskip
Supposons que la loi de $Z_1$ soit la suivante:
$${\cal P}\, (Z_1 = k) = \beta \alpha^k$$
o\`u $\alpha$ et $\beta = 1 - \alpha$ sont donn\'es (cette loi, dite {\it 
loi g\'eom\'etrique}, est celle de $Y$ dans le probl\`eme $N^\circ 2$ 
lorsque $\alpha = {5 \over 6}$ et $\beta = {1 \over 6}$). Cette loi est 
peu r\'ealiste en tant que mod\`ele de r\'eaction en cha{\^\i}ne de fission 
nucl\'eaire, mais cela importe peu pour le moment. Cette loi peu r\'ealiste
permet de calculer {\it explicitement} la fonction $G_n$,  et nous
verrons le cas g\'en\'eral ensuite.
\medskip
Il est toujours sous-entendu que les probabilit\'es ${\cal P}\, (Z_1 = k)
=  \beta \alpha^k$ sont par nature approximatives et n'ont gu\`ere de
signification pour des grandes valeurs de $k$; donc la fonction
g\'en\'eratrice est aussi approximative. D'un point de vue pratique il est
en g\'en\'eral bien rare que l'on ait \`a consid\'erer des probabilit\'es 
${\cal P}\, (Z_1 = k)$ au-del\`a de $k = 4$, car elle sont alors plus 
petites que l'incertitude sur leurs valeurs. Mais dans le cas pr\'esent 
o\`u nous supposons une loi g\'eom\'etrique $\beta \alpha^k$, si on
effectue  formellement la somme de la s\'erie $\sum_k {\cal P}\, (Z_1 =
k)\; z^k$ jusqu'\`a l'infini, on obtient une fonction simple (somme d'une
s\'erie g\'eom\'etrique); si par contre, sous le pr\'etexte que $\beta
\alpha^k$  devient n\'egligeable au-del\`a de, disons, $k=10$, on
n'effectuait la somme de la  s\'erie que jusqu'\`a $k = 10$, on obtiendrait
un polyn\^ome de degr\'e $10$ beaucoup moins maniable que la fonction
simple (homographique), et qui ne serait a priori ni plus exact, ni moins
exact que la
fonction simple. C'est pourquoi il est avantageux de sommerjusqu'\`a
l'infini, non parce que la somme infinie est plus exacte, maisparce
qu'elle est plus commode. La fonction est alors 
$$\sum_{k=0}^{\infty}\; \beta \alpha^k\; z^k\; = \; {\beta \over 1 -
\alpha z}\; = \; G_1(z)$$ 
Cette fonction homographique a en outre l'avantage d'\^etre facile a
it\'erer: la composition de deux ou plusieurs fonctions homographiques
est \'egalement une fonction homographique (c'est la raison pour 
laquelle nous avons choisi la loi g\'eom\'etrique).
\medskip
Un calcul sans surprise donne pour les premi\`eres it\'er\'ees les 
r\'esultats suivants:
$$\eqalignno{
G_2(z)\; &= \;\beta\; {1 - \alpha z \over (1 - \alpha\beta ) - 
\alpha z } \cr
G_3(z)\; &= \;\beta\; {(1 - \alpha\beta) - \alpha z \over (1 - 
2\alpha\beta) - (1 - \alpha\beta)\,\alpha\, z } \cr
G_4(z)\; &= \;\beta\; {(1 - 2\alpha\beta) - (1 - \alpha\beta)
\,\alpha\, z \over  (1 - 3\alpha\beta + \alpha^2\beta^2) - (1 -
2\alpha\beta)\,\alpha\, z } \cr } $$
Sur ces expressions, on observe des r\'egularit\'es. 
Par exemple, le num\'e\-ra\-teur de chacune est \'egal au d\'enominateur
de lapr\'ec\'edente.  En observant encore mieux,  on voit aussi que, \`a
l'int\'erieur des num\'erateurs ou des d\'eno\-mi\-na\-teurs,  le premier
terme
est toujours en avance d'un rang sur le coefficient de $\alpha z$. On
devine donc une r\'ecurrence sous la forme
$$G_n(z)\; = \;\beta\; {\strup 7 a_{n+1} - \alpha\, a_{n}\, z 
\over a_{n+2} - \alpha\, a_{n+1}\, z } $$
o\`u $a_n$ est une certaine suite num\'erique. En identifiant les
coefficients dans la relation de r\'ecurrence $G_{n+1} = G_n \circ G_1$,
on constatera que cette suite est elle-m\^eme d\'efinie par la relation 
de r\'ecurrence
$$\displaylines{
a_{n+1} = a_{n} - \alpha\beta\, a_{n-1} \hskip12mm (n \geq 2)\cr
a_1 = 0 \; ; \quad a_2 = 1 \hskip12mm (n = 0\;\; \hbox{et}\;\; 1)\cr }$$
Mais on peut encore simplifier davantage: dans une fonction
homographique $(a + bz)\; /\; (c  + dz)$, les quatre coefficients 
$a,b,c,d$ ne sont pas univoques; on peut simplifier 
l'expression ci-dessus  de $G_n(z)$ en divisant le num\'erateur et le
d\'enominateur par  $a_{n+1}$, ce qui donne
$$G_n(z)\; = \;\beta\; {\kern5pt\strup{12} {\displaystyle 1 -
\alpha\, {a_{n}\over\ldown{a_{n+1}}}\, z }\kern5pt\over\sdown{19}
{\displaystyle {a_{n+2}\over\ldown{a_{n+1}}} -  \alpha\, z } }$$ 
Si alors on introduit la nouvelle suite num\'erique $b_n = a_{n} /
a_{n+1}$ ($n \geq 1$), cela s'\'ecrit 
$$G_n(z)\; = \;\beta\; {1 - \alpha\, b_n\, z \over \sdown{19}
{\displaystyle {1\over b_{n+1}} - \alpha\, z } }\; = \;\beta\, b_{n+1}\; 
{1 - \alpha\, b_n\, z  \over 1 - \alpha\, b_{n+1}\, z } \eqno (VIII.7.)$$
et la nouvelle suite num\'erique $b_n$ est caract\'eris\'ee par la 
r\'ecurrence 
$$ b_{n+1}\; = \; {1 \over 1 - \alpha\beta\, b_n} \qquad (b_1 = 0) 
\eqno (VIII.8.)$$
qu'on d\'eduit de la relation de r\'ecurrence des $a_n$ en divisant ses
deux membres par $a_n$.
\medskip
Une petite \'etude rapide de la suite $b_n$ sera utile. On peut montrer
qu'elle est croissante, que si $\alpha > {1\over 2}$ elle tend vers
$1/\alpha$, et si $\alpha < {1\over 2}$ elle tend vers $1/\beta$. En
effet, on v\'erifie par r\'ecurrence que l'on a toujours
$$1 < b_n < {1\over\ldown{\alpha }} \quad \hbox{et} \quad 
1 < b_n < {1\over\ldown{\beta }}$$
Rappelons que $\beta = 1 - \alpha$.  Par ailleurs on peut \'ecrire
$$\openup 2\jot\eqalignno{
b_{n+1} - b_n &= {1\over 1 - \alpha\beta\, b_n} - b_n = 
{1 - b_n +\alpha\beta\, b_n^2\over 1 - \alpha\beta\, b_n} \cr
&= {(1-\alpha b_n)(1-\beta b_n)\over 1 - \alpha\beta\, b_n}
= b_{n+1}(1-\alpha b_n)(1-\beta b_n) \cr }$$
o\`u on voit que le second membre est,  d'apr\`es ce qui pr\'ec\`ede, 
toujours positif.  Cette derni\`ere \'egalit\'e montre aussi que la
limite ne peut \^etre que $1\over\textdown\alpha$ ou $1\over\beta$; 
du fait que la suite est croissante et d\'emarre \`a $b_0=0$ et $b_1=1$, 
la limite est n\'ecessairement le plus petit de ces deux nombres.
\medskip
Pour obtenir la loi de $Z_n$ il faut maintenant d\'evelopper la fonction 
g\'en\'eratrice $G_n(z)$ en puissances de $z$:
$$\eqalignno{
G_n(z)\; &= \;\beta\, b_{n+1}\; {1 - \alpha\, b_n\, z  \over 1 -
\alpha\, b_{n+1}\, z } \cr
&= \;\beta\, b_{n+1}\; \bigl( 1 - \alpha\, b_n\, z \bigr)\;
\sum_{k=0}^\infty \; (\alpha\, b_{n+1}\, z)^k \cr
&= \;\beta\, b_{n+1}\;\biggl\{  1 + \sum_{k=1}^\infty \; (\alpha\, 
b_{n+1}\, z)^k - \sum_{k=0}^\infty \; \alpha\, b_n\,(\alpha\, 
b_{n+1})^k\, z^{k+1}\;\biggr\} \cr
&= \;\beta\, b_{n+1}\; + \beta\;\bigl[ b_{n+1} -
b_n \bigr]\;\sum_{k=1}^\infty \;  (\alpha\,  b_{n+1}\, z)^k  \cr}$$
On en d\'eduit la loi de probabilit\'e de $Z_n$:
$$\eqalignno{
\hbox{pour $k=0$:}\qquad {\cal P}\, (Z_n = 0)\; &= \;\beta\, b_{n+1} \cr
\hbox{pour $k\geq 1$:}\!\qquad {\cal P}\, (Z_n = k)\; &= \;\beta\;\bigl[
b_{n+1} - b_n \bigr]\; (\alpha\,  b_{n+1})^k \cr
&= \;\beta\, b_{n+1}\, (1 - \alpha b_n)\, (1 - \beta b_n)\; (\alpha\, 
b_{n+1})^k  \cr }$$
Les valeurs de ${\cal P}\, (Z_n = k)$ pour {\it chaque} $k$ sont peu
parlantes et on voit mieux ce qui se passe avec ${\cal P}\, (Z_n \geq
k)$; or ceci s'obtient ais\'ement:
$$\eqalignno{
{\cal P}\, (Z_n \geq k)\;  &= \;\sum_{\ell\geq k}^\infty\; {\cal P}\, 
(Z_n = k)\cr
&= \;\beta\, b_{n+1}\, (1 - \alpha b_n)\, (1 - \beta b_n)\; { (\alpha\, 
b_{n+1})^k \over 1 - \alpha b_{n+1} } \cr
&= \; (1 - \beta b_n)\; (\alpha\, b_{n+1})^k \cr }$$
On voit que cette valeur d\'ecro{\^\i}t exponentiellement avec  $k$. 
Lorsque $n$ est grand, cette d\'ecroissance est tr\`es lente si $\alpha >
{1\over 2}$, car alors $\alpha\, b_{n+1} \simeq 1$; elle est par contre
rapide si $\alpha$ est nettement inf\'erieur \`a ${1\over 2}$, car alors 
$\alpha\, b_{n+1} \simeq {\alpha\over\beta}$, ce qui est nettement 
inf\'erieur \`a $1$. Dans ce second cas, cependant, on voit que 
${\cal P}\, (Z_n = 0)\; = \;\beta\, b_{n+1} \simeq 1$, ce qui signifie que
la probabilit\'e pour que la r\'eaction en cha{\^\i}ne se soit \'eteinte est
pratiquement \'egale \`a $1$ (la limite de cette probabilit\'e quand $n$
tend vers l'infini est $1$). Dans le premier cas, o\`u $\alpha > {1\over
2}$, on voit que la probabilit\'e pour que la r\'eaction en cha{\^\i}ne soit
\'eteinte \`a la $n^{\rm e}$ g\'en\'eration tend vers
$\beta\strup{1.7}\over\textdown\alpha$ (qui est strictement 
inf\'erieur  \`a $1$) lorsque $n$ tend vers l'infini, mais les probabilit\'es
pour qu'il y ait $1,\, 2,\, 3,\,  \ldots ,\, k$ neutrons d\'ecroissent tr\`es
lentement  avec $k$ (elles sont pratiquement \'equiprobables tant que $k
\ll \bigl[ {\alpha\over\beta}\bigr]^n$).
\bigskip
Sur cet exemple le calcul d\'etaill\'e nous a permis d'obtenir 
explicitement, sous forme analytique, la loi de $Z_n$. Les propri\'et\'es
int\'eressantes sont toutefois surtout les suivantes:
\smallskip
--- quand $n$ tend vers l'infini, la limite de ${\cal P}\, (Z_n = 0)$ est
$\alpha\over\beta$ si $\alpha > {1\over 2}$ et  $1$ si $\alpha \leq
{1\over 2}$; cela signifie que si $\alpha \leq {1\over 2}$ la r\'eaction
finira {\it certainement} par s'\'eteindre, et que si $\alpha > {1\over
2}$ il y a toujours une probabilit\'e non nulle pour qu'elle finisse par
s'\'eteindre.
\smallskip
--- dans le cas o\`u la probabilit\'e d'extinction ne tend pas vers $1$, 
peu importe le d\'etail de la loi, mais on constate que celle-ci est
exponentielle (g\'eom\'etrique) et tr\`es \'etal\'ee.
\medskip
En g\'en\'eral, lorsque $G_1(z)$ est quelconque, il est impossible
d'effectuer des calculs analytiques explicites. Mais nous allons voir que
les renseignements {\it int\'eressants} peuvent \^etre d\'eduits quand
m\^eme: ce sera le sujet de la section suivante (``\'Etude du cas 
g\'en\'eral''). Nous y \'etablirons que {\it quelle que soit la loi de} $Z_1$,
il n'y a que deux possibilit\'es: si ${\bf E}(Z_1) \leq 1$ la probabilit\'e
pour que  $Z_n = 0$ tend vers $1$ et si ${\bf E}(Z_1) > 1$ il y a une
probabilit\'e non nulle  que la r\'eaction se poursuive \'eternellement. En
outre, s'il est impossible de calculer explicitement la loi exacte et
compl\`ete de $Z_n$, on peut du moins calculer explicitement ${\bf
E}(Z_n)$ et ${\bf Var}(Z_n)$. Nous verrons qu'on peut m\^eme calculer
algorithmiquement sa densit\'e. Autrement dit, quoique le calcul
d\'etaill\'e soit impossible, tout ce qu'on pouvait d\'eduire
d'int\'eressant \`a partir du calcul d\'etaill\'e peut aussi \^etre obtenu
par une autre voie. 
\bigskip
La fonction g\'en\'eratrice $G(z)$ d'une loi de probabilit\'e n'est pas 
une fonction arbitraire; lorsqu'on la d\'eveloppe selon les puissances de
$z$, les coefficients sont des probabilit\'es, donc ils ne sont jamais 
n\'egatifs, et en outre leur somme est \'egale \`a 1.  Cela a pour
cons\'equence que si   $0 < x \leq 1$, $G(x)$ et ses d\'eriv\'ees
successives au point $z = x$ sont  toutes positives. Cela se voit tr\`es
facilement. Soit $G^{(j)}(x)$ la $j^{\rm e}$ d\'eriv\'ee de $G(z)$ au point
$z = x$; on a 
$$G^{(j)}(x) = \sum_{k\geq j} k(k-1)(k-2) \cdots (k-j+1)\, p_k\, x^{k-j}$$ 
dans cette somme tous les termes sont positifs donc la somme
est positive; en outre dans le cas o\`u pour simplifier on souhaite
sommer jusqu'\`a l'infini il n'y a pas de probl\`eme de convergence
puisque nous avons suppos\'e $x \leq 1$. Or une fonction $G(x)$ dont
toutes les  d\'eriv\'ees sont positives est n\'ecessairement croissante
et convexe; donc une droite ne peut pas couper son graphe en plus de
deux points. L'\'equation $G(x) = x$ a au plus deux racines. Mais $x = 1$
est une racine \'evidente. On peut comprendre ais\'ement que si $G'(1) >
1$ il existe une deuxi\`eme racine $x_0$ telle que $0 < x_0 < 1$. Par
contre si $G'(1) < 1$, il ne peut pas exister entre $0$ et $1$ d'autre
racine que $1$. La convexit\'e de $G(x)$ a en outre la cons\'equence que
si $0 < x < 1$, la suite des it\'er\'ees $G^n(x)$ tend vers la racine $x_0$
si $G'(x) > 1$ et vers la racine $1$ si $G'(x) \leq 1$. Voir la figure 18.

\midinsert
\vskip3pt
\centerline{\epsfbox{../imgEPS/ch08eps/fig18.eps} }
\centerline{\eightpoint figure 18.}
\endinsert

\medskip
Dans l'exemple que nous avons trait\'e en d\'etail, ces ph\'enom\`enes 
se produisaient. La fonction $G(z)$ \'etait $\beta / (1 - \alpha z)$ et sa
d\'eriv\'ee \'etait donc $G'(z) = \alpha\beta / (1 - \alpha z)^2$, d'o\`u
$G'(1) = \alpha /\beta$. La condition qui \'etait n\'ecessaire pour que la
probabilit\'e d'extinction ne tende pas vers $1$ \'etait que $\alpha >
{1\over 2}$, ou, ce qui est \'equivalent, que $\alpha > \beta$,
c'est-\`a-dire $G'(1) > 1$. Dans ce cas, le calcul nous avait montr\'e que 
la suite $b_n$ tendait vers $1\over\textdown\alpha$, et par
cons\'equent $G_n(z) = \beta\, b_{n+1}\, (1 - b_n\alpha z)\; /\; (1 -
b_{n+1}\alpha z)$ (cf. $VIII.7$) devait avoir pour limite $\beta / 
\alpha$ lorsque $n$ tend vers l'infini; or la racine $x_0$ de l'\'equation
$G(x) = x$ est dans  ce cas pr\'ecis\'ement \'egale \`a cette valeur.
\bigskip
Il est possible de trouver, \`a partir de la r\'ecurrence $(VIII.8)$, une 
expression analytique explicite de la suite $b_n$ en fonction de $n$. 
Pour cela, introduisons $x_n = 1-\alpha b_n$; puisque $b_n$ tend vers 
 $1 / \alpha$, il est clair que $x_n$ tend vers 0. En outre, puisque $b_n$
tend vers  $1 / \alpha$ {\it en croissant}, les $x_n$ sont toujours
positifs. On d\'eduit imm\'ediatement la r\'ecurrence des $x_n$ de celle
des $b_n$: 
$$x_{n+1} = {\beta x_n \over \alpha + \beta x_n}$$
On voit que lorsque $x_n$ est petit (donc pour $n$ assez grand), le terme 
$\beta x_n$ au d\'enominateur est n\'egligeable devant $\alpha$, de 
sorte que $x_{n+1} \simeq {\beta\strup{1.7}\over\textdown
\alpha}x_n$. Cela montre que  la d\'ecroissance de la suite $x_n$ est de
type exponentiel, de raison ${\beta\strup{1.7}\over\textdown
\alpha}$; c'est-\`a-dire qu'on s'attend \`a ce que  pour $n$ grand, $x_n
\sim \big( {\beta\strup{1.7}\over\textdown\alpha} \big)^n$. 
Posons alors  
$$y_n =  \bigg( {\alpha\over\beta} \bigg)^n\, x_n$$
La r\'ecurrence entre les $y_n$ est maintenant
$$y_{n+1} = {\strup{7} \alpha\, y_n \over \quad\sdown{19}{\displaystyle 
\alpha + \bigg(  {\beta \over \ldown \alpha} \bigg)^{n+1}\alpha
y_n}\quad } = {\strup{7} 1 \over \quad \sdown{19} {\displaystyle
{1\over\ldown{y_n}}  +  \bigg( {\beta\over \ldown\alpha} \bigg)^{n+1}}
\quad }$$   
ou encore, en consid\'erant la suite des inverses $u_n = {1
\over\textdown{y_n}}$:   
$$u_{n+1} = u_n + \bigg( {\beta\over\ldown\alpha} \bigg)^{n+1}$$  
Pour amorcer la r\'ecurrence, il faut conna{\^\i}tre $u_1$, ce qui 
s'obtient imm\'e\-dia\-te\-ment: $b_1=0$, donc $x_1=1$, $y_1=
{\alpha\over\beta}$, et $u_1 = {\beta\strup{1.7}\over\textdown
\alpha}$.  Par cons\'equent $u_n$ peut s'exprimer directement en 
fonction de $n$:   
$$u_n = \sum_{j=1}^{j=n} \bigg(
{\beta\over\ldown\alpha} \bigg)^{j} =  
{\beta\strup{1.7}\over\ldown\alpha}\; {\strup{7} 1 -  \bigl(
{\beta\strup{1.7}\over\textdown\alpha} \bigr)^{n}  \over  
\sdown{15} 1 - {\beta\strup{1.7}\over\textdown\alpha}}$$ 
En revenant en arri\`ere, on obtient alors pour $x_n$ et $b_n$: 
$$x_n ={\textstyle \bigl( {\beta\strup{1.7}\over\textdown\alpha} 
\bigr)^{n-1}} {\strup{7} 1 - {\beta\strup{1.7}\over\textdown\alpha} 
\over \sdown{15} 1 - \bigl( {\beta\strup{1.7}\over\textdown
\alpha} \bigr)^{n} } \qquad b_n = {1\over\ldown
\alpha}\; {\strup{7}  1 - \bigl({\beta\strup{1.7}\over\textdown
\alpha} \bigr)^{n-1} \over \quad\sdown{15}  1 -  \bigl( {\beta\strup{1.7}
\over\textdown\alpha} \bigr)^{n}\quad }$$    
Revenons \`a l'expression $(VIII.7)$ de $G_n(z)$ qui avait \'et\'e obtenue
plus haut \`a l'aide des $b_n$.   
$$\openup 3\jot\eqalign{
G_n(z)\; &= \;\beta\; {1 - \alpha\, b_n\, z \over \sdown{19}
{\displaystyle {1\over b_{n+1}} - \alpha\, z } }\; \cr
&= \; \beta\; {1 - \alpha\, b_n\, z \over \sdown{12}  1 - \alpha\beta
b_n - \alpha\, z }  \cr
&= \; 1 + {(\beta - \alpha - \beta x_n) w \over \alpha w + \beta x_n}\cr
} \eqno (VIII.9.)$$
o\`u l'on a introduit $w = 1 - z$. Lorsque $x_n$ tend vers z\'ero
(c'est-\`a-dire lorsque $n$ tend vers l'infini), la limite de cette 
derni\`ere expression n'est pas une bonne fonction analytique de
$w$; en effet, si $w=0$, cela tend vers 1, mais d\`es que $w\neq 0$, 
cela tend vers  $\beta\strup{1.7}\over\textdown\alpha$. Bien 
entendu, tant  que $n$ n'est pas infini, il y a continuit\'e: lorsque $w$
s'\'ecarte  de z\'ero (c'est-\`a-dire  lorsque $z$ s'\'ecarte de $1$),
l'expression ci-dessus passe continuement de la valeur 1 \`a la valeur
$\beta\strup{1.7}\over\textdown\alpha$. Mais plus $n$ est grand, 
plus  ce passage est rapide. Il se produit le ph\'enom\`ene  suivant: pour 
$w=0$, $G(1-w)=1$, et lorsque $|w| \gg \bigl( {\beta\strup{1.7} \over
\textdown\alpha } \bigr)^{n}$, $G(1-w) \simeq {\beta\strup{1.7} 
\over \textdown\alpha }$. Le saut brusque de $1$ \`a ${\beta
\strup{1.7} \over \textdown\alpha }$ se produit sur une plage de
valeurs pour $|w|$ qui est de l'ordre de $\bigl( {\beta\strup{1.7} \over
\textdown\alpha }  \bigr)^{n}$, ce qui,  lorsque $n$ est grand, est 
une tr\`es courte distance. 
\medskip
Le ph\'enom\`ene qui se produit l\`a est analogue \`a celui du chapitre 
{\bf VII}. Nous avions vu \`a cette occasion que si $X_1,\, X_2, \ldots 
X_n$ \'etaient des variables al\'eatoires stochastiquement
ind\'ependantes, alors la  loi de  $(X_1 + X_2 + \cdots + X_n) / \sqrt{n}$
avait une fonction caract\'eristique $\Phi_n(t)$ proche de $e^{-at^2}$;
mais il a fallu pour  cela diviser la somme $(X_1 + X_2 + \cdots + X_n)$
par $\sqrt{n}$; la fonction caract\'eristique de la somme {\it non
divis\'ee} par $\sqrt{n}$  est $e^{-ant^2}$ et non $e^{-at^2}$. Lorsque $n$
tend vers l'infini, il arrive donc \'egalement que la fonction
caract\'eristique de la somme non divis\'ee tende vers $1$ lorsque
$t=0$,  et vers z\'ero lorsque $t\neq 0$: c'est le m\^eme type de
discontinuit\'e. Lorsque $n$ n'est pas infini, mais grand, le passage de
$1$ \`a $0$ est continu, mais rapide, comme le montre la  fonction
$e^{-ant^2}$: le passage se fait sur une \'etroite plage de  valeurs, de
l'ordre de $1 / \sqrt{n}$. On faisait donc appara{\^\i}tre la forme
gaussienne $e^{-at^2}$ de ce passage de $1$ \`a $0$ en effectuant  un
changement d'\'echelle: on ``agrandissait'' dans un rapport $\sqrt{n}$
l'intervalle autour de $t=0$ o\`u ce changement se produit. \medskip Le
ph\'enom\`ene que nous \'etudions maintenant \'etant analogue, il est
judicieux d'effectuer \'egalement un changement d'\'echelle, cette fois 
de rapport $\bigl( {\alpha\over\beta} \bigr)^n$. Ainsi, pour agrandir le
voisinage de $w=0$ dans ce rapport, nous poserons $\zeta = \bigl(
{\alpha\over\beta} \bigr)^n\, w$. En fonction  de cette nouvelle variable,
l'expression $(VIII.9)$ de $G_n(z)$ devient  
$$G_n(z) = 1 + {(\beta - \alpha - \beta x_n) \strup{8.5}\bigl( 
{\beta\strup{1.7}\over\textdown\alpha} \bigr)^{n} \zeta \over
\sdown{15} \alpha \bigl( {\beta\strup{2}\over\textdown\alpha}
\bigr)^{n} \zeta + \beta  x_n}$$  
Or le rapport $\bigl( {\beta\strup{1.7}\over\textdown\alpha} 
\bigr)^{n}$  se factorise aussi bien au num\'erateur qu'au
d\'eno\-mi\-na\-teur, de  sorte  qu'apr\`es simplification il reste: 
$$G_n(z) = 1 + {(\beta - \alpha - \beta x_n)\;
\strup{7}
 \zeta \over \quad \sdown{23}{\displaystyle \alpha\zeta + \beta\;
{\strup{6} 1 - {\beta\strup{2}\over\textdown\alpha} \over 
\sdown{15}  1 - \bigl( {\beta\strup{2}\over\textdown\alpha}
\bigr)^{n+1} }} \quad }$$ 
Cette expression est enti\`erement explicite: $G_n(z)$ y est exprim\'e
analytiquement en fonction de $\alpha$, $\zeta$, et $n$, sans recourir 
\`a aucun interm\'ediaire qui ne serait connu que par une relation de
r\'ecurrence. En y faisant tendre $n$ vers l'infini, tout en laissant
$\alpha$ et $\zeta$ fix\'es, on voit que cela tend vers une limite, qui 
est  \'egale \`a $1 + Q(\zeta )$, o\`u $Q(\zeta )$ est la fonction:
$$Q(\zeta ) = {(\alpha - \beta )\; \zeta \over \beta - \alpha\,\zeta }
 \eqno (VIII.10.)$$
Mais il faut bien comprendre qu'il s'agit l\`a de la limite {\it lorsque
$\zeta$ reste fix\'e}. Car $z = 1 - \bigl( {\beta\strup{1.7}\over
\textdown\alpha}  \bigr)^{n}\,\zeta$, de sorte que $z$ varie avec 
$n$ et n'est pas fix\'e: si $\zeta$ est fix\'e, $z$ tend vers $1$ lorsque 
$n$ tend vers l'infini, et  n'est donc pas lui-m\^eme fix\'e. Inversement,
si on fait tendre $n$ vers l'infini {\it en maintenant $z$, et non} $\zeta$
fix\'e, la limite sera  toute diff\'erente: ce sera, comme nous l'avons vu,
$1$ si $z=1$ et ${\beta\strup{1.7}\over\textdown\alpha }$ si $z
\neq 1$. En termes de valeurs approch\'ees, cela signifie simplement que
pour  $n$ grand, $G(z)$ est une fonction \'egale  \`a
${\beta\strup{1.7}\over\textdown\alpha }$ partout except\'e dans  
un petit voisinage (de diam\`etre environ $\bigl[
{\beta \strup{1.7}\over\textdown\alpha }  \bigr]^n$) autour de
$z=1$, tandis qu'\`a l'int\'erieur de ce voisinage, {\it si  on agrandit
celui-ci}  dans le rapport $\bigl( {\beta\strup{1.7}\over\textdown 
\alpha } \bigr)^n$, la fonction $G(z)$ est approximativement \'egale \`a
$1 + Q(\zeta )$. Ce que nous avons donc effectu\'e par le passage de la
variable $z$ \`a la variable $\zeta$ est {\it un changement d'\'echelle}.
Pour $n$ grand, la fonction caract\'eristique de $Z_n$ est $\Phi_n(t)
\simeq 1 + Q\big(\bigl[ {\beta\strup{1.7}
\over\textdown\alpha }  \bigr]^{-n}\, (1 - \e^{it}) \big)$, qui est 
presque partout proche de $\beta / \alpha$ (la probabilit\'e pour que $Z_n
= 0$), et ne diff\`ere sensiblement de $\beta / \alpha$ que  dans un
intervalle minuscule, dont la largeur est de l'ordre de $\bigl[ {\beta
\strup{1.7}\over\textdown\alpha }  \bigr]^n$, autour de $t=0$. Mais
toute l'information sur la densit\'e de probabilit\'e des valeurs de $Z_n$
autres que z\'ero est pr\'ecis\'ement concentr\'ee dans les variations de 
$\Phi_n(t)$ \`a l'int\'erieur de ce petit intervalle, et le changement
d'\'echelle permet de mieux les voir.

\bigskip

{\bf \'Etude du cas g\'en\'eral.}
\medskip

Tout ce que nous venons de voir sur l'exemple, a une valeur 
g\'en\'erale; sauf, \'evidemment, qu'en g\'en\'eral on ne peut pas
effectuer  analytiquement tous les calculs, et qu'on doit alors se
contenter de calculs num\'eriques (mais nous allons voir qu'on peut
disposer pour cela d'algorithmes simples et efficaces).
\medskip
Pour commencer, nous allons \'etablir que la fonction $Q(\zeta )$ existe 
en g\'en\'eral, et non seulement dans le cas particulier de l'exemple; nous 
verrons \'egalement un algorithme simple pour la calculer. Tout cela
r\'esulte, comme nous allons voir, du fait qu'une fonction g\'en\'eratrice
n'est pas une fonction analytique quelconque, mais poss\`ede des
propri\'et\'es tr\`es particuli\`eres, notamment le fait --- signal\'e plus 
haut --- que pour $x$ r\'eel et compris entre $0$ et $1$, toutes les
d\'eriv\'ees de $G(z)$ en $z=x$ sont positives.
\medskip
La premi\`ere cons\'equence de cette propri\'et\'e est (comme nous 
l'avons d\'ej\`a vu) l'existence d'une solution $x=r$ unique de l'\'equation
$G(x) = x$ pour $0 < x < 1$ lorsque $G'(1) > 1$. Il faut noter que $G'(1)$ 
est l'esp\'erance math\'ematique de la variable al\'eatoire $Z_1$: la
condition $G'(1) > 1$ signifie donc que chaque particule-m\`ere engendre
{\it en moyenne} plus qu'une particule-fille (la moyenne n'est
\'evidemment pas forc\'ement un nombre entier).
\medskip
La raison de l'existence de cette solution $r$ est que la fonction $G(x)$
est croissante et convexe sur l'intervalle $[0,1]$. Un raisonnement
\'el\'ementaire, qui saute aux yeux en voyant la figure 18, 
montre qu'il  doit en \^etre ainsi si $G'(1) > 1$. On peut signaler les cas
d\'eg\'en\'er\'es: si  $G(0) = 0$, $r$ devient simplement $0$; si $G(z) = 
z$, il y a une infinit\'e de solutions, mais ce cas est exclu puisque nous
supposons $G'(1) > 1$. 
\medskip
Le raisonnement simple qui se d\'eduit de la figure 18 est 
valable pour  la fonction $G(x)$ de la variable r\'eelle $x$; son \'evidence
g\'eom\'etrique ne permet pas de s'assurer qu'il n'existerait pas de
solution complexe de l'\'equation $G(z) = z$. Mais cela peut \^etre
d\'emontr\'e autrement, toujours en utilisant les propri\'et\'es
particuli\`eres des fonctions g\'en\'eratrices. En effet, la s\'erie $\sum
p_k z^k$ qui d\'efinit $G(z)$ converge pour $|z| < 1$, donc on peut \'ecrire
pour n'importe quel nombre complexe $z$ de module $< 1$:
$$G(z) - r = \sum_{k=0}^{\infty } p_k (z^k - r^k) = (z-r) \sum_{k=0}^{
\infty } p_k \Biggl[ \sum_{j=0}^{k-1} z^{k-1-j}r^j \Biggr]$$ 
on a simplement utilis\'e l'identit\'e $z^k - r^k = (z-r)\sum
z^{k-1-j}r^j$. Passant aux modules, cela donne l'in\'egalit\'e
$$\openup 3\jot\eqalignno{ |G(z) - r| &\leq |z-r|
\sum_{k=0}^{\infty } p_k \Biggl[\sum_{j=0}^{k-1} |z|^{k-1-j}r^j\Biggr]\cr
&= |z-r|\sum_{k=0}^{\infty } p_k  \Biggl[ \sum_{j=0}^{k-1} {|z|^k - r^k 
\over |z|-r}\Biggr] \cr
&= |z-r| {G(|z|) - G(r) \over |z| - r} \cr }$$ 
Or si $|z| < 1$, $\big( G(|z|) - G(r) \big) / \big( |z| - r \big) = \rho$ est
\'egalement $< 1$ \`a cause de la convexit\'e de $G(x)$ ($|z|$ et $r$ sont
r\'eels).
\medskip
On obtient alors ceci: si $|z| < 1$, $|G(z) - r| \leq \rho |z-r|$ avec $\rho <
1$.  Il est donc impossible que $G(z) = z$,  sauf si $z=r$ [sinon on aurait
$|z-r| \leq \rho |z-r|$].  Donc $z=r$ est la
seule racine de l'\'equation$G(z) = z$ \`a l'int\'erieur du disque $|z| \leq 1$.  Cela signifie que s'il
existe d'autres racines,  celles-ci sont soit sur le cercle $|z| = 1$,  soit
\`a l'ext\'erieur de ce cercle.
\medskip
La fonction g\'en\'eratrice $G_n$ de $Z_n$ \'etant la $n^{\rm e}$ 
it\'er\'ee de $G$, on peut aussi d\'eduire de cette in\'egalit\'e que pour
$|z| < 1$ $$|G_n(z) - r| \leq \rho^n |z-r|$$
qui montre que si $|z| < 1$, $G_n(z)$ tend vers $r$ quand $n$ tend vers
l'infini, d'autant plus rapidement que $\rho$ est plus petit. Il en va de
m\^eme si $|z| = 1$ et $|G(z)| < 1$, car alors on peut \'ecrire
$$|G_n(z) - r| \leq \rho^{n-1} |G(z)-r|$$
Encore plus g\'en\'eralement: il suffit que l'une quelconque des 
it\'er\'ees $G_j(z)$ soit de module inf\'erieur \`a $1$.
Si $|z| \leq 1$, le seul cas o\`u $G_n(z)$ ne tend pas vers $r$ est celui 
o\`u $|z| = 1$ et o\`u pour {\it toutes} les it\'er\'ees $G_j$, $|G_j(z)| = 1$.
\medskip
Dans l'exemple particulier trait\'e pr\'ec\'edemment,  le nombre $z=1$ 
\'etait le seul,  parmi les nombres de module $1$,  pour lequel $G(z)$
est
lui aussi de module $1$;  donc pour tous les $z$ tels que $|z| \leq 1$,
except\'e $z=1$,  $G_n(z)$ tendait vers $r= {\beta\strup{1.7} \over 
\textdown\alpha }$,  de sorte que la limite de $G_n(z)$ pour $n$ tendant
vers l'infini \'etait une fonction discontinue.  Mais en effectuant
unehomoth\'etie de centre $z=1$ et de rapport $a^n$,  c'est-\`a-dire un
changement d'\'echelle,  avec $a=G'(1)={\bf E}(Z_1)$ (dans le cas particulier
$a$
\'etait \'egal \`a $\alpha / \beta$),  on avait fait appara{\^\i}tre la 
fonction limite $Q(\zeta )$,  qui est non seulement continue, mais
analytique.  Si on veut construire une telle fonction limite dans le cas
g\'en\'eral o\`u $G(z)$ est une fonction g\'en\'eratrice quelconque,  il 
faut,  pour que l'homoth\'etie de centre $z=1$ puisse \^etre introduite,
que l'on d\'eveloppe les fonctions $G_n$ en s\'erie enti\`ere au voisinage
de $z=1$ et non au voisinage de $z=0$;  c'est d'ailleurs ce que nous avions
fait dans le cas particulier pour mettre en \'evidence la fonction $Q(\zeta)$. 
\medskip
Posons donc $w = z - 1$ et $F(w) = G(1+w) - 1$. Il est clair que $F(0) = 
0$ et que la $n^{\rm e}$ it\'er\'ee de $F$ est $F_n(w) = G_n(1+w) - 1$: 
en  effet, si on suppose que c'est vrai pour $n$, on aura
$$\eqalignno{
F_{n+1}(w) &= F\big( F_n(w) \big) = G\big(1 + F_n(w) \big) - 1 \cr
&= G\big( G_n(1+w) \big) - 1 = G_{n+1}(1+w) - 1 \cr }$$
La fonction $Q(\zeta )$ doit alors \^etre la limite des fonctions
$Q_n(\zeta )$ d\'efinies par:
$$Q_n(\zeta ) = F_n(a^{-n}\zeta )$$
Le probl\`eme est de montrer que cette limite existe (c'est-\`a-dire
que la suite $Q_n(\zeta )$ n'est pas divergente), qu'elle est une
fonction continue et analytique de la variable complexe $\zeta$, et
\'eventuellement de pouvoir calculer sa s\'erie enti\`ere. Si ce travail 
est effectu\'e, alors on pourra dire que pour $n$ grand, $G_n(z)$ est
approximativement \'egale \`a $1 + Q(a^n w)$; ou encore, que la fonction
caract\'eristique de $Z_n$ est donn\'ee approximativement par
$$\Phi_n(t) \simeq 1 + Q\big( a^n (\e^{it}-1) \big)$$
Si au lieu de consid\'erer la fonction caract\'eristique de $Z_n$, on
consid\`ere plut\^ot celle de $Z_n / a^n$, qui n'est autre que
$\Phi_n(t/a^n)$, on obtient alors l'expression particuli\`erement simple:
$$\Phi_n(t/a^n) \simeq 1 + Q(it)$$
car lorsque $n$ tend vers l'infini, $a^n(\e^{i(t/a^n)}-1)$ tend vers $it$.
\medskip
Cette approximation de la fonction caract\'eristique est d'autant plus
correcte que $n$ est plus grand et que $\zeta$, ou $t$, est plus petit;
en particulier, elle n'est plus valable si $\zeta$, ou $t$, est tr\`es grand.
Comme nous l'avons d\'ej\`a constat\'e au chapitre {\bf VII}, cela 
signifie que l'approximation ci-dessus est un filtre passe-bas qui
\'elimine le {\it bruit discret} de la loi de $Z_n$ mais qui conserve 
toute l'information sur la {\it densit\'e} de cette loi. L'\'etude
d\'etaill\'ee du cas particulier avait montr\'e que la loi de $Z_n$
\'etait faite
d'une probabilit\'e appr\'eciable pour que $Z_n=0$, et d'un nuage tr\`es \'etendu de valeurs non nulles,  ayant chacune une
probabilit\'e
infinit\'esimale;  or l'information  utile est constitu\'ee de:
\smallskip
--- la probabilit\'e de $Z_n = 0$;

--- la densit\'e de probabilit\'e des autres valeurs.
\smallskip
Le {\it d\'etail} de la loi discr\`ete, \`a savoir les probabilit\'es exactes 
des valeurs non nulles de $Z_n$, n'est pas une information utile. Par
cons\'equent, la fonction $Q(\zeta )$ contient exactement l'information
utile et est d\'ebarrass\'ee de l'information inutile. L'approximation
ci-dessus est donc meilleure que la fonction caract\'eristique exacte
$\Phi_n (t/a^n)$. On ne devrait pas dire que $1 + Q(it)$ est une
approximation de $\Phi_n (t/a^n)$, mais que $\Phi_n (t/a^n)$ est une
complication de $1 + Q(it)$.
\medskip
Pour \'etudier la fonction $Q(\zeta )$ et montrer qu'elle est analytique, 
il serait commode d'avoir une relation de r\'ecurrence sur les $Q_n 
(\zeta )$. Or, cela est facile \`a obtenir, car on a la r\'ecurrence sur les
$F_n(w)$: $$F_{n+1}(w) = F\big( F_n(w) \big)$$
en prenant $\zeta = a^n w$ cela donne
$$F_{n+1}(a^{-n}\zeta ) = F\big( F_n(a^{-n}\zeta ) \big)$$
qui est \'equivalent \`a
$$Q_{n+1}(a\zeta ) = F \big( Q_n(\zeta ) \big)$$
Cette relation est pr\'ecieuse: c'est une relation de r\'ecurrence qui permet
de calculer effectivement la suite des $Q_n(\zeta )$, \`a partir de
$Q_0(\zeta ) = \zeta$.
Si on fait tendre $n$ vers l'infini dans les deux membres de la 
relation, on obtient
$$Q(a\zeta ) = F \big( Q(\zeta ) \big) \eqno (VIII.11.)$$
puisque si les deux membres sont \'egaux pour tout $n$, leurs limites 
sont \'egales aussi (notez bien que si la limite de $F \big( Q_n(\zeta )
\big)$  est $F \big( Q(\zeta )\big)$, c'est parce que la fonction $F(w)$ 
est continue).  Toutefois cette d\'eduction est incorrecte s'il n'y a pas 
de limite, ou si la fonction $F(w)$ a des discontinuit\'es. Or la fonction
$G(z)$ est certes analytique au moins dans le disque $|z| < 1$, mais peut
avoir des p\^oles ou des singularit\'es logarithmiques ou essentielles en
dehors de ce disque; la continuit\'e de $F(w)$ n'est assur\'ee que dans le
domaine o\`u elle est analytique: il faut donc que $\zeta$ soit tel que 
{\it tous} les $Q_n(\zeta )$ soient dans ce domaine d'analyticit\'e. Ces
remarques montrent d\'ej\`a que le probl\`eme de l'existence d'une limite
doit  \^etre trait\'e s\'erieusement. Nous nous occuperons de cela un peu
plus loin, voyons d\'ej\`a ce que le r\'esultat permet de faire. 
\medskip
Pour obtenir la s\'erie enti\`ere  de  $Q(\zeta )$, il suffit de la 
substituer \`a $w$ dans la s\'erie connue  de $F(w)$, puis d'identifier 
les coefficients. Soit donc  
$$F(w) = \sum_{n=0}^{\infty } a_n w^n$$
la s\'erie connue; ses coefficients $a_n$ se d\'eduisent des coefficients
$p_n$ de $G(z)$ par la relation
$$\eqalignno{ a_0 &= 0 \cr   a_1 &= a = {\bf E}(Z_1) \cr
a_n &= \sum_{k=n}^{\infty } {k\choose n} p_k \qquad \hbox{si}\quad 
n\geq 2 \cr }$$
On voit qu'ils sont tous positifs. Soit d'autre part la s\'erie inconnue:
$$Q(\zeta ) = \sum_{j=0}^{\infty } c_j \zeta^j \eqno (VIII.12.)$$
En substituant la s\'erie $(VIII.12)$ dans l'\'egalit\'e
$(VIII.11)$:
$$\sum_{k=1}^{\infty } c_k a^k \zeta^k = \sum_{n=1}^{\infty } a_n 
\Biggl[ \sum_{j=1}^{\infty } c_j \zeta^j \Biggr]^n$$
On ne peut d\'evelopper les puissances $n$-i\`emes des s\'eries 
compl\`etes lorsque $n$ est trop grand, mais on peut d\'evelopper aux
ordres les plus bas; ainsi, si on ne retient que les termes de degr\'e au
plus cinq, on obtient: 
$$\eqalignno{
\Biggl[ \sum_{j=1}^{\infty } c_j \zeta^j \Biggr]^2 &= c_1^2 \zeta^2 +
2c_1c_2\zeta^3 + (c_2^2 + 2c_1c_3)\zeta^4 + (2c_1c_4 + 2c_2c_3)
\zeta^5 + \ldots \cr
\Biggl[ \sum_{j=1}^{\infty } c_j \zeta^j \Biggr]^3 &= c_1^3 \zeta^3 +
3c_1^2c_2\zeta^4 + (3c_1^2c_3 + 3c_1c_2^2)\zeta^5 + \ldots \cr
\Biggl[ \sum_{j=1}^{\infty } c_j \zeta^j \Biggr]^4 &= c_1^4 \zeta^4 +
4c_1^3c_2\zeta^5 + \ldots \cr
\Biggl[ \sum_{j=1}^{\infty } c_j \zeta^j \Biggr]^5 &= c_1^5 \zeta^5 +
 \ldots \cr }$$
En identifiant ensuite les coefficients des deux s\'eries, celle de
$Q(a\zeta )$ et celle de $F\big( Q(\zeta ) \big)$, on obtient:
$$\eqalignno{
ac_1 &= a_1c_1 \cr
a^2c_2 &= a_1c_2 + a_2c_1^2 \cr
a^3c_3 &= a_1c_3 + 2a_2c_1c_2 + a_3c_1^3 \cr
a^4c_4 &= a_1c_4 + a_2(c_2^2+2c_1c_3) + 3a_3c_1^2c_2 + a_4c_1^4 \cr
a^5c_5 &= a_1c_5 + a_2(2c_1c_4+2c_2c_3) + a_3(3c_1^2c_3+3c_1
c_2^2) \cr
&\hskip42mm + a_4(c_1^3c_2+3c_1^2c_2) + a_5c_1^5 \cr }$$
La premi\`ere \'equation est ind\'etermin\'ee, on prendra donc $c_1 = 1$. 
La deuxi\`eme donne $c_2 = a_2 / (a^2-a)$. On voit qu'il s'agit d'un
syst\`eme  de cinq \'equations lin\'eaires \`a cinq inconnues, et ce
syst\`eme est triangulaire donc de r\'esolution ais\'ee:
$$\eqalign{
c_1 &= 1 \cr
c_2 &= a_2 / (a^2-a) \cr
c_3 &= \big[ 2a_2c_1c_2 + a_3c_1^3 \big] / (a^3-a) \cr
c_4 &= \big[ a_2(c_2^2+2c_1c_3) + 3a_3c_1^2c_2 + a_4c_1^4 \big] /
(a^4-a) \cr
c_5 &= \big[ a_2(2c_1c_4+2c_2c_3) + a_3(3c_1^2c_3+3c_1c_2^2) \cr
&\hskip23mm + a_4(c_1^3c_2+3c_1^2c_2) + a_5c_1^5\big] / (a^5-a) \cr
} \eqno (VIII.13.)$$ 
c'est-\`a-dire que chaque $c_j$ se calcule ais\'ement \`a partir des
coefficients $c_1,\, c_2,\, c_3,\, \ldots c_{j-1}$ pr\'ec\'edemment 
calcul\'es.
\medskip
On peut ainsi obtenir un d\'eveloppement limit\'e de $Q(\zeta )$ \`a
n'importe quel ordre, et par cons\'equent aussi pour la fonction
caract\'eristique approch\'ee $1 + Q(it)$. Cela permet d\'ej\`a de 
calculer  $1 + Q(it)$ pour des valeurs assez petites de $t$.
\medskip
On voit que tout cela m\`ene \`a des algorithmes pour un calcul effectif
de la fonction caract\'eristique approch\'ee. Comme au chapitre {\bf VII},
on pourra alors d\'eduire (par des op\'erations de filtrage et de
transformations de Fourier) des informations sur la densit\'e de la
variable al\'eatoire $Z_n$ pour $n$ grand.
\medskip
Toutefois ces possibilit\'es de calcul effectif pr\'esupposent que 
la suite $Q_n(\zeta )$ converge vers une limite. Nous allons maintenant
nous occuper de v\'erifier que c'est bien le cas, en pr\'ecisant le domaine
auquel doit appartenir $\zeta$ pour qu'il en soit ainsi. La m\'ethode
est classique: il revient au m\^eme de dire que la suite $Q_n(\zeta )$
a une limite, ou que la s\'erie $\sum \big[ Q_{n+1}(\zeta )
- Q_n(\zeta ) \big]$ est convergente. Et pour montrer que la s\'erie 
est convergente, nous la majorerons par une s\'erie g\'eom\'etrique.
\medskip
Il s'agit donc d'\'etudier la diff\'erence $Q_{n+1}(\zeta ) - Q_n(\zeta )$.
Pour  cela, il faut d'abord en savoir plus sur $F_n(w)$.
\medskip
Un d\'eveloppement limit\'e de $F(w)$ \`a l'ordre 2 est de la forme
$F(w) \simeq aw + a_2w^2$; si on it\`ere cela en ne retenant que les
termes en $w$ et $w^2$, on trouve:
$$\eqalignno{
F(w) &\simeq  aw + a_2w^2 \cr
F_2(w) &\simeq  a^2w + a_2(a+a^2)w^2 \cr
F_3(w) &\simeq  a^3w + a_2(a^2+a^3+a^4)w^2 \cr
F_4(w) &\simeq  a^4w + a_2(a^3+a^4+a^5+a^6)w^2 \cr 
&\ldots \cr }$$
donc on devine la r\'ecurrence
$$\eqalignno{
F_n(w) &\simeq  a^n w + a_2 (a^{n-1} + a^n + a^{n+1} + \cdots
+ a^{2n-2}) w^2  \cr 
&=  a^n w + a_2\, a^{n-1}\; {a^n - 1 \over a - 1}\; w^2 \cr } $$
qui se confirme ais\'ement. Puisque $Q_n(\zeta ) = F_n(\zeta / a^n)$,
on en d\'eduit que le d\'eveloppement \`a l'ordre 2 de la fonction 
$Q_n(\zeta )$ est
$$Q_n(\zeta ) \simeq \zeta + {a_2 \over a-1}\cdot \Biggl[ {1\over a} -
{1\over a^{n+1}} \Biggl]\, \zeta^2$$
et par cons\'equent
$$Q_n(\zeta ) - Q_{n-1}(\zeta ) \simeq {a_2 \over a-1}\cdot \Biggl[ 
{1\over a^n} - {1\over a^{n+1}} \Biggl]\, \zeta^2 $$
Donc il semble raisonnable de conjecturer que l'on doit avoir une 
in\'egalit\'e du type
$$|Q_n(\zeta ) - Q_{n-1}(\zeta )| \leq K(|\zeta |) \Biggl[ {1\over a^n} -
{1\over a^{n+1}} \Biggl] \eqno (VIII.14.)$$
o\`u $K(x)$ est une certaine fonction positive de $x$, \'equivalente \`a
$x^2a_2/(a-1)$ lorsque $x$ tend vers z\'ero, mais ind\'ependante de $n$.
Si on arrivait \`a montrer cette conjecture, on en d\'eduirait
imm\'ediatement que la suite $Q_n(\zeta )$ converge pour tout $\zeta$.
\medskip
Or nous allons voir que tout d\'epend de la courbe $y = G(x)$ telle qu'elle
est repr\'esent\'ee sur la figure 18.  Nous avons d\'ej\`a utilis\'e 
plus haut la propri\'et\'e que la fonction g\'en\'eratrice est non
seulement un polyn\^ome ou une fonction analytique,  mais que les
coefficients de son d\'eveloppement de Taylor sont tous positifs. 
Cela
avait pour cons\'equence que pour $z$ complexe on pouvait toujours
\'ecrire $|G(z)| \leq G(|z|)$.  Pour $x$ r\'eel (donc en ce qui concerne
les
propri\'et\'es g\'eom\'etriques visibles sur la figure),  la courbeest
convexe,  c'est-\`a-dire qu'elle est toujours {\it au-dessus} de sa
tangente.  Par exemple au point $x = 1$,  la tangente \`a la courbe est
la
droite d'\'equation $y-1 = a (x-1)$.  La convexit\'e entra{\^\i}ne que$G(1 +
t) - 1 - at \geq 0$ pour tout $t$ r\'eel.  En outre, la courbes'\'ecarte
plus ou moins lentement de sa tangente selon que la courburede la
courbe est plus ou moins prononc\'ee. 
\medskip
Le cas particulier que nous avons trait\'e pr\'ec\'edemment en d\'etail 
avait  le privil\`ege que la fonction g\'en\'eratrice y \'etait une 
fonction  homographique; les fonctions homographiques sont 
particuli\`erement faciles \`a it\'erer: en it\'erant un nombre 
quelconque  de fois une fonction homographique, on a toujours une 
fonction homographique; alors qu'en it\'erant un polyn\^ome, m\^eme 
aussi simple que $x^2 + x$, on obtient  des polyn\^omes de plus en plus 
en plus complexes, car le degr\'e double \`a chaque it\'eration. 
\medskip
Pour montrer la convergence de la suite des fonctions $Q_n(\zeta )$,
nous allons montrer que la {\it s\'erie} de terme g\'en\'eral 
$Q_{n+1}(\zeta ) - Q_n(\zeta )$ est convergente (du moins pour $\zeta$
assez petit), en la majorant par une s\'erie g\'eom\'etrique. Si la suite
des fonctions $Q_n(\zeta )$ converge dans un disque $|\zeta | < \rho$,
aussi petit soit-il, le principe du prolongement analytique fera le reste.
En effet, il nous faut simplement l\'egitimer les calculs effectu\'es plus
haut en vue d'obtenir le d\'ebut de la s\'erie enti\`ere de $Q(\zeta )$, et
pour cela un disque, m\^eme petit, suffit. Mais pour majorer 
$Q_{n+1}(\zeta ) - Q_n(\zeta )$, il nous faut des encadrements {\it
it\'erables}. Nous avons d\'ej\`a  la minoration $F(t) \geq at$ (pour $t$
r\'eel), due \`a la convexit\'e, qui  par it\'eration donne $F_n(t) \geq a^n
t$, si on appelle $F_n = F \circ F \circ F \cdots \circ F$ la $n^{\rm e}$
it\'er\'ee  de  $F$. Pour avoir une majoration il est indiqu\'e d'utiliser 
une fonction homographique plut\^ot qu'un polyn\^ome: on peut choisir un
coefficient $\alpha$ tel que
$$F(t) \leq {a \, t \over 1 - \alpha t}$$
de telle sorte que par it\'eration on obtienne des in\'egalit\'es de
la forme
$$F_n(t) \leq {\beta_n\, t \over 1 - \alpha_n t}$$
En {\it effectuant} les it\'erations on constate que $\beta_n$ doit \^etre
\'egal \`a $a^n$, et pour les $\alpha_n$ on trouve des expressions
compliqu\'ees, mais qui peuvent \^etre major\'ees par $\alpha\, a^n /
(a-1)$, de sorte que tout peut \^etre r\'esum\'e par
$$F_n(t) \leq {a^n\, t \over \sdown{19}{\displaystyle 1 - {\alpha\, a^n\, 
t \over a - 1}}} \eqno (VIII.15.)$$ 
L'encadrement de $F_n(t)$ ainsi obtenu (voir figure
19)  permettra de montrer que la suite $Q_n(\zeta ) = F_n(\zeta
/a^n)$ converge, ce qui l\'egitimera les calculs heuristiques que nous
avons effectu\'es plus haut pour obtenir des d\'eveloppements limit\'es
et  autres algorithmes pour le calcul de  $Q(\zeta )$

\midinsert
\vskip3pt
\centerline{\epsfbox{../imgEPS/ch08eps/fig19.eps} }
\vskip3mm
\centerline{\eightpoint figure 19.}
\vskip5mm
\endinsert

Il est bien clair que la fonction $F(w) = G(1 + w) - 1$ de la variable 
complexe $w$ a, tout comme $G(z)$ elle-m\^eme, tous ses coefficients
de Taylor positifs; de m\^eme la fonction $F(w) - aw$. De sorte que 
$$\eqalign{
| F(w) | & \leq F( |w| ) \cr
| F(w) - a\, w | & \leq F( |w| ) - a |w| \cr
| F(w') - F(w)| &\leq |w' - w|\, F'(t_0) \quad \hbox{ si $|w'| \leq t_0$ et
$|w| \leq t_0$} \cr } \eqno (VIII.16.)$$
La troisi\`eme de ces in\'egalit\'es devient aussi compr\'ehensible que 
les deux premi\`eres d\`es lors qu'on applique l'in\'egalit\'e des
accroissements  finis \`a la fonction $t \longmapsto F\bigl( w +
t(w'-w)\bigr)$ et qu'on tient compte du fait que les coefficients de
Taylor de la fonction d\'eriv\'ee $F'(w)$ sont tous positifs.
\medskip 
En ce qui concerne les it\'er\'ees, on aura aussi les in\'egalit\'es
$$\eqalign{
| F_n(w) | & \;\leq\; F_n( |w| ) \cr
\Bigl| F_{n+1}\Bigl( {\up{w}\over\ldown{a^{n+1}}}\Bigr) - F_{n}\Bigl( 
{\up{w}\over\ldown{a^{n}}}\Bigr) \Bigr|\; & \leq\; \Bigl| F_{n}\Bigl( 
{\up{w}\over \ldown{a^{n+1}}}\Bigr) - F_{n-1}\Bigl( 
{\up{w}\over\ldown{a^{n}}}\Bigr) \Bigr|
 \times F'(t_0) \cr }$$  
On obtient en effet la seconde de ces in\'egalit\'es en rempla\c{c}ant
dans la
troisi\`eme des in\'egalit\'es $(VIII.16)$ $w'$ par 
$F_{n}\bigl( w / a^{n+1}\bigr)$ et $w$ par 
$F_{n-1}\bigl( w / a^{n}\bigr)$;  il suffit que $t_0$ soit
plus grand que les modules de ces deux nombres.
\medskip
On peut alors r\'ep\'eter le processus et faire une r\'ecurrence 
descendante,  qui donnera
$$\Bigl| F_{n+1}\Bigl( {\up{w}\over\ldown{a^{n+1}}}\Bigr) - F_{n}\Bigl( 
{\up{w}\over\ldown{a^{n}}}\Bigr) \Bigr| \;\leq\; \Bigl| F\Bigl( {\up{w}\over
\ldown{a^{n+1}}}\Bigr) - {\up{w}\over\ldown{a^{n}}} \Bigr|
\times F'(t_0)^n$$
Cela est valide pourvu que les modules des nombres complexes
$F_{j+1}(w / a^j)$ soient tous, pour $j$ variant de $0$ \`a $n-1$,
inf\'erieurs \`a $t_0$: c'est en effet la condition pour que l'in\'egalit\'e
puisse \^etre utilis\'ee pour les indices compris entre $1$ et $n$.
\medskip
Or nous avons vu $(VIII.15)$ que 
$$F_j(t) \;\leq\; {a^j\, t \over \sdown{19}{\displaystyle 1 - {\alpha\, a^j\, 
t \over a - 1}}}$$ 
par cons\'equent, pour que $|F_{j+1}(w / a^n)|$ soit inf\'erieur \`a
$t_0$ pour tous les $j$ entre $0$ et $n-1$, il suffit de prendre 
$$t_0 = {|w| \over \sdown{19}{\displaystyle a - {\alpha\, |w| \over 
a-1}}}$$
En effet, pour $0 \leq j \leq n-1$ on a $n-j \geq 1$, donc 
$${|w|\over\sdown{19}{\displaystyle a - {\alpha\, |w| \over a-1}}} \;\geq\; 
{|w| \over \sdown{19}{\displaystyle a^{n-j} - {\alpha\, |w| \over a-1}}} 
\;\geq\; F_{j+1}\Bigl( {\up{w} \over \ldown{a^n}}\Bigr)$$
Cela revient \`a dire que, si $t_0$ est fix\'e par ailleurs (et nous 
verrons plus loin qu'il faudra le choisir), l'in\'egalit\'e ci-dessus sera
v\'erifi\'ee pour tous les $w$ tels que
$$|w| \; < \; {a\, t_0 \over \sdown{19}{\displaystyle 1 + {\alpha\,  t_0 
\over  a-1}}} \; =\; \rho$$
Nous avons ainsi \'etabli l'in\'egalit\'e
$$| Q_{n+1}(\zeta ) - Q_{n}(\zeta ) | \;\leq\; \Bigl|  F\Bigl( {\up{\zeta } 
\over \ldown{a^{n+1}}}\Bigr) - {\up{\zeta }\over\ldown{a^{n}}} \Bigr|
\times F'(t_0)^n \eqno (VIII.17.)$$ 
Mais par ailleurs, puisque la fonction $F(w) - a w$ poss\`ede la 
propri\'et\'e que tous ses coefficients de Taylor sont positifs, on a 
aussi
$$\Bigl| F\Bigl( {\up{\zeta }\over \ldown{a^{n+1}}}\Bigr) -
{\up{\zeta }\over\ldown{a^{n}}} \Bigr| \leq F\Bigl( {\up{|\zeta |}\over
\ldown{a^{n+1}}}\Bigr) - {\up{|\zeta |}\over\ldown{a^{n}}} \leq {a^{-n}\, 
|\zeta | \over \sdown{19}{\displaystyle 1 - {\alpha\, a^{-n}\, |\zeta | 
\over a - 1}}} - a^{-n}\, |\zeta | \eqno (VIII.18.)$$
D'autre part pour tout $t$ on a l'identit\'e
$${t \over \sdown{19}{\displaystyle 1 - {\alpha\, t \over a - 1}}} - t
\;
= \; {\alpha t^2 \over a - 1 - \alpha t}$$
donc en prenant $t = a^{-n}\, |\zeta |$,  on peut d\'eduire de $(VIII.17)$
et $(VIII.18)$ que pour tout $\zeta$ tel que $|\zeta | < \rho$ et pour tout
entier $n$:
$$| Q_{n+1}(\zeta ) - Q_{n}(\zeta ) | \;\leq\; {\alpha |\zeta |^2\,
a^{-2n}\,
F'(t_0)^n \over a - 1 - \alpha |\zeta |\, a^{-n}}$$
Puisque $a > 1$ il est clair que,  quelle que soit la valeur de $\rho$,
le d\'enominateur de l'avant-derni\`ere expression ci-dessus deviendra
sup\'erieur \`a $(a-1) / 2$ \`a partir d'un certain rang $n_0$ (d\`es que
$a^{-n}$ sera devenu inf\'erieur \`a $(a-1) / 2\alpha\rho$). Donc pour
$n \geq n_0$ on
aura 
$$| Q_{n+1}(\zeta ) - Q_{n}(\zeta ) | \;\leq\; {2\,\alpha\rho^2\over a-1} 
\times \Bigl[ {F'(t_0)\over a^2} \Big]^n \eqno (VIII.19.)$$
Il sufit donc de prendre $t_0$ tel que $F'(t_0) < a^2$ pour que
la suite $Q_n$ converge \`a l'int\'erieur du
disque $|\zeta | < \rho$. La limite $Q = \lim Q_n$ sera donc au moins
d\'efinie et analytique dans ce disque; mais bien entendu, il se peut 
qu'elle se prolonge analytiquement au-del\`a. En tous cas, au moins \`a
l'int\'erieur de ce disque, elle sera la somme d'une s\'erie enti\`ere
convergente. Cela l\'egitime les calculs que nous avons faits plus haut,
pour obtenir les premiers coefficients de cette s\'erie enti\`ere.
\bigskip
Tout au long de ces cha{\^\i}nes logiques, nous avons utilis\'e, pour les
fonctions $G(z)$, $G'(z)$, $F(w)$, $F'(w)$, etc., la propri\'et\'e typique 
des fonctions {\it absolument monotones}, \`a savoir que, leurs
coefficients de Taylor \'etant tous positifs, on a $|G(z)| \leq G(|z|)$,
$|G'(z)| \leq  G'(|z|)$, $|F(w)| \leq F(|z|)$, $|F'(w)| \leq F'(|z|)$, etc.
\medskip
Ainsi, nous venons d'\'etablir que la fonction $Q(\zeta )$ \'etait 
analytique au moins dans le disque $|\zeta | < \rho$ avec
$$\rho \; = \; {a\, t_0 \over \sdown{19}{\displaystyle 1 + {\alpha\, 
t_0 \over a-1}}}$$
o\`u $t_0$ est tel que $F'(t_0) < a^2$.  On peut m\^eme prendre $t_0$
tel que $F'(t_0) = a^2$;  cela ne changerait rien,  puisque si
$F'(t_0) = a^2$,  alors pour tout nombre
$t_1 < t_0$,  on aura$|F'(t_1)| < a^2$,  et $\rho$ est une fonction strictement croissante de $t_0$.
Mais on voit que ce disque de convergence est limit\'e:  par exemple la
pr\'esence d'un point singulier de la fonction $G(z)$ \`a proximit\'e
du disque $|z| < 1$ aura pour effet de rendre ce disque tr\`es petit.
\medskip
Or, la fonction caract\'eristique asymptotique $\Phi (t)$ que nous 
cherchons \`a d\'eterminer (qui est, rappelons-le, la limite des fonctions
caract\'eristiques $\Phi_n (a^{-n}t)$ des variables al\'eatoires $Z_n /
a^n$) est \'egale \`a $1 + Q(it)$; on a 
$$\Phi_n (a^{-n}t) \; = \; G_n \bigl( \e^{ia^{-n}t} \bigr)$$
et pour tout $t$ r\'eel, $\e^{ia^{-n}t}$ est de module 1, donc $G' \bigl(
e^{ia^{-n}t} \bigr)$ est inf\'erieur ou \'egal \`a $G'(1) = a$ en module, ce
qui est \'evidemment plus int\'eressant que de pouvoir dire seulement 
que $|G'|$  est inf\'erieur \`a $a^2$. Cela a, comme nous allons le voir
maintenant,  une cons\'equence pratique: les fonctions caract\'eristiques
$\Phi_n (a^{-n}t)$, {\it du fait qu'elles ne prennent en compte que les 
valeurs complexes de module 1} pour la variable $z$ dont d\'epend $G(z)$,
convergent (lorsque  $n$ tend vers l'infini) pour tout $t$ r\'eel, et  non pas
seulement dans un disque. Ou encore: la s\'erie de Taylor de la fonction
$Q(\zeta )$ ne converge, du moins en g\'en\'eral, que dans un disque de
rayon limit\'e, mais  elle se prolonge analytiquement jusqu'\`a l'infini
dans un voisinage  de l'axe imaginaire $\zeta = it$; en effet la relation
entre $t$ et $\zeta$ est $\zeta = a^n\, (e^{ia^{-n}t} - 1)$.  Une troisi\`eme
fa\c{c}on de dire cela:  la fonction $Q(\zeta )$ peut avoir des points
singuliers qui limitent la taille de son disque de convergence,  mais ces
points sont toujours en dehors de l'axe $\zeta = it$,  de sorte que la
relation  
$$\lim_{n \rightarrow \infty } \Phi_n (a^{-n}t) \; = \; 1 + Q(it)$$ 
est vraie pour tout $t$  r\'eel. 
Il ne reste plus qu'\`a le d\'emontrer. \medskip
Pour calculer les coefficients de Taylor de la fonction $Q( \zeta )$, il
avait \'et\'e beaucoup plus commode de partir de la fonction $F(w) = G(1
+ w) - 1$,  car celle-ci (du fait que la s\'erie de $F$ n'a pas de terme
constant) permettait d'obtenir pour les coefficients successifs un
syst\`eme triangulaire permettant une r\'ecurrence.
\medskip
Par contre, pour mettre en \'evidence que $Q$ est analytique dans un
domaine allong\'e contenant l'axe des $\zeta$ imaginaires,
il est pr\'ef\'erable d'employer la fonction $G(z)$.  Nous allons 
donc montrer, tout comme nous l'avions fait pour la suite de fonctions 
$Q_n$, que la s\'erie des $|\Phi_{n+1} (a^{-(n+1)}t) - \Phi_n (a^{-n}t)|$
converge {\it pour tout $\, t$ r\'eel}, en la majorant \`a nouveau par une
s\'erie g\'eom\'etrique. 
\medskip
Partons donc de l'in\'egalit\'e g\'en\'erale
$$|G(x) - G(y)| \;\leq\; |x - y|\; G'(r)$$
qui a lieu pour n'importe quels nombres complexes $x$ et $y$ dont le
module est inf\'erieur ou \'egal \`a $r$ (inutile de la red\'emontrer). Donc
$$\openup 3\jot \eqalignno{
\big|\, \Phi_{n+1} (a^{-(n+1)}t) - \Phi_n (a^{-n}t)\, \big|\; 
&= \;\bigg|\; G_{n+1}\Big( \e^{ia^{-(n+1)}t}\,\Big) - 
G_n\Big( \e^{ia^{-n}t}\, \Big)\, \bigg| \cr
&= \;\Bigg|\; G\bigg( G_{n}\Big( \e^{ia^{-(n+1)}t}\,\Big)\bigg) 
- G\bigg( G_{n-1}\Big( \e^{Ba^{-n}t}\,\Big)\bigg)\, \Bigg| \cr
&\leq \; \bigg|\; G_{n}\Big( \e^{ia^{-(n+1)}t}\,\Big) 
- G_{n-1}\Big( \e^{ia^{-n}t}\,\Big)\, \bigg| \times G'(1) \cr } $$
car les nombres complexes $x = G_{n}\Big( \e^{ia^{-(n+1)}t}\,\Big)$ et $y =
G_{n-1}\Big( \e^{ia^{-n}t}\,\Big)$ sont tous deux de module $\leq 1$.
On peut r\'ep\'eter le proc\'ed\'e (par r\'ecurrence descendante), car les 
nombres $G_{n-j}\Big( e^{ia^{-n}t}\,\Big)$ sont tous de module $\leq 1$.
On obtient ainsi
$$|\Phi_{n+1} (a^{-(n+1)}t) - \Phi_n (a^{-n}t)| \leq \bigg|\; G\Big( 
\e^{ia^{-(n+1)}t}\, \Big)  - \e^{ia^{-n}t}\; \bigg| \times a^n$$
Tout comme nous l'avions fait avec la fonction $F(w)$, il s'agit
maintenant de majorer l'expression $\bigl| G\big( e^{i(\theta / a)}
\big)  - e^{i\theta }\bigr|$ (on prendra ensuite $\theta = a^{-n} t $).
Cela est ais\'e, si on utilise le d\'eveloppement limit\'e de 
l'exponentielle d\'ej\`a employ\'e au chapitre {\bf VII}: $e^{i\alpha } = 1 
+ i\alpha +  R(\alpha )$, avec $|R(\alpha )| \leq \alpha^2 / 2$.
$$\openup 3\jot \eqalignno{
\bigg|\; G\Big( \e^{i(\theta / a)}\, \Big)  &- \e^{i\theta }\, \bigg|\; 
= \;\Bigg|\; \sum_{k\geq 0} p_k \bigg[ 1 + ik{\up{\theta}\over\down{a}} +
R\Big( k\, {\up{\theta}\over\down{a}} \Big) \bigg] - 1 - i\theta - 
R(\theta )\, \Bigg| \cr
&\hskip-11.35pt= \;\Bigg|\; \sum_{k\geq 0} p_k  + i\, {\up{\theta}\over\down{a}}
\sum_{k\geq 0} p_k\, k  + \sum_{k\geq 0} p_k\, R\Big( k\, {\up{\theta}  
\over\down{a}} \Big) - 1 - i\theta -  R(\theta )\, \Bigg| \cr
&\hskip-11.35pt= \;\Bigl|\; \sum_{k\geq 0} p_k\, R\Big( k\, {\up{\theta}  
\over\down{a}} \Big) -  R(\theta )\, \Bigl| \cr
&\hskip-11.35pt \leq \;{\theta^2 \over 2 }\;\Bigg[\; \sum_{k\geq 0} p_k\, 
{k^2\over a^2}
+ 1 \;\Bigg] \cr }$$
Pour la troisi\`eme \'equation ci-dessus,  on a utilis\'e le fait que
$a = \sum k\, p_k$;  dans la quatri\`eme ligne,  on voit s'introduire
le moment d'ordre 2 de la variable al\'eatoire $Z_1$,  \`a savoir 
$M_2 = \sum p_k\, k^2$.  En fin de compte:
$$\Bigl|\Phi_{n+1} (a^{-(n+1)}t) - \Phi_n (a^{-n}t)\Bigr| \leq {t^2
\over 2\, a^{2n}}
\times \bigg( {M_2 \over a^2} +1 \bigg) \times a^n = 
{t^2 \over 2\, a^{n}} \times \bigg( {M_2 \over a^2} +1 \bigg)$$
Ceci montre (puisque $a > 1$) que {\it pour tout} $t$ la s\'erie
$\Phi_{n+1} (a^{-(n+1)}t) - \Phi_n (a^{-n}t)$ est major\'ee par une 
s\'erie g\'eom\'etrique de raison $1/a$.  Du point de vue
math\'ematique, cette conclusion n'est  valable que si la s\'erie $\sum
p_k\, k^2$ converge;  mais nous avons d\'ej\`a  fait remarquer plus d'une
fois que dans les probl\`emes pratiques les $p_k$ sont toujours nuls \`a
partir d'un certain rang (et surtout dans le cas des neutrons de fission). 
\medskip
Ainsi,  pour les valeurs {\it complexes} de $t$,  la fonction limite 
$\Phi (t) = 1 + Q(it)$ n'est la somme d'une s\'erie enti\`ere que dans un
disque,  mais pour les valeurs {\it r\'eelles} de $t$,  $\Phi (t)$ est
d\'efinie  sur tout l'intervalle $] -\infty \quad +\infty\, [$ comme 
la limite de la suite des fonctions $\Phi_n (a^{-n}t)$.

\bigskip

La question est alors de savoir comment on peut calculer effectivement
cette fonction limite $\Phi (t) = 1 + Q(it)$. Nous avons d\'ej\`a vu 
comment calculer r\'ecursivement les coefficients de Taylor de la 
fonction $Q( \zeta )$: cet algorithme est efficace pour calculer 
$Q( \zeta )$, donc en particulier $Q( it )$, pour les petites valeurs de $t$,
mais la petitesse du disque de convergence ne permet pas de le calculer
pour les grandes valeurs de $t$. Pour ces derni\`eres, on utilisera alors
l'\'equation fonctionnelle v\'erifi\'ee par la fonction $Q$ (ou par la 
fonction $\Phi$):
$$\eqalign{
Q( a\zeta ) &= F\bigl( Q( \zeta ) \bigr) \cr
\Phi (at) &= G\bigl( \Phi (t) \bigr) \cr } \eqno (VIII.20.)$$
En effet, si $Q$ ou $\Phi$ est connue au voisinage de z\'ero,  on peut en
d\'eduire les valeurs dans un voisinage $a$ fois plus grand en utilisant
ces \'equations fonctionnelles; puis en r\'ep\'etant le processus, dans
un voisinage $a^2$, puis $a^3$, $a^4$, $\ldots$ $a^n$ fois plus grand. 
Si
on proc\`ede ainsi avec un nombre complexe $\zeta$ quelconque, le proc\'ed\'e finira par diverger lorsqu'on atteindra les bords
du domaine d'analyticit\'e; mais ce que nous venons de voir nous
garantit que pour $t$ r\'eel on pourra poursuivre aussi loin qu'on
voudra\ftn{2}{Ce que nous venons de voir pr\'esupposait que
$M_2 =
\sum p_k\, k^2$ ne soit pas infini;  cela revient \`a supposerque la fonction $G(x)$ est d\'erivable \`a l'ordre deux en $x = 1$.}. 
\medskip
{\eightpoint 
On peut par exemple suivre le plan que voici.  On calcule d'abord les 
valeurs de $\Phi (t)$ entre $t=10^{-5}$ et $t=10^{-5}\times a$ en
utilisant la s\'erie enti\`ere. \xsp  Par exemple, on calcule les valeurs
en dix points $t_0 = 10^{-5}$, $\; t_1 = 10^{-5}\times [1 + 0.1\, (a-1)]$, 
$\; t_2 =
10^{-5} \times [1 + 0.2\, (a-1)]$, $\; t_3 = 10^{-5} \times[1 + 0.3\, (a-1)]$,
 $\; \ldots\; $ $\; t_{10} = 10^{-5} \times a$. \xsp Ou bien $t_0 = 10^{-5}$,  $\; t_1 = 
10^{-5} \times a^{0.1}$, $\; t_2 = 10^{-5} \times a^{0.2}$, $\; t_3 = 10^{-5}
\times a^{0.3}$, $\; \ldots\; $ $\; t_{10} = 10^{-5} \times a$. \xsp  Ensuite,  on calcule
les valeurs en $t_{11} = a\, t_1$,  $\; t_{12} = a\, t_2$, 
$\;t_{13} = a\, t_3$,  $\;\ldots\;$  $\; t_{21} = a^2\, t_1$, 
$\;t_{22} = a^2\, t_2$, $\; t_{23} = a^2\, t_3$, 
$\;\ldots\, ,\; \ldots\;$ \xsp Si le choix de $10^{-5}$ ne convient pas,  soit que 
les intervalles agrandis sont trop grossiers,  soit que le disque de
convergence est trop petit,  on prendra $10^{-6}$ ou $10^{-7}$.
\medskip
On trouvera en fin de chapitre un programme \'ecrit en langage
{\eightrm PASCAL},  qui calcule la fonction $\Phi (t)$ par ce 
proc\'ed\'e,  pour toute donn\'ee de la loi initiale de $Z_1$.
Les calculs effectu\'es \`a l'aide de ce programme sont montr\'es
sur les figures $21$,  $22$,  et $23$. \par}
\vskip7pt plus7pt minus3pt
Enfin,  lorsque la fonction caract\'eristique asymptotique $\Phi$ est
connue (par exemple par le calcul num\'erique),  on peut reconstituer la
densit\'e limite en effectuant une transformation de Fourier inverse.
Il faut cependant prendre en compte deux choses. 
\medskip
a) \xsp  La probabilit\'e (non nulle lorsque
$a = G'(1) > 1$) del'extinction ultime,  qui est \'egale \`a $x_0$,  l'unique racine
$< 1$ de l'\'equation $G(x) = x$.  La loi de probabilit\'e de $Z_n$ 
pour $n$ grand est en effet compos\'ee de deux parties:  d'une part la
valeur z\'ero qui a une probabilit\'e $x_0$ non infinit\'esimale, 
et
d'autre part un nombre \'enorme de valeurs enti\`eres distribu\'ees
entre $0$ et $N^n$ ($N$ \'etant la plus grande des valeurs prises par
$Z_1$),  ayant chacune une probabilit\'e infinit\'esimale,  mais
totalisant la probabilit\'e restante $1 - x_0$.  C'est bien s\^ur la
deuxi\`eme partie qui est ad\'equatement d\'ecrite par la densit\'e,
tandis que la premi\`ere partie reste discr\`ete.
 Ainsi $x_0$ estla probabilit\'e de z\'ero,  et la transform\'ee de Fourier inverse
de $\Phi(t) - x_0$ donnera le reste de la loi.  Convenons d'appeler
{\it loi r\'esiduelle} la partie de la loi concernant les valeurs autres
que $Z_n = 0$. 
\medskip
b) La transform\'ee de Fourier inverse de $\Phi_n(a^{-n}t) - x_0$ donnera
la loi r\'esiduelle discr\`ete exacte.  La densit\'e de cette loi
r\'esiduelle exacte s'obtiendra en effectuant un lissage de cette loi
discr\`ete;  mais comme on a vu au chapitre {\bf VII},  le simple
fait de calculer,  non pas l'int\'egrale 
$$f(x) = {1 \over 2\pi }\int_{-\infty}^{+\infty}
\big[ \Phi (t) - x_0 \big]\,\e^{-ixt} dt$$
compl\`ete,  sur tout l'intervalle de $-\infty$ \`a $+\infty$, 
mais seulement sur un intervalle fini $-A\, , +A$,  \'equivaut \`a un
lissage de la loi par le filtre passe-bas $\rho(x) = \sin(Ax) \Big/
(\pi\, x)$,  dont la longueur de corr\'elation est $\varepsilon \sim 1/A$. 
Or l'\'ecartement entre les valeurs dont on cherche la densit\'e est
$a^{-n}$  (il est constant puisque les valeurs de $Z_n$ sont enti\`eres). 
Comme la fonction $\Phi(t)$ a \'et\'e construite avec le changement
d'\'echelle $\Phi(t) = \lim_{n \to \infty}\Phi_n(a^{-n}t)$,  il suffit
donc de prendre $a^{n} \gg A \gg 1$ pour avoir le lissage correct. 
\medskip
On peut donc dire que la loi asymptotique de $Z_n$ est enti\`erement
d\'etermin\'ee par la connaissance de $x_0$ (probabilit\'e que $Z_n = 
0$) et de la densit\'e $f$,  de sorte que pour $0 < a < b$,  $n$ grand, 
et $b-a \gg a^{-n}$:
$${\cal P}\, (Z_n = 0) \; \simeq \; x_0 \hskip9mm \hbox{et} \hskip9mm
{\cal P}\, (a < {Z_n \over a^n} < b) \; \simeq \; \int_a^b f(x) \, dx$$
La transform\'ee de Fourier de la loi discr\`ete exacte est la fonction 
caract\'e\-ristique compl\`ete de $Z_n$,  donn\'ee sur tout
l'intervalle $-\infty < t < +\infty$:
$$\Phi (t/a^n)\; = \; x_0\, + \sum_{1}^{k_max} p_k\,\e^{itk/a^n}\;
\simeq \;\; \Phi(t) \eqno (VIII.21.)$$
Par cons\'equent,  la loi r\'esiduelle discr\`ete exacte de $a^{-n}Z_n$
(pour $n$ grand) est la transform\'ee de
Fourier inverse de$\Phi (t) - x_0$,  tandis que la densit\'e $f$ sera donn\'ee, 
en prenant $A$ tel que $a^{n} \gg A \gg 1$,  par l'int\'egrale
$$f(x) \; = \; {1 \over 2\pi }\int_{-A}^{+A}
\big[ \Phi (t) - x_0 \big]\,\e^{-ixt} dt  \eqno (VIII.22.)$$
\medskip
{\eightpoint {Remarque.} 
Il est sans doute utile ici de rappeler ceci: l'int\'egrale $VIII.22$ est
forc\'ement convergente puisque ses bornes sont finies;  si on prenait des
bornes infinies,  elle serait divergente,  ce qui est logique puisque
la loi est discr\`ete;  il faut consid\'erer la transform\'ee de Fourier
inverse de $\Phi (t) - x_0$ au sens des distributions par exemple; 
mais le lecteur de ce livre peut ignorer compl\`etement cette subtilit\'e, 
puisque seule nous int\'eresse la densit\'e de la loi discr\`ete, 
et que celle-ci est donn\'ee par l'int\'egrale convergente $VIII.22$.\par}
\medskip
La formule $VIII.22$ fournit donc un proc\'ed\'e
num\'erique ---~on peut utiliser n'importe quel algorithme de
transformation de Fourier~---  pour d\'eterminer $f$ \`a partir de la
fonction caract\'eristique asymptotique $\Phi$,  suppos\'ee obtenue par
les proc\'ed\'es indiqu\'es pr\'ec\'edemment.  Bien entendu la 
probabilit\'e $x_0$ s'obtiendra en cherchant la racine de l'\'equation
$G(x_0) = x_0$ par it\'erations (m\'ethode de Newton).

\vfill\break

\null
\vskip3pt
\centerline{\epsfbox{../imgEPS/ch08eps/fig20.eps} }
\vskip7mm
\centerline{\eightpoint figure 20.}
\vskip3mm
{\eightpoint\baselineskip=14pt
Ce graphique repr\'esente la partie r\'eelle de la fonction 
caract\'eristique asymptotique $\Phi (t)$ de la loi $p_0 = {1\over 9}
\;   ;\; p_1 = {2\over 9}\; ;\; p_2 = 0\; ;\; p_3 = {2\over 3}$, dont la
moyenne est $a = 2.22$. Il s'agit donc de la limite, lorsque $n$ tend vers 
l'infini, des fonctions $\Phi_n(t) = G_n(e^{ia^{-n}t})$, avec $G(z) = (1 + 
2z + 6z^3)/9$. Ce graphique a \'et\'e obtenu avec 9 it\'erations, c'est
donc   le graphique de $G_9(e^{ia^{-9}t})$ (on n'obtient pas un meilleur 
graphique en it\'erant davantage, car le gain en pr\'ecision devient alors 
inf\'erieur \`a la d\'efinition de l'imprimante). 
\medskip
Cette fonction contient toute l'information utile sur la distribution
de probabilit\'e de la variable al\'eatoire $Z_n$ pour $n$ grand, 
c'est-\`a-dire qu'elle d\'ecrit la situation statistique limite de la 
fission apr\`es un certain temps de relaxation. Elle n'est pas la fonction
caract\'eristique d'une loi de probabilit\'e discr\`ete, mais d'une
loi asymptotique: la loi limite de  $Z_n / a^n$ lorsque $n$ tend vers
l'infini. Nous avons vu que cette loi limite comporte un partie discr\`ete
(la probabilit\'e que $Z_n=0$, \'egale \`a $x_0$, la racine
de l'\'equation $G(x) = x$) et une densit\'e continue qui d\'ecrit
la distribution en probabilit\'e des valeurs non nulles. On constate que 
le graphique est form\'e d'une partie centrale avec un maximum 
accentu\'e (toutes les fonctions caract\'eristiques sont maximum en
$t=0$ et ce maximum vaut toujours $1$), et d'une partie plate sur les
bords, qui correspond au fait que la limite de la fonction 
caract\'eristique asymptotique pour $t$ grand est toujours \'egale \`a
$x_0$.}

\vfill\break

\null
\vskip3pt
\centerline{\epsfbox{../imgEPS/ch08eps/fig21.eps} }
\vskip15mm
\centerline{\eightpoint figure 21.}
\vskip4mm
\centerline{\vbox{\hsize=12cm\eightpoint \baselineskip=14pt
Ce graphique presque identique au pr\'ec\'edent a \'et\'e obtenu avec un 
algorithme tr\`es diff\'erent:  on a cette fois utilis\'e les proc\'ed\'es 
\'etudi\'es dans le texte.  On a commenc\'e par d\'eterminer les
coefficients de la s\'erie de Taylor de $Q(\zeta )$; puis on a calcul\'e
$Q(it)$ en quatre points tr\`es proches de z\'ero \`a l'aide de cette 
s\'erie; enfin, pour les valeurs \'eloign\'ees de z\'ero, on a utilis\'e
l'\'equation fonctionnelle $\Phi (at) = G\big( \Phi (t)\big)$. On remarque
que loin de z\'ero, la courbe devient une ligne bris\'ee: cela est d\^u au
manque de finesse de la discr\'etisation, et montre qu'en pratique  
quatre points initiaux sont insuffisants. C'est pourquoi les deux courbes
de la page suivante sont meilleures (elles font appel \`a huit et seize
points initiaux).  Mais cet algorithme est bien plus rapide que 
l'it\'eration directe de la fonction $G$. }} 
\vskip10mm
\vfill\break

\null
\vskip-13mm plus8mm minus8mm
\centerline{\epsfbox{../imgEPS/ch08eps/fig22.eps} }
\vskip1mm
\centerline{\eightpoint figure 22.}
\vskip2mm
\centerline{\vbox{\hsize=9cm\eightpoint 
On a am\'elior\'e la courbe pr\'ec\'edente en prenant 8 points initiaux au 
lieu de quatre.}}

\vskip3mm
plus3mm minus2mm
\centerline{\epsfbox{../imgEPS/ch08eps/fig23.eps} }
\vskip1mm
\centerline{\eightpoint figure 23.}
\vskip2mm
\baselineskip=14pt 
\centerline{\vbox{\hsize=12cm\eightpoint
Encore mieux avec 16 points initiaux. Le r\'esultat est maintenant aussi
bon qu'avec l'it\'eration, et le calcul bien plus rapide.} }

\vskip1pt\break

\null\vfill
\vskip3pt
\centerline{\epsfbox{../imgEPS/ch08eps/fig24.eps} }
\vskip5mm
\centerline{\eightpoint figure 24.}
\vskip3mm
{\eightpoint\baselineskip=12pt
Et voici maintenant la {\it densit\'e} de la loi limite. Cette densit\'e
d\'ecrit la distribution en probabilit\'e des valeurs non nulles
de $a^{-n}Z_n$ (pour $n$ grand). Si on voulait repr\'esenter sur ce m\^eme
graphique la probabilit\'e discr\`ete que $Z_n = 0$, il faudrait dessiner 
un pic extr\^emement haut et \'etroit concentr\'e sur l'axe vertical 
situ\'e au milieu du graphique (cet axe marque l'abscisse z\'ero). 
\medskip
Le graphique a \'et\'e obtenu par la {\it transform\'ee de Fourier 
inverse}. La fonction caract\'eristique d'une loi de probabilit\'e est en 
effet sa transform\'ee de Fourier, on retrouve donc la loi de probabilit\'e
\`a partir de la  fonction caract\'eristique en effectuant l'op\'eration
inverse.  Mais la densit\'e repr\'esent\'ee ci-dessus n'est qu'une partie 
de la loi,  celle qui concerne les valeurs non nulles. Si $f(x)$ est la
densit\'e repr\'esent\'ee ci-dessus (de sorte que ${\cal P}\, (Z_n = k) =
f(k/a^n)$) la fonction caract\'eristique asymptotique est 
$$\Phi (t) = x_0 + \sum_{k\geq 1} f(a^{-n}\, k)\;\e^{ika^{-n}t}$$
puisque ${\cal P}\, (Z_n = 0) = x_0$, o\`u $x_0$ est la racine de 
$G(x_0) = x_0$. Lorsque $n$ est grand la somme devient une int\'egrale 
avec $dx = a^{-n}$ et $x = a^{-n}k$:
$$\Phi (t) = x_0 + \int_{x>0} f(x)\;\e^{ixt}$$
ce qui montre que $f(x)$ n'est pas la transform\'ee de Fourier 
inverse de $\Phi (t)$, mais de $\Phi (t) - x_0$.
Un lissage par convolution est effectu\'e automatiquement par le
simple fait d'avoir omis les grandes valeurs de $t$:  on s'est bien
gard\'e de calculer l'int\'egrale de Fourier ci-dessus sur l'intervalle
$]-\infty\, , +\infty\, [$,  on s'est limit\'e \`a la fen\^etre graphique
de la figure.  Comme on l'a vu \`a la section {\bf 3} du chapitre {\bf VII}, 
cela \'equivaut \`a une convolution par un filtre du type $\sin(k\, x) / x$.
C'est pourquoi aucun bruit discret n'appara{\^\i}t sur la figure ci-dessus. 
En outre il serait trop fin pour la r\'esolution de l'imprimante.
\medskip
Les m\'ethodes expos\'ees dans ce chapitre permettent de d\'eterminer
la fonction carac\-t\'eristique asymptotique de $a^{-n}Z_n$; pour obtenir
la densit\'e, il suffit ensuite d'effectuer num\'eriquement cette 
transform\'ee de Fourier inverse. } 

\vfill\break

\null
\vskip-10mm plus8mm minus8mm
\centerline{\epsfbox{../imgEPS/ch08eps/fig25.eps} }
\vskip3mm
\centerline{\eightpoint figure 25.}
\vskip2mm plus3mm minus1mm
\centerline{\vbox{\hsize=12cm\eightpoint 
Voici (ci-dessus et pages suivantes) quelques autres exemples de
fonctions caract\'eristiques asymptotiques (gra\-phi\-que du haut) et leurs
densit\'es (graphique du bas) pour diff\'erentes loi de probabilit\'e
initiales: ici,  $G(z) = (3 + 2z + z^2 + 4z^3) / 10$;  la moyenne de $Z_1$ 
est donc $a = 1.4$. Pour $x_0$, la probabilit\'e que $Z_n = 0$, le calcul
donne $0.444\, 922\, 417$.  
\medskip
On constate que la densit\'e est nulle pour les abscisses n\'egatives,
ce qui est normal puisque $Z_n$ ne prend que des valeurs positives.
Le maximum de la densit\'e correspond en g\'en\'eral \`a la valeur
moyenne de  $Z_n$, soit $k=a^n$. Bien entendu, $a^n$ tend vers l'infini
avec $n$, on  choisit donc toujours pour les abcisses une \'echelle o\`u
$a^n$ repr\'esente l'unit\'e.  }}

\vskip1pt\break

\null
\vskip3pt
\centerline{\epsfbox{../imgEPS/ch08eps/fig26.eps} }
\vskip12mm
\centerline{\eightpoint figure 26.}
\vskip3mm
\centerline{\eightpoint $G(z) = (1 + 2z + 3z^2 + 4z^3) / 10$}
\centerline{\eightpoint $a = 2.00$} 
\centerline{\eightpoint $x_0 = 0.133\, 726\, 403$}

\vfill\break

\null
\vskip3pt
\centerline{\epsfbox{../imgEPS/ch08eps/fig27.eps} }
\vskip14mm
\centerline{\eightpoint figure 27.}
\vskip3mm
\centerline{\eightpoint $G(z) = (1 + z + z^2 + z^3) / 4$}
\centerline{\eightpoint $a = 1.50$} 
\centerline{\eightpoint $x_0 = 0.418\, 888\, 986$}

\vfill\break

\null
\vskip3pt
\centerline{\epsfbox{../imgEPS/ch08eps/fig28.eps} }
\vskip10mm
\centerline{\eightpoint figure 28.}
\vskip3mm
\centerline{\eightpoint $G(z) = (1 + z + 7z^2 + 3z^3) / 12$}
\centerline{\eightpoint $a = 2.00$} 
\centerline{\eightpoint $x_0 = 0.098\, 950\, 997$}

\vfill\break

\null
\vskip3pt
\centerline{\epsfbox{../imgEPS/ch08eps/fig29.eps} }
\vskip8mm
\centerline{\eightpoint figure 29.}
\vskip3mm
\centerline{\eightpoint Loi de Poisson de moyenne $9$:}
\medskip
\centerline{\eightpoint $G(z) = \e^{9\, (z-1)}$}
\centerline{\eightpoint $a = 9.00$} 
\centerline{\eightpoint $x_0 = 0.000\, 123\, 547$}

\vfill\break

\null
\vskip-6mm plus8mm minus4mm
\centerline{\epsfbox{../imgEPS/ch08eps/fig30.eps} }
\vskip6mm
\centerline{\eightpoint figure 30.}
\vskip3mm
\centerline{\eightpoint Loi de Poisson de moyenne $19$:}
\medskip
\centerline{\eightpoint $G(z) = \e^{19\, (z-1)}$}
\centerline{\eightpoint $a = 19.00$} 
\centerline{\eightpoint $x_0 = 5.6 \cdot 10^{-9}$}
\medskip
\centerline{\vbox{\hsize=12cm\eightpoint  Dans le graphique du haut,
les abscisses sont r\'etr\'ecies d'un facteur  ${1 \over 2}$ par rapport 
\`a celui de la page pr\'ec\'edente. }}

\vskip1pt\break

\null
\vskip-13mm plus8mm minus8mm
\centerline{\epsfbox{../imgEPS/ch08eps/fig31.eps} }
\vskip3mm
\centerline{\eightpoint figure 31.}
\vskip3mm
\centerline{\eightpoint Loi de Poisson de moyenne $39$:}
\medskip
\centerline{\eightpoint $G(z) = \e^{39\, (z-1)}$}
\centerline{\eightpoint $a = 39.00$} 
\centerline{\eightpoint $x_0 = 1.2 \cdot 10^{-17}$}
\medskip
\centerline{\vbox{\hsize=12cm\eightpoint  Dans le graphique du haut,
les abscisses sont r\'etr\'ecies d'un facteur  ${1 \over 2}$ par rapport 
\`a celui de la page pr\'ec\'edente (donc d'un facteur  ${1 \over 4}$ par
rapport \`a celui de la figure 29). }}

\vskip1pt\break

\null
\vskip3pt
\centerline{\epsfbox{../imgEPS/ch08eps/fig32.eps} }
\vskip3mm
\centerline{\eightpoint figure 32.}
\vskip3mm
\centerline{\eightpoint Loi de Bernoulli de moyenne
$5$ et de degr\'e $9$:}
\medskip
\centerline{\eightpoint $G(z) = \bigl( 1 + {5\over 9}\, (z-1)\bigr)^9$}
\centerline{\eightpoint $a = 5.00$} 
\centerline{\eightpoint $x_0 = 0.000\, 301\, 865$}

\vfill\break

\null
\vskip-13mm plus8mm minus8mm
\centerline{\epsfbox{../imgEPS/ch08eps/fig33.eps} }
\vskip3mm
\centerline{\eightpoint figure 33.}
\vskip2mm
\centerline{\eightpoint Loi de Bernoulli de moyenne
$15$ et de degr\'e $19$:}
\medskip
\centerline{\eightpoint $G(z) = \bigl( 1 + {15\over 19}\,
(z-1)\bigr)^{19}$} 
\centerline{\eightpoint $a = 15.00$} 
\centerline{\eightpoint $x_0 = 2.9 \cdot 10^{-14}$}
\medskip
\centerline{\vbox{\hsize=12cm\eightpoint  Dans le graphique du haut,
les abscisses sont r\'etr\'ecies d'un facteur  ${1 \over 2}$ par rapport 
\`a celui de la page pr\'ec\'edente. }}

\vskip1pt\break

Voici un programme qui calcule la {\it fonction caract\'eristique 
asymptotique} $\Phi (t)$ de la variable al\'eatoire $a^{-n}\, Z_n$, en 
utilisant  la m\'ethode d\'evelopp\'ee dans ce chapitre: on calcule 
d'abord  les coefficients $q_1, q_2, q_3,\ldots$ de la fonction $Q(\zeta 
)$, ce qui permet de calculer $\Phi (t) = 1 + Q(it)$ pour de petites 
valeurs de  $t$; puis on utilise l'\'equation fonctionnelle $\Phi (at) = 
G\bigl( \Phi (t)  \bigr)$ pour calculer $\Phi (t)$ pour des valeurs de 
plus en plus  grandes de $t$. C'est le programme qui a engendr\'e la 
figure 23.

\vskip7mm plus2mm minus2mm

\def\q{\hskip5mm}
\def\qq{\hskip14mm}

{\obeylines

{\bf program} gener;
\medskip
 {\bf const}  disc = 16; \q {\eightpoint ** nombre de points initiaux **}
\medskip
 {\bf var}
  a, b, c, d, N, j0, $jj$, L : integer;
  p, q, x, y, s, t, retemp, imtemp, norm, init, initcour : real;
  q1, q2, q3, q4, a1, a2, a3, apow, t0, x0, y0, lna : real;
  tabj : {\bf array} [0..disc] {\bf of} integer;
  tabt, tabx, taby : {\bf array} [0..disc] {\bf of} real;
\bigskip
 {\bf function} rep (u, v : real) : real;  \q \hbox{\eightpoint ** partie\  
r\'eelle de $G$ **}
  {\bf var}
   s, t : real;
 {\bf begin}
  t := d * u;
  s := 3 * t + c;
  t := t + c;
  t := t * u + b;
  rep := (a + u * t - s * v * v) / N;
 {\bf end};
 \bigskip
 {\bf function} imp (u, v : real) : real;  \q \hbox{\eightpoint ** partie\  
imaginaire de $G$ **}
  {\bf var}
   t : real;
 {\bf begin}
  t := d * u;
  t := 3 * t + 2 * c;
  imp := v * (b + u * t - d * v * v) / N;
 {\bf end};
\bigskip
{\bf begin}
 a := 1;  \q {\eightpoint ** a, b, c, d sont les coefficients de $G$ **}
 b := 2;
 c := 0;  \q {\eightpoint ** $G(z)$ = a + b $z$ + c $z^2$ + d $z^3$ **} 
 d := 6;
 N := 9;
 a1 := (b + 2 * c + 3 * d) / N; \q \hbox{\eightpoint ** a1, a2, a3 : \ 
 coefficients de  $F$ **} 
 a2 := (c + 3 * d) / N;  \qq \hbox{\eightpoint ** $F(w)$ = \ 
a1 $w$ + a2 $w^2$  + a3 $w^3$ **} 
 a3 := d / N;
 L := 96; \q {\eightpoint ** demi-largeur (en pixels) du graphique **}
 norm := 24; \q {\eightpoint ** unit\'e (en pixels) sur les abscisses **}
 init := 0.05; \q {\eightpoint ** abscisse du point initial **}
 p := (b + 2 * c + 3 * d) / N;  \q {\eightpoint** moyenne de la loi **}
 q := 1;
\medskip
 q1 := 1; \q \hbox{\eightpoint ** calcul it\'eratif des coefficients de \ 
Taylor q1, ** }
 apow := a1; \q {\eightpoint ** q2, q3, q4 de la fonction $Q(\zeta )$ **}
 q2 := a2 / (a1 * (a1 - 1));
 apow := apow * a1;
 q3 := (2 * a2 * q2 + a3) / (a1 * (apow - 1));
 apow := apow * a1;
 q4 := (a2 * (q2 * q2 + 2 * q3) + 3 * a3 * q2) / (a1 * (apow - 1));
\medskip
 {\bf writeln}; \q {\eightpoint ** affichage **}
 {\bf writeln} (N : 1, ': ', a : 1, ',', b : 1, ',', c : 1, ',', d : 1);
 {\bf writeln};
 {\bf writeln} ('a = ', a1 : 4 : 2);
 {\bf writeln};
 {\bf writeln} ('disc = ', disc : 1);
\medskip
 {\bf MoveTo} (220 + 2 * L, 250);  \q \hbox{\eightpoint ** tra\c{c}age \ 
des axes de coordonn\'ees **}
 {\bf LineTo} (220 + 2 * L, 10);
 {\bf MoveTo} (220 - 2 * L, 250); \  {\bf LineTo} (220 - 2 * L, 10);
 {\bf MoveTo} (220, 255); \  {\bf LineTo} (220, 10);
 {\bf MoveTo} (220 - 2 * L, 250); \  {\bf LineTo} (220 + 2 * L, 250);
\medskip
 {\bf MoveTo} (28, 255);  \q \hbox{\eightpoint ** tra\c{c}age des \ 
graduations en abscisse **}
 {\bf LineTo} (28, 250);
 {\bf MoveTo} (52, 255); \  {\bf LineTo} (52, 250);
 {\bf MoveTo} (76, 255); \  {\bf LineTo} (76, 250);
 {\bf MoveTo} (100, 255); \  {\bf LineTo} (100, 250);
 {\bf MoveTo} (124, 255); \  {\bf LineTo}(124, 250);
 {\bf MoveTo} (148, 255); \  {\bf LineTo} (148, 250);
 {\bf MoveTo} (172, 255); \  {\bf LineTo} (172, 250);
 {\bf MoveTo} (196, 255); \  {\bf LineTo} (196, 250);
 {\bf MoveTo} (244, 255); \  {\bf LineTo} (244, 250);
 {\bf MoveTo} (268, 255); \  {\bf LineTo} (268, 250);
 {\bf MoveTo} (292, 255); \  {\bf LineTo} (292, 250);
 {\bf MoveTo} (316, 255); \  {\bf LineTo} (316, 250);
 {\bf MoveTo} (340, 255); \  {\bf LineTo} (340, 250);
 {\bf MoveTo} (364, 255); \  {\bf LineTo} (364, 250);
 {\bf MoveTo} (388, 255); \  {\bf LineTo} (388, 250);
 {\bf MoveTo} (412, 255); \  {\bf LineTo} (412, 250);
\medskip
 {\bf MoveTo} (220, 220 - 2 * L); \q {\eightpoint ** point de d\'epart **}
\medskip
 lna := ln(a1);
\medskip
 t0 := init;
 {\bf for} $jj$ := 0 {\bf to} disc {\bf do}  
\smallskip
  {\bf begin} \q {\eightpoint ** calcul des points initiaux dans tabt **}
   tabt$\, [jj]$ := t0;
   j0 := round(norm * t0);
   tabj$\, [jj]$ := j0;
   t0 := t0 * exp(lna / disc);
   s := tabt$\, [jj]$ * tabt$\, [jj]$;
   tabx$\, [jj]$ := 1 + s * (q4 * s - q2);
   taby$\, [jj]$ := t0 * (q1 - q3 * s);
   {\bf LineTo} (220 + tabj$\, [jj]$, 220 - round(2 * L * tabx$\, [jj]$));
  {\bf end};
\medskip
 initcour := init;
\medskip
 {\bf while} j0 $\leq$ 2 * L {\bf do}
\smallskip
  {\bf begin} \q \hbox{\eightpoint  **it\'eration \`a partir de\ 
l'\'equation fonctionnelle **}
   initcour := initcour * a1;
   t0 := initcour;
   {\bf for} jj := 0 {\bf to} disc {\bf do}
\smallskip
    {\bf begin}
     tabt$\, [jj]$ := t0;
     j0 := round(norm * t0);
     tabj$\, [jj]$ := j0;
     t0 := t0 * exp(lna / disc);
     retemp := rep(tabx$\, [jj]$, taby$\, [jj]$);
     imtemp := imp(tabx$\, [jj]$, taby$\, [jj]$);
     tabx$\, [jj]$ := retemp;
     taby$\, [jj]$ := imtemp;
     {\bf LineTo} (220 + tabj$\, [jj]$, 220 - round(2 * L * tabx$\, [jj]$));
    {\bf end};
\medskip
  {\bf end};
\medskip
 {\bf MoveTo} (220, 220 - 2 * L);
\medskip
 t0 := init;
 {\bf for} jj := 0 {\bf to} disc {\bf do}
\smallskip
 {\bf begin} \q \hbox{\eightpoint  ** r\'ep\'etition pour la partie\  
gauche du graphique **}
   tabt$\, [jj]$ := t0;
   j0 := round(norm * t0);
   tabj$\, [jj]$ := j0;
   t0 := t0 * exp(lna / disc);
   s := tabt$\, [jj]$ * tabt$\, [jj]$;
   tabx$\, [jj]$ := 1 + s * (q4 * s - q2);
   taby$\, [jj]$ := t0 * (q1 - q3 * s);
   {\bf LineTo} (220 - tabj$\, [jj]$, 220 - round(2 * L * tabx$\, [jj]$));
  {\bf end};
\medskip
 initcour := init;
\medskip
 {\bf while} j0 $\leq$ 2 * L {\bf do}
\smallskip
  {\bf begin}
   initcour := initcour * a1;
   t0 := initcour;
   {\bf for} jj := 0 {\bf to} disc {\bf do}
\smallskip
    {\bf begin}
     tabt$\, [jj]$ := t0;
     j0 := round(norm * t0);
     tabj$\, [jj]$ := j0;
     t0 := t0 * exp(lna / disc);
     retemp := rep(tabx$\, [jj]$, taby$\, [jj]$);
     imtemp := imp(tabx$\, [jj]$, taby$\, [jj]$);
     tabx$\, [jj]$ := retemp;
     taby$\, [jj]$ := imtemp;
     {\bf LineTo} (220 - tabj$\, [jj]$, 220 - round(2 * L * tabx$\, [jj]$));
    {\bf end};
\medskip
  {\bf end};
\medskip
{\bf end}.

} %%% end of \obeylines %%%

\vfill\break




\bye

 pictures 

Four1.7  : le maximum peu marqu\'e correspond \`a x=1.7^n;
Four9  : j=7 correspond \`a x=9^n (max), j=19 \`a la fin de la bosse;
Four19  : j=12 correspond \`a x=19^n, j=23 \`a la fin de la bosse
et j=5 au d\'ebut;
Four1206  : m = 2.22    r=0.145566419;
Four3214  : m = 1.60    r=0.444922417;
Four1234  : m = 2.00    r=0.133726403;
Four1111  : m = 1.50    r=0.418888986;
Four1173  : m = 2.00    r=0.098950997;
FourB9    :   m = 5.00    r=0.000301865;
FourB19   :  m = 15.00  r=2.9e-14;
FourP9   :    m = 9.00    r=0.000123547;
FourP19  :   m = 19.00  r=5.6e-9;
FourP39  :   m = 39.00  r=1.2e-17;

