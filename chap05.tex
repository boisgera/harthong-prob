\input/home/harthong/tex/formats/twelvea4.tex
\input/home/harthong/tex/formats/epsf.tex

\auteurcourant={\sl J. Harthong: probabilit\'es et statistique}
\titrecourant={\sl R\'eunions d'\'ev\'enements}

\pageno=111

\newdimen\blocksize  \blocksize=\vsize \advance\blocksize by -8pt
\font\gtm=cmdunh10
\def\struta{\vrule depth1.8pt width0pt}
\def\strutb{\vrule height7.5pt width0pt}
\def\ata{\hskip-2.5pt}
\def\aub{\hskip1pt}
\def\rom{\hbox{\gtm R}}
\def\qom{\hbox{\gtm Q}}
\def\som{\hbox{\gtm S}}
\def\bgl{\raise1.3pt\hbox{$($}}
\def\bgr{\raise1.3pt\hbox{$)$}}

\null\vskip10mm plus4mm minus3mm

\centerline{\tit V\ata .  R\'EUNIONS D'\'EV\'ENEMENTS.}

\vskip10mm plus3mm minus2mm

{\bf V\ata . 1. Qu'est-ce qu'un \'ev\'enement?}
\medskip
Les probl\`emes scolaires de probabilit\'es sont le plus souvent 
exprim\'es en  langage courant. Par exemple: {\sl on jette trois d\'es;
calculer la probabilit\'e d'avoir trois six}; ou bien: {\sl on jette trois 
d\'es; calculer la probabilit\'e d'avoir un double (c'est-\`a-dire que deux
des trois d\'es montrent la  m\^eme face)}; ou encore {\sl on  jette trois
d\'es; calculer  la probabilit\'e pour que chacun des trois d\'es marque un
nombre  impair}; etc. Dans ces probl\`emes l'\'enonc\'e d\'etermine
toujours un \'ev\'enement dont il s'agira de calculer la probabilit\'e. 
Ainsi, dans le premier des trois exemples pr\'ec\'edents, l'\'ev\'enement
est ``les  trois d\'es marquent six''. Pour r\'esoudre le probl\`eme, il faut
traduire ce langage imag\'e sous forme math\'ematique, c'est-\`a-dire
trouver d'abord l'espace $\Omega$ qui convient le mieux, puis 
d\'eterminer le sous-ensemble $A$ de $\Omega$ qui repr\'esente 
l'\'ev\'enement. Ainsi, la premi\`ere proposition de l'\'enonc\'e {\sl on
jette trois d\'es} se traduit math\'ematiquement en prenant pour espace
$\Omega$   l'ensemble des groupes de trois chiffres de 1 \`a 6, autrement
dit l'ensemble des mots de trois lettres qu'on peut \'ecrire avec l'alphabet
$\{ 1, 2, 3,4, 5, 6 \}$, soit un ensemble de cardinal $216$ ($\Omega$ est
toujours l'ensemble de tous les r\'esultats \'equiprobables possibles). 
La seconde proposition de l'\'enonc\'e d\'etermine l'\'ev\'enement:
chaque r\'esultat {\it possible} est la famille des trois chiffres donn\'es
par chaque d\'e, mais l'\'ev\'enement $A$ cherch\'e est
l'ensemble des familles distingu\'ees par l'\'enonc\'e: ici celles dont les
trois chiffres sont six; il n'y en a qu'une sur les $216$, donc $A$ est un
sous-ensemble \`a un seul \'el\'ement, de cardinal $1$, donc de
probabilit\'e $1/216$. Dans le deuxi\`eme probl\`eme, l'\'ev\'enement
d\'ecrit par l'\'enonc\'e est ``avoir un double''. Le sous-ensemble $B$ de
$\Omega$ correspondant \`a cela est l'ensemble des familles ayant deux
chiffres \'egaux, ou l'ensemble des mots ayant une lettre r\'ep\'et\'ee; le
meilleur argument pour le d\'enombrer consiste \`a prendre son
compl\'ementaire $B'$, qui est l'ensemble des mots dont les trois lettres
sont toutes distinctes: c'est donc l'ensemble des mots ayant toutes leurs
lettres distinctes (second cas de d\'enombrement du chapitre {\bf II}),
dont le cardinal est  $6 \cdot 5 \cdot 4 = 120$. Donc on conclut que $\#  
B = 216 - 120 = 96$. Dans le troisi\`eme probl\`eme, l'\'ev\'enement
sugg\'er\'e par l'\'enonc\'e est ``chacun des trois d\'es donne un chiffre
impair''. Le sous-ensemble $C$ de $\Omega$ correspondant est l'ensemble
des familles dont les trois  chiffres sont tous impairs. On peut remarquer
que $C = C_1 \cap C_2 \cap  C_3$, o\`u $C_1$ est l'ensemble des
familles dont le {\it premier} chiffre est impair (et les autres
indiff\'erents), $C_2$ l'ensemble des familles dont le {\it second}
chiffre est impair, et $C_3$ l'ensemble des familles dont le {\it
troisi\`eme} chiffre est impair. On peut dire que  $C_1$ est la
traduction math\'ematique de ''le premier d\'e donne un chiffre impair'',
et de m\^eme pour $C_2$ et $C_3$. L'intersection des trois ensembles
correspond \`a la phrase ''le premier d\'e  donne un chiffre impair {\it et}
le deuxi\`eme aussi {\it  et} le troisi\`eme aussi''. Il ne reste plus qu'\`a
remarquer que les trois \'ev\'enements $C_1, C_2, C_3$ sont
stochastiquement ind\'ependants (reflet de l'ind\'ependance causale des
trois d\'es), de sorte que ${\cal P}\, (C) = {\cal P}\, (C_1) \times {\cal 
P}\, (C_2) \times {\cal P}\, (C_3)$. Les probabilit\'es de $C_1, C_2, C_3$
sont faciles \`a calculer (elles valent  ${1 \over 2}$ car chacun des trois
d\'es s\'epar\'ement a une chance sur deux de donner un chiffre impair),
donc ${\cal P}\, (C) = {1 \over 8}$. 
\medskip 
On constate dans chacun de ces trois probl\`emes, que ce qui permet de 
{\it calculer} est la traduction math\'ematique pr\'ealable; c'est elle 
qui permet le d\'enombrement syst\'ematique. Mais elle a aussi permis 
le recours \`a des op\'erations ensemblistes (compl\'ementaire pour le
second probl\`eme et intersection pour le troisi\`eme) qui, gr\^ace a
leurs  propri\'et\'es alg\'ebriques, ont facilit\'e le calcul. 
\medskip 
Consid\'erons encore le quatri\`eme probl\`eme que voici: {\sl on jette
trois d\'es; calculer la probabilit\'e pour que l'un au moins des trois
d\'es donne un chiffre impair}. Cette fois l'\'ev\'enement $D$ d\'ecrit
par l'\'enonc\'e est ``le premier d\'e donne un chiffre impair {\it ou} le
second d\'e donne un  chiffre impair {\it ou} le troisi\`eme d\'e donne 
un chiffre impair'' ({\it ou} non exclusif). Donc $D = C_1 \cup C_2 \cup
C_3$. Pour pouvoir calculer comme dans le troisi\`eme probl\`eme, il
faudrait disposer  d'une formule donnant la probabilit\'e d'une r\'eunion
d'\'ev\'enements \ftn{1}{On peut aussi consid\'erer le compl\'ementaire
de $D$, qui est l'intersection des compl\'e\-men\-taires des $C_j$, ce qui 
nous ram\`ene au probl\`eme pr\'ec\'edent. Mais nous y reviendrons.}. Or
une telle formule existe, c'est la formule de Poincar\'e.

\vskip6mm plus3mm minus3mm


{\bf V\ata .\aub 2. Formule de Poincar\'e.}
\medskip
Commen\c{c}ons par le cas extr\^emement simple de la r\'eunion de deux
\'ev\'enements. Il est clair que si deux ensembles $A_1$ et $A_2$ sont
disjoints, le cardinal de $A_1 \cup A_2$ est la somme des cardinaux:
$\# (A_1 \cup A_2) = \# A_1 + \# A_2$. Mais si les deux ensembles ne
sont pas disjoints et qu'on \'ecrit la liste num\'erot\'ee des 
\'el\'ements de  $A_1$, suivie imm\'ediatement par la liste
num\'erot\'ee des \'el\'ements de  $A_2$ (cette op\'eration est
appel\'ee {\it concat\'enation}), on obtient une liste contenant $\# A_1
+ \# A_2$ \'el\'ements, mais dans laquelle tous  ceux qui sont \`a la
fois dans $A_1$ et dans $A_2$, c'est-\`a-dire ceux  qui sont dans $A_1
\cap A_2$, figurent deux fois. Pour compenser cette r\'ep\'etition, il
faut donc retrancher le cardinal de $A_1 \cap A_2$; ainsi l'\'egalit\'e
suivante est correcte: 
$$\# (A_1 \cup A_2) = \# A_1 + \# A_2 - \# (A_1 \cap A_2) \eqno
(V.1.)$$  
Comment traiter le cas d'une r\'eunion de trois ensembles $A_1
\cup  A_2 \cup A_3$? On pourrait proc\'eder de  m\^eme, concat\'ener 
les trois listes d'\'el\'ements, puis regarder combien sont compt\'es
plusieurs fois. On comprend ais\'ement que ceux qui sont dans $A_1 
\cap A_2 \cap A_3$ sont compt\'es trois fois, ceux qui sont dans $A_1
\cap A_2$, $A_2 \cap A_3$, ou $A_1 \cap A_3$ sans \^etre dans $A_1
\cap A_2 \cap A_3$ sont compt\'es deux fois. Il faut donc retrancher 
$\# (A_1\cap A_2)$, $\# (A_2\cap A_3)$, et $\# (A_1\cap A_3)$ \`a $\#
A_1 + \# A_2  + \# A_3$, mais du coup on aura retranch\'e trois fois
ceux qui sont dans  $A_1 \cap A_2 \cap A_3$ alors qu'il n'aurait fallu 
les retrancher que deux fois; on compensera donc encore en les {\it
rajoutant} une fois. Ainsi l'\'egalit\'e suivante sera correcte: 
$$\eqalign{
\# (&A_1 \cup A_2 \cup A_3)\quad = \quad\,\# A_1 + \# A_2 + 
\# A_3 \cr 
&- \# (A_1 \cap A_2) - \# (A_2 \cap A_3) - \# (A_1 \cap A_3) \cr
&\hskip35mm + \# (A_1 \cap A_2 \cap A_3) \cr } \eqno (V.2.)$$
En principe on pourrait traiter ainsi, par concat\'enations compens\'ees, 
les r\'eunions d'un nombre arbitraire d'\'ev\'enements; mais l'examen de
tous les cas possibles d'intersections partielles est un exercice un peu
abstrait. Il vaut mieux se contenter d'abord de {\it deviner} une formule
g\'en\'erale, qu'on d\'emontrera ensuite rigoureusement par r\'ecurrence.
La formule qu'on devine par les concat\'enations compens\'ees est la
suivante 
$$\eqalign{
\# (&A_1 \cup A_2 \cdots \cup A_n)\quad = 
\quad\sum_{j=1}^{j=n} \# A_j \cr
&\hskip24mm -\sum\sum_{\hbox{\hskip-18pt$\scriptstyle 1\leq j_1 
< j_2\leq n$}} \# (A_{j_1} \cap A_{j_2}) \cr
&\hskip18mm +\sum\;\sum\sum_{\hbox{\hskip-36pt$\scriptstyle 1\leq  
j_1 < j_2 < j_3  \leq n$}} \# (A_{j_1} \cap A_{j_2} \cap A_{j_3}) \cr
&\hskip12mm -\sum\;\sum\;\sum\;\sum_{\hbox{\hskip-55pt$
\scriptstyle 1\leq j_1 <  j_2 < j_3 < j_4 \leq n$}} \# (A_{j_1} \cap
A_{j_2} \cap A_{j_3}  \cap A_{j_4}) \cr
\noalign{\medskip}
&\hskip36mm \cdots \cr
\noalign{\medskip}
&\hskip19mm + (-1)^{n-1} \# (A_{1} \cap A_{2} \cap A_{3} 
\cdots \cap  A_{n}) \cr } \eqno (V.3.)$$
Une mani\`ere plus condens\'ee d'exprimer cette relation est la suivante:
$$\# (A_1 \cup A_2 \cdots \cup A_n) = \som_1 - \som_2 + \som_3 -
\cdots \pm\som_n \eqno (V.4.)$$
o\`u
\smallskip
--- $\som_1$ est la somme des cardinaux de tous les $A_j$ [cette
somme contient $n$ termes],  
\smallskip
--- $\som_2$ est la somme des cardinaux de toutes les intersections
de deux des $A_j$, telles que $A_1 \cap A_2$ ou $A_3 \cap A_7$
[contient ${n \choose 2} = n(n-1)/2$ termes],   
\smallskip
--- $\som_3$ est la somme des cardinaux de toutes les intersections
de trois des $A_j$, telles que $A_1 \cap A_5 \cap A_6$ ou $A_3 \cap
A_7 \cap A_{11}$ [contient ${n \choose 3} = n(n-1)(n-2)/6$
termes],   
\smallskip
--- $\som_4$ est la somme des cardinaux de toutes les intersections
de quatre des $A_j$ [contient ${n \choose 4}$ termes],   
\medskip
\hskip1cm $\ldots$
\medskip
--- $\som_{n-1}$ est la somme des cardinaux de toutes les 
intersections de $n-1$ des $A_j$, [contient ${n\choose n-1} = n$ termes],
\smallskip
--- $\som_{n}$ ne contient qu'un seul terme, $\# (A_1 \cap A_2 \cap
A_3 \cap A_4 \cdots \cap A_n$), car il n'y a qu'une seule intersection 
de $n$ d'entre les $A_j$ [${n\choose n} = 1$].
\medskip
On peut maintenant se proposer de d\'emontrer cette formule par
r\'ecur\-rence: pour amorcer la r\'ecurrence, nous disposons d\'ej\`a du 
cas $n=2$ (de toute fa\c{c}on la formule n'a vraiment de sens que pour $n
\geq 2$). Supposons qu'elle soit vraie pour $n-1$ \'ev\'enements et
montrons qu'elle doit alors automatiquement \^etre vraie aussi pour $n$
\'ev\'enements $A_1, A_2, A_3, \ldots A_n$.
\medskip
Or posons $B = A_2 \cup A_3 \cup A_4 \cdots \cup A_n$. 
On voit que la r\'eunion des $A_j$ de $j=1$ \`a $j=n$, est identique \`a 
$A_1 \cup B$. Or il est d\'ej\`a \'etabli que $\# (A_1 \cup B) = \# A_1 +  
\# B - \# (A_1 \cap B)$. Mais d'autre part $B$ est la r\'eunion de $n-1$
\'ev\'enements, on peut donc utiliser l'hypoth\`ese de r\'ecurrence pour 
d\'ecomposer $\# B$; de m\^eme $A_1 \cap B$ est \'egal \`a $(A_1 \cap 
A_2) \cup (A_1 \cap A_3) \cup (A_1 \cap A_4) \cdots\;\cup (A_1 \cap
A_n)$, c'est-\`a-dire \`a une r\'eunion de $n-1$ \'ev\'enements, auxquels
aussi on peut donc appliquer l'hypoth\`ese de r\'ecurrence. Ce qui donne 
$$\# B = \qom_1 - \qom_2 + \qom_3 \cdots \pm \qom_{n-1}$$
o\`u $\qom_j$ est la somme des cardinaux des intersections de $j$ 
parmi les $n-1$ \'ev\'enements $A_2,\, A_3,\, A_4, \ldots A_n$. Il 
s'agit donc des intersections de $j$ des $n$ \'ev\'enements $A_1,\,
A_2,\,  A_3,\, A_4, \ldots A_n$, mais {\it excluant} l'\'ev\'enement 
$A_1$.  De  m\^eme (on reprend \`a partir d'ici la notation multiplicative
pour l'intersection):
$$\# (A_1B) = \rom_1 - \rom_2 + \rom_3 \cdots \pm\rom_{n-1}$$ 
o\`u $\rom_j$ est la somme des cardinaux des
intersections de $j$  parmi les $n-1$ \'ev\'enements $A_1A_2,\,
A_1A_3,\, A_1A_4,\, \ldots A_1A_n$: ce sont donc les intersections de
$j+1$ parmi les $A_1,\, A_2,\, A_3,\, A_4, \ldots A_n$, {\it dont l'un est
obligatoirement} $A_1$. On peut donc dire que $\vrule height 12pt width 
0pt \rom_{j-1}  + \qom_j$  est la somme des cardinaux de {\it toutes} les
intersections de $j$ parmi les $n$ \'ev\'enements $A_1,\, A_2,\, A_3,\, 
A_4, \ldots A_n$: c'est la somme pour celles qui contiennent le facteur
$A_1$ $\vrule height 13pt width 0pt \bgl\;\rom_{j-1}\;\bgr$ plus la
somme pour celles qui ne contiennent pas le facteur $A_1$ $\vrule height
12pt  width 0pt \bgl\,\qom_{j}\,\bgr$. On a donc pour les expressions
$\som$  d\'efinies plus haut  
$$\som_j = \rom_{j-1} + \qom_j$$
En appliquant alors l'hypoth\`ese de r\'ecurrence comme indiqu\'e, on 
obtient 
$$\eqalignno{
\# (A_1 \cup A_2 \cup A_3 \cdots \cup A_n)\quad  &= \quad\# (A_1
\cup B) \cr
\noalign{\medskip}
&= \quad\# A_1 + \# B - \# (A_1 B) \cr
\noalign{\medskip}
&= \quad\# A_1 + \qom_1 - \qom_2 + \qom_3 \cdots \pm \qom_{n-1}\cr
&\hskip12mm  - \rom_1 + \rom_2 -\rom_3 + \cdots \pm \rom_{n-1}\cr
\noalign{\medskip}
&= \quad \# A_1 + \# A_2 + \# A_3 + \cdots + \# A_n \cr
&\hskip12mm  - (\qom_2 + \rom_1) + (\qom_3 + \rom_2)\cr 
&\hskip12mm -(\qom_4 + \rom_3)\cdots\pm (\qom_n + \rom_{n-1}) \cr
\noalign{\medskip}
&=\quad\som_1 - \som_2 + \som_3 - \som_4 \cdots \pm \som_n\cr }$$
Ceci est bien ce que nous voulions obtenir; nous avons donc ainsi
rigoureusement d\'emontr\'e la formule de Poincar\'e par r\'ecurrence.
\medskip
On peut revenir au probl\`eme : {\sl on jette trois d\'es; quelle est la
probabilit\'e pour que l'un au moins donne un r\'esultat impair?}
L'\'ev\'enement $C_1$ est form\'e des familles de trois chiffres dont 
le premier est impair; il y a donc trois possibilit\'es pour le premier
($1$, $3$, et $5$), et six possibilit\'es pour chacun des deux autres,
de sorte que $\#C_1 = 3 \times 6 \times 6 = 108$. De m\^eme 
$\#C_2 = 6 \times 3 \times 6 = 108$ et $\#C_3 = 6 \times 6 \times 
3 = 108$. Pour appliquer la formule de Poincar\'e (ici pour trois
\'ev\'enements, soit $V.2$) il faut conna{\^\i}tre \'egalement les 
cardinaux  des intersections par deux et par trois; mais cela est ais\'e:
$\#(C_1C_2) =  3 \times 3 \times 6 = 54$, $\#(C_2C_3) =  6 \times 3
\times 3 = 54$, $\#(C_1C_3) =  3 \times 6 \times 3 = 54$; enfin
$\#(C_1C_2C_3) =  3 \times 3 \times 3 = 27$. Ainsi 
$$\#(C_1 \cup C_2 \cup C_3) = 3 \times108 - 3 \times 54 + 27 = 189$$
et la probabilit\'e pour que {\sl l'un au moins} des trois d\'es marque un 
chiffre impair est $189/216 = 7/8$.

\vskip6mm plus3mm minus3mm

{\bf V\ata . 3. Le probl\`eme des co{\"\i}ncidences fortuites.}
\medskip
Le probl\`eme typique que l'on r\'esoud par la formule de Poincar\'e est 
celui des {\it co{\"\i}ncidences fortuites} ou probl\`eme de Montmort
(math\'ematicien fran\c{c}ais du $XVIII^e$ si\`ecle); dans la litt\'erature
anglo-saxonne {\it random matches problem}. Ce probl\`eme s'\'enonce 
ainsi: 
\smallskip
{\narrower
{\sl \noindent $n$ lettres  sont adress\'ees chacune \`a un destinataire 
d\'etermin\'e; mais on les distribue au hasard entre leurs $n$ 
destinataires.  Quelle est la probabilit\'e pour qu'aucun des $n$
destinataires ne re\c{c}oive  la lettre qui lui \'etait destin\'ee ?} \par }
\smallskip
Ce probl\`eme est l'un des plus anciens du Calcul des probabilit\'es. 
Il est \'evoqu\'e pour la premi\`ere fois dans le livre de Montmort \`a 
propos du {\it jeu des Treize}.  Dans ce jeu,  on m\'elangeait treize  
cartes de valeurs 1 \`a 13 (les trois derni\`eres \'etant valet, dame,  
roi,  consid\'er\'es comme num\'eros 11, 12, 13).  Puis il fallait retirer 
les cartes une \`a une et voir si la $k^{\rm e}$ tir\'ee avait aussi la 
valeur $k$.  Montmort a pos\'e le probl\`eme de trouver la probabilit\'e 
pour que cela se produise au moins une fois. 
\medskip 
L'historien Todhunter rapporte que c'est probablement Nicolas 
Bernoulli et non Montmort qui a trouv\'e la r\'eponse;  en effet,  la  
premi\`ere \'edition du livre de Montmort rapporte le r\'esultat mais ne 
comporte aucune d\'emonstration;  puis dans la seconde \'edition Montmort 
donne deux d\'emons\-tra\-tions en disant qu'il les tient de Nicolas 
Bernoulli (Montmort et N. Bernoulli entretenaient une correspondance
r\'eguli\`ere). 
\medskip 
La c\'el\'ebrit\'e de ce probl\`eme provient de ce que la limite de cette
probabilit\'e lorsque $n$ tend vers l'infini,  est \'egale \`a $1/\e$ (nous
ferons ce calcul ci-dessous).  Qu'on puisse rencontrer des nombres tels 
que $\e$ ou $\pi$ dans un probl\`eme concret de probabilit\'es,  \'etait 
per\c{c}u comme une d\'ecouverte fantastique,  et s'apparentait au miracle
de la g\'eom\'etrie qui avait d\'ej\`a fascin\'e les Anciens\ftn{2}{ 
Platon, {\it La R\'epublique}, Livre VII.}. 
\medskip
L'explication de ce paradoxe se trouve dans le principe d'invariance
qui conduit \`a postuler l'\'equiprobabilit\'e des destinataires.
L'\'enonc\'e du probl\`eme dit:  {\sl on distribue les lettres au hasard
entre les destinataires},  ce que tout le monde traduit par ``tous les
destinataires sont \'equiprobables''.  Comme cela a d\'ej\`a \'et\'e
discut\'e dans ce livre (section {\bf I. 3} {\it La signification de
l'\'equiprobabilit\'e}),  l'\'equiprobabilit\'e est postul\'ee a priori, 
\`a
partir d'une invariance que le sens commun,  d'origine empirique, fait
percevoir comme \'evidente:  chaque lettre est
suppos\'ee avoir autant de chances d'arriver chez n'importe lequel des 
$n$ destinataires.  D'un point de vue empirique,  cela signifie que si on
refait un grand nombre de fois l'exp\'erience consistant \`a distribuer les
$n$ lettres,  on constatera que statistiquement elles se r\'epartissent 
\`a peu pr\`es uniform\'ement;  nous verrons au chapitre {\bf  XI} que 
les variations correspondant \`a cet ``\`a peu pr\`es'' doivent \^etre de
l'ordre de $1/\sqrt{N}$, $N$ \'etant le nombre de fois qu'on r\'ep\`ete 
l'exp\'erience;  l'\'equivalence des destinataires sera donc 
vraie \`a $1/\sqrt{N}$ pr\`es:  pour v\'erifier exp\'erimentalement
l'\'equiprobabilit\'e au millioni\`eme pr\`es,  il faudrait refaire
l'exp\'erience $10^{12}$ fois !  Et selon toute vraisemblance,  on
constaterait alors que l'\'equiprobabilit\'e n'est pas parfaite:  pour la
distribution au hasard les lettres devraient \^etre \`a chaque fois 
rem\'elang\'ees et tir\'ees au sort dans un panier,  mais l'une pourrait 
\^etre l\'eg\`erement plus lourde et se retrouver plus souvent au fond du 
panier,  ou une autre l\'eg\`erement \'ecorn\'ee pourrait s'accrocher aux 
autres et avoir plus de chances de remonter sur le dessus du panier.  En 
un mot,  l'\'equiprobabilit\'e empirique est par nature approximative. 
Cependant,  comme l'\'equiprobabilit\'e est un principe qualitatif et non
quantitatif,  il ne se trouve pas diminu\'e par le caract\`ere approximatif
de sa v\'erification empirique.  Il en va de m\^eme dans d'autres domaines;
si par exemple on \'enonce ``la constante $c$ de l'\'electromagn\'etisme
(vitesse de la lumi\`ere dans  le vide) vaut $2.997\, 925 \cdot 10^{8}\;
m/s$'',  une mesure plus pr\'ecise augmentera l'information ainsi
exprim\'ee,  car elle est uniquement quantitative et ne contient rien
d'autre que la connaissance de d\'ecimales.  Mais si on \'enonce ``la 
constante $c$ de l'\'electromagn\'etisme est la m\^eme dans tous les
rep\`eres galil\'eens'',  une v\'erification plus pr\'ecise de ce principe 
ne l'am\'eliore plus (tout au plus elle peut conduire \`a l'infirmer).  
\medskip
Les philosophes de la Gr\`ece antique ont \'et\'e fascin\'es devant la
possibilit\'e de conna{\^\i}tre le nombre $\pi$ par les pures 
math\'ematiques,  avec une pr\'ecision infinie,  alors que la mesure 
physique d'une circonf\'erence ne pouvait jamais donner plus de trois
d\'ecimales.  Platon en d\'eduisait dans le texte c\'el\`ebre de {\it La
R\'epublique} (op. cit.) que l'on peut donc atteindre par la pens\'ee une
v\'erit\'e situ\'ee au-del\`a de tout ce que l'exp\'erience sensible permet
de conna{\^\i}tre.  La raison de ce paradoxe (ou de cette illusion) est que le
calcul de $\pi$ par la g\'eom\'etrie r\'esulte du principe {\it qualitatif}
d'une invariance de l'espace par rotation.  Ce principe qualitatif est
d'origine exp\'erimentale,  mais le fait qu'il soit qualitatif permet le
glissement conceptuel d'une validit\'e approch\'ee \`a une validit\'e
parfaite.  Si on a mesur\'e $\pi$ exp\'erimentalement avec six 
d\'ecimales,  on ne peut pas deviner les d\'ecimales suivantes;  mais si on 
a \'etabli \`a six d\'ecimales pr\`es l'invariance de l'espace par rotation, 
on peut deviner ou croire que cette invariance se maintiendra pour des
mesures plus pr\'ecises (du moins jusqu'au jour o\`u ces mesures seront
devenues pr\'ecises au point de montrer les limites de l'invariance) et en
d\'eduire math\'ematiquement une valeur de $\pi$ ``exacte'',  avec une
infinit\'e de d\'ecimales.  
\medskip
La fascination des contemporains de Montmort devant la probabilit\'e 
\'egale \`a $1/\e$ est la m\^eme que celle de Platon devant la valeur
math\'e\-ma\-ti\-que\-ment exacte de $\pi$.  L'\'equiprobabilit\'e rigoureuse 
n'existe jamais en pratique;  mais si on la traduit sous forme de v\'erit\'e 
math\'ematique (on appelle cela aujourd'hui ``construire un mod\`ele 
math\'ematique''),  elle conduit \`a des nombres qui peuvent \^etre 
calcul\'es avec une pr\'ecision infinie.  Il est cependant bien \'evident 
que si on distribue {\it r\'eellement},  {\it mat\'eriellement},  les 
lettres par tirage au sort dans un panier,  l'\'equi\-pro\-ba\-bi\-lit\'e
ne  sera satisfaite qu'approximativement (tout comme pour les dates de
naissances du probl\`eme de la section {\bf II.2.}),  de telle sorte que  
les premi\`eres d\'ecimales du mod\`ele math\'ematique seront 
correctes,  mais les suivantes fantaisistes.  Le calcul par les pures 
math\'ematiques donnera $1/\e = 0.367\, 879 \ldots$,  mais la
probabilit\'e pour qu'aucune lettre n'arrive \`a son destinataire 
sera de
$37\%$.
\medskip
Voyons maintenant comment on r\'esoud le probl\`eme de Montmort. 
L'espace $\Omega$ est l'ensemble de toutes les distributions possibles.
Chaque distribution correspond \`a une permutation des lettres
par rapport \`a la distribution correcte; comme il y a $n$ lettres,  il y a
$n!$ permutations,  soit $\#\Omega=n!$ L'\'ev\'enement d\'efini par 
l'\'enonc\'e est ${\cal A}$ : ``aucun destinataire ne re\c{c}oit la lettre qui
lui est destin\'ee''. Son compl\'ementaire est ${\cal B}$ : ``{\it au moins
un} destinataire re\c{c}oit la lettre qui lui est destin\'ee''.  L'expression 
{\it au moins un}  correspond \`a une r\'eunion, celle des \'ev\'enements:
$$\eqalignno{
{\cal B}_1 &: \hbox{``le destinataire $N^\circ 1$ re\c{c}oit la 
lettre qui lui est destin\'ee''}\cr
{\cal B}_2 &: \hbox{``le destinataire $N^\circ 2$ re\c{c}oit la 
lettre qui lui est destin\'ee''}\cr
\noalign{\vskip6pt plus8pt minus4pt}
&\hskip25mm \cdots \hskip25mm \cdots \cr
\noalign{\vskip6pt plus8pt minus4pt}
{\cal B}_n &: \hbox{``le destinataire $N^\circ n$ re\c{c}oit la 
lettre qui lui est destin\'ee''}\cr }$$
Ainsi ${\cal B} = {\cal B}_1 \cup {\cal B}_2 \cup {\cal B}_3
\cdots \cup {\cal B}_n$, et on peut obtenir la probabilit\'e de ${\cal B}$
par la formule de Poincar\'e.
\medskip
{\eightpoint
Ici, il convient peut-\^etre de faire deux remarques: 
\smallskip
{\bf 1.} Le lecteur peut trouver bizarre que le probl\`eme
de Montmort, r\'esolu au d\'ebut du $XVIII^e$ si\`ecle, fasse appel \`a la 
formule de Poincar\'e (math\'ematicien fran\c{c}ais {\oldstyle 1854} -- 
{\oldstyle 1912}). Il est clair qu'on savait calculer la probabilit\'e
d'une r\'eunion bien avant! Les d\'emonstrations de Nicolas Bernoulli,
mentionn\'ees plus haut, ne faisaient \'evidemment pas appel \`a la 
``formule de Poincar\'e'', mais \`a des proc\'ed\'es sp\'ecifiques. 
\smallskip 
L'appellation ``formule de Poincar\'e'', ou ``th\'eor\`eme de Poincar\'e'',  
ou encore ``identit\'e de Poincar\'e'' pour $V.3$ ou $V.4$ provient du 
fait qu'on la trouvait sous cette forme g\'en\'erale dans le {\it Calcul 
des probabilit\'es} (op. cit. chap {\bf I}, note 6 et bibliographie) de  
Henri Poincar\'e. Mais les cas particuliers $V.1$ et $V.2$ \'etaient 
connus d\`es la fin du $XVII^{\rm e}$ si\`ecle (Jacques Bernoulli), et  
leur g\'en\'eralisation par r\'ecurrence \'etait \'evidemment d\`es  
cette \'epoque \`a la port\'ee de n'importe quel math\'ematicien. La 
formule n'est donc pas  une d\'ecouverte de Poincar\'e: il l'a  
simplement rendue populaire. 
\smallskip
{\bf 2.} Si ${\cal B} = {\cal B}_1 \cup {\cal B}_2 \cup {\cal B}_3
\cdots \cup {\cal B}_n$, alors ${\cal A} = {\cal A}_1 \cap {\cal A}_2 
\cap {\cal A}_3 \cdots \cap {\cal A}_n$, o\`u ${\cal A}_j$ est le
compl\'ementaire de ${\cal B}_j$ et ${\cal A}$ celui de ${\cal B}$. 
Pourquoi ne pas utiliser la formule beaucoup plus simple ${\cal P}\, ({\cal
A}) = {\cal P}\, ({\cal A}_1)  \times {\cal P}\, ({\cal A}_2)  \times {\cal
P}\, ({\cal A}_3) \cdots \times {\cal P}\, ({\cal A}_n)$ ?  {\bf R\'eponse:}
ce serait faux,  car les \'ev\'enements ${\cal A}_j$ ne sont pas
stochastiquement ind\'ependants.  ${\cal A}_j$ est l'\'ev\'enement ``le
destinataire $N^\circ j$ ne re\c{c}oit pas la lettre qui lui \'etait
destin\'ee'':  donc un autre la re\c{c}oit,  \`a qui elle n'\'etait pas
destin\'ee
non plus,  ce qui implique que le fait d'appartenir \`a l'undes ${\cal A}_j$
augmente la probabilit\'e d'appartenir aussi \`a l'undes autres.  La
propri\'et\'e du produit est simple et pratique, mais ne marche
malheureusement que pour des \'ev\'enements ind\'ependants, tandis que la formule de Poincar\'e est valable pour n'importe quelle
sorte
d'\'ev\'enements. \par} 
\medskip
Pour calculer $\#{\cal B}$ par la formule de Poincar\'e, il faut 
d'abord conna{\^\i}tre les cardinaux des ${\cal B}_j$, puis des 
${\cal B}_{j_1} \cap {\cal B}_{j_2}$, puis des ${\cal B}_{j_1} \cap 
{\cal B}_{j_2} \cap {\cal B}_{j_3}$, et ainsi de suite.
De l'\'equivalence des diff\'erents destinataires, on peut d\'ej\`a 
d\'eduire  que les ${\cal B}_j$ ont tous le m\^eme cardinal (sinon, cela
indiquerait que certains destinataires seraient plus \'egaux que 
d'autres). De m\^eme, le cardinal de ${\cal B}_{j_1} \cap {\cal B}_{j_2}$
ne peut  pas d\'ependre de $j_1$ ou de $j_2$, ni celui de ${\cal B}_{j_1}
\cap {\cal B}_{j_2}  \cap {\cal B}_{j_3}$ de $j_1$, $j_2$ ou $j_3$.
\medskip
Sachant que les intersections par deux sont au nombre de ${n \choose 
2}$, que les intersections par trois sont au nombre de ${n \choose 3}$,
etc. il suffit donc de calculer les cardinaux de ${\cal B}_1$, de ${\cal
B}_1{\cal B}_2$, de ${\cal B}_1{\cal B}_2{\cal B}_3$, $\ldots$ ${\cal
B}_1{\cal B}_2 \cdots {\cal B}_n$, et on aura 
$$\#{\cal B} = n\, \#{\cal B}_1 - {n \choose 2}\, \#({\cal B}_1 {\cal
B}_2) + {n \choose 3}\, \#({\cal B}_1{\cal B}_2{\cal B}_3) \cdots \pm 
{n \choose n}\, \#({\cal B}_1{\cal B}_2\cdots {\cal B}_n) \eqno (V.5.)$$ 
Or ${\cal B}_1$ \'etant l'\'ev\'enement ``le destinataire $N^\circ 1$ re\c
coit la  lettre qui lui est destin\'ee'', son cardinal est \'evidemment le 
nombre de permutations des $n-1$ lettres restantes,  soit $(n-1)!$
L'\'ev\'enement ${\cal B}_1{\cal B}_2$ est ``les destinataires $N^\circ1$
et $N^\circ2$ ont re\c{c}u les lettres qui leur sont destin\'ees'' donc son
cardinal est le nombre de permutations des $n-2$ lettres restantes, soit
$(n-2)!$ En g\'en\'eral, $\#({\cal B}_1{\cal B}_2\cdots {\cal B}_k) =
(n-k)!$ Substituant dans $(V.5)$ on obtient
$$\eqalign{
\#{\cal B} &= n\, (n-1)! - {n \choose 2}\, (n-2)! + \cdots
- (-1)^k\, {n \choose k}\, (n-k) \cdots - (-1)^n \cr
&= -\; n!\; \sum_{k=1}^{k=n} (-1)^k\, {1 \over k!} \cr}$$
On obtient les probabilit\'es en divisant par $\#\Omega = n!$ d'o\`u
$${\cal P}\, ({\cal B}) = - \sum_{k=1}^{k=n} (-1)^k\, {1 \over k!}$$
Passant au compl\'ementaire, cela donne
$$\eqalign{
{\cal P}\, ({\cal A}) &= 1 - {\cal P}\, ({\cal B}) =
1 + \sum_{k=1}^{k=n} (-1)^k\, {1 \over k!} \cr
&= 1 - 1 + {1\over 2!} - {1\over 3!} + {1\over 4!} \cdots + (-1)^n\, 
{1\over n!} \cr } \eqno (V.6.)$$ 
En faisant tendre $n$ vers l'infini, on reconna{\^\i}t le d\'eveloppement 
en s\'erie de MacLaurin de $\e^{-1}$, ce qui montre que pour $n$ grand,
la probabilit\'e ${\cal P}\, ({\cal A})$ est proche de $1/\e$. Une autre
observation est aussi que cette probabilit\'e est la m\^eme quel que 
soit  le nombre de destinataires, pourvu qu'il soit grand.
\medskip
L'expression $V.6$, en tant qu'expression alg\'ebrique, garde la trace    
de la formule de Poincar\'e dont elle est issue; elle ne se simplifie pas
davantage, montrant par l\`a que le probl\`eme de Montmort rel\`eve
intrins\`equement d'une r\'eunion d'\'ev\'enements non disjoints et 
qu'il ne peut pas faire l'objet d'une r\'eduction \`a un probl\`eme plus
simple. Ce type de constat permet souvent en math\'ematique de 
s'assurer qu'il n'existe pas de r\'eduction sous-jacente (ou au contraire
de deviner qu'il en existe certainement une). Remarquons en passant que
c'est sur ce  type  de raisonnement alg\'ebrique que reposent les 
d\'emonstrations de  l'impossibilit\'e  de la quadrature du cercle ou de
l'impossibilit\'e de r\'esoudre alg\'ebriquement les \'equations de 
degr\'e sup\'erieur ou \'egal \`a cinq. En Calcul des probabilit\'es, on 
peut \'egalement tirer des renseignements de l'\'etude de la fraction,
dans la mesure o\`u celle-ci refl\`ete une structure alg\'ebrique: 
tant qu'il s'agit de probabilit\'es a priori et exactes, elles s'expriment
sous forme de fraction, dont le d\'enominateur est en principe le 
cardinal de l'espace des \'epreuves. Mais il peut arriver (et il arrive
souvent) que la fraction se simplifie et que le d\'enominateur devienne
alors plus petit que le cardinal de l'espace sur lequel on a travaill\'e. Ce
ph\'enom\`ene  signifie alors que la probabilit\'e obtenue aurait pu \^etre
calcul\'ee  sur un espace des \'epreuves r\'eduit. Nous avons en effet
observ\'e \`a propos des probabilit\'es conditionnelles (voir remarques 
\`a la fin de {\bf IV.3}), que les probl\`emes de probabilit\'e peuvent 
parfois \^etre r\'eduits \`a des espaces d'\'epreuves plus petits, par la
consid\'eration de probabilit\'es conditionnelles.  
\medskip
Un exemple de ce ph\'enom\`ene est fourni par le quatri\`eme probl\`eme
de la section {\bf V\ata .1.}:  {\sl on jette trois d\'es; calculer la
probabilit\'e pour que l'un au moins des trois d\'es donne un chiffre
impair}.  Nous avons introduit les \'ev\'enements $C_j$:  ``le d\'e 
$N^\circ j$ donne un chiffre impair'',  et l'\'ev\'enement du probl\`eme 
 ``au moins des trois d\'es donne un chiffre impair'' \'etait la r\'eunion 
$D = C_1 \cup C_2 \cup C_3$.  En calculant avec la formule de Poincar\'e
nous avons trouv\'e que  $\#D = 189$,  d'o\`u ${\cal P}\, (D) = {189 \over
216} = {7 \over 8}$.  On voit que cette fraction,  contrairement \`a $V.6$, 
ne garde aucune trace de la formule de Poincar\'e dont elle est issue;  en
outre,  le d\'enominateur est $8$,  ce qui laisse soup\c{c}onner qu'on devait
pouvoir traiter le probl\`eme sur un espace des \'epreuves de cardinal 
$8$ seulement.  La probabilit\'e du compl\'ementaire $\overline{\! D}$ de
$D$ est ${1 \over 8}$,  montrant par l\`a qu'il ne contient qu'une seule de
ces \'epreuves r\'eduites;  cela devrait nous mettre sur la voie.  Et en
effet,  $\overline{\! D}$ est l'intersection des compl\'ementaires des
$C_j$:  $\overline{\! D} = \overline{\! C_1}\cap \overline{\! C_2} \cap 
\overline{\! C_3}$.  Par ailleurs, $\overline{\! C_1}$, $\overline{\! C_2}$, 
et $\overline{\! C_3}$ sont relatifs \`a trois d\'es ind\'ependants et sont 
donc des \'ev\'enements stochastiquement ind\'ependants,  de sorte que
${\cal P}\, \big(\,\overline{\! D}\,\big) = {\cal P}\, \big(\,\overline{\!
C_1}\,\big) \times {\cal P}\, \big(\,\overline{\! C_2}\,\big) \times 
{\cal P}\, \big(\,\overline{\! C_3}\,\big)$.  Or $\;\overline{\! C_j}$ n'est
autre que ``le d\'e $N^\circ j$ donne un chiffre pair'',  \'ev\'enement dont
la probabilit\'e est {\it \'evidemment} ${1\over 2}$.  Par ailleurs, le
d\'enominateur $8$ signifie tout simplement qu'on pouvait se contenter
de prendre pour \'epreuves les huit triplets de {\it pair} et {\it impair}
possibles,  au lieu des $216$ triplets de chiffres.

\vskip8mm plus3mm minus3mm

{\bf V\ata . 4. Les textes al\'eatoires.}
\medskip
Un autre probl\`eme classique o\`u on est amen\'e \`a consid\'erer des
r\'eunions d'\'ev\'enements est celui des suites al\'eatoires. Nous avons 
d\'ej\`a rencontr\'e les suites al\'eatoires au chapitre {\bf I} \`a propos
du chaos d\'eterministe (voir {\bf I.4}).  Un probl\`eme c\'el\`ebre 
discut\'e par \'Emile Borel et popularis\'e par {\it La biblioth\`eque de
Babel} de J. L. Borges\ftn{3}{Jorge Luis Borges,  {\it Fictions}; 
traduction fran\c{c}aise chez Gallimard,  collection folio.} est celui de
l'apparition  al\'eatoire d'un texte sens\'e dans une suite de lettres
\'ecrites au  hasard.
\medskip
Ce probl\`eme se pr\'esente ainsi: une machine \'ecrit au hasard \`a la 
suite les uns des autres des caract\`eres de base de la typographie: $26$
lettres de l'alphabet majuscules et minuscules, signes de ponctuation,
parenth\`eses, lettres accentu\'ees telles que \'e, \^e, \`e, \"e, \^\i , 
\"\i , chiffres de 0 \`a 9, blanc s\'eparant les mots, retour-chariot avec 
ou  sans alinea; disons cent caract\`eres en tout. Chaque \'el\'ement 
successif de la suite est choisi au hasard parmi les cent caract\`eres de
base. Une suite  de $n$ caract\`eres est donc un ``mot'' de $n$ lettres
\'ecrit avec l'alphabet des cent caract\`eres de base. Le probl\`eme
rel\`eve du cas de d\'enombrement \'etudi\'e en {\bf II.1}: on peut
\'ecrire $100^n$ suites diff\'erentes. Ces suites \'etant suppos\'ees
\'equiprobables, la probabilit\'e d'en obtenir une particuli\`ere parmi
toutes celles possibles est alors $100^{-n}$.
\medskip
Le probl\`eme \'etudi\'e par \'Emile Borel est celui de l'apparition {\it 
partielle} d'un texte sens\'e particulier: quelle est la probabilit\'e pour 
que dans une suite de longueur $n$, apparaisse au moins une fois quelque
part un texte donn\'e de longueur $k$? Si la machine a \'ecrit une suite 
de $10^{1\, 000\, 000}$ de caract\`eres, quelle est la probabilit\'e d'y
trouver  quelque part, noy\'e dans un oc\'ean de gallimatias insens\'e, le
texte exact  des Voyages de Gulliver? 
\medskip
Ce probl\`eme revient \`a calculer la probabilit\'e d'une {\it r\'eunion}
d'\'ev\'e\-ne\-ments.  En effet, appelons $A_j$ l'\'ev\'enement ``entre le
rang $N^{\rm o}\, j$ inclu et le rang $N^{\rm o}\, j+k$ exclu de la suite
se trouve
exactement le texte cherch\'e''.  Il est clair que $j$ ne peut pas\^etre
sup\'erieur \`a $n-k$,  puisqu'\`a partir du rang$N^{\rm o}\, j$ il doit rester assez de place pour placer le texte de $k$
caract\`eres.  Donc $1 \leq j \leq n-k$.  L'\'ev\'enement ``le texte
pr\'ed\'efini se rencontre au moins
une fois dans la suite'' est alors lar\'eunion $E = A_1 \cup A_2 \cup
\cdots \cup A_{n-k}$. 
\medskip
D'apr\`es la formule de Poincar\'e,  pour avoir la probabilit\'e de la 
r\'eunion $E$,  nous devons d\'eterminer non seulement les probabilit\'es
de chacun des $A_j$,  mais aussi celles des intersections $A_{j_1}
A_{j_2}$,  puis $A_{j_1} A_{j_2} A_{j_3}$, etc.
\medskip
Dans le probl\`eme discut\'e avant (celui des lettres et des destinataires),
les intersections \'etaient toutes \'equivalentes car les destinataires 
\'etaient interchangeables.  Ici,  la situation est un peu plus complexe
car la probabilit\'e d'une intersection ne sera pas la m\^eme selon que les
textes peuvent ou non se superposer:  si le texte comporte \`a son d\'ebut
une partie de longueur $\ell$ qu'on retrouve \`a la fin,  il y a une 
probabilit\'e non nulle pour une intersection de la forme $A_{j} A_{j + k - 
\ell }$ impliquant la superposition de ces parties.  Mais pour un texte
qui n'est pas autosuperposable,  les intersections $A_{j_1} A_{j_2}$
pour lesquelles $j_2 - j_1 < k$ auront une probabilit\'e nulle.  En
revanche les autres intersections (pour lesquelles $j_2 - j_1 \geq k$)
auront toutes la m\^eme probabilit\'e.  Il en va de m\^eme pour les
intersections de trois, quatre, $\ldots$, des $A_j$.
\medskip
Ainsi la somme des cardinaux des intersections de $r$ parmi 
les $A_j$,  que dans la formule de Poincar\'e $V.4$ nous avons d\'esign\'ee 
par $\som_r$,  sera \'egale au nombre d'intersections possibles de 
$r$ \'ev\'enements $A_j$ sans recouvrement de texte,  multipli\'e par le
cardinal commun de ces intersections.  Appelons provisoirement 
$G_{n,k}^r$ ce nombre d'intersections possibles.  Le calcul de ces
nombres $G_{n,k}^r$ n'est pas imm\'ediat et nous y proc\'ederons d'ici
peu.  Par contre le cardinal de l'une quelconque de ces intersections
r\'esulte directement de la formule de d\'enombrement $II.1$:  en effet,
soient $A_{j_1}$, $A_{j_2}$, $\ldots$, $A_{j_r}$ tels que les textes
correspondants ne se recouvrent pas.  Cela \'equivaut \`a ce que les
diff\'erences entre deux quelconques des indices $j_\ell$ soient toutes
sup\'erieures \`a $k$.  Un \'el\'ement de cette intersection (une \'epreuve)
est une suite dont les caract\`eres situ\'es dans les intervalles disjoints
$\{ j_1 \ldots j_1 + k - 1 \}$, $\{ j_2 \ldots j_2 + k - 1 \}$, $\ldots$, 
$\{ j_r
\ldots j_r + k - 1 \}$ constituent les $r$ occurrences du textepr\'ed\'efini
et sont donc d\'etermin\'es.  Le nombre de ces caract\`eres
d\'etermin\'es est par cons\'equent $r \times k$.  La suite comporte $n$
caract\`eres en tout,  de sorte qu'il reste $n - rk$ caract\`eres \`a choisir
de toutes les fa\c{c}ons possibles,  ce qui d'apr\`es  $II.1$ fait $100^{n -
rk}$ possibilit\'es.  Ainsi $\#\, (A_{j_1} \cup
A_{j_2} \cdots  \cup A_{j_r}) = 100^{n - rk}$.  L'espace $\Omega$ de 
toutes les suites possibles ayant le cardinal $100^{n}$,  cela donne la
probabilit\'e $100^{-rk}$ pour l'intersection. 
\vskip9pt plus9pt minus6pt
En conclusion, nous aurons d'apr\`es la formule de Poincar\'e:
$${\cal P}\, (E) = {G_{n,k}^1 \strup{5.5} \over 100^{k}} - 
{G_{n,k}^2 \strup{5.5} \over 100^{2k}} + {G_{n,k}^3 \strup{5.5} \over 
100^{3k}} - {G_{n,k}^4 \strup{5.5} \over 100^{4k}} + \cdots $$
Bien entendu, si la machine utilise non pas $100$ caract\`eres mais
un nombre quelconque $m$, la probabilit\'e cherch\'ee sera
$${\cal P}\, (E)\quad = \;\sum_{1 \leq r \leq {n \over k}} (-1)^{r-1}\;
{G_{n,k}^r \strup{5.5}\over  m^{rk}} \eqno (V.7.)$$
Cette expression de ${\cal P}\, (E)$ n'est toutefois absolument exacte 
que si les recouvrements de textes sont impossibles; si les textes
peuvent se recouvrir en partie, il faut tenir compte d'une probabilit\'e
non nulle pour des intersections d'\'ev\'enements $A_j$ dont les indices
diff\`erent de moins de $k$. Le texte des voyages de Gulliver (comme la
quasi totalit\'e des textes de la litt\'erature) ne permet aucune
superposition partielle; pour avoir une telle superposition il faut
fabriquer des textes ad hoc. Par ailleurs, m\^eme pour
de tels textes sp\'eciaux, le changement quantitatif de la probabilit\'e 
est  infime et ne change rien tant qu'on n'envisage que des valeurs 
approch\'ees.
\medskip
Il reste \`a calculer les nombres $G_{n,k}^r$. Ce probl\`eme ne
rel\`eve pas directement des cas de d\'enombrement pass\'es en revue 
au chapitre {\bf II}, mais s'y ram\`ene par une simple r\'eduction. 
Repr\'esentons la suite des caract\`eres \'ecrits par la machine par des
points sur une  droite. On alignera ainsi en tout $n$ points. Puis
rempla\c{c}ons chaque apparition du texte pr\'ed\'efini (occurrence) par 
un seul segment  regroupant les $k$ points correspondants, comme le
montre la figure  ci-dessous o\`u les apparitions du texte sont 
marqu\'ees par des  crochets ($n=100$, $k=5$, $r=6$):  
\def\seg{\hbox{\kern1.25pt\vrule height0.4pt width20.3pt\kern1.25pt}} 
$$\eqalign{
&....[.....]..[.....].......................[.....].......[.....]..............[
.....].............[.....]..... \cr
&....\seg .. \seg ....................... \seg ....... \seg .............. \seg 
............. \seg .....    
\cr } $$
Si on a $r$ occurrences du texte (ne se recouvrant pas), la droite 
portera $n-rk$ points rest\'es isol\'es et $r$ segments. \`A chaque 
distribution particuli\`ere de $r$ occurrences correspond ainsi
univoquement une succession d\'etermin\'ee de $n-rk$ points isol\'es et
$r$ segments. Le d\'enombrement de toutes les intersections possibles
des  \'ev\'enements  $A_j$ \'equivaut donc au d\'enombrement des
configurations de $n-rk$ points et $r$ segments, soit $n-r[k-1]$
\'el\'ements en tout, ce qui rel\`eve du cas {\bf II.3}. Par cons\'equent 
$$G_{n,k}^r = {n-r[k-1] \choose r} \eqno (V.8.)$$
\medskip
Le probl\`eme est surtout int\'eressant pour de grandes valeurs de 
$n$. Dans ce cas on peut avoir des valeurs approch\'ees pour $G_{n,k}^r$.
Le coefficient bin\^omial qui appara{\^\i}t dans $V.8$ peut s'\'ecrire
(ici $k$ remplace $k-1$):
$$\eqalignno{ \hskip-8pt
{n-rk\choose r} &= {(n-rk)(n-rk-1)(n-rk-2)\cdots (n-rk-r+1)\over r!}\cr  
&= {(n-rk)^r \over r!}\hbox{\eightpoint $\displaystyle
\Big( 1 - {1\over n-rk}\Big)\Big( 1 - {2\over  n-rk}\Big) \cdots 
\Big( 1 - {r-1\over n-rk}\Big)$}\hskip35pt &(V.9)\cr
&= {n^r\over r!}\hbox{\eightpoint $\displaystyle\Big[  1 - 
{rk\over n}\Big]^r \cdot\;\Big( 1 - {1\over  n-rk} \Big)\Big( 1 - 
{2\over n-rk}\Big)\cdots\Big( 1 - {r-1\over n-rk}\Big)$} \cr } $$
\medskip
Reportons le tout dans la formule de Poincar\'e  $V.7$. On obtient alors
$${\cal P}\, (E)\quad = \;\sum_{1 \leq r \leq {n \over k}} 
(-1)^{r-1}\; {x^r \over r!}\, a_r \eqno(V.10.)$$
o\`u l'on a pos\'e 
$$\openup 2\jot\eqalign{
x\, &=\, {\up{n} \over m^k}\cr
a_r &= \Big[  1 - {r [k-1]\over\ldown{n}}\Big]^r \cdot
\hbox{\eightpoint $\displaystyle \Big( 1 - {1\over 
n-r [k-1]} \Big) \;\cdots\; \Big( 1 - {r-
1 \over n-r [k-1]}\Big)$} \cr}$$
On peut remarquer que pour les petites valeurs de $r$, $a_r$ est
pratiquement \'egal \`a $1$; par contre pour les  grandes valeurs de $r$,
c'est le terme ${x^r \over r !}$ qui devient tr\`es petit (en toute rigueur,
la condition pour que ces deux cas ne puissent pas  se pr\'esenter  en
m\^eme temps est que $n$, quoique tr\`es grand, reste petit devant
$m^{2k}/k$).  Sous cette condition on peut dire que la somme $V.10$ est
pratiquement \'egale \`a la s\'erie $\sum (-1)^{r-1}{x^r\over r!} =
1-e^{-x}$. Si $n$ est trop grand (du m\^eme ordre que $m^{2k}/k$, ou
plus grand encore), le terme ${x^r \over r !}$ sera encore grand lorsque 
les $a_r$ deviendront plus petits que $1$  et on ne pourra pas
assimiler  simplement la somme  $V.10$ \`a la s\'erie exponentielle,
mais il n'y en aura m\^eme pas besoin, car dans ce cas il est clair que la
probabilit\'e ${\cal P}\, (E)$ sera pratiquement \'egale \`a 1. 
\medskip
On peut donc conclure que si une machine \'ecrit au hasard $n$ 
caract\`eres (choisis dans un alphabet de $m$ caract\`eres) \`a la suite
les uns des autres, la probabilit\'e pour qu'il apparaisse au moins une 
fois un texte donn\'e de $k$ caract\`eres est \'egale \`a $1 - e^{-x}$ 
avec $x=n/m^k$. Pour $m=100$ par exemple, la probabilit\'e restera
n\'egligeable tant que $x$  sera petit, c'est-\`a-dire tant que $n$ sera
petit devant $100^k$. Les ex\'eg\`etes de la biblioth\`eque de Babel
avaient d\'ecouvert un volume qui contenait ---~comme unique passage
sens\'e~--- la phrase ``O temps tes pyramides''; cette phrase comporte
(blancs compris) 21 caract\`eres. Tant que $n$ reste petit devant
$100^{21} = 10^{42}$ la probabilit\'e de  trouver la phrase ``O temps tes
pyramides'' est nulle. \`A l'inverse, lorsque $n$ deviendra nettement plus
grand que $10^{42}$, la probabilit\'e de  voir appara{\^\i}tre ``O temps tes
pyramides'' deviendra \'egale \`a $1$. Ce sera seulement lorsque $n$ sera 
de l'ordre de $10^{42}$ que l'on verra le passage {\it progressif} de la 
probabilit\'e $0$ \`a la probabilit\'e $1$; ce passage se produira selon la
loi $1-\e^{-x}$.
\medskip
Une approche heuristique du probl\`eme,  ne faisant pas appel \`a la 
formule de Poincar\'e,  est envisageable.  On pourrait par exemple se dire
que l'apparition d'un texte sens\'e dans un texte \'ecrit au hasard \'etant
forc\'ement un \'ev\'enement rare,  on peut n\'egliger les cas o\`u un tel
texte appara{\^\i}trait deux,  trois,  quatre fois,  et approcher la
probabilit\'e ${\cal P}\, (E)$ par la somme $\sum_j {\cal P}\, (A_j)$.
Autrement dit ne retenir que le d\'ebut de la formule de Poincar\'e.  Le
r\'esultat qu'on obtiendrait alors serait $(n-k)/m^k \simeq x$.  Ceci
constitue bien une approximation de $1-e^{-x}$ lorsque $x$ est petit et
en ce sens le raisonnement heuristique est correct.  Mais lorsque $x$ n'est
pas petit,  il devient grossi\`erement faux.  L'explication en est simple:
n\'egliger l'apparition  de plusieurs occurrences est effectivement
l\'egitime pour des \'ev\'enements rares;  mais nous avons vu que lorsque
$n$ devient grand par rapport \`a $m^k$,  non seulement l'apparition d'une
occurrence cesse d'\^etre rare,  mais devient m\^eme quasi-certaine.  C'est
pourquoi on ne peut pas se passer de la formule de Poincar\'e compl\`ete.
\medskip
Jusqu'ici nous avons consid\'er\'e le cas asymptotique o\`u l'entier $n$
est grand. Dans ce cas la somme $V.7$ comporte un tr\`es grand nombre 
de termes et devient pratiquement une s\'erie, comme nous avons vu. 
\medskip
Bien s\^ur  $V.7$ est une expression ``exacte'', en ce sens qu'elle 
se d\'eduit math\'e\-ma\-ti\-que\-ment sans approximation de
l'hypoth\`ese  d'\'equiprobabilit\'e a priori des \'epreuves; elle est donc
tout aussi  exacte pour des valeurs modestes de $n$. Dans la pratique des
probabilit\'es ne peuvent pas \^etre exactes; cela n'a aucun sens r\'eel.
L'expression  $V.7$ (de m\^eme que tout calcul de probabilit\'e a priori)
n'est exacte que dans la mesure o\`u l'hypoth\`ese d'\'equiprobabilit\'e
des \'epreuves est elle-m\^eme exacte. Comme nous l'avons  d\'ej\`a
discut\'e au d\'ebut de la section {\bf V\ata . 3}, ce type d'exactitude est
comparable \`a celui  de la g\'eom\'etrie: on peut montrer que le rapport
de la circonf\'erence au diam\`etre est ``exactement'' $\pi$ ou que le
rapport de la dia\-go\-nale au c\^ot\'e du carr\'e est  ``exactement''
$\sqrt{2}$ ---~ces nombres ayant une  infinit\'e de d\'ecimales
d\'etermin\'ees~--- uniquement parce qu'on admet que les invariances 
de l'espace (par rotations et translations) sont elles-m\^emes 
absolument exactes.  
\medskip
On peut dire que s'il existe une l\'eg\`ere inexactitude dans 
l'\'equi\-pro\-ba\-bi\-lit\'e des carac\-t\`eres align\'es par la 
machine, celle-ci  se r\'epercutera sur le r\'esultat du calcul exact; 
mais comme l'hypoth\`ese de l'\'equiprobablilit\'e des \'epreuves est 
ind\'ependante  de la valeur des param\`etres tels que $n$, $k$, $m$, 
etc.,  il n'y a aucune raison pour  qu'une inexactitude \`a ce  niveau se
manifeste plut\^ot pour les petites valeurs de $n$ que pour  les grandes.
Par contre, l'approximation exponentielle reposait  explicitement sur 
l'hypoth\`ese que $n$ est grand. Donc la formule $V.7$ s'applique en
principe pour n'importe quelle valeur de $n$, tandis que l'approximation
exponentielle ne s'applique en principe que pour $n$ grand. On peut se
rendre compte de son domaine de validit\'e en comparant un calcul
effectu\'e directement \`a partir de  $V.7$ au r\'esultat donn\'e par
l'approximation exponentielle.  
\medskip
Cherchons par exemple les probabilit\'es pour que, dans une suite de 
vingt, cent, trois cents, mille, trois-mille,  et dix-mille chiffres
d\'ecimaux \'ecrits  au hasard, la suite $314$ apparaisse au moins une
fois. D\'esignons res\-pec\-ti\-ve\-ment par  ${\cal P}\, (20),\; {\cal
P}\,  (100),\; {\cal P}\, (300),\; {\cal P}\, (1\, 000),\; {\cal P}\, (3\,
000),\; {\cal P}\, (10\,  000)$ ces probabilit\'es. Le tableau suivant 
donne les valeurs obtenues d'apr\`es $V.7$, suivies entre parenth\`eses
par le calcul selon l'approximation exponentielle: 
\vskip 0pt plus4pt minus4pt 
$$\openup 2\jot\eqalignno{
{\cal P}\, (20) &= {18 \over 10^3} - {16 \cdot 15 \over  2 \cdot 10^6} +
{14 \cdot 13 \cdot 12 \over 6 \cdot 10^9} - \cdots \simeq 0.017\, 88
\quad (0.019\, 80)\cr
{\cal P}\, (100) &= {98 \over 10^3} - {96 \cdot 95 \over 2 \cdot 10^6} +
{94 \cdot 93 \cdot 92 \over 6 \cdot 10^9} - \cdots \simeq 0.093\, 57
\quad (0.095\, 16) \cr
{\cal P}\, (300) &= {298 \over 10^3} - {296 \cdot 295 \over 2 \cdot 
10^6}  + {294 \cdot 293 \cdot 292 \over 6 \cdot 10^9} - \cdots \simeq 
0.258\, 25 \quad (0.259\, 18) \cr
{\cal P}\, (1\, 000) &= {998 \over 10^3} - {996 \cdot 995 \over 2 \cdot 
10^6} + {994 \cdot 993 \cdot 992 \over 6 \cdot 10^9} - \cdots \simeq 
0.632\, 31 \quad (0.632\, 12) \cr
{\cal P}\, (3\, 000) &= {2998 \over 10^3} - {2996 \cdot 2995 \over 2
\cdot  10^6} +  \cdots \simeq  0.950\, 49 \quad (0.950\, 21) \cr
{\cal P}\, (10\, 000) &= {9998 \over 10^3} - {9996 \cdot 9995 \over 
2 \cdot 10^6} + \cdots \simeq 0.999\, 84 \quad (0.999\, 95) \cr  }$$
On constate ais\'ement que l'approximation exponentielle est d\'ej\`a
tr\`es bonne pour $n=100$; sur le tableau, l'approximation ne diff\`ere 
notablement du calcul ``exact'' que pour $n=20$ (erreur relative de
$11\%$).
\medskip
Le probl\`eme de l'apparition d'un passage sens\'e dans un texte
al\'eatoire a \'et\'e utilis\'e par \'Emile Borel\ftn{4}{\'Emile Borel 
{\it Le hasard} \'ed. Alcan, Paris, {\oldstyle 1914}, page 162.} pour 
discuter la
nature  du hasard. Borel consid\'erait surtout des suites infinies (il \'etait
math\'ematicien).  Il expliquait que si une suite de caract\`eres infinie 
est {\it vraiment}\'ecrite au hasard, elle doit comporter obligatoirement
n'importe quel texte donn\'e; donc tous les textes qui ont \'et\'e \'ecrits
et qui seront \'ecrits  un jour y figurent, et y figurent m\^eme une 
infinit\'e de fois, car d\`es que l'un appara{\^\i}t, la suite se poursuit 
comme si elle repartait de z\'ero. Une suite infinie dans laquelle
ne figurerait nulle part un certain passage ne peut pas avoir \'et\'e 
\'ecrite  au hasard, car il a fallu prescrire \`a la machine qui l'a
engendr\'ee d'\'eviter le passage manquant. Si toutes les suites de $n$
caract\`eres d'un alphabet qui  en comporte $m$ sont \'equiprobables,
alors la probabilit\'e  de voir  appara{\^\i}tre un certain passage de
longueur $k$ au bout de  $n = x\, m^k$ caract\`eres est $1-e^{-x}$, qui
tend vers $1$ quand $x$ (et donc $n$) tend vers l'infini. La contrapos\'ee 
de cette assertion  vraie est donc: si un passage donn\'e n'appara{\^\i}t
jamais, alors les  suites ne sont pas toutes \'equiprobables; en
particulier celle qui contient le passage est moins probable que les
autres. 
\medskip
Ces observations sur le hasard faites au d\'ebut du si\`ecle par \'Emile 
Borel
ont une post\'erit\'e consid\'erable, qui est la th\'eorie des suites al\'eatoires d\'ej\`a \'evoqu\'ee (section {\bf I.4}). Il s'agit de savoir 
quels sont les crit\`eres qui garantissent qu'une suite est ``vraiment''
al\'eatoire. Pour qu'il en soit ainsi, il faut donc, non seulement que 
chaque lettre y revienne aussi souvent que n'importe quelle autre, 
mais il faut aussi que n'importe quelle cha{\^\i}ne de caract\`eres fix\'ee
\`a l'avance y revienne aussi souvent que n'importe quelle autre de m\^eme 
longueur. \'Emile Borel\ftn{5}{{\it Les probabilit\'es d\'enombrables 
et leurs applications arithm\'etiques}, d\'ej\`a cit\'e.} a appel\'e 
{\it suites normales} les suites de chiffres ou de lettres qui satisfont 
ce crit\`ere.  Mais il est apparu par la suite qu'on pouvait engendrer
algorithmiquement des suites infinies satisfaisant \`a cette condition. 
De nombreux crit\`eres plus restrictifs ont \'et\'e propos\'es, dont 
il n'est pas toujours \'evident qu'ils soient \'equivalents les uns 
aux autres. Le crit\`ere le plus reconnu aujourd'hui est le crit\`ere
algorithmique de Solomonov et Kolmogorov (ann\'ees {\oldstyle 1960}): 
une suite est al\'eatoire si pour tout $n$ il est impossible d'en 
produire les $n$ premiers \'el\'ements avec un algorithme plus court 
que la simple donn\'ee de ces $n$ premiers \'el\'ements. Ce crit\`ere
pr\'esente toutefois des variantes et des subtilit\'es du fait qu'il 
n'y a pas de mesure \`a la fois universelle et absolument exacte de 
la longueur des algorithmes\ftn{6}{Voir le livre de Jean-Paul Delahaye 
{\it Information, complexit\'e, et hasard}, d\'ej\`a cit\'e.}.
\medskip
Revenons \`a notre calcul bas\'e sur la formule de Poincar\'e. Il a montr\'e 
que si on lit une telle suite infinie, pour avoir une chance non nulle 
d'{\it atteindre} un texte donn\'e de longueur $k$, il faut lire la suite
jusqu'au $m^k$-i\`eme caract\`ere environ.
\medskip
Cela implique que si nous lisons \`a la vitesse de $10^6$ caract\`eres 
par heure (vitesse de lecture excessive pour comprendre la {\it Critique
de la Raison pure} mais suffisante si on cherche seulement \`a rep\'erer
des passages sens\'es dans un texte al\'eatoire),  nous devrons lire 
pendant $10^{36}$ heures,  soit environ $10^{32}$ ann\'ees avant d'avoir
une chance de tomber sur le passage ``O temps tes pyramides''.  Si on est
moins exigeant,  on peut s'estimer heureux de tomber sur un passage
sens\'e non choisi \`a l'avance.  Une phrase semblable \`a ``O temps tes
pyramides'' peut s'obtenir en mettant \`a la suite quatre mots de la
langue fran\c{c}aise;  on peut estimer \`a $10^{12}$ le nombre de
combinaisons de quatre mots susceptibles de produire un sens
(\'eventuellement en for\c{c}ant un peu l'herm\'eneutique).  Dans ce
cas,  l'\'ev\'enement $A_j$ consid\'er\'e avant (trouver la phrase ``O 
temps  tes pyramides'' entre les rangs $j$ et $j+20$ de la suite),  dont la
probabilit\'e \'etait $100^{-21}$,  est remplac\'e par l'\'ev\'enement 
$A_j'$ (trouver n'importe quelle phrase sens\'ee de $21$ caract\`eres
entre les rangs $j$ et $j+20$ de la suite),  dont la
probabilit\'e est $10^{12}$ fois plus grande,  soit $10^{-30}$.  Pour avoir 
une chance non infime de rencontrer un tel texte au cours de la lecture 
s\'equentielle,  il faut alors parcourir $10^{30}$ caract\`eres,  op\'eration 
qui prendra $10^{20}$ ann\'ees.
\medskip
Le crit\`ere algorithmique de Solomonov et Kolmogorov est radicalement
restrictif;  la notion de hasard qu'il sous-entend exclut absolument la
possibilit\'e d'une simulation algorithmique du hasard.  Or nous avons 
vu d\`es le chapitre {\bf I} que le hasard r\'eel et pratique est
g\'en\'eralement un effet de chaos d\'eterministe;  les fonctions 
{\bf random} simulent le hasard par d\'eroulement d'un algorithme; 
et la question de savoir si dans le monde r\'eel il y a un hasard 
primordial (par exemple celui de la M\'ecanique quantique) est une 
question m\'etaphysique:  elle ne peut pas \^etre tranch\'ee par
l'exp\'erience.  C'est dire que le crit\`ere de Solomonov 
et Kolmogorov,  pris \`a la lettre,  est purement th\'eorique et ne 
concerne pas le monde r\'eel.  Par exemple,  la suite des d\'ecimales de 
$\pi$,  $\sqrt{2}$,  $e$, etc,  est engendr\'ee par un algorithme;  mais, 
bien qu'on ne sache pas le d\'emontrer rigoureusement,  ces suites sont
vraisemblablement normales au sens de Borel et utilisables pratiquement 
comme simulation de l'al\'eatoire:  si on convertit les d\'ecimales de 
$\pi$, $\sqrt{2}$, $e$, etc, en lettres,  il faudra certainement aller aussi 
loin dans ces suites pour trouver le passage ``O temps tes pyramides'' 
que dans n'importe quelle suite ``vraiment'' al\'eatoire. 
\medskip
C'est que l'algorithme qui calcule les d\'ecimales est bas\'e sur
l'arithm\'etique et n'a aucune raison sp\'eciale de favoriser ou
d\'efavoriser l'apparition de la phrase fatidique;  il y a une sorte
d'ind\'ependance causale entre l'algorithme arithm\'etique et la
phrase ``O temps tes pyramides''.
\medskip
Un des projets utopiques formul\'es par quelques math\'ematiciens (notamment 
Peano,  Hilbert,  et Russel,  mais contre une grande majorit\'e de sceptiques)
aux alentours de {\oldstyle 1900} fut la formalisation int\'egrale de la
math\'ematique.  Dans cette conception,  les d\'emonstrations math\'ematiques 
devenaient l'application automatique d'un algorithme;  un th\'eor\`eme 
aurait alors \'et\'e par d\'efinition le r\'esultat cod\'e d'un tel algorithme. 
Mais il est bien \'evident que ces algorithmes de d\'eduction logique 
formelle seraient tout aussi ind\'ependants de notre perception de 
l'espace que les algorithmes arithm\'etiques peuvent l'\^etre de la phrase 
``O temps tes pyramides''.  En sorte que si on avait automatis\'e la 
d\'eduction des th\'eor\`emes de la g\'eom\'etrie selon ce principe,  comme 
l'avait r\^ev\'e Peano,  on aurait d\^u attendre aussi longtemps 
l'arriv\'ee du premier th\'eor\`eme {\it sens\'e} que l'apparition de la phrase 
``O temps tes pyramides'' dans le d\'eroulement des d\'ecimales de $\pi$ 
ou $\sqrt{2}$. 
\medskip
Si on veut trouver des passages plus longs,  par exemple le texte 
int\'egral de {\it Madame Bovary},  qui comporte environ $980\, 000$ 
caract\`eres,  il faudra lire au moins $n = 100^{980\, 000}$ caract\`eres 
de la suite,  ce qui prendra de l'ordre de $10^{1\, 960\, 000}$ ann\'ees.
\medskip
Ces nombres prodigieusement grands (ou les probabilit\'es
prodigieusement petites qui leur correspondent) ne peuvent \^etre 
calcul\'es que par des raisonnements a priori et n'ont aucun sens 
empirique. C'est pourquoi \'Emile Borel (qui a beaucoup \'etudi\'e ces
probabilit\'es extr\^emes) a pos\'e le principe ``les \'ev\'enements dont 
la probabilit\'e est infinit\'esimale ne se produisent jamais''. 
\medskip
Les lois d\'eterministes de la Physique macroscopique, comme la loi de
Planck \'etudi\'ee en {\bf II.6}, sont d\'eduites par des raisonnements a 
priori \`a partir de certaines hypoth\`eses d'invariances. Pour la loi de
Planck, l'hypoth\`ese \'etait l'\'equiprobabilit\'e des modes d'occupation 
pour les photons du rayonnement. Ces raisonnements a priori conduisent
toujours \`a des probabilit\'es infinit\'esimales, parce que  les formules 
de d\'enombrement contiennent des puissances ou des factorielles. Ainsi
on peut calculer la probabilit\'e  a priori pour que la distribution des
photons selon les intervalles de  fr\'equences diff\`ere de la loi de 
Planck. Nous avons vu en {\bf II.6} que cette probabilit\'e diminue en
fonction de l'amplitude de l'\'ecart selon un facteur gaussien
(exponentielle du carr\'e de l'amplitude), ce qui conduit tr\`es 
rapidement \`a des probabilit\'es extr\^emes. C'est pourquoi de tels
\'ecarts ``ne ne produisent jamais'' et que la loi est d\'eterministe.
Nous y reviendrons encore au chapitre {\bf XIV}. 






\bye
