\input /home/harthong/tex/formats/twelvea4.tex
\input /home/harthong/tex/formats/epsf.tex

\auteurcourant={\sl J. Harthong: probabilit\'es et statistique}
\titrecourant={\sl Analyse statistique des d\'ependances}

\pageno=321 

\def\xx{\hskip-5pt + \hskip-5pt} 
\def\tv{\vrule height 30pt depth 6pt width0.4pt} 
\def\punkt{\vrule height0.4pt depth0pt width0.4pt} 
 
\newdimen\blocksize  \blocksize=\vsize  \advance\blocksize by -12pt 

\null\vskip11mm plus3mm minus3mm
 
\centerline{\tit XII. ANALYSE STATISTIQUE} 
\vskip7pt
\centerline{\tit DES D\'EPENDANCES.} 

\vskip11mm plus3mm minus3mm
 
{\bf XII. 1. Corr\'elation de deux variables al\'eatoires.}
\medskip 
Au chapitre {\bf VI} (variables al\'eatoires), nous avions introduit une 
grandeur appel\'ee la {\it covariance} de deux variables al\'eatoires. 
Nous avions vu alors que, lorsque deux variables al\'eatoires $X$ et $Y$ 
sont stochastiquement ind\'ependantes, leur covariance est nulle, mais 
que la covariance peut \^etre nulle aussi pour des variables d\'ependant 
l'une de l'autre. Les trois tableaux du chapitre {\bf VI} montrent trois 
possibilit\'es pour la loi conjointe de deux va\-ria\-bles al\'eatoires: 
(1) elles sont stochastiquement ind\'ependantes; (2) l'une est fonction de 
l'autre; (3) elles ne sont pas stochastiquement ind\'ependantes, et ne 
sont pas non plus fonction l'une de l'autre. Les trois exemples avaient
\'et\'e choisis de sorte que dans les trois cas, la covariance soit nulle. 
Ceci afin de bien montrer que la covariance n'est pas une mesure de la 
d\'ependance, qu'elle serait d'autant plus grande que la d\'ependance entre 
les deux varia\-bles serait plus forte (si tel \'etait le cas, la covariance 
serait nulle pour le tableau 1 seulement; elle serait maximum pour le 
tableau 2 et entre les deux pour le tableau 3). En examinant les
choses de pr\`es, il appara{\^\i}t que la covariance ne mesure pas
la d\'ependance en g\'en\'eral, mais seulement la d\'ependance
{\it lin\'eaire},  ou plus exactement la d\'ependance {\it affine}. 
Dans le tableau 2 du chapitre {\bf VI}, $Y$ d\'epend de $X$, 
mais pas lin\'eairement: $Y = X^2$, donc $Y$ est bien fonction de $X$, 
mais \`a une valeur donn\'ee de $Y$ correspondent deux valeurs 
oppos\'ees de $X$, qui s'annulent mutuellement dans le calcul de la 
covariance. On peut v\'erifier que si $Y$ \'etait une fonction lin\'eaire 
(ou affine) de $X$,  la covariance serait maximum. Si au lieu d'une 
d\'ependance lin\'eaire rigoureuse, on avait une d\'ependance ``floue'' 
(mais lin\'eaire),  la covariance serait interm\'ediaire entre 
z\'ero et le maximum. 
\medskip 
Afin de voir cela en d\'etail, le mieux est d'introduire une troisi\`eme
variable al\'eatoire $Z_\lambda = Y - \lambda X$, o\`u $\lambda$ est
un param\`etre r\'eel fix\'e. Si la loi conjointe de $X$ et $Y$ est 
donn\'ee, elle permet bien s\^ur de calculer la loi de $Z_\lambda$. Si 
par exemple la variance de $Z_\lambda$ est nulle, cela signifie que 
$Z_\lambda$ est, avec probabilit\'e 1, c'est-\`a-dire avec certitude, 
\'egale \`a sa moyenne $m = {\bf E}(Z_\lambda )$. Mais comme 
$Z_\lambda = Y - \lambda X$, cela signifierait aussi que $Y$ est avec 
certitude \'egale \`a $\lambda X + m$, donc que $Y$ est une fonction 
affine, de pente $\lambda$, de $X$. Si la variance de $Z_\lambda$ n'est 
pas nulle, mais petite, cela signifie que $Z_\lambda$ s'\'ecarte peu, ou 
ne s'\'ecarte qu'avec une faible probabilit\'e, de sa moyenne $m$, et 
cela revient \`a dire que $Y$ s'\'ecarte peu, ou avec une faible 
probabilit\'e, de $\lambda X + m$, ou encore, que la d\'ependance de $Y$ 
par rapport \`a $X$ est floue (comme dans le tableau 3), mais affine de 
pente $\lambda$. On peut donc dire que la variance de $Z_\lambda$ est 
une mesure du degr\'e de d\'ependance affine, pour une pente 
$\lambda$  donn\'ee,  de $Y$ par rapport \`a $X$. Voyons maintenant 
comment interpr\'eter la covariance de $X$ et $Y$ dans ce contexte. 
\medskip 
D\'esignons par $r_{j,k} = {\cal P}\, (X = x_j\; ; \; Y = y_k)$ la loi 
conjointe de $X$ et de $Y$, par $a = \sum_{j,k}\; r_{j,k}\> x_j$ la 
moyenne de $X$, et par $b = \sum_{j,k}\; r_{j,k}\> y_k$ la moyenne de 
$Y$. Si on calcule la variance de $Z_\lambda$ \`a partir de ces 
ingr\'edients, on obtient ceci : 
$$\eqalign{ {\bf Var}\, (Z_\lambda ) \;\; 
&= \;\; \sum_{j,k} r_{j,k}\; \bigl[ y_k - b - \lambda\, (x_j -a) \bigr]^2 \cr 
&= \;\; \sum_{j,k} r_{j,k}\; \bigl[ (y_k - b)^2 - 2\,\lambda\, (x_j -a)\,
(y_k - b) + \lambda^2\, (x_j -a)^2 \bigr] \cr 
&= \;\; {\bf Var}\, (Y) - 2\,\lambda\, {\bf Cov}\, (X,Y) + \lambda^2\, {\bf 
Var}\, (X) \cr }$$ 
Or la variance est une grandeur qui est par nature positive, de sorte
que, quelle que soit la valeur de $\lambda$, on aura toujours ${\bf 
Var}\, (Z_\lambda ) \geq 0$; d'apr\`es ce qui pr\'ec\`ede, cela a pour
cons\'equence que
$${\bf Var}\, (Y) - 2\,\lambda\, {\bf Cov}\, (X,Y) + \lambda^2\, {\bf
Var}\, (X) \;\; \geq \;\; 0$$ 
Cette in\'egalit\'e \'etant vraie quel que soit $\lambda$, on en d\'eduit
que n\'ecessairement, dans tous les cas 
$${\bf Cov}\, (X,Y) \;\; \leq \;\; \sqrt{{\bf Var}\, (X) \cdot {\bf Var}\, (Y)}
\eqno (XII.1.)$$
en outre, on peut dire que, pour que ${\bf Var}\, (Z_\lambda ) = 0$, 
il faut et il suffit que
$${\bf Cov}\, (X,Y) \;\; = \;\; \sqrt{{\bf Var}\, (X) \cdot {\bf Var}\, (Y)}
\eqno (XII.2.)$$
et que
$$\lambda \;\; = \;\; \sqrt{{\bf Var}\, (Y) \over {\bf Var}\, (X)}
\eqno (XII.3.)$$ 
Ainsi, ${\bf Cov}\, (X,Y)$ est maximum lorsque ${\bf 
Var}\, (Z_\lambda ) = 0$, $\lambda^2$ \'etant \'egal \`a 
${\bf Var}\, (Y) / {\bf Var}\, (X)$. 
\medskip 
On peut donc r\'esumer les choses ainsi: dans tous les cas la
covariance de $X$ et $Y$ peut au maximum \^etre \'egale \`a $\sqrt{{\bf 
Var}\, (X) \cdot {\bf Var}\, (Y)}$; elle {\it atteint} ce maximum lorsque 
$Y$ est une fonction affine de $X$ de pente $(XII.3)$. Lorsque la 
corr\'elation n'est pas \'egale \`a ce maximum, mais en est proche, 
$Y$ est approximativement \'egale \`a $\lambda X + m$, ce qui veut 
dire que $Y$ ne s'\'ecarte sensiblement de $\lambda X + m$ qu'avec
une faible probabilit\'e (d\'ependance ``floue''). 
\medskip 
 
\midinsert 
\epsfxsize=\hsize
\line{\epsfbox{../imgEPS/ch12eps/fig50.eps} } 
\vskip2mm 
\centerline{\eightpoint figure 50} 
\vskip8pt 
\centerline{\vbox{\hsize=11cm\eightpoint
Ce graphique repr\'esente les valeurs prises par les deux variables 
al\'eatoires $X$ et $Y$ du tableau 2 du chapitre {\bf VI} (les valeurs
de $X$ sont les abscisses des points et les valeurs de $Y$ sont les 
ordonn\'ees des points). Chaque point est affect\'e du poids 
correspondant \`a sa probabilit\'e, donn\'ee par la loi conjointe de $X$
et $Y$: le point de coordonn\'ees $x_j\, ,\, y_k$ a le poids ${\cal P}\, 
(X=x_j\, ;\, Y=y_k)$ (mais les poids ne se voient pas sur le graphique). 
La droite de r\'egression est repr\'esent\'ee: c'est la droite la plus 
proche possible (en un certain sens, d\'efini dans le texte) du nuage des 
points.  Ici, bien que ce soit {\it la plus} proche, elle n'est pas proche
car la corr\'elation entre les deux variables al\'eatoires est nulle. } }
\vskip 3mm
\endinsert 
 
La valeur maximum de la covariance donn\'ee par $(XII.2)$ d\'epend 
des variances de $X$ et $Y$; la covariance n'est donc pas la pure 
mesure du degr\'e de d\'ependance lin\'eaire entre $X$ et $Y$, mais
inclut aussi une mesure de leurs variances. Afin de s\'eparer les deux, 
il est commode d'introduire la grandeur
$$\rho \;\; = \;\; {{\bf Cov}\, (X,Y) \over \sdown{14.3}\sqrt{{\bf Var}\,
(X) \cdot {\bf Var}\, (Y)}} \eqno (XII.4.)$$ 
qu'on appelle le coefficient de corr\'elation de $X$ et $Y$; celui-ci est
n\'eces\-sai\-re\-ment compris entre $-1$ et $+1$, et il est \'egal \`a $+1$
si, et seulement si, l'\'egalit\'e $(XII.2.)$ est satisfaite, c'est-\`a-dire 
lorsque $Y$ est une fonction affine de $X$. Le coefficient de 
corr\'elation mesure donc le degr\'e de d\'ependance affine entre
$X$ et $Y$, ind\'ependamment de leurs variances res\-pec\-tives 
et sans pr\'ejuger de la pente $\lambda$. Ce degr\'e est de $100\%$
lorsque $Y$ est une fonction affine de $X$, et 
est proche de $100\%$ si $Y$ est proche d'une fonction affine de $X$, 
ou du moins si $Y$ ne s'\'ecarte qu'avec une faible probabilit\'e d'une 
fonction affine de $X$. Dans le cas du couple $X,Y$ d\'ecrit sur le 
tableau 2 du chapitre {\bf VI}, on a $Y = X^2$, c'est-\`a-dire que $Y$ est 
fonction de $X$, mais le degr\'e de d\'ependance {\it lin\'eaire} de $Y$ 
par rapport \`a $X$ est nul. Cela peut se comprendre ais\'ement: si on 
repr\'esente les valeurs que peut prendre (avec une probabilit\'e non 
nulle) le couple $X,Y$ par des points du plan ayant ces valeurs comme 
coordonn\'ees, on obtient des points situ\'es sur une parabole (voir 
figure 50); si $Y$ \'etait une fonction affine de $X$, les points seraient 
sur une droite. On peut dire que le degr\'e de d\'ependance lin\'eaire de 
$Y$ par rapport \`a $X$ est \'elev\'e si l'ensemble des points reste 
proche d'une certaine droite (par \hbox{exemple} sur la figure 51). 
Pour les points de la figure 50, aucune droite ne peut approcher 
correctement la parabole, et toutes les droites possibles sont 
\'egalement peu satisfaisantes; c'est pourquoi on peut dire que le 
degr\'e de d\'ependance lin\'eaire de $Y$ par rapport \`a $X$ est nul. 
\medskip 
 
\midinsert 
\epsfxsize=\hsize
\line{\epsfbox{../imgEPS/ch12eps/fig51.eps} } 
\vskip2mm 
\centerline{\eightpoint figure 51} 
\vskip8pt 
\centerline{\vbox{\hsize=11cm\eightpoint
Ce graphique montre un nuage de points correspondant aux valeurs qui 
seraient prises par deux variables al\'eatoires fortement corr\'el\'ees 
(c'est-\`a-dire dont le coefficient de corr\'elation est proche de 1). 
Ici, contrairement \`a la figure 50, la droite de r\'egression est 
r\'eellement proche du nuage des points. } } 
\vskip 3mm 
\endinsert 
 
En g\'en\'eral, pour un couple quelconque $X,Y$ de variables al\'eatoires, 
il existe une valeur de $\lambda$ qui rend minimum l'expression 
$(XII.1.)$, c'est-\`a-dire qui rend minimum la variance de $Z_\lambda = 
Y - \lambda X$; la droite d'\'equation $y = \lambda x + {\bf E}\, 
(Z_\lambda )$, correspondant \`a ce minimum, est appel\'ee la {\it
droite de r\'egression} de $Y$ par rapport \`a $X$. En interpr\'etant la 
variance de $Z_\lambda$ comme la somme des carr\'es des \'ecarts 
des points par rapport \`a la droite (pond\'er\'es par les probabilit\'es 
correspondantes), on peut dire que la droite de r\'egression est la 
droite des ``moindres carr\'es''. Si on calcule la valeur de $\lambda$ 
correspondant \`a ce minimum, on obtient
$$\lambda \;\; = \;\; \rho\;\sqrt{{\bf Var}\, (Y) \over {\bf Var}\, (X)}$$
ce qui redonne $(XII.3)$ lorsque $\rho = 1$. L'ordonn\'ee \`a l'origine de
la droite de r\'egression est simplement ${\bf E}\, (Z_\lambda ) = {\bf E}\,
(Y) - \lambda\, {\bf E}\, (X)$. 
\medskip 
Il faut bien comprendre ceci: on peut toujours calculer la droite de 
r\'egression, sauf si la variance de $X$ est nulle. Par exemple, dans le 
cas du tableau 2 du chapitre {\bf VI}, la droite de r\'egression existe 
(elle est repr\'esent\'ee sur la figure 50): c'est la droite de pente $0$
et d'ordonn\'ee \`a l'origine $2$. En effet $\lambda = \rho\,\sqrt{{\bf
Var} \, (Y) / {\bf Var}\, (X)} = 0$ et ${\bf E}\, (Y) - \lambda\, {\bf E}\, 
(X) = 2$. Mais bien que cette droite existe, elle n'est pas proche de la 
parabole, de sorte qu'elle n'a rien \`a voir avec la d\'ependance de $Y$ 
par rapport \`a $X$:  elle minimise bien la somme des carr\'es des 
\'ecarts, mais ce minimum n'est pas petit. Ainsi, pour n'importe quel 
couple $X,Y$ de variables al\'eatoires (sauf si ${\bf Var}\, (X) = 0$), on 
peut trouver une fonction affine des moindres carr\'es, mais celle-ci
ne repr\'esentera correctement la d\'ependance de $Y$ par rapport \`a 
$X$ que si le coefficient de corr\'elation est assez proche de 1. 
\medskip 
Dans le cas du tableau 2 du chapitre {\bf VI}, on devrait pouvoir dire
que le degr\'e de d\'ependance lin\'eaire de $Y$ par rapport \`a $X$
est $0$,  mais qu'en revanche le degr\'e de d\'ependance ``quadratique'' 
de $Y$ par rapport \`a $X$ est de $100\%$. Peut-on introduire des 
param\`etres li\'es aux variables al\'eatoires $X$ et $Y$, qui joueraient 
un r\^ole analogue \`a la covariance et au coefficient de corr\'elation, 
et qui permettraient de calculer quantitativement le degr\'e de 
d\'ependance ``quadratique'' de $Y$ par rapport \`a $X$? La r\'eponse 
est oui, mais il n'existe pas pour des fonctions non lin\'eaires de 
proc\'ed\'e aussi simple que pour le cas lin\'eaire. Nous reprendrons 
cela plus loin, dans la section {\bf 4} (``r\'egression non lin\'eaire''). 
 
\vskip6mm plus3mm minus3mm 
 
{\bf XII. 2. Moyenne, variance, et covariance empiriques.}
\medskip 
Dans le Calcul des probabilit\'es, on calcule des probabilit\'es \`a
partir d'une invariance postul\'ee. Les probabilit\'es que l'on d\'eduit 
ainsi par le calcul, sont appel\'ees des {\it probabilit\'es a priori}. 
Leurs valeurs sont exactes: par exemple on trouvera que la probabilit\'e 
de tel ou tel \'ev\'enement est $31 \over 73$ et non un nombre 
approximativement \'egal \`a $3 \over 7$, $0.42$ ou $0.425$. 
\medskip 
\`A l'inverse du Calcul des probabilit\'es, la Statistique ne traite pas
de probabilit\'es a priori. Elle ne fait que mesurer sur des 
\'echantillons. Nous avons d\'ej\`a discut\'e au chapitre {\bf X}\a ce 
qu'elle mesure.  Soit, par pr\'el\`evement d'un \'echantillon sur une 
population, elle permet (c'est la m\'ethode du sondage) de mesurer 
approximativement les proportions exactes sur la population totale; 
soit elle permet, en r\'ep\'etant un grand nombre de fois une 
exp\'erience reproductible (par exemple le lancer d'une pi\`ece de 
monnaie) de mesurer approximativement les probabilit\'es a priori de 
l'exp\'erience. Si par exemple on a une pi\`ece de monnaie non 
\'equilibr\'ee, qui a un peu plus de chances de tomber sur pile que sur 
face, on peut difficilement trouver le ``niveau o\`u intervient le hasard 
pur'' (celui-ci \'etait facile \`a trouver si la pi\`ece est sym\'etrique, 
car justement la sym\'etrie conduit \`a des invariances,  mais sans 
sym\'etrie, on ne peut plus).  Dans ce cas on peut cependant mesurer a 
posteriori la probabilit\'e de pile en lan\c{c}ant un grand nombre de 
fois la pi\`ece et en comptant combien de fois la pi\`ece est tomb\'ee 
sur pile. Pour conna{\^\i}tre les fr\'equences moyennes d'accidents
(et cette connaissance est n\'ecessaire pour organiser des services 
d'assistance ou calculer le montant des primes d'assurance) 
on ne peut pas proc\'eder autrement, car on est incapable de trouver 
une invariance a priori. Il en va de m\^eme lorsqu'on explore un 
domaine enti\`erement nouveau, o\`u il existe une invariance 
sous-jacente, mais inconnue. Par exemple pour la statistique de
Bose -- Einstein, on ignorait au d\'epart le principe d'invariance 
sous-jacent (``indiscernabilit\'e des particules''). Celui-ci n'a pas 
\'et\'e devin\'e \`a partir de rien par la seule logique: il a \'et\'e 
d\'eduit par induction \`a partir de r\'esultats {\it statistiques} 
exp\'erimentaux. 
\medskip 
Ainsi la Statistique traite des donn\'ees brutes, \`a travers lesquelles
on cherche \`a d\'etecter l'effet de probabilit\'es inconnues. Les
donn\'ees brutes sont mesur\'ees sur des objets, et un ensemble de 
donn\'ees est mesur\'e sur un ensemble d'objets appel\'e un {\it 
\'echantillon}: les donn\'ees sont des grandeurs se rapportant aux
objets de l'\'echantillon. Par exemple l'\'echantillon est un ensemble
de pi\`eces manufactur\'ees, disons pour fixer les id\'ees cent 
cinquante cylindres en acier inoxydable. On peut mesurer les longueurs 
de ces cylindres, les diam\`etres, et les poids. Si on num\'erote (ne 
serait-ce que par la pens\'ee) les cylindres de $1$ \`a $150$, appelons 
par exemple $x_i$ la longueur du $i^{\rm \,e}$ cylindre, $y_i$ son poids, 
et $z_i$ son diam\`etre. Les trois grandeurs $x_i$, $y_i$, et $z_i$ 
viennent donc du m\^eme objet $N^\circ i$. 
\medskip 
On distinguera donc l'ensemble des {\it objets}, appel\'e \'echantillon, 
et les donn\'ees num\'eriques qui s'y rapportent, qui sont les
{\it variables d'\'echantillon} ou {\it grandeurs empiriques}. 
Ainsi la longueur $x$, le poids $y$, et le diam\`etre $z$ de l'exemple 
pr\'ec\'edent sont trois variables empiriques relatives au m\^eme 
\'echantillon. De m\^eme que le Calcul des probabilit\'es traite 
essentiellement de variables al\'eatoires, la Statistique traite de 
variables d'\'echantillon.  Au paragraphe $1$ ci-dessus, nous avons vu 
comment la covariance de deux variables al\'eatoires permettait 
d'\'evaluer leur degr\'e de d\'ependance lin\'eaire mutuelle. Si on 
souhaite d\'etecter sur un \'echantillon les effets d'une relation 
lin\'eaire inconnue entre deux param\`etres, on ne peut pas utiliser les 
grandeurs introduites au paragraphe $1$, \`a savoir la covariance ou le 
coefficient de corr\'elation, puisque nous ne connaissons pas de 
probabilit\'es a priori. C'est pourquoi la Statistique fait appel \`a des 
grandeurs de nature diff\'erente, li\'ees \`a l'\'echantillon et non \`a 
des probabilit\'es a priori (mais ayant un rapport avec les 
pr\'ec\'edentes, que nous allons \'elucider), et qui sont la moyenne, la 
variance, ou la covariance {\it d'\'echantillon}. On dit aussi moyenne, 
variance, ou covariance {\it empiriques}. Par opposition, on appellera 
moyenne, variance, ou covariance {\it th\'eorique} la moyenne, 
variance, ou covariance d'une variable al\'eatoire (telles qu'elles sont 
d\'efinies au paragraphe $1$). 
\medskip 
Pour un \'echantillon comportant $n$ objets, la moyenne d'une variable 
d'\'echantillon $x_i$ ($i = 1,\, 2, \ldots n$) est d\'efinie comme suit
$$M \;\; = \;\; {1 \over n}\sum_{i=1}^{i=n} x_i \eqno (XII.5.)$$
Ceci est tr\`es facile \`a comprendre puisque c'est ainsi qu'on calcule
les moyennes aux examens (l'\'echantillon est alors l'ensemble des 
copies).  La variance d'\'echantillon de la variable $x_i$ sera 
$$P \;\; = \;\; {1 \over n-1}\sum_{i=1}^{i=n} (x_i - \overline{x})^2
\eqno (XII.6.)$$
o\`u $\overline{x} = M$ est la moyenne de la m\^eme variable. Notez 
bien le facteur $1/\hbox{\eightpoint $(n\! - \! 1)$}$ et 
non $1/n$). La covariance d'\'echantillon de deux variables $x_i$ et 
$y_i$ se rapportant au {\it m\^eme} \'echantillon est
$$Q \;\; = \;\; {1 \over n-1}\sum_{i=1}^{i=n} (x_i - 
\overline{x})\, (y_i - \overline{y}) \eqno (XII.7.)$$
o\`u $\overline{x}$ est la moyenne des $x_i$ et $\overline{y}$ celle
des $y_i$. La covariance entre deux variables ne se rapportant pas au 
m\^eme \'echantillon n'a g\'en\'eralement pas de signification. 
\medskip 
On notera tout particuli\`erement le coefficient $1\over 
\textdown{n}$ pour les moyennes, mais $1\over n-1$ pour les 
variances ou la covariance. Nous donnerons plus loin l'explication de
ce choix qui au premier abord peut sembler bizarre. 
\medskip 
La variance (th\'eorique) d'une variable al\'eatoire et la variance 
(empirique) d'une variable d'\'echantillon sont des concepts
diff\'erents, quoique appa\-rent\'es.  De m\^eme pour les moyennes ou les 
covariances. Du point de vue math\'ematique, d'abord, le calcul de la 
moyenne ou de la variance d'une variable al\'eatoire s'effectue en 
affectant \`a chaque valeur de la variable un {\it poids} \'egal \`a la 
probabilit\'e de cette valeur; par contre lorsqu'on calcule la moyenne 
ou la variance d'un \'echantillon, tous les points (ou coordonn\'ees) de 
l'\'echantillon ont le m\^eme poids (par exemple $1 \over n-1$ pour la 
variance et $1\over \textdown{n}$ pour la moyenne). Du point de 
vue pratique, ensuite, la variance d'une variable al\'eatoire se calcule 
d'apr\`es des probabilit\'es a priori,  tandis que la variance d'un 
\'echantillon se calcule d'apr\`es des r\'esultats qui se sont produits, 
qui n'ont donc pas ou qui n'ont plus de probabilit\'e. 
\medskip 
Lorsqu'on dispose de donn\'ees statistiques sur une population (par 
exem-ple la tension art\'erielle pour chaque personne d'un groupe de
cent), on peut calculer la moyenne ou la variance de la tension
art\'erielle pour cet \'echantillon. Il suffit d'appliquer une formule, 
et cela peut se faire sans se poser aucune question. Les probl\`emes 
commencent lorsqu'on se demande ce que repr\'esentent exactement
les grandeurs ainsi calcul\'ees: quel est leur sens, quelle information 
peut-on en tirer concernant l'\'echantillon? Et aussi: pourquoi la 
variance doit-elle \^etre calcul\'ee avec le coefficient $1 \over n-1$ 
alors que la moyenne l'est avec le coefficient $1\over 
\textdown{n}$ ? 
\medskip 
Pour \'eclaircir ces myst\`eres, il faut revenir \`a une remarque faite
au chapitre {\bf X}: \`a propos de l'exp\'erience consistant \`a tester 
l'effet d'un m\'edicament sur un groupe, nous avons distingu\'e 
soigneusement, d'une part la probabilit\'e a priori pour que le 
m\'edicament agisse sur une personne donn\'ee (cette probabilit\'e
r\'esulte du hasard cr\'e\'e par le chaos des m\'ecanismes 
mol\'eculaires), et d'autre part la distribution des effets parmi les 
diff\'erentes personnes d'un groupe. On peut enregistrer les effets du 
m\'edicament sur le groupe en notant pour chaque personne la baisse de 
tension sur les 24 heures qui suivent l'administration du m\'edicament; 
l'ensemble de ces donn\'ees chiffr\'ees constitue alors une variable 
d'\'echantillon: on pourra calculer la moyenne d'\'echantillon et 
la variance d'\'echantillon. 
\medskip
Par ailleurs, on peut imaginer {\it pour chaque personne consid\'er\'ee
s\'epar\'e\-ment}, la variable al\'eatoire dont les valeurs sont les
baisses de tension possibles avec les probabilit\'es a priori
correspondantes (ces probabilit\'es sont \'evidemment difficiles
\`a mesurer et impossibles \`a calculer a priori, et en outre
d\'ependent du temps, mais cela importe peu pour comprendre
le principe). On peut alors imaginer aussi ce que signifie 
la moyenne (ou esp\'erance math\'ematique) de cette variable 
al\'eatoire, ou sa variance, et compendre que ces grandeurs n'ont aucun 
rapport avec la moyenne ou la variance d'\'echantillon calcul\'ee sur le 
groupe. Comme nous l'avions soulign\'e au chapitre {\bf X}, il n'y a 
aucune raison que la moyenne d'\'echantillon calcul\'ee sur le groupe 
soit \'egale \`a la moyenne de la variable al\'eatoire li\'ee \`a une 
personne particuli\`ere, et de m\^eme pour la variance: cela provient
de ce que les ph\'enom\`enes se produisant dans le m\'etabolisme d'une 
personne particuli\`ere ne peuvent pas influer sur ce qui se passera 
dans le m\'etabolisme des autres. Il s'agit de deux grandeurs qui n'ont
aucun rapport entre elles. On peut m\^eme affirmer, en l'absence de 
ph\'enom\`enes tels que la contagion ou des comportements de groupe 
avec incidences biologiques (tabagisme, alcoolisme, habitudes 
ali\-men\-taires, etc.) que certaines variables al\'eatoires li\'ees
au m\'etabolisme d'individus diff\'erents sont stochastiquement 
ind\'ependantes. 
\vskip6pt plus 4pt minus3pt 
Il en va tout diff\'eremment si on consid\`ere un \'echantillon
provenant de la r\'ep\'etition d'une {\it exp\'erience reproductible}. 
Si la variable al\'eatoire (appelons-la $X$) est le r\'esultat d'une 
exp\'erience reproductible, elle aura une moyenne $m = {\bf E}\, (X)$
et une variance $v = {\bf Var}\, (X)$ (dites {\it th\'eoriques}). Par 
exemple pour le jeu de pile ou face, $X$ vaudrait 0 pour face et 1 pour 
pile, avec probabilit\'e $1 \over 2$, ce qui donnerait $m = {1 \over 2}$ 
et $v = {1 \over 4}$. Si on reproduit l'exp\'erience un grand nombre de 
fois, la loi des grands nombres aura pour effet que la proportion de 
chacun des r\'esultats possibles sera proche de la probabilit\'e a 
priori; par exemple, si on lance mille fois la pi\`ece, on obtiendra 
environ $500$ pile et $500$ face (avec, comme nous l'avons d\'ej\`a
vu, une incertitude de l'ordre de $\pm 30$). Si on consid\`ere les
$1000$ r\'esultats comme un \'echantillon statistique, et qu'on calcule 
la moyenne et la variance {\it empiriques} pour cet \'echantillon, alors 
elles seront proches de la moyenne et de la variance th\'eoriques de la 
variable al\'eatoire; si on poursuit le lancement de la pi\`ece, $2000$
fois, $10\, 000$ fois, $100\, 000$ fois, etc., la taille de l'\'echantillon 
augmentera et la moyenne empirique sera de plus en plus proche de 
l'esp\'erance math\'ematique $m$ de la variable al\'eatoire $X$ 
(respectivement: la variance empirique sera de plus en plus proche de
la variance th\'eorique $v$). 
\vskip6pt plus 4pt minus3pt 
On peut donc dire ceci: la moyenne (resp. la variance) d'une grandeur 
empirique sur un \'echantillon ${\cal A}$ et la moyenne (resp: la
variance) th\'eorique d'une variable al\'eatoire $X$ qui repr\'esente le 
r\'esultat possible d'une exp\'erience reproductible ${\cal E}$, sont 
deux grandeurs pratiquement identiques {\bf lorsque l'\'echantillon 
${\cal A}$ est constitu\'e par la r\'ep\'etition d'un grand nombre de 
fois l'exp\'erience ${\cal E}$} (ou encore: les grandeurs empiriques 
tendent vers les valeurs th\'eoriques lorsque la taille de l'\'echantillon 
tend vers l'infini). Mais cette \'equivalence entre les grandeurs 
empiriques et th\'eoriques n'est va\-lable que pour des exp\'eriences 
reproductibles. Lorsqu'on fait appel \`a cette \'equivalence pour 
interpr\'eter des donn\'ees relatives \`a des organismes vivants,  par 
exemple, il faut s'assurer que l'hypoth\`ese de reproductibilit\'e est 
\`a peu pr\`es l\'egitime. 
\vskip6pt plus 4pt minus3pt 
Les d\'efinitions des grandeurs empiriques donn\'ees en $(XII.5.)$ et 
$(XII.6.)$ ont \'et\'e convenues d\'elib\'er\'ement de mani\`ere \`a
satisfaire cette \'equivalence. En particulier, le myst\'erieux facteur
$1 \over n-1$ qui appara{\^\i}t dans ces expressions de la variance (ou
de la covariance) empirique, n'a pas d'autre justification que d'\^etre
le facteur qui rend la grandeur empirique le plus proche possible de la 
grandeur th\'eorique correspondante {\it dans le cas o\`u cette
\'equivalence s'applique}. Si on avait pris le facteur $1 \over \textdown n$
dans la variance, au lieu du facteur $1 \over n-1$, la grandeur empirique
co{\"\i}nciderait moins bien avec la grandeur th\'eorique. 
Tout cela s'explique math\'ematiquement comme suit. 
\medskip 
Ayant en vue le principe d'\'equivalence \'enonc\'e ci-dessus, 
consid\'erons l'\'echan\-til\-lon comme r\'esultant de la r\'ep\'etition 
d'une m\^eme exp\'erience, pour laquelle des probabilit\'es a priori 
existent. Soit donc la variable al\'eatoire  $X$ dont les $r$ valeurs 
$x_j$ ($j = 1,2, \ldots r$) sont prises avec probabilit\'e $p_j$. Si on 
r\'ep\`ete $n$ fois l'exp\'erience d\'ecrite par $X$, on obtiendra un 
certain nombre $n_j$ de fois la valeur $x_j$, de sorte que $\sum_j  
n_j =  n$.  On peut alors former les grandeurs
$$S_0 \;\; = \;\; \sum_{j=1}^{j=r} n_j\, (x_j - m)^2$$ 
o\`u $m$ est l'esp\'erance math\'ematique de la variable $X$, 
c'est-\`a-dire la moyenne th\'eorique a priori, et
$$S_1 \;\; = \;\; \sum_{j=1}^{j=r} n_j\, (x_j - M)^2$$ 
o\`u $M = {1\over\textdown{n}}\sum n_j\, x_j$ est la moyenne 
empirique de l'\'echantillon. Si on part du principe que les r\'esultats 
$x_j$ sont tous des r\'ealisations d'une m\^eme exp\'erience 
reproductible, cela signifie que $S_0$ et $S_1$ sont des variables 
al\'eatoires qu'on peut \'ecrire
$$\eqalign{ 
S_0\;\; &= \;\; \sum_{i=1}^{i=n} (X_i - m)^2 \cr 
S_1\;\; &= \;\; \sum_{i=1}^{i=n} (X_i - M)^2 \cr }$$ 
o\`u les $X_i$ sont $n$ variables al\'eatoires stochastiquement 
ind\'ependantes et de m\^eme loi (celle de $X$), et o\`u $M = 
{1\over\textdown{n}}\sum_i X_i$. Cela n'est que l'expression 
math\'ematique de l'hypoth\`ese que l'exp\'erience est reproductible, 
et signifie simplement qu'on reproduit $n$ fois, {\it ind\'ependamment} 
et {\it \`a l'identique} la variable al\'eatoire $X$. 
\medskip 
La diff\'erence essentielle entre $S_0$ et $S_1$ est que les variables 
al\'eatoires $X_i - m$ sont stochastiquement ind\'ependantes, alors
que les $X_i - M$ sont li\'ees par la relation que leur somme est nulle. 
Entre $S_0$ et $S_1$ on a l'identit\'e 
$$S_0 - S_1 \;\; = \;\; n\, (M-m)^2 \eqno (XII.8)$$ 
Cela est facile \`a \'etablir, il suffit de d\'evelopper les expressions
de $S_0$ et $S_1$. Pour la premi\`ere: 
$$\eqalignno{ 
S_0 \;\; &= \;\; \sum_{i=1}^{i=n} (X_i - m)^2 \cr 
&= \;\; \sum_{i=1}^{i=n} X_i^2 - 2m\sum_{i=1}^{i=n} X_i + n\, m^2\cr
&= \;\; \sum_{i=1}^{i=n} X_i^2 - 2n\, mM + n\, m^2\cr }$$ 
Pour la seconde: 
$$\eqalignno{ 
S_1 \;\; &= \;\; \sum_{i=1}^{i=n} (X_i - M)^2 \cr 
&= \;\; \sum_{i=1}^{i=n} X_i^2 - 2M \sum_{i=1}^{i=n} X_i + n\, M^2 \cr 
&= \;\; \sum_{i=1}^{i=n} X_i^2 - 2n\, M^2 + n\, M^2\cr 
&= \;\; \sum_{i=1}^{i=n} X_i^2 - n\, M^2 \cr }$$ 
On voit en soustrayant membre \`a membre que les termes $\sum
X_i^2$ s'annulent et il reste 
$$S_0 - S_1 \;\; = \;\; - 2n\, mM + n\, m^2 + n M^2 = n\, (M-m)^2$$ 
ce qui montre bien $(XII.8)$. 
\medskip 
L'esp\'erance math\'ematique de $S_0$ est par d\'efinition 
$n\, {\bf Var} (X)$, donc celle de $S_1$, ${\bf E} (S_1)$, est \'egale \`a 
$n\, {\bf Var} (X) - {\bf E} (n[M-m]^2)$. 
\medskip 
Il est facile de calculer ${\bf E} ([M-m]^2)$: 
$${\bf E} ([M-m]^2) \;\; = \;\; {\bf E} \bigg( \bigg[ {\up{1}\over\down{n}} 
\sum_{i=1}^{i=n} (X_i - m)^2\bigg] \bigg) \;\; = \;\; {\up{1}\over\down{n}} 
{\bf Var} (X)$$ 
Par cons\'equent ${\bf E} (S_1) = (n-1)\, {\bf Var} (X)$, alors que
${\bf E} (S_0) = n\, {\bf Var} (X)$. On comprend ainsi pourquoi on a 
introduit le coefficient $1\, /\,\hbox{\eightpoint $(n-1)$}$ 
dans $XII.6$:  d'apr\`es la loi des grands nombres appliqu\'ee \`a 
l'exp\'erience reproductible,  si $n$ est grand,  les valeurs de $S_1$ 
fluctueront autour de leur moyenne ${\bf E} (S_1) = (n-1)\, {\bf Var} (X)$, 
avec des \'ecarts de l'ordre de $\sqrt{n}$. La diff\'erence entre ${\bf E} 
(S_1)$ et ${\bf E} (S_0)$, \'egale \`a ${\bf Var} (X)$, est certes faible 
compar\'ee \`a $\sqrt{n}$ (amplitude moyenne des fluctuations), mais 
introduit un d\'ecalage syst\'ematique ou {\it biais}.  La valeur ${1\over 
n-1} {\bf E} (S_1)$ est une meilleure estimation de ${\bf Var} (X)$ que 
${1\over \textdown{n}} {\bf E} (S_1)$, parce qu'elle est 
bas\'ee sur le centre exact des fluctuations de la variable $S_1$. 
\medskip 
Il faut bien noter que le choix du coefficient ${1 \over n-1}$ repose
sur une id\'ee pr\'econ\c{c}ue de la ``vraie'' variance, \`a savoir que
la ``vraie'' variance est ${\bf Var}(X)$. Cette id\'ee pr\'econ\c{c}ue
n'a de sens que par le principe de l'\'equivalence entre grandeurs
th\'eoriques et grandeurs empiriques pour les exp\'eriences
reproductibles. Le coefficient ${1 \over n-1}$ ne sert qu'\`a ajuster
la varian\-ce empirique \`a la variance th\'eorique, de mani\`ere \`a
satisfaire au plus juste l'\'equivalence. 
\medskip 
Si on \'etudie une r\'epartition statistique qui ne provient pas d'une 
exp\'erience reproductible, par exemple dans l'\'etude de l'effet d'un 
m\'edi\-ca\-ment sur diff\'erentes personnes (cf chapitre {\bf X}), 
il n'est plus possible d'avoir une id\'ee pr\'econ\c{c}ue de la ``vraie'' 
variance. Dans ce cas n'importe quel coef\-fi\-cient autre que ${1 \over 
n-1}$ est tout aussi bon, car la variance empirique (qui seule existe) 
ne peut servir qu'\`a comparer les dispersions statistiques sur 
diff\'erents groupes; bien s\^ur,  pour que la comparaison d'\'etudes
provenant de pays diff\'erents soit possible et sens\'ee, il faut une 
norme internationale (par exemple, que {\it par convention} tout le 
monde adopte la somme $S_1$ des carr\'es des \'ecarts, ou ${1\over 
\textdown{n}}\, S_1$, ou encore ${1\over n-1}\, S_1$), mais la 
d\'efinition $XII.6$, c'est-\`a-dire ${1\over n-1}\, S_1$, n'est alors
pas plus juste qu'une autre. 
\medskip 
La norme internationale qui a \'et\'e adopt\'ee est celle de la
d\'efinition $XII.6$, par analogie avec les exp\'eriences reproductibles. 
Mais en dehors de ces derni\`eres, ce n'est qu'une convention. Il ne faut 
donc pas en \^etre dupe. 
 
\vskip6mm plus3mm minus3mm 
 
{\bf XII. 3. Corr\'elation statistique.}
\medskip 
Nous avons vu au paragraphe $1$ que la corr\'elation \'etait la 
d\'ependance lin\'eaire (mesur\'ee par le coefficient de corr\'elation). 
Mais cela a \'et\'e trait\'e pour des variables al\'eatoires, dont la loi 
conjointe est connue. De m\^eme qu'on a pu, dans la section {\bf 
XII.1}, d\'eriver la corr\'elation de deux variables al\'eatoires de 
leur covariance et de leurs variances, on doit pouvoir d\'eriver une 
corr\'elation empirique de la covariance et de la variance empiriques 
d'un \'echantillon.  Formellement, cela ne pose aucun probl\`eme; mais
il faudra en interpr\'eter la signification. 
\medskip 
Consid\'erons l'exemple suivant. Dans un groupe de cent personnes 
(num\'erot\'ees de $i=1$ \`a $i=100$ par randomisation), on rel\`eve
la tension art\'erielle. On repr\'esente celle-ci sur un graphique de la 
mani\`ere sui\-vante: \`a chaque personne $i$ correspond un point, dont 
l'abscisse $x_i$ est l'\^age, et l'ordonn\'ee $y_i$ la tension (disons la 
tension systolique pour fixer les id\'ees). On dispose ainsi d'un 
\'echantillon de cent points dans le plan $\{ x,y \}$, mais ceux-ci ne
sont pond\'er\'es par aucune loi de probabilit\'e. Les {\it moyennes 
d'\'echantillon} sont alors les nombres $\overline{x} = {1 \over 
100}\sum_{i=1}^{100} x_i$ (\^age moyen de l'\'echantillon) et
$\overline{y} = {1 \over 100}\sum_{i=1}^{100} y_i$ (tension moyenne
de l'\'echantillon). La {\it covariance d'\'echantillon} de l'\^age et de la 
tension est la grandeur
$$Q = {\up{1} \over 99}\sum_{i=1}^{i=100} (x_i - \overline{x})\, (y_i 
- \overline{y})$$
et de m\^eme les {\it variances d'\'echantillon} sont 
$$\eqalignno{ 
P \;\; &= \;\; {\up{1} \over 99}\sum_{i=1}^{i=100} (x_i - \overline{x})^2
\quad \hbox{(variance de l'\^age)} \cr 
R \;\; &= \;\; {\up{1} \over 99}\sum_{i=1}^{i=100} (y_i - \overline{y})^2
\quad \hbox{(variance de la tension)} \cr }$$ 
\medskip
Rien n'emp\^eche d'introduire un coefficient de corr\'elation empirique 
formel\-le\-ment analogue \`a celui du paragraphe $1$: 
$$\rho \;\; = \;\; {\hbox{covariance d'\'echantillon (entre la tension
et l'\^age)} \strup{8} \over \sdown{17} \sqrt{\sdown{9.5} \hbox{variance
de l'\^age} \times \hbox{variance de la tension}}}$$ 
Sur le graphique on a cent points (dont l'abscisse repr\'esente l'\^age et 
l'ordonn\'ee la tension). Il est facile de v\'erifier que si $\rho$ \'etait 
\'egal \`a $\pm 1$ ces cent points seraient sur une droite: il n'est en
effet pas n\'ecessaire de refaire les calculs du paragraphe $1$, puisque 
formellement la situation est la m\^eme (au paragraphe $1$, les cent 
points avaient pour poids les probabilit\'es a priori, ici ils ont tous le 
m\^eme poids $1 / 99$, mais les calculs effectu\'es formellement ne 
sont pas affect\'es par la diff\'erence de signification). La pente de la 
droite serait \'egale \`a
$$\lambda \;\; = \;\; \cases{ + \sqrt{\hbox{\vrule depth4pt width0pt
variance de la tension} \over \hbox{\vrule height10pt width0pt variance
de l'\^age}} & si $\rho = +1$ \cr
\noalign{\vskip9pt plus3pt minus3pt} 
- \sqrt{\hbox{\vrule depth4pt width0pt variance de la tension}\over 
\hbox{\vrule height10pt width0pt variance de l'\^age}} & si $\rho = 
-1$ \cr }$$
Bien entendu il n'arrive jamais dans la r\'ealit\'e que les cent points 
soient sur une droite (cela signifierait que la tension serait exactement 
une fonction lin\'eaire de l'\^age!). Mais ce qu'on observe, c'est que la 
tension est en moyenne plus \'elev\'ee chez des personnes plus
\^ag\'ees. Sur le graphique, cela devrait avoir l'effet que les points 
dont l'abscisse est plus grande doivent avoir aussi (``en moyenne'') une 
ordonn\'ee plus grande, ce qui se traduit par un nuage de points 
allong\'e et inclin\'e comme ceux qui sont repr\'esent\'es sur la figure 
$52$; un graphique de la tension et de l'\^age aurait l'aspect de la 
figure $52.2$ ou $52.3$, donc la corr\'elation est assez faible. 
\vskip6pt plus6pt minus3pt
En Statistique, la connaissance de la droite de r\'egression joue un
r\^ole capital. Elle sert en effet \`a corriger des donn\'ees brutes et
\`a s\'eparer les diverses causes qui influencent un ph\'enom\`ene. 
Supposons que nous cherchions \`a d\'eterminer une corr\'elation entre 
le tabagisme et l'hypertension. Nous effectuons une \'etude clinique sur 
deux groupes de personnes, un groupe de cinquante fumeurs consommant 
au moins vingt cigarettes par jour, et un groupe t\'emoin de cinquante 
personnes qui ne fument jamais. \`A premi\`ere vue, il peut sembler
que, pour que l'\'etude soit valide, il suffit de choisir ces groupes par 
randomisation. En effet, il s'agit de montrer que sur l'ensemble de la
population (disons la population fran\c{c}aise pour fixer les id\'ees) 
la tension art\'erielle moyenne de l'ensemble des fumeurs est
nettement sup\'erieure \`a la tension moyenne sur les non-fumeurs; 
comme il est impossible de faire une \'etude sur tous les fumeurs 
fran\c{c}ais, on utilise la m\'ethode du sondage en pr\'elevant des 
\'echantillons de cinquante personnes dans ces deux populations 
(comme on ne s'int\'eresse pas \`a conna{\^\i}tre la tension art\'erielle 
moyenne \`a trois d\'ecimales pr\`es, mais seulement de v\'erifier que 
l'une est nettement sup\'erieure \`a l'autre, un \'echantillon de taille 
mo\-deste suffit). La randomisation garantira alors que les 
\'echantillons sont bien ``pris au hasard'', conform\'ement \`a ce qui 
avait \'et\'e dit au chapitre {\bf X}. 
\vskip6pt plus6pt minus3pt
Mais il se peut que les fumeurs ne soient pas r\'epartis uniform\'ement 
selon l'\^age. Par exemple, une mode anti-tabac aurait pu, au cours des 
ann\'ees mille-neuf-cent-quatre-vingt-dix, inciter beaucoup de jeunes \`a
ne jamais fumer, habitude qu'ils garderont ensuite, tandis que chez leurs
ain\'es les non-fumeurs seraient rest\'es rares. Si des \'echantillons
sont choisis par randomisation dans l'ensemble de la population, on aura
certes un reflet de la population totale, mais comment savoir si la tension 
art\'erielle plus \'elev\'ee qu'on observera dans le groupe des fumeurs 
provient du tabagisme et non tout simplement du seul fait que les 
personnes plus \^ag\'ees y sont plus nombreuses que dans le groupe des 
non-fumeurs? Ce probl\`eme demeurerait, m\^eme si l'\'etude \'etait 
effectu\'ee sur la population totale, et ne peut \^etre attribu\'e \`a la
taille modeste de l'\'echantillon. 
\vskip6pt plus6pt minus3pt
Une solution est \'evidemment de r\'eunir, non pas deux \'echantillons, 
mais quarante r\'epartis selon l'\^age, de mani\`ere \`a disposer de
vingt couples d'\'echantillons fumeurs $/$ non-fumeurs, chacun de ces 
couples \'etant homog\`ene quant \`a l'\^age. Mais cette solution est 
beaucoup plus on\'ereuse. L'\'etude statistique des corr\'elations
permet de s'en passer. 
\medskip 
 
\midinsert 
\vbox to \blocksize {
\epsfxsize=\hsize
\line{\epsfbox{../imgEPS/ch12eps/fig52-1.eps} } 
\vskip4mm 
\centerline{\eightpoint figure 52.1} 
\vfill\epsfxsize=\hsize
\line{\epsfbox{../imgEPS/ch12eps/fig52-2.eps} } 
\vskip4mm 
\centerline{\eightpoint figure 52.2} 
\vskip8pt 
\centerline{\eightpoint Nuages de points faiblement corr\'el\'es} 
\vskip15pt}
\endinsert 
 
\midinsert 
\vbox to \blocksize {\epsfxsize=\hsize
\line{\epsfbox{../imgEPS/ch12eps/fig52-3.eps} } 
\vskip2mm 
\centerline{\eightpoint figure 52.3} 
\vfill
\epsfxsize=\hsize
\line{\epsfbox{../imgEPS/ch12eps/fig52-4.eps} } 
\vskip2mm 
\centerline{\eightpoint figure 52.4} 
\vskip8pt 
\centerline{\vbox{\hsize=11cm\eightpoint
Ici, la corr\'elation est plus grande qu'\`a la page pr\'ec\'edente (en
fait le coefficient de corr\'elation $\rho$ dans les diff\'erentes 
figures est tel que d'une figure \`a la figure suivante 
$\sqrt{1-\rho^2}$ est diminu\'e de moiti\'e). 
 } } 
\vskip15pt} 
\endinsert 
 
\midinsert 
\vbox to \blocksize {\epsfxsize=\hsize
\line{\epsfbox{../imgEPS/ch12eps/fig52-5.eps} } 
\vskip4mm 
\centerline{\eightpoint figure 52.5} 
\vfill\epsfxsize=\hsize
\line{\epsfbox{../imgEPS/ch12eps/fig52-6.eps} } 
\vskip4mm 
\centerline{\eightpoint figure 52.6} 
\vskip8pt 
\centerline{\vbox{\hsize=11cm\eightpoint 
Les nuages de points sont maintenant tr\`es nettement distribu\'es \`a 
proximit\'e de leur droite de r\'egression. On peut dire que celle-ci a
une v\'eritable signification. 
 } } 
\vfill} 
\endinsert 
 
\midinsert 
\vbox to \blocksize {\epsfxsize=\hsize
\line{\epsfbox{../imgEPS/ch12eps/fig52-7.eps} } 
\vskip4mm 
\centerline{\eightpoint figure 52.7} 
\vfill\epsfxsize=\hsize
\line{\epsfbox{../imgEPS/ch12eps/fig52-8.eps} } 
\vskip4mm 
\centerline{\eightpoint figure 52.8} 
\vskip8pt 
\centerline{\vbox{\hsize=11cm\eightpoint 
Enfin, voici l'aspect pris par le nuage de points lorsque la corr\'elation
est vraiment tr\`es forte. Sur $52.8$, $1-\rho^2$ est $4000$ fois plus 
petit qu'en $52.1$, ce qui veut dire que $\rho$ est pratiquement \'egal 
\`a $1$. Dans ce cas on peut dire que l'ordonn\'ee est pratiquement une 
fonction lin\'eaire de l'abscisse. 
 } } 
\vfill} 
\endinsert 
 
Il suffit en effet de repr\'esenter les r\'esultats concernant les deux 
\'echan\-til\-lons sur deux graphiques, chaque personne correspondant
\`a un point dont l'abscisse est l'\^age et l'ordonn\'ee la tension; 
on obtient ainsi deux graphiques de cinquante points chacun,  l'un 
repr\'esente les non-fumeurs et l'autre les fumeurs. Si, comme il 
\'etait \`a craindre,  le groupe des fumeurs contient davantage de 
personnes \^ag\'ees de plus de quarante ans que le groupe des 
non-fumeurs, cela se traduira sur le graphique par une in\'egale 
r\'epartition de la densit\'e des points: sur le graphique fumeurs,  
le nuage de points sera plus dense vers la droite,  sur le graphique 
non-fumeurs, il sera plus dense vers la gauche. Mais la droite de 
r\'egression n'est pas affect\'ee par ces diff\'erences de densit\'e. 
Bien s\^ur si on prend un autre \'echantillon, on peut observer une 
droite de r\'egression un peu diff\'erente, et une r\'epartition de 
densit\'e \'egalement diff\'erente, mais ce ne sont que des fluctuations 
par rapport \`a une valeur moyenne. Ce qui int\'eresse alors le clinicien 
n'est pas la r\'epartition, in\'egale, de la densit\'e selon l'\^age, mais 
la droite de r\'egression: si $y = ax +b$ est la droite de r\'egression 
des non-fumeurs, et $y=cx + d$ celle des fumeurs, l'influence du 
tabagisme sur la tension art\'erielle sera consid\'er\'ee comme 
\'etablie si {\bf pour tout $x$, $cx + d$ est sup\'erieur \`a $ax + b$}. On 
comprend que, si la corr\'elation est forte, (c'est-\`a-dire si le nuage 
de points ne s'\'ecarte pas beaucoup de la droite), le fait que pour une 
certaine valeur de $x$ on ait $cx + d > ax + b$ signifie que parmi les 
personnes d'\^age $x$, la tension art\'erielle est plus \'elev\'ee chez 
les fumeurs que chez les non-fumeurs. Si cette in\'egalit\'e a lieu 
pour tout $x$, cela signifie qu'ind\'ependamment de l'\^age, la tension
est plus \'elev\'ee chez les fumeurs que chez les non-fumeurs. Toutefois, 
pour que cette conclusion soit fond\'ee, certaines conditions doivent 
\^etre r\'eunies, qui sont les trois suivantes. 
\vskip6pt plus4pt minus4pt 
1. Si $cx + d - ax - b$, quoique formellement positif, est nettement
plus petit que la largeur (mesur\'ee par la variance de $y - ax$) du 
nuage de points, la signification du r\'esultat est nulle, car la valeur 
moyenne de $cx + d - ax - b$ pourrait \^etre $0$ et la valeur positive 
trouv\'ee une fluctuation pratiquement aussi fr\'equente que la valeur 
$0$ elle-m\^eme; en particulier, si le nuage de points est tr\`es large 
(donc si le coefficient de corr\'elation est petit), il est pratiquement 
impossible de fonder la conclusion. Si on veut estimer 
quantitativement la signification du r\'esultat, autrement dit d\'ecider 
si la valeur positive est significative ou au contraire due \`a des 
fluctuations normales, on effectuera un test du $\chi^2$ ou apparent\'e. 
\vskip4pt plus3pt minus3pt 
2. L'\'etude de la r\'egression a \'et\'e rendue n\'ecessaire (au lieu que
nous nous contentions d'un simple calcul de moyenne) par le soup\c{c}on 
d'une corr\'elation entre le tabagisme et l'\^age, corr\'elation qui se 
superposerait \`a la corr\'elation cherch\'ee entre le tabagisme et 
l'hypertension et fausserait ainsi les observations. Mais qui nous
prouve qu'il n'existe pas d'autres corr\'elations, que nous ne
soup\c{c}onnons pas, et qui viendraient fausser \'egalement les
r\'esultats ?  Il existe par exemple une corr\'elation entre la tension
et le poids (les ob\`eses sont davantage hypertendus), or fumer agit
{\it n\'egativement} sur le poids.  Il faudrait alors, en toute rigueur, 
\'etudier aussi la r\'egression tension $/$ poids dans les deux groupes, 
et effectuer une nouvelle comparaison.  Il faudrait refaire la m\^eme 
op\'eration pour {\it toutes} les corr\'elations possibles (du moins 
celles qui sont suffisamment fortes et par cons\'equent non 
n\'egligeables), et pouvoir s'assurer qu'on n'en ignore aucune. 
\vskip4pt plus3pt minus3pt 
3. Enfin, toutes ces \'etudes de r\'egression sont valables tant qu'il
s'agit de d\'ependances lin\'eaires; la corr\'elation entre la tension et 
l'\^age est suf\-fisam\-ment nette pour qu'on puisse dire ---~avec la 
pr\'ecision correspondante~--- que la tension augmente en moyenne 
lin\'eairement avec l'\^age. Mais si la corr\'elation est nulle, qu'il
existe cependant une nette d\'ependance non lin\'eaire (comme cela a 
\'et\'e illustr\'e sur la figure 50), les droites de r\'egression ne nous 
apprennent plus rien et ne nous permettent plus de corriger les 
donn\'ees statistiques. Il faut alors faire appel \`a la r\'egression 
non lin\'eaire (voir section suivante). On peut donc avoir calcul\'e 
soigneusement toutes les corr\'elations possibles et effectu\'e les 
corrections correspondantes, mais aboutir \`a des conclusions 
fausses simplement parce qu'une d\'ependance non lin\'eaire 
tr\`es forte aura \'echapp\'e \`a toutes ces pr\'ecautions. 
\vskip8pt plus5pt minus4pt 
Toutes ces r\'eserves sont n\'ecessaires pour montrer les limites de 
l'analyse statistique. Dans beaucoup de situations pratiques on 
rencontre suffisamment peu de corr\'elations ou de d\'ependances pour 
que ces m\'ethodes aient n\'eanmoins une valeur empirique correcte. 
Mais il est bien clair que tous les abus (pressentis et d\'enonc\'es par
le bon sens populaire) proviennent du non-respect de ces r\'eserves. 
Aucune m\'ethode math\'ematique ne permet d'en faire l'\'economie: on 
peut corriger des donn\'ees par des calculs de r\'egression (lin\'eaire
ou non lin\'eaire), mais les r\'esultats ne sont pleinement corrig\'es 
que si l'ensemble des d\'ependances a \'et\'e pris en compte; il est 
l\'egitime de n\'egliger les influences faiblement corr\'el\'ees avec le 
param\`etre \'etudi\'e, mais il suffit d'avoir ignor\'e un seul facteur 
fortement corr\'el\'e avec le param\`etre \'etudi\'e pour que la
conclusion perde toute validit\'e. Il est vain d'appliquer des
m\'ethodes math\'ematiques, aussi sophistiqu\'ees soient-elles: en 
aucun cas on ne pourra ainsi compenser l'omission des facteurs 
occultes. 
 
\vskip6mm plus3mm minus3mm 
\penalty-600
 
{\bf XII. 4. R\'egression non lin\'eaire.} 

\penalty600
\medskip 
La recherche de la droite de r\'egression d'une variable $Y$ par rapport
\`a une variable $X$ se ramenait \`a un probl\`eme de moindres
carr\'es: trouver, pour un nuage de points donn\'e (avec ou sans poids) 
la fonction affine $Y = f(X) = aX+b$ qui {\it minimise} les \'ecarts 
al\'eatoires.  Il va de soi que cette op\'eration est possible pour 
n'importe quelle fonction $Y = f(a,b\, ; X)$ d\'ependant de deux 
param\`etres $a$ et $b$ (ou plus).  On aurait pu chercher la {\it 
parabole des moindres carr\'es} pour les nuages de points des figures 
52. Mais on comprend bien que cette parabole n'aurait aucune 
pertinence: c'est flagrant pour la figure $52.8$, quoiqu'il puisse y avoir 
un doute pour la figure $52.1$. En effet, que la parabole qui rend 
minimum la somme des carr\'es des \'ecarts existe est une chose; que 
ce minimum soit petit en est une autre. C'est-\`a-dire que la parabole 
des moindres carr\'es a aussi peu de pertinence dans la figure $52.8$ 
que la droite de r\'egression n'en avait pour la figure 50. 
\medskip 
L'id\'ee de la r\'egression {\it non} lin\'eaire est d'effectuer cette 
g\'en\'eralisation: au lieu de chercher la droite de r\'egression d'un
nuage de points, chercher une courbe de r\'egression. Mais il n'existe 
\'evidemment aucune recette permettant de d\'eterminer le bon type
de courbe, qu'il n'y aurait ensuite plus qu'\`a ajuster. Si on a une
raison de penser que le nuage de points suit une parabole plut\^ot que
n'importe quel autre type de courbe, on peut poser pour la fonction
$Y = f(X) = a X^2 + b X + c$ (tout comme on avait pos\'e $f(X) = a X + b$ 
pour la r\'egression lin\'eaire), puis d\'eterminer les coefficients $a$, 
$b$, et $c$ de mani\`ere \`a rendre minimum la somme des carr\'es
des \'ecarts. Cherchons cette parabole des moindres carr\'es. 
\medskip 
La somme des carr\'es des \'ecarts est
$$S \;\; = \;\; \sum_{i=1}^n p_i\,\big[ y_i - f(x_i) \big]^2 \;\; 
= \;\; \sum_{i=1}^n p_i\,\big[ y_i - a x_i^2 - b x_i - c \big]^2$$ 
o\`u on a tenu compte du poids ou probabilit\'e $p_i$ de chaque point 
$(x_i,y_i)$: s'il s'agit de variables al\'eatoires $X$ et $Y$, les $p_i$
sont donn\'es par la loi conjointe, s'il s'agit d'un nuage empirique, on
aura $p_i=1/n$. En d\'eveloppant le carr\'e et en introduisant les
corr\'elations suivantes, qu'on peut appeler {\it corr\'elations
non lin\'eaires}: 
$$M_{\alpha,\beta}(X,Y) \;\; = \;\; \sum_{i=1}^n p_i\, x_i^\alpha y_i^\beta$$
on obtient
$$\eqalignno{ 
S \;\; &= \;\; M_{0,2} + a^2\, M_{4,0} - 2a\, M_{2,1} + 2ab\, M_{3,0}\, + \cr 
&\hskip6mm + (b^2 +2ac)\, M_{2,0} - 2b\, M_{1,1} - 2c\, M_{0,1} +  
2bc\, M_{1,0} + c^2 \cr }$$ 
Cette expression est minimum lorsque ses d\'eriv\'ees partielles par 
rapport \`a $a$, $b$, et $c$ s'annulent. La condition (n\'ecessaire) de 
minimum est donc le syst\`eme de trois \'equations lin\'eaires \`a 
trois inconnues $a$, $b$, $c$ suivant: 
$$\matrix{ 
a\; M_{4,0} &\xx &b\; M_{3,0} &\xx &c\; M_{2,0} &= & M_{2,1} \cr 
a\; M_{3,0} &\xx &b\; M_{2,0} &\xx &c\; M_{1,0} &= & M_{1,1} \cr 
a\; M_{2,0} &\xx &b\; M_{1,0} &\xx &c &= & M_{0,1} \cr } 
\eqno(XII.9)$$ 
Si au lieu d'une parabole, on cherche la courbe de r\'egression non 
lin\'eaire sous la forme d'un graphe de polyn\^ome de degr\'e $Q$, 
soit $Y = P(X) = \sum a_j X^j$, on obtiendra un syst\`eme de $Q+1$ 
\'equations lin\'eaires \`a $Q+1$ inconnues $a_j$ (les coefficients du 
polyn\^ome): 
$$\sum_{j=0}^Q a_j M_{k+j,0} \;\; = \;\; M_{k,1} \eqno(XII.10)$$ 
pour $k=0,1,2, \ldots Q$. 
\medskip
Un tel syst\`eme se r\'esoud num\'eriquement par la {\it m\'ethode du pivot}
bien connue.
\medskip 
Rien n'impose d'ailleurs les polyn\^omes et n'importe quelle fonction 
de la forme $Y = f(X) = \sum a_j \Phi_j(X)$, o\`u les $\Phi_j$ sont des 
fonctions de base, conduiront \`a un syst\`eme d'\'equations lin\'eaires 
dont les inconnues seront les coefficients $a_j$. Mais la lin\'earit\'e 
des conditions de minimum $XII.9$ et $XII.10$ provient \'evidemment
du fait que la fonction $f$ d\'epend elle-m\^eme lin\'eairement des 
param\`etres.  Si on cherche la courbe de r\'egression sous la forme
$Y = f(X) = a\e^{\alpha X}$, qui d\'epend des deux param\`etres $a$ et 
$\alpha$ (mais non lin\'eairement en $\alpha$), les conditions de 
moindre carr\'e seront alors, et pour cause, un syst\`eme de deux 
\'equations \`a deux inconnues, mais non lin\'eaires. 
\medskip 
On comprend que pour les calculs pratiques il vaut mieux chercher les 
courbes de r\'egression sous une forme qui soit lin\'eaire selon les
param\`etres. 
\medskip
Toutefois lorsqu'on a de bonnes raisons de chercher une d\'ependance
non lin\'eaire, il existe un proc\'ed\'e algorithmique efficace (du 
moins si le nombre de param\`etres n'est pas trop \'elev\'e) connu sous 
le nom de {\it m\'ethode de Levenberg-Marquardt}. Ce proc\'ed\'e est 
pr\'esent\'e dans l'annexe qui suit ce chapitre. L'algorithme de
Levenberg-Marquardt est parfaitement robuste quand il n'y a que deux 
param\`etres. Sa robustesse d\'ecro{\^\i}t ensuite avec le nombre de 
param\`etres: au del\`a de six, l'algorithme n'est plus gu\`ere fiable, 
mais il est tr\`es rare que la d\'ependance soit non lin\'eaire selon 
six param\`etres; presque toujours, on aura une d\'ependance lin\'eaire 
par rapport \`a la plupart des param\`etres, en sorte que seuls deux ou trois 
d'entre eux produiront la d\'ependance non lin\'eaire. Il suffira alors 
d'\'eliminer pr\'ealablement les premiers par la m\'ethode lin\'eaire 
du pivot, et traiter ensuite les deux ou trois param\`etres restants par 
la m\'ethode de Levenberg-Marquardt. 
\medskip 
Le probl\`eme g\'en\'eral de la d\'ependance n'est cependant pas
r\'esolu pour autant. Car si on peut ajuster les param\`etres d'une 
famille de courbes, on ne sait pas comment choisir cette famille de 
courbes.  Or la m\'ethode de Levenberg-Marquardt ne r\'esoud \'evidemment
pas ce probl\`eme l\`a;  elle permet seulement de calculer les param\`etres 
optimaux {\it pour une famille donn\'ee}.
\medskip 
Si la r\'egression lin\'eaire ne donne aucun r\'esultat, en ce sens 
que la {\it droite} de r\'egression correspond \`a un minimum {\it 
qui n'est pas petit} (ce qui se produit sur la figure 50), on peut 
chercher une {\it courbe} des moindres carr\'es qui correspond \`a un 
minimum petit; on peut par exemple augmenter le degr\'e du polyn\^ome 
jusqu'\`a ce que le minimum devienne petit. En proc\'edant ainsi, on est 
assur\'e du succ\`es, puisqu'on sait \`a l'avance que pour un polyn\^ome 
dont le degr\'e est {\it \'egal} au nombre de points du nuage, la somme
des carr\'es des \'ecarts sera nulle (polyn\^ome d'interpolation des 
points). Mais ce proc\'ed\'e, qui aboutira donc {\it forc\'ement} au 
succ\`es si le nombre de points du nuage est fini, fournira aussi une 
courbe de r\'egression d\'epourvue de la moindre pertinence. La courbe 
doit repr\'esenter une loi, dont le nuage de points serait une expression 
bruit\'ee; si on interpole les points du nuage, et qu'on refait une autre 
s\'erie de mesures qui fourniront un autre nuage, la courbe 
d'interpolation aura chang\'e. Or la {\it loi} du ph\'enom\`ene observ\'e 
(si elle existe) est ce qui ne change pas d'une s\'erie de mesures \`a 
l'autre, le {\it bruit} est ce qui change. 
\medskip 
On peut donc pr\'eciser comme suit les conditions dans lesquelles la 
r\'egression non lin\'eaire est recommand\'ee.  La r\'egression {\it 
lin\'eaire} est un probl\`eme bien pos\'e parce que les fonctions 
lin\'eaires, c'est-\`a-dire les rapports de proportionnalit\'e, jouent un 
r\^ole privil\'egi\'e dans l'\'etude des grandeurs. Math\'ematiquement
la r\'egression {\it non} lin\'eaire n'est un probl\`eme bien pos\'e que
si on choisit a priori une famille param\'etr\'ee de fonctions. 
Le probl\`eme formul\'e ainsi: ``parmi toutes les fonctions 
possibles et imaginables, laquelle minimise la somme des carr\'es
des \'ecarts?'' est \'evidemment absurde. En effet, parmi {\it toutes} 
les fonctions possibles et imaginables, il y en a forc\'ement une, et 
m\^eme une infinit\'e, qui rendent exactement nulle la somme des 
carr\'es des \'ecarts. Si $P(x)$ est le polyn\^ome
d'interpolation des $n$ points, pour lequel $y_i = P(x_i)$, on aura 
$S = \sum \big[ y_i - P(x_i)\big]^2 = 0$. \`A partir de ce polyn\^ome, 
on peut ensuite fabriquer une infinit\'e de fonctions ayant la m\^eme 
propri\'et\'e; soit en effet $Q(x)$ une fonction \'egale \`a $1$ lorsque 
$x$ est \'egal \`a l'un des $x_i$, mais \`a n'importe quoi en dehors 
des valeurs $x_i$ (rien n'emp\^eche une telle fonction d'\^etre 
diff\'erentiable ou analytique).  Alors toute fonction $f(x)$ de la forme 
$P(x) \cdot Q(x)$ remplira \'egalement les conditions requises. 
\medskip 
La r\'egression non lin\'eaire est donc une technique qui n'a de sens
que lorsqu'on a d\'ej\`a de bonnes raisons th\'eoriques de postuler un 
type particulier de d\'ependance, et qu'on souhaite seulement en
ajuster empiriquement les param\`etres. 
\medskip 
On appellera {\it mod\`ele de d\'ependance} un choix particulier d'une 
famille param\'etr\'ee de fonctions; selon le nombre des param\`etres, 
le mod\`ele sera dit {\it \`a $n$ param\`etres}. 
 \medskip 
Mais il ne suffit pas que le probl\`eme soit {\it math\'ematiquement} 
bien pos\'e gr\^ace \`a la pr\'esence d'un mod\`ele de d\'ependance. Il 
faut aussi que le mod\`ele postul\'e soit pertinent. Pratiquer la 
r\'egression non lin\'eaire en utilisant la m\'ethode de Levenberg-Marquardt, 
ou un logiciel qui la met en \oe uvre, est une 
simple technique. Trouver le mod\`ele pertinent est une question 
d'imagination, mais surtout d'intuition et d'exp\'erience, voire de 
g\'enie. Car la difficult\'e n'est pas d'ajuster des param\`etres pour 
minimiser la somme des carr\'es des \'ecarts, ni de postuler un 
mod\`ele plus ou moins arbitraire; la difficult\'e est de s\'eparer ce 
qui constitue la loi du ph\'enom\`ene (qui reste invariable d'une s\'erie 
de mesures \`a l'autre), et ce qui constitue le bruit. On peut tester a 
posteriori un mod\`ele de r\'egression non lin\'eaire par les deux 
crit\`eres suivants: 
\smallskip 
a) le mod\`ele doit rester valide pour toutes les s\'eries de mesures, 
c'est-\`a-dire que le minimum de la somme des carr\'es des \'ecarts
doit \^etre petit pour {\it toutes} les s\'eries de mesures que l'on 
effectue; 
\smallskip 
b) le bruit constitu\'e par les \'ecarts doit \'egalement conserver des 
caract\`eres statistiques constants d'une s\'erie de mesures \`a
l'autre. Par exemple si les \'ecarts sont distribu\'es selon une 
loi gaussienne (cas de loin le plus commun), et si 
l'\'ecart-type empirique change notablement d'une s\'erie de mesures 
\`a l'autre, il faudrait conclure que la s\'eparation entre loi et bruit 
effectu\'ee par le mod\`ele postul\'e n'\'etait pas correcte. 
\medskip 
Le crit\`ere $b$ ne peut \^etre n\'eglig\'e au profit du seul crit\`ere
$a$.  Imaginons par exemple qu'en \'etudiant la r\'egression lin\'eaire 
de deux variables $X$ et $Y$, on constate que les \'ecarts sont
\'elev\'es pour $X$ petit et deviennent de plus en plus faibles quand
$X$ augmente (voir par exemple figure 53.2). Lorsqu'on effectue des 
mesures il arrive bien plus souvent que les \'ecarts augmentent 
proportionnellement \`a $X$ au lieu de diminuer, car il est fr\'equent 
que les erreurs soient relatives plut\^ot qu'absolues. L'inverse est 
donc surprenant et devrait mettre la puce \`a l'oreille.  Si les 
fluctuations sont caus\'ees (comme les r\'esultats du chapitre {\bf 
VII} le font soup\c{c}onner) par d'innombrables petites perturbations 
al\'eatoires ind\'ependantes qui s'accumulent, on comprend 
difficilement pourquoi ces perturbations s'\'evanouissent lorsque $X$ 
devient grand. Il s'est donc produit un ph\'enom\`ene qui est 
int\'eressant par lui-m\^eme \ftn 1{Pour l'explication d'un tel 
ph\'enom\`ene, voir plus loin dans cette section.}. Si on veut analyser 
la situation de mani\`ere compl\`ete, la seule droite de r\'egression
ne suffit pas, il faut inclure une pr\'ediction concernant la variation 
\'eventuelle des \'ecarts. On appellera donc {\it mod\`ele de
r\'egression} (lin\'eaire ou non) un mod\`ele de d\'ependance (famille 
param\'etr\'ee de fonctions) accompagn\'e d'une pr\'ediction 
concernant le bruit (par exemple que le bruit est gaussien et que son 
\'ecart-type est proportionnel \`a $X$). En pratique les mod\`eles de 
r\'egression non lin\'eaire sont rarement autres que gaussiens et 
incluent toujours un \'ecart-type constant ou proportionnel \`a $X$, 
car les ph\'enom\`enes du type de la figure 53.2 sont des artefacts, 
comme nous le verrons plus loin. 
\medskip 
L'importance du crit\`ere $b$ intervient encore sous un autre aspect. 
Dans des \'etudes exp\'erimentales, on ne peut pas donner \`a la
variable $X$ des valeurs arbitrairement grandes ou arbitrairement 
petites; les valeurs accessibles sont en effet toujours limit\'ees par 
les possibilit\'es pratiques. C'est pourquoi il n'est pas toujours facile 
de distinguer une d\'ependance exponentielle d'une d\'ependance 
polyn\^omiale: par exemple $\e^X \simeq 1 + X + {1\over 2} X^2$; on 
pourrait ais\'ement d\'epartager l'exponentielle du polyn\^ome, 
m\^eme si le nuage est tr\`es flou, si on pouvait faire tendre $X$ vers 
l'infini.  Mais si les valeurs mesurables de $X$ sont limit\'ees, on ne le 
peut pas. Existe-t-il alors un moyen statistique de remarquer la 
diff\'erence? La r\'eponse est oui: supposons par exemple que le bruit 
soit gaussien et qu'on le constate pour les petites valeurs de $X$; 
supposons en outre que le ph\'enom\`ene soit exponentiel ``en 
r\'ealit\'e'', mais qu'on ait choisit un mod\`ele de d\'ependance de la 
forme $a + bX + cX^2$, qui apr\`es ajustage par moindres carr\'es donne 
$a=1$, $b=1$, $c={1\over 2}$.  Alors les \'ecarts ne seront plus 
exactement gaussiens pour les grandes valeurs de $X$, \`a cause d'une 
erreur syst\'ematique sur la moyenne.  Le mod\`ele exponentiel 
s\'eparerait mieux le signal du bruit que le mod\`ele polyn\^omial. 
\medskip 
Ainsi l'analyse statistique du seul bruit permet aussi de d\'etecter si
le mod\`ele de d\'ependance est correct. 
\medskip 
Ce qui implique que si l'un des deux crit\`eres $a$ et $b$ manque, le 
mod\`ele de r\'egression doit \^etre revu. Mais il n'existe pas de 
m\'ethode pour trouver un mod\`ele. La Statistique ne fournit que
des crit\`eres de v\'erification a posteriori. 
\medskip 
Un point doit encore \^etre pr\'ecis\'e. Lorsque les mesures donnent
des \'ecarts faibles, c'est-\`a-dire que le bruit est tr\`es petit par 
rapport au \hbox{signal}, que le nuage de points est concentr\'e au 
voisinage imm\'ediat d'une courbe qui saute aux yeux, comme sur la 
figure 52.8, alors les techniques statistiques de r\'egression sont 
\'evidemment superflues. Ces techniques deviennent utiles lorsque le 
bruit est important (ou lorsqu'il varie nettement avec l'une des 
variables,  comme sur les figures 53 \`a 57), c'est-\`a-dire lorsque le 
nuage est flou,  mais qu'on a de bonnes raisons de penser qu'il existe 
une loi noy\'ee dans ce bruit, et qu'il s'agit de la d\'ecouvrir. Comme 
nous avons pu le constater avec les tests statistiques (chapitre {\bf 
XI}),  la Statistique est un outil qui devient inutile quand les 
fluctuations al\'eatoires sont faibles. Il est inutile d'appliquer le test 
du $\chi^2$ pour v\'erifier l'\'equilibre d'un d\'e si on peut le lancer un 
million de fois et donc obtenir des fluctuations de l'ordre de $0.1\%$; 
ce test ne prend son sens que si, n'ayant jet\'e le d\'e que cent fois, les 
fluctuations sont trop fortes pour donner un r\'esultat flagrant. De 
m\^eme, les m\'ethodes de r\'egression ne sont utiles que si 
l'amplitude des fluctuations, c'est-\`a-dire l'\'epaisseur du nuage, est 
si grande qu'elle efface la loi du ph\'enom\`ene.  Ces m\'ethodes 
statistiques n'apportent cependant pas de certitude quant \`a la loi en 
question; elles ne doivent pas dispenser d'une \'etude plus exacte du 
ph\'enom\`ene (tout comme un alcootest ne doit pas dispenser d'une 
prise de sang), mais permettent tout au plus d'\'economiser du temps 
et des moyens en \'evitant de retenir pour une \'etude plus approfondie 
un mod\`ele d\'ej\`a incompatible avec les premi\`eres donn\'ees. 
\medskip 
Il faut aussi signaler que la recherche d'une {\it courbe} de
r\'egression se ram\`ene le plus souvent \`a la r\'egression lin\'eaire 
apr\`es un changement de variable ad\'equat. En pratique ce proc\'ed\'e 
est infiniment plus courant que la recherche directe des moindres 
carr\'es \`a partir d'un mod\`ele non lin\'eaire: il est tout 
particuli\`erement r\'epandu sous la forme du papier logarithmique, 
analogue au papier millim\'etr\'e, sauf que les abscisses et les
ordonn\'ees sont toutes deux gradu\'ees logarithmiquement (ou les
abscisses seulement sur papier semilogarithmique) et non
arithm\'etiquement. Ainsi une d\'ependance de la forme $Y = 
aX^\alpha$, c'est-\`a-dire une fonction puissance, se traduit sur
papier logarithmique par une droite de pente $\alpha$ et d'ordonn\'ee 
\`a l'origine $\log (a)$, puisque les nouvelles variables $\log (X)$ et 
$\log (Y)$ sont li\'ees par la d\'ependance lin\'eaire $\log (Y) = \alpha 
\log (X) + \log (a)$. Par cons\'equent, au lieu de chercher une courbe
de r\'egression de la forme $Y = a\, X^\alpha$ en d\'eterminant $a$ et 
$\alpha$ par les moindres carr\'es (ce qui est d'autant plus difficile 
que le param\`etre $\alpha$ n'intervient pas lin\'eairement), il est bien 
plus pratique et plus parlant de chercher la droite de r\'egression de 
$\log (Y)$ par rapport \`a $\log (X)$. Mais si on cherche \`a analyser 
l'\'ecart lui-m\^eme, par exemple pour savoir si l'\'ecart-type est 
constant en valeur relative ($\Delta X / X$) ou au contraire en valeur 
absolue ($\Delta X$), il peut \^etre pr\'ef\'erable d'employer la 
m\'ethode directe. 
\medskip 
Afin d'illustrer cela, on a donn\'e sur les figures 53 \`a 57 divers 
exemples de telles transformations non lin\'eaires. On peut constater
que si sur le nuage avant transformation, les \'ecarts sont homog\`enes, 
soit en valeur absolue ($\Delta X$), soit en valeur relative (${\Delta X 
\over X}$), il n'en est plus du tout de m\^eme apr\`es transformation. 
Il est ais\'e de donner une description math\'ematique de ces 
transformations: supposons que $Y = f(X) + \varepsilon$, o\`u $f$ est
une fonction non lin\'eaire et $\varepsilon$ une variable al\'eatoire 
repr\'esentant le bruit. Les points de coordonn\'ees $\big(x_i, 
y_i=f(x_i) + \varepsilon_i\big)$ sont les points du nuage. 
\medskip 

\midinsert 
\vbox to \blocksize{ 
\epsfxsize=\hsize
\line{\epsfbox{../imgEPS/ch12eps/fig53-1.eps} }
\vskip3mm 
\centerline{\eightpoint figure 53.1} 
\vskip3mm 
\centerline{\vbox{\hsize=11cm\eightpoint 
$y = ax^3 + \varepsilon$ 
} } 
\vfill 
\epsfxsize=\hsize
\line{\epsfbox{../imgEPS/ch12eps/fig53-2.eps} } 
\vskip3mm
\centerline{\eightpoint figure 53.2} 
\vskip3mm 
\centerline{\vbox{\hsize=11cm\eightpoint 
$y = \root 3 \of {ax^3 + \varepsilon }$ 
} } \vskip15pt }
\endinsert 

\midinsert 
\vbox to \blocksize{\epsfxsize=\hsize
\line{\epsfbox{../imgEPS/ch12eps/fig54-1.eps} } 
\vskip3mm
\centerline{\eightpoint figure 54.1} 
\vskip3mm 
\centerline{\vbox{\hsize=11cm\eightpoint 
$y = ax^3 \cdot (1 + \varepsilon )$ 
} } 
\vfill 
\epsfxsize=\hsize
\line{\epsfbox{../imgEPS/ch12eps/fig54-2.eps} } 
\vskip3mm
\centerline{\eightpoint figure 54.2} 
\vskip3mm 
\centerline{\vbox{\hsize=11cm\eightpoint 
$y = \root 3 \of {ax^3 \cdot (1 + \varepsilon )}$ 
} } \vskip15pt } 
\endinsert 
 
\midinsert 
\vbox to \blocksize{
\epsfxsize=\hsize
\line{\epsfbox{../imgEPS/ch12eps/fig55.eps} } 
\vskip-1mm
\centerline{\eightpoint figure 55} 
\vskip3mm 
\centerline{\vbox{\hsize=11cm\eightpoint \noindent
$\log y = \log (ax^3 + \varepsilon )$ avec $\varepsilon$ extr\^emement 
petit. Le trait horizontal est l'axe des $\log x$ (correspondant \`a
$y=0$). La pente de la droite de r\'egression est $3$, mais l'axe des 
$\log y$ a \'et\'e comprim\'e pour une meilleure mise en page. } } 
\vfill 
\epsfxsize=\hsize
\line{\epsfbox{../imgEPS/ch12eps/fig56.eps} } 
\vskip-1mm 
\centerline{\eightpoint figure 56} 
\vskip3mm 
\centerline{\vbox{\hsize=11cm\eightpoint 
$\log y = \log (ax^3 + \varepsilon )$ avec $\varepsilon$ mod\'er\'ement 
petit. } } \vskip15pt } 
\endinsert 

Repr\'esentons maintenant non plus les points $\big(x_i,y_i\big)$, 
mais les points $\big(x_i, g(y_i)\big)$, o\`u $x=g(y)$ est la fonction 
inverse de $y=f(x)$. Si $\varepsilon$ reste assez petit, on peut faire 
un d\'eveloppement limit\'e de $g(y_i) = g\big( f(x_i) + 
\varepsilon_i\big)$ en puissances de $\varepsilon_i$: 
$$\eqalign{ 
g(y_i) \;\; &\simeq \;\; g\big( f(x_i)\big) + g'\big( f(x_i)\big)\,
\varepsilon_i + {1 \over 2}\, g''\big( f(x_i)\big)\, \varepsilon_i^2 \cr 
&= \;\; x_i + {1 \over f'(x_i)}\;\varepsilon_i 
- {1 \over 2}\, {f''(x_i)\over f'(x_i)^3}\;\varepsilon_i^2 \cr }$$ 
On a utilis\'e les d\'eriv\'ees connues de la fonction inverse: 
$g'\big( f(x)\big) = 1/f'(x)$ et $g''\big( f(x)\big) = -f''(x)/f'(x)^3$. 
On en conclut que la d\'ependance non lin\'eaire entre $Y$ et $X$ 
devient une d\'ependance lin\'eaire entre $g(Y)$ et $X$, mais la 
r\'epartition des \'ecarts al\'eatoires $\big( 1/f'(x)\big)\, \varepsilon$ 
(le bruit) subit une distorsion. Si l'\'ecart-type de $\varepsilon$ est 
constant, alors dans la nouvelle repr\'esentation il variera en $1/f'(X)$. 
Si $f'(x)$ est croissante, le nouveau nuage de points sera flou pour les 
petits $x$, mais se resserrera au voisinage de la droite de r\'egression 
$y=x$ (voir figure 53.2). Si $f'(x)$ est d\'ecroissante, le nuage 
s'\'evasera pour les grandes valeurs de $x$. Le bruit sera r\'eparti de 
fa\c{c}on inhomog\`ene le long de la droite de r\'egression. 
\medskip 
 
\midinsert 
\epsfxsize=\hsize
\line{\epsfbox{../imgEPS/ch12eps/fig57.eps} } 
\vskip-1mm
\centerline{\eightpoint figure 57} 
\vskip3mm 
\centerline{\vbox{\hsize=11cm\eightpoint \noindent
$\log y = \log\big( ax^3 \cdot [1 + \varepsilon ]\big)$ avec 
$\varepsilon$ mod\'er\'ement petit. Le trait horizontal est l'axe des 
$\log x$ (correspondant \`a $y=0$). } } 
\endinsert 

Pour reprendre une remarque faite plus haut: si dans une s\'erie 
d'ob\-ser\-va\-tions on constate un tel ph\'enom\`ene, on peut en d\'eduire 
que les variables $X$ et $Y$ ne sont pas rapport\'ees \`a la bonne 
repr\'esentation. Une transformation non lin\'eaire ad\'equate des 
coordonn\'ees r\'etablira l'ordre. 
\medskip 
Lorsque le bruit appara{\^\i}t en valeurs relatives, le m\^eme 
d\'eveloppement limit\'e conduit \`a un bruit de la forme
$\big( f(x)/f'(x)\big)\, \varepsilon$ apr\`es transformation (figures 54.1
et 54.2). 
\medskip 
Supposons qu'on recherche par la m\'ethode des moindres carr\'es sur
les param\`etres $a$ et $\alpha$ la fonction $y = ax^\alpha$ de 
r\'egression du nuage de points de la figure 53.1, puis la droite de 
r\'egression du nuage de points de la figure 53.2 (ou bien la m\^eme 
chose pour les figures 54.1 et 54.2). Techniquement cela est plus 
compliqu\'e que si le param\`etre $\alpha$ intervenait lin\'eairement, 
mais ce n'est qu'une affaire de programmation facile et sans 
importance pour le propos qui suit. Il n'y a aucune raison pour que la 
droite soit exactement la transform\'ee de la courbe $y=a\, x^\alpha$ par
$g(Y) = \root 3\of {Y}$ (toutefois elle en sera assez proche si le bruit 
$\varepsilon$ est petit). C'est-\`a-dire que le proc\'ed\'e d'optimisation 
par les moindres carr\'es n'est pas covariant dans la transformation. 
Il ne revient au m\^eme, ni de rechercher la loi du ph\'enom\`ene, ni 
d'analyser le bruit dans les deux repr\'esentations, surtout si le bruit 
est de forte amplitude. 
\medskip 
On ne doit donc pas conclure que dans n'importe quel cas {\og il suffit de 
passer \`a une repr\'esentation o\`u la d\'ependance est lin\'eaire\fg}. 
Si le bruit est une perturbation d'un signal, on ne peut pas comprendre 
que sa dispersion varie bizarrement selon la valeur du signal, si ce 
n'est en admettant que la repr\'esentation est artificielle. En comparant 
par exemple les figures 53.1 et 53.2, on sera conduit \`a conclure 
que la r\'epartition du bruit en 53.1 est naturelle, mais que celle de
53.2 ne l'est pas. La repr\'esentation dans laquelle la d\'ependance
est lin\'eaire est donc {\og artificielle\fg}. Cela montre bien que la
r\'egression non lin\'eaire n'est pas \'equivalente \`a une r\'egression
lin\'eaire apr\`es transformation. 
\medskip 
On peut aussi faire subir aux deux coordonn\'ees \`a la fois une 
transformation non lin\'eaire; c'est ce que par exemple on effectue 
automatiquement lorsqu'on reporte des mesures sur papier 
logarithmique. Dans ce cas, au lieu de repr\'esenter $X$ en abscisse et 
$Y$ en ordonn\'ee,  on repr\'esente $\log (X)$ en abscisse et $\log (Y)$ en 
ordonn\'ee.  Cette repr\'esentation logarithmique est particuli\`erement 
indiqu\'ee pour les fonctions puissance $f(x) = a\, X^\alpha$: 
\medskip 
--- si $Y = a\, X^\alpha + \varepsilon$ (\'ecart absolu): 
$$\ln (Y) = \alpha\ln (X) + \ln (a) + \ln\Big(1 + {\varepsilon\over a\, 
X^\alpha}\Big)$$ 
 
--- si $Y = a\, X^\alpha (1+ \varepsilon )$ (\'ecart relatif): 
$$\ln (Y) = \alpha\ln (X) + \ln (a) + \ln\Big(1 + \varepsilon\Big)$$ 
 
Si $\varepsilon$ est petit et repr\'esente un \'ecart absolu 
uniform\'ement distribu\'e, on aura dans la repr\'esentation 
logarithmique un bruit en $\varepsilon / a\, X^\alpha$, qui est donc 
fortement variable le long de la droite de r\'egression. Si $\varepsilon$ 
est un \'ecart relatif, la transformation donnera un \'ecart absolu \'egal 
\`a $\varepsilon$. 
\medskip 
Le r\'esultat est montr\'e sur les figures 55, 56, et 57 pour $\alpha = 
3$. La figure 55 doit \^etre compar\'ee \`a 53.1: elle repr\'esente le 
m\^eme nuage de points, mais sur papier logarithmique. De m\^eme 56 
doit \^etre compar\'ee \`a 54.1. 
\medskip 
Ces figures illustrent tr\`es nettement le ph\'enom\`ene; si dans la 
repr\'e\-sen\-ta\-tion {\og naturelle\fg}, le bruit est distribu\'e 
uniform\'ement,  alors dans la nouvelle repr\'esentation o\`u la 
d\'ependance est rendue lin\'eaire, l'inhomog\'en\'eit\'e des \'ecarts 
peut \^etre consid\'erable. Cette id\'ee de repr\'esentation {\og naturelle\fg} 
est \`a rapprocher de la discussion \`a la fin de la section {\bf IX.1}. La 
r\'epartition du bruit est en effet li\'ee aux invariances naturelles de 
la Physique (spatiale et temporelle). Prenons pour exemple le cas 
illustr\'e par les figures 53.1 et 55 o\`u le bruit $\varepsilon$ est un 
\'ecart absolu uniforme, de densit\'e gaussienne. La d\'ependance de $y$ 
par rapport \`a $x$ est alors telle que $y = ax^3 + \varepsilon$. On peut 
dire que dans cette repr\'esentation {\og naturelle\fg}, l'\'ecart de $y$ par 
rapport \`a $ax^3$ est la somme d'un grand nombre de perturbations 
stochastiquement ind\'ependantes, puisque c'est g\'en\'eralement ainsi 
que sont engendr\'ees les fluctuations gaussiennes. Pour que des 
perturbations puissent s'ajouter, il faut qu'elles repr\'esentent des 
quantit\'es additives, par exemple des d\'eplacements dans l'espace. 
Notons que des erreurs de mesure se ram\`enent toujours en derni\`ere 
instance \`a des sommes de d\'eplacements dans l'espace; par exemple
les tremblements des aiguilles d'instruments en face d'une graduation 
sont bien des fluctuations spatiales (les affichages par cristaux
liquides rendent cela moins flagrant, mais ce n'est qu'un masque et on 
sera toujours ramen\'e \`a l'espace et au temps par une analyse plus 
pouss\'ee).  Si le bruit est distribu\'e uniform\'ement, c'est-\`a-dire
si l'\'ecart-type de $\varepsilon$ est constant (ind\'ependant de $x$), 
cela a une signification concr\`ete: c'est que les nombreux effets 
ind\'ependants qui contribuent \`a cr\'eer la fluctuation gaussienne ne 
sont pas influenc\'es par la valeur de $x$. L'uniformit\'e du bruit est 
donc le reflet d'une invariance occulte. 
\medskip 
En repr\'esentant alors les variables $x$ et $y$ sur papier 
logarithmique, on fait dispara{\^\i}tre, ou du moins on occulte davantage 
cette invariance. On voit bien que dans ce cas la repr\'esentation $y = 
ax^3 + \varepsilon$ est r\'eellement plus {\og naturelle\fg} que la 
repr\'esentation logarithmique, de m\^eme que les rep\`eres galil\'eens 
sont r\'eellement plus {\og naturels\fg} que les autres. 
\medskip 
En conclusion, le papier logarithmique (ou n'importe quel proc\'ed\'e
faisant appara{\^\i}tre artificiellement une d\'ependance lin\'eaire) est
certes un moyen commode pour faciliter la mise en \'evidence des 
d\'ependances, mais il ne faut jamais oublier que la structure du bruit 
peut elle aussi contenir des informations essentielles. L'\'etude des
corr\'elations (lin\'eaires) sur le graphique 55 effacerait ces
informations; seule l'\'etude de la r\'egression non lin\'eaire sur le 
graphique 53.1 permet de les obtenir. 
 
\vskip6mm plus3mm minus3mm

{\bf XII. 5. Comment trouver le bon mod\`ele: l'exemple de Planck.}
\medskip 
Lorsqu'on est confront\'e \`a des r\'esultats de mesures ou
d'exp\'eriences qui, report\'ees sur un papier millim\'etr\'e ou 
logarithmique se traduisent par un nuage de points, la grande 
difficult\'e n'est pas d'appliquer les techniques de Statistique, mais
de trouver le bon mod\`ele. Cela est encore plus vrai aujourd'hui (o\`u
le travail purement technique de mise en oeuvre des recettes 
statistiques est enti\`erement effectu\'e par des logiciels 
sp\'ecialis\'es) qu'autrefois. Les logiciels de r\'egression 
non lin\'eaire les plus performants attendent de l'utilisateur qu'il 
choisisse un {\it mod\`ele de r\'egression}. Si l'utilisateur n'en
propose pas de pertinent, les r\'esultats des calculs seront grossiers 
et peu significatifs, et la sophistication du logiciel ne compensera 
pas cette lacune (tout au plus elle servira \`a mieux tromper les 
profanes, objectif certes essentiel \`a notre \'epoque, mais hors de 
notre sujet).  On peut \'ecrire des manuels sur les techniques
statistiques, mais enseigner l'intuition, l'imagination, l'astuce, est 
impossible. Le pr\'esent ouvrage ne le pr\'etend pas non plus. 
Toutefois, cet ouvrage \'etant vou\'e \`a la signification et \`a la 
compr\'ehension des concepts scientifiques, ne peut laisser de
c\^ot\'e ce qui cons\-titue justement {\it la seule d\'emarche porteuse 
de sens et de compr\'ehension}, \`a savoir la cr\'eation de mod\`eles 
pertinents.  C'est pourquoi nous consacrons une section \`a pr\'esenter 
un exemple particuli\`erement f\'econd: la d\'ecouverte de la Physique 
quantique, Berlin, {\oldstyle 1899}\ftn{1}{Nous suivons en gros 
l'expos\'e original de Max Planck \`a la Soci\'et\'e allemande de 
Physique (s\'eance du 19 octobre {\oldstyle 1900}: {\it \"Uber eine 
Verbesserung der Wienschen Spectralgleichung} Verhandlungen der 
Deutschen Physikalischen Gesellschaft, Band 2, {\oldstyle 1900}, 
pages 237 -- 245.}. 
\medskip 
\`A la fin du chapitre {\bf II} nous avons rencontr\'e la loi de Planck, 
qui s'exprime par une fonction du type
$$f(\nu ) = {A\, \nu^3 \over\sdown{17}\e^{\alpha \nu \over T} - 1} 
\eqno (XII.\, 11)$$ 
o\`u $\nu$ est la fr\'equence et $T$ la temp\'erature absolue du
rayonnement. Les constantes $A$ et $\alpha$ \'etaient li\'ees \`a la 
constante de Planck $\hbar$ (voir {\bf II. 6}). 
Cette loi de d\'ependance de l'\'energie rayonn\'ee par le 
corps noir en fonction de la fr\'equence \'etait inconnue avant Planck. 
Planck ne l'a pas d\'eduite comme nous de raisonnements probabilistes 
faisant appel \`a la M\'ecanique quantique, c'est au contraire la 
M\'ecanique quantique qui a \'et\'e d\'eduite de la loi. Il l'a obtenue 
directement \`a partir de lois empiriques ant\'erieures \`a la 
Physique quantique. Dans ces lois classiques il n'y avait \'evidemment 
nulle trace de la constante $\hbar$, qui est justement issue de ce 
travail. 
\medskip 
Ce qui \'etait connu avant \'etaient les deux cas asymptotiques 
suivants: 
\smallskip 
a) lorsque la fr\'equence $\nu$ est petite 
$$f(\nu ) = {8\pi\nu^2 \over c^3}\; k\, T \eqno (XII.\, 12)$$ 
o\`u $c$ est la vitesse de la lumi\`ere et $k$ la constante de 
Boltzmann. Cette loi portait le nom de Rayleigh--Jeans\ftn 1{En toute 
rigueur historique, la loi de Rayleigh--Jeans n'\'etait pas encore 
connue officiellement au moment o\`u Planck r\'efl\'echissait au 
probl\`eme ({\oldstyle 1899}) puisque l'article de Rayleigh o\`u elle 
fut pr\'esent\'ee a paru en juin {\oldstyle 1900}, mais elle \'etait 
connue ``officieusement''. Ce d\'etail historique est toutefois sans 
int\'er\^et pour notre propos.}; elle avait \'et\'e obtenue par voie 
th\'eorique \`a partir des lois classiques du rayonnement,  mais ne se 
v\'erifiait exp\'erimentalement que dans l'infra-rouge,  c'est-\`a-dire 
pour les petites fr\'equences. 
\smallskip 
b) lorsque la fr\'equence $\nu$ est grande 
$$f(\nu ) = A\, \nu^3 \e^{-{\alpha \nu \over T}} \eqno (XII.\, 13)$$ 
appel\'ee loi de Wien, plut\^ot empirique, et r\'esultant 
des observations sur les rayonnements de haute fr\'equence 
(ultra-violet); les constantes $A$ et $\alpha$ \'etaient empiriques et 
non (comme pour la loi de Rayleigh--Jeans) exprim\'ees en fonction de 
constantes d\'ej\`a connues. On ne savait pas justifier th\'eoriquement 
la loi de Wien; c'\'etait une \'enigme de la Physique. 
\medskip 
Peut-on deviner la loi $XII.\, 11$ \`a partir de $XII.\, 12$ et $XII.\, 
13$?  Si on pose \`a un math\'ematicien le probl\`eme: ''trouver une 
fonction qui se comporte comme $1\over\textdown x$ pour $x$ 
petit et comme $\e^{-x}$ pour $x$ grand, il est probable qu'apr\`es un 
temps de r\'eflexion assez court il propose la fonction $1 / (\e^x - 1)$. 
Il est donc possible de deviner. Toutefois si on lit les publications de 
Planck \`a cette \'epoque charni\`ere de la Physique ({\oldstyle 1899} 
-- {\oldstyle 1900}) on constatera qu'il n'a pas proc\'ed\'e ainsi. Il a 
commenc\'e par chercher quelle expression pour l'entropie $S$ du 
rayonnement en fonction de la grandeur $U = c^3 f(\nu ) / 8\pi\nu^2$ 
impliquaient les relations $XII.\, 12$ et $XII.\, 13$, sachant que ${dS 
\over dU} = {1\over T}$.  La grandeur $U$ avait dans ce contexte un
sens particulier, qu'il est trop long de d\'evelopper ici.  Mais si on
r\'e\'ecrit $XII.\, 12$ et $XII.\, 13$ pour la grandeur $U$, cela donne 
respectivement: 
$$U = k\, T \eqno (XII.\, 12\, a.)$$ 
pour la loi de Rayleigh-Jeans et 
$$U = b\,\nu\, \e^{-{\alpha\nu\over T}} \eqno (XII.\, 13\, a.)$$ 
o\`u $b = c^3 A / 8\pi$,  pour la loi de Wien. 
\medskip 
On d\'eduit ais\'ement de $XII.\, 12\, a$ que ${1\over T} = 
{k \over U}$ d'o\`u
$${d^2S \over dU^2} \; = \; -{k \over U^2}$$
et de $XII.\, 13\, a$ que ${1\over T} = -{1 \over
\textdown{\alpha\nu }} \ln (U / b\nu )$ d'o\`u
$${d^2S \over dU^2} \; = \; -{1 \over \alpha\nu }\cdot {1\over U}$$ 
Ce d\'etour par l'entropie \'etait n\'ecessaire pour Planck; sa
sp\'ecialit\'e \'etait la Thermodynamique et il pensait pouvoir {\it 
comprendre} la vraie nature du ph\'enom\`ene par ce biais car (dit-il 
dans son autobiographie scientifique) ``l\`a, il se trouvait en terrain 
connu''. En introduisant la grandeur $R = 1\big/ {d^2S \over dU^2}$, on 
obtient dans le cas correspondant \`a $XII.\, 12\, a$ 
$$ R \; = \; -{1 \over k}\, U^2$$ 
et dans le cas correspondant \`a $XII.\, 13\, a$ 
$$ R \; = \; -\alpha\nu\, U$$ 
Autrement dit, la loi de Rayleigh--Jeans (pour les petites fr\'equences) 
conduisait \`a une d\'ependance quadratique entre $R$ et $U$, tandis 
que la loi de Wien (pour les hautes fr\'equences) conduisait \`a une 
d\'ependance lin\'eaire. 
\medskip 
Planck proposa alors d'essayer la combinaison la plus simple de ces
deux formules, sous la forme 
$$R \; = \; -{1 \over k}\, U^2 -\alpha\nu\, U \eqno (XII.\, 14)$$ 
de sorte que si $\nu$ est petite, le second terme devient n\'egligeable 
et le premier pr\'edomine, par contre si $\nu$ est grande c'est
l'inverse. Il suffit alors de revenir en arri\`ere pour retrouver $U$
en fonction de $T$: 
$${d^2S\over dU^2} \; = \; {\up{1}\over R} \; = \; - \, {1\over 
\; {1 \over k}\, U^2+\alpha\nu\, U \sdown{13}\; } \eqno (XII.\, 15)$$ 
La primitive de cette fonction de $U$ est ${1\over 
\textdown{\alpha\nu }} \ln ( 1 + b\nu/U )$, d'o\`u on d\'eduit que 
$${1\over T} \; = \; {dS\over dU} \; = \; {1\over\ldown{\alpha\nu }}\,
\ln\Big( 1 + {b\nu \over U}\Big) \eqno (XII.\, 16)$$ 
Ceci est une expression de ${1/ T}$ en fonction de $U$, qu'on peut 
inverser pour obtenir une expression de $U$ en fonction de $1/T$, qui
est
$$U \; = \; {b\nu\over\sdown{17}\e^{b\nu\over T} - 1}\eqno (XII.\, 17)$$ 
On constate que Planck a largement us\'e de la relation 
thermodynamique ${dS \over dU} = {1 \over T}$. En comparant $XII.\, 
17$ avec $II.\, 14$ on voit que $\alpha$ est \'egal \`a $2\pi\hbar / k$ 
(rappelons que la fr\'equence $\nu$ n'est pas \'egale \`a la pulsation 
$\omega$, mais \`a $\omega / 2\pi$). Les physiciens de {\oldstyle 
1900} ne pouvaient pas faire cette comparaison. Mais ils savaient que
le rayonnement de corps noir est ind\'ependant des mat\'eriaux qui 
constituent la cavit\'e, et que par cons\'equent les constantes $b$ et 
$\alpha$ devaient \^etre (comme $c$ ou $k$) des constantes
universelles de la Physique. 
\medskip 
C'est ainsi que $\hbar$ (ou plut\^ot $h = 2\pi\hbar$) apparut pour la
premi\`ere fois,  d'o\`u son nom de {\it constante de Planck}. 
\medskip 
Cette histoire est int\'eressante pour nous ici parce qu'elle montre 
comment on devine un mod\`ele. Planck n'aurait jamais trouv\'e 
l'explication fondamentale du rayonnement du corps noir s'il avait 
cherch\'e un polyn\^ome qui approche la courbe empirique \`a toutes
les fr\'equences. Si l'ambition de la Physique consistait \`a approcher 
les courbes empiriques par des mod\`eles math\'ematiques d\'epourvus 
d'une signification plus profonde, et ayant pour seule vertu de se 
rapprocher le plus possible des courbes fournies par l'exp\'erience, ce 
ne serait plus la Physique et Planck n'aurait pas d\'ecouvert $\hbar$
(ni Newton la gravitation, etc). La vraie nature de la d\'emarche de 
Planck d\'ecrite ci-dessus est d'avoir attribu\'e a priori aux deux lois
empiriques $XII.\, 12$ et $XII.\, 13$ un sens plus profond que la simple 
description des faits exp\'erimentaux. Au d\'epart de sa d\'emarche 
figurait la conviction intime que les courbes exp\'erimentales sont
des apparences, sous-tendues par des causes {\it intelligibles} qu'on
ne peut atteindre que par la pens\'ee (cf. Platon, {\sl La R\'epublique} 
Livre {\bf VII});  sa tactique n'a pas consist\'e \`a approcher co\^ute 
que co\^ute la courbe exp\'erimentale pour toutes les fr\'equences, 
mais \`a partir du principe (dict\'e par une intime conviction et non
par l'observation) que la {\it cause intelligible} du ph\'enom\`ene se 
r\'ev\'elerait mieux dans les deux cas asymptotiques $\nu \rightarrow 
0$ et $\nu \rightarrow \infty$ que dans le cas interm\'ediaire des 
fr\'equences moyennes. Le passage par l'entropie $S$ est la meilleure 
preuve que son approche fut bien celle-l\`a; mais il en t\'emoigne 
aussi lui-m\^eme dans nombre de textes. 
\medskip 
Un point essentiel de la d\'emarche (revendiqu\'e par Planck) est 
\'egalement le passage par les relations les plus simples possibles. 
Deviner la loi g\'en\'erale \`a partir des deux cas asymptotiques 
directement sous la forme $XII.\, 12$ et $XII.\, 13$ e\^ut \'et\'e 
possible comme nous l'avons indiqu\'e plus haut. Mais Planck a 
pr\'ef\'er\'e passer par l'interm\'ediaire de $R = 1\big/ 
\hbox {\eightpoint ${d^2S \over \sdown{6.7}dU^2}$}$ car 
les relations atteignaient alors le maximum de simplicit\'e (relation 
lin\'eaire ou quadratique) et surtout, le maximum de {\bf 
signification}. Le proc\'ed\'e est toujours le m\^eme:  passer par 
l'intelligible plut\^ot que par une description purement empirique. 
\medskip 
On peut tirer de cet exemple magistral deux le\c{c}ons 
essentielles \`a retenir lorsqu'on doit chercher un mod\`ele. Ces
le\c{c}ons sont valables m\^eme quand on ne nourrit pas l'ambition de 
r\'evolutionner la Physique: 
\smallskip 
{\leftskip=20pt 
a) Ne pas chercher simplement \`a approcher le nuage de points ou la 
courbe empirique, mais r\'efl\'echir d'abord et essayer de trouver \`a 
quelle n\'ecessit\'e purement intelligible le mod\`ele doit ob\'eir 
(sym\'etries ou invariances, analogie avec des situations connues, 
n\'ecessit\'es logiques). Nous avons d\'ej\`a fait appel avec insistance 
\`a ce type de d\'emarche dans le pr\'esent ouvrage, en indiquant que 
pour r\'esoudre un probl\`eme de probabilit\'e ``il faut commencer par 
chercher ce qui est \'equiprobable''). 
\smallskip 
b) Effectuer sur les variables qui sont en jeu des transformations 
qui ram\`enent autant que possible aux types de d\'ependance les plus 
simples: proportionnalit\'e ou \`a la rigueur d\'ependance quadratique. 
Ainsi une d\'ependance du type $Y = a\, X^\alpha$ devient une
r\'egression lin\'eaire entre $\log (Y)$ et $\log (X)$ (papier 
logarithmique), une d\'ependance du type $Y = a\,\e^{\alpha X}$
devient une r\'egression lin\'eaire entre $\log (Y)$ et $X$ (papier 
semilogarithmique). \par }
\medskip
Bien entendu, pour d\'ecouvrir la M\'ecanique quantique, il a fallu
pousser l'imagination un peu plus loin, mais le principe reste le m\^eme. 

\vskip6mm plus3mm minus3mm

{\bf XII. 6. Corr\'elation et causalit\'e.} 
\medskip 
L'id\'ee essentielle qui se cache derri\`ere ces probl\`emes de 
corr\'elation ou de r\'egression est celle de la {\it causalit\'e}. 
Dans le langage courant, y compris les jargons techniques (et m\^eme
{\it surtout} dans les jargons techniques) la diff\'erence entre la 
d\'ependance statistique et la causalit\'e n'est pas clairement 
d\'efinie. C'est pourquoi nous allons encore discuter du sens des
diff\'erentes notions introduites dans ce chapitre, \`a savoir 
la corr\'elation ou la r\'egression. Cette discussion compl\'etera les 
discussions d\'ej\`a amorc\'ees au chapitre {\bf X}\hskip6pt sur la 
signification des mesures statistiques. Nous avions alors bien insist\'e 
sur la diff\'erence entre la mesure d'une probabilit\'e a priori (cas de 
l'exp\'erience reproductible) et la mesure d'une proportion dans une 
population (cas du sondage), les deux pouvant parfois se recouvrir. Ici, 
nous ne discutons plus les mesures simples, mais la d\'ependance. 
\medskip 
Nous pouvons en effet interpr\'eter une forte corr\'elation, par
exemple dans le cas extr\^eme o\`u le coefficient de corr\'elation est 
\'egal \`a $+1$ ou $-1$, comme une d\'ependance lin\'eaire. Une telle 
d\'ependance n'est pas flagrante dans le cas de la corr\'elation entre 
la tension art\'erielle et l'\^age, car cette corr\'elation est m\'ediocre, 
et correspond \`a peu pr\`es \`a ce qu'on peut voir sur les figures 
$52.2$ ou $52.3$. La situation serait beaucoup plus claire avec un nuage 
de points tel que celui de la figure $52.7$ ou $52.8$. Ce type de figure 
peut vous rappeler un T.P. d'\'electricit\'e o\`u il s'agissait de 
v\'erifier la loi d'Ohm: supposons qu'on reporte la tension \'electrique 
$U$ entre les bornes de la r\'esistance en abscisse et l'intensit\'e $I$ 
du courant en ordonn\'ee; on obtiendrait bien des graphiques de ce type. 
\medskip 
Les lois de l'\'electromagn\'etisme sont souvent enseign\'ees
aujourd'hui de fa\c{c}on d\'eductive. Les \'equations de Maxwell sont 
postul\'ees comme s'il s'agissait d'une v\'erit\'e math\'ematique a 
priori, dont on d\'eduit tout le reste; la r\'esistance d'un conducteur 
n'est alors qu'un effet macroscopique sur un nombre \'enorme de 
mol\'ecules et d'\'electrons et on retrouverait la loi d'Ohm, en 
expliquant {\og tout\fg} par moyennisation \`a partir des seules \'equations 
fondamentales de Maxwell.  Mais le cheminement historique va en sens 
inverse, et les \'equations de Maxwell ont \'et\'e peu \`a peu {\it 
d\'egag\'ees} par induction \`a partir d'observations exp\'erimentales, 
telles que justement cette loi d'Ohm qui a \'et\'e \'etablie par Georg S. 
Ohm en effectuant des mesures de tension \'electrique et d'intensit\'e. 
\medskip 
Si on effectue r\'eellement ces mesures, ce qu'on obtient est un 
tableau de chiffres ou, si on les repr\'esente graphiquement, un nuage 
de points comme celui de la figure $52.8$. Si on calcule le coefficient 
de corr\'elation du nuage on obtiendra une valeur proche de $1$ et une 
droite de r\'egression dont la pente est appel\'ee la r\'esistance $R$ 
du conducteur. $U = RI$ est l'\'equation de la droite de r\'egression. 
\medskip 
Toutes les lois de la physique ont \'et\'e, soit obtenues directement 
par un tel proc\'ed\'e, soit d\'eduites ou induites math\'ematiquement 
\`a partir de tels proc\'ed\'es. Ce n'est que lorsque la forme 
math\'ematique issue de tout ce travail devient vraiment abstraite 
(par exemple lorsqu'elle prend la forme des \'equations de Maxwell) 
qu'on croit pouvoir lui donner une valeur de v\'erit\'e sup\'erieure, 
qui serait plus exacte que ce que la variance des mesures exp\'erimentales 
laisse supposer. On dit alors que les lois abstraites sont les
{\og vraies\fg} lois (par exemple $U=RI$) et que les petits \'ecarts
du nuage de points par rapport \`a cette loi sont {\og du bruit\fg}. 
Mais cette l\'egende sur la nature des lois physiques ne doit pas faire
oublier qu'au d\'epart, toute l'information a \'et\'e recueillie dans
des graphiques tels que ceux de la figure $52.8$. Il n'est pas difficile
de comprendre le proc\'ed\'e litt\'eraire qui permet de passer d'un nuage
de points comme celui de la figure $52.8$ \`a une loi math\'ematique
comme les \'equations de Maxwell: la premi\`ere \'etape consiste \`a
dire que $U$ est {\it \'egal} \`a $RI$ au lieu de dire que la variance
d'\'echantillon de $U - RI$ est petite. 
\medskip 
Mais dans la loi d'Ohm il y a m\^eme bien plus que la seule relation 
math\'ematique $U = RI$. On dit en effet que la tension \'electrique 
appliqu\'ee aux bornes est la {\it cause} du courant. 
\medskip
Cette id\'ee de causalit\'e est tout \`a fait \'evidente si on con\c{c}oit 
le courant \'electrique selon l'\'electromagn\'etisme moderne. En effet, le
courant \'electrique est un d\'eplacement d'\'electrons dans le conducteur, 
et les \'electrons ne se d\'eplacent que si on leur applique un champ 
\'electrique. Le champ \'electrique est alors la cause du mouvement des 
\'electrons comme le champ de gravit\'e est la cause de la chute des 
corps. Or on cr\'ee un tel champ dans le conducteur en appliquant une 
tension \'electrique aux bornes, ce qui veut dire que la tension 
\'electrique est la cause du champ et donc aussi la cause du courant 
\'electrique. 
\medskip
Mais une simple d\'ependance ne suffit pas \`a d\'eterminer
ce qui est la cause et ce qui est l'effet. Par exemple, on aurait tout
aussi bien pu \'ecrire $I = U/R$, cela n'aurait pas pour autant fait de 
$I$ la cause et de $U$ l'effet. On voit donc qu'une corr\'elation 
statistique, m\^eme tr\`es forte, ne peut pas \`a elle seule d\'ecider
ce qui est la cause ou ce qui est l'effet. La simple constatation d'une 
corr\'elation entre le tabagisme et l'hypertension ne suffit pas \`a 
certifier que le tabagisme est la cause et l'hypertension l'effet; il se 
pourrait par exemple que l'hypertension ait une cause biologique qui 
favorise chez la m\^eme personne le go\^ut pour le tabac.  
\medskip
La causalit\'e est donc une relation plus forte que la d\'ependance 
statistique. Si on a observ\'e sur un \'echantillon de mesures les
valeurs de deux param\`etres (tels que \^age et tension art\'erielle, 
ou intensit\'e et tension \'electrique), et que le nuage de points est 
distribu\'e le long d'une courbe d'\'equation $y = f(x)$ de telle sorte que 
la variance de $y-f(x)$ sur l'\'echantillon de tous les points est petite, 
on dira qu'il y a une forte d\'ependance statistique entre les deux 
param\`etres (si la fonction $f$ est lin\'eaire, on appellera cela une 
corr\'elation). Pour qu'en outre on puisse dire que le premier 
param\`etre (celui qui est en abscisse) est la {\it cause} de l'autre, il 
faut que la r\'ealisation du premier pr\'ec\`ede toujours la r\'ealisation 
du second (en Relativit\'e, la condition est encore plus forte: il faut que 
le premier soit r\'ealis\'e avant le second, mais de sorte qu'un photon 
parti du premier imm\'ediatement apr\`es sa r\'ealisation puisse
arriver au second avant sa r\'ealisation). 
\medskip
On peut r\'esumer cela en disant que
$$\hbox{causalit\'e} = \hbox{d\'ependance statistique} + 
\hbox{ant\'eriorit\'e} \eqno (XII.18.)$$
Au del\`a de cette d\'efinition de la causalit\'e, on peut 
s'interroger sur le probl\`eme de savoir si la causalit\'e se r\'eduit 
enti\`erement \`a $(XII.18.)$; si deux param\`etres ont entre eux une 
forte d\'ependance statistique et un rapport d'ant\'eriorit\'e, mais sans 
qu'on puisse pour autant {\it comprendre} la relation de n\'ecessit\'e 
qui les relie,  peut-on encore parler de causalit\'e?  Cette question a 
\'et\'e \'etudi\'ee au $XVIII^{e}$ si\`ecle par David Hume\ftn1{David 
HUME {\it Enqu\^ete sur l'entendement humain} ({\oldstyle 1748}).} qui 
a r\'epondu que la causalit\'e ne peut pas \^etre {\it plus} que ce qui 
est exprim\'e par $(XII.18.)$ Quoique sa r\'eponse ait \'et\'e critiqu\'ee, 
je dois lui donner enti\`erement raison. Les critiques qui lui ont \'et\'e 
oppos\'ees reposent essentiellement sur l'argument que voici: 
supposons que, chaque fois qu'un \'ev\'enement $A$ se produit, un 
\'ev\'enement $B$ se produit peu apr\`es, et $B$ ne se produit jamais 
sans que $A$ se soit produit juste avant; on ne peut alors 
s\'erieusement tenir $A$ pour la cause de $B$ que si le rapport de 
n\'ecessit\'e entre $A$ et $B$ est clair pour l'entendement. 
C'est-\`a-dire que $(XII.18.)$ ne suffit pas, il faut en outre un rapport 
de n\'ecessit\'e. Cette situation est illustr\'ee sous forme comique 
dans {\it La vie criminelle d'Archibald de la Cruz}, un film de Luis 
Bunuel ({\oldstyle 1955}), dont le personnage principal (Archibald de 
la Cruz) constate que, chaque fois que dans sa vie il a \'eprouv\'e de
l'hostilit\'e envers une personne, celle-ci est morte quelques heures
ou quelques jours apr\`es. Il acquiert ainsi la certitude que, par un 
m\'ecanisme inconnu, son sentiment de haine est la {\it cause} de la 
mort de ses victimes. Lorsque, tortur\'e par les remords, il se livre
\`a la police, personne ne le croit. Or, une situation toute analogue
s'est produite lorsque Kepler a d\'ecouvert les lois du mouvement des 
plan\`etes. En effet, Kepler a constat\'e que le temps mis par une une 
plan\`ete quelconque pour parcourir une portion de son orbite est 
proportionnel \`a l'aire balay\'ee par le segment Soleil -- plan\`ete 
(loi des aires). Quoique \`a cette \'epoque les m\'ethodes
statistiques n'\'etaient pas encore aussi syst\'ematiquement 
quantitatives, ni surtout aussi sophistiqu\'ees math\'ematiquement 
qu'aujourd'hui, on peut n\'eanmoins, avec un anachronisme certain, mais 
qui ne touche pas \`a l'essentiel, imaginer Kepler repr\'esentant les 
r\'esultats de ses mesures et de ses calculs sur un graphique, avec les 
dur\'ees en ordonn\'ee et les aires balay\'ees en abscisse. Les 
points trac\'es ont bien la caract\'eristique de former un nuage 
concentr\'e au voisinage d'une droite. Il y a donc une d\'ependance 
entre la vitesse et la distance au Soleil; lorsque la plan\`ete s'approche 
du Soleil, elle va plus vite, et la loi des aires permet de calculer cette 
variation. 
\medskip
Or, constatant cela, Kepler ne s'est pas content\'e d'en tirer une loi 
purement descriptive; voyant que le Soleil \'etait au foyer de toutes
les orbites des plan\`etes, il a estim\'e que le Soleil ne pouvait pas ne 
pas \^etre la {\it cause} du mouvement des plan\`etes: en effet, les 
autres plan\`etes ne pouvaient pas \^etre tenues pour la cause du 
mouvement de l'une d'entre elles, car elles \'etaient interchangeables; 
or par d\'efinition, la cause n'est pas interchangeable avec l'effet. 
Seul le Soleil pouvait jouer ce r\^ole. On voit bien que la simple 
corr\'elation constat\'ee ne suffit pas \`a conclure \`a une causalit\'e, 
il a fallu faire appel \`a un raisonnement a priori. 
\medskip 
Mais ce n'est pas tout. La distance entre le Soleil et la plan\`ete
\'etant \'enorme, comment la plan\`ete, lorsqu'elle passe \`a un endroit
de son orbite, peut-elle {\og savoir\fg} que, s'\'etant par exemple 
rapproch\'ee du Soleil, il lui convient d'augmenter sa vitesse pour 
respecter la loi des aires? D'autre part, la loi des aires est valable 
pour chaque plan\`ete s\'epar\'ement, et ne fait intervenir que le 
couple plan\`ete-Soleil, de sorte que le mouvement d'une plan\`ete
est ind\'ependant des autres plan\`etes. Il faut dire que du temps de 
Kepler, on ne connaissait pas encore la gravitation, et \`a plus forte 
raison, on ne savait pas que chaque plan\`ete subit aussi l'attraction 
des autres plan\`etes. La gravitation du Soleil \'etant de beaucoup la 
plus forte, et la pr\'ecision des observations \'etant insuffisante, 
Kepler ne pouvait pas percevoir l'influence mutuelle des plan\`etes. 
Si donc une des plan\`etes disparaissait brusquement, cela ne devrait 
rien changer \`a la validit\'e de la loi des aires pour celles qui restent. 
Donc le Soleil (c'est-\`a-dire la cause) agissait {\it sans savoir} si 
les plan\`etes \'etaient l\`a pour en subir les effets. Comment la 
corr\'elation observ\'ee est-elle possible s'il n'y a pas une 
communication entre le Soleil et la plan\`ete? L'\'etude des textes 
laiss\'es par Kepler montre qu'il avait \'et\'e troubl\'e par cette 
question et qu'il avait cherch\'e une r\'eponse. Son explication a \'et\'e 
la suivante. Puisque le Soleil est la cause et que cette cause produit 
son effet \`a des centaines de millions de kilom\`etres, c'est que le 
Soleil doit \'emettre constamment un fluide qui se diffuse dans 
l'espace, et ce fluide, lorsqu'il arrive \`a la plan\`ete, lui donne de la 
vitesse. Kepler a alors tent\'e de calculer la loi d'action de ce fluide 
sur la plan\`ete, en partant du principe que, puisque le Soleil agit sans 
savoir s'il existe ou non des plan\`etes, il doit \'emettre son fluide 
ind\'ependamment des plan\`etes, donc isotropiquement (c'est-\`a-dire 
uniform\'ement dans toutes les directions). La quantit\'e de fluide 
pr\'esente sur un petit morceau d'orbite est alors facile \`a calculer, 
mais ce qu'on trouve n'est pas proportionnel \`a la vitesse. Kepler, 
persuad\'e qu'en vertu d'une causalit\'e n\'ecessaire il {\it devait} y 
avoir quelque chose comme un fluide, n'a jamais r\'eussi \`a faire 
co{\"\i}ncider cette id\'ee avec les calculs quantitatifs, mais les
textes montrent qu'il a beaucoup cherch\'e. C'est Newton qui a trouv\'e 
la solution du probl\`eme, en disant que ce n'est pas un fluide (dont la 
densit\'e est forc\'ement une grandeur scalaire) mais une force (qui 
est un vecteur) qu'il convenait d'invoquer; et qu'en outre, suivant 
Galil\'ee, la force agit sur l'acc\'el\'eration et non sur la vitesse. 
\medskip
La moralit\'e de cette histoire est la suivante: l'observation n'a
jamais fourni que des corr\'elations, mais les corr\'elations brutes
ne fournissent aucune {\it compr\'ehension} du ph\'enom\`ene. 
Comprendre un ph\'enom\`ene signifie y distinguer des \'el\'ements
qui sont des causes et d'autres qui sont des effets, et pour cela, 
l'entendement est oblig\'e d'inventer de toutes pi\`eces des 
m\'ecanismes abstraits qui ne sont pas observables en tant que 
tels. Ainsi Archibald de la Cruz ne pouvait pas comprendre la 
corr\'elation entre ses fantasmes meurtriers et la mort violente de
ses victimes sans faire appel \`a un fluide, ou \`a une force, \'emis par 
son esprit et se propageant dans l'espace, ou tout autre m\'ecanisme 
bizarre d\'eclench\'e dans son cerveau et finissant par produire le 
d\'eplacement d'objets contondants \`a proximit\'e de ses victimes. 
C'est en ce sens que David Hume a raison: si la science, d'une fa\c{c}on 
ou d'une autre, d\'ecouvre par l'observation brute une relation telle
que $(XII.18.)$, mais sans que l'entendement humain puisse y trouver 
aucune relation de n\'ecessit\'e a priori, la r\'eaction de la 
post\'erit\'e ne sera pas de nier que $(XII.18.)$ soit une v\'eritable 
relation de cause \`a effet, mais de chercher par tous les moyens un 
artifice conceptuel, tel que les fluides ou les forces, qui {\it ajoutera} 
\`a $(XII.18.)$ le caract\`ere de n\'ecessit\'e logique qui lui manquait, 
afin de rassurer l'entendement. Autrement dit, si les \'ev\'enements 
$A$ et $B$ sont li\'es par $(XII.18)$,  l'absence d'une n\'ecessit\'e 
logique entre eux n'est pas une raison valable pour nier que $A$ soit
la cause de $B$, car une telle n\'ecessit\'e logique peut toujours
\^etre invent\'ee: il suffit d'avoir de l'imagination. En revanche, la 
d\'ependance statistique et l'ant\'eriorit\'e ne peuvent pas \^etre 
invent\'ees, car elles doivent se conformer aux observations. 
\medskip 
Ainsi les r\'eflexions de Hume dans {\it Enqu\^ete sur 
l'entendement humain} ne sont pas des sp\'eculations philosophiques
que les scientifiques peuvent n\'egliger, mais sont \`a la base m\^eme 
de l'esprit scientifique et ont une incidence {\it pratique} sur la 
m\'ethode: les ignorer m\`ene tout droit \`a l'erreur. L'insistance de 
Hume \`a vouloir \'evacuer la relation de n\'ecessit\'e n'est pas 
l'expression d'un positivisme dogmatique,  mais d'une exigence de 
rigueur: d\'efinir la causalit\'e comme une simple corr\'elation avec 
ant\'eriorit\'e, c'est la soumettre totalement et exclusivement au 
verdict de l'exp\'erience objective; au contraire, la d\'efinir comme une
corr\'elation avec relation de n\'ecessit\'e, c'est introduire un 
\'el\'ement de subjectivit\'e dans la connaissance. Autrement dit, 
refuser de reconna{\^\i}tre une corr\'elation avec ant\'eriorit\'e comme 
causalit\'e au pr\'etexte qu'il y manque une relation de n\'ecessit\'e 
\'equivaut \`a s'interdire de {\it chercher} une nouvelle relation de 
n\'ecessit\'e que la science ne poss\`ede pas encore, (donc interdire
\`a Kepler d'inventer son histoire de fluide, \`a Newton d'inventer les 
forces, \`a Faraday d'inventer les champs, etc.) et donc \`a soumettre 
la connaissance future \`a la connaissance pr\'esente. De m\^eme, 
voir de la causalit\'e dans une corr\'elation {\it sans ant\'eriorit\'e 
\'etablie} au pr\'etexte qu'on y voit une relation de n\'ecessit\'e 
\'equivaut \`a donner \`a un pr\'ejug\'e la priorit\'e sur l'exp\'erience. 

\vskip20mm plus4mm minus3mm 

\centerline{\tit ANNEXE DU CHAPITRE XII.} 

\vskip10mm plus4mm minus3mm 

\centerline{\bf LA M\'ETHODE DE LEVENBERG-MARQUARDT} 
\smallskip 
\centerline{\bf pour le calcul de la r\'egression non lin\'eaire.} 
 
\vskip6mm plus4mm minus3mm 
 
Comme on l'a vu \`a la section {\bf 12. 4}, un probl\`eme de
r\'egression non lin\'eaire consiste \`a chercher le minimum de la 
fonction 
$$S(\alpha ) \; = \; \sum_{i=1}^{n} p_i\, \big[ f(\alpha , x_i) - y_i\big]^2 
\eqno (12\, A.1)$$ 
en faisant varier les param\`etres $\alpha = (\alpha_1, \alpha_2, 
\ldots \alpha_Q)$. Ce minimum \'etant un point o\`u le gradient de $S$
s'annule, cela revient \`a trouver les $\alpha$ tels que $\grad S = 0$. 
\medskip 
Lorsque le mod\`ele de r\'egression non lin\'eaire $f(\alpha , 
x_i)$ d\'epend lin\'eairement des param\`etres \`a optimiser,  la
fonction $S$ est du second degr\'e, donc son gradient est lin\'eaire. 
L'\'equation $\grad S = 0$ est alors un syst\`eme lin\'eaire de $Q$
\'equations \`a $Q$ inconnues qu'on r\'esoud num\'eriquement par la
m\'ethode du pivot de Gauss. 
\smallskip 
{\bf N. B.} On notera que le terme {\it lin\'eaire} (ou le terme 
{\it non lin\'eaire}) a, dans le pr\'esent contexte, deux sens ind\'ependants 
\`a ne pas confondre: le mod\`ele de r\'egression {\it non lin\'eaire}
$f(\alpha , x_i)$ peut d\'ependre {\it lin\'eairement} ou non des
param\`etres $\alpha$ : $y=f(\alpha , x)$ est un mod\`ele de r\'egression 
{\it non lin\'eaire} si $f$ d\'epend non lin\'eairement de $x$, mais il peut 
d\'ependre lin\'eairement de $\alpha$; ainsi $f(\alpha , x) = \alpha_1 x^2 
+ \alpha_2 x + \alpha_3$ est un mod\`ele de r\'egression non lin\'eaire, qui 
d\'epend lin\'eairement de $\alpha$. 
\medskip 
Lorsque la d\'ependance en $\alpha$ n'est pas lin\'eaire on doit traiter
un syst\`eme non lin\'eaire de $Q$ \'equations \`a $Q$ inconnues et il 
existe pour cela un algorithme num\'erique qui a fait ses preuves, 
connu sous le nom de {\sl Levenberg -- Marquardt}\ftn{1}{D. W. 
Marquardt J. Soc. Ind. Appl. Math. vol {\bf 11} ({\oldstyle 1963}) pp 431 
-- 441.}.  Mon exp\'erience d'enseignement dans une \'ecole 
d'ing\'enieur m'a appris que les probl\`emes de moindres carr\'es sont 
de loin les plus fr\'equemment rencontr\'es par les praticiens. Il me 
semble donc utile d'inclure ici une description de cette m\'ethode, 
accompagn\'ee d'explications conformes \`a l'esprit de cet ouvrage. 
\medskip 
\'Etant donn\'e que la fonction $S(\alpha )$ s'exprime analytiquement 
\`a l'aide de fonctions \'el\'ementaires (la fonction $f(\alpha , x)$ ne 
peut offrir un bon mod\`ele de r\'egression que si elle est
\'el\'ementaire), elle est facile \`a d\'eriver. On a 
$$S_\ell \;\; = \;\; {\eightup{\partial S} \over \partial\alpha_\ell} \;\; 
= \;\; \sum_{i=1}^{n} 2\, p_i \, \big[ f(\alpha , x_i) - y_i \big]\, 
{\partial f \over \partial\alpha_\ell} (\alpha , x_i) \eqno (12\, A.2)$$ 
Le vecteur $\{ S_\ell \}_{\ell = 1, 2, \ldots Q}$ est le gradient 
d\'ej\`a mentionn\'e. Les d\'eriv\'ees secondes seront 
$$\eqalignno{ 
S_{k \ell}\;\;  &= \;\; {\partial S_\ell \over \partial\alpha_k} \;\; 
= \;\; {\partial S_k \over \partial\alpha_\ell} \;\; = &(12\, A.3) \cr 
&= \;\; \sum_{i=1}^{n} 2\, p_i \, \Bigg\{ \big[ f(\alpha , x_i) - y_i
\big]\, {\partial^2 f \over \partial\alpha_\ell \partial\alpha_k}
(\alpha , x_i) + {\partial f \over \partial\alpha_k} (\alpha , x_i) \cdot 
{\partial f \over \partial\alpha_\ell} (\alpha , x_i) \Bigg\} \cr }$$ 
Ces expressions se pr\`etent parfaitement bien \`a la programmation 
et par cons\'equent l'algorithme it\'eratif qui conviendra le mieux 
\`a la r\'esolution du syst\`eme d'\'equations $\grad S = 0$ est la 
m\'ethode de Newton ou m\'ethode de la tangente. Elle consiste \`a 
choisir des valeurs initiales $\alpha_1^{(0)}, \alpha_2^{(0)}, \ldots 
\alpha_Q^{(0)}$ ---~qu'on notera collectivement $\alpha^{(0)}$~--- 
\`a partir desquelles on calcule $\alpha^{(1)}$ par 
$$\alpha^{(1)} \; = \; \alpha^{(0)} - \, {\goth S}_2^{-1} \circ {\goth S}_1 
\eqno (12\, A.4)$$ 
o\`u ${\goth S}_1 = \grad S (\alpha^{(0)})$ et ${\goth S}_2$ est 
la matrice des $S_{k \ell}(\alpha^{(0)})$. Apr\`es quoi on
recommence en rempla\c{c}ant $\alpha^{(0)}$ par $\alpha^{(1)}$, puis 
$\alpha^{(1)}$ par $\alpha^{(2)}$, et ainsi de suite. Rappelons que 
cette m\'ethode de Newton s'interpr\`ete g\'eom\'etriquement de la
mani\`ere suivante :  l'\'equation 
$$\beta \; = \; \grad S (\alpha_1, \alpha_2, \ldots \alpha_Q)
\eqno (12\, A.5)$$
repr\'esente une hypersurface de dimension $Q$ dans l'espace $\R^{2Q}$
des coordonn\'ees $\alpha_1$, $\alpha_2$,  $\ldots$ $\alpha_Q$, 
$\beta_1$, $\beta_2$,  $\ldots$ $\beta_Q$; la solution du syst\`eme 
d'\'equations $\grad S = 0$ (ce qu'on cherche) correspond \`a 
l'intersection de cette hypersurface avec le sous-espace (de dimension $Q$) 
$\beta = 0$. La m\'ethode consiste alors \`a se rapprocher de ce point
en suivant l'hyperplan tangent \`a l'hypersurface;  celui-ci en effet
coupe le sous-espace $\beta = 0$ en un seul point.  Ainsi $\alpha^{(1)}$
est ce point d'intersection; puis $\alpha^{(2)}$ sera le point d'intersection
de l'hyperplan tangent en $\alpha^{(1)}$ avec le sous-espace $\beta = 0$, 
et ainsi de suite. 
\medskip 
La m\'ethode de Newton est recommand\'ee lorsque la d\'eriv\'ee de
la fonction (ici $\grad S$) est facile \`a programmer, ce qui est 
justement le cas. Elle converge extr\^emement vite, \`a condition de 
prendre $\alpha^{(0)}$ d\'ej\`a pr\`es du point qu'on veut atteindre: 
l'erreur est alors \`a chaque it\'eration de l'ordre du carr\'e de l'erreur 
pr\'ec\'edente, ce qui signifie que le nombre de d\'ecimales exactes
double \`a chaque it\'eration. Mais si le point de d\'epart $\alpha^{(0)}$ 
est \'eloign\'e du point qu'on veut atteindre, la m\'ethode ne converge 
plus du tout ou converge trop lentement pour \^etre efficace. Or dans
le probl\`eme des moindres carr\'es on ne peut pas deviner a priori un 
bon point initial. 
\medskip 
L'id\'ee essentielle de la m\'ethode de Marquardt est de proposer un
moyen simple et algorithmiquement \'economique pour, dans un
premier temps, se rapprocher du minimum (appelons cela la {\it phase 
de descente rapide}); apr\`es quoi, dans un second temps, on peut 
amorcer avec profit l'it\'eration de Newton (qu'on appellera la {\it 
phase critique}). 
\medskip 
Le principe est le suivant (la mise en \oe uvre pratique exigera 
quelques am\'enagements semi-empiriques qu'on d\'ecrira apr\`es). Il
s'agit, partant d'un point fix\'e arbitrairement, de se rapprocher aussi 
rapidement que possible du minimum (suffisamment pour que 
l'it\'eration de Newton devienne efficace). Pour cela il est logique de 
suivre la ligne de plus grande pente sur la surface d'\'equation 
$$\gamma \; = \; S(\alpha_1, \alpha_2, \ldots \alpha_Q) \eqno (12\, A.6)$$ 
{\eightpoint {\bf N. B. } L'\'equation $12\, A.5$ consid\'er\'ee avant
est vectorielle:  c'est un {\it syst\`eme} de $Q$ \'equations, il lui 
correspond donc une hypersurface de dimension $Q$ dans l'espace $\R^{2Q}$. 
Par contre l'\'equation $12\, A.6$ est scalaire, il lui correspond aussi
une hypersurface de dimension $Q$,  mais dans l'espace $\R^{Q+1}$ des 
param\`etres $\alpha_1$, $\alpha_2$, $\ldots$ $\alpha_Q$, $\gamma$. \par }
\medskip 
Or la ligne de plus grande pente sur une surface d'\'equation
$12\, A.6$ (projet\'ee sur le sous-espace $\R^Q$ des coordonn\'ees 
$\alpha_1$,  $\alpha_2$,  $\ldots$ $\alpha_Q$) a pour vecteur tangent 
le vecteur $S_1, S_2, \ldots, S_Q$. Ce vecteur est centrifuge par 
rapport au minimum. Donc pour se rapprocher du minimum \`a partir 
d'un point $\alpha^{(0)}$ initial situ\'e sur le plan des $\alpha$, il
suffit de prendre la direction $-\grad S(\alpha^{(0)})$. Ainsi on
choisira le second point $\alpha^{(1)} = \alpha^{(0)} - \lambda \grad 
S(\alpha^{(0)})$ (avec \'evidemment $\lambda > 0$).  Mais il faut 
choisir $\lambda$ de mani\`ere optimale. Si on prend un $\lambda$
trop petit il faudra un nombre trop grand d'it\'erations pour arriver 
dans le voisinage du minimum; si on le prend trop grand,  on risque de
d\'epasser ce minimum. C'est alors pour d\'eterminer le meilleur
choix de ce param\`etre $\lambda$ qu'interviennent plusieurs
am\'enagements empiriques que nous d\'ecrivons maintenant. 
\medskip 
D'abord les maxima ou minima locaux sont fr\'equents, surtout si 
la dimension du mod\`ele, c'est-\`a-dire le nombre de param\`etres
$\alpha$, est grande. Marquardt a donc propos\'e de placer dans 
le programme le test simple suivant: on commence avec un 
$\lambda$ petit, par exemple $1/100$ et on calcule $\alpha^{(1)} = 
\alpha^{(0)} - \lambda \grad S(\alpha^{(0)})$; si $S(\alpha^{(1)}) < 
S(\alpha^{(0)})$, cela indique qu'on descend la pente; on recommence 
alors avec un $\lambda$ beaucoup plus grand (ainsi on \'evite des 
maxima locaux). 
\medskip 
Par ailleurs il n'est pas indispensable de suivre {\it exactement} la
ligne de plus grande pente; en pratique il suffit de ne pas trop s'en 
\'ecarter, et sur ce point une approximation m\^eme grossi\`ere est 
fortement recommand\'ee si elle \'economise du temps de calcul. 
C'est pourquoi Marquardt propose de suivre la direction du vecteur
$R$ de composantes $R_{\ell} = S_{\ell} / S_{\ell , \ell}$. 
La bonne raison pour cela est que de cette mani\`ere le param\`etre 
$\lambda$ sera sans dimension. Ce point n'est pas indiff\'erent car 
le test doit choisir les valeurs petites ou grandes de $\lambda$ 
sans conna{\^\i}tre les ordres de grandeurs sur le terrain. En 
introduisant le vecteur $R$ on diminue les risques entra{\^\i}n\'es par 
un mauvais choix. Toutefois, d'apr\`es la d\'efinition $12\, A.3$ des 
$S_{k , \ell}$, les \'el\'ements diagonaux $S_{\ell , \ell}$ ne sont pas
forc\'ement positifs et peuvent parfois s'annuler ou devenir trop
petits; c'est pourquoi Marquardt a propos\'e aussi de remplacer la 
d\'efinition $12\, A.3$ par 
$$S_{k \ell} = \sum_{i=1}^{n} 2\, p_i \cdot {\partial f \over \partial 
\alpha_k} (\alpha , x_i) \cdot {\partial f \over \partial\alpha_\ell} 
(\alpha , x_i) \eqno (12\, A.7)$$ 
qui garantit que les termes diagonaux seront positifs: ils ne peuvent 
devenir nuls que si $\grad_\alpha f$ peut lui-m\^eme devenir nul, ce 
qu'on \'evitera par le choix du mod\`ele. Cela implique que la m\'ethode 
de Newton est quelque peu modifi\'ee. Mais les termes n\'eglig\'es (les 
d\'eriv\'ees secondes de $f$ par rapport aux param\`etres $\alpha$) 
sont multipli\'es par les facteurs $f(\alpha , x_i) - y_i$. Or de deux 
choses l'une: 
\smallskip 
--- ou bien ces facteurs ne deviennent pas tous petits lorsqu'on se 
rapproche du minimum; alors les termes n\'eglig\'es ne sont pas 
n\'egligeables et le calcul sera faux; 
\smallskip 
--- ou bien ces facteurs deviennent tous petits lorsqu'on se 
rapproche du minimum, et alors le calcul sera correct. 
\medskip 
Mais si les facteurs $f(\alpha , x_i) - y_i$ ne deviennent pas tous
petits lorsqu'on se rapproche du minimum, c'est que le minimum de 
$S(\alpha )$ n'est pas petit et que donc le mod\`ele est mauvais: il ne 
permet pas d'approcher les points $x_i,y_i$. La m\'ethode des moindres 
carr\'es n'ayant de toute fa\c{c}on aucun sens dans ce cas, on voit que 
la proposition de Marquardt est justifi\'ee. 
\medskip 
Il existe encore une autre raison, au moins aussi importante que la 
positivit\'e des $S_{\ell ,\ell}$, de remplacer $12\, A.3$ par $12\, 
 A.7$. Comme nous venons de le voir, ce changement ne s'\'ecarte 
gu\`ere de la m\'ethode de Newton si le minimum de $S(\alpha )$ est 
petit; par contre si les facteurs $f(x_i) - y_i$ restent appr\'eciables, 
l'it\'eration ne converge plus vers le minimum puisqu'on ne suit plus
la tangente. {\it Or cela est m\^eme un avantage}: en effet, l'un des 
d\'efauts bien connus de la m\'ethode de Newton est qu'elle converge 
trop facilement vers des minima locaux (il suffit de l'amorcer avec 
une valeur initiale $\alpha_0$ situ\'ee dans le bassin d'attraction d'un 
tel minimum local pour qu'elle converge vers celui-ci et non vers celui 
qu'on veut); mais si on applique $12\, A.7$ au lieu de $12\, A.3$, la 
m\'ethode ne convergera justement {\it que} si on est pr\`es du vrai 
minimum (le minimum absolu). Si on se trouve par hasard au voisinage 
d'un minimum local o\`u $S(\alpha )$ n'est pas petit, l'it\'eration ne 
conduira pas \`a s'en rapprocher; au contraire, au bout de quelques 
it\'erations on sortira de son bassin d'attraction et on recommencera 
le processus ailleurs. Ainsi seuls les minima effectivement petits 
feront converger le proc\'ed\'e. En optant pour $12\, A.7$ au lieu de 
$12\, A.3$,  Marquardt fait donc d'une pierre deux coups. 
\medskip
Une troisi\`eme astuce de Marquardt est encore la suivante. 
La tactique g\'en\'erale est, comme on l'a vu plus haut, de suivre la
pente (ou plut\^ot le vecteur $R$) lorsqu'on est loin du minimum, puis
de passer \`a la m\'ethode de la tangente ---~modifi\'ee comme on
vient de voir~--- lorsqu'on est suffisamment pr\`es. Marquardt 
introduit pour cela la matrice $D$ dont les \'el\'ements sont 
$$D_{j,k}(\alpha ) = 
\cases{ S_{j,j}(\alpha ) \cdot (1+\lambda ) & si $j = k$ \cr 
\noalign{\medskip} 
S_{j,k}(\alpha ) & si $j \neq k$ \cr } \eqno (12\, A.8)$$ 
o\`u $S_{j,k}(\alpha )$ est d\'efini par $12\, A.7$, et propose 
d'appliquer la formule $12\, A.4$ avec la matrice $D$ ainsi 
d\'efinie au lieu de la matrice ${\goth S}$: 
$$\alpha^{(1)} = \alpha^{(0)} - D(\alpha^{(0)})^{-1} \circ \grad S 
(\alpha^{(0)})\eqno (12\, A.9)$$ 
 
Cela se comprend ainsi: lorsque $\lambda$ est petit, la matrice 
$D_{j,k}(\alpha^{(0)})$ est pratiquement identique \`a la matrice 
$S_{j,k} (\alpha^{(0)})$, donc avec $12\, A.9$ on applique en fait la
m\'ethode de Newton; si au contraire $\lambda$ est grand, la matrice
$D_{j,k}(\alpha^{(0)})$ est pratiquement identique \`a la matrice
diagonale d'\'el\'ements $S_{j,j}(\alpha^{(0)})$, par cons\'equent
$D^{-1} \circ \grad S$ est pratiquement identique au vecteur $R$ 
introduit ci-dessus. Ainsi on dispose d'une grande souplesse 
algorithmique pour passer progressivement de la phase de descente
rapide \`a la phase critique: il suffit de jouer sur les valeurs du 
param\`etre $\lambda$. 
\medskip 
Tous ces am\'enagements sont des recettes empiriques: choisir
$\lambda$ grand ou petit selon des crit\`eres qui ne sont pas
absolument fiables, suivre le vecteur $R$ plut\^ot que la ligne de plus 
grande pente, remplacer $12\, A.3$ par $12\, A.7$, tout cela ne peut 
pas \^etre justifi\'e par des d\'emonstrations rigoureuses et formelles 
et c'est uniquement l'usage pratique qui a tranch\'e. En effet, 
cette m\'ethode de Marquardt a aujourd'hui convaincu tous les 
praticiens et elle est impl\'ement\'ee dans les logiciels de calcul 
(Matlab, Statistica, etc). Mais il est clair qu'elle ne marche pas \`a 
coup s\^ur. Son succ\`es est d\^u \`a ce qu'on n'a pas trouv\'e mieux. 
\medskip 
Parmi les causes de ``plantage'' possibles, la plus courante est 
de tourner en rond dans une zone de minima d\'eg\'en\'er\'es; le 
test qui sert \`a d\'eterminer $\lambda$ saute alors \'eternellement 
entre un petit $\lambda$ et un grand. Un programme bien con\c{c}u doit 
donc pr\'evoir un test d'arr\^et avec message d'erreur pour sortir 
d'une telle boucle sans fin; par exemple si $\lambda$ passe plus de 
dix fois d'une grande valeur \`a une petite et vice-versa, arr\^eter 
le processus et afficher ``Sorry, but it seems I am in a wrong track''. 
La meilleure solution est que le programme pr\'evoie alors une sortie 
de secours avec entr\'ee manuelle des valeurs de $\lambda$. 
\medskip 
On peut donc r\'esumer la m\'ethode comme suit. 
\smallskip 
{\bf a)} Cr\'eer des routines qui calculent le vecteur $\grad S (\alpha )$ 
donn\'e par $12\, A.2$ et la matrice $D$ donn\'ee par $12\, A.8$ en 
fonction de $\alpha$ et $\lambda$, ainsi que son inverse $D^{-1}$ par 
la m\'ethode du pivot. 
\smallskip 
{\bf b)} Choisir une valeur initiale $\alpha^{(0)}$ (en g\'en\'eral 
compl\`etement arbitraire car on n'a aucun crit\`ere de choix), calculer 
$S(\alpha^{(0)})$, et commencer l'it\'eration de $12\, A.9$ avec un 
$\lambda$ moyen mais plut\^ot petit (de l'ordre de $0.01$ ou $0.001$), 
ce qui va donner $S(\alpha^{(1)})$. Le fait de choisir ce premier 
$\lambda$ plut\^ot petit revient \`a appliquer la m\'ethode de Newton 
modifi\'ee par $12\, A.7$. 
\smallskip 
{\bf c)} Calculer $S(\alpha^{(1)})$ et le comparer \`a $S(\alpha^{(0)})$: 
\smallskip 
\leftskip=12pt 
--- {\bf c1)} si $S(\alpha^{(1)}) \geq S(\alpha^{(0)})$ (ce qui veut dire 
qu'on n'\'etait certainement pas au voisinage du bon minimum: on 
devait \^etre au voisinage d'une instabilit\'e, ou d'un minimum local 
non petit dont on s'est \'ecart\'e du fait qu'on n'applique pas $12\, 
A.3$, mais $12\, A.7$), prendre un $\lambda$ beaucoup plus grand, 
par exemple $\lambda = 1$ ou $\lambda = 0.1$ (ce qui veut dire qu'on 
suit maintenant plut\^ot le vecteur $R$ afin de s'\'ecarter nettement
de ce point qui n'\'etait visiblement pas bon) et recommencer {\bf b}; 
\smallskip 
--- {\bf c2)} si $S(\alpha^{(1)}) < S(\alpha^{(0)})$ (on peut alors penser
qu'on est sur la bonne voie vers le minimum) prendre un $\lambda$ 
encore plus petit (disons dix fois, ce qui veut dire qu'on applique 
maintenant la m\'ethode de Newton modifi\'ee comme on a vu) et 
recommencer aussi longtemps que $S(\alpha^{(n)})$ diminue, ce qui va 
donner $\alpha^{(2)}$, $\alpha^{(3)}$, \dots. Continuer l'it\'eration tant 
que $\Delta S = S(\alpha^{(n-1)}) - S(\alpha^{(n)})$ est positif et 
appr\'eciable; revenir \`a {\bf c1} d\`es que $\Delta S$ devient $< 0$; 
stopper lorsque $\Delta S$ devient petit en restant $> 0$, on est alors
arriv\'e \`a destination (la petitesse ici est aussi question de flair: 
en g\'en\'eral on convient que l'it\'eration peut s'arr\^eter quand 
$\Delta S$ devient inf\'erieur \`a $0.01$; en r\'ealit\'e, cela d\'epend 
fortement du mod\`ele et de l'\'echantillon). 
\medskip 
\leftskip=0pt 
{\bf N. B.} Au stade {\bf c2} rien ne prouve encore d\'efinitivement 
qu'on est sur la bonne voie: si on l'est, on s'\'ecartera peu de
l'hyperplan tangent \`a l'hypersurface d'\'equation $12\, A.5$ et
---~si le minimum n'est pas d\'eg\'en\'er\'e~--- l'it\'eration convergera; 
sinon, il arrivera t\^ot ou tard que $S(\alpha^{(n)})$ augmente \`a nouveau, 
puisqu'on ne suit pas l'hyperplan tangent. 
\medskip 
Les logiciels du commerce sont contraints de choisir forfaitairement 
les crit\`eres tels que le choix de $\lambda$ ou la petitesse de $\Delta 
S$. Mais les meilleurs sont programmables (par exemple Matlab) et 
permettent un ajustement par l'utilisateur.  Cette possibilit\'e est 
essentielle, car ces choix ne peuvent vraiment \^etre rendus optimaux 
que pour un mod\`ele donn\'e,  d'apr\`es les exigences de pr\'ecision et 
de temps de calcul impos\'es par les conditions concr\`etes du 
probl\`eme. Le meilleur ajustement est toujours empirique, mais 
l'utilisateur ne peut intervenir \`a bon escient que s'il a compris les 
principes th\'eoriques expos\'es ci-dessus. 
\medskip 
Il est int\'eressant de comparer les calculs effectu\'es selon la 
m\'ethode de Marquardt, lorsque le mod\`ele $y = f(\alpha , x)$ ne 
d\'epend pas lin\'eairement des param\`etres $\alpha$, avec les 
calculs \`a effectuer lorsque le mod\`ele d\'epend lin\'eairement des 
param\`etres. Dans ce dernier cas, il y a seulement \`a r\'esoudre un 
syst\`eme lin\'eaire de $Q$ \'equations \`a $Q$ inconnues, donc on 
applique une seule fois la m\'ethode du pivot.  Dans la m\'ethode de 
Marquardt, la m\'ethode du pivot est r\'eappliqu\'ee \`a chaque 
it\'eration (cf {\bf a}). Cette partie du programme repr\'esente 
\'evidemment la quasi totalit\'e du temps de calcul, sauf si la 
dimension $Q$ est petite. Ainsi, si $N$ it\'erations en tout sont 
n\'ecessaires pour parvenir au r\'esultat,  le mod\`ele sera $N$ fois 
plus dispendieux qu'un mod\`ele \`a d\'ependance lin\'eaire. Cet 
argument doit \^etre pris en compte lorsqu'on d\'ecide de choisir un 
mod\`ele de grande dimension. En revanche, un mod\`ele \`a 
d\'ependance non lin\'eaire de dimension $Q=2$ ou $Q=3$, est 
certainement pr\'ef\'erable \`a un mod\`ele \`a d\'ependance
lin\'eaire de dimension \'elev\'ee. Par exemple un mod\`ele du type 
$f(\alpha_1, \alpha_1, x) = x\alpha_1\e^{-\alpha_2 x}$ sera bien 
meilleur (s'il convient \`a l'\'echantillon) qu'un polyn\^ome de degr\'e 
$10$. 

\vskip20pt

{\eightpoint On trouvera dans les pages suivantes un programme qui 
ex\'ecute l'algorithme. Le langage {\eightrm PASCAL} a \'et\'e choisi 
pour sa facilit\'e de lecture, une traduction en $C$ peut se faire 
rapidement. \par} 

\vfill\break 

\hsize=160mm
\larg=160mm

\font\ko=cmr10
\def\iA{\hskip4.8mm}
\def\iB{\hskip9.6mm} 
\def\iC{\hskip14.4mm} 
 
\null\vskip10mm
{\centerline{\bf Le programme que voici ex\'ecute la m\'ethode de 
Marquardt.} \vskip10mm 
{\obeylines 
{\tt 
{\bf program Marquardt}; 
\medskip 
{\bf label}   
   \iA  99; 
\medskip 
 {\bf const} 
   \iA N = 12; \hskip4mm {(\ko nombre de points)} 
   \iA stop = 10; \hskip4mm {(\ko pour limiter le nombre d'it\'erations.)} 
  \iA  eps = 1e-5;  \hskip4mm {(\ko pr\'ecision du calcul.)} 
\medskip 
 {\bf var} 
   \iA p, q, u, v, u0, v0, u1, v1, w, z, max, scale : double; 
   \iA det, lambda, delta, S0, S1 : double; 
   \iA loi, xx, yy : array[1..N] of double; 
   \iA i, j, k : integer; 
 
\medskip 
\filbreak 
\vskip-12pt 
\line{\hfill \smash{\lower12mm\hbox{\vbox{\hsize=40mm  
\rightskip=0mm plus4mm minus4mm 
\eightpoint Cette fonction d\'efinit le mod\`ele de r\'egression choisi.  }}}} 
 {\bf function} phi (x, y, p : double) : double;      
 \smallskip 
  \iA {\bf begin}                                               
  \iA  phi := exp(x * ln(p) - y * p); 
  \iA {\bf  end}; 
 
\medskip 
\filbreak 
\vskip-12pt 
\line{\hfill \smash{\lower12mm\hbox{\vbox{\hsize=40mm  
\rightskip=0mm plus4mm minus4mm 
\eightpoint D\'eriv\'ee par rapport \`a $x$ de la fonction {\tt phi}.  }}}} 
 {\bf function} phix (x, y, p : double) : double; 
\smallskip 
  \iA {\bf begin} 
  \iA phix := ln(p) * exp(x * ln(p) - y * p); 
  \iA {\bf end}; 
 
\medskip 
\filbreak 
\vskip-12pt 
\line{\hfill \smash{\lower12mm\hbox{\vbox{\hsize=40mm  
\rightskip=0mm plus4mm minus4mm 
\eightpoint D\'eriv\'ee par rapport \`a $y$ de la fonction {\tt phi}.  }}}} 
{\bf  function} phiy (x, y, p : double) : double; 
\smallskip 
  \iA {\bf begin} 
  \iA phiy := -p * exp(x * ln(p) - y * p); 
  \iA {\bf end}; 
 
\vskip5mm 
\line{\hfill \smash{\lower12mm\hbox{\vbox{\hsize=90mm  
\rightskip=0mm plus4mm minus4mm 
\eightpoint  
Lorsqu'on voudra changer le mod\`ele de r\'egression sans changer le % 
nombre de param\`etres, il suffira de remplacer ces trois fonctions % 
sans toucher \`a la suite du programme. Mais si on veut passer % 
\`a trois param\`etres ou plus, il faudra modifier aussi les fonctions % 
SQ, Sx, etc.  }}}} 
 
\medskip 
\filbreak 
\vskip-12pt 
\line{\hfill \smash{\lower12mm\hbox{\vbox{\hsize=40mm  
\rightskip=0mm plus4mm minus4mm 
\eightpoint {\tt SQ} est la somme des carr\'es des \'ecarts.  }}}} 
 {\bf function} SQ (x, y : double) : double; 
\smallskip 
  \iA {\bf var} 
    \iB s : double; 
    \iB l : integer; 
\smallskip 
  \iA {\bf begin} 
  \iA s := 0; 
  \iA {\bf for} l := 1 {\bf to} N {\bf do} 
    \iB {\bf begin} 
    \iB  s := s + loi[l] * sqr(phi(x, y, xx[l]) - yy[l]); 
    \iB {\bf end}; 
  \iA SQ := s; 
  \iA {\bf end}; 
 
\medskip 
\filbreak 
\vskip-12pt 
\line{\hfill \smash{\lower12mm\hbox{\vbox{\hsize=40mm  
\rightskip=0mm plus4mm minus4mm 
\eightpoint Calcul de $\displaystyle {\partial {\tt SQ} \over \partial x}$.  
}}}} 
 {\bf function} Sx (x, y : double) : double; 
\smallskip 
   \iA {\bf var} 
      \iB s : double; 
      \iB l : integer; 
\smallskip 
   \iA {\bf begin} 
   \iA s := 0; 
   \iA {\bf for} l := 1 {\bf to} N {\bf do} 
      \iB {\bf begin} 
      \iB s := s + loi[l] * (phi(x, y, xx[l]) - yy[l]) * phix(x, y, xx[l]); 
      \iB {\bf end}; 
   \iA Sx := s; 
   \iA{\bf end}; 
 
\medskip 
\filbreak 
\vskip-12pt 
\line{\hfill \smash{\lower12mm\hbox{\vbox{\hsize=40mm  
\rightskip=0mm plus4mm minus4mm  \eightpoint  
Calcul de $\displaystyle {\partial {\tt SQ} \over \partial y}$. }}}}  
{\bf  function} Sy (x, y : double) : double; 
\smallskip 
   \iA var 
   \iB  s : double; 
   \iB  l : integer; 
\smallskip 
  \iA {\bf begin} 
   \iA s := 0; 
   \iA {\bf for} l := 1 {\bf to} N {\bf do} 
    \iB {\bf begin} 
     \iB s := s + loi[l] * (phi(x, y, xx[l]) - yy[l]) * phiy(x, y, xx[l]); 
    \iB {\bf end}; 
   \iA Sy := s; 
  \iA {\bf end}; 
 
\medskip 
\filbreak 
\vskip-12pt 
\line{\hfill \smash{\lower12mm\hbox{\vbox{\hsize=40mm  
\rightskip=0mm plus4mm minus4mm \eightpoint  
Calcul de $\displaystyle {\partial^2 {\tt SQ} \over \partial x^2}$.  }}}} 
 {\bf function} Sxx (x, y : double) : double; 
\smallskip 
   \iA {\bf var} 
    \iB s : double; 
    \iB l : integer; 
\smallskip 
  \iA {\bf begin} 
   \iA s := 0; 
   \iA {\bf for} l := 1 {\bf to} N {\bf do} 
    \iB {\bf begin} 
     \iB s := s + loi[l] * phix(x, y, xx[l]) * phix(x, y, xx[l]); 
    \iB {\bf end}; 
   \iA Sxx := s * (1 + lambda); 
  \iA {\bf end}; 
 
\medskip 
\filbreak 
\vskip-12pt 
\line{\hfill \smash{\lower12mm\hbox{\vbox{\hsize=40mm  
\rightskip=0mm plus4mm minus4mm \eightpoint  
Calcul de $\displaystyle {\partial^2{\tt SQ} \over \partial x \, \partial y}$. 
}}}} 
 {\bf function} Sxy (x, y : double) : double; 
\smallskip 
   \iA {\bf var} 
   \iB  s : double; 
   \iB  l : integer; 
\smallskip 
   \iA  {\bf begin} 
   \iA s := 0; 
   \iA {\bf for} l := 1 {\bf to} N {\bf do} 
    \iB {\bf begin} 
    \iB  s := s + loi[l] * phix(x, y, xx[l]) * phiy(x, y, xx[l]); 
    \iB {\bf end}; 
   \iA Sxy := s; 
  \iA {\bf end}; 
 
\medskip 
\filbreak 
\vskip-12pt 
\line{\hfill \smash{\lower12mm\hbox{\vbox{\hsize=40mm  
\rightskip=0mm plus4mm minus4mm \eightpoint  
Calcul de $\displaystyle {\partial^2 {\tt SQ} \over \partial y^2}$.  }}}} 
 {\bf function} Syy (x, y : double) : double; 
\smallskip 
  \iA {\bf var} 
     \iB s : double; 
     \iB l : integer; 
\smallskip 
  \iA {\bf begin} 
   \iA s := 0; 
   \iA {\bf for} l := 1 {\bf to} N {\bf do} 
     \iB {\bf begin} 
     \iB s := s + loi[l] * phiy(x, y, xx[l]) * phiy(x, y, xx[l]); 
     \iB {\bf end}; 
  \iA  Syy := s * (1 + lambda); 
  \iA {\bf end}; 
 
\bigskip 
 
\filbreak 
\vskip-12pt 
\line{\hfill \smash{\lower12pt\hbox{\vbox{\hsize=70mm  
\rightskip=0mm plus4mm minus4mm 
\eightpoint D\'ebut du programme principal.  }}}} 
{\bf begin} 
\medskip 
 z := 0; 
\vskip-12pt 
\line{\hfill \smash{\lower19mm\hbox{\vbox{\hsize=85mm  
\rightskip=0mm plus4mm minus4mm 
\eightpoint Mise en m\'emoire du nuage de points. {\tt xx[i]} est % 
l'abcisse du point $N^\circ$ {\tt i}, {\tt yy[i]} son ordonn\'ee. % 
Pour pouvoir tester plus facilement le programme,  % 
on a choisi ici un nuage de points situ\'e exactement sur la courbe % 
$y = x^u\,\e^{-vx}$, avec $u=1.75$ et $v=\sqrt{0.3}\simeq 0.5477$,  % 
de sorte que le mod\`ele de r\'egression sera exact.   }}}} 
{\bf  for} i := 1 {\bf to} N {\bf do} 
 \iA   {\bf begin} 
 \iA   w := z; 
 \iA   xx[i] := i; 
 \iA   q := sqrt(xx[i]); 
 \iA   z := xx[i] * q * sqrt(q) * exp(-sqrt(0.3) * xx[i]); 
 \iA   yy[i] := z; 
\vskip-12pt 
\line{\hfill \smash{\lower12mm\hbox{\vbox{\hsize=70mm  
\rightskip=0mm plus4mm minus4mm 
\eightpoint Recherche du maximum de {\tt yy[i]} afin de trouver la % 
bonne \'echelle graphique ({\tt scale}).  }}}} 
 \iA   {\bf if} z > w {\bf then} 
   \iB   {\bf begin} 
   \iB    max := z; 
   \iB   {\bf end}; 
 \iA  {\bf end}; 
 scale := 220 / max; 
 
\medskip 
\filbreak 
\vskip-12pt 
\line{\hfill \smash{\lower12mm\hbox{\vbox{\hsize=50mm  
\rightskip=0mm plus4mm minus4mm 
\eightpoint {\tt loi[i]} est le poids du {\tt i}-i\`eme point.  
Ici il est uniforme.  }}}} 
 {\bf for} i := 1 {\bf to} N {\bf do} 
   \iA {\bf begin} 
   \iA  loi[i] := 1; 
   \iA  {\bf end}; 
 
 \medskip 
\filbreak 
\vskip-12pt 
\line{\hfill \smash{\lower12mm\hbox{\vbox{\hsize=50mm  
\rightskip=0mm plus4mm minus4mm 
\eightpoint Repr\'esentation graphique du nuage de points.  }}}} 
 {\bf MoveTo}(10, 242); 
  {\bf LineTo}(10, 229); 
  {\bf MoveTo}(9, 240); 
  {\bf LineTo}(490, 240); 
  {\bf for} i := 1 {\bf to} N {\bf do} 
   \iA {\bf begin} 
   \iA k := {\bf round}(scale * yy[i]); 
   \iA {\bf MoveTo}(10 + 40 * i, 241 - k); 
   \iA {\bf LineTo}(10 + 40 * i, 239 - k); 
   \iA {\bf MoveTo}(9 + 40 * i, 240 - k); 
   \iA {\bf LineTo}(11 + 40 * i, 240 - k); 
   \iA {\bf end}; 
 
 \medskip 
 \filbreak 
\vskip-12pt 
\line{\hfill \smash{\lower24mm\hbox{\vbox{\hsize=72mm  
\rightskip=0mm plus4mm minus4mm 
\eightpoint Initialisation manuelle (au clavier) de l'it\'eration.  
Cela laisse l'initiative \`a l'utilisateur. }}}} 
  {\bf writeln}; 
  {\bf  writeln}('   enter the initial parameters:'); 
  {\bf write}('                        u = '); 
  {\bf readln}(u); 
  {\bf write}('                        v = '); 
  {\bf readln}(v); 
  {\bf writeln}; 
 
  \medskip 
  \filbreak 
 
  99 :   \hfill {\eightpoint D\'ebut de l'it\'eration.  } \hfill  
\smallskip 
  lambda := 0.0\iB  delta := 1; 
  i := 0; 
\smallskip 
  {\bf while} ((delta < -eps) {\bf or} (delta > 0)) {\bf and} % 
(i <= stop) {\bf do}  
\smallskip 
    \iA {\bf begin} 
\smallskip 
    \iA {\bf MoveTo}(10, 240); 
\smallskip 
\vskip-12pt 
   \line{\hfill \smash{\lower11.3mm\hbox{\vbox{\hsize=85mm  
\rightskip=0mm plus4mm minus4mm 
\eightpoint On repr\'esente graphiquement l'\'evolution du mod\`ele % 
au cours de l'it\'eration. Si tout se passe bien, les courbes obtenues % 
doivent finir par se rapprocher de plus en plus du nuage de points.  }}}} 
 \iA {\bf for} j := 1 {\bf to} 120 {\bf do} 
      \iB {\bf begin} 
      \iB w := j / 10; 
      \iB z := scale * phi(u, v, w); 
      \iB {\bf if} z >= 500 {\bf then} \hfill % 
{\eightpoint Ici, simple pr\'ecaution pour \'eviter {\it overflow}.  } 
      \iB k := 500 
      \iB {\bf else if} z <= -500 {\bf then} 
      \iB k := -500 
      \iB {\bf else} 
      \iB k := {\bf round}(z); 
      \iB {\bf LineTo}(10 + 4 * j, 240 - k); 
      \iB {\bf end}; 
\smallskip 
  \filbreak 
\vskip-12pt 
 \line{\hfill \smash{\lower40mm\hbox{\vbox{\hsize=50mm  
\rightskip=0mm plus4mm minus4mm 
\eightpoint Ci-contre on reconna{\^\i}t l'algorithme de Marquardt.  }}}} 
  \iA i := i + 1; 
   \iA S0 := SQ(u, v); 
   \iA det := Sxx(u, v) * Syy(u, v) - Sxy(u, v) * Sxy(u, v); 
   \iA u1 := Sx(u, v) * Syy(u, v) - Sy(u, v) * Sxy(u, v); 
   \iA v1 := -Sx(u, v) * Sxy(u, v) + Sy(u, v) * Sxx(u, v); 
   \iA u0 := u - u1 / det; 
   \iA v0 := v - v1 / det; 
   \iA S1 := SQ(u0, v0); 
   \iA delta := S1 - S0; 
  \filbreak 
   \iA {\bf writeln}(i : 3, '   lambda = ', lambda : 11); 
   \iA {\bf writeln}('u = ', u0 : 16 : 14, ' v = ', v0 : 16 : 14); 
   \iA {\bf writeln}('delta = ', delta : 15); 
   \iA {\bf writeln}('S0 = ', S0 : 15); 
   \iA {\bf writeln}; 
\smallskip 
  \filbreak 
\vskip-12pt 
\line{\hfill \smash{\lower8.3mm\hbox{\vbox{\hsize=75mm  
\rightskip=0mm plus4mm minus4mm 
\eightpoint On redonne l'initiative \`a l'utilisateur s'il appara{\^\i}t % 
que l'it\'eration diverge trop.  }}}} 
   \iA {\bf if} (S0 >= 10000) {\bf then} 
\smallskip 
      \iB {\bf begin} 
\smallskip 
      \iB {\bf writeln}; 
      \iB {\bf writeln}('    I think your initial parameters are highly 
      \iB unrealistic. '); 
      \iB {\bf writeln}('    The process will probably not converge. '); 
      \iB {\bf writeln}('    Could you choose other values? '); 
      \iB {\bf writeln}; 
      \iB {\bf writeln}('   enter new initial parameters:'); 
      \iB {\bf write}('                        u = '); 
      \iB {\bf readln}(u); 
      \iB {\bf write}('                        v = '); 
      \iB {\bf readln}(v); 
      \iB {\bf writeln}; 
     \iB {\bf goto} 99; \hfill % 
{\eightpoint On revient au d\'ebut de l'it\'eration.  } 
      \iB {\bf end}; 
\smallskip 
   \filbreak 
\vskip-12pt 
 \line{\hfill \smash{\lower25mm\hbox{\vbox{\hsize=55mm  
\rightskip=0mm plus4mm minus4mm 
\eightpoint Ci-contre on reconna{\^\i}t la seconde partie % 
de l'algorithme de Marquardt.  }}}} 
  \iA {\bf if} (delta >= 0) {\bf then} 
\smallskip 
      \iB {\bf begin} 
      \iB lambda := 10 * lambda; 
      \iB {\bf end} 
\smallskip 
   \filbreak 
   \iA {\bf else} 
\smallskip 
      \iB {\bf begin} 
      \iB lambda := lambda / 10; 
      \iB u := u0; 
      \iB v := v0; 
      \iB {\bf end}; 
\smallskip 
   \iA {\bf end}; 
 
\medskip 
\filbreak 
 \vskip-12pt 
\line{\hfill \smash{\lower18mm\hbox{\vbox{\hsize=60mm  
\rightskip=0mm plus4mm minus4mm 
\eightpoint On trace un pointill\'e horizontal.  }}}} 
    {\bf writeln}; 
    {\bf write}('     '); 
    {\bf for} j := 1 {\bf to} 21 {\bf do} 
    {\bf write}('---'); 
    {\bf writeln}; 
    {\bf writeln}; 
\smallskip 
\filbreak 
    {\bf if} (delta >= -eps) {\bf and} (delta <= 0) {\bf then} 
\smallskip 
\vskip-12pt 
 \line{\hfill \smash{\lower12mm\hbox{\vbox{\hsize=40mm  
\rightskip=0mm plus4mm minus4mm 
\eightpoint \`A la fin de l'it\'eration, si la convergence est stabilis\'ee, %   
on affiche le r\'esultat final \dots }}}} 
      \iA {\bf begin} 
      \iA {\bf writeln}('    final result:          S1 = ', S1 : 15); 
      \iA {\bf writeln}('                            u =  ', u : 16 : 14); 
      \iA {\bf writeln}('                            v =  ', v : 16 : 14); 
 
\medskip 
\filbreak 
\vskip-12pt 
 \line{\hfill \smash{\lower12mm\hbox{\vbox{\hsize=60mm  
\rightskip=0mm plus4mm minus4mm 
\eightpoint \dots avec la courbe correspondante \dots}}}} 
     \iA {\bf MoveTo}(10, 240); 
 \smallskip 
     \iA {\bf for} j := 1 {\bf to} 120 {\bf do} 
 \smallskip 
        \iB {\bf begin} 
         \iB w := j / 10; 
         \iB z := phi(u, v, w); 
         \iB k := {\bf round}(scale * z); 
         \iB {\bf LineTo}(10 + 4 * j, 240 - k); 
         \iB {\bf end}; 
\smallskip 
      \iA {\bf end} 
\smallskip 
\filbreak 
\vskip-12pt 
\line{\hfill \smash{\lower8.3mm\hbox{\vbox{\hsize=107mm  
\rightskip=0mm plus4mm minus4mm 
\eightpoint \dots  et si la convergence n'est pas stabilis\'ee,  % 
on rend l'initiative \`a l'utilisateur. \dots }}}} 
    {\bf else} 
\smallskip 
      \iA {\bf begin} 
       \iA {\bf writeln}('  I have stopped computing because there is no'); 
       \iA {\bf writeln}('appreciable progress after ', stop + 1 : 1, '  
iterations . '); 
       \iA {\bf writeln}('  Try other values for u and v or change the model.'); 
       \iA {\bf end};
\smallskip
   {\bf end.}

\medskip
}
}%end of \tt
}%end of \obeylines


\bye 
