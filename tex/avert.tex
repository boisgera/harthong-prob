\input twelvea4.tex
\input epsf.tex

\pageno=-3 
\null 
\vskip10mm plus4mm minus3mm 
 
\centerline{\bf AVERTISSEMENT} 
\vskip8mm plus4mm minus3mm 
Ce livre est \'ecrit pour tous ceux qui, en suivant un enseignement 
scientifique, ont \'eprouv\'e le sentiment typique qu'il manquait quelque  
chose d'essentiel, qui aurait permis de {\it vraiment} comprendre. Il 
ne s'adresse donc qu'\`a ceux qui veulent r\'eellement comprendre. 
\medskip 
L'enseignement scientifique laisse les esprits curieux sur leur faim: 
Beaucoup de jeunes sont impressionn\'es en regardant des \'emissions  
ou  en  lisant des revues de vulgarisation scientifique qui font miroiter 
des myst\`eres  passionnants,  et choisissent une fili\`ere d'\'etudes 
scientifique. L'enseignement sup\'erieur scientifique a par contre pour 
but unique de transmettre des techniques. De ces techniques, la formation 
minimale indispensable pour faire un ing\'enieur, un chercheur, ou un 
enseignant, en comporte d\'ej\`a plus  qu'un jeune cerveau ne peut en 
supporter. Tout enseignant arrive \`a peine, dans le nombre d'heures de 
cours maximal qui lui est consenti compte tenu des possibilit\'es  
horaires et du temps de travail personnel consid\'erable que les 
\'etudiants devraient th\'eoriquement fournir pour leur assimilation, \`a 
placer la  totalit\'e des techniques qu'il est charg\'e de transmettre. 
Comment voulez-vous, dans ces conditions, prendre du temps pour 
expliquer la {\it signification} des concepts introduits, ou bien pour 
expliquer {\it comment} et {\it pourquoi} les techniques qu'on enseigne 
ont \'et\'e invent\'ees?    
\medskip 
Beaucoup de scientifiques, \'eminents ou non, se plaignent  de voir ainsi 
la {\it culture} scientifique peu \`a peu \'evinc\'ee par des formations 
purement techniques. On s'en plaignait d\'ej\`a au si\`ecle dernier. 
Peut-\^etre croyait-on alors qu'en agitant publiquement le spectre 
d'un retour progressif \`a l'obscurantisme, on parviendrait \`a 
susciter un sursaut collectif. Il n'en a rien \'et\'e. On a envoy\'e en 
{\oldstyle 1914} les jeunes polytechniciens, auxquels les meilleurs 
professeurs avaient enseign\'e cette culture, au front se faire tuer. 
Puis la masse des connaissances \`a transmettre a continu\'e \`a 
augmenter sans limite. Peut-\^etre croit-on aujourd'hui qu'en r\'ep\'etant 
publiquement les m\^emes craintes, dans les m\^emes termes, mais 
\`a la t\'el\'evision cette fois, on r\'eussira l\`a o\`u nos anc\^etres ont 
\'echou\'e?  
\medskip 
Je n'en crois \'evidemment rien. Il m'est apparu apr\`es r\'eflexion que la 
seule action susceptible d'atteindre une certaine efficacit\'e \'etait 
d'\'ecrire des livres comme celui-ci. Sans trop d'illusions, car contre  
les principaux obstacles, je n'ai aucun moyen d'action. Je ne peux  
agir qu'en affaiblissant des obstacles secondaires:  si je n'ai pas le 
pouvoir de donner plus de temps \`a ceux qui souhaitent approfondir et 
comprendre tout ce qui est rest\'e cach\'e derri\`ere les masses de 
connaissances qu'ils ont  d\^u assimiler \`a toute vitesse, je peux au 
moins les aider dans leur qu\^ete, raccourcissant par l\`a le temps qui 
leur serait n\'ecessaire pour  faire tout le chemin eux-m\^emes. Mais 
j'\'ecris pour les lecteurs ayant une {\it volont\'e pr\'ealable} 
d'approfondir et de comprendre; je ne cherche pas \`a la susciter chez 
ceux qui ne l'ont pas.    
\medskip 
Il faudra cependant que cette volont\'e soit assez forte, car je n'ai pas 
\'et\'e capable de fournir une mati\`ere pr\'edig\'er\'ee au point qu'un  
dimanche apr\`es-midi puisse suffire pour la comprendre. 
Ce  livre exigera donc malgr\'e tout un effort important de la 
part du lecteur.  
\medskip 
Jusqu'ici, je n'ai fait que d\'ecrire mon objectif. Cela montre d\'ej\`a  
que mon but n'\'etait pas d'\'ecrire un manuel scolaire; cela n'aurait eu 
aucun sens, puisqu'il en existe d\'ej\`a beaucoup. On peut certes  
estimer que les manuels existants sont mauvais, qu'ils expliquent le 
sujet d'une mani\`ere trop compliqu\'ee, trop abstraite, ou au contraire  
pas assez rigoureuse, etc., et nourrir l'ambition d'en \'ecrire un qui 
soit enfin parfait. Mais pour \'eviter tout malentendu, je me permets 
d'insister: ce n'est pas cette ambition qui m'a pouss\'e \`a r\'ealiser ce 
livre. Bien s\^ur, je me suis toujours efforc\'e de ne pas rendre les 
choses plus compliqu\'ees qu'elles  ne le sont naturellement et d'\^etre 
aussi p\'edagogique que possible (par exemple en ins\'erant de 
nombreuses figures). Mais mon but n'a pas  \'et\'e de faire un manuel 
``meilleur''; j'ai voulu faire {\it autre chose}  qu'un manuel. J'ai voulu 
expliquer ce qui est pass\'e sous silence dans les manuels. Toutefois j'ai 
construit ce livre autour d'un cours d'initiation  (au d\'epart purement 
technique, lui aussi) au Calcul des probabilit\'es;  il est donc possible  
\`a un bachelier scientifique d'apprendre les techniques du Calcul des 
probabilit\'es dans ce livre, bien que ce ne soit pas le but de l'ouvrage. 
Dans ce cas, il ne faut pas h\'esiter \`a laisser de c\^ot\'e les discussions 
qui \'emaillent l'expos\'e. Mais tout d\'epend du niveau technique d'o\`u 
part le lecteur. Je m'adresse aussi \`a ceux qui  ont d\'ej\`a suivi avec 
succ\`es des cours de Calcul des probabilit\'es, au  niveau de la licence 
ou de la ma\^\i trise ou m\^eme plus, et je veux leur raconter ce qu'ils n'y 
ont pas appris.    \medskip 
Je me suis efforc\'e d'\'ecrire pour plusieurs niveaux de  lecture tr\`es  
diff\'erents \`a la fois. Beaucoup d'exemples (souvent emprunt\'es \`a la 
Physique) peuvent \^etre  difficiles pour un d\'ebutant, ou pour un lecteur 
ayant  re\c cu une formation purement math\'ematique, mais ils ne sont 
pas  n\'ecessaires pour  comprendre le reste et peuvent donc \^etre omis. 
Inversement, le physicien aura  beaucoup de mal \`a suivre certaines 
d\'emonstrations math\'ematiques  d\'etaill\'ees et rigoureuses; le  
mieux pour lui sera de les sauter.  Les lecteurs qui connaissent d\'ej\`a  
le Calcul des probabilit\'es et qui cherchent ce  qu'on ne trouve pas dans 
les manuels, pourront, eux, laisser les passages qu'ils connaissent 
d\'ej\`a, et ne lire que les  discussions et les exemples.   
\medskip 
Quel que soit l'objectif vis\'e, quelle que soit  la formation scientifique 
ant\'erieure, il ne faut donc pas vouloir absolument tout lire. Une lecture 
lacunaire est possible, ce livre a \'et\'e construit pour le permettre;  
c'est pourquoi, si on le lit en entier, on trouvera bon nombre de redites.  
\medskip 
Je ne donne pas ici d'instructions de lecture plus pr\'ecises; ce serait  
vain car chaque lecteur est diff\'erent et nul ne  sait mieux que 
lui-m\^eme comment il convient de lire.  
\bigskip 
Apr\`es ces remarques pragmatiques destin\'ees au lecteur, j'\'eprouve 
encore le besoin de me justifier quant aux principes. 
\medskip 
Je ne discuterai pas les raisons ---~\'evoqu\'ees plus haut~--- pour 
lesquelles l'enseignement  scientifique se borne \`a aligner des formules 
et des techniques sans se pr\'eoccuper des questions qui se posent \`a 
leur sujet. Il est devenu courant de d\'evaloriser  comme ``philosophique'' 
toute question non strictement technique, mais les questions  auxquelles 
je tente de r\'epondre dans ce livre ne sont pas de la philosophie, elles 
sont partie int\'egrante de la science au sens le plus strict. Il arrivera 
certes deux ou trois fois qu'un philosophe soit cit\'e, mais  c'est alors 
pour des remarques d'ordre scientifique. Je quali\-fierai ces questions  
de {\it s\'emantiques}, faute d'un meilleur terme.  
\medskip 
Pour me faire bien comprendre, voici un exemple.  En Calcul des 
probabilit\'es, on d\'efinit formellement l'ind\'ependance stochastique: 
deux \'ev\'enements $A$ et $B$ sont stochastiquement ind\'ependants si 
la probabilit\'e de leur intersection  est \'egale au produit de leurs 
probabilit\'es. Les questions {\it formelles} sont celles qui concernent 
les  r\`egles et astuces purement alg\'ebriques de calcul qui d\'erivent 
de  cette d\'efinition. Les questions {\it s\'emantiques} par contre 
concernent la signification concr\`ete de ce concept: \`a quelle  
r\'ealit\'e correspond cette d\'efinition? \`A quoi reconna\^\i t-on que 
des \'ev\'enements r\'eels sont ind\'ependants? Pourquoi des 
\'ev\'enements r\'eels qui ne s'influencent pas mutuellement 
satisfont-ils cette d\'efinition? Et que signifie au juste ``ne pas 
s'influencer''?  \medskip Les questions formelles ne se posent qu'\`a 
l'int\'erieur de l'axiomatique. Les questions s\'emantiques se posent au 
contraire \`a l'ext\'erieur: elles concernent la justification  des 
d\'efinitions et axiomes, leur rapport  avec une r\'ealit\'e qu'on 
mod\'elise; elles peuvent \^etre critiques et conduire  \`a relativiser le 
formalisme, ou m\^eme \`a en contester la pertinence et \`a en percevoir 
les limites de validit\'e.   
\medskip 
Le danger d'une formation scientifique d\'epourvue de tout 
questionnement et de toute pens\'ee critique (danger d\'ej\`a largement 
d\'enonc\'e au $XIX^{\rm e}$  si\`ecle) est la perte progressive de 
cr\'eativit\'e de la pens\'ee scientifique. Si on prend la peine de faire 
certaines \'etudes historiques, et de se plonger  dans l'ambiance tr\`es 
intime des grandes d\'ecouvertes scientifiques du pass\'e (de Kepler  
\`a Einstein), en lisant les articles originaux, les auto\-bio\-graphies,  
les commentaires et les pol\'emiques des contemporains, etc., on est 
frapp\'e par la pr\'edominance des questions s\'emantiques. Ces  
derni\`eres sont beaucoup moins pr\'esentes dans les 
p\'eriodes de d\'eveloppement qui suivent les d\'ecouvertes.  C'est 
pourquoi ma  principale source d'inspiration a \'et\'e le contexte 
historique des d\'ecouvertes scientifiques majeures, o\`u j'ai puis\'e la 
plupart de mes exemples. Les questions s\'emantiques sont aussi celles 
que posent les \'etudiants vraiment curieux, s'ils n'ont pas \'et\'e trop 
d\'ecourag\'es de le faire ou s'ils ne se sentent pas trop ridicules.  
\medskip 
Bien entendu il ne suffit pas de poser des questions s\'emantiques pour 
faire des d\'ecouvertes, mais les  v\'eritables d\'ecouvertes 
commencent toujours par de telles questions.  Ces d\'ecouvertes 
consistent g\'en\'eralement \`a  s'apercevoir que  certaines \'evidences  
n'en \'etaient pas: par exemple que   le mouvement diurne du firmament 
n'est pas d\^u \`a sa rotation, que la lune ne reste pas dans le ciel parce 
qu'elle est l\'eg\`ere, que les calculs effectu\'es par Euler ne marchent 
pas pour n'importe quelles fonctions mais seulement pour les fonctions 
analytiques, que la simultan\'eit\'e de deux  \'ev\'enements \'eloign\'es 
d\'epend de la vitesse de l'observateur, etc.  Chacune de ces d\'ecouvertes  
est la r\'eponse \`a une certaine question s\'emantique. 
\medskip 
Un enseignement purement technique, \'epur\'e de toute question 
s\'eman\-ti\-que, se pr\'esente g\'en\'eralement sous la forme  
d'une axiomatique pos\'ee a priori, dont on ignore l'origine et les 
motivations;  on ignore {\it pourquoi}  et {\it comment} ces axiomes ont 
\'et\'e pos\'es un jour parce que ce savoir l\`a n'est pas transmis.  Il 
devient impossible de les mettre en doute,  d'en imaginer les limites. 
Peu \`a peu, la  critique et donc la cr\'eation  deviennent impossibles.  
\medskip 
Ce mal, ou du moins le risque d'y succomber, est aussi vieux que la 
science, il suffit de lire les Anciens pour s'en convaincre. Il n'est d\^u 
essentiellement, ni \`a la massification de l'enseignement scientifique, 
ni au culte effr\'en\'e du rendement qui  carac\-t\'e\-risent la 
soci\'et\'e moderne, puisque les philosophes grecs  s'en  plai\-gnaient 
d\'ej\`a. Il a simplement pris une {\it forme} moderne. 
\medskip 
Un autre mal, r\'eellement moderne celui-l\`a (il est typique du $XX^{\rm 
e}$ si\`ecle), est le cloisonnement de la science en {\it sp\'ecialit\'es}, 
gard\'ees par des clans jaloux de leur propri\'et\'e intellectuelle. Cette 
derni\`ere se r\'eduit d'ailleurs le plus souvent \`a la simple possession 
de  {\it mots-cl\'es}, au contenu scientifique pauvre, et d'un jargon  
incompr\'ehensible destin\'e \`a masquer ladite pauvret\'e. La science 
s'appelait autrefois {\it Philosophie de la Nature} et formait un tout. 
Jusqu'au tournant du si\`ecle, tout grand math\'ematicien comprenait la  
Physique aussi bien que la Math\'ematique; plus que cela: il ne s\'eparait 
pas les deux. Une compr\'ehension v\'eritable des concepts du Calcul des 
probabilit\'es  est impossible \`a l'int\'erieur de l'un de ces clans 
d\'efinis par leurs mots-cl\'es. C'est  pourquoi je me suis efforc\'e de 
renouer avec la Philosophie de la Nature. Et pour que l'ouvrage ne tourne 
pas \`a l'encyclop\'edie (ce qui aurait fortement  nui \`a son but), j'ai 
choisi de sacrifier un aspect secondaire du sujet, en ne parlant que de 
probabilit\'es discr\`etes. Cela \'evite par exemple de recourir \`a 
l'int\'egrale de Lebesgue, et la place ainsi gagn\'ee est mise \`a profit 
pour approfondir {\it la signification} des concepts. Cette tactique est 
d'autant plus efficace ici que les probabilit\'es discr\`etes sont 
conceptuellement tout aussi riches que les probabilit\'es d\'enombrables. 
\medskip 
D\'eplorer les maux mentionn\'es ci-dessus ne changera rien \`a la  
situation sociologique.  Il est vain de chercher \`a convaincre la grande 
masse ou les gouvernements de la gravit\'e de  la situation;  la grande 
masse a d'autres soucis, et pour ce  qui est de la gravit\'e, il  se passe 
tant de choses bien plus tra\-gi\-ques qu'on ne mobilisera gu\`ere sur  
le th\`eme de la science menac\'ee. Publier un  appel dans la presse, sur 
un ton de d\'etresse invoquant le d\'eclin de la France ou de l'Europe, 
\'equivaut \`a jeter une goutte d'eau dans le Rhin.  D'ailleurs il n'est pas 
du tout \'evident que la science ait vraiment gagn\'e \`a se transformer 
en cette gigantesque bureaucratie de type sovi\'etique que nous 
connaissons aujourd'hui, et  les appels \`a sauver l'{\it esprit} de la 
science ne seraient  pas seulement noy\'es dans les innombrables appels 
d'intellectuels p\'etitionnaires qui se tuent \`a attirer l'attention sur des 
probl\`emes  bien plus urgents; ils seraient aussi noy\'es parmi les 
innombrables appels de la nomenklatura scientifique pour l'augmentation 
de ses cr\'edits.  
\medskip 
En cons\'equence, tout en \'etant bien conscient de la t\'enuit\'e de mon 
action, j'ai pens\'e  qu'il serait plus utile de jeter une bou\'ee de 
sauvetage \`a ceux  qui se noient. Je signale \`a ce propos que la bou\'ee 
que je lance avait auparavant servi \`a me sauver moi-m\^eme de la 
noyade.  
\vskip12mm 
\line{\hfill\hfill Strasbourg, le 15 mars {\oldstyle 1995}\hfill } 
 
 
 
 
 
 
\bye 
