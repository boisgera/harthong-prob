\input twelvea4.tex
\input epsf.tex
\vsize=219mm

\auteurcourant={\sl J. Harthong: probabilit\'es et statistique}
\titrecourant={\sl Le hasard}
 
\newdimen\blocksize  \blocksize=\vsize \advance\blocksize by-8pt 
\medskipamount=6pt plus3pt minus3pt 
 
 
\null\vskip10mm plus4mm minus3mm 
 
\centerline{\tit I. LE HASARD.} 
\vskip10mm 
{\bf I. 1. Le langage du Calcul des probabilit\'es.} 
\medskip 
{\cit ``La th\'eorie des hasards consiste \`a r\'eduire tous les 
\'ev\'enements du m\^eme genre \`a un certain nombre de cas \'egalement
possibles et \`a d\'eterminer le nombre de cas favorables \`a l'\'ev\'enement
dont on cherche la probabilit\'e. Le rapport de ce nombre \`a celui de tous 
les cas possibles est la mesure de cette probabilit\'e, qui n'est ainsi 
qu'une fraction dont le num\'erateur est le nombre de cas favorables, et 
dont le d\'enominateur est le nombre de tous les cas possibles.''\par}
\smallskip
\line{\hfill \vbox{\hsize=66mm \eightpoint 
\centerline{Pierre-Simon Laplace}
\centerline{\sl Th\'eorie analytique des probabilit\'es ({\oldstyle 1819}).}}}
\medskip
Il n'existe pas de meilleure d\'efinition de la probabilit\'e que celle-ci, vieille de plusieurs si\`ecles, et c'est d'elle que nous partirons.
\medskip
Par exemple si on tire ``au  
hasard''  un nombre entre 1 et 36, la probabilit\'e d'obtenir 11 est $1 
\over 36$; si  la probabilit\'e d'obtenir 11 \'etait $1 \over 32$, qui est 
sup\'erieur \`a $1 \over 36$, cela signifierait que le nombre 11 est 
privil\'egi\'e par la Fortune ou que la roulette est truqu\'ee. Il y aurait 
une  cause agissant secr\`etement et le hasard pur ne pourrait 
\'eventuellement \^etre retrouv\'e qu'au-del\`a de cette cause. On 
ne pourrait pas consid\'erer que le tirage est soumis au seul hasard.  
\medskip 
On appelle {\it \'epreuve} un tel r\'esultat possible. Par principe, toutes  
les \'epreuves sont suppos\'ees \'equiprobables. 
\medskip 
{\eightpoint N.B.  Lorsqu'on applique le Calcul des probabilit\'es \`a  
la r\'ealit\'e il est bien rare que l'on ait le loisir de s'assurer  
rigoureusement  et \`a l'avance que cette hypoth\`ese 
d'\'equiprobabilit\'e est bien v\'erifi\'ee, car d'innombrables {\it 
causes} peuvent intervenir \`a notre insu pour fausser les chances. Le 
mod\`ele math\'ematique postulant l'\'equiprobabilit\'e peut \^etre 
valable lorsque les diff\'erentes causes sont suffisamment nombreuses 
et de forces \'egales pour qu'aucune ne domine (par exemple dans le cas 
des accidents de la route), ou bien que des sym\'etries fondamentales  
de l'espace-temps assurent a priori qu'il ne peut y avoir de domaine 
privil\'egi\'e (par exemple dans le cas des ph\'enom\`enes physiques). 
Dans le jeu de la roulette, c'est la conception m\^eme de la roulette qui 
garantit l'\'equiprobabilit\'e avec une grande pr\'ecision. Toutefois, 
nous verrons (\S 2) que l'hypoth\`ese  d'\'equiprobabilit\'e ne signifie 
rien en soi: encore faut-il savoir quelles sont les \'epreuves 
\'equiprobables.} 
\medskip 
Un r\'esultat attendu peut \^etre tout un ensemble d'\'epreuves. Pour  
donner un exemple, revenons au tirage d'un nombre entre 1 et 36: 
on peut se demander quelle est la probabilit\'e d'obtenir un nombre 
premier. La r\'eponse est $11 \over 36$. Le seul moyen de trouver cette 
probabilit\'e est de faire la liste exhaustive des nombres premiers 
compris entre 1 et 36: 2, 3, 5, 7, 11, 13, 17, 19, 23, 29, 31. On constate 
qu'ils sont au nombre de onze, soit un peu moins que le tiers de 36. Le
r\'esultat attendu n'\'etait pas une \'epreuve isol\'ee, mais tout un 
ensemble d'\'epreuves r\'eunies par une propri\'et\'e commune.  
\medskip 
On appelle {\it \'ev\'enement} un tel ensemble d'\'epreuves.  
L'\'equiprobabilit\'e des \'epreuves a pour cons\'equence imm\'ediate  
que  la probabilit\'e d'un \'ev\'enement est le rapport: 
$$\hbox{nombre d'\'epreuves appartenant \`a l'\'ev\'enement} \over 
\hbox{nombre de toutes les \'epreuves possibles}\eqno (I.1)$$ 
\medskip 
Dans l'exemple des nombres premiers, le seul moyen de 
conna\^\i tre la pro\-ba\-bi\-lit\'e \'etait de compter exhaustivement. Mais  
dans beaucoup de probl\`emes comportant des r\'egularit\'es, des 
sym\'etries,  ou des invariances, une m\'ethode math\'ematique 
appropri\'ee permet de calculer, sans avoir \`a compter. C'est 
l'ensemble de ces m\'ethodes qui constitue le {\it Calcul de 
Probabilit\'es}.   
\medskip 
L'ensemble de toutes les \'epreuves possibles est \'evidemment  
sp\'ecifique du probl\`eme \'etudi\'e. On l'appelle {\it espace des 
\'epreuves} (en anglais {\it sample space}) et on le note tr\`es souvent 
$\Omega$. Dans le langage math\'ematique on \'ecrit  
$$\eqalign{ 
\Omega = \{\, &1\, , \, 2\, , \, 3\, , \, 4\, , \, 5\, , \, 6\, , \, 7\, , \,  
8\, ,  
 \, 9\, , \, 10\, , \, 11\, , \, 12\, , \,  \cr 
&13\, , \, 14\, , \, 15\, , \, 16\, , \, 17\, ,  \, 18\, , \, 19\, , \, 20\, ,  
\, 
21\, , \, 22\, , \, 23\, , \, 24\, , \,  \cr 
&25\, , \, 26\, ,  \, 27\, , \, 28\, , \, 29\, , \, 30\, , \, 31\, , \, 32\, ,  
\,  
33\, , \, 34\,  , \, 35\, , \, 36\,\} \cr }$$  
Le sous-ensemble des nombres premiers est, dans le m\^eme
langage math\'e\-ma\-tique  
$$A = \{\, 2\, , \, 3\, , \, 5\, , \, 7\, , \, 11\, , \, 13\, , \,  
17\, 
,  \, 19\, , \, 23\, , \, 29\, , \, 31\,\}$$ 
L'\'ev\'enement $A$ est ainsi un sous-ensemble de $\Omega$, qui est 
g\'en\'eralement d\'efini par une propri\'et\'e commune \`a tous ses 
\'el\'ements, \`a savoir ici, ``\^etre un nombre premier''. Dans le langage 
du Calcul des probabilit\'es, on d\'esigne souvent un \'ev\'enement par 
une phrase  imag\'ee qui exprime cette propri\'et\'e commune \`a tous 
ses \'el\'ements. On dira que $A$ est  l'\'ev\'enement ``on a tir\'e un 
nombre premier''.  Ainsi on retrouve le  sens du mot \'ev\'enement dans 
le langage courant. Dans le langage du Calcul des probabilit\'es, on 
d\'esigne indiff\'eremment un \'ev\'enement  par une phrase 
imag\'ee exprimant une certaine propri\'et\'e, ou comme  un 
sous-ensemble de $\Omega$. Le sous-ensemble en question est alors 
l'ensemble des \'epreuves qui poss\`edent ladite propri\'et\'e. La phrase 
imag\'ee permet de se faire comprendre facilement, elle exprime aussi  
le  sens du probl\`eme; la traduction math\'ematique sous forme 
d'ensemble sert par contre \`a calculer. Il faut s'exercer \`a passer de 
l'un \`a l'autre. Nous verrons beaucoup d'exemples qui permettront de s'y 
entra\^\i ner.  
\medskip 
Si $A$ est un \'ev\'enement ($A \subset \Omega$) on note $\#A$ le 
nombre d'\'el\'ements (le cardinal) de l'ensemble $A$, c'est-\`a-dire le 
nombre d'\'epreuves qui constituent l'\'ev\'enement $A$. La probabilit\'e  
d'un \'ev\'enement $A$ est ainsi: 
$${\cal P}(A) = {\#A \over \#\Omega}\eqno (I.2)$$ 
qui est une traduction imm\'ediate de $(I.1.)$ 
\medskip 
La r\'esolution de n'importe quel probl\`eme concret  
comporte traditionnellement deux  \'etapes:  
\smallskip 
$a$) la mod\'elisation (nom moderne de ce qu'on appelait autrefois 
``mise  en \'equations") 
\smallskip 
$b$) la r\'esolution analytique ou le traitement num\'erique de ces  
\'equations.  
\medskip 
\noindent En Calcul des probabilit\'es, la phase $a)$ de mod\'elisation  
consiste \`a trouver l'espace des \'epreuves $\Omega$ correspondant  
au probl\`eme; la phase b) consiste \`a calculer  les probabilit\'es \`a 
l'aide de $I.2.$ 
\medskip 
La th\'eorie math\'ematique des probabilit\'es d\'eveloppe une  {\it 
alg\`ebre}, c'est-\`a-dire des r\`egles de calcul, qui s'appliquent en 
g\'en\'eral aux probabilit\'es des \'ev\'enements dans des espaces 
d'\'epreuves non n\'ecessairement finis. Pour des espaces d'\'epreuves  
{\it finis} (le  seul cas qui nous int\'eresse) cette th\'eorie est 
\'el\'ementaire, mais suffit \`a traiter tous les probl\`emes. La suite de 
l'ouvrage traitera de probl\`emes classiques  qu'il faut conna\^\i tre, 
de m\'ethodes de calcul approch\'e, de lois asymptotiques, d'applications 
diverses.  
\medskip 
Ainsi le langage du Calcul des probabilit\'es est extr\^emement simple. 
Pour le moment nous en avons pr\'esent\'e les bases, qui se r\'eduisent 
au vocabulaire suivant. 
\smallskip 
--- {\bf \'epreuve:} un des r\'esultats possibles, que le 
hasard peut choisir, sans en favoriser aucun. 
\smallskip 
--- {\bf espace des \'epreuves:} l'ensemble de toutes les \'epreuves. 
\smallskip 
--- {\bf \'ev\'enement:} un ensemble d'\'epreuves, g\'en\'eralement  
d\'efini par une propri\'et\'e commune \`a ses \'el\'ements. 
\smallskip 
--- {\bf cardinal:}  nombre d'\'el\'ements d'un ensemble fini. 
\smallskip 
--- {\bf probabilit\'e:} la probabilit\'e d'un \'ev\'enement $A$ est le  
rapport du cardinal de $A$ au cardinal de $\Omega$. C'est donc un 
nombre toujours compris entre 0 et 1. 
\medskip 
Peu \`a peu ce vocabulaire de base sera enrichi par de nouvelles 
d\'efinitions. 
\medskip 
Le mot {\it hasard} ne fait pas partie du vocabulaire technique du  
Calcul des probabilit\'es, car ce dernier ne se pr\'eoccupe que des  
r\`egles et des m\'ethodes de calcul. Une discussion sur la nature du 
hasard est cependant essentielle; sans cela, le Calcul des probabilit\'es 
n'est qu'un formalisme vide, qu'on saura utiliser pour r\'esoudre des 
probl\`emes scolaires \'enonc\'es dans le m\^eme formalisme, mais 
qu'on ne saura  plus utiliser devant des probl\`emes concrets {\it 
d\'epourvus} de formalisation pr\'ealable. Nous avons dit que la {\it 
mod\'elisation} d'un probl\`eme de probabilit\'es consistait \`a trouver 
l'espace des \'epreuves ad\'equat. Mais ``espace des \'epreuves'' 
sous-entend \'equiprobabilit\'e, et il n'est pas \'evident de savoir ce qui 
est \'equiprobable. C'est cela qui soul\`eve toute une discussion sur la 
nature du hasard, \`a laquelle le reste de ce chapitre est consacr\'e.  
 
\vskip7mm plus6mm minus5mm
 
{\bf I. 2. \'Etude de deux exemples.} 
\medskip 
L'exemple extr\^emement simple du tirage au hasard d'un nombre entre 1 
et 36 ne pose aucun probl\`eme de mod\'elisation: le choix de l'ensemble 
$\Omega$ ad\'equat est imm\'ediat. Il ne laisse rien soup\c conner de la 
profondeur du concept. C'est pourquoi nous \'etudions dans cette section 
deux probl\`emes non triviaux (quoique simples), afin d'illustrer d\`es 
maintenant le v\'eritable sens des concepts d\'ej\`a introduits. 

\bigskip 

{\bf Premier exemple: distribution au hasard de trois boules dans deux 
bo\^\i tes.} 
 
\midinsert 
\def\trv{\vrule height8pt depth4pt width0.5pt} 
\newdimen\lrg  
\lrg=15mm \advance\lrg by -0.5pt 
\def\trh{\hrule height0.5pt depth0pt width32.94mm} 
\def\sp{\hskip\lrg} 
\def\boule#1{\hbox to 12pt{$\bigcirc$\hskip-8pt 
\raise0.5pt\hbox{$\scriptstyle #1$}\hfill }} 
\def\ball{\hbox to 12pt{ \hfill $\bigcirc$ \hfill }} 
\vskip12pt 
\centerline{ 
\vtop{\hsize=4cm 
\hbox{\trv 
\hbox to \lrg { \hfill \boule1 \hfill \boule2 \hfill \boule3 \hfill } 
\trv \hbox to \lrg { \hfill } \trv}\trh  
\vskip0pt 
\hbox{\hskip-1pt\vbox{$$\left. \vcenter{
\hbox{\trv 
\hbox to \lrg { \hfill \hskip5pt \boule1 \hfill \boule2 \hfill } 
\trv \hbox to \lrg { \hfill \hskip5pt \boule3 \hfill } \trv } \trh  
\vskip12pt 
\hbox{\trv 
\hbox to \lrg { \hfill \hskip5pt \boule2 \hfill \boule3 \hfill } 
\trv \hbox to \lrg { \hfill \hskip5pt \boule1 \hfill } \trv } \trh  
\vskip12pt 
\hbox{\trv 
\hbox to \lrg { \hfill \hskip5pt \boule1 \hfill \boule3 \hfill } 
\trv \hbox to \lrg { \hfill \hskip5pt \boule2 \hfill } \trv } \trh }
\hskip10pt \right\}$$ } }
\vskip-12pt 
\hbox{\hskip-1pt\vbox{$$\left. \vcenter{
\hbox{\trv 
\hbox to \lrg { \hfill \hskip5pt \boule1 \hfill  } 
\trv \hbox to \lrg { \hfill \hskip5pt \boule2 \hfill \boule3 \hfill } \trv } \trh  
\vskip12pt 
\hbox{\trv 
\hbox to \lrg { \hfill \hskip5pt \boule2 \hfill  } 
\trv \hbox to \lrg { \hfill \hskip5pt \boule1 \hfill \boule3 \hfill } \trv } \trh  
\vskip12pt 
\hbox{\trv 
\hbox to \lrg { \hfill \hskip5pt \boule3 \hfill  } 
\trv \hbox to \lrg { \hfill \hskip5pt \boule1 \hfill \boule2 \hfill } \trv }
\trh } \hskip10pt \right\}$$ } } 
\vskip-6pt 
\hbox{\trv \hbox to \lrg { \hfill } \trv 
\hbox to \lrg { \hfill \boule1 \hfill \boule2 \hfill \boule3 \hfill } 
\trv} \trh  
\vskip12pt } 
\hskip2cm 
\def\trh{\hrule height0.5pt depth0pt width32.9mm} 
\vtop{\hsize=4cm 
\hbox{\trv 
\hbox to \lrg { \hfill \ball \hfill \ball \hfill \ball \hskip6pt \hfill } 
\trv \hbox to \lrg { \hfill } \trv}\trh  
\vskip36.8pt 
\hbox{\trv 
\hbox to \lrg { \hfill \ball \hfill \ball \hskip5pt \hfill } 
\trv \hbox to \lrg { \hfill \ball\hskip5pt  \hfill } \trv } \trh  
\vskip61.6pt 
\hbox{\trv 
\hbox to \lrg { \hfill \ball \hskip5pt \hfill  } 
\trv \hbox to \lrg { \hfill \ball \hfill \ball \hskip5pt \hfill } \trv } \trh  
\vskip36.6pt 
\hbox{\trv \hbox to \lrg { \hfill } \trv 
\hbox to \lrg { \hfill \ball \hfill \ball \hfill \ball \hskip5pt \hfill } 
\trv} \trh  
\vskip12pt } 
}

\centerline{\eightpoint figure 1} 
\vskip5mm 
\endinsert 
 
On voit sur la figure 1: dans la colonne de gauche, toutes les 
r\'epartitions possibles de trois boules num\'erot\'ees 1, 2, 3 dans les  
deux bo\^\i tes; il y en a huit en tout. Si on supprime les marques sur les 
boules, on ne peut plus distinguer les r\'epartitions $(1,2 \mid 3)$, 
$(2,3 \mid 1)$, et $(1,3 \mid 2)$, ni $(1 \mid 2,3)$, $(2 \mid 1,3)$, et 
$(3 \mid 1,2)$. C'est pourquoi, sur la colonne de droite, il n'y a plus que 
quatre r\'epartitions.  
\medskip 
Question: pour des boules sans marques, l'espace des \'epreuves est-il 
form\'e des huit r\'epartitions de la colonne de gauche, ou des quatre de 
la colonne de droite ?   
\medskip 
Il se trouve que la r\'eponse d\'epend de la nature des boules: s'il s'agit 
de boules macroscopiques ob\'eissant \`a la M\'ecanique classique, 
c'est la premi\`ere r\'eponse qui convient (et peu importe  qu'elles 
soient num\'erot\'ees ou non, car le probl\`eme n'est pas qu'elles soient 
subjectivement num\'erot\'ees, mais qu'elles soient objectivement 
num\'erotables); s'il s'agit de particules quantiques ob\'eissant \`a la 
statistique de Bose-Einstein, c'est la deuxi\`eme r\'eponse qui convient.  
L'\'equiprobabilit\'e est bien dans les deux cas l'expression du {\it pur 
hasard}, mais il se trouve que le pur hasard ne se situe pas au m\^eme 
niveau  de causalit\'e selon que les particules sont classiques ou  
quantiques.   
\medskip 
 
{\eightpoint 
Dans la plupart des ouvrages sur le Calcul des probabilit\'es, on ne 
donne pas de la probabilit\'e ${\cal P}\, (A)$ une d\'efinition 
directement d\'eriv\'ee de l'hypoth\`ese d'\'equiprobabilit\'e comme 
nous le faisons ici, mais une d\'efinition  axiomatique du type 
suivant: ``\'etant donn\'e un ensemble fini $\Omega$, appel\'e espace des 
\'epreuves, on appelle {\it probabilit\'e} sur  $\Omega$ une application 
${\cal P}$ de l'ensemble des parties de $\Omega$ sur l'intervalle $[0, \; 
1 [$ qui v\'erifie la propri\'et\'e d'additivit\'e  ${\cal P}\, (A \cup 
B) = {\cal P}\, (A) + {\cal P}\, (B)$ pour tout couple de parties $A$ et 
$B$ disjointes, ainsi que la propri\'et\'e ${\cal P}\, (\Omega ) = 1$''. Il 
est bien clair que la probabilit\'e telle que nous l'avons d\'efinie dans 
$(I.2.)$ satisfait \`a cette d\'efinition axiomatique. Mais ce que la 
d\'efinition axiomatique recouvre est plus g\'en\'eral et n'exige pas 
n\'ecessairement que toutes les \'epreuves soient \'equiprobables; on 
peut donner \`a chaque \'epreuve un {\it poids}, c'est-\`a-dire un atome 
de probabilit\'e. Si les poids sont diff\'erents il n'y a pas 
\'equiprobabilit\'e et $(I.2.)$ doit \^etre remplac\'e par   
$${\cal P}\, (A) = {\hbox{somme des poids des \'epreuves appartenant 
\`a}\; A \over \hbox{somme des poids de toutes les \'epreuves} }\eqno 
(II.2\; bis.)$$  
Ceci v\'erifie aussi la d\'efinition axiomatique et inversement, 
toute application de l'ensemble des parties de $\Omega$ sur l'intervalle 
$[0, \; +\infty [$ qui v\'erifie la propri\'et\'e d'additivit\'e sera de la  
forme $(II.2\; bis.)$, avec le poids ${\cal P}\, (\{\omega\})$ pour chaque 
\'epreuve $\omega$.  
\medskip  
Les auteurs qui proc\`edent ainsi ont pour cela une excellente raison: 
exposer le Calcul des probabilit\'es pour des \'epreuves non 
n\'ecessairement \'equiprobables ne co\^ute pas une seule phrase de plus 
que pour des \'epreuves \'equiprobables, car tout se d\'eduit de 
l'additivit\'e (ce qu'on d\'eduit de I.2. passe toujours par l'interm\'ediaire 
logique de l'additivit\'e). Pourquoi donc se priver de quelque chose 
d'absolument gratuit?   
\medskip 
En fait l'enjeu est le {\it sens} du Calcul des probabilit\'es, ou  
autrement dit, la signification du mot hasard. 
\medskip 
La caract\'eristique du pur hasard {\it est} l'\'equiprobabilit\'e. Nous 
reviendrons encore sur ce point au paragraphe suivant. Si des 
\'epreuves ne sont pas \'equiprobables, c'est que certaines sont  
favoris\'ees et donc qu'une cause, qui n'est pas le hasard, agit 
secr\`etement. Il ne s'agit pas l\`a d'un choix philosophique: si une  
\'etude statistique \'etablit que le cancer du pancr\'eas est plus 
fr\'equent dans la haute vall\'ee de l'Is\`ere que partout ailleurs, on en 
d\'eduit qu'une cause, agissant dans cette haute vall\'ee et non ailleurs 
(ou agissant plus fortement dans cette haute vall\'ee qu'ailleurs) est 
responsable de cette surfr\'equence. M\^eme si apr\`es de longues 
recherches cette cause reste introuvable, il n'y aura pas de querelle 
philosophique sur la question de savoir si la Fortune est ou n'est pas 
aveugle; on ne parlera de hasard que si la fr\'equence est distribu\'ee 
uniform\'ement, car c'est l\`a le sens m\^eme du mot hasard.  
\medskip 
Il peut \^etre commode de disposer de mod\`eles th\'eoriques o\`u 
l'\'equiprobabilit\'e n'est pas postul\'ee a priori. Par exemple dans un  
probl\`eme o\`u on jette $r$ boules dans $k$ cases identiques, mais o\`u  
on ne s'int\'eresse qu'\`a l'alternative:  ``case 1'' ou ``autre case'', 
il est pr\'ef\'erable de mod\'eliser la situation en ne parlant que de 
deux possibilit\'es seulement, mais de sorte que la probabilit\'e de  
la premi\`ere soit $1/k$ et la probabilit\'e de la 
deuxi\`eme $1-1/k$.  Il s'agit alors d'un mod\`ele 
ph\'enom\'enologique et on retrouve le vrai hasard en creusant 
davantage. Il faut cependant admettre que dans ce cas le gain en 
simplicit\'e qui r\'esulte du mod\`ele ph\'enom\'enologique (compar\'e 
au mod\`ele avec \'equiprobabilit\'e) est m\'ediocre. Une exp\'erience 
plus pouss\'ee du Calcul des probabilit\'es montrerait que le gain est 
m\'ediocre dans tous les cas. Nous verrons au chapitre {\bf IV} 
la {\it formule des probabilit\'es conditionnelles} qui \'etablit le lien  
entre de tels mod\`eles ph\'enom\'enologiques et le mod\`ele 
\'equiprobable sous-jacent. 
\medskip 
En revanche le fait de poser les probl\`emes de probabilit\'e en  
remontant syst\'e\-ma\-ti\-que\-ment au {\it pur hasard}, sans 
s'arr\^eter \`a des repr\'esentations ph\'enom\'enologiques, est 
fonci\`erement instructif.   
\medskip 
C'est justement ce que montre l'exemple ci-dessus de trois boules \`a 
disposer dans deux cases . On peut pour les deux situations discut\'ees 
consid\'erer un espace des \'epreuves \`a quatre \'el\'ements,  les  
quatre \'el\'ements \'etant:  
$$\eqalign{ 
&\omega_1 : \quad \left\{ \vcenter{ 
\hbox{trois boules dans la case 1}  
\vskip-5pt 
\hbox{z\'ero boule dans la case 2 \vrule depth2pt width0pt }} \right. \cr 
&\omega_2 : \quad \left\{ \vcenter{ 
\hbox{deux boules dans la case 1}  
\vskip-5pt 
\hbox{une boule dans la case 2 \vrule depth2pt width0pt }} \right.  \cr 
&\omega_3 : \quad \left\{ \vcenter{ 
\hbox{une boule dans la case 1}  
\vskip-5pt 
\hbox{deux boules dans la case 2 \vrule depth2pt width0pt }} \right. \cr  
&\omega_4 : \quad \left\{ \vcenter{ 
\hbox{z\'ero boule dans la case 1}  
\vskip-5pt 
\hbox{trois boules dans la case 2\vrule depth2pt width0pt}}\right.\cr }$$ 
Dans le cas de la statistique de Bose-Einstein, les boules sont par  
exemple des photons et les cases des \'etats quantiques; les quatre 
\'epreuves $\omega_1,\,\omega_2,\,\omega_3,\,\omega_4$ sont 
\'equiprobables;  dans le cas des boules macroscopiques, par contre (et 
peu importe qu'elles  soient num\'erot\'ees ou non, elles sont toujours 
{\it objectivement} discernables, du fait qu'elles sont et restent 
situ\'ees dans l'espace-temps), les quatre \'epreuves ne sont pas 
\'equiprobables; leurs probabilit\'es respectives sont ${1 \over 8},\, {3 
\over 8},\, {3 \over 8},\, {1 \over 8}$. On peut alors calculer tout ce 
qu'on veut sur les probabilit\'es de tous les \'ev\'enements concernant 
ces \'epreuves, et on aboutira aux m\^emes  r\'esultats dans le mod\`ele 
\`a quatre \'el\'ements non \'equiprobables que dans le mod\`ele \`a huit 
\'el\'ements \'equiprobables. Mais bien entendu, il aura fallu auparavant 
trouver les valeurs ${1 \over 8},\, {3 \over 8},\, {3 \over 8},\,  {1 
\over 8}$. Et pour cela il n'y avait aucun autre moyen que de sortir du  
mod\`ele \`a quatre \'el\'ements, venir en secret dans le mod\`ele \`a 
huit \'el\'ements \'equiprobables pour y d\'eterminer les valeurs des 
poids, puis retourner au mod\`ele \`a quatre \'el\'ements en faisant 
semblant de les avoir devin\'ees gr\^ace au g\'enie.  
\medskip  
Si on analyse bien cet exemple, on comprend que le {\it pur hasard} est   
une expression des sym\'etries fondamentales: si les huit \'epreuves  
 du  mod\`ele \`a huit \'el\'ements sont effectivement \'equiprobables 
c'est parce que les trois boules peuvent \^etre isol\'ees dans l'espace. 
Quoiqu'il arrive \`a l'une des trois boules, cela n'affecte pas l'\'etat des 
deux autres. Lorsque la Fortune dispose la deuxi\`eme dans l'une des 
deux cases, sa  c\'ecit\'e consiste \`a ne pas percevoir l'\'etat 
d'occupation des deux cases par la premi\`ere boule, mais seulement les 
positions spatio-temporelles relatives de la deuxi\`eme boule et des 
deux cases. En revanche, lorsque la m\^eme Fortune dispose le second 
photon dans l'un des deux \'etats, sa c\'ecit\'e consiste \`a ne pas 
percevoir les positions spatio-temporelles relatives des photons
(ces positions n'ont d'ailleurs aucun sens objectif) et des 
deux \'etats quantiques, mais seulement les \'etats d'occupation. Ainsi 
dans le premier cas aucune des boules ne favorise une case, tandis que 
dans le second cas aucun des photons ne favorise un \'etat d'occupation. 
Dans le premier cas le pur hasard est le reflet d'une invariance 
spatio-temporelle (les deux cases sont \'equivalentes car ce sont des 
lieux sym\'etriques de l'espace, et chacune des trois boules agit 
ind\'ependamment des autres, c'est-\`a-dire que sa trajectoire ne 
d\'epend pas du fait que les autres boules ont \'et\'e lanc\'ees avant ou 
apr\`es). Dans le second cas le pur hasard est le reflet d'une 
indiscernabilit\'e objective des trois photons; seuls peuvent \^etre 
objectivement distingu\'es les quatre \'etats d'occupation.   
\medskip  
Or aucun probl\`eme de probabilit\'e ne peut \^etre r\'esolu sans 
recourir directement au postulat que la Fortune est aveugle. Si une 
possibilit\'e est favoris\'ee par rapport \`a une autre (dans le sens 
qu'elle est plus probable) on ne peut calculer a priori sa probabilit\'e 
qu'en recherchant le hasard pur \`a un niveau sup\'erieur. Mais on ne 
peut \'evidemment appliquer le  postulat que si on sait quelle 
sym\'etrie intervient dans le probl\`eme. Cela est tellement vrai qu'on 
peut le v\'erifier par exemple dans le livre  de William F{\eightrm 
ELLER} {\it An Introduction to Probability Theory  and its Applications} 
(John Wiley ed.) Je choisis ce livre parce que c'est, sur le sujet, {\it le} 
livre de r\'ef\'erence: il est pratiquement impossible d'enseigner mieux 
que F{\eightrm ELLER} les questions trait\'ees dans son livre; c'est 
pourquoi certaines parties de ce cours sont simplement reprises 
d'apr\`es F{\eightrm ELLER} (voir par exemple le chapitre {\it marches 
al\'eatoires}). Bien qu'ennemi jur\'e du formalisme  math\'ematique, 
F{\eightrm ELLER} ne renonce pas \`a la commodit\'e de l'approche 
axiomatique qui, comme nous l'avons dit plus haut, ne co\^ute aucun 
effort suppl\'ementaire, et laisse la porte ouverte aux espaces 
d'\'epreuves non \'equiprobables. De ce fait il est conduit \`a donner des 
exemples ou des exercices o\`u interviennent des espaces d'\'epreuves 
non \'equiprobables. Mais bien s\^ur dans chacun de ces exemples ou 
exercices, les poids qui sont ``parachut\'es'' au lecteur ont \'et\'e 
d\'etermin\'es par l'auteur \`a partir d'un mod\`ele \`a \'epreuves 
\'equiprobables.   
\medskip  

Ainsi, exercice 4. page 24: 
\smallskip 
{\leftskip=28pt \rightskip=10pt  ``A coin is tossed until for the first time the same result 
appears twice in succession. To every possible outcome requiring $n$ 
tosses attribute probability $1 / 2^{n-1}$. Describe the sample space.  
Find the probability of the following events: (a) the experiment ends 
before the sixth toss, (b) an {\it even} number of tosses is required.''\par}
\smallskip 
Voir aussi les deux exercices qui suivent celui-ci.  
\smallskip 
Le choix du poids $1 / 2^{n-1}$ provient d'une \'equiprobabilit\'e \`a un  
niveau sup\'erieur (nous reviendrons l\`a-dessus au chapitre {\it 
probabilit\'es conditionnelles}). On trouvera encore un tel exemple  
pages 47 -- 48 (chap $II$ \S 7: Examples for waiting times). 
Fr\'equemment dans son ouvrage, F{\eightrm ELLER} exprime la 
condition ``assuming perfect randomness, $\ldots$''. Cette condition 
signifie toujours que le probl\`eme poss\`ede un mod\`ele avec 
\'equiprobabilit\'e, et dans lequel cette \'equiprobabilit\'e r\'esulte 
d'une sym\'etrie \'evidente (le tout \'etant de la trouver). On appellera  
{\it probabilit\'es a priori} les probabilit\'es ainsi obtenues par 
consid\'eration d'une sym\'etrie. En l'absence de sym\'etrie connue on  
peut aussi mesurer a posteriori les probabilit\'es par la statistique; on 
ignore alors o\`u est intervenu le pur hasard, et on parlera de {\it 
probabilit\'es empiriques}. On peut toujours construire a posteriori un 
mod\`ele ph\'enom\'enologique reprenant ces probabilit\'es empiriques. 
Mais j'ai estim\'e pr\'ef\'erable de recourir \`a la formule des 
probabilit\'es conditionnelles (chap. {\bf IV}). 
 
} %%% end of \eightpoint 
 
\vskip7mm plus6mm minus5mm
\penalty-600 
 
{\bf Deuxi\`eme exemple: tirage au hasard d'une corde 
dans un cercle.} 

\penalty600 
\medskip 
Cet exemple est c\'el\`ebre: il est tir\'e du {\it Calcul des 
probabilit\'es} de Joseph Bertrand, un ouvrage paru en {\oldstyle 1888}. 
Sa c\'el\'ebrit\'e provient de ce qu'il a \'et\'e repris et comment\'e 
par Henri Poincar\'e, puis encore par \'Emile Borel. Il examine trois
mani\`eres diff\'erentes de tirer au hasard une corde dans un cercle 
de rayon $R$: 
\smallskip 
{\bf 1.} on tire au hasard deux points sur le cercle (ce qui \'equivaut 
\`a deux nombres compris entre 0 et $2\pi$): ils d\'efinissent une corde,  
celle qui joint les deux points; 
\smallskip 
{\bf 2.} on tire au hasard une direction (ce qui \'equivaut \`a un nombre 
$\theta$ entre 0 et $2\pi$) et une distance $d$ au centre du cercle (ce  
qui \'equivaut \`a un nombre entre 0 et $R$); ces deux donn\'ees 
d\'efinissent univoquement une corde perpendiculaire \`a la direction 
donn\'ee et \`a la distance $d$ du centre;  
\smallskip   
{\bf 3.} on tire au hasard un point \`a l'int\'erieur du cercle: ce point  
d\'efinit univoquement une corde dont le centre est ce point.  
\medskip 
\`A premi\`ere vue le probl\`eme rel\`eve du Continu: il y a un continuum  
de cordes, et les nombres qui les d\'eterminent sont des nombres 
r\'eels. Cela ne pose aucune difficult\'e particuli\`ere, mais il est de 
toute fa\c con ais\'e de discr\'etiser le probl\`eme: on d\'ecoupe les 
intervalles dans  lesquels les nombres doivent \^etre choisis au hasard 
en un grand nombre  de parties \'egales. Cela est montr\'e sur la figure 2 
pour les trois cas. 
 
\midinsert 
\vskip3pt 
\centerline{ \epsfbox{../images/fig2.eps }}  
\vskip3mm 
\centerline{\eightpoint figure 2} 
\vskip6pt 
\centerline{\vbox{\hsize=12cm \eightpoint On a repr\'esent\'e la 
discr\'etisation du probl\`eme dans les trois cas. Afin de ne pas  
surcharger la figure, on n'a repr\'esent\'e la discr\'etisation que 
radialement pour les cas 1 et 2. Dans le cas 3, le quadrillage d\'etruit 
bien s\^ur la sym\'etrie sph\'erique, mais celle-ci est conserv\'ee 
statistiquement. }}  
\vskip3mm 
 
 \endinsert 
 
{\bf Probl\`eme:} quelle est la probabilit\'e pour que la distance de la  
corde au centre du cercle soit inf\'erieur \`a $R/2$ ou ce qui est 
\'equivalent, quelle  est la probabilit\'e pour que la longueur de la 
corde soit sup\'erieure \`a $R\sqrt{3}$? 
 
\midinsert 
\vbox to \blocksize { 
\vfill 
\centerline{\epsfbox{../images/fig3-1.eps}} 
\vskip3mm 
\centerline{\eightpoint figure 3.1.} 
\vfill 
\centerline{\epsfbox{../images/fig3-2.eps}} 
\vskip3mm 
\centerline{\eightpoint figure 3.2.} 
\vfill } 
\endinsert 

Les r\'eponses d\'ependent du mod\`ele: dans le cas 1 elle vaut $1\over  
3$,  dans  le cas 2 elle vaut $1\over 2$, dans le cas 3 elle vaut $1\over 
4$. La diff\'erence provient du fait que l'espace $\Omega$ de toutes les 
cordes possibles n'est pas le m\^eme dans les trois cas. Les figures 3.1, 
3.2, et 3.3 montrent quel est l'espace $\Omega$ respectivement dans les 
cas 1, 2 et 3 (on n'a repr\'esent\'e que la partie radiale). Dans son
livre, Joseph Bertrand pr\'esentait cet exemple dans un but de critique, 
comme un probl\`eme  mal pos\'e. Il avait raison, puisque l'\'enonc\'e 
ci-dessus ne permet pas \`a lui seul  de d\'eterminer lequel des trois 
mod\`eles est le bon.    
\medskip 
Une analyse d\'etaill\'ee du ph\'enom\`ene permet de  
comprendre la diff\'erence entre les trois figures. Introduisons les 
coordonn\'ees suivantes:  
$$\eqalignno{ 
&u \quad \hbox{l'angle polaire de la m\'ediatrice de la corde}\cr 
&v \quad \hbox{le demi-angle d'ouverture de la corde}\cr 
\noalign{\smallskip} 
&d\quad\hbox{la distance de la corde au centre du cercle\hskip5mm et} 
\quad t = {\up{d}\over R} \cr 
\noalign{\smallskip} 
&x,y \quad \hbox{les coordonn\'ees cart\'esiennes du milieu de la 
corde.}\cr}$$  
On peut dire que dans le mod\`ele 1 on choisit les nombres $u$ et $v$ au 
hasard, c'est \`a dire que $u$ est pris au hasard dans l'intervalle 
$[0,2\pi  ]$ et $v$ au hasard dans l'intervalle $[0,{\pi \over 2}]$ 
(en pr\'esentant ce mod\`ele, nous avons dit qu'on choisissait au hasard 
les deux extr\'emit\'es $u_1$ et $u_2$ de la corde, et non les 
param\`etres $u$ et $v$ introduits ci-dessus, mais nous verrons que  
cela est \'equivalent). L'expression {\it au hasard} signifie que dans ces 
intervalles il n'y a pas de r\'egion privil\'egi\'ee.  Si on discr\'etise
le probl\`eme comme dans la figure 2,  cela conduit  \`a diviser ces 
intervalles en parties \'egales.   
\smallskip 
Par contre dans le mod\`ele 2 ce sont les nombres $u$ et $t$ qui sont 
choisis au hasard dans les intervalles $[0,2\pi ]$ et $[0,1]$. 
\smallskip 
Enfin, dans le mod\`ele 3 ce sont les nombres $x,y$ qui sont choisis au 
hasard dans le domaine $x^2 + y^2 < R^2$. 
\midinsert 
\vskip3pt
\centerline{\epsfbox{../images/fig3-3.eps}} 
\vskip3mm 
\centerline{\eightpoint figure 3.3} 
\vskip3mm 
\endinsert 
\medskip 
Ces choix {\it au hasard} se traduisent math\'ematiquement par le fait  
que  les probabilit\'es sont les rapports des aires: 
\medskip 
\noindent {\bf cas 1.}  
$$P_1 = {\int_0^{2\pi}\int_{\pi \over 3}^{\pi \over 2} 
\strup{8} \; dv \, du \over 
\int_0^{2\pi}\int_{0}^{\pi \over 2} 
\sdown{15}  \; dv \, du  }$$ 
{\bf cas 2.}  
$$P_2 = {\int_0^{2\pi}\int_{0}^{1 \over 2} 
\strup{6.5}  \; dt \, du \over 
\int_0^{2\pi}\int_{0}^{1} 
\sdown{13} \; dt \, du  }$$ 
{\bf cas 3.}  
$$P_3 = {\int\!\!\int_{x^2 + y^2 < R^2/4} 
\strup{6}  \; dx \, dy \over 
\int\!\!\int_{x^2 + y^2 < R^2}  
\sdown{11} \; dx \, dy  }$$ 
\medskip 
Entre ces diff\'erents param\`etres on a les relations suivantes: 
$$\eqalignno{&d = R\cos v \cr   &x = d\cos u \cr   &y = d\sin u \cr}$$ 
de sorte que si on les exprime tous en fonction de $u,t$ cela donne 
$$\eqalignno{ 
&v = \arccos t \cr  &x = R\; t \cos u \cr  &y = R\; t \sin u \cr}$$  
Si maintenant on rapporte les cas 1 et 3 aux coordonn\'ees $u,t$, 
c'est-\`a-dire si on fait le changement de variable correspondant dans  
les int\'egrales, les probabilit\'es deviennent respectivement:  
$$P_1 =  {\int_{\pi \over 3}^{\pi \over 2} 
\strup{8} \; dv \over 
\int_{0}^{\pi \over 2} \sdown{15}  \; dv }  
= {\int_{0}^{1 \over 2} \strup{8}
{1 \over \sdown{9} \sqrt{1-t^2}}  
\; dt \over \int_{0}^{1}\sdown{13}
{1 \over \sdown{8.5} \sqrt{1-t^2} } \; dt }\hskip9mm 
P_2 =  {\int_{0}^{1 \over 2} \strup{6.5} \; dt \over 
\int_{0}^{1} \sdown{13} \; dt }\hskip9mm 
P_3 =  {\int_{0}^{1 \over 2} t \strup{6} \; dt \over 
\int_{0}^{1}t \sdown{13}\; dt }$$ 
(Les int\'egrales en $du$ se factorisent puis se simplifient, ce qui  
r\'esulte simplement de la sym\'etrie de rotation). 
\medskip 
On voit ainsi que le changement de coordonn\'ees a introduit des  
densit\'es, $1/\sqrt{1-t^2}$ dans le cas 1 et $t$ dans le cas deux, qui 
sont plus grandes pour $t$ proche de 1 que pour $t$ proche de 0. Cela 
signifie que le choix au hasard du nombre $v$ dans l'intervalle 
$[0,{\pi\over 2} ]$  ou le choix au hasard d'un point $x,y$ sur le cercle ne 
correspondent plus \`a un choix au hasard du nombre $t$. Si on 
choisit au hasard $v$ dans  l'intervalle $[0,{\pi\over 2} ]$, de sorte 
qu'aucune r\'egion de cet intervalle ne soit favoris\'ee, alors, pour les 
valeurs correspondantes de $t$, on favorise la r\'egion proche de 1 par 
rapport \`a la r\'egion proche de 0.  
\medskip 
Cette diff\'erence de r\'epartition est due \`a la non lin\'earit\'e du  
changement de variable. Elle appara\^\i t clairement sur les figures 3.1 
et 3.3 : les cordes proches du bord (ce  qui correspond \`a $t$ proche de 1) 
sont favoris\'ees par rapport \`a la figure 3.2.  Par contre le passage des 
coordonn\'ees $u_1,u_2$ (abscisses angulaires des extr\'emit\'es) aux 
coordonn\'ees $u,v$ est lin\'eaire: $u_1 = u - v/2$ et $u_2 = u + v/2$;  
c'est pourquoi dans le cas {\bf 1} il est indiff\'erent  de choisir $u_1, 
u_2$ au hasard ou $u,v$. 
\medskip 
C'est ce qui explique que les probabilit\'es pour que $t < {1/ 2}$ soient 
plus petites dans les cas 1 et 3 (resp. ${1/ 3}$ et ${1/ 4}$) que 
dans le cas 2 (${1/ 2}$). Joseph Bertrand attribue l'ambigu\"\i t\'e au 
continu; il est vrai que lorsqu'on discr\'etise, comme sur les figures $3$, 
on voit bien la diff\'erence, qui ne se verrait pas dans le continu.  
Cependant \'Emile Borel\ftn{1}{\'Emile Borel {\it Le hasard} (Librairie 
F\'elix Alcan, {\oldstyle 1914}), page 84} a critiqu\'e cette critique de 
Bertrand en montrant que l'ambigu\"\i t\'e ne vient pas du continu, mais  
du fait que le niveau o\`u intervient le hasard n'est pas pr\'ecis\'e. 
\medskip 
Ainsi {\it il n'est pas \'equivalent} de choisir au hasard les deux 
extr\'emit\'es de la corde (cas 1), ou de choisir au hasard son milieu 
$(x,y)$ (cas 3), ou  encore de choisir au hasard sa direction et sa distance 
au centre (cas 2). Mais supposons qu'on fasse l'exp\'erience concr\`ete que 
voici: on jette, \`a partir d'une ouverture pratiqu\'ee dans le plafond, des 
millions de f\'etus de paille sur le  plancher. Afin d'assurer une bonne 
dispersion des brins, on place un puissant ventilateur pr\`es de 
l'ouverture dans le plafond. Sur le sol, on a trac\'e un grand cercle \`a la 
craie. Pour chaque brin de paille, on consid\`ere la droite qui le prolonge; 
dans beaucoup de cas, la droite ne coupera pas le cercle, mais on ne 
compte pas ces brins. Pour chaque brin dont le prolongement coupe le 
cercle, on consid\`ere la corde que ce prolongement d\'ecoupe sur le 
cercle.   
\medskip 
Comment se distribuent ces cordes? 
\medskip 
On a ici affaire \`a une situation pratique: ce n'est plus {\it nous} qui 
d\'ecidons d'une mani\`ere de choisir les cordes, les cordes se  
distribuent toutes seules. O\`u intervient alors le hasard? L'\'etude 
des trois cas a montr\'e qu'il y a diff\'erentes  
distributions possibles pour les cordes, mais dans une exp\'erience 
r\'eelle il y aura forc\'ement une certaine distribution particuli\`ere. 
\medskip 
Dans cette exp\'erience, c'est le mod\`ele $N^\circ 2$ qui est le bon.  
On peut s'en convaincre sans faire l'exp\'erience, et sans calculer le 
mouvement des brins. Pour cela, il suffit de remarquer que la 
distribution des brins sur le plancher ne peut pas d\'ependre de la 
pr\'esence du cercle: il serait absurde que les brins se distribuent 
diff\'eremment selon qu'on a trac\'e un cercle ou qu'on n'en a pas trac\'e 
(sauf si par exemple le cercle n'\'etait pas mat\'erialis\'e par de la  
craie, mais par de petits aimants, qui attireraient les brins de paille 
suppos\'es aimant\'es, ou tout autre artifice de ce type). De m\^eme,  
les brins seront distribu\'es \`a peu pr\`es de la m\^eme fa\c con dans  
toutes les r\'egions du plancher (sauf bien s\^ur si on s'\'ecarte trop du 
trou dans le plafond). Par cons\'equent, la distribution des cordes que  
le prolongement des brins d\'ecoupe sur un cercle sera la m\^eme quelle 
que soit la r\'egion o\`u on trace le cercle. Il est donc impossible que se 
produisent les situations 1 ou 3, dans lesquelles les cordes sont plus 
denses pr\`es du bord du cercle: si tel \'etait le cas pour un cercle 
particulier, ce ne pourrait plus l'\^etre pour un autre cercle trac\'e un 
m\`etre plus loin. Il faudrait que les brins ``sentent'' la pr\'esence du 
cercle et tombent d\'elib\'er\'ement d'une mani\`ere qui favorise le 
bord du cercle. Seul le mod\`ele $N^\circ 2$ correspond \`a une 
distribution des droites qui soit ind\'ependante de l'existence du cercle. 
 
\vskip7mm plus6mm minus5mm
 
{\bf I. 3. La signification de l'\'equiprobabilit\'e.} 
\medskip  
Jusqu'ici nous avons souvent \'evoqu\'e le ``pur hasard'', qui ferait des 
``choix \'equiprobables''. Ces consid\'erations sur le {\it pur hasard} ne 
sont pas n\'ecessaires pour un expos\'e axiomatique et d\'eductif, qui 
suppose un mod\`ele math\'ematique (c'est-\`a-dire un espace des 
\'epreuves $\Omega$) d\'ej\`a donn\'e, et r\'ecuse toute question sur 
l'origine ou sur le sens de ce mod\`ele. Elles le deviennent cependant si 
on veut ma\^\i triser l'art de la mod\'elisation et \^etre capable devant 
une situation concr\`ete de construire soi-m\^eme le mod\`ele ad\'equat. 
\medskip 
L'affirmation que {\it le pur hasard} fait des choix \'equiprobables 
n'est pas un postulat m\'etaphysique. C'est un {\it principe}, qui est 
au Calcul des probabilit\'es ce que le principe de relativit\'e de  
Galil\'ee est \`a la M\'ecanique. Toutefois il peut laisser le 
sentiment qu'il s'agit de m\'etaphysique \`a cause de l'invocation du 
``pur hasard''. Ce n'est pas le mot hasard qui insulte ainsi le 
positivisme, mais plut\^ot l'\'epith\`ete qui l'accompagne et qui 
\'evoque ``les pures id\'ees'', ``la Raison pure'', etc. C'est pourquoi 
l'explication qui va suivre est n\'ecessaire. 
\medskip  
Afin que la discussion soit bien concr\`ete, voyons comment les 
choses se passent dans l'exemple  simple des  cordes sur un cercle. 
Nous avons vu \`a ce propos qu'il y avait diff\'erentes mani\`eres de 
choisir une corde sur un cercle. Imaginez maintenant un physicien 
qui observe pendant longtemps l'apparition al\'eatoire de cordes sur 
un cercle, mais qui ignore ``comment le hasard a choisi les cordes''  
(vous pouvez imaginer que c'est un programme \'ecrit par 
vous-m\^eme qui fait appara\^\i tre, au rythme de une par seconde, 
des cordes sur l'\'ecran de l'ordinateur; pour le choix de chaque corde 
le programme fait appel deux fois --~puisqu'il faut deux 
coordonn\'ees~-- \`a une fonction {\bf random}; le physicien qui 
observe le r\'esultat ignore comment sont \'ecrits les programmes). 
Il accumule les observations et, sachant qu'il y a la sym\'etrie de 
rotation, il ne note la position des cordes que si, \`a $\varepsilon$ 
pr\`es, elles sont parall\`eles  \`a une direction fix\'ee; il obtient 
ainsi, disons, la figure 3.1. Ce physicien peut {\it en d\'eduire} que les 
cordes ont \'et\'e distribu\'ees en laissant le hasard choisir les deux 
extr\'emit\'es, ind\'ependamment l'une de l'autre, sur le cercle. Dans 
le graphique qu'il obtient (semblable, donc, \`a la figure 3.1), les 
cordes ne sont pas distribu\'ees uniform\'ement le long des 
ordonn\'ees, c'est-\`a-dire le long du rayon. ``Le hasard pur'' n'est  
donc pas intervenu \`a ce niveau (celui de la distribution le long du 
rayon, qui est aussi le choix de la coordonn\'ee $t$) mais  
au niveau du choix des extr\'emit\'es.  (si les cordes sont trac\'ees  
par votre programme, le physicien peut ainsi savoir que la fonction 
{\bf random} choisit les deux extr\'emit\'es, comme dans le cas 1, et 
non les coordonn\'ees $x,y$ du centre ou les coordonn\'ees $t,u$). Le 
physicien peut donc, par l'observation, d\'epartager les diff\'erentes 
mani\`eres d'agir du hasard.  Cela montre qu'il s'agit d'une 
information pratique et observable, et non d'une hypoth\`ese 
m\'etaphysique. 
\medskip 
Lorsqu'on dit  ``le hasard pur est intervenu \`a ce niveau'' ou ``le  
hasard pur choisit les deux extr\'emit\'es'', cette mani\`ere imag\'ee 
de s'exprimer peut sugg\'erer qu'il  y aurait une divinit\'e 
dou\'ee de volont\'e et veillant, par une vigilance de tous les 
instants, \`a ne favoriser aucun de ses choix par rapport aux autres, 
et que par cons\'equent le probl\`eme pos\'e ci-dessus (trouver le 
niveau o\`u le hasard fait son choix), consisterait \`a  
p\'en\'etrer les desseins de cette divinit\'e. Il ne s'agit que d'une  
fa\c con de parler, comme par exemple lorsqu'on dit ``la Nature 
ob\'eit \`a des lois''.   
\medskip 
On ne postule aucune divinit\'e, et on peut au contraire postuler que le 
hasard provient de causes toutes rationnelles, qu'il  n'est que l'effet 
d'un enchev\^etrement d'innombrables influences qui se contrarient 
mutuellement, cr\'eant ainsi l'illusion appel\'ee hasard. Mais si ces 
causes, rationnelles ou non, favorisent certains r\'esultats au  
d\'etriment d'autres, on dira que ce favoritisme a une explication 
causale  et non qu'il est un effet du hasard.    
\medskip 
Pour illustrer cela, revenons encore \`a l'exemple des cordes sur un 
cercle. Dans ce probl\`eme il n'y avait pas de ph\'enom\`ene physique; 
il s'agissait de g\'eom\'etrie abstraite et ``les choix du hasard'' 
portaient sur des nombres. Ainsi ``le hasard'' devait choisir, soit deux 
nombres $\alpha\, , \,\beta$ dans l'intervalle $[ 0\, , \, 2\pi [$, soit 
deux nombres $u\, , \, t$ dans (respectivement) les intervalles $[ 0\, 
, \, 2\pi [$ et $[ 0\, , \, 1]$, soit un point dans le disque. Pour tout 
calculer, nous sommes partis du seul postulat que quand ``le pur 
hasard'' choisit un nombre dans un intervalle, il ne favorise aucune 
r\'egion de l'intervalle: par exemple les nombres de $[ 0\, , \, 1]$ 
n'ont pas plus de chances d'\^etre choisis s'ils sont proches de 0,  
que s'ils sont proches de 1. Ce qu'on exprime math\'ematiquement en 
disant que la probabilit\'e pour que le nombre choisi dans  $[ 0\, , \, 
1]$ se trouve ``par hasard'' entre $a$ et $b$ est \'egale \`a $b-a$. Il  
n'y a rien de plus dans cette expression math\'ematique que l'intuition 
premi\`ere, que tout le monde comprend a priori. De m\^eme pour un 
point ``pris au hasard'' dans le disque: la probabilit\'e pour que ``par 
hasard'' il se trouve dans la r\'egion $A$ du disque est le rapport  
$${\hbox{aire de $A$} \over\hbox{aire du disque} }$$ 
Or l'\'etude d\'etaill\'ee des diff\'erentes formes de choix a montr\'e  
que si par exemple les deux extr\'emit\'es $\alpha\, , \,\beta$ de la 
corde  sont choisies dans l'intervalle $[ 0\, , \, 2\pi [$, sans favoriser 
aucune r\'egion de cet intervalle, alors {\it ipso facto} la coordonn\'ee 
$t$ (distance au centre rapport\'ee au rayon) sera choisie --~dans 
l'intervalle $[ 0\, , \, 1]$~-- de mani\`ere que les nombres proches 
de $1$ soient favoris\'es par rapport \`a ceux qui sont proches de 0. 
Math\'ematiquement, la probabilit\'e pour que $t$ se trouve ``par 
hasard'' entre $a$ et $b$ n'est plus $b-a$, mais  
$$\int_a^b {dt \over \sdown{14}\sqrt{1-t^2}}$$ 
Le physicien que nous \'evoquions plus haut, qui observe ce 
favoritisme, ne dira pas ({\it personne} ne le dira) ``le hasard 
pr\'ef\`ere les cordes proches du bord que les cordes proches du  
centre''. Il dira ``il y a un choix al\'eatoire, {\it biais\'e} par un 
ph\'enom\`ene d\'eterministe''. 
\medskip 
Attention! Il ne s'agit pas d'une querelle de mots. Si on change les mots, 
on ne supprime pas l'insatisfaction du physicien, qui veut savoir 
pourquoi les valeurs de $t$ ont plus de chances d'\^etre proches de 1 que 
de 0. On pourrait {\it convenir} --~dans le sens o\`u Henri Poincar\'e\ftn 
{2}{Henri Poincar\'e {\it La valeur de la science}, chapitre II ``La mesure 
du temps''.} disait que la mesure du temps est une convention et 
qu'aucune n'est plus objective qu'une autre~-- que le hasard choisit $t$ 
de fa\c con non \'equiprobable, avec la densit\'e $1 \bigl/ \sqrt{1-t^2}$. 
Mais cela ne nous \'epargnerait pas les questions, cela ne ferait que nous 
rendre l'ensemble des calculs effectu\'es pour  les trois cas plus 
compliqu\'es, car nous devrions tra\^\i ner partout la r\'epercussion de 
cette densit\'e, de m\^eme que si on voulait absolument mesurer le 
temps selon une \'echelle diff\'erente, et  non lin\'eaire par rapport \`a 
l'usuelle, on devrait tra\^\i ner la r\'epercussion de ce changement de 
variable dans toutes les \'equations de la Physique. Poincar\'e a certes 
\'ecrit que la mesure du temps est  une convention et qu'aucune n'est 
plus objective qu'une autre, mais la phrase qui est \'ecrite 
imm\'ediatement avant dans son texte est ``le temps doit \^etre d\'efini 
de telle fa\c con  que les \'equations de la M\'ecanique soient aussi 
simples que possible''.  
\medskip 
C'est de l\`a que d\'erive  la notion de rep\`ere galil\'een, et c'est  
pourquoi ``le niveau o\`u les choix du hasard sont \'equiprobables'' joue en 
Calcul des probabilit\'es un r\^ole analogue \`a celui du rep\`ere galil\'een 
en M\'ecanique. Lorsque le physicien observe les cordes, distribu\'ees le 
long du rayon selon la loi $1 \bigl/\sqrt{1-t^2}$, il peut {\it en d\'eduire} 
que le niveau o\`u les choix \'etaient \'equiprobables \'etait celui du choix 
des extr\'emit\'es de la corde, de m\^eme qu'un physicien enferm\'e dans 
une capsule loin de toute source de gravitation, qui observe que les billes 
qu'il l\^ache dans sa cabine suivent un mouvement uniform\'ement 
acc\'el\'er\'e, peut en d\'eduire  que sa capsule subit de l'ext\'erieur un 
mouvement uniform\'ement  acc\'el\'er\'e en sens oppos\'e \`a celui des 
billes.  
\medskip
Afin de ne pas trop surcharger ce chapitre, la discussion sur le 
``conventionnalisme'' de Poincar\'e, ses propres commentaires sur le 
paradoxe de Bertrand, et la parent\'e avec le principe de relativit\'e de
Galil\'ee ne sont ici que r\'esum\'es succintement. Le sujet est davantage 
d\'evelopp\'e dans un article, ``{\it Le paradoxe de Bertrand}'', qui 
peut \^etre trouv\'e \`a l'adresse U.R.L.:
\smallskip
\centerline{\tt http: //moire4.u-strasbg.fr/hist/bertrand.htm}
\medskip  
Tout cela est bien beau, direz-vous, mais ne nous dit pas comment on va 
trouver le niveau o\`u agit le hasard pur, c'est-\`a-dire le niveau o\`u on 
a affaire \`a des choix \'equiprobables. En outre, il pourrait y avoir des 
niveaux diff\'erents, mais \'equivalents, et on ne pourrait pas savoir 
dans lequel des deux le hasard agirait ``vraiment''.   
\medskip 
Il se trouve qu'il n'existe pas de m\'ethode pour cela. Quant \`a la 
question de trancher entre deux niveaux \'equivalents, elle par contre 
est m\'etaphysique; elle est exactement aussi m\'etaphysique que la 
question de savoir quel rep\`ere galil\'een est  ``vraiment'' immobile. 
Dans le cas des cordes sur un cercle, la recherche  du niveau o\`u agit 
le pur hasard a \'et\'e facile, car nous connaissions d\'ej\`a trois 
diff\'erentes mani\`eres  de distribuer les cordes et il a suffi d'en 
reconna\^\i tre une parmi les trois. Dans les probl\`emes d'urnes, de 
boules, de marches al\'eatoires, etc. (autrement dit dans les 
probl\`emes scolaires de Calcul des probabilit\'es) cette recherche 
du ``niveau o\`u agit le hasard pur'' rel\`eve de la {\it mod\'elisation}; 
le  ``niveau o\`u agit le hasard pur'' est tr\`es souvent sugg\'er\'e, ou 
m\^eme tout simplement donn\'e, par  l'\'enonc\'e du probl\`eme.  
\medskip 
La mod\'elisation a toujours fait peur aux \'etudiants car elle demande 
de l'imagination, de l'astuce, et sa r\'eussite n'est pas pr\'evisible ou 
chiffrable; tandis que si la mod\'elisation est d\'ej\`a faite, ou 
sugg\'er\'ee de fa\c con \'evidente, il n'y a plus qu'\`a calculer en 
appliquant quelques th\'eor\`emes ou formules vues dans le cours. 
Tout cela est classique.  
\medskip  
Mais il y a plus grave. Apr\`es tout, lorsqu'un \'etudiant n'arrive pas 
\`a trouver le  mod\`ele correct, c'est parce qu'il n'est pas encore 
habitu\'e  aux poncifs  des probl\`emes scolaires, et dans ce cas  
l'enseignant conna{\^\i}t la bonne r\'eponse, qu'il avait lui-m\^eme  
apprise quand il \'etait \'etudiant, par mim\'etisme plus que par 
imagination. Par contre il y a des probl\`emes pos\'es par la Physique  
pour lesquels {\it personne} ne sait au d\'epart \`a quel niveau agit le 
hasard pur, et le trouver est une affaire de g\'enie (ce qui justifie 
tout-\`a-fait, d'ailleurs, le malaise que les \'etudiants \'eprouvent 
face \`a la mod\'elisation). L'exemple historique le plus spectaculaire 
est \'evidemment la d\'ecouverte de la M\'ecanique quantique: \`a 
partir de faits exp\'erimentaux (rayonnement du corps noir, 
diffraction des \'electrons par des cristaux, etc.) il a fallu trouver 
par la recherche le niveau auquel le hasard pur agissait; on ne 
l'a toujours pas vraiment trouv\'e. Il serait donc absolument na\"\i f 
de vouloir disposer d'une m\'ethode g\'en\'erale et syst\'ematique 
pour le trouver. 
\medskip 
Il semble ressortir de cette discussion que l'\'equiprobabilit\'e des 
choix du hasard, tout comme pour la mesure du temps ou les rep\`eres  
galil\'eens, est en quelque sorte {\it la d\'efinition m\^eme} du 
hasard. On pourrait penser que le hasard peut se d\'efinir autrement, 
ind\'ependamment de  la mani\`ere dont  s'op\`erent ses choix, \`a 
partir de consid\'erations plus fondamentales, de telle sorte que 
l'\'equiprobabilit\'e serait une propri\'et\'e qui se d\'eduirait de 
la d\'efinition, et non un postulat a priori. Il se trouve que pour le 
temps ou les rep\`eres galil\'eens, la Physique ne fournit pas de 
connaissance plus profonde que ce conventionnalisme (c'est le nom 
donn\'e \`a la position de Poincar\'e cit\'ee pr\'ec\'edemment), du 
moins jusqu'\`a nos jours. En revanche, l'Histoire de la Physique 
contient beaucoup d'exemples o\`u un progr\`es dans la connaissance 
apporte une explication plus approfondie d'un ph\'enom\`ene dont la 
description avait \'et\'e jusque l\`a purement conventionnaliste. 
Ainsi nous comprenons aujourd'hui {\it pourquoi} le mouvement 
apparent des plan\`etes dans le firmament (c'est-\`a-dire par rapport 
\`a la {\it sph\`ere des fixes}) est comme il est; par exemple nous 
comprenons pourquoi une plan\`ete avance (dans le m\^eme sens que 
les \'etoiles), puis ralentit, puis revient en arri\`ere, selon une 
trajectoire apparemment erratique: l'explication en est que nous la 
voyons en perspective \`a partir de la Terre qui est elle-m\^eme en 
mouvement.  
\medskip 
Cette explication est l'{\it id\'ee} du syst\`eme de Copernic. Mais dans le 
syst\`eme de Ptol\'em\'ee, o\`u la Terre \'etait suppos\'ee immobile au 
centre, il fallait supposer plusieurs sph\`eres imbriqu\'ees les unes dans 
les autres et tournant autour d'axes diff\'erents, la plus \'eloign\'ee 
\'etant la sph\`ere des fixes qui portait les \'etoiles, les sph\`eres 
int\'erieures portant chacune une plan\`ete. Dans ce syst\`eme de 
Ptol\'em\'ee le mouvement erratique des plan\`etes \'etait certes 
expliqu\'e par les mouvements relatifs des diff\'erentes sph\`eres, mais 
il n'y avait pas {\it une} combinaison unique de sph\`eres compatible avec 
le mouvement observ\'e. De sorte que dans ce syst\`eme la question de 
savoir  laquelle de ces combinaisons \'etait ``r\'eellement'' en fonction 
dans le ciel \'etait une question m\'etaphysique; on ne pouvait r\'epondre 
que par le conventionnalisme, en disant que parmi toutes les 
combinaisons de sph\`eres produisant les mouvements plan\'etaires 
apparents observ\'es, il fallait choisir celle pour laquelle les calculs 
sont les plus simples possibles.  Dans l'astronomie mo\-derne, au 
contraire, il n'y a pas de choix conventionnel: le syst\`eme solaire est 
tel qu'il est, nous ne savons pas pourquoi il est pr\'ecis\'ement 
comme cela plut\^ot qu'autrement, mais nous ne pouvons pas 
d\'ecider de d\'ecrire autrement la trajectoire de Mars dans le seul 
but de simplifier les calculs. Si  nous n'avons plus cette libert\'e, 
c'est parce que nous en savons trop: les  astronomes du Moyen-Age 
consid\'eraient que seul le mouvement apparent des plan\`etes dans 
le ciel (donc sa projection radiale sur la vo\^ute c\'eleste) \'etait une 
connaissance objective.  Aujourd'hui leur mouvement dans les {\it 
trois} dimensions est soumis \`a la connaissance objective, ce qui 
d\'etruit toute possibilit\'e de conventionnalisme \`a ce niveau.  
L'Histoire de la Physique regorge d'autres exemples semblables.  
\medskip 
C'est pourquoi la question pos\'ee plus haut \`a propos du hasard 
m\'erite d'\^etre approfondie. Le hasard peut-il \^etre con\c cu  
comme objectif, de sorte que l'\'equiprobabilit\'e de ses choix soit 
une n\'ecessit\'e, ou au contraire, \`a l'instar du temps et de l'espace, 
est-il --~du moins jusqu'\`a nouvel ordre~-- impensable autrement 
que comme convention? 
\medskip 
Il se trouve que, selon les pr\'esuppos\'es, l'une ou l'autre r\'eponse est 
possible. C'est ce que nous allons examiner dans les deux sections 
suivantes. 
 
\vskip7mm plus6mm minus5mm
 
{\bf I. 4. Hasard et d\'eterminisme math\'ematiques.} 
\medskip
{\cit Tous les \'ev\'enements, ceux m\^emes qui par leur petitesse semblent 
ne pas tenir aux grandes lois de la nature, en sont une suite aussi
n\'ecessaire que les r\'evolutions du soleil. Dans l'ignorance des liens 
qui les unissent au syst\`eme entier de l'univers, on les a fait d\'ependre 
des causes finales, ou du hasard, suivant qu'ils arrivaient et se 
succ\'edaient avec r\'egularit\'e, ou sans ordre apparent; mais ces causes
imaginaires ont \'et\'e successivement recul\'ees avec les bornes de nos
connaissances, et disparaissent enti\`erement devant la saine philosophie, 
qui ne voit en elles que l'expression de l'ignorance o\`u nous sommes des
v\'eritables causes. \par}
\smallskip
\line{\hfill \vbox{\hsize=66mm \eightpoint 
\centerline{Pierre-Simon Laplace}
\centerline{\sl Th\'eorie analytique des probabilit\'es ({\oldstyle 1819}).}}}
\medskip

Ce passage c\'el\`ebre est le manifeste du d\'eterminisme: rien n'est  
al\'eatoire,  tout est rigoureusement d\'etermin\'e, mais souvent le
d\'eveloppement d\'eter\-mi\-niste d'un processus est tellement chaotique 
que nous ne pouvons rien calculer du processus exact, que seules 
des probabilit\'es peuvent \^etre calcul\'ees. La pr\'esente section et 
la suivante sont consacr\'es \`a cette explication du hasard. L'affirmation 
de Laplace n'a jamais \'et\'e d\'ementie depuis, puisque l\`a o\`u elle
pourrait \'eventuellement l'\^etre nous sommes ignorants: c'est ce 
qu'on appelle un postulat m\'etaphysique (il ne peut \^etre tranch\'e par
l'exp\'erience). Avec la 
M\'ecanique quantique nous {\it ne savons pas} d'o\`u provient le 
hasard qu'elle postule. Mais par exemple le hasard dans le jeu de la 
roulette ob\'eit enti\`erement \`a la conception de Laplace. La question 
de savoir si le hasard postul\'e par la M\'ecanique quantique peut 
s'expliquer \`a la mani\`ere de Laplace est ouverte, mais il faut dire que 
tant que la valeur d'une telle explication ne peut pas \^etre tranch\'ee 
par l'exp\'erience, elle reste sp\'eculative. Peut-\^etre une telle 
explication, m\^eme sans \^etre tranch\'ee par l'exp\'erience, 
pourrait-elle du moins r\'eduire le myst\`ere de ce hasard dont nul ne 
comprend la cause; mais pourquoi un d\'eterminisme serait-il moins 
myst\'erieux que le hasard?  
\medskip 
Aujourd'hui on appelle {\it chaos d\'eterministe} le m\'ecanisme par 
lequel, comme l'entendait Laplace, le d\'eterminisme se transforme en 
hasard. Mais il faut aussit\^ot ajouter que le d\'eterminisme aussi peut 
\^etre l'effet du hasard, par l'interm\'ediaire de la loi des grands 
nombres (nous reviendrons l\`a-dessus aux chapitres suivants, 
notamment \`a la fin du chapitre {\bf II}, ainsi qu'aux chapitres {\bf VI}  
et {\bf VII}). De sorte que, du d\'eterminisme ou du hasard, chacun peut 
\^etre la cause premi\`ere de l'autre. 
\medskip 
Comme nous l'avons toujours fait jusqu'ici (et, rassurez-vous, 
nous continuerons) nous allons rendre la discussion concr\`ete par 
l'\'etude de quelques exemples simples.  
\medskip 
Voici un premier exemple extr\^emement simple de chaos 
d\'eterministe: on prend un nombre irrationnel, disons $e \simeq 
2.718\,  281\, 828 \ldots$ pour fixer les id\'ees. Puis on consid\`ere  
la suite de nombres $u_n = $ la partie d\'ecimale de $n \cdot e$ 
(c'est-\`a-dire qu'on  ne retient du nombre $n \cdot e$ que les  
chiffres apr\`es la virgule). Ainsi les nombres $u_n$ sont tous 
compris entre 0 et 1. Pour donner une id\'ee, voici les cinq premiers 
termes de cette suite:  
$$\eqalign{ 
u_1 &= 0.718\, 281\, 828 \ldots \cr 
u_2 &= 0.436\, 563\, 656 \ldots \cr 
u_3 &= 0.154\, 845\, 485 \ldots \cr 
u_4 &= 0.873\, 127\, 313 \ldots \cr 
u_5 &= 0.591\, 409\, 142 \ldots \cr }$$ 
Si on calcule un tr\`es grand nombre de termes de cette suite, disons 
un milliard, on obtiendra donc un gros \'echantillon de nombres  
compris entre 0 et 1.  
\medskip 
Il se trouve que cette suite de nombres a apparemment les m\^emes  
propri\'et\'es qu'une suite tir\'ee au hasard; d'apr\`es le pr\'ejug\'e que 
nous avons sur le hasard, \`a savoir qu'il ne favorise pas une r\'egion 
particuli\`ere de l'intervalle, $10^9$ nombres ``pris au hasard'' se  
r\'epartissent uniform\'ement dans l'intervalle: dans n'importe quel 
sous-intervalle  $[a,b] \subset [0,1[$, il y en aura environ $(b-a)\cdot 
10^9$. Mais il en va de m\^eme pour la suite d\'eterministe $u_n$: ces 
nombres se r\'epartissent uniform\'ement dans l'intervalle. Si, en 
\'evitant soigneusement de r\'ev\'eler comment la suite $u_n$ a 
\'et\'e engendr\'ee, on en confiait le r\'esultat \`a un statisticien 
pour qu'il le soumette aux tests usuels de la statistique, il 
serait impossible \`a celui-ci de d\'etecter un \'ecart par rapport aux 
lois du hasard; le seul \'ecart,  infime, qui pourrait \^etre observ\'e, 
diminuerait en augmentant la taille de l'\'echantillon (nous verrons 
au chapitre {\bf XI} comment on peut effectuer de tels tests). Dans 
l'exemple des cordes sur un cercle, si au lieu de confier le choix des 
coordonn\'ees au hasard, on avait tir\'e les nombres selon la suite 
$u_n$ (ou $2\pi u_n$ pour les nombres devant \^etre pris entre $0$ et 
$2\pi$), tout se serait pass\'e exactement de la m\^eme fa\c con. 
Cette d\'efinition math\'ematique du hasard vient d'\'Emile 
Borel\ftn{3}{\'Emile Borel {\it Les probabilit\'es d\'enombrables et leurs 
applications arithm\'etiques} (Rendiconti del Circolo matematico di 
Palermo, vol. {\bf 27} ({\oldstyle 1909}, pp. 247 -- 270.} 
\medskip 
Plusieurs fois nous avons \'evoqu\'e l'emploi d'une fonction {\bf 
random} pour simuler le hasard, ce qui fait \'evidemment plus 
s\'erieux que d'\'evoquer la Fortune.  Mais  les fonctions {\bf 
random} du commerce fonctionnent exactement sur le m\^eme 
principe que la suite $u_n$; tout au plus elles font appel \`a des 
algorithmes plus sophistiqu\'es afin de mieux brouiller les 
pistes\ftn 4{On peut cependant consid\'erer que plus l'algorithme est 
tortueux et cryptique, plus le hasard qui en r\'esulte est de bonne 
qualit\'e. Ce point de vue est \`a la base de la th\'eorie des suites 
al\'eatoires. Voir par exemple le livre de Donald Knuth {\it The Art  
of Computer Programming} Vol. 2: Seminumerical Algorithms 
(Addison-Wesley, {\oldstyle 1969}) ou celui de Jean-Paul Delahaye 
{\it Information, complexit\'e, et hasard} (\'Ed. Herm\`es, Paris, 
{\oldstyle 1994}).}. On va voir en effet que la suite  
$u_n$ ne brouille pas suffisamment bien les pistes.  
\medskip 
Supposons que, au lieu d'avoir pris le nombre $e \simeq 2.718281828$, 
nous ayons pris le nombre $0.5$. La suite des  $v_n = $ partie d\'ecimale 
de $n \cdot 0.5$  ne respecte pas la r\`egle de  ne favoriser aucune 
r\'egion, puisqu'elle  ne donne que $0$ ou $0.5$, et d\'efavorise donc les 
nombres situ\'es  dans $[0.1\, ,\, 0.4]$ ou dans  $[0.6\, , \, 0.9]$. Dans ce 
cas on ne peut plus parler de hasard. La diff\'erence entre les deux 
situations ne provient pas du d\'eterminisme, qui serait absent dans un 
cas et pr\'esent dans l'autre (et si m\^eme il y avait ind\'eterminisme, on 
ne serait pas plus avanc\'e, puisqu'on aurait juste remplac\'e le mot 
hasard par le mot ind\'eterminisme). Il y a d\'eterminisme dans les deux 
cas, mais dans le premier la r\`egle de l'uniformit\'e est 
respect\'ee, tandis que dans le second elle ne l'est pas. Toutefois, si le 
statisticien auquel on   a confi\'e la suite $u_n$ pour expertise, 
consid\`ere la suite $w_n = u_{n+1} - u_n$, il constatera que les nombres 
$w_n$ ont le m\^eme d\'efaut que les $v_n$, car ils sont tous \'egaux, 
soit \`a $e-2 \simeq 0.718\, 281\, 828$, soit \`a $e-3 \simeq -0.281\, 
718\, 171$.  
\medskip 
On ne peut donc pas vraiment admettre  que la suite $u_n$ 
soit au hasard car ``il fallait avoir fait expr\`es'' de choisir les $u_n$ 
pour que leurs diff\'erences soient toujours les m\^emes. On a donc en 
t\^ete le pr\'ejug\'e que le hasard ne peut pas avoir choisi des nombres  
que l'application d'un algorithme aurait \'egalement pu engendrer (nous 
avions d\'ej\`a le pr\'ejug\'e que le hasard ne favorise pas certains 
nombres au d\'etriment d'autres; voil\`a donc un autre pr\'ejug\'e). On 
pourrait alors donner du hasard la d\'efinition suivante: un ensemble de 
$N$ nombres r\'eels pris dans l'intervalle $[0 , 1]$ sont distribu\'es au 
hasard si leur liste ne peut \^etre engendr\'ee par aucun algorithme. 
L'ennui, c'est que (du moins pour un nombre fini $N$ de nombres) il y a 
toujours un algorithme, ne serait-ce que la donn\'ee directe de ces $N$ 
nombres. On peut corriger cette incoh\'erence en pr\'ecisant davantage:  
on dira que $N$ nombres r\'eels de l'intervalle $[0 , 1]$ sont distribu\'es 
au hasard si leur liste ne peut \^etre engendr\'ee par aucun algorithme  
{\it plus court que la simple liste de ces $N$ nombres}. Pour bien  
comprendre cela, imaginons qu'on veuille \'ecrire un programme qui   
\'ecrit les $N$ nombres dans un fichier de donn\'ees. Pour $N = 10^9$ et 
huit octets par nombre (il s'agit de flottants), ce fichier est tr\`es gros: 
$8$ gigaoctets. Mais le programme qui engendre la suite $u_n$ est tr\`es 
court, car l'algorithme est tr\`es simple:  
\def\program{\font\gras=cmbx10 \def\&{\cleartabs &} 
\cleartabs \def\<##1>{\hbox{\gras##1\/}} 
\def\\##1\\{\hbox{\it##1\/}} \catcode`\;=\active \def;{\unskip\kern 
2pt\string; \kern 3pt}}  
$$\vbox{\program  
\+\<var> \hskip3pt  &$i\;$ &:\hskip3pt\<integer> ;   \cr  
\+                 &$N\;$ &:\hskip3pt\<longinteger> ;   \cr  
\vskip6pt 
\+\<begin> \cr 
\+ $N := 1e{\scriptscriptstyle +} {\scriptstyle 9}$ ; \cr 
\+\<for> \&$i:=1$ \<to> $N$ \<do>  \cr 
\+ &\<begin> \cr 
\+ &\<writeln>$(i*\<exp>(1) - \<trunc>(i*\<exp>(1)))$ ;  \cr 
\+ &\<end> \cr 
\+\<end.> \cr }$$ 
Ce programme, m\^eme en code ex\'ecutable, va occuper au plus  
$1000$ octets de m\'emoire. Ce qui permet cette \'enorme \'economie  
par rapport \`a la {\it liste} des nombres, est la boucle qui n'est \'ecrite 
qu'une fois, bien que dans l'{\it ex\'ecution} du programme elle soit 
r\'ep\'et\'ee  $10^9$ fois. Si les $10^9$ nombres n'\'etaient li\'es entre  
eux par aucune r\'ecurrence, on ne pourrait pas utiliser de boucle; le 
programme devrait comporter autant d'entr\'ees ou d'initialisations que 
de nombres, et serait donc aussi long que la liste des nombres. 
\medskip 
Autrement dit la suite $u_n$ peut \^etre \'ecrite par un programme  
court car il suffit de conna\^\i tre $u_1 = e$ pour d\'eduire par 
r\'ecurrence tous les autres. Si les $10^9$ nombres n'\'etaient li\'es 
entre eux par aucune relation, il ne suffirait pas d'en conna\^\i tre un ou 
m\^eme quelques uns pour en d\'eduire les autres, de sorte qu'on ne 
pourrait pas trouver de programme {\it plus court} que la liste directe. 
\medskip 
Ce qui a permis \`a notre statisticien de d\'ecouvrir la tricherie, est la 
possibilit\'e de faire appara\^\i tre le d\'eterminisme de la suite par  
les soustractions. Mais si {\it le plus court} algorithme qui engendre la 
suite est horriblement long, celui-ci \'echappera forc\'ement aux 
investigations et aux tests. On peut donc mesurer le degr\'e de hasard 
ou --~inversement~-- le degr\'e de d\'eterminisme d'une suite de  
nombres \`a la longueur en $K\! o$ du plus court programme qui engendre 
la suite. Ainsi un degr\'e \'elev\'e de hasard signifie uniquement que la 
suite est engendr\'ee par un algorithme suffisamment tortueux pour 
pouvoir \'echapper \`a la surveillance des inspecteurs. Cette id\'ee  
d\'ecoule naturellement de celle de Borel mentionn\'ee plus haut, mais 
elle n'a  \'et\'e \'erig\'ee en syst\`eme que plus tard, et est le r\'esultat 
d'une lente maturation qui s'\'etale de 1900 \`a  1960 environ. Le lecteur 
int\'eress\'e pourra lire  le livre de Knuth ou celui de Delahaye 
(d\'ej\`a cit\'es en note 4) pour en savoir plus. 
\medskip
La conclusion ultime de toute cette lente maturation d'id\'ees 
sur ce  que seraient ``vraiment'' des nombres pris au hasard dans 
un intervalle, est qu'il n'y a pas de ``vrai'' hasard, il n'y a que des 
algorithmes d\'eterministes, mais suffisamment crypt\'es et tortueux 
pour que {\it en pratique} aucun expert ne puisse d\'ecouvrir le 
d\'eterminisme sous-jacent. C'est ainsi, en tous cas, que fonctionnent les 
fonctions {\bf random} des logiciels de calcul. 
\medskip 
Dans le domaine des choix de nombres, c'est-\`a-dire du hasard 
{\it math\'e\-ma\-ti\-que}, il n'y a pas d'autre forme de hasard que le 
cryptage du d\'eterminisme. La  boutade de Laplace est donc 
parfaitement juste.  
 
\vskip7mm plus6mm minus5mm
 
{\bf I. 5. Hasard et d\'eterminisme dans la Nature.} 
\medskip 
Se pourrait-il alors que dans la Nature, au moins, on rencontre du 
v\'eritable hasard, c'est-\`a-dire des processus pour lesquels, m\^eme si 
on pouvait les reproduire avec un algorithme d\'eterministe, du moins la 
Nature n'y aurait pas recouru. L\`a, le probl\`eme est plus d\'elicat, car 
nous n'avons de la Nature qu'une connaissance approch\'ee, ce qui a pour 
premi\`ere cons\'equence que si deux mod\`eles math\'ematiques 
diff\'erents  donnent des pr\'evisions qui ne diff\`erent num\'eriquement 
que d'une petite quantit\'e, il est impossible de dire que l'un est plus vrai 
que l'autre, au cas o\`u ces pr\'evisions seraient toutes deux conformes 
\`a l'observation. Il est donc absurde de conf\'erer une validit\'e absolue  
\`a une description de la M\'ecanique en termes d'\'equations 
diff\'erentielles. Par exemple, on peut dire que la bille de la roulette 
ob\'eit \`a la M\'ecanique classique et que son mouvement est d\'ecrit 
par des \'equations diff\'erentielles, mais cette description n'est qu'un 
mod\`ele math\'ematique approch\'e. Le mouvement de la bille est un 
exemple de chaos d\'eterministe, mais lorsqu'on dit cela il faut avoir  
en vue que le d\'eterminisme n'existe que dans le mod\`ele 
math\'ematique; dans le mouvement r\'eel de la bille, rien ne prouve  
qu'il ne pourrait \^etre d\'ecrit tout aussi bien par un autre mod\`ele 
math\'ematique, pr\'edisant le m\^eme mouvement pour la bille {\it dans 
les limites de pr\'ecision exp\'erimentales}, mais ind\'eterministe; le 
d\'eterminisme appro\-xi\-matif du mouvement observ\'e \'etant alors  
expliqu\'e comme un effet de la loi des grands nombres et non d'un 
d\'eterminisme absolu. Or si on parle de mod\`ele ind\'eterministe, on 
entend par l\`a que le mod\`ele postule un hasard {\it ontologique}, 
c'est-\`a-dire un hasard premier, suppos\'e exister sans explication plus 
profonde, comme celui qui est invoqu\'e par la M\'ecanique quantique, par 
exemple. Mais il suffirait d'ajouter \`a ce mod\`ele math\'ematique une 
annexe ``expliquant'' le hasard invoqu\'e comme l'application d'un 
algorithme du type indiqu\'e ci-dessus, pour  le transformer en mod\`ele 
d\'eterministe. De sorte que la question du d\'eterminisme ou de 
l'ind\'eterminisme {\it de la Nature} est tout simplement d\'epourvue de 
sens logique. Pour que cette question puisse acqu\'erir un sens logique,  
il faudrait avoir de la Nature une connaissance math\'ematique 
rigoureusement exacte et non approch\'ee. L'existence d'innombrables 
ex\'eg\`eses philosophiques sur le sujet ne suffit pas \`a transmuter  
cette question absurde en question sens\'ee (et d'autant moins que 
la plupart de ces ex\'eg\`eses philosophiques, surtout depuis 
Emmanuel Kant, s'efforcent pr\'ecis\'ement de montrer l'absurdit\'e 
de cette question). On pourra lire \`a ce sujet Richard P. Feynman 
{\it Lectures on Physics}, Tome {\bf 3}, {\bf II. 6.}).\looseness=-1 
\medskip
Poincar\'e aurait dit qu'entre les deux mod\`eles il faut choisir le plus   
simple. Cela ne signifie pas que le plus simple est le plus vrai, mais 
ne signifie pas non plus que rien n'existe r\'eellement et que tout  
serait convention. Et cela signifie encore moins que si par hasard le 
plus simple des deux se trouve \^etre d\'eterministe, on soit fond\'e \`a  
en d\'eduire que la science (en  faisant ce choix) aurait prouv\'e que la 
Nature est d\'eterministe, ou inversement. Ind\'ependamment de 
Poincar\'e, un certain bon sens nous ferait juger tr\`es artificiel, ou 
m\^eme farfelu, d'ajouter \`a un mod\`ele ind\'eterministe une annexe qui 
ne le rendrait pas plus juste dans ses pr\'edictions, et dont la seule 
fonction serait de le rendre  d\'eterministe. Ce bon sens est simplement 
le sentiment de la na\"\i vet\'e d'une telle entreprise, et ce sentiment de 
na\"\i vet\'e  provient d'une exp\'erience mill\'enaire, qui  nous a 
toujours montr\'e que lorsqu'on d\'ecouvre une explication plus profonde 
\`a un ph\'enom\`ene, celle-ci est bien plus surprenante et plus retorse 
que ce qu'on aurait pu imaginer. 
\medskip 
Pour comprendre de fa\c con plus pr\'ecise les cons\'equences du 
caract\`ere {\it approch\'e}, et non math\'ematique, de la connaissance,  
nous allons analyser  un mod\`ele simplifi\'e de roulette: nous supposons 
que la bille se d\'eplace \`a l'int\'erieur d'un cercle, mais sur un plan (et 
non une cuvette); qu'elle rebondit sur le bord du cercle comme une boule  
de billard; et pour simuler le frottement nous supposons simplement que  
la vitesse de la bille diminue exponentiellement avec le temps. La 
description de la vraie roulette serait plus difficile et plus 
compliqu\'ee, mais ne comporterait rien d'essentiellement diff\'erent. 
\medskip 
Cette roulette simplifi\'ee (ou plut\^ot, ce {\it mod\`ele} de roulette  
simplifi\'ee) est un cas de chaos d\'eterministe interm\'ediaire entre 
l'exemple extr\^emement simple de la suite $u_n = $ partie d\'ecimale  
de $n \cdot e$ et la vraie roulette. En outre, une bille dans un billard 
circulaire est un exemple qui, contrairement \`a la suite $u_n$, n'est 
pas uniquement math\'ematique, car on peut \`a la fois imaginer une  
bille mat\'erielle sur un tapis vert, rebondissant sur  une bordure 
circulaire, et le mod\`ele math\'ematique correspondant. Ce mod\`ele est 
donc exactement ce qu'il nous faut pour comprendre le hasard dans la 
Nature. Fondamentalement ce mod\`ele  permet non seulement de deviner 
par analogie ce qui se produit pour la roulette, mais \'egalement pour de 
nombreux ph\'enom\`enes analogues dans leur principe, tels que par 
exemple l'agitation thermique, le mouvement brownien, la 
diss\'emination de poussi\`eres dans l'atmosph\`ere, etc.   
\medskip 
 
\midinsert 
\centerline{\epsfbox{../images/fig4.eps}} 
\vskip 3mm 
\centerline{\eightpoint figure 4} 
\vskip 6pt 
\centerline{\eightpoint \'Evolution de la trajectoire d'une bille dans un 
cercle.}  
\vskip3mm
\endinsert 
 
La bille est lanc\'ee \`a l'instant 0, avec une vitesse initiale $v$. La 
trajectoire qu'elle va d\'ecrire est form\'ee de cordes successives, 
l'extr\'emit\'e de chacune \'etant l'origine de la suivante (voir la figure 
4).  La r\'eflexion sur le bord ob\'eit bien s\^ur \`a la loi de la r\'eflexion,  
\`a savoir que  la bissectrice de l'angle form\'e par deux cordes 
cons\'ecutives passe par le centre du cercle. Cet angle reste le m\^eme 
pour toutes les cordes; appelons le $\theta$. Si la bordure du billard  
n'\'etait pas un cercle, cet angle changerait \`a chaque rebondissement; 
le probl\`eme en serait rendu plus compliqu\'e, mais sans aucun 
b\'en\'efice pour la discussion.  
\medskip 
Si l'angle $\theta$ est commensurable avec $\pi$, c'est-\`a-dire s'il 
existe un nombre {\it rationnel} $\alpha = p/q$ (quotient de deux 
entiers $p$ et $q$) tel que $\theta = \alpha\pi$, alors la trajectoire  
de la bille, form\'ee d'une suite de cordes contigu\"es,  est un polygone 
r\'egulier: les cordes se superposent p\'eriodiquement, et la bille  
repasse p\'eriodiquement sur ses traces. Ce cas particulier est 
l'analogue de la suite $v_n = $ partie d\'ecimale de $n \cdot 0.5$, ou  
plus g\'en\'eralement $v_n = $ partie d\'ecimale de $n \cdot \alpha$  
(avec $\alpha$ fractionnaire). Par contre si $\theta$ est 
incommensurable avec $\pi$, par exemple si $\pi / \theta = e$, alors  
la bille ne repasse  jamais deux fois sur la m\^eme corde. Si la bille 
roule tr\`es longtemps (si donc le frottement est faible) elle parcourt  
un tr\`es grand nombre de cordes, toutes distinctes, et ces cordes 
remplissent compl\`etement une couronne (voir figure 4). Remplir 
compl\`etement signifie que n'importe quelle petite r\'egion de la 
couronne, aussi petite soit-elle, finira t\^ot ou tard par recevoir la 
visite de la bille.   
\medskip 
Supposons maintenant que le disque soit subdivis\'e en beaucoup de 
petites r\'egions. On lance la bille avec une vitesse initiale $v_0$, un 
angle  $\theta = \pi / e$, et le coefficient de frottement est $k$, de 
sorte que la vitesse de la bille d\'ecro{\^\i}t en fonction du temps $t$
selon la loi $v(t) = e^{-kt}$.  Les petites r\'egions situ\'ees en dehors
de la couronne ne seront de toute fa\c con jamais atteintes par la bille. 
Par contre la bille finira par s'arr\^eter (plus exactement, elle se 
rapprochera ind\'efiniment d'une position limite), forc\'ement dans une  
r\'egion situ\'ee \`a une distance du centre sup\'erieure \`a  
$R\sin (\pi / 2\, e)$. Le mod\`ele est enti\`erement d\'eterministe, 
puisque si  on conna\^\i t $v_0$, $\theta$, $k$, et la position initiale 
$M_0$ de la bille, alors le point o\`u la bille s'arr\^etera est parfaitement 
d\'etermin\'e, et la petite r\'egion dans laquelle ce point se trouve peut 
donc \^etre pr\'edite. Voyons comment.  
\medskip 
La trajectoire est de toute fa\c con d\'etermin\'ee par la seule donn\'ee  
de  $M_0$ et $\theta$. Le param\`etre de frottement $k$ ne d\'etermine  
que la distance parcourue sur cette trajectoire. On supposera $M_0$ 
situ\'e sur le bord (c'est-\`a-dire qu'on lance la bille \`a partir du bord),  
et en outre l'invariance par rotation permet de ne faire le calcul que pour 
un point $M_0$ particulier: si on change $M_0$ en $M_0'$ par une rotation, 
le point d'arriv\'ee sera chang\'e selon la m\^eme rotation. La  
d\'ependance par rapport \`a $M_0$ ne montre donc rien d'int\'eressant. 
Mais supposons qu'on change un tout petit peu $\theta$ en $\theta' = 
\theta + \varepsilon$. La premi\`ere corde parcourue par la bille sera 
l\'eg\`erement tourn\'ee, d'un angle \'egal  \`a  $\varepsilon$; la seconde 
sera tourn\'ee du double, la troisi\`eme  du triple, etc. Si on appelle 
$M_1$, $M_2$, $\cdots$ les points de la circonf\'erence touch\'es 
successivement --~apr\`es $M_0$~-- par la  bille, la distance angulaire 
de deux cons\'ecutifs de ces points est $\pi  - \theta$, qui se trouve donc 
modifi\'e en $\pi - \theta'$. \'Etant donn\'e que $M_0$ est fix\'e, cela  
se traduit par un d\'eplacement angulaire de  $-\varepsilon$ pour $M_1$, 
de $-2\varepsilon$ pour $M_2$, de $-3\varepsilon$ pour $M_3$, etc. Au 
bout d'un nombre $N$ de r\'eflexions, le point $M_N$ sera d\'eplac\'e, par 
rapport \`a la  trajectoire de r\'ef\'erence, de $-N\varepsilon$; si $N$ est 
assez grand pour compenser la petitesse de $\varepsilon$, la bille pourra 
se trouver  dans une r\'egion \'eloign\'ee de celle qu'elle aurait atteinte 
apr\`es le  m\^eme nombre de r\'eflexions sur la trajectoire de 
r\'ef\'erence.  
\medskip 
Voyons maintenant comment le mouvement d\'epend de $k$ (param\`etre  
de frottement). La trajectoire, qu'elle soit de r\'ef\'erence ou 
modifi\'ee,  ne change pas lorsqu'on change la valeur de $k$; seule 
change la {\it distance parcourue} sur cette trajectoire. Les 
r\'eflexions successives ne changent rien \`a la vitesse, seule la 
direction du mouvement est affect\'ee. On peut donc, bien que la 
trajectoire soit une ligne bris\'ee, traiter le mouvement {\it sur 
cette trajectoire} comme s'il s'agissait d'un mouvement sur un axe. Soit 
donc $x(t)$ la distance parcourue {\it sur cette trajectoire} par la bille 
\`a l'instant $t$,  depuis l'instant $0$ o\`u elle \'etait en $M_0$. La 
d\'eriv\'ee de $x(t)$ est  sa vitesse \`a l'instant $t$ , dont nous avons 
suppos\'e qu'elle \'etait  \'egale \`a $v_0 e^{-kt}$. Ainsi:    
$$\eqalign{ 
x'(t)  &= v_0 e^{-kt} \quad ; \cr 
x(0) &= 0 \quad . \cr }$$ 
Par un calcul de primitive imm\'ediat, on obtient: 
$$x(t) = {v_0 \over k}\bigl( 1 - e^{-kt} \bigr)$$ 
On voit que lorsque $t$ tend vers l'infini, $x(t)$ tend vers $v_0 / k$. En 
pratique $x(t)$ sera pratiquement \'egal \`a $v_0 / k$ d\'ej\`a pour $t 
\sim 10/k$, et n'en bougera pratiquement plus ensuite, ce qui signifie 
que la bille s'arr\^etera au bout d'un temps de l'ordre de $10/k$, apr\`es 
avoir parcouru une distance \'egale \`a $v_0 / k$. 
\medskip 
La question est maintenant de savoir en quel lieu du disque la bille 
aboutira. Pour des valeurs donn\'ees des param\`etres $k$ et $\theta$,  
la longueur d'une seule corde est $2 R \cos (\theta /2)$ et la distance 
parcourue $v_0 / k$, donc le nombre de cordes parcourues (ou, ce qui 
revient au m\^eme, le nombre de rebroussements) est $N = $ partie 
enti\`ere de $v_0 / 2 k R \cos (\theta /2)$, et la position de la bille  
sur la $N+1$ -i\`eme corde sera donn\'ee par la partie d\'ecimale de  
$v_0 / 2 k   R \cos (\theta /2)$ : si cette partie d\'ecimale est par 
exemple 0.75,  cela signifiera que la $N+1$ -i\`eme corde aura \'et\'e 
parcourue aux trois quarts.  Or nous avons vu que si $\theta$ est 
modifi\'e d'une tr\`es petite quantit\'e $\varepsilon$, et que $N$ est de 
l'ordre de $1 / \varepsilon$, la  $N+1$ -i\`eme corde ne sera pas 
modifi\'ee un tout petit peu, mais beaucoup. Si $k$ n'a pas \'et\'e 
modifi\'e en m\^eme temps que $\theta$, la bille, bien qu'ayant parcouru 
la {\it m\^eme} distance sur la trajectoire, s'arr\^etera n\'eanmoins en  
un point compl\`etement diff\'erent du disque.  Il suffit pour cela que $N 
\sim 1 / \varepsilon$, soit  $v_0 / 2 k R \cos (\theta /2) \sim 1 / 
\varepsilon$ ou, ce qui revient au m\^eme: 
$$k \sim {v_0 \, \varepsilon \over 2\, R\, \cos ({1 \over 2}\theta )} 
\eqno (I.3.)$$ 
D'autre part, si on modifie $k$ mais pas $\theta$, mettons qu'on 
augmente ou diminue $k$ d'une proportion $\eta$, qu'on le remplace 
donc par $k' = k(1+\eta )$, alors la trajectoire reste inchang\'ee, mais  
la distance parcourue sur cette trajectoire passera de $v_0 /k$ \`a 
$v_0  /k'$, c'est-\`a-dire qu'elle sera divis\'ee par $1+\eta$, ou 
multipli\'ee  par $1 - \eta$ puisque pour $\eta$ petit  $1 /(1 + \eta ) 
\simeq 1 - \eta$. La diff\'erence entre les deux est donc \`a peu pr\`es 
\'egale \`a $\eta v_0 / k$, ce qui n'est pas petit si $k$ est du m\^eme 
ordre de grandeur que  $\eta v_0$. Cela peut m\^eme d\'epasser la 
longueur d'une corde si   
$$k < {v_0 \, \eta \over 2\, R\, \cos ({1 \over 2}\theta )} \eqno (I.4.)$$ 
On voit alors ce qui se passe: tant que le probl\`eme reste purement 
math\'ematique, la position finale est toujours exactement pr\'edictible 
\`a partir de la donn\'ee des param\`etres $\theta$ et $k$. Mais d'apr\`es 
$(I.3)$ et $(I.4.)$ l'incertitude sur la position finale est beaucoup plus 
grande que l'incertitude sur les valeurs de ces param\`etres, et cela 
d'autant plus que $k$ est plus petit. Autrement dit: plus le frottement 
est faible, plus cette disproportion entre les incertitudes est grande.   
Pour en donner une id\'ee num\'erique: supposons que le temps soit  
mesur\'e en secondes et les distances en d\'ecim\`etres, que le rayon  
du disque soit $R = 1$ d\'ecim\`etre, et $v_0 = 1$ d\'ecim\`etre par 
seconde. Prenons  $k = 10^{-3}$, $\theta$ tel que $\cos (\theta / 2) = 
0.75$, ce qui fait $\theta \simeq 1.445\, 468\, 496$ et $\pi / \theta 
\simeq 2.173\, 407\, 904$. La longueur d'une corde sera alors $1.5$, la 
distance parcourue par la bille $10^3$, c'est-\`a-dire $100$ m\`etres, le 
nombre de cordes parcourues sera $N = 666$, et la bille s'arr\^etera aux 
deux tiers  de la $667^{\rm e}$ corde. On peut se donner un rep\`ere 
orthonorm\'e pour les coordonn\'ees des points, tel que $M_0$ ait pour 
coordonn\'ees $(1,0)$. Puisqu'on passe de chaque corde \`a la suivante par 
une rotation d'angle $\pi - \theta$, on aura la  $667^{\rm e}$ corde par 
une rotation d'angle $666 \cdot (\pi -\theta) \simeq 1\, 129.618\, 
689\, 203$, ce qui est \'egal \`a $4.928\, 519\, 218$ modulo $2\pi$. Le 
point situ\'e aux deux tiers de cette corde est alors le point de 
coordonn\'ees $x=+0.699\, 662\, 113$ et $y=-0.102\, 337\, 320$. Si on 
prend $\varepsilon = 10^{-3}$, et $\eta = 0$,  le nouvel angle $\theta' / 
2$ a pour cosinus non plus $0.75$, mais $0.749\, 669\, 187$, la longueur 
des cordes n'est plus $1.5$, mais $1.499\, 338\, 375$, le nombre de  
cordes parcourues est toujours $666$, mais la proportion parcourue de la 
$667^{\rm e}$ est $0.960\, 852\, 116$. Si on refait les m\^emes calculs 
de rotations, on trouve que cette fois le point atteint par la bille est le  
point de coordonn\'ees $x=+0.893\, 362\, 414$ et $y=-0.342\, 554\, 750$. 
Les deux points sont \`a l'int\'erieur du disque, mais leur distance 
mutuelle est $0.308\, 584\, 218$, \`a peu pr\`es le tiers du rayon. Le 
calcul confirme donc ce qui \'etait qualitativement pr\'evu, \`a savoir 
qu'un changement de un milli\`eme de l'angle $\theta$ entra\^\i ne un 
changement macroscopique du point atteint par la bille. On observerait la 
m\^eme chose si au lieu de faire varier $\theta$, on avait fait varier  $k$, 
en prenant $\varepsilon = 0$ et $\eta = 10^{-3}$, ou en faisant varier les 
deux \`a la fois. Cela veut dire que si on avait voulu pr\'edire le point o\`u 
la bille viendra s'immobiliser au milli\`eme pr\`es, il aurait fallu 
conna\^\i tre $\theta$ et $k$ avec une pr\'ecision non du milli\`eme, mais 
du millioni\`eme. C'est ce que montrent les relations $(I.3.)$ et $(I.4.)$. 
Le point o\`u la bille s'arr\^etera peut donc \^etre pr\'edit, mais la 
pr\'ecision de cette pr\'ediction exige une pr\'ecision encore bien plus 
grande sur les valeurs des param\`etres initiaux $\theta$ et $k$, 
d'autant plus grande d'ailleurs que $k$ est plus petit, conform\'ement \`a 
$(I.3.)$. Il s'agit l\`a d'une propri\'et\'e caract\'eristique du chaos 
d\'eterministe, qui est l'{\it amplification} des perturbations, ou la 
{\it tr\`es haute sensibilit\'e aux conditions initiales}. 
\medskip  
D'autre part, il se produit --~mais en deux dimensions cette fois~-- le 
m\^eme ph\'enom\`ene que pour la suite $u_n$, \`a savoir que si le 
frottement \'etait absolument nul, la bille ne s'arr\^eterait jamais,  mais 
parcourrait la couronne \'eternellement, et si on d\'elimitait de petites 
r\'egions dans la couronne, alors  
\smallskip 
a) chacune de ces petites r\'egions serait travers\'ee \'episodiquement  
par la bille (d'autant plus souvent que la dur\'ee de l'observation serait 
longue);  
\smallskip 
b) le temps total (form\'e donc de nombreuses travers\'ees br\`eves)  
que la bille passerait \`a l'int\'erieur de chacune de ces petites 
r\'egions, d\'ependrait de l'aire de cette petite r\'egion. 
\medskip 
Pour un \^etre microscopique vivant dans l'une de ces petites r\'egions, 
et voyant de temps en temps passer la bille, les passages de cette bille 
sembleraient survenir \`a des moments al\'eatoires (cela provient de ce 
que $\pi / \theta$ est irrationnel). L'\^etre  microscopique, ne pouvant  
pas imaginer la cause d\'eterministe des passages de la bille, penserait 
avoir affaire \`a un hasard ontologique. En cas de frottement, il 
s'apercevrait cependant que  la vitesse de la bille est plus lente \`a 
chaque passage, et il pourrait m\^eme avoir le privil\`ege fantastique de 
voir un jour la bille venir s'arr\^eter dans son jardin. 
 \medskip 
\`A une toute autre \'echelle, et avec une perspective compl\`etement 
diff\'erente, le physicien humain qui lance cette bille serait un dieu  
pour l'\^etre microscopique, mais, sa connaissance des choses \'etant 
approch\'ee et non math\'ematiquement exacte, il ne pourrait  
lui-m\^eme pr\'evoir le point o\`u la bille s'arr\^etera, faute de pouvoir 
mesurer avec une pr\'ecision suffisante les conditions initiales du 
mouvement. Ainsi, le postulat selon lequel le mouvement est 
d\'eterministe a un sens dans le mod\`ele math\'ematique, mais n'en a 
absolument aucun dans la r\'ealit\'e.  
\medskip 
Une question int\'eressante est maintenant de savoir comment sont 
r\'epartis ces temps de s\'ejour dans chaque petite r\'egion. Avec la  
suite $u_n$, nous avons dit que (pour $n$ grand) le nombre de termes  
qui ``tombent'' dans un intervalle $[a,b]$ de $[0,1]$ est proportionnel  
\`a la longueur $b-a$. Pour la bille, nous allons voir que la dur\'ee  
totale de s\'ejour dans chacune des petites r\'egions n'est pas 
proportionnelle \`a son aire: les r\'egions situ\'ees pr\`es du bord 
int\'erieur de la couronne seraient --~\`a aire \'egale~-- fortement 
favoris\'ees par rapport aux r\'egions situ\'ees pr\`es du bord 
ext\'erieur. On peut comprendre cela facilement:  ce qui est r\'eparti 
uniform\'ement n'est pas la densit\'e des passages de la bille, mais les 
angles d'inclinaison des cordes: en effet, ces angles sont, comme nous 
l'avons vu, en progression arithm\'etique, et la suite des angles, 
compt\'es modulo $2\pi$, est donc du m\^eme type que la suite $u_n$.  
Si on se fixe deux cordes  proches l'une de l'autre, leurs angles  
d'inclinaison sont proches \'egalement, ils diff\`erent, disons, d'un  
petit angle $\delta$ (figure 5). Cela veut dire que ces deux cordes ont 
entre elles un angle  $\delta$; elles se coupent tout pr\`es du bord 
int\'erieur de  la couronne, mais s'\'ecartent l'une de l'autre en 
s'\'eloignant de l'intersection. Parmi toutes les  cordes de la trajectoire, 
certaines se situeront entre les deux cordes que nous nous sommes 
fix\'ees; leur nombre sera proportionnel \`a l'angle $\delta$,  tout comme 
le nombre d'\'el\'ements de la suite $u_n$  situ\'es \`a l'int\'erieur d'un 
petit intervalle de longueur $\delta$ est proportionnel \`a $\delta$. Le 
nombre de ces cordes sera \`a peu pr\`es \'egal \`a $\delta /2\pi$ fois 
le nombre de toutes les cordes parcourues. Or les cordes sont plus 
denses, ou plus serr\'ees, pr\`es du bord  int\'erieur de la couronne: leur 
distance est nulle tout pr\`es du bord int\'erieur et cro\^\i t ensuite 
lin\'eairement, ce qui entra\^\i ne que leur densit\'e est tr\`es grande 
pr\`es du bord int\'erieur (\`a la limite infinie), mais  diminue en se 
rapprochant du bord ext\'erieur. Cela se voit imm\'ediatement  rien 
qu'en regardant la  figure 4.   
\medskip  
Pour calculer, introduisons la coordonn\'ee radiale $t = \sqrt{x^2 + y^2}  
/ R$ (la distance au centre du disque rapport\'ee au rayon). La couronne 
correspond aux valeurs de $t$ comprises entre $r=\sin\bigl( {1\over 2} 
\theta\bigr)$ et $1$ ($rR$ est ainsi le rayon int\'erieur de la couronne). 
\medskip 
 
\midinsert 
\vbox{\vskip3pt 
\centerline{\epsfbox{../images/fig5.eps}} 
\vskip2mm 
\centerline{\eightpoint figure 5} 
\vskip6pt 
\centerline{\eightpoint Petits losanges d\'elimit\'es par des faisceaux  
de  cordes.} 
\vskip2mm} 
\endinsert 
 
Les aires infinit\'esimales $\sigma$ d\'elimit\'ees par deux couples de 
cordes tr\`es proches, formant entre elles de petits angles $\delta$ 
(voir figure 5) sont \'egales \`a   
$$\sigma = \hbox{${1 \over 2}$} \; t^2 \; \delta^2 \;  \sqrt{{t^2 \over 
r^2} - 1}$$    
Mais puisque, entre deux cordes fix\'ees, ayant entre elles un angle  
$\delta$, le nombre de passages de la bille est proportionnel \`a  
$\delta$ (plus pr\'ecis\'ement \'egal \`a $N\delta / 2\pi$, $N$ \'etant  
le  nombre total de  cordes parcourues), et le petit intervalle que la  
bille parcourt \`a  chaque passage \`a l'int\'erieur du losange \'etant de  
longueur $t^2 \delta / 2r$, on en d\'eduit que la distance totale (somme 
de  tous ces petits intervalles) parcourue par la bille dans le petit 
losange  d'aire $\sigma$ est  
$$\tau = {N \over 2\,\pi\, r } \; t^2 \; \delta^2$$ 
Si on exprime $\tau$ en fonction de $\sigma$ on trouve 
$$\tau = {N \over\sdown{12} \pi\,\sqrt{t^2 - r^2}}\; \sigma$$ 
Conform\'ement \`a ce que nous avions pressenti auparavant par des 
estimations qualitatives, $\tau$ n'est pas simplement proportionnel \`a 
$\sigma$; on voit appara\^\i tre la densit\'e $1 /\,\sqrt{t^2 - r^2}$, qui  
exprime quantitativement le favoritisme dont b\'en\'eficient, \`a aire 
\'egale, les petits losanges situ\'es pr\`es du bord int\'erieur de la  
couronne.  Cette loi valable pour les petits losanges s'\'etend ensuite \`a 
des petites r\'egions de forme quelconque, puisqu'on peut toujours 
quadriller une r\'egion selon de tels petits losanges, comme on quadrille 
en petits carr\'es dans la th\'eorie usuelle de l'int\'egration. 
\medskip 
Si au lieu de compter le {\it chemin} parcouru \`a l'int\'erieur de chaque 
petit losange on avait voulu compter simplement le {\it nombre de 
passages}, sans consid\'eration du chemin parcouru, on aurait aussi 
obtenu une loi statistique, dont on aurait d\'eduit une probabilit\'e 
empirique; dans ce cas, ce seraient les dimensions lin\'eaires, et non  
plus l'aire, qui interviendraient: par exemple, si la petite r\'egion est un 
petit disque de diam\`etre $\rho$, le nombre de passages de la bille \`a 
travers ce petit disque serait \`a peu pr\`es \'egal \`a   
$$\nu = {N \,\over\sdown{12}\pi\,\sqrt{t^2 - r^2}}\; \rho$$  
Bien entendu la densit\'e $1 /\sqrt{t^2 - r^2}$ est toujours l\`a.  
\medskip 
Ainsi les \^etres microscopiques qui habitent dans la couronne 
observeraient des lois statistiques, v\'erifi\'ees par les passages de la 
bille: pour toutes les r\'egions  situ\'ees {\it \`a une distance donn\'ee}  
du centre du disque, le total du temps pass\'e par la boule \`a l'int\'erieur 
d'une r\'egion du disque est --~sur un grand nombre de passages~-- 
proportionnel \`a l'aire de cette r\'egion; ou encore: le nombre total de 
passages \`a travers une petite r\'egion est proportionnel \`a son 
diam\`etre. Mais lorsqu'on consid\`ere des r\'egions situ\'ees \`a des  
distances diff\'erentes du centre, le coefficient de proportionnalit\'e 
varie selon $1 /\sqrt{t^2 - r^2}$.  
\medskip 
En observant vraiment bien, les \^etres microscopiques apercevraient  
des r\'egularit\'es dans les passages, car si la suite des  
passages semble al\'eatoire \`a premi\`ere vue, elle a les m\^emes 
d\'efauts que la suite $u_n$. Ceci provient de ce que le mod\`ele 
est trop simple, on peut tout calculer explicitement \`a l'aide de 
formules ou d'algorithmes simples, que des inspecteurs peuvent 
d\'ecrypter. Mais on peut rendre le mod\`ele plus compliqu\'e: par 
exemple, au lieu d'une bordure circulaire, on pourrait prendre une 
bordure non circulaire, pas m\^eme elliptique (car l'ellipse poss\`ede 
encore trop de r\'egularit\'es). Point n'est besoin de modifier beaucoup 
la bordure: celle-ci peut ne jamais s'\'ecarter du  cercle de plus de un 
centi\`eme de millim\`etre: nous avions vu qu'en ne changeant 
l'angle $\theta$ que d'un milli\`eme, on perturbait macroscopiquement 
la forme prise par la trajectoire apr\`es plusieurs centaines de 
rebroussements; il en irait de m\^eme en modifiant la bordure de 
quelques milli\`emes; en effet, modifier cette bordure revient \`a 
modifier {\it tous} les angles de r\'eflexion, chacun d'une quantit\'e  
petite, mais diff\'erente, et non pas seulement le premier. En agissant 
ainsi, on produirait le m\^eme type de perturbation amplifi\'ee: le  
d\'ebut de la trajectoire serait peu modifi\'e, mais  {\it apr\`es un 
certain temps}, inversement proportionnel  \`a l'amplitude des 
modifications, elle serait compl\`etement diff\'erente. Le simple fait 
d'avoir une bordure qui, bien que presque circulaire, n'est plus 
rigoureusement un cercle, rend d\'ej\`a le calcul exact de la trajectoire 
si compliqu\'e que les inspecteurs n'y verraient que du feu. Pour d\'ejouer 
encore davantage leur vigilance, on pourrait donner au fond du disque (le 
tapis vert) une forme courbe de cuvette au lieu d'un plan. Le mouvement 
serait toujours rigoureusement d\'eterministe. Les propri\'et\'es de 
chaos que nous avons mises en \'evidence dans le mod\`ele simple 
subsisteraient  (\`a savoir l'{\it  amplification}, le fait qu'une tr\`es 
faible modification des conditions initiales entra\^\i nerait une 
modification macroscopique du mouvement apr\`es un nombre de  tours 
inversement proportionnel \`a cette modification). Mais l'algorithme pour 
les calculer serait alors tellement complexe que ce mouvement {\it ne 
pourrait pas \^etre calcul\'e}. Les \^etres microscopiques ne pourraient 
plus observer des r\'egularit\'es dans les passages individuels de la bille: 
si leurs observations \'etaient exactes (c'est-\`a-dire sans erreur de 
mesure), ce serait la complexit\'e du processus qui les emp\^echerait de 
voir les r\'egularit\'es. Et pour \^etre bien certain que les habitants de la 
couronne, qu'il ne faudrait tout de m\^eme pas sous-estimer, ne pourront 
jamais percer le myst\`ere des passages de la bille, nous supposerons que 
la zone situ\'ee pr\`es du bord  leur est inaccessible (que les lois 
physiques de leur monde sont telles qu'il leur faudrait une \' energie 
infinie pour atteindre ce bord) et qu'ils ne peuvent donc pas conna\^\i tre 
exactement sa forme. Si en outre  leurs observations sont approch\'ees et 
entach\'ees d'erreurs de mesure, le d\'eterminisme du mouvement leur 
sera encore plus cach\'e. Les seules propri\'et\'es suffisamment simples 
qu'ils pourront observer seront les deux propri\'et\'es suivantes:   
\smallskip 
$a)$ la dur\'ee totale de s\'ejour dans une petite r\'egion d'aire  
$\sigma$ situ\'ee \`a une distance $t$ du centre du disque est 
proportionnelle \`a $\sigma$ et inversement proportionnelle \`a 
$\sqrt{t^2 - r^2}$.  
\smallskip 
$b)$ la variation en $1 / \sqrt{t^2 - r^2}$ s'explique {\it simplement}  
par une  propri\'et\'e g\'eom\'etrique. C'est-\`a-dire que les \^etres 
microscopiques peuvent ais\'ement comprendre que les trajectoires sont 
des segments de droites qui s'\'ecartent lin\'eairement et sont donc moins 
denses vers le bord.    
\medskip 
Ainsi le ph\'enom\`ene se d\'ecompose en deux parties: 1. une distribution  
myst\'erieuse des segments de trajectoire, et 2. un effet d\^u uniquement  
aux propri\'et\'es g\'eom\'etriques communes \`a tous les segments  
(th\'eor\`eme de Thal\`es, etc.). La partie 2 du ph\'enom\`ene est 
compr\'ehensible et ``explique'' la densit\'e $1 / \sqrt{t^2 - r^2}$. Par 
contre, la partie du ph\'enom\`ene que les \^etres microscopiques  {\it  
ne peuvent pas comprendre} est la distribution des segments. Il leur est 
ais\'e de voir que les segments se coupent pr\`es du bord int\'erieur 
(comme on voit  sur la figure 5) et utiliser le th\'eor\`eme de Thal\`es 
pour retrouver par  le calcul la densit\'e  $1 / \sqrt{t^2 - r^2}$, mais ils 
ne peuvent pas pr\'evoir la position des segments successifs. Ils peuvent 
alors d\'ecouvrir apr\`es quelques t\^atonnements math\'ematiques que, 
si au lieu de mesurer le temps de s\'ejour dans chaque petite r\'egion, on 
avait mesur\'e par exemple l'orientation des segments (l'angle 
d'inclinaison par rapport \`a la direction radiale), alors il n'y aurait plus 
de densit\'e variable: les diff\'erentes orientations se distribueraient de 
fa\c con uniforme. \'Etant alors donn\'e que la distribution des segments 
est incompr\'ehensible, il n'est plus possible de progresser au-del\`a de  
cette loi uniforme: de m\^eme que dans un calcul alg\'ebrique (par  
exemple factoriser un polyn\^ome ou r\'eduire une fraction du type 
$(\sqrt{2} + \sqrt{3}) /  (\sqrt{5} -\sqrt{3})$ \`a sa forme la plus 
simple),  il arrive forc\'ement un moment o\`u ``on ne peut plus r\'eduire 
davantage'', de m\^eme lorsqu'un ph\'enom\`ene est  d\'ecompos\'e en une 
partie incompr\'ehensible et une partie compr\'ehensible, il arrive  
forc\'ement un moment o\`u ``on ne peut plus r\'eduire davantage'' la  
partie incompr\'ehensible. On a alors atteint un stade de 
{\it compr\'ehension maximum} du ph\'enom\`ene. \`A un tel stade, la 
partie incompr\'ehensible est r\'eduite \`a une distribution al\'eatoire 
uniforme, car tant qu'il subsiste une non-uniformit\'e, celle-ci peut faire 
l'objet de recherches pour en comprendre l'origine, comme ce fut le cas 
pour la densit\'e $1/\sqrt{t^2-r^2}$. En revanche, une fois la partie 
incompr\'ehensible r\'eduite ``\`a sa plus simple expression'', le seul 
progr\`es encore possible consisterait \`a lever l'obstacle du brouillage 
par le chaos (par exemple disposer d'ordinateurs  tellement puissants et 
d'instruments de mesure tellement pr\'ecis que  le chaos d\'eterministe 
puisse \^etre lui aussi ma\^\i tris\'e).  
\medskip 
Une fois qu'on a regroup\'e tout ce qui est compr\'ehensible et r\'eduit   
l'incompr\'ehensible \`a sa plus  simple expression, uniforme par 
nature, ce dernier re\c coit alors un nom: le hasard. 
 
\vfill\break 
 
 
 
\bye 
