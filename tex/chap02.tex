
\input/home/harthong/tex/formats/twelvea4.tex

\auteurcourant={\sl J. Harthong: probabilit\'es et statistique}
\titrecourant={\sl D\'enombrement}

\font\twelvemi=cmmi12 
\def\struta{\vrule depth2.4pt width0pt} 
\def\strutb{\vrule depth4.8pt width0pt} 
\def\strutc{\vrule depth3.6pt width0pt} 

\pageno=34 
 
\null\vskip10mm plus4mm minus3mm 
 
\centerline{\tit II. D\'ENOMBREMENT.} 
\vskip10mm 
 
Comme il a \'et\'e dit au chapitre I, le calcul des probabilit\'es  
consiste d'abord \`a mod\'eliser une situation sous la forme d'un 
ensemble fini,  l'{\it espace des \'epreuves}, puis \`a calculer les 
probabilit\'es des \'ev\'enements. Or dans un mod\`ele, calculer la 
probabilit\'e d'un \'ev\'enement $A$ revient, en vertu de $(I.2.)$ \`a 
calculer le nombre d'\'el\'ements (\'epreuves) que contient l'ensemble 
$A$. Compter exhaustivement n'est que tr\`es rarement une m\'ethode 
efficace,  surtout lorsque les ensembles sont gros. Il faut donc 
commencer par apprendre les techniques \'el\'ementaires de {\it 
d\'enombrement}, qui permettent de d\'eterminer sans fatigue le 
nombre d'\'el\'ements des ensembles les plus ty\-pi\-ques. Dans ce 
chapitre, nous apprendrons \`a r\'esoudre les probl\`emes sui\-vants:  
\smallskip 
--- avec un alphabet de $r$ lettres, combien de mots diff\'erents de  
$n$ lettres peut-on \'ecrire? (suites avec r\'ep\'etition) 
\smallskip 
--- avec un alphabet de $r$ lettres, combien peut-on \'ecrire de mots  
diff\'erents form\'es de $n$ lettres {\it toutes distinctes}? (suites  
sans r\'ep\'etition ou arrangements)  
\smallskip 
--- combien de sous-ensembles diff\'erents \`a $n$ \'el\'ements y a-t-il 
dans un ensemble \`a  $r$  \'el\'ements? (ensembles non ordonn\'es sans 
r\'ep\'etition ou combinaisons) 
\smallskip 
--- on se donne $k$ nombres entiers $\geq 0$, $n_1$, $n_2 \ldots n_k$, 
dont la somme est $n$; de combien de mani\`eres diff\'erentes peut-on  
ranger $n$ \'el\'ements en $k$ groupes de (respectivement) $n_1$, $n_2 
\ldots n_k$ \'el\'ements? (partitions en groupes de taille donn\'ee) 
\smallskip 
--- $n$ \'etant donn\'e, de combien de mani\`eres diff\'erentes peut-on  
choisir les nombres $n_1$, $n_2 \ldots n_k$ du probl\`eme pr\'ec\'edent? 
(subdivisions) 
 
\medskip 
La connaissance de ces cinq cas \'el\'ementaires de d\'enombrement  
permet d\'ej\`a de r\'esoudre une quantit\'e de probl\`emes de 
probabilit\'es. En fait, la quasi totalit\'e des probl\`emes usuels se 
ram\`ene, apr\`es un travail math\'ematique ad\'equat (et plus ou moins 
long selon  la difficult\'e intrins\`eque du probl\`eme), \`a un ou 
plusieurs de ces cas de d\'enombrement.  
\medskip 
Ce chapitre est form\'e de cinq sections, chacune correspondant \`a 
l'un de ces cinq cas. On pr\'esentera au fur et \`a mesure des exemples 
simples illustrant  chaque formule. 
 
\bigskip 
\penalty-500 
{\bf II. 1. Suites avec r\'ep\'etition ou tirages avec remise.}  
\penalty500 
\medskip 
Avec un alphabet de $r$ lettres, combien de mots diff\'erents de $n$ 
lettres peut-on \'ecrire?  
\smallskip 
Imaginons que notre alphabet soit form\'e des trois lettres $A,B,C$  
($r=3$). On peut former trois mots diff\'erents de une lettre ($n=1$):  
ces mots sont  $A$, $B$, et $C$. Mais on peut former neuf mots de deux  
lettres ($n=2$):  
$$\matrix{ &AA\; , \quad  &AB\; , \quad  &AC\; , \quad \cr  
                  &BA\; , \quad  &BB\; , \quad  &BC\; , \quad \cr  
                  &CA\; , \quad  &CB\; , \quad  &CC\; . \quad \cr }$$ 
Il y a $27$ mots de trois lettres: 
$$\matrix{ &AAA\; , \!\!\!  &AAB\; , \!\!\!  &AAC\; , \!\!\!  
                  &ABA\; , \!\!\!  &ABB\; , \!\!\!  &ABC\; ,  \!\!\!  
                  &ACA\; , \!\!\!  &ACB\; , \!\!\!  &ACC\; ,  \cr  
                  &BAA\; , \!\!\!  &BAB\; , \!\!\!  &BAC\; ,  \!\!\!  
                  &BBA\; , \!\!\!  &BBB\; , \!\!\!  &BBC\; ,  \!\!\!  
                  &BCA\; , \!\!\!  &BCB\; , \!\!\!  &BCC\; ,  \cr                    
                  &CAA\; , \!\!\!  &CAB\; , \!\!\!  &CAC\; , \!\!\!  
                  &CBA\; , \!\!\!  &CBB\; , \!\!\!  &CBC\; ,  \!\!\!  
                  &CCA\; , \!\!\!  &CCB\; , \!\!\!  &CCC\; .  \cr }$$ 
Nous n'irons pas plus loin. On voit sur ces tableaux comment engendrer 
syst\'ematiquement {\it tous} les mots possibles: si la premi\`ere  
lettre est donn\'ee, par exemple $A$, il y a pour la seconde les trois 
possibilit\'es $AA$, $AB$, et $AC$. Si les deux premi\`eres sont 
donn\'ees, par exemple $AA$, il y a trois possibilit\'es pour la 
troisi\`eme: $AAA$, $AAB$, et $AAC$. Autrement dit, pour chaque mot 
possible de $n-1$ lettres, il y a  trois possibilit\'es pour la $n^{\rm e}$. 
Si l'alphabet est form\'e de $r$ lettres, il y aura $r$ possibilit\'es 
chaque fois qu'on ajoute une lettre de plus. Ainsi: avec une seule lettre, 
il y a $r$ possibilit\'es; pour {\it chacune} de ces $r$ possibilit\'es, il y 
a $r$ possibilit\'es pour la deuxi\`eme lettre, ce qui fait $r \times r = 
r^2$ possibilit\'es. Pour {\it chacune} de ces $r^2$ possibilit\'es, il y a 
\`a nouveau $r$ possibilit\'es pour la troisi\`eme lettre, ce qui fait $r^2 
\times r = r^3$ possibilit\'es, et ainsi de suite. Il y aura donc $r^n$ mots 
de $n$ lettres. On peut donc retenir que {\bf le nombre de mots de 
$n$ lettres qu'on peut \'ecrire dans un alphabet de $r$ lettres} est  
$$r^n \eqno (II.1.)$$ 
Beaucoup de probl\`emes qui peuvent para\^\i tre \`a premi\`ere vue 
diff\'erents sont en fait \'equivalents \`a la formation de mots avec  
des lettres. Le probl\`eme que nous venons d'\'etudier est donc un {\it 
mod\`ele} pour des probl\`emes diff\'erents. 
\medskip 
Ainsi, imaginons que nous ayons  $n$ boules num\'erot\'ees et $r$ 
bo\^\i tes. Nous avons d\'ej\`a rencontr\'e ces histoires de boules et de 
bo\^\i tes au chapitre {\bf I}: on voit cela dans la figure 1, o\`u on a 
repr\'esent\'e toutes les mani\`eres diff\'erentes de disposer trois 
boules dans deux bo\^\i tes. En les dessinant toutes sur la figure 1, 
nous avons vu qu'il y avait huit mani\`eres diff\'erentes. En g\'en\'eral,  
pour  $n$ boules et $r$ bo\^\i tes, il y a $r^n$ distributions 
diff\'erentes,  de m\^eme qu'il y a $r^n$ mots de $n$ lettres avec un 
alphabet de $r$ lettres. Les deux probl\`emes sont en r\'ealit\'e 
identiques.  
\medskip 
On peut mettre cette identit\'e en \'evidence en {\it codant} 
les distributions de  boules dans des bo\^\i tes; supposons en effet que 
nous voulions repr\'esenter une distribution de boules par un code 
chiffr\'e: on pourrait d\'esigner chaque bo\^\i te par une lettre, et 
chaque boule par son num\'ero (elles sont suppos\'ees num\'erot\'ees). 
Pour une distribution de boules donn\'ee,  on regarde dans quelle bo\^\i 
te se trouve la boule $N^{\rm o}1$, disons que c'est la bo\^\i te $G$.  
Puis on regarde dans quelle bo\^\i te se trouve la boule $N^{\rm o}2$, 
disons que c'est la  bo\^\i te $U$, la boule $N^{\rm o}3$ serait dans la  
bo\^\i te $S$, la  boule $N^{\rm o}4$ dans la bo\^\i te $T$, la boule 
$N^{\rm o}5$ dans la bo\^\i te $A$, la boule $N^{\rm o}6$ dans  
la bo\^\i te $V$, et la boule $N^{\rm o}7$ serait dans la bo\^\i te 
$E$. On obtient ainsi un mot de sept lettres, $GU\! ST\! AV\! E$. 
Quiconque conna\^\i t la r\`egle de codage peut reconstituer la 
distribution des boules dans les bo\^\i tes \`a partir du mot  
$GU\! ST\! AV\! E$: il prend la premi\`ere lettre, $G$, et place donc la 
boule $N^{\rm o}1$ dans la   bo\^\i te $G$, et ainsi de suite. \`A chaque 
distribution possible de $n$ boules dans $r$ bo\^\i tes, correspond de 
fa\c con biunivoque un mot de $n$ lettres \'ecrit avec les $r$ lettres 
qui ont servi \`a d\'esigner les bo\^\i tes. Il y a donc un {\it 
isomorphisme} entre les deux probl\`emes.  
\medskip 
Un autre exemple est celui des tirages dans une urne. Une urne contient  
$r$ boules de couleurs diff\'erentes, par exemple ($r=5$) une boule 
blanche, une boule noire, une rouge, une verte, une jaune. On tire  une 
boule, puis on la remet dans l'urne (apr\`es avoir not\'e sa couleur) on  
tire \`a nouveau, on remet, etc. Combien de r\'esulats diff\'erents 
sont possibles en $n$ tirages ? r\'eponse: $r^n$, puisque chaque  
r\'esultat possible est la liste des $n$ couleurs successives: si on a 
tir\'e, disons, la noire, puis la rouge, puis \`a nouveau la noire, puis la 
verte ($n=4$),  on aura la liste $N\! RNV$, de sorte que chaque tirage est 
cod\'e par un mot de quatre lettres \'ecrit dans l'alphabet $B,\, N,\, R,\, 
V,\, J$. C'est pourquoi ce type de probl\`eme est souvent --~comme  
dans le titre du paragraphe~--\hskip6pt d\'esign\'e par l'expression 
``tirages  avec remise'' ou en anglais ``samples with replacement''. 
 
\vskip12pt plus8pt minus5pt 
 
{\bf II. 2. Suites sans r\'ep\'etition ou tirages sans remise.}  
\vskip6pt plus4pt minus3pt 
 
Avec un alphabet de $r$ lettres, combien de mots diff\'erents de $n$ 
lettres {\it toutes distinctes} peut-on \'ecrire? Il s'agit du m\^eme 
probl\`eme qu'au paragraphe pr\'ec\'edent, except\'e que les mots ne  
doivent pas contenir deux fois la m\^eme lettre. 
\medskip 
Avec l'alphabet des trois lettres $A,B,C$, on peut \'ecrire trois 
mots diff\'erents form\'es d'une seule lettre: $A$, $B$, et $C$. Mais on 
ne peut plus \'ecrire neuf mots de deux lettres, car dans les mots $AA$, 
$BB$, et $CC$ une m\^eme lettre est r\'ep\'et\'ee. Il n'y a donc que six 
possibilit\'es:   
$$\matrix{  
                  &          \quad  &AB\; , \quad  &AC\; , \quad \cr  
                  &BA\; , \quad  &                     &BC\; , \quad \cr  
                  &CA\; , \quad  &CB\; , \quad  &                    \cr }$$ 
Il n'y a plus que six mots de trois lettres: 
$$\matrix{ &\hphantom{ABC\; ,  \!\!\! } &\hphantom{ABC\; ,  \!\!\! }   
&\hphantom{ABC\; ,  \!\!\! }  
 &\hphantom{ABC\; ,  \!\!\! }  &\hphantom{ABC\; ,  \!\!\! }   &ABC\; ,  
\!\!\! &\hphantom{ABC\; ,  \!\!\! }  &ACB\; , \!\!\!  &\hphantom{ABC\; ,  
\!\!\! }  \cr  
&\hphantom{ABC\; , \!\!\! }   &\hphantom{ABC\; , \!\!\! }  &BAC\; , \!\!\!  
&\hphantom{ABC\; ,  \!\!\! }  &\hphantom{ABC\; ,  \!\!\! }  
&\hphantom{ABC\; ,  \!\!\! }   &BCA\; , \!\!\!  &\hphantom{ABC\; , \!\!\! }  
&\hphantom{ABC\; ,  \!\!\! } \cr                    
&\hphantom{ABC\; ,  \!\!\! }  &CAB\; , \!\!\!  &\hphantom{ABC\; , \!\!\! }  
&CBA\; , \!\!\!  &\hphantom{ABC\; , \!\!\! }  & \hphantom{ABC\; , \!\!\! }  
&\hphantom{ABC\; ,  \!\!\! }   &\hphantom{ABC\; ,  \!\!\! }   
&\hphantom{ABC\; ,  \!\!\! }  \cr }$$ 
Les deux tableaux sont identiques \`a ceux du paragraphe pr\'ec\'edent, 
sauf qu'on y a effac\'e les mots contenant deux fois une m\^eme lettre. 
En effa\c cant simplement sur les tableaux d\'ej\`a constitu\'es du 
paragraphe pr\'ec\'edent ce qui ne doit plus y figurer, on ne fait pas 
appara\^\i tre un proc\'ed\'e syst\'ematique, mais il n'est pas difficile  
d'en faire appara\^\i tre un. En effet, si la premi\`ere  lettre est 
donn\'ee, par exemple $A$, il y a pour la seconde non plus trois, mais 
deux possibilit\'es, car $A$ ayant \'et\'e choisie pour la premi\`ere 
place, elle est maintenant exclue du choix suivant, qui se fera entre 
$B$ et $C$; de m\^eme si $B$ avait \'et\'e pris en premier, le choix 
pour la deuxi\`eme place se ferait entre $A$ et $C$. Pour {\it chacun} 
des trois choix possibles de la premi\`ere lettre, il y a donc  deux 
choix possibles pour la deuxi\`eme lettre, soit entre  $B$ et $C$, soit  
entre  $A$ et $C$, soit entre  $A$ et $B$, ce qui fait $3 \times 2 = 6$ 
choix en tout. Si les deux premi\`eres lettres sont donn\'ees, il n'y a 
plus qu'un choix possible pour la troisi\`eme, qui est $C$ si les deux 
premi\`eres sont $AB$ ou $BA$, $B$ si les deux premi\`eres sont $AC$ 
ou $CA$, $A$ si les deux premi\`eres sont $BC$ ou $CB$, ce qui fait 
bien six choix en tout. Si on voulait poursuivre et ajouter une 
quatri\`eme lettre, on verrait qu'il reste z\'ero choix possibles, 
c'est-\`a-dire qu'on ne peut pas ajouter une quatri\`eme lettre sans 
forc\'ement r\'ep\'eter l'une des trois premi\`eres.   
\medskip 
Plus g\'en\'eralement, avec un alphabet de $r$ lettres, il y a $r$ choix 
possibles pour la premi\`ere lettre; celle-ci ne pouvant plus \^etre 
r\'eutilis\'ee, il reste $r-1$ choix possibles pour la deuxi\`eme; 
c'est-\`a-dire que pour {\it chacun} des $r$ choix de la premi\`ere, il y  
a  $r-1$ choix pour la deuxi\`eme, ce qui fait $r\cdot (r-1)$ choix pour 
les deux premi\`eres; ensuite, pour {\it chacun} des $r\cdot (r-1)$  
choix des deux premi\`eres, il reste $r-2$ choix pour la troisi\`eme, ce 
qui fait $r\cdot (r-1)\cdot (r-2)$ choix pour les trois premi\`eres; et 
ainsi de suite. Lorsque les $n-1$ premi\`eres lettres sont d\'ej\`a 
choisies, il reste $r-n+1$  possibilit\'es pour la $n$-i\`eme. De sorte 
que pour avoir un mot de $n$ lettres, on aura en tout  
$$r(r-1)(r-2) \cdots (r-n+1) \eqno (II.2.)$$ 
possibilit\'es. On remarquera que si $n\leq r$ ce nombre est aussi  
\'egal  \`a $r! / (r-n)!$ Bien entendu, si $r < n$ c'est nul, puisqu'apr\`es 
avoir \'ecrit une fois chaque lettre de l'alphabet disponible, on ne peut 
en ajouter une $(r+1)$-i\`eme sans r\'ep\'etition. 
\medskip 
Tout comme le probl\`eme du \S1, celui-ci sert de mod\`ele pour  
d'autres situations. Ainsi nous avions vu que les mots de $n$ lettres 
\'ecrits dans un alphabet de $r$ lettres permettaient de coder 
biunivoquement des distributions de boules dans des bo\^\i tes: si on 
consid\`ere uniquement des mots dont toutes les lettres sont 
diff\'erentes, les distributions de boules correspondantes sont des 
distributions o\`u aucune bo\^\i te ne re\c coit deux boules (aucune 
lettre n'est \'ecrite \`a deux places dans le mot): ce sont donc les 
distributions o\`u il y a au plus une boule par bo\^\i te. Il faut donc 
retenir que {\bf le nombre de distributions de $n$ boules dans $r$  
bo\^\i tes avec au plus une boule par bo\^\i te} est donn\'e par $(II.2.)$ 
\medskip 
Autre probl\`eme r\'epondant au m\^eme mod\`ele: les tirages sans  
remises: une urne contient $r$ boules de couleurs diff\'erentes; on tire 
successivement une premi\`ere boule, puis une deuxi\`eme, puis une 
troisi\`eme, sans jamais les remettre dans l'urne; la deuxi\`eme ne peut 
donc pas avoir la m\^eme couleur que la premi\`ere puisqu'elle n'est 
plus dans l'urne, ni la troisi\`eme avoir la couleur de l'une des deux 
premi\`eres, etc. Le nombre de tirages possibles est donc \'egalement 
donn\'e par $(II.2.)$ 
 
\bigskip 
 
{\eightpoint On peut illustrer les deux pr\'ec\'edents paragraphes 
imm\'ediatement par le probl\`eme suivant: 
\smallskip 
{\bf on \'ecrit 5 chiffres d\'ecimaux au hasard; quelle est la probabilit\'e  
pour qu'ils soient tous diff\'erents?} 
\smallskip 
{\bf r\'eponse:} l'ensemble $\Omega$ de toutes les \'epreuves possibles  
est l'ensemble de toutes les suites de cinq chiffres pris dans l'alphabet 
$\{ 0,\,  1,\,  2,\,  3,\,  4,\,  5,\,  6,\,  7,\,  8,\,  9 \}$. 
(ce qui revient \`a l'ensemble de tous les nombres entiers de 0 \`a $99\, 
999$, en consid\'erant que le nombre 67 par exemple s'\'ecrit $00067$);  
il y en a $100\, 000$ en tout d'apr\`es $(II.1.)$ L'\'ev\'enement qui nous 
int\'eresse  est l'ensemble $A$ de tous les nombres dont les chiffres sont 
tous diff\'erents: il y en a $10 \cdot 9 \cdot 8 \cdot 7  \cdot 6 = 30\, 
240$ d'apr\`es $(II.2.)$. La probabilit\'e de cet \'ev\'enement est donc    
$${\cal P}(A) = {\# A \over \#\Omega } = {30\, 240 \over 100\, 000} = 
0,302\, 4$$   
De m\^eme, pour des nombres \`a $j$ chiffres, si on d\'esigne par $A_j$ 
l'\'ev\'enement: ``les $j$ chiffres  sont tous diff\'erents'': 
 $$\openup 2\jot \eqalignno{ 
 &{\cal P} (A_6) = {10 \cdot 9 \cdot 8 \cdot 7 \cdot 6 \cdot 5 \over 
1\, 000\, 000} = 0.151\, 2 \cr 
 &{\cal P} (A_7) = {10 \cdot 9 \cdot 8 \cdot 7 \cdot 6 \cdot 5 \cdot 4  
\over 10\, 000\, 000} = 0.060\, 48 \cr 
 &{\cal P} (A_8) = {10 \cdot 9 \cdot 8 \cdot 7 \cdot 6 \cdot 5 \cdot 4  
\cdot 3 \over 100\, 000\, 000} = 0.018\, 144 \cr 
 &{\cal P} (A_9) = {10 \cdot 9 \cdot 8 \cdot 7 \cdot 6 \cdot 5 \cdot 4  
\cdot 3 \cdot 2 \over 1\, 000\, 000\, 000} = 0.003\, 628\, 8 \cr 
 &{\cal P} (A_{10}) = {10 \cdot 9 \cdot 8 \cdot 7 \cdot 6 \cdot 5 \cdot 4  
\cdot 3 \cdot 2 \cdot 1 \over 10\, 000\, 000\, 000} = 0.000\, 362\, 88  
\cr &{\cal P} (A_j) = 0 \quad \hbox{pour $j > 10$} \cr }$$ 
 
\medskip 
 
Une variante de ce probl\`eme est le suivant:  
\smallskip 
{\bf On prend au hasard un groupe de vingt \'etudiants; quelle est la 
probabilit\'e pour que deux (au moins) d'entre eux aient leur anniversaire  
le m\^eme jour ?} 
\smallskip 
{\bf r\'eponse:} le probl\`eme se mod\'elise de la mani\`ere suivante. On 
consid\`ere que les jours de l'ann\'ee sont \'equiprobables pour les  
naissances (ce qui est faux, mais grossi\`erement approximatif); de plus  
on n\'eglige les cas de naissance un 29 f\'evrier. Pour une liste donn\'ee  
des vingt \'etudiants, par exemple dans l'ordre alphab\'etique, on a \`a 
consid\'erer toutes les listes possibles pour leurs jours de naissance, 
c'est-\`a-dire toutes les listes possibles de vingt dates parmi 365 
possibles; ces listes sont au nombre de $365^{20}$. Par ailleurs, 
l'\'ev\'enement $A$ ``au moins deux sont n\'es le m\^eme jour'' est le 
compl\'ementaire de l'\'ev\'enement $B$ ``toutes les dates sont 
diff\'erentes''.  L'\'ev\'enement $B$ contient $365 \cdot 364 \cdot  363 
\cdots 346$ \'el\'ements, de sorte que  
$$ \eqalign{  
{\cal P} (B) &= {365 \cdot 364 \cdot  363 \cdots 346 \over 365^{20}} \cr 
&= \Bigl(1-{1 \over 365}\Bigr) \cdot \Bigl(1-{2 \over 365}\Bigr)  
\cdot \Bigl(1-{3 \over 365}\Bigr) \cdots \Bigl(1-{19 \over 365}\Bigr) 
\cr } \eqno (II.3.)$$ 
On \'evalue cette expression en prenant le logarithme, sachant que 
 $\ln (1 - \varepsilon ) \simeq -\varepsilon $.  Le logarithme du produit  
est la somme des logarithmes, donc 
$$\eqalign{\ln  {\scriptstyle \Bigl\{ \!\! \Bigl( \! 
\hbox{$ 1 {\sevenmi -} { \hbox{1} \over \hbox{\vrule height8pt 
width0pt $365$} }$} \!  \Bigr)\!\! \Bigl( \!  
\hbox{$ 1 {\sevenmi -} { \hbox{2} \over \hbox{\vrule height8pt 
width0pt $365$} }$} \! \Bigr) \cdots 
\Bigl( \!  \hbox{$ 1 {\sevenmi -} { \hbox{$19$} \over \hbox{\vrule 
height8pt width0pt $365$} }$} \!  
\Bigr)\!\! \Bigr\} } &\simeq   - { 1+2+ \cdots + 19 \over 365 } \cr  
 &= - {19 \cdot 20 \over 2 \cdot 365} \simeq -0.5 \cr }$$ 
Ainsi on obtient la valeur approch\'ee 
$${\cal P} (B) \simeq e^{-0.5} \simeq 0.6$$ 
d'o\`u on d\'eduit 
$${\cal P} (A)  \simeq 0.4$$ 
Autrement dit, il y a $40\%$ de chances pour que deux \'etudiants aient  
leur anniversaire le m\^eme jour. Le calcul approch\'e qui a \'et\'e fait 
pour obtenir ce r\'esultat peut sembler tr\`es grossier (en fait la valeur 
exacte de l'expression (II.3.) est $0.588\, 561\, 535$ soit une erreur de 
$1\%$), mais l'hypoth\`ese de l'\'equiprobabilit\'e des jours de l'ann\'ee  
est au moins aussi grossi\`ere; on pourrait aussi s'interroger sur 
l'int\'er\^et de conna\^\i tre le r\'esultat au milli\`eme pr\`es. 
}  %%% end of \eightpoint 
\medskip 
Un cas particulier qui vaut la peine d'\^etre soulign\'e est le cas o\`u la 
suite ordonn\'ee utilise {\it toutes} les lettres disponibles, c'est-\`a-dire 
$n = r$. La question devient alors: {\bf de combien de mani\`eres  
diff\'erentes peut-on ordonner  $r$ objets ?}. Il s'agit du {\it nombre de 
permutations} de $r$ objets. En prenant simplement $n = r$ dans $(II.2.)$  
on obtient:  
$$ \hbox{nombre de permutations de $r$ objets} = r! \eqno (II.4.)$$ 
Dans l'exemple ci-dessus (probabilit\'e pour qu'un nombre de $j$ chiffres 
d\'ecimaux pris au hasard ait tous ses chiffres diff\'erents) cela 
correspond au cas de dix chiffres (nombres compris entre z\'ero inclu  
et dix milliards exclu). 
\medskip 
Bien entendu, dans aucun cas on ne peut avoir $n > r$, puisque si on doit 
\'ecrire plus de lettres qu'il n'y en a dans l'alphabet disponible, on ne  
peut qu'en r\'ep\'eter.  
 
\bigskip 
 
{\bf II. 3.  Combinaisons.} 
\medskip 
Lorsqu'on fait le produit de deux expressions alg\'ebriques, par exemple  
$(x_1 + y_1) \cdot (x_2 + y_2)$ on utilise la distributivit\'e de la 
multiplication: $(x_1 + y_1) \cdot (x_2 + y_2) = x_1x_2 + x_1y_2 + 
y_1x_2 + y_1x_1$. De m\^eme $(x_1 + y_1) \cdot (x_2 + y_2)  \cdot 
(x_3 + y_3) = x_1x_2x_3 + x_1y_2x_3 + y_1x_2x_3 + y_1x_1x_3 +  
x_1x_2y_3 + x_1y_2y_3 + y_1x_2y_3 + y_1x_1y_3$. Les mon\^omes 
qu'on obtient dans l'expression d\'evelopp\'ee sont les mots de deux  
(pour le produit de deux facteurs) ou trois (pour le produit de trois 
facteurs) lettres \'ecrits avec l'alphabet $\{ x,\, y \}$; l'indice ne sert 
ici qu'\`a indiquer la place de la lettre dans le mot. Le nombre de mots 
est, conform\'ement \`a $(II.1.)$, $2^2 = 4$ pour deux facteurs, $2^3 = 8$ 
pour trois facteurs. Si on fait un produit de deux ou trois facteurs 
identiques, on obtient les d\'eveloppements sans indices: 
$$\displaylines{ 
(x+y) \cdot (x+y) = xx + xy + yx + yy \cr 
(x+y) \cdot (x+y) \cdot (x+y) = xxx + xyx + yxx + yyx + xxy + xyy + yxy + 
yyy \cr }$$ 
Plus g\'en\'eralement, si on d\'eveloppe un produit de $n$ fois le  
facteur $(x+y)$, le d\'eveloppement est la somme de tous les mon\^omes  
possibles, c'est-\`a-dire la somme de tous les mots de $n$ lettres 
qu'on peut former avec les deux lettres $x$ et $y$. Si on tient compte 
de la commutativit\'e de la multiplication, les mon\^omes $xyx$, $yxx$, 
et $xxy$ sont \'egaux; on regroupe donc leur somme en $3x^2y$; de 
m\^eme les mon\^omes $yyx$, $xyy$, et $yxy$ peuvent \^etre regroup\'es 
en $3xy^2$. Ainsi se pose la question: parmi les $2^n$ mots de $n$ 
lettres qu'on peut former avec les deux lettres $x$ et $y$, combien 
comportent $k$ fois la lettre $x$ et $n-k$ fois la lettre $y$? La 
formule connue du bin\^ome de Newton r\'epond \`a cette question: elle 
nous dit que  
$$(x+y)^n = \sum_{k=0}^{k=n} {n \choose k}\; x^ky^{n-k}\quad ,$$ 
c'est-\`a-dire que le nombre de mon\^omes \'egaux \`a $x^k y^{n-k}$ (et 
qui par cons\'equent peuvent \^etre regroup\'es) est \'egal au  
$k$-i\`eme coefficient du bin\^ome d'ordre $n$  
$${n \choose k} = {n! \over k!\, (n-k)!} = {n(n-1(n-2) \cdots (n-k+1) \over 
k!} \eqno (II.5.)$$ 
Ainsi, {\bf le nombre de mots de $n$ lettres qu'on peut former avec un 
alphabet de  deux lettres, et qui comportent $k$ fois l'une et $n-k$ fois 
l'autre} est \'egal au coefficient bin\^omial ${n \choose k}$. 
\medskip 
Nous avons vu au \S1 qu'il y avait isomorphisme ou \'equivalence entre  
les mots et les distributions de boules dans des bo\^\i tes; traduisons 
ce que nous venons de constater pour les mon\^omes en termes de 
boules: cela correspond donc \`a $n$ boules \`a distribuer dans $r=2$ 
bo\^\i tes. Ainsi il y a ${n \choose k}$ mani\`eres de distribuer $n$  
boules dans deux bo\^\i tes, de telle fa\c con qu'il y en ait $k$ dans  
l'une et $n-k$ dans l'autre. Placer $k$ (parmi $n$) boules dans une  
bo\^\i te (il en restera alors forc\'ement $n-k$ pour l'autre bo\^\i te) 
revient \`a choisir un sous-ensemble de $k$ boules parmi l'ensemble  
des $n$ boules.  
\medskip 
On remarquera qu'il s'agit de sous-ensembles, c'est-\`a-dire que  
l'ordre dans lequel les {\it \'el\'ements} du sous-ensemble sont  
donn\'es ne joue pas: les num\'eros des $k$ boules qui sont dans la 
bo\^\i te $x$ forment un sous-ensemble \`a $k$ \'el\'ements de 
l'ensemble $\{ 1,\,   2,\,  3,\,  \cdots n \}$. Mais ce sous-ensemble non 
ordonn\'e d\'etermine l'ordre dans lequel les lettres $x$ et $y$  
apparaissent dans le mon\^ome. Si  par exemple $n=5$ et $k=3$, le  
sous-ensemble  $\{ 2,\,  4,\,  5 \}$ est  le m\^eme que $\{ 4,\,  5,\,  2  
\}$ ou $\{ 5,\,  4,\,  2 \}$. Le mon\^ome correspondant \`a ce  
sous-ensemble est  $yxyxx$ (c'est-\`a-dire que la lettre $x$ occupe  
les places $2$, $4$, et $5$, peu importe dans quel ordre ces num\'eros 
de places ont \'et\'e dict\'es); si on m\'elange l'ordre des lettres {\it 
dans le  mon\^ome}, pour obtenir par exemple le mon\^ome $xyxyx$ (qui 
est   donc \'egal au pr\'ec\'edent compte tenu de  la commutativit\'e de 
la  multiplication), ce nouveau  mon\^ome correspond \`a un {\it autre}  
sous-ensemble, \`a savoir $\{ 1,\,  3,\,  5 \}$.  
\medskip 
Traduit en termes de tirages dans des urnes, le probl\`eme revient \`a 
ceci: au \S1 on caract\'erisait un tirage par la suite ordonn\'ee des 
couleurs obtenues successivement (cela formait un mot). Si on traduit 
de la m\^eme fa\c con, nous avons ici deux boules, par exemple une noire 
et une rouge. On tire $n$ fois avec remise. En notant \`a chaque fois la 
couleur tir\'ee, $x$ pour rouge et $y$ pour noire, on obtient un  
mon\^ome en $x$ et $y$. 
\smallskip 
Une autre fa\c con d'habiller le m\^eme probl\`eme, toujours  
\'equivalente,  consiste, puisqu'il n'y a que deux couleurs ou deux 
lettres, \`a d\'enombrer les suites de r\'esultats possibles obtenus en 
lan\c cant $n$ fois une pi\`ece de monnaie: on \'ecrit $x$ quand sort  
pile et $y$ quand sort face. Si un joueur donne un franc \`a l'autre  
joueur chaque fois que sort pile (et inversement quand sort face), le 
gain est \'egal au nombre de $x$ moins le nombre de $y$; si $k$ est le 
nombre de $x$, le nombre de $y$ sera $n-k$, donc le gain sera $w = k - 
(n-k) = 2k - n$.  Ce gain ne d\'epend donc pas de la place des $x$ et des 
$y$ dans le mon\^ome, mais seulement du nombre $k$ de $x$, 
c'est-\`a-dire du nombre de pile. Par cons\'equent le nombre de  
r\'esultats diff\'erents apportant un  m\^eme gain $w = 2k-n$ est ${n 
\choose k}$.  
\medskip 
{\eightpoint 
Les coefficients bin\^omiaux sont aussi donn\'es par le triangle de 
Pascal, qu'on obtient en \'ecrivant pour chaque $n$ les $2n+1$ 
coefficients bin\^omiaux  d'ordre $n$ sur une ligne, la premi\`ere ligne 
correspondant \`a $n=1$, la seconde \`a $n=2$, etc. Cela donne pour les 
lignes 0 \`a 13:  
$$\hskip-12pt\matrix{ 
&1  &    &    &    &    &    &    &    &    &    &    &    &    &  \cr 
&1  &1   &    &    &    &    &    &    &    &    &    &    &    &  \cr 
&1  &2   &1   &    &    &    &    &    &    &    &    &    &    &  \cr 
&1  &3   &3   &1   &    &    &    &    &    &    &    &    &    &  \cr 
&1  &4   &6   &4   &1   &    &    &    &    &    &    &    &    &  \cr 
&1  &5   &10  &10  &5   &1   &    &    &    &    &    &    &    &  \cr 
&1  &6   &15  &20  &15  &6   &1   &    &    &    &    &    &    &  \cr  
&1  &7   &21  &35  &35  &21  &7   &1   &    &    &    &    &    &  \cr  
&1  &8   &28  &56  &70  &56  &28  &8   &1   &    &    &    &    &  \cr  
&1  &9   &36  &84  &126 &126 &84  &36  &9   &1   &    &    &    &  \cr  
&1  &10  &45  &120 &210 &252 &210 &120 &45  &10  &1   &    &    &  \cr  
&1  &11  &55  &165 &330 &462 &462 &330 &165 &55  &11  &1   &    &  \cr  
&1  &12  &66  &220 &495 &792 &924 &792 &495 &220 &66  &12  &1   &  \cr  
&1  &13  &78  &286 &715 &1287&1716&1716&1287&715 &286 &78  &13  &1 \cr  }$$ 
Le triangle de Pascal a la propri\'et\'e suivante: chaque \'el\'ement  
d'une ligne est la somme de deux \'el\'ements de la ligne pr\'ec\'edente: 
celui qui est au-dessus et son voisin de gauche;  ainsi $1716 = 924 + 
792$,  $165 =120 + 45$, etc. 
Cela correspond \`a la relation de r\'ecurrence suivante: 
$${n-1 \choose k} + {n-1 \choose k-1} = {n \choose k}$$ 
Cette relation de r\'ecurrence est tr\`es facile \`a v\'erifier  
directement.  En effet, en \'ecrivant les coefficients bin\^omiaux avec 
des factorielles,  selon $(II.5.)$ cela revient \`a 
$${(n-1)! \over k!\, (n-1-k)!} + {(n-1)! \over (k-1)!\, (n-k)!} = {(n-1)!  
\over k!\, (n-k)!}\; \bigl[ (n-k) + k \bigr] = {n! \over k!\, (n-k)!}$$ 
Si on se propose de calculer {\it num\'eriquement} les coefficients  
bin\^omiaux, il se trouve que cette formule de r\'ecurrence est 
l'algorithme le plus efficace qu'on connaisse. La raison en est que si on 
le met en oeuvre, il n'effectue que des additions, alors qu'un calcul 
m\^eme habile des factorielles demande des multiplications et des 
divisions. Son  d\'efaut est que, m\^eme si on veut ne calculer qu'un seul 
coefficient bin\^omial, par exemple ${72 \choose 27}$, on doit calculer les 
lignes compl\`etes du triangle de Pascal; le nombre d'additions n\'ecessaire  
est alors environ $2500$. Calculer directement en passant par les 
factorielles est visiblement maladroit:  $72!$ exige $72$ 
multiplications, $27$ en exige $27$ et $(72-27)!$ en exige $72-27$, soit 
$144$ multiplications en tout. En outre, en proc\'edant ainsi, on calcule 
d'abord s\'epar\'ement des nombres {\it bien plus gros} que celui qu'on veut 
finalement obtenir (${72 \choose 27}$ est grand aussi,  mais nettement 
moins que $72!$);  on gaspille ainsi du temps \`a calculer des nombres 
entiers tr\`es longs,  que l'on divise ensuite.  On pourrait certes faire
une \'economie en calculant par exemple ${72 \over 27} \cdot {71 \over 26}  
\cdot {70 \over 25} \cdots {47 \over 2} \cdot 46$, mais les diff\'erents 
facteurs sont alors des fractions et ne peuvent donc \^etre manipul\'es 
qu'en virgule flottante (donc en perdant l'exactitude). En utilisant le 
triangle de Pascal, les calculs interm\'ediaires ne font jamais appel \` a 
des nombres plus gros que celui qu'on veut atteindre, et on ne fait 
intervenir que des entiers. En fin de compte on y gagne, surtout si on n'en 
veut pas un seul coefficient, mais par exemple toute une ligne. \par} 
\medskip 
 
\def\ldown#1{\hbox{\raise1.5pt\hbox{$#1$}}} 
 
Lorsque $k$ varie de $0$ \`a $n$, le coefficient bin\^omial  ${n 
\choose k}$ est croissant tant que $2k < n$, puis d\'ecroissant quand 
$2k > n$; si $n$ est pair (disons $n = 2p$), il est maximum pour  
$k=p$ et vaut alors $(2p)! / p!^2$; si $n$ est impair (disons $n = 2p-1$), 
il est maximum ex aequo pour  $k=p-1$ et $k=p$ et vaut alors $(2p-1)! / 
p!\, (p-1)! = {1 \over 2}\cdot (2p)! / p!^2$. Pour s'en rendre compte, il 
suffit de comparer ${n \choose k+1}$ \`a ${n \choose k}$: en divisant 
les factorielles, on voit que  
$${n \choose k+1} = {n-k \over k+1} \cdot {n \choose k}$$ 
la suite des ${n \choose k}$ est donc croissante tant que $(n-k) / 
(k+1) \geq 1\quad$ ($\Leftrightarrow 2k \leq n-1$)  et d\'ecroissante 
quand $(n-k) / (k+1) \leq 1\quad$($\Leftrightarrow 2k \geq n-1$), 
{\eightrm C.Q.F.D.} 
\medskip 
Le calcul num\'erique exact des coefficients bin\^omiaux ${n \choose k}$  
(par exemple avec l'algorithme du triangle de Pascal) devient tr\`es ardu 
lorsque $n$ est tr\`es grand; le calcul op\`ere sur des entiers, mais 
lorsqu'ils sont grands on ne peut se contenter de deux octets et les temps 
de calcul deviennent vite prohibitifs. En outre, il est bien rare que pour  
$n$ grand une expression exacte pr\'esente seulement un int\'er\^et. Or  
il se trouve qu'il existe une  expression approch\'ee simple et tr\`es utile 
des coefficients bin\^omiaux. Voyons le cas o\`u $n$ est pair ($n=2p$); le   
cas $n$ impair doit \^etre trait\'e s\'epar\'ement, mais il est analogue. 
Posons $k = p+j$ pour sym\'etriser ($j$ est alors nul quand $k = p$ et $j$ 
varie de $-p$ \`a $+p$).  On a bien s\^ur ${2p \choose p-j} = {2p \choose 
p+j}$, et il sufit  d'examiner le cas o\`u $j > 0$.  
\vskip8pt plus7pt minus4pt 
En d\'ecomposant convenablement les factorielles qui figurent au 
nu\-m\'e\-ra\-teur et au d\'e\-no\-mi\-na\-teur des coefficients 
bin\^omiaux, on peut \'ecrire, si $j$ est positif:   
$$\eqalignno {{2p 
\choose p+j} &= {2p \choose p} \; {p(p-1)(p-2) \cdots (p-j+1) \over 
(p+1)(p+2) \cdots (p+j) } \cr 
\noalign{\medskip} 
&= {2p \choose p}\; {\vrule depth12pt width0pt  
p^j \; \biggl[{\displaystyle \biggl( 1 - {1 
\over\ldown p}\biggr) \biggl( 1 - {2 \over\ldown p}\biggr) \cdots 
\biggl( 1 - {j-1 \over\ldown p}\biggr)}\biggr]  
\over  
\vrule height19pt width0pt p^j \; \biggl[{\displaystyle\biggl( 1 + {1  
\over\ldown p}\biggr) \biggl( 1 + {2 \over\ldown p}\biggr) \cdots  
\biggl( 1 + {j  \over\ldown p}\biggr)}\biggr] } \cr }$$ 
Pour \'evaluer les expressions entre crochets en num\'erateur et  
d\'enominateur on prend leurs logarithmes; on sait que  
$$\ln (1 + \varepsilon ) = \varepsilon - {\varepsilon^2\over 2} + 
{\varepsilon^3\over 3} - \cdots$$ 
de sorte que pour l'expression au d\'enominateur on obtient 
$$\eqalign{\ln  {\scriptstyle \Bigl\{ \!\! \Bigl( \! 
\hbox{$ 1 {\sevenmi +} { \hbox{1} \over \hbox{\vrule height5pt 
width0pt $p$} }$} \!  \Bigr)\!\! \Bigl( \!  
\hbox{$ 1 {\sevenmi +} { \hbox{2} \over \hbox{\vrule height5pt 
width0pt $p$} }$} \! \Bigr) \cdots 
\Bigl( \!  \hbox{$ 1 {\sevenmi +} { \hbox{$j$}\vrule depth3pt width0pt  
\over \hbox{\vrule height5pt width0pt $p$} }$} \!  
\Bigr)\!\! \Bigr\} } &=   { 1+2+ \cdots + j \over p } - { 1+4+ \cdots +  
j^2 \over 2\, p^2 } + \cdots \cr  
 &= {j(j+1) \over 2\, p} - {O(j^3) \over 6\, p^2} \cr  
&\simeq  {j^2 \over 2\, p} }$$ 
Pour l'expression au num\'erateur (en utilisant cette fois $\ln (1 - 
\varepsilon ) = -\varepsilon - {\varepsilon^2\over 2} - {\varepsilon^3 
\over 3} - \cdots$) on obtiendrait de m\^eme  
$$\eqalign{\ln  {\scriptstyle \Bigl\{ \!\! \Bigl( \! 
\hbox{$ 1 {\sevenmi -} { \hbox{1} \over \hbox{\vrule height6pt 
width0pt $p$} }$} \!  \Bigr)\!\! \Bigl( \!  
\hbox{$ 1 {\sevenmi -} { \hbox{2} \over \hbox{\vrule height6pt 
width0pt $p$} }$} \! \Bigr) \cdots 
\Bigl( \!  \hbox{$ 1 {\sevenmi -} { \hbox{$j\! - \!1$}\vrule depth3pt 
width0pt  \over \hbox{\vrule height6pt width0pt $p$} }$} \!  
\Bigr)\!\! \Bigr\} } &=   -{1+\cdots + (j\! - \! 1) \over p} - { 
1+ \cdots +  j^2 \over 2\, p^2 } + \cdots \cr  
 &= -{j(j-1) \over 2\, p} - {O(j^3) \over 6\, p^2} \cr  
&\simeq  {-j^2 \over 2\, p} }$$ 
En revenant aux exponentielles des logarithmes et en regroupant tout, 
il appara\^\i t que  
$${2p \choose p+j} \simeq {2p \choose p} \; \exp \Bigl[ - {j^2\over 
\ldown p}\Bigr]$$  
Cette approximation est valable pourvu que l'erreur commise, qui  
comme nous l'avons vu au cours du calcul, est de l'ordre de $j^3 / p^2$, 
soit n\'egligeable. Or si $j$ est de l'ordre de $\sqrt{p}$ ou plus petit, 
$j^3  / p^2$ sera de l'ordre de $1/\sqrt{p}$ qui est n\'egligeable si $p$ 
est grand. Si $j$ est plus grand que $\sqrt{p}$ en ordre de grandeur, 
l'erreur n'est  plus n\'egligeable en ce sens que $j^3 / p^2$ n'est pas 
petit devant $j^2 /p$, mais  en fait cela ne porte pas \`a cons\'equence 
car dans ce cas le rapport ${2p \choose p+j} \bigl/ {2p \choose p}$  
est tellement petit que, erreur ou pas,  on peut le remplacer aussi bien 
par $0$ que par $\exp (-j^2/n)$.  
\medskip 
Quand \`a la valeur maximum ${2p \choose p} = (2p)! / p!^2$ 
elle-m\^eme, elle peut \^etre approch\'ee en utilisant la formule 
de Stirling: $p! \simeq p^pe^{-p} \sqrt{2\pi p}$. Ainsi 
$${2p \choose p} = (2p)! / p!^2 \simeq {2^{2p}\, p^{2p}\, 
e^{-2p}\,\sqrt{4\pi p} \over p^{2p}\, e^{-2p}\, 2\pi p} 
= {2^{2p} \over  \sqrt{\pi p}}$$ 
On peut donc conclure que pour $p$ grand:  
$${2p \choose p + j} = {2p \choose p - j} \simeq 
{ 2^{2p} \over \sqrt{\pi p} } \cdot \exp\Bigl[ - {j^2 \over\ldown 
p}\Bigr] \eqno (II.6.)$$ 
Pour le cas impair on aurait obtenu par des voies semblables: 
$${2p-1 \choose p + j} = {2p-1 \choose p - j - 1} \simeq 
{ 2^{2p-1} \over \sqrt{\pi p} } \cdot \exp\Bigl[ - {j^2 \over\ldown 
p}\Bigr] \eqno (II.6a.)$$ 
 
\bigskip 
 
{\bf II. 4.  Partitions en groupes de taille donn\'ee.} 
\medskip 
Maintenant la question est celle-ci: 
on se donne $m$ nombres entiers $\geq 0$, $n_1$, $n_2 \ldots n_m$, 
dont la somme est $n$; de combien de mani\`eres diff\'erentes peut-on  
ranger $n$ \'el\'ements en $m$ groupes de (respectivement) $n_1$, $n_2 
\ldots n_m$ \'el\'ements?  
\medskip 
On peut interpr\'eter le probl\`eme \'etudi\'e dans le paragraphe  
pr\'ec\'edent en disant qu'on cherchait le nombre de partitions de  
l'ensemble \`a $n$ \'el\'ements  en {\it deux} sous-ensembles ayant 
respectivement $k$ et $n-k$ \'el\'ements, ce qui \'etait donc le cas 
particulier correspondant \`a $m=2$, $n_1=k$ et $n_2=n-k$. C'est ce qui 
apparaissait quand le probl\`eme \'etait interpr\'et\'e en termes de boules 
\`a distribuer  dans des bo\^\i tes: il y avait alors $2^n$ distributions 
possibles de $n$ boules dans deux bo\^\i tes, et parmi celles-ci il y en  
avait ${n \choose k}$ pour lesquelles la premi\`ere bo\^\i te recevait $k$ 
boules et la deuxi\`eme $n-k$. Si le nombre de bo\^\i tes est quelconque,  
disons $m$, il s'agit des distributions de boules dans $m$ bo\^\i tes: le 
probl\`eme se g\'en\'eralise alors ainsi: {\bf quel est le nombre de 
partitions d'un ensemble \`a $n$ \'el\'ements en $m$ sous-ensembles  
ayant respectivement $n_1, \; n_2, \ldots  n_m$ \'el\'ements?} Les  
nombres $n_1, \; n_2, \ldots  n_m$ sont appel\'es les nombres  
d'occupation; $n_j$ est le nombre d'occupation de  la $j$-i\`eme bo\^\i te. 
\medskip 
Dans le paragraphe pr\'ec\'edent nous avons abord\'e le probl\`eme en  
\'etudiant le d\'eveloppement du bin\^ome: $(x+y)^n$. De la m\^eme fa\c 
con on peut aborder le probl\`eme du nombre de partitions \`a partir, non 
plus du {\it bi}n\^ome, mais de $(x_1 + x_2 + \cdots x_m)^n$: on est alors 
ramen\'e \`a compter, sur les $m^n$ mon\^omes de l'expression 
d\'evelopp\'ee, combien sont \'egaux \`a $x_1^{n_1} x_2^{n_2} x_3^{n_3} 
\cdots x_m^{n_m}$, ou encore (en termes de mots et de lettres): {\bf  
parmi les $m^n$ mots de $n$ lettres qu'on peut former avec l'alphabet  
$\{ x_1, \, x_2, \,  x_3, \ldots  x_m \}$  combien y en a-t-il qui 
contiennent $n_1$ fois la lettre $x_1$,  $n_2$ fois la lettre $x_2$,   
$n_3$ fois la lettre $x_3$, $\ldots$  $n_m$ fois la lettre $x_m$?}  
(nous y reviendrons un peu plus loin). 
\medskip 
Toujours au paragraphe pr\'ec\'edent, au lieu de poser le probl\`eme en  
termes de mots \'ecrits avec les deux lettres $x$ et $y$, on aurait aussi  
pu le poser de la mani\`ere suivante: pour un ensemble $E$ de $n$ objets il  
y a $n!$ permutations. Si on consid\`ere une partition en {\it deux} 
sous-ensembles $E_k$ et $E_{n-k}$, alors toute permutation qui permute 
les \'el\'ements de $E_{k}$ entre  eux et les \'el\'ements de $E_{n-k}$  
entre eux laisse ces deux ensembles intacts; or il y a $k!\, (n-k)!$ telles 
permutations. Toute autre  permutation de $E$ mo\-di\-fie les deux 
sous-ensembles, mais non leur nombre (2), ni le nombre de leurs 
\'el\'ements respectifs ($k$ et $n-k$); autrement dit, toute autre 
permutation transforme une partition en deux sous-ensembles \`a $k$ et 
$n-k$ \'el\'ements en une autre partition en deux sous-ensembles \`a $k$  
et $n-k$ \'el\'ements. Le nombre total de permutations (soit $n!$) est donc  
le produit du nombre de partitions en deux sous-ensembles \`a $k$ et   
$n-k$ \'el\'ements par le nombre de permutations qui laissent chacune 
intacte,  ou inversement, le nombre de partitions en deux sous-ensembles 
\`a $k$ et $n-k$ \'el\'ements est le quotient du nombre total de 
permutations par le nombre  de permutations qui laissent chacune intacte, 
soit $n!/k!\, (n-k)!$.  On retrouve ainsi $(II.5)$.  
\medskip  
Cette fa\c con de compter donne imm\'ediatement la r\'eponse \`a la 
question du pr\'esent paragraphe. Pour une partition en $m$ 
sous-ensembles $E_{n_1},\; E_{n_2}, \ldots  E_{n_m}$, toute 
permutation qui permute les \'el\'ements de chacun des $E_{n_j}$  entre 
eux (pour $j$ de 1 \`a $m$)  laisse ces $m$ ensembles intacts; or il y  a 
$n_1!\, n_2! \cdots n_m!$ telles permutations. Toute autre permutation  
de $E$ mo\-di\-fie les $m$ sous-ensembles, mais non leur nombre ($m$), ni 
le nombre de leurs  \'el\'ements, c'est-\`a-dire les nombres 
d'occupation ($n_1, \; n_2, \ldots  n_m$); autrement dit, toute autre 
permutation transforme une partition en $m$ sous-ensembles \`a $n_1, 
\; n_2, \ldots  n_m$ \'el\'ements en une autre partition en $m$ 
sous-ensembles \`a $n_1,  \; n_2, \ldots  n_m$ \'el\'ements. Le nombre  
total de permutations (soit $n!$) est donc le produit du nombre de 
partitions en $m$ sous-ensembles  \`a $n_1, \; n_2, \ldots  n_m$ 
\'el\'ements par le nombre de  permutations qui laissent chacune 
intacte, ou encore, le nombre de partitions en $m$ sous-ensembles \`a 
$n_1, \; n_2, \ldots  n_m$ \'el\'ements  est le quotient du nombre total 
de permutations par le nombre  de permutations qui laissent chacune 
intacte, soit    
$$n! \over n_1!\, n_2! \cdots n_m!\; . \eqno (II.7.)$$    
Pour donner un exemple de ce type de d\'enombrement, prenons notre 
promotion de $72$ \'etudiants. De combien de mani\`eres diff\'erentes  
peut-on les diviser en trois groupes de T.D. de 24 \'etudiants chacun? 
R\'eponse:  ${72! / 24!\, 24!\, 24!} \simeq 2.564 \cdot 10^{32}$.  De 
m\^eme, le nombre de mani\`eres diff\'erentes de r\'epartir $72$  
\'etudiants dans trois groupes de respectivement 30, 25, et 17 
\'etudiants est ${72! / 30!\, 25!\, 17!} \simeq 4.184 \cdot 10^{31}$. Dans 
des groupes de 32, 22, et 18: ${72! / 32!\, 22!\, 18!} \simeq 3.234 \cdot 
10^{32}$. Si les tailles des trois groupes sont davantage 
diff\'erenci\'ees, le nombre devient plus petit: ${72! / 60!\, 8!\, 4!} 
\simeq 7.605 \cdot 10^{15}$. En revanche pour des groupes plus nombreux 
le nombre de possibilit\'es augmente: ${72! / 32!\, 22!\, 10!\, 8!} \simeq 
1.4155 \cdot 10^{36}$. On pourrait montrer (d'ailleurs on le fera plus 
loin) que pour un nombre de groupes donn\'e, le nombre de possibilit\'es 
est maximum lorsque les  groupes sont de tailles \'egales, 
c'est-\`a-dire lorsque les nombres d'occupation sont \'egaux entre eux.  
\medskip  
On retrouve encore ces nombres lorsqu'on d\'eveloppe le polyn\^ome 
$(x_1 + x_2 + \cdots + x_m)^n$. Cela n'a rien de surprenant puisqu'ils ne 
font que g\'en\'eraliser les nombres $II.5.$ Ainsi: 
$$(x_1 + x_2 + \cdots + x_m)^n = \sum_{n_1+n_2+ \cdots +n_m=n}  {n! 
\over n_1!\,  n_2! \cdots n_m!}\, x_1^{n_1}\, x_2^{n_2} \cdots  
 x_m^{n_m} $$  
la sommation portant sur toutes les familles d'indices possibles qui 
v\'erifient $n_1 + n_2 +  \cdots  + n_m = n$. C'est pourquoi ces nombres  
sont appel\'es {\it coefficients multin\^omiaux}, de m\^eme que ceux de  
la section pr\'ec\'edente \'etaient appel\'es coefficients bin\^omiaux. 
Quoique cette notation soit moins c\'el\`ebre que celle des coefficients 
bin\^omiaux, on d\'esigne les coefficients  multin\^omiaux de mani\`ere 
analogue:  
$${n \choose n_1,\,  n_2,  \cdots   n_m} = {n \over n_1!\,  n_2! 
\cdots   n_m!}$$ 
 
De m\^eme que tous les probl\`emes de combinaisons de lettres que  
nous avons abord\'e jusqu'ici, le probl\`eme du nombre de partitions en 
groupes de taille donn\'ee peut s'interpr\'eter en termes de boules \`a 
ranger dans des bo\^\i tes. Lorsqu'il s'agissait de mots (de $n$ lettres) 
\'ecrits avec un alphabet de $m$ lettres, nous avons vu l'\'equivalence 
avec la r\'epartition de $n$ boules dans $m$ bo\^\i tes. Ici, il s'agit du 
nombre de mots diff\'erents qui contiennent {\it un nombre donn\'e de 
fois} chacune des $m$ lettres de l'alphabet. En termes de boules et de 
bo\^\i tes,  chaque bo\^\i te correspond \`a une lettre de l'alphabet, et 
chaque boule correspond \`a la place occup\'ee par une lettre dans le 
mot. Le probl\`eme revient alors au nombre de r\'epartitions pour 
lesquelles la bo\^\i te correspondant \`a la lettre $x_j$ contient $n_j$ 
boules: il s'agit donc du nombre de r\'epartitions diff\'erentes pour 
lesquelles chaque bo\^\i te contient un nombre fix\'e \`a l'avance de 
boules: {\bf de combien de mani\`eres diff\'erentes peut-on r\'epartir 
$n$ boules dans $m$ bo\^\i tes, de telle sorte que la premi\`ere en 
contienne $n_1$, la deuxi\`eme $n_2$, la troisi\`eme $n_3$, etc ?}  
\medskip 
De m\^eme que pour les coefficients bin\^omiaux, on peut avoir une 
approximation simple et pratique pour $n$ grand. On peut v\'erifier que  
$n! \bigl/ n_1!\, n_2!\, n_3! \cdots n_m!$ est maximum lorsque les 
nombres $n_1,\, n_2,\, n_3, \ldots n_m$ sont \'egaux (si $n$ est un 
multiple de $m$) ou \'egaux \`a une unit\'e pr\`es (si $n$ n'est pas un 
multiple de $m$). En utilisant la formule de Stirling, comme nous 
l'avons fait pour le maximum des coefficients bin\^omiaux, on obtient 
pour ce maximum l'approximation 
$${n! \over n_1!\, n_2!\, n_3! \cdots n_m!} \simeq 
{m^{n+{m\over 2}} \over \sdown{16}\sqrt{(2\pi n)^{m-1}}}$$ 
(\`a partir d'ici $n_1,\, n_2,\, n_3, \ldots n_m$ d\'esignent les 
valeurs correspondant au {\it maximum}). 
Pour conna\^\i tre la variation {\it autour} du maximum, on proc\`ede 
comme au \S 3, on introduit $j_1,\, j_2,\, j_3, \ldots j_m$ et on 
remarque que  
$$\eqalign{ &{n! \over (n_1+j_1)!\, (n_2+j_2)!\, (n_3+j_3)! \cdots 
(n_m+j_m)!}  = \cr 
=  &{n! \over n_1!\, n_2!\, n_3! \cdots n_m!} \cdot  {n_1! \over 
(n_1+j_1)!} \cdot {n_2! \over (n_2+j_2)!}\cdot {n_3! \over 
(n_3+j_3)!}\;\cdots\; {n_m! \over (n_m+j_m)!} \cr }$$ 
Chacun des facteurs ${n_i! / (n_i+j_i)!}$ peut \^etre approch\'e en 
utilisant le proc\'ed\'e que nous avons d\'ej\`a vu. Si $j_i > 0$  on  
\'ecrit 
$$\eqalign{ 
{n_i! \over (n_i+j_i)!} &= {1 \over \quad (n_i+1)(n_i+2)(n_i+3) \cdots 
(n_i+j_i)\quad }\cr  
\noalign{\medskip} 
&= {1 \over \quad\vrule height17pt width0pt n_i^{j_i} \; \biggl[ 
{\displaystyle\biggl( 1 + {1  \over\ldown n_i}\biggr) \biggl( 1 + {2 
\over\ldown n_i}\biggr) \cdots  \biggl( 1 + {j_i  \over\ldown 
n_i}\biggr)}\biggr]\quad } \cr 
\noalign{\medskip} 
&\simeq {1 \over n_i^{j_i}\vrule height11pt width0pt}\cdot 
\e^{-{\; j_i^2\strutb\over 
2n_i}} \cr }$$  
et si $j_i < 0$  
$$\eqalign{  
{n_i! \over (n_i+j_i)!} &= n_i(n_i-1)(n_i-2)(n_i-3)\cdots (n_i-|j_i|+1) \cr 
&= n_i^{|j_i|} \; \biggl[\biggl( 1 - {1  \over\ldown n_i}\biggr) \biggl( 1 -  
{2 \over\ldown n_i}\biggr) \cdots  \biggl( 1 - {|j_i|-1  \over\ldown 
n_i}\biggr)\biggr] \cr 
&\simeq {1 \over n_i^{j_i}\vrule height11pt width0pt}\cdot 
\e^{-{\; j_i^2\strutb\over 
2n_i}} \cr }$$  
(on voit que l'approximation a la m\^eme expression analytique, 
ind\'e\-pen\-dam\-ment du signe de $j_i$). Si maintenant on applique ce 
r\'esultat \`a  chacun des facteurs ${n_i! \bigl/ (n_i+j_i)!}$, on obtient  
$$\eqalign{ &{n! \over (n_1+j_1)!\, (n_2+j_2)!\, (n_3+j_3)! 
\cdots (n_m+j_m)!} \quad \simeq \cr 
\noalign{\medskip} 
\simeq \quad &{n! \over n_1!\, n_2!\, n_3! \cdots n_m!} \;\cdot\;   
{\e^{- {\; j_1^2\strutb\over 2n_1} - {\; j_2^2\strutb\over 2n_2} - {\; 
j_3^2\strutb\over 2n_3} \cdots - {\; j_m^2\strutb\over 2n_m} }\over 
n_1^{j_1} n_2^{j_2} n_3^{j_3}\vrule height13pt width0pt \cdots  
n_m^{j_m} } \qquad \cr } \eqno (II.8.)$$ 
Il ne faut pas oublier que la somme des $n_i$ est toujours \'egale \`a  
$n$ donc la somme des $j_i$ est nulle; l'exposant de l'exponentielle est  
la somme des $j_i^2 / 2n_i$, mais les $j_i$ ne sont pas ind\'ependants. 
\medskip 
Pour obtenir $(II.8.)$ nous n'avons pas utilis\'e le fait que les valeurs 
des nombres $n_1,\, n_2,\, n_3,\ldots n_m$   correspondent au 
maximum du coefficient multin\^omial; formellement, $(II.8.)$ est  
vrai pour {\it n'importe quelle valeur de} $n_1,\, n_2,\, n_3,\,\ldots 
n_m$. Mais si le facteur $1 \bigl/ n_1^{j_1} n_2^{j_2} n_3^{j_3} \cdots 
n_m^{j_m}$ n'est pas {\it stationnaire}, cette formule ne sert \`a rien, 
car lorsque les $j_i$ varient, ce facteur peut augmenter bien plus vite 
que la fonction $\exp (-j_i^2/2n_i)$ ne diminue, de sorte qu'il ne sert 
plus \`a rien d'avoir mis en \'evidence ce facteur gaussien. Dire que le 
facteur est stationnaire signifie qu'il ne varie pas (ou tr\`es peu)  
lorsque les $j_i$ s'\'ecartent de $0$; or cela se produit pr\'ecis\'ement 
lorsque les $n_i$ sont \'egaux: dans ce cas $n_1^{j_1} n_2^{j_2}  
n_3^{j_3} \cdots n_m^{j_m} = n_1^{j_1+j_2+j_3+ \cdots +j_m} = n_1^0 = 
1$. Il est \'evident que la fonction $\exp\, (-\sum_i j_i^2/2n_i)$ est 
maximum quand les  $j_i$ sont tous nuls; mais pour que le {\it produit} 
de $1 \bigl/ n_1^{j_1} n_2^{j_2} n_3^{j_3} \cdots n_m^{j_m}$ par $\exp\,  
(-\sum_i  j_i^2/2n_i)$ soit maximum, il faut que $1 \bigl/ 
n_1^{j_1} n_2^{j_2} n_3^{j_3} \cdots n_m^{j_m}$ soit maximum en 
m\^eme temps, ou du moins qu'il reste constant. La stationnarit\'e de  
ce facteur est donc la condition sine qua non pour que le produit soit 
maximum. 
 
\bigskip 
 
{\bf II. 5.  Subdivisions.} 
\medskip 
La section pr\'ec\'edente traitait du nombre de r\'epartitions en 
sous-ensembles de taille fix\'ee \`a l'avance (par la donn\'ee des  
nombres d'oc\-cu\-pa\-tion). Mais il reste la question de conna\^\i tre 
le nombre  de choix possibles pour les nombres d'occupation. Ceux-ci 
v\'erifient  n\'ecessairement la relation $n_1 + n_2 + \cdots + n_m = n$. 
{\bf Combien  y a-t-il de possibilit\'es de choisir $m$ nombres 
v\'erifiant cette \'egalit\'e ?} On peut encore formuler la question en 
terme de d\'eveloppement multin\^omial: nous avons vu au paragraphe 
pr\'ec\'edent comment les {\it coefficients multin\^omiaux} 
interviennent dans le d\'eveloppement 
$$\sum_{n_1+n_2+ \cdots +n_m=n}  {n! \over n_1!\,  n_2! \cdots n_m!}\, 
x_1^{n_1}\, x_2^{n_2} \cdots  x_m^{n_m} $$  
La question que nous traitons maintenant est celle du {\it nombre de 
termes} qu'il y a dans cette somme. On remarquera que les nombres 
d'occupation peuvent \^etre nuls: quand on consid\`ere toutes les  
distributions possibles de boules dans des bo\^\i tes, on inclut le cas  
o\`u des bo\^\i tes sont enti\`erement vides; ou encore, dans le  
d\'eveloppement multin\^omial ci-dessus, beaucoup de mon\^omes ne 
contiennent pas toutes les $n$ variables, par exemple le mon\^ome 
$x_1^n$. Mais on peut aussi d\'enombrer les distributions qui ne laissent 
{\it aucune} bo\^\i te vide: nous \'etudierons \'egalement ce cas.  
\medskip  
Consid\'erons d'abord le cas o\`u les sous-ensembles vides sont admis.  
On r\'esoud ce probl\`eme en le ramenant \`a un autre de la fa\c con 
suivante. Le probl\`eme d'une distribution de $n$ objets dans $m$ 
sous-ensembles est \'evidemment identique \`a celui d'une distribution 
de $n$ boules dans $m$ cases. On peut repr\'esenter graphiquement une 
telle distribution (m\^eme  si on ne peut le faire qu'abstraitement et 
non sur du papier, comme ce serait le cas si par exemple on devait avoir 
$m=10^{1000000}$) comme dans la figure 1 (chapitre {\bf I}). On 
sch\'ematise alors la s\'eparation entre deux cases adjacentes par une 
barre verticale, ce qui donne un graphique du type suivant: \vskip0mm plus3mm minus2mm
$$\bigcirc\;\bigcirc\mid\bigcirc\mid\bigcirc\mid\bigcirc\;\bigcirc 
\mid\bigcirc\;\bigcirc\mid\bigcirc\bigcirc\bigcirc\mid\bigcirc  
\mid\;\mid\bigcirc\mid\bigcirc\bigcirc\bigcirc\mid\;\mid 
\;\mid\bigcirc \mid\bigcirc$$ 
\vskip4mm plus3mm minus2mm
\noindent on a ici 18 boules dans 14 cases, les nombres d'occupation \'etant 2, 1,  
1,  2, 2, 3, 1, 0, 1, 3, 0, 0, 1, 1. On peut voir que le probl\`eme, vu sous 
cet angle, est simplement le probl\`eme de la r\'epartition de $n$ ronds 
et $m-1$ barres sur $n + m -1$ places: \`a chaque sch\'ema de ce type 
correspond une et une seule distribution de $n$ boules dans $m$ cases et 
vice-versa. Si on change l'ordre des ronds (entre eux) ou celui des barres 
(entre elles) on ne change rien au sch\'ema de sorte que le nombre de 
sch\'emas diff\'erents est donn\'e par la section {\bf II. 3.}: il y en a 
$(n+m-1)! / n!\, (m-1)!$  
\medskip 
Pour le cas o\`u les sous-ensembles vides ne sont pas admis, le sch\'ema 
ci-dessus ne marche plus car il y a des cases vides (correspondant \`a  
des barres non s\'epar\'ees par un rond). Ne correspondent alors \`a notre 
probl\`eme que les sch\'emas sans barres contigu\"es. On peut 
interpr\'eter un tel sch\'ema comme une combinaison, non plus de $\mid$  
et de $\bigcirc$, mais de $\mid\bigcirc$ et de $\bigcirc$. Comme les 
sch\'emas ont un rond \`a leur extr\'emit\'e gauche, qui est obligatoire  
et  ne change donc rien au {\it nombre} de possibilit\'es, le choix ne 
portera que sur les $n-1$ autres ronds. De plus, comme chacun des $m-1$ 
symboles  $\mid \bigcirc$ repr\'esente une case {\it et} une boule, il 
faudra n'ajouter que $n-m$ symboles $\bigcirc$ pour  avoir $n-1$ boules 
en tout. De sorte  que le nombre total de signes $\mid\bigcirc$ ou 
$\bigcirc$ \`a choisir  sera $n-1$, et par cons\'equent le nombre de 
toutes les combinaisons sera $(n-1)! / (m-1)!\, (n-m)!$     
\medskip 
 
En fin de compte: 
$$\eqalignno{ \vcenter{\hsize=42mm \baselineskip=0pt 
\line{\hbox{nombre de subdivisions}\hfil} 
\vskip-3pt 
\line{\raise2pt\hbox{en $m$ sous-ensembles}\hfil} 
\vskip-3pt 
\line{\hbox{pouvant \^etre vides}\hfil} } 
 &= {(n+m-1)! \over n!\, (m-1)!}\; ; &(II.9.)\cr 
\noalign{\medskip} 
\vcenter{\hsize=42mm \baselineskip=0pt 
\line{\hbox{nombre de subdivisions}\hfil} 
\vskip-3pt 
\line{\raise2pt\hbox{en $m$ sous-ensembles}\hfil} 
\vskip-3pt 
\line{\hbox{non vides}\hfil} } 
&= {(n-1)! \over (n-m)!\, (m-1)!}\; .  &(II.10.)\cr }$$  
\medskip 
{\eightpoint 
Le nombre donn\'e en $II.9.$ est aussi le nombre de d\'eriv\'ees 
partielles d'ordre $n$ d'une fonction de $m$ variables. Par exemple on  
peut voir dans les tableaux suivants le nombre de d\'eriv\'ees partielles 
d'ordre 1 \`a 5 pour des fonctions de deux, trois, quatre, et cinq 
variables:  
\bigskip 
\line{\hfill 
\vtop{\hsize=27mm 
\centerline{fonctions de}  
\centerline{deux variables} 
\vskip-10pt 
$$\matrix{ 
\hbox{ordre}  &\hbox{nombre}  \cr 
\noalign{\medskip} 
1                   &2               \cr 
2                   &3               \cr 
3                   &4               \cr 
4                   &5               \cr 
5                   &6               \cr } 
$$} 
\hfill 
\vtop{\hsize=27mm 
\centerline{fonctions de}  
\centerline{trois variables} 
\vskip-10pt 
$$\matrix{ 
\hbox{ordre}  &\hbox{nombre}  \cr 
\noalign{\smallskip} 
1                   &3                \cr 
2                   &6                \cr 
3                   &10              \cr 
4                   &15              \cr 
5                   &21              \cr } 
$$} 
\hfill 
\vtop{\hsize=29mm 
\centerline{fonctions de}  
\centerline{quatre variables} 
\vskip-10pt 
$$\matrix{ 
\hbox{ordre}  &\hbox{nombre}  \cr 
\noalign{\smallskip} 
1                   &4                \cr 
2                   &10              \cr 
3                   &20              \cr 
4                   &35              \cr 
5                   &56              \cr } 
$$} 
\hfill 
\vtop{\hsize=27mm 
\centerline{fonctions de}  
\centerline{cinq variables} 
\vskip-10pt 
$$\matrix{ 
\hbox{ordre}  &\hbox{nombre}  \cr 
\noalign{\medskip} 
1                   &5                \cr 
2                   &15              \cr 
3                   &35              \cr 
4                   &70              \cr 
5                   &126            \cr } 
$$} 
\hfill } 
 
}  %%% end of \eightpoint 
  
\medskip 
La formule de d\'enombrement $(II.9.)$ est essentielle en physique 
statistique.  Elle est \`a la base de la {\it statistique de Bose-Einstein}. 
Au chapitre {\bf I} nous avions donn\'e l'exemple de trois particules de 
Bose \`a placer dans deux \'etats quantiques (voir figure 1,  colonne de 
droite).  Cet exemple \'etait oppos\'e \`a celui de trois boules non 
quantiques \`a ranger dans deux bo\^\i tes (figure 1,  colonne de 
gauche).  \`A cette occasion, nous avons insist\'e sur le fait que
pour les boules,  les \'epreuves \'equiprobables \'etaient les huit
distributions possibles (on consid\'erait que deux distributions qui
diff\`erent par une permutation des boules constituent deux \'epreuves
distinctes),  tandis que pour les particules de Bose,  les \'epreuves
\'equiprobables \'etaient les quatre modes d'occupation possibles
(on consid\'erait que deux distributions qui diff\`erent par une
permutation des particules ne constituent pas deux \'epreuves distinctes).  
\medskip 
La formule $(II.9.)$ g\'en\'eralise cela pour $n$ particules et $m$  
\'etats quantiques. Si on prend $n = 3$ et $m = 2$ on a bien 
$${(n+m-1)! \over n!\, (m-1)!} = {(3+2-1)! \over 3!\, (2-1)!} = {4! \over 
3!\, 1!} = 4$$ 
Ainsi ${(n+m-1)! \bigl/ n!\, (m-1)!}$ est le nombre de modes  
d'occupation de $m$ \'etats quantiques par $n$ particules de Bose. 
\medskip 
Le postulat de base de la statistique de Bose-Einstein est que les 
modes d'occupation sont \'equiprobables (ils constituent les \'epreuves 
parmi lesquelles ``le hasard choisit''). Mais il faut bien comprendre que 
ce hasard, tout comme dans les exemples de chaos d\'eterministe que 
nous avons analys\'es au chapitre {\bf I}, n'est pas une propri\'et\'e 
premi\`ere de la nature,  mais r\'esulte d'un brouillage chaotique, qui  
est l'agitation thermique.  L'agitation thermique est d'autant plus intense
que la temp\'erature du gaz de particules est plus \'el\'ev\'ee;  elle a
pour effet que les particules sont sans cesse d\'elog\'ees des \'etats
quantiques qu'elles occupent par toutes sortes d'interactions;  des milliards
de milliards de fois par seconde,  des milliards de milliards de photons ou 
d'\'electrons se prom\`enent \`a travers l'espace pour apporter ou enlever
de petites quantit\'es d'\'energie \`a ces particules;  ainsi ces derni\`eres
passent sans cesse (des milliards de milliards de fois par seconde) d'un
\'etat quantique \`a un autre:  chaque fois qu'elles absorbent un photon
elles passent \`a un \'etat d'\'energie sup\'erieure,  chaque fois qu'elles 
\'emettent un photon elles passent \`a un \'etat d'\'energie inf\'erieure. 
Ce brouillage est incomparablement plus puissant que par exemple celui 
qui r\'esulte des nombreuses r\'eflexions sur le bord de la roulette;  le 
mod\`ele simplifi\'e  de roulette discut\'e au chapitre {\bf I} \'etait un 
exemple choisi d\'elib\'er\'ement pour sa simplicit\'e et restait dans 
les limites du calculable. Il n'est pas question d'en faire autant pour un 
gaz form\'e de $10^{24}$ particules (environ le nombre d'Avogadro), 
c'est-\`a-dire appliquer la M\'ecanique quantique {\it exacte} \`a un   
tel syst\`eme.  Mais il est remarquable que ce brouillage produise 
l'\'equiprobabilit\'e des modes d'occupation et non par exemple celui des 
distributions de particules discernables.  Ceci est une loi fondamentale 
de la Physique et ne peut se d\'eduire de consid\'erations a priori sur 
les invariances spatio-temporelles. 
\medskip 
Attention!  Ce sont les modes d'occupations pour des \'etats {\it  
d'\'energie \'egale} qui sont \'equiprobables.  Si on consid\`ere deux 
\'etats  d'\'energie diff\'erente,  celui dont l'\'energie est plus basse a 
plus de chances d'\^etre occup\'e. On peut comprendre cela en imaginant 
qu'on remue un pierrier  sur une pente; bien que les pierres se mettent en 
mouvement de mani\`ere al\'eatoire et impr\'evisible, elles ont plus de 
chances de descendre que de monter: pour monter, il faut qu'une pierre  
ait par exemple heurt\'e  en descendant une autre pierre plus lente, et 
rebondi sur elle, de sorte que la quantit\'e de mouvement totale soit 
conserv\'ee; pour que la pierre monte, il faut que l'autre pierre descende 
plus vite. Il en va de m\^eme pour les particules: l'agitation thermique ne 
fait que r\'epartir l'\'energie; l'\'energie totale se conserve, de sorte que 
pour qu'une particule ``monte'' dans un \'etat de plus grande \'energie, il 
faut qu'en compensation une ou plusieurs autres ``descendent''. On con\c 
coit donc qu'en moyenne, il est plus difficile pour les particules de 
monter que de descendre, et on s'attend \`a ce  que les nombres 
d'occupation des \'etats de grande \'energie soient en moyenne plus 
petits que les nombres d'occupation des \'etats de faible \'energie.  
\medskip 
S'il n'y avait aucune agitation thermique, les particules seraient toutes 
dans l'\'etat dit fondamental, celui dont l'\'energie est la plus basse  
(c'est ce qui se produirait si la temp\'erature devenait exactement 
\'egale \`a $0$ degr\'e Kelvin). C'est donc uniquement l'agitation 
thermique, c'est-\`a-dire l'\'echange incessant d'\'energie entre les 
particules, qui permet \`a certaines de monter (au d\'etriment des 
autres). 
\medskip 
Afin de d\'enombrer les diff\'erents modes d'occupation entre tous les 
\'etats quantiques, quelle que soit leur \'energie, on d\'ecoupe les  
valeurs possibles de l'\'energie des \'etats en petits intervalles de 
largeur  $\delta$; ainsi $\varepsilon_0$ sera l'\'energie la plus basse;  
on consid\'erera les valeurs discr\`etes $\varepsilon_i = \varepsilon_0 
+ i \delta$. Si $\delta$ est petit, les \'etats dont l'\'energie est 
comprise entre $\varepsilon_i$ et $\varepsilon_{i+1}$ ont pratiquement 
la m\^eme \'energie; soit $m_i$ leur nombre. Il est bien clair que les  
nombres $m_i$ sont \`a peu pr\`es proportionnels \`a $\delta$: si par 
exemple $\delta$ est doubl\'e, il y aura deux fois moins d'intervalles 
d'\'energie, mais les nombres $m_i$ seront aussi deux fois plus grands.  
Puisque par construction les \'etats d'un m\^eme intervalle ont \`a peu  
pr\`es la m\^eme \'energie, les modes d'occupation des $m_i$ \'etats de  
l'intervalle seront tous \`a peu pr\`es \'equiprobables. On serait tent\'e de 
dire que plus $\delta$ est petit, plus cette \'equiprobabilit\'e est  exacte; 
mais cela n'a gu\`ere de sens car le principe de l'\'equiprobabilit\'e des 
modes d'occupations est par nature approximatif, et si $\delta$ est si 
petit que $m_i$ devient \'egal \`a 0, 1, ou 2, le principe devient m\^eme 
carr\'ement faux; en fait il est essentiel que $\delta$ ne soit ni trop 
grand, ni trop petit, et $m_i$ doit \^etre un grand nombre, plut\^ot de 
l'ordre du nombre d'Avogadro que de l'ordre de cent ou mille.   
\medskip  
Pour les $m_i$ \'etats quantiques d'\'energie comprise entre  
$\varepsilon_i$ et $\varepsilon_{i+1}$, il y aura donc pour $n_i$   
particules $(n_i + m_i -1)!  \bigl/ n_i!\, (m_i-1)!$ modes d'occupation: 
cela veut dire que si $n_i$  particules sont distribu\'ees entre les $m_i$ 
\'etats d'\'energie  comprise entre $\varepsilon_i$ et $\varepsilon_{i+1}$, 
alors elles peuvent se r\'epartir selon $(n_i + m_i -1)! \bigl/ n_i!\, 
(m_i-1)!$  modes d'occupation diff\'erents. A priori on ne peut pas 
conna\^\i tre les  valeurs des $n_i$, mais par contre celles des $m_i$ sont 
d\'etermin\'ees par les caract\'eristiques macroscopiques du syst\`eme. 
Ce sont g\'en\'eralement les \'equations de la M\'ecanique {\it classique} 
qui permettent de les d\'eterminer. En effet, les $m_i$ font partie des 
propri\'et\'es globales du syst\`eme, qui constituent en quelque sorte 
l'environnement de la population de particules. Par exemple, si on sait  
que les particules sont des photons de rayonnement \`a l'int\'erieur d'une 
cavit\'e, les \'etats qu'ils peuvent occuper sont caract\'eris\'es par les 
fr\'equences propres de la cavit\'e: pour chacune de ces fr\'equences 
propres, il y aura deux \'etats quantiques possibles (diff\'erant par la 
polarisation) pour le photon. Calculer l'\'evolution du syst\`eme complexe 
qu'est la population de photons par la M\'ecanique quantique exacte est 
d'une complexit\'e inou\"\i e, mais calculer les fr\'equences propres de  
la cavit\'e est assez facile car la forme g\'eom\'etrique de la cavit\'e  
ne subit pas le brouillage thermique; on effectuera donc des calculs 
classiques d\'etaill\'es pour la cavit\'e, mais on appliquera les lois du 
hasard \`a  la population de photons. Cette fa\c con d'aborder les 
syst\`emes form\'es d'un tr\`es grand nombre de particules ou de 
mol\'ecules constitue la  {\it Physique statistique}. Afin d'en donner une 
id\'ee plus pr\'ecise, le mieux est de traiter en d\'etail un exemple  
concret, qui est aussi une application directe de ce que nous venons de  
voir concernant la statistique de Bose-Einstein: la loi de Planck. 
 
\bigskip
 
{\bf II. 6.  Une introduction \`a la physique statistique: la loi de Planck  
et le rayonnement du corps noir.}
\medskip
Le probl\`eme du {\it corps noir} est un des grands probl\`emes  
historiques de la Physique, car il est \`a l'origine de la M\'ecanique 
quantique (Max Planck, {\oldstyle1900}). Concr\`etement, un corps noir  
est par exemple une cavit\'e \`a l'int\'erieur d'un corps opaque, qui est 
chaud \`a l'int\'erieur mais thermiquement isol\'e \`a l'ext\'erieur: dans  
la cavit\'e il y a un rayonnement d\^u \`a l'\'emission thermique par le 
corps; plus le corps est chaud, plus nombreux sont les photons de haute 
fr\'equence (le corps peut avoir \'et\'e chauff\'e ``au rouge'', ou ``\`a 
blanc''). Planck a d'abord trouv\'e la loi qui porte son nom en interpolant 
entre la loi de Rayleigh-Jeans (qui \'etait correcte pour les faibles 
fr\'equences) et la loi de Wien (correcte pour les hautes fr\'equences), 
et non par le raisonnement probabiliste que nous pr\'esentons ici, 
puisque les photons et la statistique de Bose-Einstein n'\'etaient 
\'evidemment pas encore connus. Voir \`a ce sujet {\bf XII.5.}  
Planck  a cherch\'e ensuite une explication statistique, 
mais c'est Einstein qui, en {\oldstyle 1905}, a eu l'id\'ee de 
consid\'erer des {\it quanta de lumi\`ere} pour expliquer la  
loi de Planck ({\sl Annalen der Physik}, vol {\bf 17}, 
1905, pages 132 -- 148). Toutefois, dans cette publication de 1905,  
rien n'est dit sur l'aspect probabiliste. C'est Satyandranath Bose qui a 
propos\'e un fondement statistique ({\sl Zeitschrift f\"ur Physik}, 
vol {\bf 26}, 1924, pages 178 -- 181): 
\smallskip 
{\cit Supposons que le rayonnement soit enferm\'e dans un 
volume $V$ et que son \'energie totale soit $E$. Il existe diff\'erentes 
sortes de quanta en nombre $N_s$ et d'\'energie $h \nu_s$, o\`u $s$ 
va de $s=0$ \`a $s=\infty$ (\dots ) Si nous pouvions exprimer la  
probabilit\'e de chacune des distributions caract\'eris\'ees par un  
nombre $N_s$ arbitraire, alors la solution sera obtenue en prenant le 
crit\`ere de rendre cette probabilit\'e maximale. \par } 
\medskip 
Cette probabilit\'e est alors cherch\'ee par le raisonnement suivant: 
\smallskip 
{\cit Soit $N_s$ le nombre de quanta appartenant au domaine de 
fr\'equences $d\nu_s$. De combien de mani\`eres peut-on les distribuer 
entre les cellules appartenant \`a $d\nu_s$? Soit $p_{0,s}$ le nombre 
de cellules vides, $p_{1,s}$ le nombre de celles contenant un quantum, 
$p_{2,s}$ le nombre de celles en contenant deux, etc. Le nombre de 
distributions possibles est alors 
$${A_s \over p_{0,s}!\, p_{1,s}!\, \ldots }$$ 
o\`u $A_s = (8\pi\nu^2/c^3)\, d\nu_s$, et o\`u $N_s = 0 \cdot p_{0,s} +  
1 \cdot p_{1,s} + 2 \cdot p_{2,s} + \cdots$ est le nombre de quanta  
appartenant \`a la cellule $d\nu_s$. \par } 
\medskip 
On voit donc que le raisonnement suivi par Bose est celui d'une 
distribution de boules entre des cellules, mais dans son approche les 
nombres $p_{j,s}$ ne sont pas les nombres d'occupation, qui sont les 
$N_s$. 
\medskip
En {\oldstyle 1900}, quoique ne poss\'edant pas le concept quantique de
``mode d'occupation'', Planck a postul\'e que pour chaque fr\'equence le
rayonnement se  distribuait par unit\'es discr\`etes entre des
``r\'esonateurs'' hypoth\'etiques, et a suivi, pour calculer l'entropie 
de ces r\'esonateurs, un raisonnement  finalement assez proche de celui 
que nous pr\'esentons ici. On pourra consulter l'article original  
{\it  \"Uber das Gesetz der Energieverteilung im Normalspectrum.}  
Annalen  der Physik, vol {\bf 4}, {\oldstyle 1901},  pages 553 -- 563).
Pour une traduction fran\c caise, voir la bibliographie. 
\medskip 
Si au lieu de consid\'erer uniquement des \'etats de m\^eme \'energie,  
ou d'\'energie voisine, on consid\`ere l'ensemble de {\it tous} les 
\'etats quelle  que soit leur \'energie, le nombre total de modes 
d'occupations sera  bien  s\^ur le produit 
$$N = \prod_{i} {(n_i + m_i -1)! \over n_i!\, (m_i-1)!} \eqno (II.11)$$ 
Comme les nombres $m_i$ et $n_i$ sont tr\`es grands, on ne peut  
gu\`ere  tirer de conclusions de ces grosses factorielles, et c'est 
pourquoi nous allons, comme pour les autres formules de 
d\'enombrement, trouver une  expression approch\'ee. Le proc\'ed\'e 
pour cela est toujours le m\^eme: d'abord trouver le maximum, et autour 
du maximum les factorielles peuvent \^etre approch\'ees par une 
fonction de la forme $\exp(-x^2)$. 
\medskip 
Les $m_i$ \'etant d\'etermin\'es, consid\'erons une perturbation $n_i +  
j_i$ des nombres  $n_i$. Le nombre $N$ sera alors chang\'e en  
$$N' = \prod_{i} {(n_i+j_i + m_i -1)! \over (n_i+j_i)!\, (m_i-1)!}$$ 
Pour comparer $N'$ \`a $N$ on examinera $N'/N$.  
Or pour $j_i > 0$ on peut \'ecrire  
$$\eqalignno{ 
&{(n_i+j_i + m_i -1)! \over (n_i+j_i)!} \biggl/ 
{(n_i+ m_i -1)! \over n_i!} \quad = \cr 
\noalign{\medskip} 
=\quad &{(n_i + m_i)(n_i + m_i+1) \cdots  
(n_i + m_i+j_i-1) \over (n_i+1)(n_i+2)\cdots (n_i+j_i)} \cr 
\noalign{\medskip} 
=\quad &\Bigl[ {n_i + m_i \over n_i}\Bigr]^{j_i} \cdot {\Bigl(1 +  
{1\strutc \over n_i + m_i}\Bigr) \Bigl(1 + {2\strutc\over n_i + 
m_i}\Bigr) \cdots \Bigl(1 + {j_i-1\strutc\over n_i + m_i}\Bigr)  
\strup{8} \over \sdown{13}\Bigl(1 + {1\strutc 
\over n_i}\Bigr) \Bigl(1 + {2\strutc\over n_i}\Bigr) \cdots 
\Bigl(1 + {j_i\strutc\over n_i}\Bigr)} \cr 
\noalign{\medskip} 
\simeq\quad &\Bigl[ {n_i + m_i \over n_i}\Bigr]^{j_i} \exp\biggl[ {\;  
j_i^2 \over 2(n_i +m_i)} - {\; j_i^2 \over 2n_i}\biggr] \cr 
\noalign{\medskip} 
=\quad &\Bigl[ {n_i + m_i \over n_i}\Bigr]^{j_i} \exp\biggl[ -{ m_i \; 
j_i^2 \over 2 n_i (n_i + m_i)}\biggr] 
\cr }$$ 
Pour $j_i < 0$, quoique le calcul soit l\'eg\`erement diff\'erent, on  
obtient  la m\^eme expression analytique pour l'approximation. En 
regroupant tout dans le produit: 
$$N' = N\; \prod_i\Bigl[ 1 + {m_i \over n_i}\Bigr]^{j_i}\; \exp\Bigl[ 
-\sum_i { m_i\; j_i^2\strutc \over 2 n_i (n_i + m_i)}\Bigr]$$ 
Comme nous avons vu au paragraphe pr\'ec\'edent pour les coefficients 
multin\^omiaux, on peut alors dire que la condition pour que $N$ soit 
maximum est que le facteur  
$$\prod_i\Bigl[ 1 + {m_i \over n_i}\Bigr]^{j_i}$$ 
soit stationnaire lorsque les $j_i$ s'\'ecartent de $0$. Il est clair que  
ce facteur {\it ne peut pas} rester stationnaire si les $j_i$ sont tous 
ind\'ependants les uns des autres (il faudrait pour cela que les $m_i$ 
soient tous nuls).  Dans le cas des coefficients multin\^omiaux, pour 
obtenir $(II.7.)$, nous avions utilis\'e le fait que  la somme des $j_i$ 
\'etait nulle. Cette condition exprime que le nombre total de particules 
ne varie pas. Mais une autre condition est possible: que l'\'energie  
totale ne varie pas; ou m\^eme les deux \`a la fois. Le maximum ne sera 
pas le m\^eme selon la contrainte qui lie les $j_i$ entre eux, et par 
cons\'equent on n'obtient pas la m\^eme statistique. Par exemple s'il 
s'agit d'un gaz de photons (rayonnement  de corps noir) 
l'\'energie totale du rayonnement se conserve (c'est justement pour cela 
qu'il est dit ``noir''),  mais pas le nombre de photons, car pour maintenir 
l'\'equilibre thermique avec le corps noir  il faut constamment que 
celui-ci absorbe ou \'emette des photons.  Si par contre il ne s'agit  
pas de rayonnement, mais d'helium liquide, les particules de Bose 
consid\'er\'ees sont alors des noyaux d'$H\! e\, 4$, dont le nombre  
total doit se conserver. Ici c'est le cas du rayonnement qui nous 
int\'eresse, c'est-\`a-dire le gaz de photons,  o\`u c'est l'\'energie totale 
et non le nombre de particules qui se conserve. Alors la contrainte sur 
les $j_i$  est $\sum_i  \varepsilon_i j_i = 0$; on peut donc exprimer l'un 
quelconque des $j_i$ en fonction des autres: par exemple   
$$j_a = -{1 \over \varepsilon_a} \sum_{i \neq a} \varepsilon_i j_i $$  
En rempla\c cant $j_a$ par cette expression dans le facteur qui doit  
rester stationnaire, on obtient  
$$\eqalignno{ 
\prod_i\Bigl[ 1 + {m_i \over\ldown{n_i}}\Bigr]^{j_i}\; &=  
\quad\Bigl[ 1 + {m_a \over\ldown{n_a}}\Bigr]^{j_a} \cdot  
\prod_{i \neq a}\Bigl[ 1 + {m_i \over\ldown{n_i}}\Bigr]^{j_i} \cr 
\noalign{\bigskip} 
&= \quad\Bigl[ 1 + {m_a \over\ldown{n_a}}\Bigr]^{-{1\over 
\varepsilon_a} \sum_{i\neq a}j_i\varepsilon_i} \cdot\prod_{i \neq 
a}\Bigl[ 1 + {m_i \over\ldown{n_i}}\Bigr]^{j_i}\cr 
\noalign{\medskip} 
&= \quad\prod_{i \neq a}\left[ {1 + {m_i \vrule depth3pt width0pt 
\over n_i} \vrule depth10pt width0pt 
\over \Bigl(1 + {m_a \vrule depth3pt width0pt\over n_a} 
\Bigr)^{\varepsilon_i \vrule height4pt depth3pt width0pt\over 
\varepsilon_a} }\right]^{j_i}\cr } $$ 
Cette fois les $j_i$ (pour $i \neq a$) sont ind\'ependants, donc la 
condition pour que ce produit soit stationnaire est que chacun des 
facteurs soit \'egal \`a 1, soit 
$$1 + {m_i \over\ldown{n_i}} = \Bigl(1 + {m_a \over\ldown{ 
n_a}}\Bigr)^{\varepsilon_i\strutc \over \varepsilon_a}$$ 
Cette \'egalit\'e devant \^etre v\'erifi\'ee pour $i$ et $a$ arbitraires,  
on peut donc dire que  
$$K = \Bigl(1 + {m_i \over\ldown{n_i}}\Bigr)^{1 \over \varepsilon_i} 
\eqno (II.12.)$$ 
est ind\'ependant de $i$; c'est-\`a-dire qu'il existe une constante $K$ 
telle que 
$$1 + {m_i \over\ldown{n_i}} = K^{\varepsilon_i}$$ 
et par cons\'equent 
$$n_i = {m_i \over K^{\varepsilon_i} - 1} \eqno (II.13.)$$ 
Pour les photons, les \'etats quantiques qu'ils peuvent occuper sont 
carac\-t\'e\-ris\'es par la fr\'equence $\nu$; l'\'energie d'un photon de 
fr\'equence $\nu_i$ est $\varepsilon_i = h\nu_i$, $h$ \'etant la constante  
de Planck. Le nombre $m_i$ est alors le nombre de fr\'equences propres  
de rayonnement  (dans la cavit\'e) comprises entre $\nu_i$ et $\nu_{i+1}  
= \nu_i + \delta /h$ (en fait le double car pour chaque fr\'equence propre  
il y a deux \'etats de polarisation). La condition de {\it maximum} pour le 
nombre total $N$ de modes d'occupation est donc  que les $n_i$ soient  
tous \'egaux \`a  $m_i \bigl/ (K^{\varepsilon_i} - 1)$.   
\medskip 
Autour de leurs valeurs optimales correspondant au maximum, les  
$n_i$ peuvent varier: cela aura pour effet de diminuer la valeur de $N$; 
la diminution de $N$ par rapport \`a sa valeur maximum 
$N_{\hbox{\sevenrm max}}$ est alors donn\'ee par le facteur gaussien 
$$N = N_{\hbox{\sevenrm max}}\;\exp\Bigl[ -\sum_i { m_i\; j_i^2 
\strutc \over 2 n_i (n_i + m_i)}\Bigr]$$ 
Ce facteur gaussien devient infinit\'esimal lorsque l'exposant d\'epasse  
environ $10$ (en valeur absolue); cela se produit lorsque les $j_i$ sont 
en moyenne sup\'erieurs \`a plusieurs fois la quantit\'e $\sqrt{n_i\, 
(1+(n_i/m_i)}$.  En principe les nombres $m_i$ et $n_i$ sont tous les 
deux grands, mais du m\^eme ordre (de l'ordre du nombre d'Avogadro, 
$\sim 10^{24}$); on peut donc consid\'erer que $N$ reste proche de sa 
valeur maximum tant que les fluctuations $j_i$ des $n_i$ sont de 
l'ordre de $\sqrt{n_i} \sim 10^{12}$; au del\`a, $N$ devient tr\`es vite 
infiniment plus petit que sa valeur maximum. Les modes d'occupation 
pour lesquels les nombres d'occupation $n_i$ diff\`erent de la valeur 
optimale $m_i \bigl/  (K^{\varepsilon_i} - 1)$ de plus que quatre ou  
cinq fois leur racine  carr\'ee sont donc extr\^emement peu nombreux. 
\medskip 
Or, le hasard pur ``choisit'' parmi les modes d'occupation, qui sont tous 
\'equi\-pro\-bables. Nous venons de voir que pour la quasi totalit\'e 
de ces modes d'occupation, les nombres $n_i$ sont, \`a $j_i 
\sim \sqrt{n_i}$ pr\`es, \'egaux \`a $m_i \bigl/ (K^{\varepsilon_i} - 
1)$, tandis que les $n_i$ ne s'\'ecartent de cette valeur optimale que 
pour un nombre incomparablement plus petit de modes d'occupation.  
Par cons\'equent le choix du hasard pur sera presque s\^urement un  
mode d'occupation pour lequel les $n_i$ seront \`a peu pr\`es \'egaux  
\`a $m_i \bigl/ (K^{\varepsilon_i} - 1)$. La cause de cela n'est pas que  
le hasard ``pr\'ef\`ere'' les modes d'occupations pour lesquels 
$n_i \simeq m_i \bigl/ (K^{\varepsilon_i} - 1)$, mais tout simplement  
que ces modes d'occupation sont de tr\`es loin les plus nombreux. 
\medskip   
M\^eme s'il arrivait que {\it par hasard} un mode d'occupation rare se 
produise, il ne durerait qu'un temps infinit\'esimal ($10^{-12}$ seconde 
ou m\^eme bien moins), car l'agitation thermique a pour effet de 
d\'eloger sans cesse les particules de l'\'etat qu'elles occupent, de 
sorte que les modes d'occupation changent sans cesse; il est donc 
logique que ceux qui constituent l'immense majorit\'e soient presque 
toujours en place.  
\medskip 
Le myst\'erieux param\`etre $K$ qui appara\^\i t dans $II.12$ et $II.13$ 
doit \^etre li\'e \`a la temp\'erature absolue $T$ du corps noir. En effet,  
si la temp\'erature est grande, les  hautes fr\'equences doivent \^etre 
occup\'ees par davantage de  photons, et si elle est nulle, seule la 
fr\'equence 0 doit \^etre  occup\'ee. Autrement dit, pour $\varepsilon_i > 
0$ le d\'enominateur $K^{\varepsilon_i} - 1$ doit devenir infini lorsque  
la temp\'erature est nulle afin que $n_i$ soit nul, donc $K$ doit \^etre 
infini pour $T=0$. Inversement, le d\'enominateur doit \^etre de plus en 
plus petit pour les grandes valeurs de $T$, afin que le $n_i$ 
correspondant soit grand,  donc $K$ doit tendre vers 1 quand la 
temp\'erature $T$ tend  vers l'infini.  Dans l'article cit\'e, Planck a  
obtenu une expression pr\'ecise de ce param\`etre $K$ en comparant  
$II.13.$ \`a la loi de Wien, qui \'etait connue et v\'erifi\'ee par  
l'exp\'erience pour le rayonnement ultra-violet. Celle-ci comportait un  
facteur $\exp (-\alpha\nu_i / T)$, dans lequel $\nu$ est la fr\'equence  
du rayonnement (pour nous, les photons), $T$ la temp\'erature absolue en 
Kelvin, et $\alpha$ une constante empirique.  
\medskip 
Cette loi de Wien n'\'etait valable que pour les grandes fr\'equences.  
Compar\'ee \`a celle de Planck, elle 
correspond au cas o\`u on peut n\'egliger le terme $-1$ dans le 
d\'enominateur de $II.13$. Planck soup\c connait que la loi de Wien devait 
\^etre une version asymptotique de la v\'eritable loi qu'il \'etait en train 
de chercher; les raisonnements probabilistes de la Physique statistique 
laissent les param\`etres tels  que les nombres $m_i$ ind\'etermin\'es, 
car on ne  peut pas mesurer directement leur valeur. Il fallait donc 
rattacher la loi obtenue $II.13$  \`a des grandeurs mesurables 
exp\'erimentalement, or justement la loi de Wien, quoique obtenue au 
d\'epart par des arguments th\'eoriques, \'etait une loi empirique: \'etant 
de forme exponentielle, il suffisait de reporter sur une \'echelle 
logarithmique les valeurs  mesur\'ees de l'intensit\'e du rayonnement dans 
chaque intervalle de fr\'equence, qui se r\'epartissaient alors le long d'une 
droite.  Partant donc de cette id\'ee que le facteur $1 / (K^\varepsilon - 
1)$ dans $II.13$ doit \^etre asymptotiquement \'equivalent au facteur 
$\exp (-\alpha\nu / T)$ de la loi de Wien, on obtient  
$$K^{-\varepsilon_i} = e^{-{\alpha\nu 
\over T}}$$  
Pour que l'identification ait un sens physique, il faut que 
l'argument  dans l'exponentielle soit sans dimension; or  $\varepsilon_i$ 
ayant la dimension d'une \'energie, Planck a choisi de prendre $K = \exp (1 
/ kT)$  et $\varepsilon_i = k\alpha\nu_i$ o\`u $k$  est la constante de 
Boltzmann ($\simeq 1.38 \cdot 10^{-16}\;  \hbox{\eightrm 
erg}/\hbox{\eightrm deg}$). C'est la constante $h = k\alpha \simeq 6.55 
\cdot 10^{-27}\; \hbox{\eightrm erg} \cdot \hbox{\eightrm sec}$ qui a  
pris le nom  de  {\it constante  de Planck}, et la relation $\varepsilon_i = 
h\nu_i$, d\'eduite par identification \`a la loi empirique de Wien, est \`a 
l'origine de la M\'ecanique quantique.  
\medskip 
En conclusion, pour un rayonnement {\it de corps noir}, c'est-\`a-dire  
un rayonnement emprisonn\'e dans une cavit\'e et en \'equilibre  
thermique avec la mati\`ere formant les parois de la cavit\'e, le  
nombre $n_i$  de photons dont la fr\'equence est comprise entre 
$\nu_i$ et $\nu_{i+1}$ est \'egal \`a  
$${m_i \over \vrule height15pt width0pt \e^{{h\nu_i\vrule depth2.2pt width0pt 
\over \vrule height4.5pt width0pt kT}} - 1}$$ 
o\`u $T$ est la temp\'erature qui r\`egne dans la cavit\'e et $h$ la 
constante de Planck. L'\'etude purement \'electromagn\'etique des 
fr\'equences propres de la cavit\'e montre que le nombre d'\'etats $m_i$ 
correspondant \`a un intervalle  de fr\'equences $[\nu_i,  \nu_{i+1}[$ de 
largeur $\eta = \delta  / h$ est proportionnel \`a  $\nu_i^2 \eta$ \ftn 
1{voir {\it Quantique} par Levy-Leblond et Balibar, page 453.}; de plus,  
un photon  de fr\'equence $\nu$ porte une \'energie $h\nu$; on aboutit 
ainsi \`a la {\it loi de Planck}: l'intensit\'e (l'\'energie) de la part de 
rayonnement dont la fr\'equence est comprise entre $\nu_i$ et 
$\nu_{i+1}$ est proportionnelle \`a    
$${h\nu_i^3\eta \over \vrule height15pt width0pt \e^{{h\nu_i \vrule 
depth2.2pt width0pt \over\vrule height4.5pt width0pt kT}} - 1}  
\eqno (II.14.)$$
Il faut bien comprendre que cette loi est due {\it uniquement} 
\`a l'action du hasard qui choisit, des milliards de milliards de fois par  
seconde,  des modes d'occupation parmi l'ensemble de tous les modes  
d'occupation possibles, sans en favoriser aucun. La loi de Planck est 
v\'erifi\'ee, non parce que les photons lui ``ob\'eissent'', mais parce 
que les photons n'ob\'eissent \`a rien.




\bye 
