\input twelvea4.tex
\input epsf.tex

\auteurcourant={\tensl J. Harthong: probabilit\'es et statistique}
\titrecourant={\tensl Tests statistiques}

\pageno=289

\def\w{\hskip1.2pt$\bullet$\hskip1.2pt}
\def\trv{\vrule height26.2pt depth0pt width0.4pt}

\newdimen\blocksize  \blocksize=\vsize  \advance\blocksize by -12pt

\def\six{\lower9.6pt\hbox{\trv\vbox to 26.2pt{\offinterlineskip
\hrule
\vskip2.4pt
\hbox to 25.2pt {\w\hfill\w\hfill\w}
\vfill
\hbox to 25.2pt {\w\hfill\w\hfill\w}
\vskip2pt
\hrule } \hskip-4pt\trv } }

\def\cinq{\lower9.6pt\hbox{\trv\vbox to 26.2pt{\offinterlineskip
\hrule
\vskip2pt
\hbox to 25.2pt {\w\hfill\w}
\vfill
\hbox to 25.2pt {\hfill\w\hfill}
\vfill
\hbox to 25.2pt {\w\hfill\w}
\vskip2.4pt
\hrule } \hskip-4pt\trv } }

\def\quatre{\lower9.6pt\hbox{\trv\vbox to 26.2pt{\offinterlineskip
\hrule
\vskip2.4pt
\hbox to 25.2pt {\w\hfill\w}
\vfill
\hbox to 25.2pt {\w\hfill\w}
\vskip2.4pt
\hrule } \hskip-4pt\trv } }

\def\trois{\lower9.6pt\hbox{\trv\vbox to 26.2pt{\offinterlineskip
\hrule
\vskip2.4pt
\hbox to 25.2pt {\hfill\w}
\vfill
\hbox to 25.2pt {\hfill\w\hfill}
\vfill
\hbox to 25.2pt {\w\hfill}
\vskip2.4pt
\hrule } \hskip-4pt\trv } }

\def\deux{\lower9.6pt\hbox{\trv\vbox to 26.2pt{\offinterlineskip
\hrule
\vskip4.8pt
\hbox to 25.2pt {\hfill\w\hskip2.4pt}
\vfill
\hbox to 25.2pt {\hskip2.4pt\w\hfill}
\vskip4.8pt
\hrule } \hskip-4pt\trv } }

\def\as{\lower9.6pt\hbox{\trv\vbox to 26.2pt{\offinterlineskip
\hrule
\vfill
\hbox to 25.2pt {\hfill\w\hfill}
\vfill
\hrule } \hskip-4pt\trv } }

\null\vskip10mm plus3mm minus3mm

\centerline{\tit XI. TESTS STATISTIQUES}

\vskip10mm plus3mm minus3mm

{\bf XI. 1. Les densit\'es $\chi_r^2(t)$}  
\medskip
Au chapitre pr\'ec\'edent, nous avons vu comment utiliser la loi 
gaussienne pour distinguer une v\'eritable anomalie d'une simple
fluctuation al\'eatoire. L'exemple que nous avons retenu pour discuter
\'etait celui d'une pi\`ece de monnaie qu'on lance un grand nombre $n$ de
fois. La loi gaussienne des fluctuations nous dit que la probabilit\'e pour
que le nombre de pile ou de face s'\'ecarte de la valeur m\'ediane $n/2$ 
de plus que, disons, cinq fois l'\'ecart-type $\sigma = \sqrt{n/2}$ 
est si faible qu'il est pratiquement impossible qu'un tel \'ecart se
produise, ce qui permet, si cela se produit, d'affirmer avec un tr\`es 
haut degr\'e de certitude que la pi\`ece \'etait donc mal \'equilibr\'ee. 
\medskip
Les tests statistiques sont des recettes bas\'ees sur ce principe. Tous ne
font que transposer le raisonnement pr\'ec\'edent, en introduisant un
interm\'ediaire math\'ematique qui ne change rien quand au fond, mais
facilite le calcul. Dans le cas de la pi\`ece de monnaie le recours direct 
\`a la loi gaussienne \'etait facile: il suffisait de compter le nombre de 
pile, puis de comparer son \'ecart par rapport \`a $n/2$ \`a l'\'ecart-type 
$\sigma = \sqrt{n/2}$, et on pouvait alors sans calcul suppl\'ementaire 
se servir d'une table de la loi gaussienne.
\medskip
Les choses deviennent plus complexes lorsqu'au lieu d'une pi\`ece de
monnaie, on teste un d\'e. Si on lance le d\'e $6000$ fois, chaque face du 
d\'e doit appara{\^\i}tre environ $1000$ fois, et c'est alors l'\'ecart par
rapport \`a $1000$ qui mesure la fluctuation. Mais il y a {\it six}
fluctuations, une par face. En fait, il n'y en a que cinq ind\'ependantes, 
car, de m\^eme que pour la pi\`ece de monnaie le nombre de pile
d\'eterminait automatiquement le nombre de face, pour le d\'e la
connaissance de cinq fluctuations d\'etermine la sixi\`eme. Il est bien
entendu que chacune de ces six fluctuations ob\'eit \`a une loi
approximativement gaussienne, et qu'on peut donc analyser chacune
s\'epar\'ement, avec sa propre loi gaussienne. Mais il est assez rare en
Statistique qu'une information aussi d\'etaill\'ee soit utile: il suffit de
savoir si le d\'e est globalement bon ou mauvais et il est inutile de
s'encombrer de ce qui arrive \`a chacune des faces.  Le d\'e sera
consid\'er\'e comme mal \'equilibr\'e m\^eme si un seul des six \'ecarts
est trop grand (par exemple sup\'erieur \`a deux et demi ou trois fois
l'\'ecart-type); autrement dit: un d\'e qui est parfaitement \'equilibr\'e pour
les faces $1,2,5,6$, mais mal \'eqilibr\'e pour les faces $3$ et $4$, n'est 
pas meilleur qu'un d\'e qui est mal \'equilibr\'e sur toutes les faces. Dans un 
tel cas tester l'\'ecart pour chacune des six faces s\'epar\'ement est une
perte de temps, et on souhaiterait plut\^ot pouvoir mesurer un \'ecart
global, par exemple la somme des carr\'es des fluctuations. Certes, la
somme des valeurs absolues des fluctuations, ou la somme des
quatri\`emes puissances pourraient convenir \'egalement, mais, comme
cela a d\'ej\`a \'et\'e discut\'e \`a propos de la variance (voir {\bf VI.2.}),
la somme des carr\'es est consid\'er\'ee pour diverses raisons comme la
plus pratique (bien entendu la somme {\it alg\'ebrique} des fluctuations
est par nature toujours nulle et ne mesure donc rien). 
\medskip
Pour que la discussion soit plus concr\`ete, imaginons que le d\'e,
lanc\'e $6000$ fois, donne les r\'esultats suivants:
$$\eqalign{
&\as \hbox to 20mm {\hfill 1064 fois} \hskip25mm
\quatre \hbox to 20mm {\hfill 1009 fois} \cr
&\deux \hbox to 20mm {\hfill 987 fois} \hskip25mm 
\cinq \hbox to 20mm {\hfill 977 fois} \cr
&\trois \hbox to 20mm {\hfill 961 fois} \hskip25mm
\six \hbox to 20mm {\hfill 1002 fois} \cr } \eqno (XI.1.)$$
Si on veut faire comme nous avons fait pour la pi\`ece de monnaie,
il faudra comparer les six r\'esultats aux fluctuations gaussiennes autour 
de la moyenne 1000; dans ce cas il faut d\'eterminer l'\'ecart-type: en
l'occurrence il vaut $\sqrt{1000} \simeq 32$. Ici l'\'ecart-type est le 
m\^eme pour chacune des six faces, ce qui est \'evident puisqu'il y a
sym\'etrie (les six faces sont interchangeables). On peut aussi 
calculer la somme des carr\'es qui est alors 
$$S = 64^2 + 13^2 + 39^2 + 9^2 + 23^2 + 2^2 = 6400 \eqno (XI.2.)$$
Mais plus question de comparer cette valeur \`a l'\'ecart-type 32. Il y a 
\`a cela trois raisons: la premi\`ere est qu'on fait la somme de six
contributions, il faudrait donc comparer \`a six fois la valeur moyenne
attendue pour l'une; la seconde est qu'on prend les carr\'es (qui sont
beaucoup plus grands), il faudrait donc comparer \`a six fois le carr\'e de
l'\'ecart-type. Mais ce raisonnement serait faux: ce n'est pas \`a six fois
le carr\'e de l'\'ecart-type qu'il faut comparer ce nombre, car (c'est la
troisi\`eme raison) cette somme de carr\'es ne suit pas une loi gaussienne.
Il se trouve en effet que, si les fluctuations ob\'eissent \`a une loi
gaussienne, ce n'est plus le cas de leurs carr\'es.
\medskip

\midinsert
\epsfxsize = \hsize
\line{\epsfbox{../images/fig41.eps} }
\vskip2mm
\centerline{\eightpoint figure 41}
\vskip8mm
\epsfxsize = \hsize
\line{\epsfbox{../images/fig42.eps} }
\vskip1mm
\centerline{\eightpoint figure 42}
\vskip3mm
\endinsert

Si on n'a pas du tout r\'efl\'echi \`a la question, on peut \^etre surpris
d'apprendre que le carr\'e d'une variable al\'eatoire n'ob\'eit pas \`a la 
m\^eme loi que la variable elle-m\^eme, mais cela est logique et a d\'ej\`a
\'et\'e \'evoqu\'e: par exemple dans le probl\`eme des distributions au 
hasard de cordes sur un cercle (chapitre {\bf I}) la diff\'erence entre les
trois lois se traduisait math\'ematiquement par le fait que les variables
al\'eatoires correspondantes sont li\'ees par des relations non lin\'eaires.
Sur la figure $41$ on peut voir deux graphiques de loi. Celui du haut
repr\'esente une loi $\{ x_j\; ; \; (p_j) \}$ telle que les $p_j$ soient tous
\'egaux \`a $1 \over 17$ et $x_j = {1 \over 10}j$: les valeurs sont donc 
\'equidistantes et vont de $0$ \`a $1.6$; elles sont toutes \'equiprobables
et, comme il y en a $17$ en tout,  la probabilit\'e de chacune est $1 \over
17$. Le graphique du bas repr\'esente la loi du carr\'e: les $x_j$ sont 
remplac\'ees par les $x_j^2$, mais les $p_j$ ne changent pas: le passage 
au carr\'e rend plus petit ce qui est inf\'erieur \`a $1$, mais rend plus 
grand ce qui est sup\'erieur \`a $1$. Ce qui a pour effet que les valeurs
inf\'erieures \`a $1$ se trouvent davantage comprim\'ees vers $0$, 
tandis que les valeurs sup\'erieures \`a $1$ sont dilat\'ees. Par
cons\'equent la {\it densit\'e},  qui \'etait uniforme, se trouve 
augment\'ee entre $0$ et $1$ (d'autant plus qu'on est plus pr\`es de $0$) 
et diminu\'ee au-del\`a de $1$. Si on effectue un lissage de cette loi
discr\`ete (celle du carr\'e) pour faire appara{\^\i}tre sa densit\'e, on
obtient le graphique de la figure 42; le lissage de la loi initiale $\{ x_j\; ;
\; (p_j) \}$ aurait \'evidemment donn\'e une densit\'e constante entre $0$
et $1.6$. Ainsi, comme nous en avions d\'ej\`a eu des exemples ailleurs, 
ce n'est pas une modification des probabilit\'es qui change la loi, mais 
une modification dans la r\'epartition des valeurs, cons\'ecutive \`a
une transformation non lin\'eaire. 
\medskip

\midinsert
\epsfxsize = \hsize
\line{\epsfbox{../images/fig43.eps} }
\vskip6mm
\centerline{\eightpoint figure 43}
\vskip6mm
\endinsert

On s'attend donc \`a un effet analogue pour la loi bin\^omiale. Sur la
figure $43$ on peut voir en $(a)$ la loi bin\^omiale et en $(b)$ la m\^eme 
loi apr\`es lissage par convolution; ces graphiques sont les m\^emes que 
sur les figures 11 (en haut), et 14 ($\varepsilon = 0.42$) du chapitre {\bf
VII}. Si $X$ est une variable al\'eatoire de loi $\{ x_j\; ; \; (p_j) \}$, son
carr\'e a pour loi $\{ x_j^2\; ; \; (p_j)\}$: c'est-\`a-dire que les
probabilit\'es $p_j$ ne changent pas, mais les valeurs $x_j$ sont
remplac\'ees par les carr\'es. Si parmi les $x_j$ il y en a deux distinctes
$x_{j_1}$ et $x_{j_2}$ qui ont le m\^eme carr\'e, la probabilit\'e de ce
carr\'e sera \'evidemment $p_{j_1} + p_{j_2}$.
Dans le cas de la loi bin\^omiale repr\'esent\'ee
sur la figure $43\; a$, qui est sym\'etrique,  cela se produit pour 
chacune des $x_j$, sauf $0$. Ainsi la probabilit\'e de $+x_j$ est
(approximativement) $p_0\exp (- x_j^2/2)$, et la probabilit\'e de $-x_j$
est aussi $p_0\exp (- x_j^2/2)$, de sorte que la probabilit\'e de $x_j^2$
sera $2p_0 \exp (- x_j^2/2)$, \`a l'exception de $x_0 = 0$ qui est seule 
\`a avoir $0$ pour carr\'e, et dont la probabilit\'e ne sera donc pas $2p_0$,
mais $p_0$. C'est pourquoi on voit en $(c)$ que les b\^atons du graphique
(dont la longueur est proportionnelle \`a la probabilit\'e correspondante)
sont deux fois plus longs que les b\^atons du graphique $(a)$, \`a
l'exception du tout premier,  qui se confond presque avec le second. 
Mais ce n'est pas la seule diff\'erence;  car sur
le graphique $(a)$ les b\^atons sont \'equidistants (en effet les valeurs
$x_j$ de la loi bin\^omiale sont toutes des multiples entiers de $x_1$), et
sur le graphique $(c)$ ils ne le sont plus car leurs abscisses sont les
carr\'es des pr\'ec\'edentes. Ainsi les valeurs prises par le carr\'e sont
plus denses au voisinage de z\'ero et moins denses loin de l'origine. Si on
effectue une convolution sur la loi $(c)$, afin de mieux visualiser la {\it
densit\'e} de la loi, on obtient le graphique $(d)$, qui ne ressemble plus 
du tout au graphique $(b)$. 
\medskip
La loi de probabilit\'e qui est repr\'esent\'ee par le graphique $(c)$ ou, 
sous forme liss\'ee, par le graphique $(d)$, est appel\'ee {\it la loi du
$\chi^2$ \`a un degr\'e de libert\'e}. En toute rigueur, la loi du $\chi^2$ 
\`a un degr\'e de libert\'e n'est qu'asymptotiquement la loi du carr\'e
d'une variable {\it bin\^omiale}, 
c'est-\`a-dire que la loi discr\`ete$(c)$ a pour limite la loi du $\chi^2$ lorsque le nombre de b\^atons tend
vers l'infini et que la
distance entre eux tend conjointement vers z\'ero, de m\^eme que la loi
gaussienne est la limite de la loi bin\^omiale. Les graphiques $(c)$ et $(d)$ ne correspondent donc qu'asymptotiquement
\`a la loi du $\chi^2$ \`a un degr\'e de libert\'e. 
\vskip12pt plus6pt minus4pt
On peut r\'esumer:
\medskip
\line{\hfill \vrule height59.75pt depth 0pt width 0.4pt
\vbox{\hsize=11.5cm \hrule
\vskip5pt
\centerline{\vbox{\hsize=108mm
La densit\'e normalis\'ee $\chi_1^2 (t)$ \`a un degr\'e de libert\'e est la
densit\'e du carr\'e d'une variable al\'eatoire de densit\'e gaus\-sienne 
${1\over\sdown{9}{\sqrt{2\pi}}}\exp (-x^2/2)$.
 } }
\vskip7pt
\hrule }\vrule height59.75pt depth 0pt width 0.4pt \hfill }
\vskip12pt plus6pt minus4pt
Mais dans $(XI.2.)$ il y avait une somme de carr\'es. Lorsqu'on addi\-tionne
des variables al\'eatoires gaussiennes ind\'ependantes, leur somme est
\'egalement gaussienne, avec une variance qui est la somme des 
variances de chacune;  cela r\'esulte d'une propri\'et\'e tr\`es
sp\'eciale que les densit\'es gaus\-siennes poss\`edent, \`a savoir
que la convolution de plusieurs densit\'es gaus\-siennes est aussi une
densit\'e gaus\-sienne. Les carr\'es ont pour densit\'e $\chi^2(t)$ et non
$\e^{-x^2/2}$, donc il n'y a aucune raison pour que leur somme soit aussi 
une densit\'e $\chi^2(t)$. Nous allons voir ci-dessous lorsque nous
calculerons ces fonctions que la densit\'e de la somme de plusieurs
carr\'es (disons $r$ carr\'es) n'est pas semblable \`a la densit\'e d'un 
seul carr\'e, et on l'appelle la densit\'e $\chi_r^2(t)$, ou densit\'e
$\chi^2(t)$ \`a $r$ degr\'es de libert\'e. 
\bigskip
On pourra donc retenir la d\'efinition suivante:
\medskip
\penalty-200
\line{\hfill \vrule height75.3pt depth 0pt width 0.4pt
\vbox{\hsize=11.5cm \hrule
\vskip5pt
\centerline{\vbox{\hsize=108mm
La densit\'e normalis\'ee $\chi_r^2 (t)$ \`a $r$ degr\'es de libert\'e est 
la densit\'e de la somme des carr\'es de $r$ variables al\'eatoires
sto\-chas\-ti\-que\-ment ind\'e\-pen\-dantes,  chacune de densit\'e 
gaussienne ${1\over\sdown{9}{\sqrt{2\pi}}}\exp (-x^2/2)$.
 } }
\vskip8.4pt
\hrule }\vrule height75.3pt depth 0pt width 0.4pt \hfill }

\bigskip

{\eightpoint
Soit $X$ une variable de loi $\e^{-x^2/2}$. On sous-entendra qu'il 
s'agit de
lois asymptotiques et qu'en r\'ealit\'e la loi est discr\`ete. Cela signifie que
$${\cal P}\, (a \leq X < b) \;\; \simeq \;\; {1\over \sdown{11} \sqrt{2\pi}}
\int_a^b \e^{-{\hbox{\sevenmi x}^{\hbox{\fiverm 2}} \over \sdown{4.5} 2}}dx$$
si $b-a$ est suffisamment grand pour contenir un nombre
statistiquement significatif de valeurs discr\`etes. Il suffit d'ailleurs
pour caract\'eriser la densit\'e, de conna{\^\i}tre la fonction
$$F(x) \;\; = \;\; {\cal P}\, (X<x) \;\; \simeq \;\; {1\over \sdown{11}
\sqrt{2\pi}} \int_{-\infty}^x 
\e^{-{\hbox{\sevenmi t}^{\hbox{\fiverm 2}}\over \sdown{4.5} 2}}dt \eqno (XI.3.)$$
appel\'ee {\it fonction de r\'epartition} de $X$; en effet, ${\cal P}\,
(a<X<b) = F(b) - F(a)$. Pour d\'eterminer la densit\'e du carr\'e de $X$, 
nous partirons donc du fait que la fonction de r\'epartition de $X$ est 
celle donn\'ee par $(XI.3.)$ et il s'agit de calculer la fonction de
r\'epartition $G(x)$ de $X^2$. 
\medskip
Or 
$$G(x) \;\; = \;\; {\cal P}\, (X^2<x) \;\; = \;\; 
{\cal P}\, (-\sqrt{x}<X<\sqrt{x})\;\;  
\simeq \;\; {1\over \sdown{11} \sqrt{2\pi}}\int_{-\sqrt{x}}^{\sqrt{x}} 
\e^{-{\hbox{\sevenmi t}^{\hbox{\fiverm 2}} \over
\sdown{4.5} 2}}dt $$
Puisque la fonction \`a int\'egrer ci-dessus est paire, on a aussi
$$G(x) \;\; \simeq \;\;\; {2\over \sdown{11} \sqrt{2\pi}}\int_{0}^{\sqrt{x}} 
\e^{-{\hbox{\sevenmi t}^{\hbox{\fiverm 2}}
\over \sdown{4.5} 2}}dt $$
Effectuons dans l'int\'egrale le changement de variable $s = t^2$; on obtient 
$$G(x) \;\; \simeq \;\;\; {1\over \sdown{11} \sqrt{2\pi}}\int_{0}^{x} 
\hbox{\tenmi s \vrule
height7.5pt width0pt}^{-{1 \over \sdown{4.5} 2}} \e^{-{\hbox{\sevenmi s}\over \sdown{4.5} 2}}\, ds $$
L'astuce, dans le changement de variable, \'etait de faire en sorte que
la
borne sup\'erieure de l'int\'egrale,  qui \'etait devenue $\sqrt{x}$, 
redevienne $x$.  Ainsi la probabilit\'e pour que $X^2$ soit inf\'erieur
\`a $x$ est l'int\'egrale de $0$ \`a $x$ (et non plus de $0$ \`a
$\sqrt{x}$) de la
densit\'e $\sqrt{1/2\pi }\; s^{-{1 / 2}}\, \e^{-{s / 2}}$;  c'est la preuve que la fonction sous l'int\'egrale est
bien la densit\'e cherch\'ee.  La densit\'e $\chi_1^2 (t)$ sera donc
$$\chi_1^2 (t) \;\; = \;\; \sqrt{{1\over 2\pi }}\; t^{-{1 \over \sdown{4.5} 
2}}\, \e^{-{\hbox{\sevenmi s} \over
\sdown{4.5} 2}} \eqno (XI.4.)$$
On voit que cette densit\'e tend vers l'infini comme $1/\sqrt{t}$ quand 
$t$ tend vers $0$.  Cette singularit\'e est \'evidemment un artefact du
passage \`a une limite continue:  il ne peut pas y avoir une telle densit\'e
infinie pour des lois discr\`etes et r\'eelles,  comme le montrent les
figures $42$ et $43\; (d)$.  Le ph\'enom\`ene doit \^etre compris de la 
fa\c{c}on suivante.  Nous avons vu au chapitre {\bf VII} que lorsqu'on dit
qu'une loi discr\`ete tend vers une densit\'e continue,  cela signifie que
c'est la loi \'echantillonn\'ee ou liss\'ee par convolution qui tend (au 
sens de la convergence uniforme des fonctions) vers la densit\'e continue;
mais la loi \'echantillonn\'ee ou liss\'ee est d'autant plus proche de sa
limite que les valeurs de la loi discr\`ete sont plus nombreuses et plus
proches les unes des autres:  ce qui fait qu'une loi discr\`ete ``tend''
vers une densit\'e continue,  est que la distance moyenne entre les valeurs
cons\'ecutives (le pas de discr\'etisation) tend vers z\'ero.  Si dans le
graphique de la figure $41$ ou celui de la figure $43\; (a)$ on augmente 
le nombre de valeurs en m\^eme temps qu'on les rapproche les unes des
autres (par exemple,  au lieu de prendre les $17$ valeurs $x_j = {1\over
10}j$ pour $j=0,\, 1,\, \ldots 16$,  chacune ayant la probabilit\'e ${1 \over
17}$,  on prend $1,7 \cdot 10^{n}$ valeurs $x_j = 10^{-n}j$ pour $j=0,\,
1,\, \ldots 1,7 \cdot 10^n-1$,  chacune ayant la probabilit\'e $1/1,7\cdot
10^n$, et on fait tendre $n$ vers l'infini),  on ne change rien au fait
que la
loi liss\'ee sera uniforme;  par contre la loi liss\'ee du carr\'e(figure
$42$) va tendre vers la densit\'e $1 / \sqrt{t}$ (bien entendu le
param\`etre $\varepsilon$ --- la longueur de corr\'elation --- du filtre 
de lissage devra tendre corr\'elativement vers z\'ero).  Il n'est pas
possible,  \`a cause du faible pouvoir s\'eparateur des graphiques,  de
pr\'esenter les dessins qui illustreraient le ph\'enom\`ene;  mais si on le
pouvait,  on verrait que la courbe de la figure $42$ tendrait vers la courbe 
$y = 1 / \sqrt{x}$:  en particulier l'abscisse de son maximum,  qui est
d\'ej\`a proche de z\'ero sur le graphique, s'en rapprocherait de plus en
plus,  tandis que son ordonn\'ee prendrait des valeurs de plus en plus
grandes.  Cela se comprend ais\'ement:  si les valeurs discr\`etes se
densifient,  le fait de les \'elever au carr\'e produira une densit\'e 
\'enorme tout pr\`es de z\'ero (par exemple si la distance entre les 
valeurs est $\varepsilon$,  la distance entre les valeurs du carr\'e au
voisinage imm\'ediat de z\'ero sera de l'ordre de $\varepsilon^2$).  Tant 
que la loi est discr\`ete,  cette densit\'e \'enorme ne peut toutefois pas
\^etre infinie,  et c'est pourquoi la courbe liss\'ee poss\`ede pr\`es de 
z\'ero un maximum aigu,  mais non une singularit\'e.  Ce maximum se
transforme cependant en singularit\'e \`a la limite continue. 
\medskip

\midinsert
\vbox to \blocksize{\null
\epsfxsize = \hsize
\line{\epsfbox{../images/fig44a.eps} }
\vskip4mm
\centerline{\eightpoint La densit\'e $\chi^2$ \`a un degr\'e de libert\'e.}
\vfill
\epsfxsize = \hsize
\line{\epsfbox{../images/fig44b.eps} }
\vskip4mm
\centerline{\eightpoint La densit\'e $\chi^2$ \`a deux degr\'es de
libert\'e.}
\vskip12mm
\centerline{\eightpoint figure 44}
\vskip15pt }
\endinsert

\midinsert
\vbox to \blocksize{\null
\epsfxsize = \hsize
\line{\epsfbox{../images/fig45a.eps} }
\vskip4mm
\centerline{\eightpoint La densit\'e $\chi^2$ \`a huit degr\'es de libert\'e.}
\vfill
\epsfxsize = \hsize
\line{\epsfbox{../images/fig45b.eps} }
\vskip4mm
\centerline{\eightpoint La densit\'e $\chi^2$ \`a seize degr\'es de
libert\'e.}
\vskip12mm
\centerline{\eightpoint figure 45}
\vskip8pt
\centerline{\vbox{\hsize=11cm\eightpoint 
Voici quatre exemplaires (figures 44 et 45) choisis parmi les densit\'es
$\chi^2$; les quatre
graphiques sont \`a la m\^eme \'echelle.} }
\vskip15pt }
\endinsert

\midinsert
\vbox to \blocksize{\null
\vfill
\epsfxsize = \hsize
\line{\epsfbox{../images/fig46.eps} }
\vskip8mm
\centerline{\eightpoint figure 46}
\vskip8mm
\centerline{\vbox{\hsize=11cm\eightpoint 
Voici, toujours \`a la m\^eme \'echelle que sur les figures 44 et 45, les
graphiques superpos\'es des six premi\`eres densit\'es $\chi_r^2$ ($r = 
1,\, 2,\, 3,\, 4,\, 5,\, 6$).} } 
\vfill }
\endinsert

On peut de la m\^eme mani\`ere calculer la densit\'e $\chi_r^2(t)$ \`a $r$
degr\'es de libert\'e. Cette fois il s'agit de la {\it somme} de $r$ carr\'es
ind\'ependants, on a donc $r$ variables al\'eatoires $X_1,\, X_2,\, X_3,\,
\ldots X_r$ stochastiquement ind\'ependantes, et pour chacune on a
$${\cal P}\, (X_j < x) \;\; \simeq \;\; \int_{-\infty }^x 
{1\over \sdown{11} \sqrt{2\pi}}\;
\e^{-{\hbox{\sevenmi t}^{\hbox{\fiverm 2}}\over \sdown{4.5} 2}}dt$$
Puisque les $X_j$ sont stochastiquement ind\'ependantes, leur loi 
conjointe s'obtient par factorisation, de sorte que pour la somme des
carr\'es on aura 
$$\eqalign{
{\cal P}\, (&X_1^2 + X_2^2 + X_3^2 + \cdots X_r^2 < x)\quad \simeq \cr
\noalign{\smallskip}
&\simeq\quad\int\!\!\int\!\!\int\!\!\cdots\int_{ \kern-56pt
\lower8pt\hbox{ $\scriptstyle t_1^2 + t_2^2 + t_3^2 +\cdots t_r^2 < x$}}
\biggl[ {1\over \sdown{11} \sqrt{2\pi}}\biggr]^r \e^{-{\hbox{\vrule depth2.4pt
width0pt$\scriptstyle
t_1$}^{\hbox{\fiverm 2}} \over \sdown{4.5} 2}}\e^{-{\hbox{\vrule depth2.4pt width0pt$\scriptstyle
t_2$}^{\hbox{\fiverm 2}} \over \sdown{4.5} 2}} \e^{-{\hbox{\vrule depth2pt width0pt$\scriptstyle
t_3$}^{\hbox{\fiverm 2}} \over \sdown{4.5} 2}} \cdots \; 
\e^{-{\hbox{\vrule depth2.4pt width0pt$\scriptstyle
t_r$}^{\hbox{\fiverm 2}}\over \sdown{4.5} 2}} \; dt_1\, dt_2\, dt_3 \cdots dt_r \cr
\noalign{\smallskip}
&=\quad\int\!\!\int\!\!\int\!\!\cdots\int_{ \kern-56pt
\lower8pt\hbox{ $\scriptstyle t_1^2 + t_2^2 + t_3^2 +\cdots t_r^2 < x$}}
\biggl[ {1\over \sdown{11} \sqrt{2\pi}}\biggr]^r
\e^{-{\hbox{\vrule depth2.4pt width0pt$\scriptstyle
t_1$}^{\hbox{\fiverm 2}} + \hbox{\vrule depth2.4pt width0pt$\scriptstyle
t_2$}^{\hbox{\fiverm 2}} + \hbox{\vrule depth2.4pt width0pt$\scriptstyle
t_3$}^{\hbox{\fiverm 2}} + \cdots + \hbox{\vrule depth2.4pt
width0pt$\scriptstyle t_r$}^{\hbox{\fiverm 2}} \over \sdown{4.5} 2}}\; dt_1\, 
dt_2\, dt_3 \cdots dt_r \cr }$$
On voit que dans cette int\'egrale multiple tout est \`a sym\'etrie 
sph\'erique, aussi bien la fonction \`a int\'egrer que le domaine
d'int\'egration. On peut donc avantageusement passer en coordonn\'ees
sph\'eriques, ce qui donne 
$${\cal P}\, (X_1^2 + X_2^2 + X_3^2 + \cdots X_r^2 < x)\;\; \simeq 
\;\; \biggl[ {1\over \sdown{11} \sqrt{2\pi}}\biggr]^r 
\int_{\rho^2 < x} \rho^{r-1} \e^{-{\hbox{\vrule depth2.4pt width0pt
$\scriptstyle
\rho$}^{\hbox{\fiverm 2}} \over \sdown{4.5} 2}} d\rho
\;\cdot\; {2\,\pi^{r \over \sdown{4.5} 2} \over \sdown{11} \Gamma\bigl(
{\hbox{\eightmi r} \over
2}\bigr)}$$le facteur $2\,\pi^{r/2} \big/ \,\Gamma (r/2)$ provenant de l'int\'egration
des coordonn\'ees sph\'e\-riques angulaires (c'est l'aire de la sph\`ere
unit\'e dans l'espace \`a $r$ dimensions).
\medskip
\`A partir de l\`a nous pouvons proc\'eder comme dans le calcul 
pr\'ec\'edent \`a une dimension: nous faisons le changement de variable 
$s = \rho^2$, ce qui donne
$${\cal P}\, (X_1^2 + X_2^2 + X_3^2 + \cdots X_r^2 < x)\quad \simeq
\quad {1\over \sdown{11.5} 2^{r \over \sdown{4.5} 2}\,\Gamma\bigl(
{\hbox{\eightmi r} \over \sdown{4.5} 2}\bigr)} \int_0^x s^{{r \over 
\sdown{4.5} 2} - 1} \e^{-{\hbox{$\scriptstyle
s$} \over \sdown{4.5} 2}} ds$$
Sous cette forme on voit appara{\^\i}tre la densit\'e que nous cherchions. 
On peut donc conclure
$$\chi_r^2(t) \;\; = \;\; {1\over \sdown{11.5} 2^{r \over \sdown{4.5}
2}\,\Gamma\bigl({\hbox{\eightmi r} \over \sdown{4.5} 2}\bigr)}\;
t^{{r \over \sdown{4.5} 2} - 1} \e^{-{\hbox{$\scriptstyle
t$} \over \sdown{4.5} 2}} \eqno (XI. 5.)$$
} %%%% end of \eightpoint

\midinsert
\epsfxsize = \hsize
\line{\epsfbox{../images/fig47.eps} }
\vskip3mm
\centerline{\eightpoint figure 47}
\vskip8pt
\centerline{\vbox{\hsize=11cm\eightpoint Ici sont superpos\'es les
graphiques des densit\'es $\chi_r^2$ pour $r = 3$ \`a $r = 16$. 
L'\'echelle des abscisses est la m\^eme que sur les figures 44, 45, 46, 
mais les ordonn\'ees ont \'et\'e agrandies $2,5$ fois.} }
\vskip4mm
\endinsert

{\bf XI. 2. Le test du $\chi^2$}
\medskip
Revenons maintenant \`a $(XI.2.)$ On avait calcul\'e la somme des 
carr\'es des six fluctuations du d\'e, et il s'agissait de d\'eterminer la
loi de probabilit\'e de cette somme de carr\'es, afin de pouvoir en 
d\'eduire si la valeur calcul\'ee \'etait probable ou pas: si la valeur
effectivement observ\'ee est tr\`es peu probable, on en d\'eduit que le
d\'e est tr\`es probablement mal \'equilibr\'e; si au contraire la valeur
effectivement observ\'ee est une valeur probable, on n'a plus aucune 
raison de conclure que le d\'e est mauvais. Mais on ne peut plus utiliser 
la loi gaussienne pour estimer ces probabilit\'es, puisque la somme des
carr\'es ne suit pas une loi gaussienne. Nous allons \'etablir maintenant
que la somme des six carr\'es suit une loi qui {\it asymptotiquement} a
pour densit\'e $\chi_5^2(t)$ (et non $\chi_6^2(t)$). Math\'ematiquement,
cela s'\'enonce ainsi:
\medskip
{\narrower {\bf Th\'eor\`eme:} Soient $X_1$, $X_2$, $\ldots$, $X_r$ des
variables al\'eatoires gaus\-siennes normalis\'ees (c'est-\`a-dire
que leurs lois ont une densit\'e \'egale \`a $(1/2\pi )\; \e^{-x^2/2}$, 
de moyenne $0$ et
d'\'ecart-type $1$). Si la loi conjointe de ces variables a pour densit\'e
$(1/2\pi )^r\exp\{ -(x_1^2 + x_2^2 + \cdots x_r^2)/2\}$ et si elles sont
li\'ees par la relation $X_1 + X_2 + \cdots + X_r = 0$, alors la somme 
de leurs carr\'es a pour densit\'e $\chi_{r-1}^2(t)$. \par }
\medskip
D'apr\`es ce qui a \'et\'e vu au \S 1 ci-dessus, on pourrait dire que 
si les $r = 6$ fluctuations \'etaient stochastiquement ind\'ependantes
les unes des autres, la somme de leurs carr\'es aurait pour densit\'e
(approximativement) $\chi_6^2(t)$. C'est bien s\^ur parce que les six
fluctuations $X_1,\,  X_2,\, X_3,\, X_4,\, X_5,\, X_6$ sont li\'ees
par la relation lin\'eaire $X_1 + X_2 + X_3 + X_4 + X_5 + X_6 = 0$ que 
la dimension baisse ainsi d'une unit\'e. Mais il n'est pas \'evident 
sans v\'erification math\'ematique que cette liaison lin\'eaire ne modifie
pas par exemple l'\'ecart-type. Afin de bien comprendre ce qui se passe, 
il faut examiner cette affaire de plus pr\`es, et pour cela nous allons
{\it d\'emontrer} le th\'eor\`eme. Pour que cette d\'emonstration soit plus
imag\'ee, nous gardons le cas particulier $r=6$ du d\'e.
\medskip
Suivre cette d\'emonstration demande un effort: ne la lisez que si vous
souhaitez vraiment comprendre {\it pourquoi} la relation lin\'eaire entre
les $X_j$ (qui fait baisser d'une unit\'e le nombre de degr\'es de libert\'es)
ne change pas le caract\`ere gaussien de la densit\'e.
\medskip
Si les six fluctuations $X_1,\, X_2,\, X_3,\, X_4,\, X_5,\, X_6$ \'etaient
stochastiquement ind\'ependantes les unes des autres, la densit\'e de leur
loi conjointe serait 
$$f_6(t_1,\, t_2,\, t_3,\, t_4,\, t_5,\, t_6) \;\; = \;\; \biggl[
{1\over \sdown{11} \sqrt{2\pi}}\biggr]^r
\e^{-{\hbox{\vrule depth2.4pt width0pt$\scriptstyle
t_1$}^{\hbox{\fiverm 2}} + \hbox{\vrule depth2.4pt width0pt$\scriptstyle
t_2$}^{\hbox{\fiverm 2}} + \hbox{\vrule depth2.4pt width0pt$\scriptstyle
t_3$}^{\hbox{\fiverm 2}} + \cdots + \hbox{\vrule depth2.4pt
width0pt$\scriptstyle t_r$}^{\hbox{\fiverm 2}} \over 2}} \quad \eqno (XI.6.)$$
Si elles sont li\'ees par la relation $X_1 + X_2 + X_3 + X_4 + X_5 + X_6 =
0$, cela signifie que le point $M$ de coordonn\'ees $X_1,\, X_2,\, X_3,\, 
X_4,\, X_5,\, X_6$ est situ\'e sur l'hyperplan $H$ d'\'equation $X_1 + 
X_2 + X_3 + X_4 + X_5 + X_6 = 0$. Le fait que les valeurs des $X_j$
soient enti\`eres et que leur densit\'e ne soit donn\'ee par $(XI.6.)$
qu'asymptotiquement ne change rien \`a cela. Supposons qu'on effectue un
\'echantillonnage (ce serait la m\^eme chose pour une convolution) des
valeurs discr\`etes. Pour effectuer cet \'echantillonnage on peut par
exemple quadriller l'espace \`a six dimensions en hypercubes ou pav\'es
dont
les ar\`etes sont parall\`eles aux six axes $OX_1,\, OX_2,\, OX_3,\, 
OX_4,\, OX_5,\, OX_6$,  mais on peut tout aussi bien le quadriller selon 
six autres directions mutuellement perpendiculaires,  et telles que les 
cinq premi\`eres soient parall\`eles \`a l'hyperplan $H$.  On choisira bien
s\^ur la taille $\varepsilon$ des pav\'es de telle sorte qu'ils contiennent 
chacun un nombre assez grand de points de la distribution discr\`ete, 
tout en \'etant suffisamment petit pour conserver un sens \`a la notion 
de densit\'e continue.  Dans ces conditions il est bien clair que la
probabilit\'e pour que le point $M$ de coordonn\'ees $X_1,\, X_2,\,
X_3,\, X_4,\, X_5,\, X_6$ soit situ\'e dans un pav\'e $P$ est
approximativement donn\'ee par la densit\'e asymptotique continue: 
$$\eqalignno{ 
{\cal P}\, (M \in P)\;\;  &\simeq \;\; \int_P f_6(t_1,\, t_2,\, t_3,\,
t_4,\, t_5,\,
t_6)\; dt_1\, dt_2\, dt_3\, dt_4\, dt_5\, dt_6 \cr&\simeq \;\; \varepsilon^6
\cdot f(M_0) &(XI.7.)\cr } $$o\`u $M_0$ d\'esigne par exemple le centre du pav\'e.  Or la densit\'e 
$f(M)$ ne d\'epend en fait que de $\rho = OM$ (la distance de $M$ \`a
l'origine). Si on impose aux six variables $X_1,\, X_2,\, X_3,\, X_4,\,
X_5,\, X_6$ d'avoir une somme nulle, cela signifie qu'on ne consid\`ere 
que les points $M$ situ\'es sur l'hyperplan $H$, ou, ce qui revient au 
m\^eme puisque la densit\'e $f$ n'a pas de discontinuit\'e, les points qui
sont distribu\'es dans une bande $H_\varepsilon$ de largeur $\varepsilon$
autour de l'hyperplan (on a suppos\'e que le quadrillage de l'espace en
pav\'es est parall\`ele \`a l'hyperplan).  Ainsi la densit\'e de la loi
conjointe des $X_j$ li\'ees par la relation $\sum X_j = 0$ sera donn\'ee
par la
r\'epartition des probabilit\'es selon les pav\'es contigus \`al'hyperplan: 
il s'agit donc simplement des probabilit\'es conditionnelles${\cal P}\, (M
\in P \mid M \in H_\varepsilon)$. Or cette probabilit\'econditionnelle est, 
conform\'ement \`a $(IV.1.)$,  donn\'ee par$$\eqalign{ 
&{\cal P}\, (M \in P \mid M \in H_\varepsilon) \quad = \quad {{\cal P}\, 
(M \in P) \over {\cal P}\, (M \in H_\varepsilon)} \quad \simeq\cr 
&\simeq \quad { {\displaystyle \int_P \vrule depth13pt width0pt
f_6(s_1,\, s_2,\, s_3,\, s_4,\,
s_5,\, s_6)\; ds_1\, ds_2\, ds_3\, ds_4\, ds_5\, ds_6} \over
{\displaystyle\int_{0 \leq s_6 < \varepsilon}\vrule height19pt width0pt
f_6(s_1,\, s_2,\, s_3,\,
s_4,\, s_5,\, s_6)\; ds_1\, ds_2\, ds_3\, ds_4\, ds_5\, ds_6} }\cr } 
\eqno (XI.8.)$$ 
En effet, la densit\'e $f(M)$ ne d\'ependant que de $\rho = OM$
est invariante par changement de rep\`ere orthonorm\'e, de sorte que 
nous pouvons int\'egrer par rapport aux coordonn\'ees $s_1,\, s_2,\, 
s_3,\, s_4,\, s_5,\, s_6$ du nouveau rep\`ere, dans lequel l'\'equation de
l'hyperplan $H$ est $s_6 = 0$: la bande de largeur $\varepsilon$ autour 
de $H$, form\'ee par l'ensemble des pav\'es contigus \`a $H$ (qui est le
domaine de l'int\'egrale au d\'enominateur), est la r\'egion de l'espace
o\`u $0 \leq s_6 < \varepsilon$. La densit\'e $f(M)$ peut \^etre
refactoris\'ee dans le nouveau rep\`ere. En effet: 
$$\eqalignno{ 
f(M)\;\; &= \;\; \e^{-{\hbox{\vrule depth2.4pt width0pt$\scriptstyle
t_1$}^{\hbox{\fiverm 2}} \over 2}} \cdot 
\e^{-{\hbox{\vrule depth2.4pt width0pt$\scriptstyle
t_2$}^{\hbox{\fiverm 2}} \over 2}} \cdot 
\e^{-{\hbox{\vrule depth2.4pt width0pt$\scriptstyle
t_3$}^{\hbox{\fiverm 2}} \over 2}} \cdot 
\e^{-{\hbox{\vrule depth2.4pt width0pt$\scriptstyle
t_4$}^{\hbox{\fiverm 2}} \over 2}} \cdot 
\e^{-{\hbox{\vrule depth2.4pt width0pt$\scriptstyle
t_5$}^{\hbox{\fiverm 2}} \over 2}} \cdot 
\e^{-{\hbox{\vrule depth2.4pt width0pt$\scriptstyle
t_6$}^{\hbox{\fiverm 2}} \over 2}} \cr
&= \;\; \e^{-{\hbox{\vrule depth2pt width0pt$\scriptstyle
t_1$}^{\hbox{\fiverm 2}} + \hbox{\vrule depth2.4pt width0pt$\scriptstyle
t_2$}^{\hbox{\fiverm 2}} + \hbox{\vrule depth2.4pt width0pt$\scriptstyle
t_3$}^{\hbox{\fiverm 2}} + \hbox{\vrule depth2.4pt width0pt$\scriptstyle
t_4$}^{\hbox{\fiverm 2}} + \hbox{\vrule depth2.4pt width0pt$\scriptstyle
t_5$}^{\hbox{\fiverm 2}} + \hbox{\vrule depth2.4pt width0pt$\scriptstyle
t_6$}^{\hbox{\fiverm 2}} \over 2}} \cr
&= \;\; \e^{-{\hbox{\vrule depth2.4pt width0pt$\scriptstyle
s_1$}^{\hbox{\fiverm 2}} + \hbox{\vrule depth2.4pt width0pt$\scriptstyle
s_2$}^{\hbox{\fiverm 2}} + \hbox{\vrule depth2.4pt width0pt$\scriptstyle
s_3$}^{\hbox{\fiverm 2}} + \hbox{\vrule depth2.4pt width0pt$\scriptstyle
s_4$}^{\hbox{\fiverm 2}} + \hbox{\vrule depth2.4pt width0pt$\scriptstyle
s_5$}^{\hbox{\fiverm 2}} + \hbox{\vrule depth2.4pt width0pt$\scriptstyle
s_6$}^{\hbox{\fiverm 2}} \over 2}} \cr
&= \;\; \e^{-{\hbox{\vrule depth2.4pt width0pt$\scriptstyle
s_1$}^{\hbox{\fiverm 2}} \over 2}} \cdot 
\e^{-{\hbox{\vrule depth2.4pt width0pt$\scriptstyle
s_2$}^{\hbox{\fiverm 2}} \over 2}} \cdot 
\e^{-{\hbox{\vrule depth2.4pt width0pt$\scriptstyle
s_3$}^{\hbox{\fiverm 2}} \over 2}} \cdot 
\e^{-{\hbox{\vrule depth2.4pt width0pt$\scriptstyle
s_4$}^{\hbox{\fiverm 2}} \over 2}} \cdot 
\e^{-{\hbox{\vrule depth2.4pt width0pt$\scriptstyle
s_5$}^{\hbox{\fiverm 2}} \over 2}} \cdot 
\e^{-{\hbox{\vrule depth2.4pt width0pt$\scriptstyle
s_6$}^{\hbox{\fiverm 2}} \over 2}} \cr }$$
Si en utilisant cela on refactorise l'int\'egrale au d\'enominateur de
$(XI.8.)$ et qu'on remplace celle du num\'erateur par son expression
approch\'ee (pour $\varepsilon$ petit) donn\'ee par $(XI.7.)$, on obtient 
pour la probabilit\'e conditionnelle
$${\cal P}\, (M \in P \mid M \in H)\;\;  
\simeq \;\; {\varepsilon^6 f_6(s_1,\, s_2,\,s_3,\, s_4,\, s_5,\, 0) \over 
\varepsilon \int f_6(s_1,\, s_2,\,s_3,\, s_4,\, s_5,\, 0)\; ds_1\, 
ds_2\, ds_3\, ds_4\, ds_5 } $$
Mais $f_6(s_1,\, s_2,\,s_3,\, s_4,\, s_5,\, 0) = f_5(s_1,\, s_2,\,s_3,\,
s_4,\, s_5)\, \big/ \sqrt{2\pi}$, de sorte que finalement apr\`es
simplification 
$${\cal P}\, (M \in P \mid M \in H)\;\;  
\simeq \;\; \varepsilon^5 f_5(s_1,\, s_2,\,s_3,\, s_4,\, s_5)$$
On comprend ainsi que 
\smallskip
a) la loi conjointe des six variables $X_1,\, X_2,\, X_3,\, X_4,\, X_5,\,
X_6$ li\'ees par la relation $X_1 + X_2 + X_3 + X_4 + X_5 + X_6 = 0$
est une loi discr\`ete de valeurs distribu\'ees sur l'espace \R${}^6$, mais
toutes situ\'ees sur l'hyperplan $H$; 
\smallskip
b) si {\it \`a l'int\'erieur de} $H$ on choisit un rep\`ere orthonorm\'e, les 
valeurs discr\`etes de cette loi conjointe peuvent \^etre rep\'er\'ees par
les {\it cinq} coordonn\'ees $s_1,\, s_2,\, s_3,\, s_4,\, s_5$, et (apr\`es
\'echantillonnage par pav\'es) elles sont distribu\'ees selon la densit\'e 
$$f_5(s_1,\, s_2,\, s_3,\, s_4,\, s_5)\;\;  = \;\; 
\e^{-{\hbox{\vrule depth2.4pt width0pt$\scriptstyle
s_1$}^{\hbox{\fiverm 2}} + \hbox{\vrule depth2.4pt width0pt$\scriptstyle
s_2$}^{\hbox{\fiverm 2}} + \hbox{\vrule depth2.4pt width0pt$\scriptstyle
s_3$}^{\hbox{\fiverm 2}} + \hbox{\vrule depth2.4pt width0pt$\scriptstyle
s_4$}^{\hbox{\fiverm 2}} + \hbox{\vrule depth2.4pt width0pt$\scriptstyle
s_5$}^{\hbox{\fiverm 2}} \over 2}} $$
Or la propri\'et\'e caract\'eristique de la fonction exponentielle (\`a
savoir que l'expo\-nen\-tielle d'une somme est le produit des
expo\-nen\-tielles) a pour cons\'equence que cette densit\'e se factorise, 
de sorte que tout se passe comme si les valeurs discr\`etes prises par le
vecteur al\'eatoire $X_1,\, X_2,\, X_3,\, X_4,\, X_5,\, X_6$, qui sont toutes
situ\'ees sur l'hyperplan $H$, \'etaient les valeurs discr\`etes d'un vecteur
al\'eatoire $S_1,\, S_2,\, S_3,\, S_4,\, S_5$ \`a cinq composantes {\it
stochastiquement ind\'ependantes} (mais dans le nouveau rep\`ere). 
\medskip
La d\'emonstration du th\'eor\`eme s'ach\`eve ici.
\medskip
On peut conclure que lorsque six variables al\'eatoires $X_1$,
$X_2$, $X_3$, $X_4$, $X_5$, $X_6$, chacune de densit\'e $e^{-t^2/2}$, sont
li\'ees par la relation $X_1 + X_2 + X_3 + X_4 + X_5 + X_6 = 0$, leur loi
conjointe est agenc\'ee comme suit: 
\smallskip
a) la probabilit\'e pour que le vecteur $X_1,\, X_2,\, X_3,\, X_4,\, X_5,\,
X_6$ soit en dehors de l'hyperplan est nulle;
\smallskip
b) \`a l'int\'erieur de l'hyperplan elle est approximativement la
m\^eme que celle de cinq
variables al\'eatoires {\it stochastiquementind\'ependantes} et ayant
chacune la densit\'e $e^{-t^2/2}$.
\medskip
Pour d\'emontrer cela, nous avons utilis\'e, de fa\c{c}on essentielle, 
deux
propri\'et\'es de la densit\'e gaussienne: la premi\`ere est qu'ellene d\'epend que de la somme des carr\'es des coordonn\'ees et est donc
invariante par changement de rep\`ere orthonorm\'e (c'est cette
propri\'et\'e qui entra{\^\i}ne que la trace de $f_6$ sur l'hyperplan est
$f_5$) et la seconde est que l'exponentielle d'une somme est le produit 
des exponentielles (c'est cette propri\'et\'e qui permet la factorisation 
de $f_5$ en produit, prouvant ainsi l'ind\'ependance stochastique des 
cinq nouvelles coordonn\'ees).  Le th\'eor\`eme \'enonc\'e ci-dessus 
n'est donc vrai qu'\`a la limite o\`u les $X_j$ sont gaussiennes; 
il n'est
qu'approximativement vrai pour la loi discr\`ete {\it exacte}. Cela peut se comprendre \`a partir d'arguments g\'eom\'etriques simples: 
supposons
pour all\'eger qu'on ait trois variables $X_1$, $X_2$, $X_3$au lieu de six,
 avec $X_1 + X_2 + X_3 = 0$.  On peut les exprimer enfonction de deux
variables ind\'ependantes $S_1$,  $S_2$, par exemple$$\eqalign{
X_1 &= S_1 + S_2 \cr
X_2 &= S_1 - S_2 \cr
X_3 &= -2 S_1 \cr }$$
Or si $S_1$ et $S_2$ prennent toutes les valeurs enti\`eres possibles, 
$X_1$ et $X_2$ auront n\'ecessairement la m\^eme parit\'e et $X_3$ sera
toujours paire, donc on ne couvrira pas ainsi toutes les valeurs que
peuvent prendre $X_1$, $X_2$, $X_3$. On pourra essayer toutes les
combinaisons pour exprimer $X_1$, $X_2$, $X_3$ en fonction de $S_1$, 
$S_2$ (de sorte que $X_1$, $X_2$, $X_3$ prennent des valeurs
enti\`eres et $X_1 + X_2 + X_3 = 0$), il sera impossible de recouvrir
exactement les valeurs prises par $X_1$, $X_2$, $X_3$; soit on n'en
retrouvera qu'une partie, soit on devra y ajouter des demi-entiers. Cela 
est d\^u simplement \`a ce que, si l'espace \R${}^3$ est invariant par
rotation, il n'en est plus de m\^eme pour une partie {\it discr\`ete} de
\R${}^3$. Par contre la propri\'et\'e est {\it approximativement} vraie 
lorsque les lois discr\`etes sont {\it approximativement} gaussiennes.
\medskip
Par ailleurs, en y regardant de pr\`es, on comprendra
\'egalement que les $X_j$ doivent avoir {\it le m\^eme \'ecart-type} (ici
il valait $1$): en effet, si les \'ecarts-types des $X_j$ \'etaient
diff\'erents, il n'y aurait plus d'invariance par rotation (toutefois on 
aurait pu pallier cet inconv\'enient en choisissant dans $H$ un rep\`ere
orthogonal, mais non norm\'e).
\medskip
L'importante propri\'et\'e que nous venons d'\'etablir, que la loi conjointe
de $r$ variables al\'eatoires gaussiennes $X_1,\, X_2,\, X_3, \ldots X_r$
d'\'ecart-type $1$ et de somme nulle est \'egalement une gaussienne, 
mais avec un degr\'e de libert\'e (une dimension) de moins, a donc pour 
cons\'equence que la somme de leurs carr\'es suit une loi dont la densit\'e 
est $\chi_{r-1}^2$; en effet, la somme des carr\'es des $X_j$ est alors
\'egale \`a la somme des carr\'es de $r-1$ variables al\'eatoires,
\'egalement gaussiennes et d'\'ecart-type $1$, mais stochastiquement
ind\'ependantes. 
\medskip
Si on veut tester un d\'e \`a l'aide de la somme $S$ des carr\'es des
fluctuations $(XI.2.)$, il faut donc utiliser la densit\'e $\chi_{5}^2$; 
mais comme celle-ci est pr\'evue pour des variables al\'eatoires 
d'\'ecart-type $1$, il faut encore se ramener \`a ce cas. En effet, les
fluctuations d'un d\'e qu'on a jet\'e $6n$ fois ont un \'ecart-type \'egal 
\`a $\sqrt{n}$; on se ram\`enera donc \`a un \'ecart-type \'egal \`a $1$ en 
divisant chaque fluctuation par $\sqrt{n}$.  Donc la somme $S$ doit \^etre
divis\'ee par $1000$ pour qu'on puisse lui appliquer la loi $\chi_{5}^2$.
La probabilit\'e pour que $S$ soit sup\'erieure \`a un seuil donn\'e $x$ 
peut donc \^etre calcul\'ee approximativement par
$${\cal P}\, \Bigl( {S \over 1000} > x \Bigr) \;\; \simeq \;\; 
\int_x^\infty
\chi_5^2 (t) \, dt$$
Bien s\^ur l'int\'egrale ci-dessus ne se calcule pas par primitives, mais
nu\-m\'e\-ri\-que\-ment (il y a pour cela des logiciels ou des tables); dans 
l'exemple pris ici, $S = 6400$, donc $S/1000 = 6,4$; si on consulte alors
une table on voit que la probabilit\'e pour que $S$ soit sup\'erieur ou 
\'egal \`a $6,4$ est environ $0,27$; ces fluctuations n'\'etant pas ``peu
probables'' on n'est donc pas fond\'e \`a conclure que le d\'e est truqu\'e 
(ce qui ne signifie pas qu'on est fond\'e \`a conclure que le d\'e n'est pas
truqu\'e). Si par contre la somme $S$ des carr\'es des six fluctuations
avait \'et\'e \'egale \`a $15\, 000$, la m\^eme table aurait montr\'e
qu'avec un d\'e normal une telle amplitude n'avait qu'une chance sur cent 
de se produire; alors on aurait \'et\'e fond\'e \`a conclure que le d\'e est 
probablement truqu\'e. 
\medskip
On voit que les op\'erations \`a faire sont simples:
\smallskip
a) se fixer un seuil de certitude $\alpha$, par exemple $\alpha = 0.99$;
\smallskip
b) calculer la somme $S$ des carr\'es des fluctuations;
\smallskip
c) diviser par la variance ($n/6$ si on a lanc\'e le d\'e $n$ fois) pour 
obtenir $x = S \bigl/ (n/6)$; 
\smallskip
d) chercher sur une table la valeur de $\int_0^x\chi_5^2 (t)\, dt$;  si
celle-ci est sup\'erieure \`a $\alpha$, on a d\'epass\'e le seuil, ce qui 
signifie que la probabilit\'e pour que les fluctuations aient une telle
amplitude est inf\'erieure \`a $1-\alpha$, et que donc le d\'e est
probablement fauss\'e. 
\medskip
C'est cette s\'erie d'op\'erations qui constitue le {\it test du} $\chi^2$.
\medskip
Tout ce qui a \'et\'e dit depuis le d\'ebut de ce chapitre avait pour but de
montrer que ce test du $\chi^2$ est math\'ematiquement \'equivalent \`a 
ce qu'aurait donn\'e une analyse s\'epar\'ee de chacune des six
fluctuations, et utilisant la loi gaussienne. En effet, on aurait pu
\'egalement chercher s\'epar\'ement pour chacune des six fluctuations
quelle probabilit\'e elle avait de se produire (ou plut\^ot: quelle \'etait la
probabilit\'e pour que la fluctuation ait une amplitude au moins \'egale 
\`a la fluctuation observ\'ee), puis accepter ou rejeter, selon le r\'esultat 
de ces six calculs, l'hypoth\`ese que le d\'e est fauss\'e. L'avantage d'une
\'etude s\'epar\'ee des six fluctuations est qu'elle permet de savoir
laquelle ou lesquelles des faces du d\'e sont favoris\'ees: si on souhaite
disposer de cette information, il ne faut pas faire le test du $\chi^2$, 
mais l'analyse s\'epar\'ee. Par contre le test du $\chi^2$ est bien plus
rapide et est donc pr\'ef\'erable d\`es lors que seul importe le r\'esultat
global.
\medskip
Il reste une derni\`ere remarque \`a faire. Les densit\'es $\chi_1^2 (t)$ 
et $\chi_2^2 (t)$ sont grandes au voisinage de $t = 0$, et deviennent
petites pour $t$ grand, comme on peut voir sur la figure 44. Par contre
on peut voir sur la figure 45 que pour $r$ grand (par exemple $r =
8$ ou $r = 16$) la densit\'e est petite non seulement pour les grandes
valeurs de $t$, mais aussi pour les petites. Cela signifie que lorsqu'il y 
a beaucoup de fluctuations, c'est-\`a-dire beaucoup de degr\'es de
libert\'e,  la somme $S$ des carr\'es de ces fluctuations a non 
seulement une faible probabilit\'e d'\^etre grande, mais aussi une faible
probabilit\'e d'\^etre petite. Or cela se comprend ais\'ement si on y
r\'efl\'echit. En effet, $S$ est la somme des carr\'es, donc si $S$ est
petit, c'est que {\it toutes} les fluctuations sont petites; la faible
densit\'e au voisinage de z\'ero signifie donc qu'il est peu probable que
toutes les fluctuations soient petites en m\^eme temps. De fa\c{c}on
plus quantitative: la moyenne des carr\'es des fluctuations est la 
variance $V$; il est peu probable que ces carr\'es deviennent beaucoup 
plus grands que la variance, mais il est peu probable aussi (s'ils sont 
nombreux) qu'ils soient {\it tous en m\^eme temps} nettement plus
petits que la variance. Cela signifierait en quelque sorte que le hasard a
{\it trop bien} fait les choses. Ainsi pour une pi\`ece de monnaie qu'on
lance $2000$ fois, le r\'esultat le plus probable est d'avoir $1000$ fois
pile (en avoir $999$ est un tout petit peu moins probable). Mais si on
lance $6000$ fois un d\'e, obtenir {\it exactement} $1000$ fois chacune
des six faces n'est pas le r\'esultat le plus probable, c'est au contraire
improbable, et si au lieu de six faces, il y en avait douze, il serait
encore plus improbable que toutes les douze apparaissent exactement
$500$ fois chacune. En calculant un peu, on peut constater que la
densit\'e $\chi_5 (t)$ est maximum pour $t = 3,5$; dans l'exemple o\`u 
on lance $6000$ fois un d\'e, cela signifie que la valeur ``la plus
probable'' de $S / 1000$ est $3,5$ et non z\'ero. 
\medskip
Toutefois lorsqu'on pratique le test du $\chi^2$ pour savoir par exemple 
si le d\'e est mal \'equilibr\'e, on ne peut pas interpr\'eter les faibles 
valeurs de $S$ de la m\^eme fa\c{c}on que les grandes. En effet, si le d\'e 
est mal \'equilibr\'e, il sera encore moins probable d'avoir des valeurs
``trop'' petites pour $S$; par cons\'equent on ne pourra conclure que le 
d\'e est mal \'equilibr\'e que si $S$ se situe dans les grandes valeurs
improbables. Les valeurs trop petites de $S$ signifient que les r\'esultats
\'etaient {\it trop} exacts. Imaginons un exp\'erimentateur qui fraude; il
annonce comme r\'esultats de mesure des r\'esultats invent\'es,
c'est-\`a-dire calcul\'es d'apr\`es ce qu'il souhaite prouver
frauduleusement. Si $V$ est alors la variance des erreurs de mesures (qui
est donc un param\`etre caract\'eristique des appareils utilis\'es), cet
exp\'erimentateur devra veiller \`a inventer \'egalement des fluctuations
autour des r\'esultats invent\'es, de telle sorte que la somme des carr\'es
de ces fluctuations, divis\'ee par $V$, corresponde \`a un maximum de la
densit\'e $\chi^2$ (avec le nombre convenable de degr\'es de libert\'e). S'il
ne prend pas cette pr\'ecaution, les experts appel\'es par des coll\`egues
soup\c{c}onneux ne manqueront pas d'invoquer un test du $\chi^2$ \`a sa
charge. Dans ce cas, le test du $\chi^2$ portera sur les petites valeurs de
$S$ et non sur les grandes, et servira \`a conclure qu'une accumulation
de r\'esultats trop exacts est anormale.
\medskip
On pourra donc retenir en conclusion que le test du $\chi^2$ peut servir
\`a estimer si des fluctuations sont trop grandes, mais aussi trop
petites. Bien entendu l'interpr\'etation du test ne sera pas du tout la 
m\^eme dans les deux cas. En revanche le principe du test est toujours 
le m\^eme;  on teste une hypoth\`ese. Le raisonnement est le suivant: si 
l'hypoth\`ese est vraie, le r\'esultat observ\'e est improbable, donc on
conclut que, l'\'ev\'enement improbable {\it ayant eu lieu}, c'est
l'hypoth\`ese qui est probablement fausse; ou inversement: si le test
conclut que le r\'esultat observ\'e \'etait probable, il n'y a aucune
incompatibilit\'e entre l'hypoth\`ese et l'observation. Un tel test permet
une quantification pr\'ecise et objective des seuils de certitude, mais 
cela ne prot\`ege \'evidemment pas contre la l\'eg\`eret\'e dans
l'interpr\'etation.

\vskip5mm plus3mm minus3mm

{\bf XI. 3. Comment tester une hypoth\`ese.}
\medskip
Toute la discussion de la section pr\'ec\'edente concerne un exemple
(le d\'e).
 Mais la m\'ethode du test peut s'appliquer de mani\`erebien plus g\'en\'erale.
\medskip
Il se trouve que l'hypoth\`ese ``le d\'e est correctement \'equilibr\'e'' 
se traduit math\'ematiquement par ``les probabilit\'es des six faces 
sont toutes \'egales \`a ${1 \over 6}$''.  Autrement dit, l'hypoth\`ese se 
traduit math\'ematiquement par la donn\'ee d'une loi de probabilit\'e. 
Imaginons qu'apr\`es un test n\'egatif, c'est-\`a-dire un test ayant 
donn\'e un r\'esultat tr\`es peu probable dans l'hypoth\`ese o\`u le 
d\'e est correctement \'equilibr\'e,  nous souhaitions trouver la loi de
probabilit\'e r\'eelle (et non uniforme) du d\'e. Le test ayant consist\'e
\`a lancer $6\, 000$ fois le d\'e, on peut prendre les fr\'equences
relatives de chaque face comme un premi\`ere indication. Mettons pour
fixer les id\'ees que les r\'esultats obtenus, au lieu d'\^etre ceux de
$(XI.1.)$ soient les suivants
$$\openup 3 \jot \eqalignno{ 
\noalign{\vskip1pt plus3pt minus1pt}\cr
&\as \hbox to 20mm {\hfill
980 fois} \hskip25mm \quatre \hbox to 20mm {\hfill 993 fois} \cr
\noalign{\vskip1pt plus3pt minus1pt}\cr
&\deux \hbox to 20mm {\hfill 812 fois} \hskip25mm 
\cinq \hbox to 20mm {\hfill 1133 fois} &(XI.9.)\cr
\noalign{\vskip1pt plus3pt minus1pt}\cr
&\trois \hbox to 20mm {\hfill 1019 fois} \hskip25mm
\six \hbox to 20mm {\hfill 1063 fois} \cr 
\noalign{\vskip1pt plus3pt minus1pt}\cr
}$$
Il saute aux yeux que les r\'esultats anormaux sont le deux qui ne sort 
que $812$ fois et le cinq qui sort $1133$ fois. \`A l'aide de la loi
gaussienne (cette fois il est pr\'ef\'erable de ne pas utiliser le test du
$\chi^2$, puisqu'on veut analyser les r\'esultats pour la face deux et la
face cinq, et non globalement) on peut voir que la probabilit\'e d'avoir
plus de $1100$ fois le cinq sur $6000$ lancers (dans l'hypoth\`ese d'un 
d\'e correctement \'equilibr\'e) est $0.0008$; la probabilit\'e d'avoir 
moins de $850$ fois le deux est inf\'erieure \`a $10^{-4}$. Comme le
premier coup d'oeil le laissait pressentir, le test est donc n\'egatif, 
et on est conduit \`a conclure que la probabilit\'e du cinq est en
r\'ealit\'e sup\'erieure \`a ${1 \over 6}$ tandis que la probabilit\'e 
du deux est inf\'erieure \`a ${1 \over 6}$. On peut \'evaluer ces
probabilit\'es par les fr\'equences observ\'ees dans $(XI.9.)$ qui
seraient donc
$${980 \over 6000}\; , \quad {812 \over 6000}\; , \quad {1019
\over 6000}\; , \quad {993 \over 6000}\; , \quad {1133 \over 6000}
\; , \quad
{1063 \over 6000}\; ,$$mais il serait absurde de les \'evaluer avec une telle
pr\'ecision
puisque nous savons que les fluctuations normales autour deces
valeurs sont de l'ordre de $\sqrt{1000}/6000 \simeq 0.005$.  Donc le
r\'esultat observ\'e pour le un,  \`a savoir $980 / 6000$, ne doit pas
\^etre tenu pour significativement diff\'erent de $1 / 6$. On peut alors
faire la {\it nouvelle} hypoth\`ese que les probabilit\'es sont ${1
\over 6}\, ,\; 0.13\, ,\; {1 \over 6}\, ,\; {1 \over 6}\, ,\; 0.20\, ,\;
0.17$.  Telles qu'elles sont choisies, ces valeurs donneront
\'evidemment un test positif (les fluctuations autour de ces valeurs
sont petites puisqu'elles ont \'et\'e choisies pour cela). Toutefois il
serait prudent de tester la nouvelle hypoth\`ese ind\'ependamment en
lan\c{c}ant \`a nouveau le d\'e $6000$ fois (cela co\^ute \'evidemment
du temps et de l'argent, mais comme toutes les bonnes choses, la
certitude se paie).
\medskip
Si avec la nouvelle hypoth\`ese on veut effectuer le second test pour
la somme des carr\'es (c'est-\`a-dire globalement et non pour une ou
plusieurs faces particuli\`eres), il faudra tenir compte d'un fait 
nouveau par rapport \`a la loi uniforme: les fluctuations $X_j$ n'ont pas
toutes le m\^eme \'ecart-type; en effet si les probabilit\'es ne sont 
pas toutes les six \'egales \`a ${1 \over 6}$, mais ont des valeurs
diff\'erentes $p_j$, les \'ecarts-types correspondants vaudront
$\sqrt{6000\, p_j}$ au lieu de $\sqrt{1000}$.  Dans ce cas il serait
erron\'e de calculer la somme des carr\'es des fluctuations, de la 
diviser par $1000$, puis de la situer par rapport \`a la r\'epartition
$\chi_5^2$, car la densit\'e $\chi_5^2$ est celle de la somme de six
fluctuations (de somme nulle) ayant {\it chacune} un \'ecart-type \'egal
\`a 1.  La loi de la variable al\'eatoire ``somme des carr\'es'' n'a pas
pour densit\'e $\chi_5^2$ et il convient de la remplacer par la variable
al\'eatoire 
$$S = \Bigl( {X_1 \over \sigma_1}\Bigr)^2 + \Bigl( {X_2
\over \sigma_2}\Bigr)^2 + \Bigl( {X_3 \over \sigma_3}\Bigr)^2 +
\Bigl( {X_4 \over \sigma_4}\Bigr)^2 + \Bigl( {X_5 \over
\sigma_5}\Bigr)^2 + \Bigl( {X_6 \over \sigma_6}\Bigr)^2$$
qui est la somme des carr\'es des variables al\'eatoires $X_j' = X_1 /
\sigma_1$, qui, elles, ont bien un \'ecart-type \'egal \`a 1. On pourrait
objecter ici que les variables al\'eatoires $X_j'$ ont certes chacune un
\'ecart-type \'egal \`a 1, mais que leur somme n'est pas nulle, car c'est 
la somme des $X_j$ qui est nulle. Or d'apr\`es le th\'eor\`eme \'enonc\'e
\`a la section {\bf 2}, si
six variables al\'eatoires ont chacune un\'ecart-type \'egal \`a 1 et une
somme nulle, alors la somme de leurscarr\'es a une loi de densit\'e
$\chi_5^2$. Cela r\'esultait de lapropri\'et\'e typique que la densit\'e gaussienne \`a six dimensions, 
$\exp\{ -{1 \over 2}( x_1^2 + x_2^2 +
x_3^2 + x_4^2 + x_5^2 + x_6^2 )\}$, est invariante par rotation et que sa trace sur un sous-espace de
dimension cinq est une densit\'e
gaussienne \`a cinq dimensions. Mais cela aurait \'et\'e vrai pour {\it
n'importe quel} sous-espacede dimension cinq, \`a cause justement de
l'invariance par rotation. Ce serait donc tout aussi vrai pour l'hyperplan
d'\'equation$\sigma_1 X'_1 + \sigma_2 X'_2 + \sigma_3 X'_3 +
\sigma_4 X'_4 + \sigma_5 X'_5 + \sigma_6 X'_6 = 0$ que pour
l'hyperplan d'\'equation $X'_1 + X'_2 + X'_3 + X'_4 + X'_5 + X'_6 = 0$.
Donc malgr\'e la modification, il demeure que la loi de la variable
al\'eatoire $S = \sum {X'}_j^2$ a bien la densit\'e $\chi_5^2$. 
\medskip
Si on refait alors le test avec la nouvelle hypoth\`ese que les six faces
ne sont pas \'equiprobables, mais ont les probabilit\'es $p_1 = {1
\over 6}\, ,\; p_2 = 0.13\, ,\; p_3 = {1 \over 6}\, ,\; p_4 = {1 \over
6}\, ,\; p_5 = 0.20\, ,\; p_6 = 0.17$, et qu'on trouve sur $N$
lancers $N_1$ fois la face 1, $N_2$ fois la face 2, etc. on calculera 
la grandeur 
$$\eqalignno{
S \;= \; {(N_1 - p_1 N)^2 \over p_1 N} \; &+ \; {(N_2 - p_2 N)^2
\over p_2 N} \; + \; {(N_3 - p_3 N)^2 \over p_3 N} \; + \; 
{(N_4 - p_4 N)^2 \over p_4
N} \; + \; \cr
&\hskip13mm + {(N_5 - p_5 N)^2 \over p_5 N} \; + \; {(N_6 - p_6 N)^2 
\over p_6 N} &(XI.10)\cr } $$
C'est donc sur ce mod\`ele que sont pratiqu\'es les tests d'hypoth\`ese 
dits ``du $\chi^2$''. L'hypoth\`ese est une loi de probabilit\'e qu'on
postule a priori soit parce qu'on y est conduit par des consid\'erations
th\'eoriques (par exemple que les six faces d'un d\'e sont 
\'equiprobables), soit par une observation pr\'ealable, soit par un
cahier des charges (si on veut d\'etecter des d\'efauts sur une
cha{\^\i}ne de fabrication) ou pour toute autre raison. Ce que nous venons
de voir montre que l'hypoth\`ese faite ne comprend pas seulement la loi de
probabilit\'e, mais \'egalement les \'ecarts-types (ou les variances)
des fluctuations. En effet, le calcul de la grandeur $S$ dans $(XI.10)$
fait appel non seulement \`a la donn\'ee des probabilit\'es $p_j$, 
mais
{\it aussi} \`a la donn\'ee des \'ecarts-types $\sigma_j$ des
fluctuations. On n'est donc pas en mesure d'appliquer un test du
$\chi^2$ si on n'a pas une id\'ee pr\'econ\c{c}ue sur l'\'ecart-type de
chaque fluctuation. On peut tr\`es bien avoir postul\'e une loi de
probabilit\'e $p_j$ correcte, et cependant avoir un test n\'egatif parce
qu'on s'est tromp\'e sur les \'ecarts-types. Les \'ecarts-types retenus
dans $(XI.10)$ sont ceux de la loi multin\^omiale (``sextin\^omiale''
pour \^etre pr\'ecis), puisque c'est cette loi qui r\'egit les lancers de
d\'es. Donc on ne conna{\^\i}t pas a priori les probabilit\'es $p_j$ car
les d\'es ont un d\'efaut non contr\^ol\'e d'\'equilibrage, mais on est
s\^ur d\`es le d\'epart que les r\'esultats d'un grand nombre de lancers
sont soumis \`a la loi multin\^omiale, et les \'ecarts-types sont alors
une fonction connue des $p_j$. Il suffit de faire une hypoth\`ese sur les
$p_j$ et de calculer ensuite $S$ selon $(XI.10)$. 
\medskip
Toutes les explications pr\'ec\'edentes ont \'et\'e d\'evelopp\'ees \`a 
propos d'une exp\'erience de d\'e qu'on lance six mille fois. Il est bien
entendu que cette exp\'erience de pens\'ee n'\'etait ici qu'un support
didactique, permettant de mieux comprendre la {\it signification} du test.
\medskip
En pratique le test du $\chi^2$ est souvent utilis\'e dans des situations 
o\`u la nature gaussienne des fluctuations n'est pas assur\'ee \`a l'avance
avec autant de rigueur. Toutefois l'exp\'erience de pens\'ee du d\'e reste
un {\it mod\`ele} ou un {\it id\'eal}. Il repr\'esente l'{\it esprit des
sciences exactes} dans des contextes exp\'erimentaux o\`u cet esprit
n'est qu'un id\'eal inaccessible.
\medskip
Voici un type d'exp\'erience tr\`es simple sur des souris qui est typique
des applications possibles de tests statistiques en biologie ou 
psychologie exp\'erimentale. Quoique de nos jours la sophistication des
m\'ethodes statistiques d\'epasse presque toujours celle de cet exemple
scolaire,  il n'y a rien de nouveau quant au principe. Nous discuterons
cette exp\'erience afin d'en montrer les limites.
\smallskip 
\midinsert
\epsfxsize = \hsize
\line{\epsfbox{../images/fig48.eps} }
\vskip6.33mm
\centerline{\eightpoint figure 48}
\vskip7mm
\endinsert
\smallskip
La figure 48 montre un dispositif destin\'e \`a \'etudier l'apprentissage 
(ou toute autre facult\'e, sens de l'orientation inn\'e, etc.) chez les 
souris. Sur une ar\`ene centrale d\'ebouchent $n$ couloirs (ici $n=8$)
absolument identiques, num\'erot\'es de $1$ \`a $8$ sur la figure. On 
veut tester si les souris sont capables de reconna{\^\i}tre le num\'ero en
leur offrant par exemple, pendant une p\'eriode d'apprentissage, dix fois 
de suite de la nourriture dans le couloir $N^\circ 7$. Apr\`es cette
p\'eriode d'apprentissage, on veut savoir si les souris se dirigeront
pr\'ef\'erentiellement du premier coup dans le couloir $N^\circ 7$, ou si 
au contraire, comme la toute premi\`ere fois, elles choisiront au hasard
(``au hasard'' voulant dire que les $n$ couloirs sont \'equiprobables).
On peut aussi tester leur sens de l'orientation en pla\c{c}ant la nourriture
dans un couloir dont le num\'ero change al\'eatoirement, mais qui reste
toujours celui qui va vers le nord. Avec ce proc\'ed\'e, on peut 
tester toutes sortes de facult\'es sensorielles.
\medskip
Apr\`es la p\'eriode d'apprentissage, on place successivement disons 
deux cents souris dans l'ar\`ene et on r\'ep\`ete exactement les 
conditions qui ont r\'egn\'e durant l'apprentissage; par exemple, si
l'arriv\'ee de nourriture \'etait annonc\'ee par un son, on produira ce
m\^eme son. Puis on note pour chaque souris le num\'ero du couloir qu'elle
choisit {\it en premier}. On obtient alors une distribution statistique: 
pour chacun des huit couloirs, le nombre de souris l'ayant choisi. 
\medskip
Si toutes les deux cents souris (ou presque, par exemple cent 
quatre-vingt-dix-sept) choisissent le bon tunnel, point n'est besoin de
test et la conclusion est claire. Mais en g\'en\'eral, dans ce genre
d'exp\'eriences, les r\'esultats sont loin d'\^etre aussi tranch\'es. Il 
faudra alors savoir si une l\'eg\`ere pr\'ef\'erence pour le bon couloir 
peut \^etre l'effet du hasard (donc une fluctuation normale), ou bien si,
sous l'hypoth\`ese de l'\'equiprobabilit\'e des huit choix, il est peu
vraisemblable que la pr\'ef\'erence soit due au hasard.
\medskip
Il y a une analogie avec le probl\`eme longuement discut\'e ci-dessus du 
d\'e. Toutefois des diff\'erences sautent aux yeux: avec le d\'e on savait
que les fluctuations seraient gaussiennes et on en connaissait les
\'ecarts-types, car on avait la loi bin\^omiale qui r\'esultait des
sym\'etries du d\'e.  On ne peut avoir de telles certitudes avec des
organismes vivants. Il \'etait raisonnable d'admettre que les lancers
successifs du d\'e \'etaient stochastiquement ind\'ependants: le d\'e 
n'est pas influenc\'e par ce qui s'est pass\'e avant. Mais qui nous prouve
que le couloir choisi par la premi\`ere souris ne garde pas une trace (une
odeur par exemple) qui poussera les souris suivantes \`a emprunter le
m\^eme couloir? Si on {\it veut} tenir compte d'un tel effet, on {\it peut}
am\'eliorer le protocole de l'exp\'erience pour \'eliminer ce facteur: par
exemple permuter al\'eatoirement les num\'eros des couloirs, de sorte
que la nourriture ne soit pas toujours dans le m\^eme couloir, mais
toujours au $N^\circ 7$. Cependant on n'est jamais \`a l'abri de facteurs
insoup\c{c}onn\'es; on aura pens\'e \`a l'odorat, mais on ne sait pas tout
sur la vie des souris.
\medskip
Une autre diff\'erence est aussi le nombre: en r\'ep\'etant six mille fois,
les r\'esultats sont bien plus concluants qu'avec deux cents. Mais on peut
aussi y rem\'edier en prenant six mille souris. C'est le sens des
exp\'eriences m\'edico-pharmacologiques \`a tr\`es grande \'echelle, 
comme celle qui est cit\'ee du journal {\sl The Lancet} au chapitre {\bf X}.
\medskip
Une diff\'erence plus subtile est la suivante: supposons que l'on ait
utilis\'e six mille souris, ou m\^eme un million si vous voulez; le test 
du $\chi^2$ suppose que, comme pour le d\'e, les \'ecarts-types des 
huit fluctuations soient ceux de la loi multin\^omiale.  Cela est
n\'ecessairement le cas si l'exp\'erience est {\it parfaitement 
reproductible}: quelles que soient les probabilit\'es a priori $p_j$ pour 
que chaque souris prenne le couloir $N^\circ j$, la r\'ep\'etition de 
l'exp\'erience conduira forc\'ement \`a la loi multin\^omiale et 
l'expression $XI.10$ (pour huit possibilit\'es au lieu de six) suivra la 
loi $\chi^2$ \`a sept degr\'es de libert\'e. Mais l'exp\'erience est-elle
exactement reproductible? On retrouve ici la discussion du chapitre {\bf
X}. Des \^etres vivants aussi \'evolu\'es que les souris ne pourraient-ils 
pas \^etre influenc\'es par des d\'etails perceptifs insignifiants pour
l'exp\'erimentateur, qui conduiraient \`a augmenter l'\'ecart-type du
quatri\`eme couloir et \`a reserrer celui du sixi\`eme, par exemple?
Imaginons que pendant l'apprentissage le couloir $N^\circ 7$ o\`u
\'etait dispos\'ee la nour\-ri\-ture \'etait trop souvent au nord,
l'exp\'erimentateur {\it n'ayant pas song\'e} \`a ce d\'etail.
L'exp\'erimentateur croit que seul le num\'ero distingue les couloirs. 
Mais essayez de vous mettre \`a la place d'une de ces souris (elles sont 
des mammif\`eres comme vous et moi, et ont une certaine forme de
pens\'ee): elles ont faim, et pour manger doivent deviner le bon couloir.
Pourquoi penseraient-elles d'abord au num\'ero, qui pour elles est bien 
plus abstrait et \'eloign\'e de leur mode de perception que par exemple
l'orientation (peu importe ici de savoir comment elles arrivent \`a
s'orienter). On observerait alors des effets diff\'erents selon que les
deux crit\`eres, le num\'ero {\it et} l'orientation,  se conjuguent ou
s'opposent.  Si les deux crit\`eres se conjuguent, les deux cat\'egories 
de souris, d'une part les intellectuelles qui pensent d'abord au num\'ero,
et d'autre part les instinctives qui pensent d'abord \`a l'orientation,
vont agir de m\^eme et ``pr\'ef\'erer'' le couloir $N^\circ 7$. Si au
contraire les deux crit\`eres s'opposent, les souris ayant d'abord 
pens\'e \`a l'orientation choisiront un autre couloir que le $N^\circ 7$
et le r\'esultat statistique sera une augmentation de l'\'ecart-type. 
\medskip 
Si l'exp\'erimentateur pense \`a randomiser soigneusement
l'orientation,  il \'evitera l'artefact correspondant. Mais nos
connaissances sur les souris sont beaucoup moins compl\`etes que nos
connaissances sur les d\'es. Combien restera-t-il de facteurs ignor\'es?
Nul ne le sait. Donc l'hypoth\`ese que les huit \'ecarts-types sont ceux
de la loi multin\^omiale est peut-\^etre syst\'ematiquement fausse. 
\medskip
Les mammif\`eres sont plus proches de l'homme que du d\'e. On peut 
mieux comprendre les souris en mettant un enfant humain \`a leur place
qu'en mettant un d\'e \`a leur place.  C'est pourquoi on pourra m\'editer
l'exemple suivant \ftn 1{Exemple emprunt\'e \`a Stella Baruk, mais je
ne sais plus dans quel livre.}. Des \'el\`eves du primaire sont 
confront\'es au probl\`eme math\'ematique: ``calculer $(10 \times 7)$, 
puis $(3 \times 7)$; en d\'eduire $13 \times 7$.'' Un des \'el\`eves
r\'epond respectivement ``$70$, $21$, et $0$''. Le ma{\^\i}tre {\it n'avait
pas pens\'e} que l'expression ``en d\'eduire'' avait deux sens diff\'erents
en fran\c{c}ais et conclut que l'\'el\`eve n'a pas compris. Si l'\'enonc\'e
pr\'ec\'edent figure dans un questionnaire dont les r\'esultats seront
trait\'es statistiquement, le mot ambigu aura pour effet d'augmenter
l'\'ecart-type des notes (et de baisser leur moyenne). On peut r\'ealiser 
des exp\'eriences consistant \`a donner des \'enonc\'es diff\'erents du
m\^eme probl\`eme math\'ematique; s'ils sont donn\'es successivement 
aux m\^emes \'el\`eves,  ceux-ci ayant en m\'emoire les pr\'ec\'edents
\'enonc\'es ne r\'epondront pas ind\'ependamment; si on change la
population, on introduit une nouvelle incertitude (les nouveaux \'el\`eves,
venant peut-\^etre d'un milieu social diff\'erent, n'ayant pas forc\'ement 
la m\^eme perception de la langue que les pr\'ec\'edents). 
\medskip
Lorsqu'on publie des r\'esultats exp\'erimentaux sur les animaux de
laboratoire, les revues sp\'ecialis\'ees exigent des tests quantitatifs. 
La raison en est que m\^eme avec les limites indiqu\'ees ci-dessus, la
pr\'esence de statistiques quantifi\'ees dans une publication 
scientifique permet une comparaison plus objective: m\^eme si le test 
du $\chi^2$ est fallacieux pour une des raisons indiqu\'ees, il permet 
dans une certaine mesure de comparer des r\'esultats obtenus par une
\'equipe japonaise \`a ceux obtenus par une \'equipe danoise. 
\medskip
Une telle comparaison des r\'esultats est relativement sens\'ee, plus
parce que les erreurs commises dans la manipulation du test sont 
partout les m\^emes (par exemple, que les ``facteurs ignor\'es'' sont les
m\^emes pour l'\'equipe japonaise que pour l'\'equipe danoise), que 
parce que le test est bien appliqu\'e dans son principe.

\bigskip

{\bf XI.4. Le test de Student.}
\medskip
Le test du $\chi^2$, comme nous avons vu, convient pour tester
l'hypoth\`ese que plusieurs possibilit\'es sont \'equiprobables, ou plus
g\'en\'eralement que ces possibilit\'es ob\'eissent \`a une loi de 
probabilit\'e suppos\'ee (et qu'il s'agit justement de tester), mais
\`a condition de conna{\^\i}tre les \'ecarts-types.
\medskip
Le test dit {\it de Student} (invent\'e en fait par le
statisticien anglais G{\eightrm OSSET} $\sim$ {\oldstyle 1900}) que
nous allons \'etudier maintenant, est g\'en\'eralement utilis\'e pour 
tester des moyennes empiriques, sans conna{\^\i}tre l'\'ecart-type. 
\medskip
Afin de bien comprendre l'{\it id\'ee} du test de Student,
imaginons une exp\'erience reproductible mod\'elis\'ee par une
variable al\'eatoire $X$ dont la loi est approximativement gaussienne.
Pour fixer les id\'ees on peut imaginer que $X$ repr\'esente les
fluctuations de la mesure d'une grandeur. Si on mesure un tr\`es grand
nombre de fois une grandeur (suppos\'ee \'evidemment reproductible),
on retient g\'en\'eralement la moyenne empirique des r\'esultats comme 
\'etant la vraie valeur; mais supposons que, par exemple \`a la suite 
d'une pr\'ediction th\'eorique, on ait d\'ej\`a une bonne raison de penser
que la valeur de la grandeur soit $m$. La moyenne empirique $M$ obtenue
par les mesures ne sera pas forc\'ement \'egale \`a $m$, mais comment 
savoir si la diff\'erence est significative et n'est pas une fluctuation 
due au hasard? Dans le test du $\chi^2$ on voulait tester si (un d\'e
\'etant lanc\'e six mille fois, ou des souris \'etant l\^ach\'ees huit
cents fois) la diff\'erence entre les probabilit\'es mesur\'ees et les
probabilit\'es suppos\'ees est due au hasard ou non. Maintenant nous
voudrions tester si la diff\'erence entre la moyenne mesur\'ee et la
moyenne suppos\'ee est due au hasard ou non. On ne peut pas utiliser le
test du $\chi^2$ pour cela. Ou plut\^ot, on pourrait l'utiliser si on
connaissait l'\'ecart-type des erreurs de mesure (c'est-\`a-dire
l'\'ecart-type de la variable al\'eatoire $X$): il suffirait alors de
calculer la somme normalis\'ee des carr\'es des \'ecarts par rapport 
\`a $m$. En effet, une valeur fausse de $m$ rendrait cette somme trop
grande et le test du $\chi^2$ serait n\'egatif. Mais pour calculer la
somme {\it normalis\'ee} des carr\'es des \'ecarts, il faut diviser par
l'\'ecart-type; si celui-ci n'est pas connu, on ne peut pas proc\'eder
ainsi. 
\medskip
Dans le cas d'une exp\'erience reproductible avec un nombre plut\^ot 
petit de possibilit\'es (six pour les d\'es, huit pour les souris, etc.) on
pouvait d\'eduire l'\'ecart-type de la loi multin\^omiale (cf. $IX.10$). 
Mais ici nous avons des erreurs de mesure dont nous ne connaissons pas
l'origine; il est raisonnable de les supposer gaussiennes, mais rien de
raisonnable ne permet de deviner leur \'ecart-type. Il faut donc un test
ind\'ependant de cet \'ecart-type. C'est le test de Student.
\medskip
Pour l'analyse math\'ematique du probl\`eme, appelons $m$ la moyenne 
de $X$ et $\sigma$ son \'ecart-type. Cela signifie qu'en r\'ep\'etant 
l'exp\'erience on obtient des r\'esultats $x_1,\; x_2, \ldots x_n$ qui 
ont chaque fois une probabilit\'e a priori ${\cal P}\, (X=x_j)$ de se
produire, ou, en termes de densit\'e, que la probabilit\'e pour que 
$x_j$ soit compris entre $x_j-\varepsilon$ et $x_j+\varepsilon$ est
\'egale \`a
$${\cal P}\, (x_j-\varepsilon < X < x_j+\varepsilon ) \;\; \simeq
\;\; {1\over \sdown{13} \sigma\sqrt{2\pi }}\int_{x_j-\varepsilon}^{x_j+\varepsilon}
\e^{-{(x-m)^2\strup{2}\over\sdown{6.6}2\sigma^2 }}\; dx$$
La r\'ep\'etition \`a l'identique de l'exp\'erience implique que les 
r\'esultats successifs $x_j$ sont stochastiquement ind\'ependants et 
tous distribu\'es selon la loi de $X$; autrement dit, les variables
al\'eatoires $(x_j - m)/ \sigma$ sont ind\'ependantes et de densit\'e
$\e^{-x^2/2}$. Par cons\'equent la somme de leurs carr\'es, soit
$$S_{0} = \sum_{j=1}^{j=n} {(x_j - m)^2\over\sigma^2}$$
ob\'eit \`a la loi du $\chi^2$ \`a $n$ degr\'es de libert\'es.
Mais ici nous ne connaissons pas forc\'ement $\sigma$,  qui est
l'\'ecart-type {\it th\'eorique}. 
\medskip
Remarquons \`a ce propos que, si pour le test du $\chi^2$ la 
connaissance a priori de l'\'ecart-type est n\'ecessaire, celle de la
moyenne th\'eorique en revanche ne l'est pas. Lorsque la moyenne 
th\'eorique $m$ est inconnue, on peut la remplacer par la moyenne {\it
empirique} $M = (x_1 + x_2 + \cdots +x_n)/n$. Au lieu de calculer la
somme $S_0$, on calculera 
$$S_{1} = \sum_{j=1}^{j=n} {(x_j - M)^2\over\sigma^2}$$ 
La diff\'erence essentielle avec $S_0$ est que les variables al\'eatoires
$x_j - M$ ne sont plus stochastiquement ind\'ependantes,  puisque $M$
d\'epend de chacune d'elles. Toutefois il est clair que la somme des 
$x_j - M$ est nulle; on est donc dans la situation du th\'eor\`eme de la
section pr\'ec\'edente {\bf IX.2}: les variables al\'eatoires $x_j$ sont
ind\'ependantes, mais la somme des $x_j - M$ est nulle. D'apr\`es le
th\'eor\`eme, la somme $S_1$ ob\'eit \`a la loi $\chi^2$, mais avec
$n-1$ degr\'es de libert\'e. La contrainte que la somme est nulle fait
baisser d'une unit\'e le nombre de degr\'es de libert\'e. Que cet
abaissement ne modifie pas le caract\`ere gaussien de la loi est une
propri\'et\'e sp\'eciale de la loi gaussienne,  voir {\bf XI.2}. Toujours
est-il que la somme {\it normalis\'ee} $S_0$ ou $S_1$ des carr\'es des
\'ecarts par rapport \`a $m$ ou $M$ ne peut \^etre calcul\'ee \`a partir
des mesures $x_1,\; x_2, \ldots x_n$ que si on conna{\^\i}t a priori
l'\'ecart-type $\sigma$. 
\vskip6pt plus5pt minus4pt 
Pour faire dispara{\^\i}tre le param\`etre $\sigma$, l'id\'ee du
test de Student consiste alors \`a consid\'erer la variance empirique
\ftn 1{on notera que dans l'expression de $V_{\rm emp}$ la somme des 
carr\'es des \'ecarts est divis\'ee par $n-1$ et non par $n$. La raison de
ce choix est expliqu\'ee au chapitre suivant ({\bf XII.2}).} 
$$V_{\rm emp} \;\; = \;\; {1\over n-1}\sum_{j=1}^{j=n} (x_j - M)^2$$
et d'\'etudier la loi de probabilit\'e (ou plut\^ot la densit\'e) du
quotient
$$S \;\; = \;\; {1\over\sdown{15}\sqrt{V_{\rm emp}}}
\sum_{j=1}^{j=n} (x_j - m)$$ 
Pour calculer la loi de ce quotient nous commen\c{c}ons par chercher
celle de la grandeur $V_{\rm emp} / \sigma^2$. Chacune des valeurs 
$x_j$ peut fluctuer autour de sa moyenne $m$ selon une loi gaussienne
d'\'ecart-type inconnu $\sigma$. Donc la grandeur $V_{\rm emp} /
\sigma^2$, qui est la somme {\it normalis\'ee} des carr\'es des 
\'ecarts par rapport \`a $M$, ob\'eit \`a la loi du $\chi^2$ \`a $n-1$
degr\'es de libert\'e, qui ne fait pas intervenir $\sigma$. D'autre part 
la grandeur ${1\over\textdown{\sigma }}\sum_j (x_j-m)$, qui est 
la somme normalis\'ee des \'ecarts par rapport \`a $m$, ob\'eit \`a la 
loi gaussienne centr\'ee et normalis\'ee \`a $n$ degr\'es de libert\'e
${1\over 2\pi}\exp\{ -(x_1^2 + x_2^2 + \cdots + x_n^2) /2\}$ 
qui ne fait pas non plus intervenir $\sigma$. Or $S$ est le quotient des
deux, dans lequel les $\sigma$ inconnus se simplifient: 
$$\eqalign{
S\;\;  &= \;\; {1\over\sdown{15}\sqrt{V_{\rm emp}}}
\sum_{j=1}^{j=n} (x_j - m) \cr
&= \;\; {1\over\sdown{15}\sqrt{{V_{\rm emp}/\sigma^2}} } 
\;\; \cdot \;\; {{\displaystyle \sum_{j=1}^{j=n} (x_j - m)} \strup{18}
\over \sdown{15}\sigma }\cr }$$
On peut donc conclure que la loi de probabilit\'e de $S$ est celle d'un
quotient de la forme $Y/\sqrt{Z}$ o\`u $Y$ est une variable al\'eatoire 
qui suit la loi gaussienne centr\'ee et normalis\'ee ${1\over 2\pi}\,
\e^{-{x^2/2}}$ et $Z$ une autre variable al\'eatoire, ind\'ependante
de $Y$, qui suit la loi $\chi^2$ \`a $n-1$ degr\'es de libert\'es. Aucune 
de ces deux lois ne fait intervenir $\sigma$, donc la loi de $S$ est
indiff\'erente \`a la quantit\'e inconnue $\sigma$. C'est l\`a le principe
du test de Student. 
\medskip
Il ne reste plus qu'\`a calculer la densit\'e de Student elle-m\^eme.
Nous proc\'edons comme au chapitre {\bf IX} pour calculer la densit\'e 
de Cauchy ou au paragraphe {\bf XI.1} pour calculer celle du $\chi^2$. 
La densit\'e sera en effet donn\'ee par les probabilit\'es ${\cal P}\, (a 
< S < b)$ lorsqu'on les aura exprim\'ees sous la forme d'une int\'egrale
entre les bornes $a$ et $b$. Puisque $S = Y/\sqrt{\hbox{\eightpoint
$Z$}}$ l'\'ev\'enement $\{ a < S < b \}$ est identique 
\`a l'\'ev\'enement $\{ a\,\sqrt{\hbox{\eightpoint $Z$}} < 
Y < b\,\sqrt{\hbox{\eightpoint $Z$}} \}$, et on peut \'ecrire 
$$\eqalign{
{\cal P}\, (a < S < b)\quad &\simeq\quad \int\quad\int_{\hbox{
\lower8.5pt \hbox{ $\scriptstyle \kern-47pt a\sqrt{z} < y < b\sqrt{z}$}}}
{1\over\sdown{13} \sqrt{2\pi }}\;\e^{-{y^2\strup{2}\over 2}} 
\chi_{n-1}^2 (z)\; dy\, dz\cr
&=\quad {1\over\sdown{13} \sqrt{2\pi }}\; {1\over\sdown{13} 
2^{{r\over 2}}\Gamma \big( {r\strup{1.8}\over 2}\big) }\quad\int\quad
\int_{\hbox{ \lower8.5pt \hbox{ $\scriptstyle \kern-50pt a\sqrt{z} < y <
b\sqrt{z}$}}} \e^{-{y^2\strup{2} \over 2}} z^{{r\over 2} - 1}
\e^{-{z\over 2}} \; dy\, dz\cr }$$ 
Nous avons simplement report\'e ici l'expression $XI.5$ de la
densit\'e $\chi_r^2$ \`a $r$ degr\'es de libert\'e. Dans le pr\'esent 
contexte, $r = n-1$, mais nous oublions l'entier $n$ pour la dur\'ee du
calcul. 
\medskip 
\midinsert
\vbox to \blocksize{\eightpoint
\epsfxsize = \hsize
\line{\epsfbox{../images/fig49.eps} }
\vfill
\centerline{\vbox{\hsize=11cm figure 49 : \hskip10pt
Les graphiques ci-dessus sont ceux des densit\'es de Student \`a $r$
degr\'es de libert\'e pour $r = 1,\, 2,\, 3,\, 4,\, 5,\, 12,\, 20,\, 30,\, 
60,\, 100$. Les derniers ne se distinguent pratiquement plus de leur
limite gaussienne.} } }
\endinsert 
Effectuons maintenant le changement de variable $s=y/\sqrt{z}$, de
mani\`ere \`a passer des variables $y,z$ aux variables $s,z$. Le 
jacobien de la transformation est $\sqrt{z}$, de sorte que notre
int\'egrale devient 
$$\eqalign{
{\cal P}\, (a < S < b)\;\;  &\simeq \;\; {1\over\sdown{13}\sqrt{2\pi }}\; 
{1\over\sdown{13} 2^{{r\over 2}} \Gamma \big( {r\strup{1.5}\over 2}
\big) }\; \int\int_{\hbox{\lower6pt \hbox{ $\scriptstyle \kern-23pt a < 
s < b$}}}\;\e^{-{z s^2\strup{1.7} \over 2}} z^{r\over 2}\,  
\e^{-{z\over 2}} \; ds\, dz \;\; \cr
 &= \;\; {1\over\sdown{13} \sqrt{2\pi }}\; {1\over\sdown{13}2^{{r\over 2}} 
\Gamma\big( {r\strup{1.5}\over 2}\big)}\;\int_a^b\left\{\int_0^\infty 
\e^{-{s^2 + 1 \strup{2} \over 2}z}\,  z^{r-1\over 2}\, dz
\right\} ds \cr }$$ 
Ceci montre d\'ej\`a que la densit\'e cherch\'ee sera
l'expression entre accolades (multipli\'ee par le coefficient), 
qui est une int\'egrale eul\'erienne connue. En effet, puisque
$$\Gamma (x) \;\; = \;\; \int_0^\infty \e^{-z} z^{x-1} dz$$ 
on voit que notre densit\'e sera
$$f(s)\;\; = \;\; {1\over\sdown{13}\sqrt{2\pi }}\; 
{1\over\sdown{13}2^{{r\over
2}} \Gamma \big( {r\strup{1.5}\over 2}\big)}\;\int_0^\infty \e^{-{s^2 +
1\over 2}z^2} z^{r-1\over 2} dz \;\; = \;\; {1\over\sdown{13}\sqrt{2\pi }}\; 
{1\over \sdown{13} 2^{{r\over 2}}\Gamma \big( {r\strup{1.5}\over 2}\big) }\;
{\Gamma \big(
{r+1\strup{2}\over 2}\big)\strup{5.3}\over \sdown{15.5}\bigl( {s^2 + 1
\over 2}\bigr)^{{r+1\strup{2}\over 2}} }$$
En regroupant toutes les puissances de $2$ \'eparpill\'ees dans cette
derni\`ere expression, on peut simplifier: $\sqrt{2} \cdot 2^{{r\over 2}}
\cdot 2^{-{r+1\strup{2}\over 2}} = 1$; d'o\`u
$$f(s) \;\; = \;\; {1\over\sdown{13}\sqrt{\pi }}\; {1\over\sdown{13}\Gamma
\big( {r\strup{1.5}\over 2}\big) }\; {\Gamma\big( {r+1\strup{2}\over
2 }\big)\strup{5.3}\over\sdown{14}( s^2 + 1 )^{{r+1\strup{2}\over 2}} }
\eqno (XI.11)$$
Rappelons que dans ces calculs $r = n-1$. La variable $Z$ est ici la
somme des carr\'es des \'ecarts par rapport \`a la moyenne th\'eorique
$m$ suppos\'ee connue. Mais la variance empirique est \'egale \`a
$Z/\hbox{\eightpoint $(n\! -\! 1)$}$ (cf. chapitre suivant 
$XII.\, 6$).  Pour {\it pratiquer} le test de Student on peut
indiff\'eremment consid\'erer $Y/\sqrt{\hbox{\eightpoint
$Z$}}$ ou bien $Y/\sqrt{\hbox{\eightpoint $Z/(n\! -\! 1)$}} =
\hbox{\eightpoint $\sqrt{n\! -\! 1}$} \cdot
Y/\sqrt{\hbox{\eightpoint $Z$}}$.  Toutefois la convention
commun\'ement admise est de consid\'erer $Y/\sqrt{
\hbox{\eightpoint $Z/(n\! -\! 1)$}}$. La loi de Student
(ou plut\^ot la {\it densit\'e} de Student) est celle de $Y/\sqrt{ 
\hbox{\eightpoint $Z/(n\! -\! 1)$}}$ et non celle de $Y/\sqrt{
\hbox{\eightpoint $Z$}}$. Ce n'est qu'une convention, 
mais \'etant donn\'e que tous les logiciels de statistique la
respectent, on doit la respecter aussi si on veut employer ces derniers
tels qu'ils sont livr\'es au client.  La relation entre les deux densit\'es
est simple:  c'est la relation entre les densit\'es de deux variables
al\'eatoires proportionnelles:  si $f(t)$ est la densit\'e d'une variable
al\'eatoire $X$, et $g(t)$ celle de $\alpha X$, on a bien s\^ur
$${\cal P}\, (a < X < b) \;\; \simeq \;\; \int_a^b f(t)\, dt$$
d'o\`u 
$$\eqalignno{ 
\int_a^b g(t)\, dt \quad &\simeq\quad {\cal P}\, (a < \alpha X < b)\quad =
\quad {\cal P}\, \bigl( {\up{a}\over\down\alpha } < X < {\up{b}\over
\down\alpha }\bigr) \cr
&\simeq\quad\int_{a/\alpha }^{\, b/\alpha } f(t)\, dt \quad
= \quad\int_{a}^{b} f\big( {\up{s}\over\down\alpha }\big)\,
{\up{1}\over\down\alpha } ds \cr }$$
o\`u la derni\`ere \'egalit\'e s'obtient par le changement de variable
$s = \alpha t$. Pour faire appara{\^\i}tre la densit\'e d'une variable
al\'eatoire $W$, il suffit d'exprimer la probabilit\'e ${\cal P}\, (a < W < 
b)$ sous la forme d'une int\'egrale dont les bornes sont $a$ et $b$,
ce qui est le cas ci-dessus. Par cons\'equent 
$$g(t) \;\; = \;\; {\up{1}\over\down\alpha }\, f\bigl(
{\up{t}\over\down\alpha }
\bigr) \eqno (XI.12)$$ 
formule qui permet de calculer tr\`es facilement la densit\'e de $\alpha
X$ \`a partir de celle de $X$. Or la densit\'e de Student est celle de
$\hbox{\eightpoint $\sqrt{n\! -\! 1}$}\cdot Y/\sqrt{\hbox{
\eightpoint $ Z$}}$, il suffit donc de prendre $\alpha =
\hbox{\eightpoint $\sqrt{n\! - \! 1}$} = \sqrt{r}$ dans $XI.12$: 
la
densit\'e $f(t)$ de $Y/\sqrt{\hbox{\eightpoint $ Z$}}$ 
\'etant donn\'ee par $XI.11$, on obtient pour la densit\'e de Student 
$$g(t) \;\; = \;\; {1\over\sdown{12}\sqrt{\pi r}}\; {\Gamma \big(
{r+1\strup{1.5} \over 2 }\big)\strup{6.3}\over\sdown{12}\Gamma\big(
{r\strup{1.5} \over 2} \big) }\; {1\over\sdown{14}\big( {t^2\over
\textdown{r}}
+ 1 \big)^{{r+1 \strup{1.5} \over 2}} } \eqno (XI.13)$$ 
On l'appelle densit\'e de Student {\it \`a $r$ degr\'es de libert\'e}. 
Mais $r$ est \'egal \`a $n-1$, car, tout comme pour la densit\'e $\chi^2$, 
les $n$ variables al\'eatoires gaussiennes dont la somme des carr\'es est
$Z$ (les $n$ fluctuations autour de $m$) ont une somme nulle, ce qui
diminue d'une unit\'e leurs degr\'es de libert\'e. 
\medskip
Lorsque $r$ tend vers l'infini, cette densit\'e de Student tend vers une
limite. En effet, on sait que le quotient $\Gamma\big(x+{1\over 2}\big)
\big/\Gamma (x)$ est \'equivalent \`a $x^{1\over 2}$ lorsque $x$ tend 
vers l'infini (ici $x = r/2$) et que $(1 + t^2 / r)^{-{r+1 \over 2}}$ tend
vers $\e^{-{t^2\over 2}}$, donc la densit\'e limite sera
$(1/\hbox{\eightpoint $\sqrt{2\pi}$}\, )\, \e^{-{t^2\over 2}}$, 
c'est-\`a-dire la densit\'e gaussienne normalis\'ee. En pratique, pour les
exigences de pr\'ecision courantes,  cette limite est d\'ej\`a atteinte
pour $r=20$ (voir figure 49). 
\medskip
{\parfillskip=7cm
Une fois connue la densit\'e de $S = Y/\sqrt{Z}$ ou de $S' = Y/\sqrt{
\hbox{\eightpoint $Z/(n\! - \! 1)$}}$ on pourra effectuer un
test d'hypoth\`ese: on calcule \`a partir de la valeur a priori $m$ et 
des donn\'ees statistiques la valeur {\it observ\'ee} de la va\-ria\-ble
al\'eatoire $S$ ci-dessus. Si elle d\'epasse un seuil d\'etermin\'e, on
conclura que, {\it ind\'ependamment de la valeur inconnue de} $\sigma$,
les \'ecarts par rapport \`a la moyenne sont trop forts pour que
l'hypoth\`ese d'\'equiprobabilit\'e puisse \^etre retenue comme 
plausible. Pour la d\'etermination de ce seuil on proc\'edera comme 
pour le test du $\chi^2$: il sera tel que la probabilit\'e de le d\'epasser
soit inf\'erieure \`a une norme de certitude conventionnelle, par 
exemple $0.05$ ou $0.01$. \par }
\medskip
Ainsi le test du $\chi^2$ doit \^etre utilis\'e lorsqu'on conna{\^\i}t
a
priori les \'ecarts-types: on forme alors la somme normalis\'ee des carr\'es des \'ecarts par rapport \`a la moyenne empirique. En revanche 
le test de Student sera utilis\'e lorsqu'on conna{\^\i}t la moyenne
th\'eorique $m$ a priori, et qu'on sait \'egalement a priori (par des
consid\'erations de sym\'etrie et d'invariance) que les \'ecarts par 
rapport \`a $m$ ont tous le m\^eme \'ecart-type $\sigma$, mais que ce
dernier est inconnu. 
\medskip
Il est essentiel, lorsqu'on recourt \`a ces tests, de les employer {\it 
\`a bon escient}. Des tests pharmaceutiques effectu\'es en clinique (par
exemple l'observation sur un groupe de cinquante personnes recevant un
m\'edicament hypotenseur, comme nous en avons discut\'e au chapitre
{\bf X}) ne s'apparentent pas \`a l'exp\'erience avec les d\'es, m\^eme si
on n'en conteste pas la reproductibilit\'e. En effet les fluctuations de la
tension dans un groupe (ou au cours du temps chez une m\^eme personne)
n'ont aucune raison d'ob\'eir \`a la loi multin\^omiale; d'ailleurs, la
tension ne prend pas que six ou huit valeurs, c'est un param\`etre
pratiquement continu. Il est raisonnable de consid\'erer les fluctuations
de tension comme gaussiennes, mais on en ignore l'\'ecart-type.
\medskip 
On a \'etabli empiriquement avec une pr\'ecision assez pouss\'ee que
la r\'epartition des $Q.I.$ (quotient intellectuel) dans une tr\`es grande 
population est gaussienne, car on a pu disposer de donn\'ees assez 
nombreuses. Le $Q.I.$ semble en effet suffisamment passionner les 
d\'ecideurs pour que des exp\'eriences \`a grande \'echelle puissent 
trouver un financement. Mais l'\'ecart-type varie beaucoup selon
les populations; il n'y a pas pour le $Q.I.$ un \'ecart-type universel.
\medskip
Les m\'ecanismes mol\'eculaires de l'h\'er\'edit\'e ob\'eissent 
\`a une causalit\'e spatio-temporelle simple qui permettent
de cerner ``le niveau o\`u intervient le hasard pur''. Si par d\'eduction 
a priori on \'etablit une loi multin\^omiale pour la combinaison de
certains g\`enes, et qu'on souhaite tester sur un \'echantillon une
hypoth\`ese sur les effets observables de cette combinaison, le test 
du $\chi^2$ prend tout son sens, comme dans l'exemple du d\'e. En
revanche, le m\'etabolisme d'un organisme vivant complexe, ou la
psychologie animale, ne sont pas r\'eductibles \`a une causalit\'e
spatio-temporelle simple (plus exactement: m\^eme si cette r\'eduction
est possible, nous ne savons pas l'effectuer); les tests ne reposent 
alors que sur des extrapolations plausibles. Dans la plupart des cas qui
concernent le vivant, l'hypoth\`ese des fluctuations gaussiennes est
simplement plausible. 
\medskip
Le choix du test pertinent ne doit donc jamais \^etre laiss\'e au hasard.
Lorsque ce choix est d\^ument motiv\'e, il exprime toujours une 
hypoth\`ese sur le r\^ole jou\'e par le hasard dans le ph\'enom\`ene
test\'e. Ces hypoth\`eses devraient ---~selon le v\'eritable esprit
scientifique~--- \^etre formul\'ees explicitement et discut\'ees, mais
c'est rarement le cas dans la pratique. 
\medskip
Bien entendu il existe d'innombrables autres tests statistiques que les
deux que nous venons d'\'etudier, mais ce n'est pas l'ambition de cet
ouvrage de pr\'esenter un expos\'e exhaustif de tous les tests. Et bien
entendu, on peut rencontrer des situations o\`u {\it aucun} des
deux tests \'etudi\'es ne convient; c'est pourquoi les statisticiens en 
ont cr\'e\'e d'autres. Mais tout test, aussi sophistiqu\'e qu'il soit, 
repose sur des hypoth\`eses implicites.






\bye
