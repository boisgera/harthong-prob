\input fram2col.tex
\input epsf.tex

\auteurcourant={\sl J. Harthong: probabilit\'es et statistique}
\titrecourant={\sl Index}
\baselineskip=15pt

\def\term#1#2{\hbox to 78mm{#1\dotfill\hskip4pt #2}} 
\def\w{\hskip15pt} 
 
\pageno=470 
 
\titre={\tit Index alphab\'etique.} 
 
\hbox{A} 
\medskip 
\term{absolument monotone (fonction)}{204} 
\term{accroissements finis}{199} 
\term{agitation thermique}{51--52,58,418} 
\term{alcoolisme}{270} 
\hbox{al\'eatoire} 
\term{\w suite ---}{20,121,127} 
\term{\w texte ---}{120-127} 
\term{\w variable ---}{60,128-148} 
\term{algorithme}{21--22, 32, 127--128, 206} 
\term{all\`eles}{75--77, 107} 
\term{amplification (par le chaos)}{28,32,96} 
\term{analyseur (de Stern-Gerlach ou d'Aspect)}{} 
\term{}{379--380, 393} 
\term{analytique (fonction)}{178, 196--197, 202} 
\term{\w prolongement ---}{203} 
\term{anniversaires}{39} 
\term{Avogadro (nombre d')}{57--58,253,424} 
\vskip8.6mm \vfill  
\hbox{B} 
\medskip 
\term{Bayes (relation de)}{280} 
\term{belle de nuit (mirabilis)}{74} 
\term{biais}{332} 
\term{biblioth\`eque de Babel}{121} 
\hbox{Boltzmann}
\term{\w constante de ---}{59,354} 
\term{\w grandeur ${\cal H}$ de ---}{457,459} 
\term{\w r\'eponse \`a Poincar\'e}{457,459} 
\term{\w Sto{\ss}zahlansatz de ---}{451} 
\term{Bose-Einstein (statistique de)}{5,51,54,422} 
\term{brouillage (par le chaos)}{33, 52, 95--96, 423, 426} 
\term{brownien (mouvement)}{62}  
\term{bruit discret}{160,162,165,167,196} 
\vskip8.6mm \vfill  
\hbox{C} 
\medskip 
\term{cancer}{270} 
\term{cardinal d'un ensemble}{3,113--114,120} 
\term{causalit\'e }{84,90, 357,360} 
\term{\w effacement de la ---}{90-97} 
\term{changement d'\'echelle}{193,195,208} 
\term{chaos d\'eterministe}{19,24,28,91--92,423} 
\term{chromosomes}{75,107} 
\term{coefficients bin\^omiaux}{42--45,151} 
\term{coefficients de Taylor}{200--206} 
\term{coefficients multin\^omiaux}{47,49,56} 
\term{co{\"\i}ncidences fortuites}{115-120} 
\term{concat\'enation d'\'ev\'enements}{112} 
\term{configuration en phase}{424} 
\term{consanguinit\'e (facteur de)}{110} 
\term{constante de Boltzmann}{59,354} 
\term{constante de Planck}{59,356} 
\term{conventionnalisme (de Poincar\'e)}{16,18} 
\term{convexit\'e}{189,194,200} 
\term{convolution}{158,286,292} 
\term{cordes sur un cercle}{8--13,132--133,291} 
\term{corps noir}{54--59, 354} 
\smallskip 
\hbox{corr\'elation} 
\term{\w ---  et causalit\'e}{357--363} 
\term{\w coefficient de --- }{323} 
\term{\w --- de deux variables al\'eatoires}{321--325} 
\term{\w ---  E.P.R. en g\'en\'eral}{377-417} 
\term{\w ---  E.P.R. empirique}{385} 
\term{\w ---  E.P.R. th\'eorique selon J. Bell}{389} 
\term{\w ---  E.P.R. th\'eorique selon la M.Q.}{391--392} 
\term{\w ---  E.P.R.}{377-417} 
\term{\w --- statistique}{270, 332-340} 
\smallskip 
\term{covariance de deux variables al\'eatoires}{149} 
\term{crit\`ere algorithmique de Solomonov et Kolmogorov}{} 
\term{}{127--128} 
\term{cut-off}{208} 
\vskip8mm \vfill  
\hbox{D} 
\medskip 
\term{d\'e (lancer de)}{290,306--307} 
\term{d\'emon de Maxwell}{444,446,452--453} 
\term{densit\'e continue}{160, 196, 294--295} 
\term{densit\'e de Cauchy}{229} 
\term{densit\'e gaussienne}{230--235, 264, 283} 
\term{densit\'es $\chi^2$}{294--299} 
\term{densit\'es de Student}{317--319} 
\term{dernier retour \`a z\'ero}{71--73, 254--255, 256--258} 
\term{D\'esir\'e Andr\'e (principe de sym\'etrie)}{69--70} 
\term{d\'eterminisme}{22,135--136,153,176} 
\term{\w --- laplacien}{18} 
\term{\w --- m\'ecaniste absolu}{267} 
\term{d\'eveloppement en s\'erie de Laurent}{182} 
\term{d\'eveloppement limit\'e}{153,198,199} 
\term{Dieu}{384}
\term{distribution de boules}{4,35,38,47} 
\term{division des chromosomes}{75}
\vskip8mm \vfill  
\hbox{E} 
\medskip 
\term{\'ecart-type}{139,151,} 
\hbox to 78.6mm{\hfil 155,\hfil 278,\hfil 286,\hfil 290,\hfil  
303,\hfil 307--308,\hfil 311--312,\hfil 313--315\hfil} 
\term{\'echantillon}{273--279, 326--329} 
\term{\w covariance d'---}{327,333} 
\term{\w moyenne d'---}{327} 
\term{\w variable d'---}{327--333} 
\term{\w variance d'---}{333} 
\term{\'echantillonnage}{157,300} 
\term{\'echelle (changement d')}{193,195} 
\term{\'emission thermique}{54}
\term{empirique (grandeur)}{326} 
\term{\w covariance ---}{327} 
\term{\w moyenne ---}{314,327,329} 
\term{\w probabilit\'e ---}{106--107} 
\term{\w variance ---}{315,327} 
\term{\'energie d'un photon}{57} 
\term{entropie}{419,436,460} 
\term{\'epreuves}{1,3,131,400} 
\term{\'equilibre thermique}{56,58--59} 
\term{\'equiprobabilit\'e}{5,13--18,116,234,284} 
\term{espace des \'epreuves}{2,86,131,178,179,279,400} 
\term{esp\'erance d'une variable al\'eatoire}{137} 
\smallskip
\hbox{\'etat}
\term{\w --- asymptotique stable}{419} 
\term{\w --- d'\'equilibre}{419} 
\term{\w --- macroscopique}{419,433--434,442} 
\term{\w --- microscopique}{424,433--434,442} 
\term{\w --- microscopique exceptionnel}{442--443} 
\term{\'etats d'occupation (de bosons)}{6--7} 
\term{\'ev\'enements}{1,2,111,398--399}  
\term{\w r\'eunions d'---}{111-117}  
\term{\w intersections d'---}{83-85}  
\term{\w --- ind\'ependants}{85}  
\term{\w famille exhaustive d'---}{103,280}  
\hbox{exp\'erience} 
\term{\w --- d'Aspect}{378--379} 
\term{\w --- E.P.R.}{378} 
\term{\w --- de Stern-Gerlach}{378} 
\term{\w --- reproductible}{270,285,311,329--332,386,406} 
\term{extinction (d'une r\'eaction en cha{\^\i}ne)}{184,189} 
\vskip8mm \vfill  
\hbox{F} 
\medskip 
\term{facteur de consanguinit\'e}{110} 
\term{facteur Rh\'esus}{74} 
\term{Fermi-Dirac (statistique de)}{422} 
\term{filtre passe-bas}{158,196,208} 
\term{fission nucl\'eaire}{183--185} 
\term{fluctuations}{152--153,\hskip3pt 155--156,} 
\hbox to 78.2mm{175,\hfil 235,\hfil 268,\hfil 286--288,\hfil 289--290,\hfil
299--300,\hfil 304--313} 
\term{flux instantan\'e}{251--252} 
\term{flux moyen}{239,241,243,251--252} 
\term{fonction absolument monotone}{204} 
\term{fonction analytique}{178,196--197,202} 
\term{fonction caract\'eristique}{140,153--155,195--196} 
\term{\w --- asymptotique}{204,207--208} 
\term{\w --- conjointe}{149} 
\term{fonction convexe}{189} 
\term{fonction erf (${\cal N}$)}{230,283} 
\term{fonction g\'en\'eratrice}{140,177--208} 
\term{\w fonction g\'en\'eratrice conjointe}{149} 
\term{\w m\'ethode des fonctions g\'en\'eratrices}{178} 
\term{\w it\'eration des fonctions g\'en\'eratrices}{181,185} 
\term{fonction homographique}{186,200} 
\term{fonction random}{14,20,22,71,91,284,429,433} 
\smallskip
\hbox{formule} 
\term{\w --- du bin\^ome de Newton}{40} 
\term{\w --- des probabilit\'es conditionnelles}{6,103,398} 
\term{\w --- de Poincar\'e}{112--115} 
\term{\w --- de Stirling}{45,48,254} 
\smallskip 
\term{fourchette (sondage)}{134,282} 
\term{Fourier (transformation de)}{} 
\term{}{140,160,169,171--172,198,207--208,226} 
\term{fr\'equences propres}{57} 
\vskip7mm \vfill  
\hbox{G} 
\medskip 
\term{gaz de photons}{56} 
\term{g\`ene}{74} 
\term{g\'enotype}{76,81} 
\term{\w mod\`ele de ---}{79} 
\term{grandeur empirique}{326} 
\term{grandeur ${\cal H}$ de Boltzmann}{457,459} 
\term{groupe t\'emoin}{263--264,272,273} 
\term{gu\'erison spontan\'ee}{265} 
\vskip7.5mm \vfill  
\hbox{H} 
\medskip 
\term{Hardy-Weinberg (loi de)}{76--81, 108} 
\term{hasard ontologique}{23} 
\term{h\'eparine}{272} 
\term{h\'er\'edit\'e}{73-81, 107-110} 
\term{h\'et\'erozygote}{74,77--78,109,110} 
\term{homographique (fonction)}{186,200} 
\term{homozygote}{74,78,107,110} 
\term{hypersurface d'\'energie}{454}
\term{hypertension}{263,269--270,334,340,359}
\vskip7.5mm \vfill  
\hbox{I} 
\medskip 
\term{ind\'ependance causale}{88,90--92,97} 
\term{ind\'ependance stochastique}{82-97,176,272,284,393} 
\term{\w --- des variables al\'eatoires}{} 
\term{}{142--144,153,167,178--179,192} 
\term{indiscernabilit\'e (des quantons)}{7} 
\term{in\'egalit\'es de Bell}{390--391} 
\term{infarctus du myocarde}{272} 
\term{intervalle de confiance (d'un sondage)}{282} 
\term{invariance}{} 
\term{}{2,,25,52,72,86,116--117,129, 233--234,252--253} 
\term{irr\'eversibilit\'e}{418--464} 
\vskip7.5mm \vfill  
\hbox{L} 
\medskip 
\term{lancer de d\'e}{290,306--307} 
\term{lancer de pi\`ece (pile ou face)}{62,152,177,262,266}
\term{Levenberg--Marquardt (m\'ethode)}{342,344,363--369} 
\term{limites humaines}{424,444} 
\term{lissage (par convolution)}{208,292} 
\term{locus (sur chromosome)}{75--76,107} 
\term{logarithmique (papier)}{346,351--353,357} 
\term{\w singularit\'e ---}{197} 
\smallskip 
\hbox{loi} 
\term{lois asymptotiques}{226-261} 
\term{loi de Bernoulli}{220--221,258--260,276} 
\term{loi bin\^omiale}{133,263}
\hbox to 78.2mm{\hfil (voir aussi:  coefficients bin\^omiaux)}
\term{loi de Borel (s\'eries)}{244--246} 
\term{loi de Carnot}{420} 
\term{loi de Cauchy}{227--230, 260} 
\term{loi du dernier retour \`a z\'ero}{71--73,254--258} 
\term{loi gaussienne}{151,156,230-235,278,283} 
\term{loi g\'eom\'etrique}{185} 
\term{loi des grands nombres}{135--136,153,253,273} 
\term{loi de Hardy--Weinberg}{76--81, 108} 
\term{loi hyperg\'eom\'etrique}{259,276} 
\term{lois marginales}{148} 
\term{loi de Mendel}{73-76} 
\term{loi multin\^omiale}{302,313} 
\hbox to 78.2mm{\hfil (voir aussi:  coefficients multin\^omiaux)}
\term{loi normale}{151-176} 
\term{loi d'Ohm}{358--359} 
\term{loi de Planck}{54--59,129,135,353} 
\term{loi de Poisson}{217-219,235-240}
\term{loi des queues}{242} 
\term{loi de Rayleigh-Jeans}{54,354} 
\term{loi de Student}{260--261} 
\term{loi de Wien}{54,58,354} 
\term{longueur de corr\'elation}{158, 208} 
\term{Loschmidt (paradoxe de)}{421,422,437,440,450} 
\vskip7.6mm \vfill  
\hbox{M} 
\medskip 
\term{malentendu (entre Poincar\'e et la Physique)}{456} 
\term{marches al\'eatoires}{61-73,133,151} 
\term{marge d'erreur (d'un sondage)}{282} 
\term{marginales (lois)}{148} 
\term{mariages consanguins}{107-110} 
\term{Maxwell (d\'emon de)}{444,446,452--453} 
\term{Maxwell (postulat de)}{444, 455--456} 
\term{m\'edicaments (tests)}{263,268--270,272} 
\term{Mendel (loi de)}{73-76} 
\term{m\'etabolisme}{268--269,273,288}
\smallskip
\hbox{m\'ethode}
\term{\w --- des fonctions g\'en\'eratrices}{178}
\term{\w --- de Levenberg--Marquardt}{342,344,363--369} 
\term{\w --- de Newton}{208,364--365} 
\smallskip
\term{mirabilis}{74,76--77} 
\term{mod\`ele de d\'ependance}{344} 
\term{mod\`ele de r\'egression non lin\'eaire}{345,353} 
\term{mod\'elisation}{3--4,13,16--17} 
\term{modes d'occupation}{51--53,55--60} 
\term{moindres carr\'es}{325,341,343,345--347,363--369} 
\term{\w moindres carr\'es (droite des)}{325} 
\term{moments d'une variable al\'eatoire}{139} 
\term{moment cin\'etique (spin)}{379} 
\term{Montmort (probl\`eme de)}{115-120} 
\term{mouvement brownien}{62} 
\term{moyenne empirique}{314,325,327} 
\term{moyennisation}{208} 
\term{mucoviscidose}{74,76--78,110,271} 
\term{multifactorialit\'e}{270} 
\term{mutation (g\'en\'etique)}{75} 
\vskip9mm \vfill  
\hbox{N} 
\medskip 
\term{neutrons}{183--185,188} 
\term{nombre d'Avogadro}{57--58,253,424} 
\term{nombre de d\'eriv\'ees partielles}{51} 
\term{nombres au hasard}{20--22,126} 
\term{nombre de mon\^omes}{41} 
\term{nombre de mots de $n$ lettres}{35,41} 
\term{nombres d'occupation}{45,46,49,55,135} 
\term{nombre de partitions (coef. multin\^omiaux)}{45--49} 
\term{nombre de permutations (de $r$ objets)}{45} 
\term{nombre de subdivisions (coef. multin\^omiaux)}{49--50} 
\term{noyau d'$U^{235}$}{183--184} 
\vskip9mm \vfill  
\hbox{P} 
\medskip 
\term{panmixie (random mating)}{77,79} 
\term{papier logarithmique}{346,351--353,357} 
\term{paradoxe de Loschmidt}{421,422,437,440,450} 
\term{particules de Bose}{5,6,51,79,88,136} 
\term{partitions en groupes de taille donn\'ee}{45--49} 
\term{particule-m\`ere}{184} 
\term{particule-fille}{184} 
\term{{\eightrm PASCAL} (langage)}{207} 
\term{{\eightrm PASCAL} (triangle de)}{42} 
\term{passe-bas (filtre)}{158,196} 
\term{p\'eage d'autoroute}{238} 
\term{permutations}{40,46,119} 
\term{ph\'enom\`ene quantique elementaire}{397,404--405} 
\term{ph\'enotype}{73--74,81} 
\term{Physique statistique}{54--60,135,418--462} 
\term{pile ou face}{62,152,177,230,262,266} 
\term{placebo}{263,269} 
\term{Planck (constante de)}{59,356} 
\term{\w loi de ---}{54--59,135,353}
\hbox{Poincar\'e}
\term{\w conventionnalisme de ---}{16--18} 
\term{\w critique de la M\'ecanique statistique}{454--457} 
\term{\w formule de ---}{112--115} 
\term{\w malentendu entre --- et la Physique r\'eelle}{456} 
\term{\w th\'eor\`eme du retour de ---}{422,456--458} 
\term{positivisme}{416--417} 
\term{premier principe de la Thermodynamique}{446} 
\term{premier retour \`a z\'ero}{67,71} 
\term{principe d'\'equivalence statistique}{329--330} 
\hbox{principes de la Thermodynamique}
\term{\w premier ---}{446} 
\term{\w second ---}{419,441--442,446} 
\term{processus en cascade}{177-225, 264} 
\smallskip 
\hbox{probabilit\'es}  
\term{\w --- a priori}{7,267,311,325}   
\term{\w --- biochimiques}{268--270}  
\term{\w --- conditionnelles}{6, 97-105, 179--180,284,302}  
\term{\w formule des --- conditionnelles}{6,103,398}  
\term{\w --- empiriques}{7,31,107}  
\term{\w --- statistiques}{268--270}  
\term{prolongement analytique}{203} 
\vskip9mm \vfill  
\hbox{Q} 
\medskip 
\term{Q.I. (quotient intellectuel)}{320} 
\term{quanta de lumi\`ere)}{54} 
\vskip8mm \vfill 
\hbox{R} 
\medskip 
\term{random (fonction)}{14,20,22,71,91,284,429,433} 
\term{randomisation}{271,272,334} 
\term{random mating (panmixie)}{77,79} 
\term{r\'eaction en cha\^\i ne}{183--185,188} 
\term{r\'ealit\'e}{400--401} 
\term{r\'egression}{} 
\term{\w droite de ---}{323--325,334--335,338,340} 
\term{\w courbe de r\'egression non lin\'eaire}{341--342} 
\term{\w r\'egression non lin\'eaire}{341-353} 
\term{\w mod\`ele de r\'egression non lin\'eaire}{345,353} 
\term{relativit\'e de Galil\'ee}{13,16, 233} 
\term{relativit\'e du hasard}{105-110} 
\term{rep\`ere galil\'een}{16,17} 
\term{reproductible (exp\'erience)}{} 
\term{}{270,285,311,329--332,386,406} 
\hbox{retour}
\term{\w dernier retour \`a z\'ero}{71--73} 
\term{\w premier retour \`a z\'ero}{67,71} 
\term{\w th\'eor\`eme du --- de Poincar\'e}{422,456--458} 
\term{roulette}{22,268} 
\term{roulette (mod\`ele simplifi\'e)}{24,90--97,265} 
\vskip8mm \vfill  
\hbox{S} 
\medskip 
\term{second principe de la Thermodynamique}{} 
\term{}{419,441--442,446} 
\term{sensibilit\'e aux conditions initiales}{28,32,96} 
\term{s\'erie (dans les queues)}{244--246} 
\term{s\'erie enti\`ere}{140,197} 
\term{s\'erie de Laurent}{140,182} 
\term{seuil de certitude (sondage)}{134,282} 
\term{singularit\'e logarithmique}{197} 
\term{sondage}{265--266,268,269,334} 
\term{\w fourchette du ---}{134,282}
\term{\w intervalle de confiance du ---}{282} 
\term{\w marge d'erreur du ---}{282} 
\term{\w seuil de certitude du ---}{134,282} 
\term{\w th\'eorie du ---}{273-288} 
\term{souris (exp\'eriences sur)}{309--312} 
\term{statistique de Bose}{5,6,51,88,136} 
\term{statistique de Fermi-Dirac}{422} 
\term{Stern-Gerlach (aimant)}{379} 
\term{Stirling (formule)}{45,48,254} 
\term{Sto{\ss}zahlansatz (de Boltzmann)}{451} 
\term{Student (test)}{313--320} 
\term{streptokinase}{272} 
\term{subdivisions}{49--53} 
\term{suites al\'eatoires}{20,121,127} 
\term{suites normales (de Borel)}{127--128} 
\term{syst\`eme de Copernic}{17} 
\term{syst\`eme de Ptol\'em\'ee}{17} 
\term{syst\`emes macroscopiques}{418} 
\vskip7mm \vfill  
\hbox{T} 
\medskip 
\term{tabagisme}{329,334--335,340,359} 
\term{Taylor (coefficients)}{200--206} 
\term{temp\'erature absolue}{58,353,420} 
\term{test du $\chi^2$}{299-305} 
\term{tests statistiques}{262-271, 289-320} 
\term{\w --- en simple ou double aveugle}{263} 
\term{\w --- de m\'edicaments}{263--264,268,272} 
\term{test de Student}{313-320} 
\term{textes al\'eatoires}{121-129} 
\term{Thermodynamique (second principe de la)}{} 
\term{}{419,441--442,446} 
\term{tirages avec remise}{35--36} 
\term{tirages sans remise}{36--40} 
\term{transformation de Fourier}{} 
\term{}{140,160,169,171--172,198,207--208,226} 
\term{triangle de Pascal}{42} 
\term{trous de Young}{398} 
\vskip7mm \vfill  
\hbox{V} 
\medskip 
\term{variables al\'eatoires}{60,130-150} 
\term{\w --- esp\'erance}{137,138} 
\term{\w --- loi}{132} 
\term{\w --- loi conjointe}{148,300} 
\term{\w --- moments}{139} 
\term{\w --- variance}{137} 
\term{variable $\lambda$ de John Bell)}{387--388} 
\term{variance empirique}{314} 
\smallskip 
\term{variance d'une variable al\'eatoire}{137} 
\vskip7mm \vfill  
\hbox{Y} 
\medskip 
\term{Young (trous de) }{378} 
\vfill\break 
 
 
\end 
 
 
 

