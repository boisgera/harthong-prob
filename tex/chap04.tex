\input twelvea4.tex
\input epsf.tex
\vsize=218mm

\auteurcourant={\sl J. Harthong: probabilit\'es et statistique}
\titrecourant={\sl L'ind\'ependance stochastique}

\pageno=82

\def\punkt{\vrule height0.4pt depth0pt width0.4pt}
\def\struta{\vrule depth2.2pt width0pt}
\def\ata{\hskip-2.5pt}
\def\aub{\hskip1pt}

\null\vskip10mm plus4mm minus3mm

\centerline{\bf IV.  L'IND\'EPENDANCE STOCHASTIQUE.}
\vskip10mm

{\bf IV\ata .\aub 1. La signification de l'ind\'ependance stochastique.}
\medskip

L'id\'ee d'ind\'ependance stochastique est assez naturelle; reprenons par
exemple le probl\`eme des boules qu'on jette successivement dans des 
cases. Au chapitre $I$ nous avions vu que l'\'equiprobabilit\'e des huit
distributions dans la colonne de gauche de la figure 1 r\'esultait des 
deux faits suivants: 
\smallskip
a) les deux cases sont sym\'etriques de par leurs positions dans l'espace
(aucune des deux n'est privil\'egi\'ee).

b) chaque boule {\it ignore} ce qui est arriv\'e aux pr\'ec\'edentes.
\smallskip
La condition b) est bien s\^ur aussi n\'ecessaire que la premi\`ere, car si 
elle n'\'etait pas v\'erifi\'ee, les deux cases cesseraient d'\^etre
sym\'etriques apr\`es que la premi\`ere boule ait \'et\'e plac\'ee (ce qui
justement est le cas pour des particules de Bose). La condition b) signifie
donc qu'il y a {\it ind\'ependance} entre les lancers successifs. 
\medskip
C'est cette sorte d'ind\'ependance qui correspond au concept d'{\it
ind\'e\-pen\-dance stochastique}.
\medskip
Lorsque le Calcul des probabilit\'es fut axiomatis\'e, on chercha \`a 
donner une d\'efinition purement math\'ematique de l'ind\'ependance
stochastique. Une d\'efinition purement math\'ematique, cela signifie une
d\'efinition qui ne recourt pas \`a des consid\'erations physiques (telles
que la s\'eparabilit\'e spatio-temporelle), ni \`a des consid\'erations
philosophiques (telles que la ``nature'' de l'ignorance mutuelle des 
boules). Une d\'efinition purement math\'ematique doit porter
uniquement sur ce qui se passe \`a l'int\'erieur du mod\`ele
math\'ematique, c'est-\`a-dire, dans le cas du Calcul de probabilit\'es,
\`a l'int\'erieur de l'espace $\Omega$. Mais $\Omega$ n'a aucune 
structure math\'ematique en dehors de la structure nue d'ensemble 
\ftn 1{Bien s\^ur, comme nous l'avons vu au chapitre $I$, la {\it
structure} de cet ensemble doit contenir sous forme implicite toute
l'information qu'on poss\`ede sur le probl\`eme concret sous-jacent, 
mais la construction ou le choix de cette structure concerne la phase de
{\it mod\'elisation}, et non la partie purement math\'ematique du Calcul
des probabilit\'es qui, elle,  pr\'esuppose simplement un ensemble 
$\Omega$ d\'ej\`a donn\'e.}.  Il faut donc fournir une d\'efinition qui 
porte sur cette seule structure d'ensemble,  c'est-\`a-dire sur les {\it
\'el\'ements} (\'epreuves),  les {\it parties} (\'ev\'enements),  les {\it
ensembles de parties},  etc.  
\medskip
Commen\c cons par observer la premi\`ere colonne de la figure 1. 
Dans ce cas $\Omega$ est un ensemble \`a huit \'el\'ements, il y a donc
exactement $2^8=256$ parties (en incluant la partie vide et le tout). 
Il
est bien clair qu'une \'etude exhaustive, comme celle qui va suivre, serait difficile avec un $\Omega$ \`a plusieurs milliards d'\'el\'ements! 
Pour faciliter la nomenclature, num\'erotons les \'epreuves de 1 \`a 8, 
en allant du haut vers le bas sur la figure 1: ainsi l'\'epreuve $5$ est
la cinqui\`eme en partant du haut sur la colonne de gauche de la figure 1
(voir cette figure) et correspond au cas: ``boule 1 dans la case de gauche
ou case ${\cal A}$, boules 2 et 3 dans la case de droite ou case ${\cal
B}$'', etc.  Parmi les $256$ sous-ensembles possibles, on peut remarquer
les six suivants, qui expriment chacun un \'ev\'enement concernant l'une
des trois boules: 
$$\eqalign{
&A_1 : \hbox{``la boule 1 est dans la case ${\cal A}$''} = \{ 1,2,4,5 \} \cr
&A_2 : \hbox{``la boule 2 est dans la case ${\cal A}$''} = \{ 1,2,3,6 \} \cr
&A_3 : \hbox{``la boule 3 est dans la case ${\cal A}$''} = \{ 1,3,4,7 \} \cr
&B_1 : \hbox{``la boule 1 est dans la case ${\cal B}$''} = \{ 3,6,7,8 \} \cr
&B_2 : \hbox{``la boule 2 est dans la case ${\cal B}$''} = \{ 4,5,7,8 \} \cr 
&B_3 : \hbox{``la boule 3 est dans la case ${\cal B}$''} = \{ 2,5,6,8 \} \cr }
$$
Par exemple,  $A_1$ est constitu\'e des \'el\'ements $N^{\rm os}\, 1,2,4,$
et $5$ c'est-\`a-dire des \'epreuves que voici:
\def\trv{\vrule height8pt depth3pt width0.5pt}
\newdimen\lrg 
\lrg=13mm
\def\trh{\hrule height0.4pt depth0pt width29.285mm}
\def\sp{\hskip\lrg}
\def\boule#1{\hbox to 7.76pt{$\bigcirc$\hskip-8pt
\raise0.5pt\hbox{$\scriptstyle #1$}\hfill }}
\vskip12pt
\line{\hfill\vbox{
\hbox{\trv
\hbox to \lrg { \hfill \boule1 \hfill \boule2 \hfill \boule3 \hfill } 
\trv \hbox to \lrg { \hfill } \trv}\trh } \hfill 
\vbox{
\hbox{\trv
\hbox to \lrg { \hfill \boule1 \hfill \boule2 \hfill }
\trv \hbox to \lrg { \hfill \boule3 \hfill } \trv}\trh } \hfill 
\vbox{
\hbox{\trv
\hbox to \lrg { \hfill \boule1 \hfill \boule3 \hfill }
\trv \hbox to \lrg { \hfill \boule2 \hfill } \trv}\trh } \hfill 
\vbox{
\hbox{\trv
\hbox to \lrg { \hfill \boule1 \hfill }
\trv \hbox to \lrg { \hfill \boule2 \hfill \boule3 \hfill } \trv}\trh } \hfill
}
\vskip12pt
En regardant la figure 1 on voit bien que ces quatre \'epreuves sont la
premi\`ere, la deuxi\`eme, la quatri\`eme, et la cinqui\`eme en partant du 
haut. 
\medskip
De m\^eme, $A_3$ est constitu\'e des \'el\'ements $N^{\rm os}\, 1,3,4,$
et $7$ c'est-\`a-dire:
\vskip12pt
\line{\hfill\vbox{
\hbox{\trv
\hbox to \lrg { \hfill \hskip-0.8pt\boule1 \hfill \boule2 \hfill \boule3
\hskip0.8pt \hfill } \trv \hbox to \lrg { \hfill } \trv}\trh } \hfill 
\vbox{
\hbox{\trv
\hbox to \lrg { \hfill \boule2 \hfill \boule3 \hfill }
\trv \hbox to \lrg { \hfill \boule1 \hfill } \trv}\trh } \hfill 
\vbox{
\hbox{\trv
\hbox to \lrg { \hfill \boule1 \hfill \boule3 \hfill }
\trv \hbox to \lrg { \hfill \boule2 \hfill } \trv}\trh } \hfill 
\vbox{
\hbox{\trv
\hbox to \lrg { \hfill \boule3 \hfill }
\trv \hbox to \lrg { \hfill \boule1 \hfill \boule2 \hfill } \trv}\trh }
\hfill } \vskip12pt
Les intersections deux par deux de ces six \'ev\'enements sont: 
$$\eqalign{
&A_1A_2 = \{ 1,2 \} \quad A_1A_3 = \{ 1,4 \} \quad 
A_2A_3 = \{ 1,3 \} \cr
&A_1B_2 = \{ 4,5 \} \quad A_1B_3 = \{ 2,5 \} \quad 
A_2B_3 = \{ 2,6 \} \cr
&A_2B_1 = \{ 3,6 \} \quad A_3B_1 = \{ 3,7 \} \quad 
A_3B_2 = \{ 4,7 \} \cr
&B_1B_2 = \{ 7,8 \} \quad B_1B_3 = \{ 6,8 \} \quad 
B_2B_3 = \{ 5,8 \} \cr }$$
(on n'a pas compt\'e $A_1 B_1$, $A_2 B_2$, $A_3 B_3$, qui sont
\'evidemment vides puis\-qu'une boule ne peut pas \^etre \`a la fois dans la
case ${\cal A}$ et dans la case ${\cal B}$).
\medskip
Les intersections par trois peuvent \'egalement \^etre not\'ees; les 
seules qui sont non vides sont les suivantes:
 $$\eqalign{
& A_1A_2A_3 = \{ 1 \} \quad A_1A_2B_3 = \{ 2 \} \quad 
  A_1B_2A_3 = \{ 4 \} \quad  B_1A_2A_3 = \{ 3 \} \cr
& B_1A_2B_3 = \{ 6 \} \quad A_1B_2B_3 = \{ 5 \}  \quad 
  B_1B_2A_3 = \{ 7 \} \quad B_1B_2B_3 = \{ 8 \} \cr }$$
On voit que toutes les intersections par deux sont compos\'ees de deux
\'el\'ements, et que toutes les intersections par trois sont compos\'ees 
d'un seul \'el\'ement (si elles ne sont pas vides); en termes de
probabilit\'es cela se traduit par le fait que les probabilit\'es des
\'ev\'enements $A_j$ ou $B_j$ valent $1 \over 2$, celles des 
intersections par deux valent $1 \over 4$ (qui est le carr\'e de $1 \over
2$), et celles des intersections par trois valent $1 \over 8$ (qui est le
cube de $1 \over 2$); autrement dit les \'egalit\'es ${\cal P}\, (E \cap F)
= {\cal P}\, (E) \times {\cal P}\, (F)$ et ${\cal P}\, (E \cap F \cap G) =
{\cal P}\, (E) \times {\cal P}\, (F) \times {\cal P}\, (G)$ s'appliquent pour
les intersections retenues ci-dessus. En revanche elles ne s'appliquent
pas pour les intersections vides, telles que $A_1B_1$, $A_2B_2$,
$A_1A_2B_1$, $A_1B_2A_2$, etc, ni pour les intersections trop pleines
telles que $A_1A_1 = A_1$, $A_1 B_2 B_2$, etc. Un examen plus attentif
montre que les intersections qui {\it ne v\'erifient pas} la r\`egle
``probabilit\'e de l'intersection \'egale produit des probabilit\'es'' sont
celles o\`u un indice est r\'ep\'et\'e, tandis que les intersections qui 
{\it v\'erifient} la r\`egle sont celles o\`u les indices sont distincts. Or
l'indice est le num\'ero de la boule; on peut donc dire que la r\`egle ${\cal
P}\, (E \cap F) = {\cal P}\, (E) \times {\cal P}\, (F)$ s'applique si les
\'ev\'enements $E$ et $F$ concernent\ftn 1{Lorsqu'on dit par exemple que
l'\'ev\'enement $E$ {\it concerne} la boule $N^{\rm o} 1$, il faut
comprendre par l\`a qu'en tant que sous-ensemble de $\Omega$ il
regroupe toutes les \'epreuves o\`u la boule $N^{\rm o} 1$ joue un r\^ole
particulier: par exemple l'\'ev\'enement $A_1$ est l'ensemble de toutes
les \'epreuves o\`u la boule $N^{\rm o} 1$ est dans la case ${\cal A}$.} des
boules diff\'erentes, et ne s'applique plus lorsque les \'ev\'enements $E$
et $F$ concernent une seule et m\^eme boule. C'est ainsi que la r\`egle du
produit exprime math\'ematiquement le fait que par exemple la boule
$N^{\rm o} 2$ ``ignore'' ce qui est arriv\'e \`a la boule $N^{\rm o} 1$.

\bigskip

Un examen encore plus approfondi de ce ph\'enom\`ene montre cependant 
que le mod\`ele math\'ematique (c'est-\`a-dire l'ensemble $\Omega$) est
plus riche en \'ev\'enements ind\'ependants que la {\it v\'eritable}
causalit\'e physique ne l'impose. Consid\'erons par exemple les
\'ev\'enements suivants: 
$$C_1 = \{ 1,2,3,4 \} \hskip15mm C_2 = \{ 3,4,5,6 \} \hskip15mm 
C_3  = \{ 2,4,6,8 \}$$
On observera que la r\`egle du produit s'applique aussi \`a ces trois
\'ev\'enements: en effet 
$$C_1C_2 = \{ 3,4 \} \quad C_1C_3 = \{ 2,4 \} \quad C_2C_3 = \{ 4,6 \} $$
ainsi que
$$C_1C_2C_3 = \{ 4 \}$$
de sorte qu'on a \'egalement ${\cal P}\, (C_1C_2) = {\cal P}\, (C_2C_3) =
{\cal P}\, (C_1C_3) = {1 \over 4}$ et ${\cal P}\, (C_1C_2C_3) = {1 \over
8}$, mais on ne voit pas tr\`es bien quelles propri\'et\'es se trouvent 
ainsi {\it s\'epar\'ees} quant \`a la causalit\'e. En r\'ealit\'e le cas des
\'ev\'enements $C_1$,  $C_2$,  $C_3$,  n'est qu'un art\'efact du mod\`ele
math\'ematique.  Cela appara{\^\i}t encore plus clairement si au lieu de se
limiter \`a trois boules et deux cases on consid\`ere le cas g\'en\'eral de
$m$ boules et $N$ cases: les \'ev\'enements du type $A_j$, $B_j$, etc.
conservent leur sens ainsi que la r\`egle du produit; on trouvera pour tout
$m$ et tout $N$ des \'ev\'enements tels que les $C_j$, qui v\'erifient
aussi la r\`egle du produit,  mais sans qu'on puisse leur trouver un sens
qui soit conserv\'e lorsque le nombre de boules ou de cases change.

\bigskip

On peut donc introduire la d\'efinition math\'ematique suivante:
\medskip
Deux \'ev\'enements $E$ et $F$ seront dits {\it stochastiquement
ind\'ependants} si ${\cal P}\, (E \cap F) = {\cal P}\, (E) \times {\cal P}\,
(F)$; plus g\'en\'eralement, si $E_1,E_2,E_3,\ldots E_m$ et
$F_1,F_2,F_3,\ldots F_n$ sont deux familles d'\'ev\'enements sur un 
espace d'\'epreuves $\Omega$, on dira que la famille $\{ E_i \}$ est {\it
stochastiquement ind\'ependante} de la famille $\{ F_j \}$ si pour tout
couple $(i,j)$ on a
$${\cal P}\, (E_i \cap F_j) = {\cal P}\, (E_i) \times {\cal P}\, (F_j)$$ 
Dans ce cas n'importe quelle r\'eunion de plusieurs d'entre
les $E_i$ sera stochastiquement ind\'ependante de n'importe quelle 
r\'eunion de plusieurs d'entre les $F_j$. 
\medskip
Mais il ne faut pas oublier qu'une telle d\'efinition formelle, par nature,
introduit toujours des art\'efacts.

\bigskip

On comprendra mieux la signification de l'ind\'ependance stochastique
formelle en la reliant \`a l'arithm\'etique: elle n'est rien d'autre 
que le reflet de propri\'et\'es purement arithm\'etiques li\'ees \`a la
divisibilit\'e du nombre entier $\#\Omega$. On pouvait d\'ej\`a soup\c
conner cela d'apr\`es la propri\'et\'e du produit: en effet dire que ${\cal
P}\, (AB) = {\cal P}\, (A) \times {\cal P}\, (B)$ revient \`a dire que $\#
(AB) \times \# \Omega = \# (A) \times \# (B)$, ce qui montre bien que
$\# \Omega$ et $\# (A)$ ou $\# (B)$ doivent avoir des diviseurs 
communs. Tout cela devient plus clair si on repr\'esente l'espace
$\Omega$ non comme une liste s\'equentielle, mais comme un tableau qui
refl\`ete la divisibilit\'e de $\#\Omega$ par exemple si $\#\Omega =
35$ on peut pr\'esenter la liste compl\`ete de ses \'el\'ements comme un
tableau \`a cinq lignes et sept colonnes (ou vice-versa). La colonne de
gauche de la figure 1, dans laquelle $\#\Omega = 8 = 4 \times 2$ peut
ainsi \^etre repr\'esent\'ee comme sur la figure 8.

\midinsert
\def\trv{\vrule height8pt depth4pt width0.5pt}
\newdimen\lrg 
\lrg=13mm \advance\lrg by -0.5pt
\def\trh{\hrule height0.5pt depth0pt width2.895cm}
\def\sp{\hskip\lrg}
\def\boule#1{\hbox to 9.7pt{$\bigcirc$\hskip-8pt
\raise0.5pt\hbox{$\scriptstyle #1$}\hfill }}
\vskip12pt
\centerline{
\vtop{\hsize=4cm
\hbox{\trv
\hbox to \lrg { \hfill \boule1 \hfill \boule2 \hfill \boule3 \hfill }
\trv \hbox to \lrg { \hfill } \trv}\trh 
\vskip12pt
\hbox{\trv
\hbox to \lrg { \hfill \boule1 \hfill \boule2 \hfill }
\trv \hbox to \lrg { \hfill \boule3 \hfill } \trv } \trh 
\vskip12pt
\hbox{\trv
\hbox to \lrg { \hfill \boule1 \hfill }
\trv \hbox to \lrg { \hfill \boule2 \hfill \boule3 \hfill } \trv } \trh 
\vskip12pt
\hbox{\trv
\hbox to \lrg { \hfill \boule1 \hfill \boule3 \hfill }
\trv \hbox to \lrg { \hfill \boule2 \hfill } \trv } \trh 
\vskip12pt }
\hskip1cm
\vtop{\hsize=4cm
\hbox{\trv
\hbox to \lrg { \hfill \boule2 \hfill \boule3 \hfill }
\trv \hbox to \lrg { \hfill \boule1 \hfill } \trv } \trh 
\vskip12pt
\hbox{\trv
\hbox to \lrg { \hfill \boule2 \hfill }
\trv \hbox to \lrg { \hfill \boule1 \hfill \boule3 \hfill } \trv } \trh 
\vskip12pt
\hbox{\trv \hbox to \lrg { \hfill } \trv
\hbox to \lrg { \hfill \boule1 \hfill \boule2 \hfill \boule3 \hfill }
\trv} \trh 
\vskip12pt
\hbox{\trv \hbox to \lrg { \hfill \boule3 \hfill }
\trv \hbox to \lrg { \hfill \boule1 \hfill \boule2 \hfill } \trv }\trh 
\vskip12pt }  }
\centerline{\eightpoint figure 8}

\bigskip
\endinsert
\medskip
La colonne de gauche de la figure 8 est l'\'ev\'enement 
que nous avions d\'esign\'e 
ci-dessus par $A_1$ et la colonne de droite l'\'ev\'enement $B_1$.
L'\'ev\'enement $A_2$ est form\'e des deux premi\`eres {\it lignes},
l'\'ev\'enement $B_2$ des deux derni\`eres lignes,  l'\'ev\'enement $B_3$
des lignes $N^{\rm os}$ 2 et 3, l'\'ev\'enement $A_3$ des lignes $N^{\rm
os}$ 1 et 4. La propri\'et\'e du produit est donc simplement l'expression 
du fait bien connu que le nombre d'\'el\'ements d'un tableau est \'egal au
nombre de lignes multipli\'e par le nombre de colonnes. 
\medskip
Bien s\^ur dans la figure 8 les configurations ont \'et\'e rang\'ees en 
tenant compte du {\it sens} de l'exp\'erience, c'est-\`a-dire du r\^ole
jou\'e par les trois boules, de m\^eme que le choix des \'ev\'enements
$A_1,\, A_2,\, A_3,\, B_1,\, B_2,\, B_3,\, $ avait \'et\'e dict\'e par ce
sens. Mais on conserve cette propri\'et\'e de multiplication en permutant
n'importe comment les \'el\'ements de la deuxi\`eme colonne: dans ce cas
les colonnes sont globalement inchang\'ees (elles consistent toujours 
dans les \'ev\'enements $A_1$ et $B_1$) mais les lignes sont
diff\'erentes:  par exemple sur la figure 8 les lignes $N^{\rm os}$ 2 et 3
avaient en commun que la boule $N^{\rm o}$ 3 \'etait dans la case ${\cal
B}$ (\'ev\'enement $B_3$);  apr\`es permutation de la deuxi\`eme colonne
il ne reste plus rien qui ait un rapport avec la boule $N^{\rm o}$ 3: le sens
est perdu, mais la propri\'et\'e de multiplication subsiste.

\medskip

En fait $\#\Omega = 8$ peut se d\'ecomposer davantage; au lieu de $2
\times 4$, c'est $2 \times 2 \times 2$, donc au lieu d'un tableau \`a
quatre lignes et deux colonnes, c'est un cube, ou un tableau \`a trois 
entr\'ees de deux \'el\'ements chacune: la premi\`ere entr\'ee correspond 
aux \'ev\'enements $A_1$ et $B_1$, la deuxi\`eme entr\'ee aux
\'ev\'enements $A_2$ et $B_2$, et la troisi\`eme entr\'ee aux
\'ev\'enements $A_3$ et $B_3$ (voir figure 9). 

\medskip

\midinsert
\vskip3pt
\centerline{\epsfbox{../images/figure.eps} }
\vskip3mm
\centerline{\eightpoint figure 9}
\vskip6pt
\endinsert

\medskip

On retrouve ainsi ce qui en r\'ealit\'e avait pr\'esid\'e d\`es le d\'epart 
\`a la construction du mod\`ele math\'ematique $\Omega$.  L'espace des
\'epreuves a une structure qui refl\`ete directement les invariances
physiques ou la causalit\'e qui sont \`a l'origine de l'\'equiprobabilit\'e.
Mais en construisant un tel mod\`ele, on introduit ipso facto des
r\'egularit\'es suppl\'ementaires (qui d\'erivent de celles qu'on a 
postul\'e a priori par des permutations) qui n'ont plus aucune
signification par rapport au probl\`eme qu'on voulait mod\'eliser.
\medskip
Ainsi, on peut voir sur la figure 9 que l'\'ev\'enement $A_1$ (``la boule
$N^{\rm o} 1$ est dans la case ${\cal A}$'') correspond \`a la face gauche 
du cube; l'\'ev\'enement $A_2$ \`a la face avant; l'\'ev\'enement $A_3$
\`a la face de dessous;  l'\'ev\'enement $B_1$ \`a la face de droite;
l'\'ev\'enement $B_2$ \`a la face arri\`ere;  l'\'ev\'enement $B_3$ \`a la
face de dessus.  Les intersections $A_1 A_2$,  $A_1 A_3$, etc. sont donc 
des intersections de deux faces,  c'est-\`a-dire des ar\`etes (ou rien si
les
faces sont oppos\'ees comme $A_1 B_1$).  Les intersections par trois
$A_1 A_2 A_3$, $A_1 B_2 A_3$, etc. sont des sommets.  Tout ce qui avait
\'et\'e dit plus haut s'interpr\`ete donc selon la structure de {\it produit
cart\'esien} de l'espace $\Omega$.  Le fait que chacune des trois boules
ignore ce que font les deux autres (ce qui constitue la propri\'et\'e
causale caract\'eristique du probl\`eme \`a mod\'eliser) est traduit dans
le mod\`ele $\Omega$ par un produit cart\'esien d'ensembles:  pour chaque
boule il y a deux possibilit\'es ${\cal A}$ ou ${\cal B}$,  et il y a trois
boules qui s'ignorent mutuellement {\it donc} l'ensemble $\Omega$ sera
de la forme $\{ {\cal A},{\cal B}\}^3$.  L'espace $\Omega$ de toutes les
\'epreuves possibles pour $n$ cases et $r$ boules serait un hypercube \`a
$r$ dimensions, avec des ar\`etes \`a $n$ \'el\'ements;  en toute logique,
son cardinal est $n^r$,  conform\'ement \`a $(II.1.)$  Si maintenant on
permute n'importe comment chaque ar\`ete,  on ne change pas la structure
de produit,  mais on fait dispara{\^\i}tre le rapport avec les boules. 
\medskip
L'ind\'ependance stochastique formelle d'une famille d'\'ev\'enements 
par rapport \`a une autre,  c'est-\`a-dire le simple fait d'avoir ${\cal P}\,
(AB) = {\cal P}\, (A) \times {\cal P}\, (B)$,  est toujours li\'ee \`a une
d\'ecomposition de $\Omega$ en produit d'ensembles plus petits (et donc
aussi \`a la d\'ecomposition de l'entier $\#\Omega$ en produit de 
facteurs premiers).  Tout probl\`eme de probabilit\'e qui comporte une 
ind\'ependance {\it causale} se mod\'elisera sous forme d'un produit
cart\'esien qui refl\`ete directement et par construction cette
ind\'ependance causale;  mais cela n'entra{\^\i}ne de loin pas que
r\'eciproquement,  toute autre d\'ecomposition de $\Omega$ en produit
implique une signification causale.
\medskip

\midinsert
\def\trv{\vrule height6pt depth4pt width0.5pt}
\newdimen\lrg 
\lrg=10mm \advance\lrg by -0.5pt
\def\trh{\hrule height0.5pt depth0pt width3.43cm}
\def\sp{\hskip\lrg}
\def\ball{\hbox{$\scriptscriptstyle\bigcirc$}}

\line{\vbox{
\hbox{\trv
\hbox to \lrg { \hskip-1.1pt \hfill \ball \hfill \ball \hfill \ball \hfill \ball \hfill }
\trv \hbox to \lrg { \hfill } \trv \hbox to \lrg { \hfill } \trv}\trh }
\hfill\vbox{
\hbox{\trv\hbox to \lrg { \hfill } \trv 
\hbox to \lrg { \hskip-1.1pt \hfill \ball \hfill \ball \hfill \ball \hfill \ball \hfill }
\trv \hbox to \lrg { \hfill } \trv}\trh }
\hfill\vbox{
\hbox{\trv \hbox to \lrg { \hfill } \trv \hbox to \lrg { \hfill } \trv
\hbox to \lrg { \hskip-1.1pt \hfill \ball \hfill \ball \hfill \ball \hfill \ball \hfill }
\trv} \trh }
}
\bigskip
\line{\vbox{
\hbox{\trv
\hbox to \lrg { \hfill \ball \hfill \ball \hfill \ball \hfill }
\trv \hbox to \lrg { \hfill \ball \hfill } \trv \hbox to \lrg { \hfill } \trv}
\trh } 
\hfill\vbox{
\hbox{\trv \hbox to \lrg { \hfill } \trv
\hbox to \lrg { \hfill \ball \hfill \ball \hfill \ball \hfill }
\trv \hbox to \lrg { \hfill \ball \hfill } \trv}\trh }
\hfill\vbox{
\hbox{\trv\hbox to \lrg { \hfill \ball \hfill } \trv \hbox to \lrg { \hfill } 
\trv \hbox to \lrg { \hfill \ball \hfill \ball \hfill \ball \hfill }
\trv }\trh }
}
\bigskip
\line{\vbox{
\hbox{\trv
\hbox to \lrg { \hfill \ball \hfill \ball \hfill \ball \hfill } \trv 
\hbox to \lrg { \hfill } \trv 
\hbox to \lrg { \hfill \ball \hfill } \trv} \trh }
\hfill\vbox{
\hbox{\trv \hbox to \lrg { \hfill \ball \hfill } \trv
\hbox to \lrg { \hfill \ball \hfill \ball \hfill \ball \hfill }
 \trv \hbox to \lrg { \hfill } \trv} \trh }
\hfill\vbox{
\hbox{\trv \hbox to \lrg { \hfill } \trv \hbox to \lrg { \hfill \ball \hfill } 
\trv \hbox to \lrg { \hfill \ball \hfill \ball \hfill \ball \hfill }
\trv}\trh }
}
\bigskip
\line{\vbox{
\hbox{\trv
\hbox to \lrg { \hfill \ball \hfill \ball \hfill }
\trv \hbox to \lrg { \hfill \ball \hfill } \trv \hbox to \lrg { \hfill \ball 
\hfill } \trv}\trh } 
\hfill\vbox{
\hbox{\trv\hbox to \lrg { \hfill \ball \hfill } \trv 
\hbox to \lrg { \hfill \ball \hfill \ball \hfill }
\trv \hbox to \lrg { \hfill \ball \hfill } \trv}\trh }
\hfill\vbox{
\hbox{\trv \hbox to \lrg { \hfill \ball \hfill } \trv  \hbox to \lrg { \hfill
\ball \hfill } \trv \hbox to \lrg { \hfill \ball \hfill \ball \hfill }
\trv}\trh }
}
\bigskip
\line{\vbox{
\hbox{\trv \hbox to \lrg { \hfill } \trv  \hbox to \lrg { \hfill \ball \hfill
\ball \hfill } \trv \hbox to \lrg { \hfill \ball \hfill \ball \hfill }
\trv}\trh }
\hfill\vbox{
\hbox{\trv\hbox to \lrg { \hfill \ball \hfill \ball \hfill } \trv 
\hbox to \lrg { \hfill }
\trv \hbox to \lrg { \hfill \ball \hfill \ball \hfill } \trv}\trh }
\hfill\vbox{
\hbox{\trv
\hbox to \lrg { \hfill \ball \hfill \ball \hfill }
\trv \hbox to \lrg { \hfill \ball \hfill \ball \hfill } \trv \hbox to \lrg {
\hfill } \trv}\trh } 
}
\bigskip
\centerline{\eightpoint figure 10}
\bigskip
\endinsert

C'est ce qu'on peut constater par exemple dans le cas de la statistique de
Bose-Einstein.  Le nombre de configurations pour $r$ particules et $N$
\'etats est $(r+N-1)!\; / \; r!\, (N-1)!$ Ce nombre n'est certes jamais
premier et on peut donc toujours d\'ecomposer $\Omega$ d'une fa\c con 
ou d'une autre en produit. Mais contrairement \`a ce qui \'etait le cas pour
les boules, o\`u le nombre de facteurs du produit \'etait toujours \'egal 
au nombre de boules, l'espace $\Omega$ pour $m$ particules de Bose et
$N$ \'etats quantiques est bien d\'ecomposable en produit, mais de
diff\'erentes mani\`eres, qui varient selon les valeurs de $m$ et de $N$,
sans qu'on puisse observer une r\`egle permanente; ceci refl\`ete le fait
que lorsque des particules de Bose viennent occuper des \'etats 
quantiques, elles ne s'y rangent pas ind\'ependamment les unes des autres
(cette propri\'et\'e est appel\'ee la {\it non-s\'eparabilit\'e quantique}).
Il n'y a pas d'ind\'ependance causale entre les particules, mais cela
n'emp\^eche pas de trouver dans chaque cas particulier (c'est-\`a-dire
pour chaque valeur particuli\`ere de $m$ ou $N$) des factorisations
possibles pour l'espace $\Omega$. 
\medskip
Pour illustrer cela,  on a repr\'esent\'e sur la figure 10 l'espace $\Omega$
pour une statistique de Bose \`a $4$ particules et $3$ \'etats;  il y a
$15 = 5
\times 3$ modes d'occupation possibles, donc une factorisation en 
tableau \`a deux dimensions. 
\medskip
La m\^eme structure que celle repr\'esent\'ee sur la figure 10 peut \^etre
d\'ecrite par le tableau de chiffres que voici:
$$\matrix{
& (4,\, 0,\, 0)\quad    & (0,\, 4,\, 0)\quad    & (0,\, 0,\, 4)\quad    \cr
& (3,\, 1,\, 0)\quad    & (0,\, 3,\, 1)\quad    & (1,\, 0,\, 3)\quad    \cr
& (3,\, 0,\, 1)\quad    & (1,\, 3,\, 0)\quad    & (0,\, 1,\, 3)\quad    \cr
& (2,\, 1,\, 1)\quad    & (1,\, 2,\, 1)\quad    & (1,\, 1,\, 2)\quad    \cr
& (0,\, 2,\, 2)\quad    & (2,\, 0,\, 2)\quad    & (2,\, 2,\, 0)\quad    \cr }$$
On peut s'amuser \`a constater que par exemple la
famille d'\'ev\'enements $\{ A_j \}\; (j = 1,2,3)$ correspondant aux
colonnes du tableau peut recevoir le sens physique suivant: $A_j$ est
l'\'ev\'enement ``l'\'etat $N^{\rm o} j$ poss\`ede le nombre d'occupation 
non ex-aequo le plus \'elev\'e''. La famille $\{ B_i \}$ correspondant \`a
des lignes recevrait le sens suivant: 
\smallskip
$B_1$ ``les nombres d'occupation sont 4, 0, et 0'' ($1^{\rm re}$ ligne);

$B_2$ ``les nombres d'occupation sont 3, 1, et 0'' ($2^{\rm e}$ et $3^{\rm
e}$ lignes); 

$B_3$ ``les nombres d'occupation sont 2, 1, et 1'' ($4^{\rm e}$ ligne); 

$B_4$ ``les nombres d'occupation sont 2, 2, et 0'' ($5^{\rm e}$ ligne);
\smallskip
Les \'ev\'enements de la famille $A$ sont stochastiquement 
ind\'ependants de ceux de la famille $B$.  Ceci a-t-il une signification
quant \`a la causalit\'e physique?  Par exemple cela aurait-il la
signification que les particules,  une fois que leur r\'epartition en 
groupes d'effectifs fix\'es est faite, vont placer le groupe le plus
nombreux {\it ind\'ependamment} de la r\'epartition?  Pour que cela
corresponde \`a une propri\'et\'e physique,  il faudrait au moins que cela
reste vrai quel que soit le nombre d'\'etats et le nombre de particules; 
or si on prend trois particules occupant trois \'etats,  il y a $10 = 2 
\times 5$ configurations,  donn\'ees par le tableau suivant:  
$$\matrix{
& (3,\, 0,\, 0)\qquad    & (0,\, 2,\, 1)    \cr
& (2,\, 0,\, 1)\qquad    & (0,\, 1,\, 2)    \cr
& (1,\, 2,\, 0)\qquad    & (0,\, 3,\, 0)    \cr
& (1,\, 0,\, 2)\qquad    & (0,\, 0,\, 3)    \cr
& (1,\, 1,\, 1)\qquad    & (2,\, 1,\, 0)    \cr }$$
On peut toujours fabriquer des paires d'\'ev\'enements ind\'ependants en
regroupant des lignes et des colonnes,  mais la propri\'et\'e physique qui
avait sembl\'e appara{\^\i}tre dans le cas pr\'ec\'edent a compl\`etement
disparu:  il n'y a aucune chance de finir par la faire appara{\^\i}tre en
permutant correctement la deuxi\`eme colonne par rapport \`a la
premi\`ere,  puisque tout simplement le nombre de colonnes n'est plus
\'egal au nombre d'\'etats,  et ne {\it peut pas} l'\^etre, parce que 3 n'est
pas un diviseur de 10.
\medskip  
En fin de compte tout est li\'e \`a la divisibilit\'e de l'entier $\#\Omega$:
plus il poss\`ede de diviseurs,  plus nombreuses sont les familles
d'\'ev\'enements stochastiquement ind\'ependantes.  Mais il est 
rarissime que cette ind\'ependance stochastique formelle soit le 
reflet d'une ind\'ependance causale.  Lorsque c'est le cas,  c'est que
l'ind\'ependance causale a \'et\'e postul\'ee a priori et a {\it pr\'esid\'e}
\`a la fabri\-ca\-tion du mod\`ele:  comme  dans le cas des trois boules, 
le
mod\`ele est alors fabriqu\'e d\'elib\'er\'ement sous forme de produit
cart\'esien,  pr\'ecis\'ement pour refl\'eter une ind\'ependance causale
sous la forme de l'ind\'ependance stochastique.  Par permutations il s'y
ajoute encore une foule d'autres familles stochastiquement
ind\'ependantes d'\'ev\'enements,  mais qui n'ont aucune signification
causale.  Dans le cas des particules de Bose,  le mod\`ele n'est pas
fabriqu\'e au d\'epart sous forme de produit cart\'esien,  et s'il se trouve
qu'il est factorisable,  cela ne refl\`ete rien de physique. 
\medskip 
Il est bien s\^ur impossible d'affirmer avec certitude qu'aucune de ces
factorisations ``parasites'' ne peut \^etre le reflet d'une causalit\'e 
encore inconnue; mais ces factorisations non voulues \`a l'avance
proviennent de propri\'et\'es arithm\'etiques des nombres entiers, et
dispara{\^\i}traient sous leur forme exacte si par exemple
l'\'equiprobabilit\'e postul\'ee a priori n'\'etait qu'approximative au lieu
d'\^etre math\'ematiquement exacte. Or on ne peut donner un sens
physique \`a une propri\'et\'e qui dispara{\^\i}t d\`es lors que les
param\`etres physiques subissent une variation arbitrairement petite.
C'est pourquoi une propri\'et\'e du type ${\cal P}\, (AB) \simeq {\cal P}\,
(A) \times {\cal P}\, (B)$ (avec $\simeq$ au lieu de $=$), mais qui ne
changerait pas compl\`etement pour chaque nouvelle valeur de $m$ ou
$N$, aurait bien plus de chances d'avoir un sens physique que des
factorisations exactes.

\bigskip

{\bf IV\ata .\aub 2. L'effacement de la causalit\'e.}
\medskip

Afin de bien comprendre ce que signifie r\'eellement l'ind\'ependance
stochastique,  il faut encore la relier \`a la {\it nature} du hasard.
Tout au long de ce chapitre, nous avons beaucoup insist\'e sur
l'ind\'ependance stochastique en tant que reflet d'une ind\'ependance
causale sous-jacente. Ainsi l'exemple des trois boules ``qui s'ignorent
mutuellement''. Dans ce cas, l'ignorance mutuelle s'expliquait par la
M\'ecanique classique. On pourrait \'etudier plus en d\'etail le r\^ole 
jou\'e par la M\'ecanique en reprenant le mod\`ele simplifi\'e de
roulette pr\'esent\'e au chapitre {\bf I}.  Les trois boules seraient 
une bille de roulette qu'on aurait lanc\'ee trois fois.  L'\'ev\'enement
que la boule tombe dans la bo{\^\i}te ${\cal A}$ correspondrait \`a un
r\'esultat tel que ``rouge'',  ``impair'',  ou ``manque''.  Par exemple
on peut
supposer que le disque est divis\'e en trente six secteurs \'egaux
(d'ouverture 10 degr\'es chacun),  num\'erot\'es de 1 \`a 36;  on dira que 
le r\'esultat est ``impair'' si la bille s'arr\^ete dans un secteur portant un
num\'ero impair.  Lorsqu'on lance la bille avec un certain angle $\theta$ 
et  un coefficient de frottement $k$,  le point du disque o\`u la bille
s'arr\^etera est,  comme nous avons vu au chapitre {\bf I},  parfaitement
d\'etermin\'e,  mais il y a un ph\'enom\`ene de chaos d\'eterministe qui
cr\'ee du hasard.  Le r\'esultat, par exemple ``impair'',  est d\'etermin\'e
par les conditions initiales de lancement,  c'est-\`a-dire par les valeurs
--~rigoureusement exactes~-- de $\theta$ et $k$.  L'ind\'ependance
causale entre les trois lancers signifie que les valeurs de $\theta$ et
$k$ au moment o\`u on lance une deuxi\`eme fois la bille sont
ind\'ependantes du point o\`u celle-ci \'etait arriv\'ee la premi\`ere
fois,  et de m\^eme,  que ces valeurs au moment o\`u on lance une
troisi\`eme fois la bille sont ind\'ependantes des points o\`u celle-ci
\'etait arriv\'ee la premi\`ere et la deuxi\`eme fois.  Lorsqu'on voit les
choses ainsi,  il est  clair qu'il n'y a eu aucune intervention du hasard
entre le moment o\`u la bille a \'et\'e l\^ach\'ee sur le disque et la
manifestation du r\'esultat:  le hasard serait intervenu au moment du
``choix'' des valeurs de $\theta$ et $k$ et plus jamais ensuite.  Or toute
la discussion pr\'esent\'ee en {\bf I.4.} sur  le hasard montrait au
contraire,  que non seulement {\it du} hasard est cr\'e\'e par le 
mouvement lui-m\^eme,  mais que c'est m\^eme {\it le} hasard qui 
par nature est toujours cr\'e\'e comme cela.  M\^eme si un {\it autre}
hasard est intervenu au moment du choix des valeurs de $\theta$ et $k$,
celui-ci n'est peut-\^etre pas d'une autre nature,  il  peut lui aussi avoir
\'et\'e produit par un effet de chaos (par exemple les valeurs de $\theta$
et $k$ peuvent \^etre choisies par une fonction {\bf random},  ou bien la
bille peut avoir \'et\'e lanc\'ee par un croupier alcoolique dont la main
tremble),  mais en renvoyant tout \`a ce hasard premier nous n'expliquons
rien,  nous ne faisons que d\'eplacer le probl\`eme.  
\medskip 
C'est pourquoi il nous faut encore examiner de plus pr\`es comment le
mouvement
d\'eterministe de la bille cr\'ee l'ind\'ependance stochastiquel\`a o\`u au d\'epart il n'y a pas d'ind\'ependance causale,  en effa\c{c}ant
en quelque sorte la causalit\'e.  On compl\`etera ainsi la discussion du
chapitre {\bf I},  car le m\'ecanisme par lequel le chaos transforme
le d\'eterminisme en hasard consiste pr\'ecis\'ement \`a rendre 
{\it stochastiquement} ind\'ependants des ph\'enom\`enes qui sont 
{\it causalement} d\'ependants.
\medskip
Imaginons que la bille soit bien lanc\'ee trois fois,  mais que les choix
initiaux de $\theta$ et $k$,  au lieu d'\^etre causalement ind\'ependants
entre eux,  soient \'etroitement d\'ependants:  pour fixer les id\'ees,
disons que,  si $\theta_1$ et $k_1$ sont les valeurs initiales du
premier lancer et sont,  elles,  ``choisies au hasard'',  alors le second
lancer sera effectu\'e avec $\theta_2 = \alpha\,\theta_1$ et $k_2 = 
k_1$,  et le troisi\`eme avec $\theta_3 = \theta_1$ et  $k_3 = \beta\,
k_1$,  $\alpha$ et $\beta$ \'etant des param\`etres fix\'es \`a l'avance.   
On suppose bien entendu comme au chapitre {\bf I} que $k_1$ est petit,    
de l'ordre de $10^{-4}$,  de sorte que la bille ne s'arr\^etera qu'apr\`es
plusieurs milliers de tours.  Dans ces conditions les trois lancers ne sont
absolument pas ind\'ependants,  puisque $\theta_2$,  $\theta_3$, $k_2$, 
et $k_3$ sont fonctions de $\theta_1$ et $k_1$.  De ce fait les choix de
$\theta_1$,  $\theta_2$,  et $\theta_3$ (ou de $k_1$, $k_2$, et $k_3$) ne
sont pas stochastiquement ind\'ependants,  ce qu'on peut  v\'erifier
ais\'ement:  si on divise l'intervalle des valeurs possibles de $\theta$, 
soit $] 0\, , \, \pi  [$ en deux demi-intervalles,  par exemple en  $] 0\, , \,
{\pi \over 2} [$ et $[ {\pi \over 2}\, , \, \pi [$, alors la probabilit\'e pour
que $\theta_1$,  qui est choisi au hasard,  soit dans l'un ou l'autre de ces
deux demi-intervalles,  par exemple $] 0\, , \, {\pi \over 2} [$, est bien
s\^ur  ${1 \over 2}$;  mais (compte tenu que les valeurs de $\theta_1$,
$\theta_2$,  et $\theta_3$ sont fonctions les unes des autres) la
probabilit\'e pour que $\theta_1$, $\theta_2$,  et $\theta_3$ soient {\it
tous les trois} dans ce m\^eme demi-intervalle n'est pas ${1 \over 8}$; 
un rapide calcul montre qu'elle est \'egale \`a ${1 \over 2 \alpha }$ si
$\alpha > 1$ et \`a ${1 \over 2}$ si $\alpha \leq 1$; en effet, puisque
$\theta_2 = \alpha\,\theta_1$ et $\theta_3 = \theta_1$,  on voit bien 
que si $\alpha > 1$, la condition n\'ecessaire et suffisante pour que
$\theta_1$,  $\theta_2$,  et $\theta_3$ soient tous les trois dans
l'intervalle $]0\, , \, {\pi\over 2}[$ est que $\theta_1$ soit dans $]0\, , \,
{\pi\over 2\alpha }[$, \'ev\'enement dont la probabilit\'e est  bien  ${1
\over 2\alpha}$;  si par contre $\alpha <1$,  cette condition n\'ecessaire 
et suffisante est que $\theta_1$ soit dans $]0\, , \, {\pi\over 2}[$ (il est
vrai que si $\alpha = 4$,  la probabilit\'e est quand m\^eme ${1\over 8}$,
mais ceci est un cas particulier sans signification).  Cela exprime la 
forte d\'ependance entre les trois valeurs.  S'il y avait ind\'ependance
stochastique entre les trois choix de $\theta_1$, $\theta_2$,  et
$\theta_3$,  alors,  puisque chacun des trois choix aurait une probabilit\'e
${1 \over 2}$ de tomber dans le demi-intervalle $] 0\, , \, {\pi \over 2}
[\,$, la probabilit\'e pour que tous les trois y tombent \`a la fois serait 
le produit ${1 \over 2} \times {1 \over 2}  \times {1 \over 2} =  {1 \over
8}$.  Mais comme $\theta_3 = \theta_1$, si $\theta_1$ y tombe,
$\theta_3$ y sera automatiquement aussi et $\theta_2$ y sera
\'egalement si $\theta_1$ est dans $] 0\, , \, {\pi \over 2\alpha }[$.  Un
constat analogue vaut pour $k_1$, $k_2$, et $k_3$, ou  pour les neuf
couples $(\theta_i\, , k_j)$.     
\medskip  
Puisque les conditions initiales des trois lancers ne sont pas
ind\'e\-pen\-dan\-tes, et que les positions finales atteintes par
la bille
sont rigoureusement d\'etermin\'ees par les conditions initiales, il ne
peut pas y avoir non plus ind\'ependance causale rigoureuse entreles
trois positions finales. Et pourtant, si on fait l'exp\'erience, on obtiendra
malgr\'e cela des r\'esultats stochastiquement ind\'ependants.
C'est-\`a-dire que la d\'ependance causale des choix de $\theta_1$,
$\theta_2$, et $\theta_3$ se traduit bien par leur d\'ependance
stochastique(la pro\-ba\-bi\-lit\'e que les trois soient dans le demi-intervalle
$] 0\, , \, {\pi \over 2} [$ est ${1 \over 2\alpha }$ et non ${1 \over
8}$),  mais les trois positions {\it en fin de trajectoire},  bien qu'\'etant
toujours,  en toute rigueur,  causalement li\'ees,  seront,
elles, 
stochastiquement ind\'ependantes. \medskip
Le ph\'enom\`ene de chaos
d\'eterministe produit par letr\`es grand nombre de r\'eflexions sur le
bord, que nousavons analys\'e au chapitre {\bf I}, aura eu pour effet
d'{\it
effacer} la d\'ependance entre les trois lancers. 
\medskip  
Faire l'exp\'erience signifie ici recommencer les trois lancers 
un tr\`es grand nombre de fois, par exemple un million de fois, et mesurer
les probabilit\'es par la fr\'equence de chaque chiffre. Cette exp\'erience
peut fort bien n'\^etre qu'une simulation num\'erique. La probabilit\'e
d'avoir ``impair'' \'etant ${1 \over 2}$ avec le premier  des trois lancers,
on aura environ $500\,000$ fois ``impair'' et $500\,000$ fois ``pair'' pour
le premier  des trois r\'esultats. Mais,  bien que les trois lancers ne
soient pas du  tout ind\'ependants, et que par exemple sur le million
d'essais il  arrivera environ $500\,000  / \alpha$ fois que les trois
valeurs  $\theta_1$, $\theta_2$, et $\theta_3$ soient toutes trois
inf\'erieures \`a ${\pi \over 2}$, il n'arrivera que $125\, 000$ fois
environ que les  trois r\'esultats \`a la fois soient impairs (r\'ev\'elant
ainsi une probabilit\'e de ${1 \over 8}$), c'est-\`a-dire  que tout se
passera comme si les trois r\'esultats \'etaient stochastiquement
ind\'ependants. C'est ainsi que le chaos d\'eterministe {\it efface} la
causalit\'e. Comment cela est-il possible?  
\medskip
Dans cette question nous laissons volontairement inexpliqu\'e le hasard qui
choisit la valeur de $\theta_1$; nous ne nous int\'eressons qu'au hasard
relatif qui est fabriqu\'e par le chaos pour les positions finales aux 
second et troisi\`eme lancer.
\medskip
Si on r\'ealisait effectivement cette exp\'erience par simulation
num\'erique, on {\it constaterait} qu'au premier lancer la bille s'arr\^ete 
une fois sur deux sur un chiffre impair, et que une fois sur huit 
toutes les trois s'arr\^etent sur un chiffre impair. On {\it constaterait},
mais on ne {\it comprendrait} pas pour autant. Une explication qualitative
est dans de tels cas plus instructive. La cl\'e du probl\`eme  est bien
s\^ur que de petits changements sur $\theta$ entra{\^\i}nent de grands
changements sur le r\'esultat final. Ainsi, si on prend $\theta_2 =
\theta_1 + \varepsilon$, avec $\varepsilon$ tr\`es petit, nous savons
d'apr\`es ce qui a \'et\'e vu  en {\bf I.4.} que le point o\`u la bille aboutira
au second lancer se situera  \`a une distance de l'ordre de
$NR\varepsilon$ ($N$ \'etant le nombre de cordes parcourues et $R$ le
rayon du disque) du point atteint au premier lancer (en supposant que $k$
n'a pas chang\'e). Cette distance est en g\'en\'eral nettement sup\'erieure
\`a la largeur d'un secteur de 10 degr\'es. Si $\theta_2 =
\alpha\,\theta_1$, $\alpha$ n'\'etant pas sp\'ecialement proche de 1, la
disparit\'e des positions finales sera  encore bien plus consid\'erable. La
probabilit\'e pour que $\theta_1$ et $\theta_2$ soient tous deux dans le
m\^eme demi-intervalle $] 0\, , \, {\pi \over 2} [$ est  ${1 \over 2\alpha
}$ et non ${1 \over 4}$, parce que les deux valeurs $\theta_1$ et
$\theta_2$ sont li\'ees. Supposons maintenant que, au lieu d'avoir deux
demi-intervalles de longueur 90 degr\'es, on ait d\'ecoup\'e l'intervalle
de 0 \`a 180 degr\'es  en 10800 petits intervalles de longueur une minute
d'angle, alternativement pairs  et impairs. On peut repr\'esenter cela
graphiquement en noircissant les intervalles impairs et en laissant en
blanc les intervalles pairs:  
\vskip8pt plus4pt minus4pt
\setbox11=\hbox to 22mm{\hfill\vrule height0.4pt depth0pt width11mm}
\setbox7=\hbox to 14mm{\hfill\vrule height0.4pt depth0pt width7mm}
\line{\punkt\copy11\copy11\copy11\copy11\copy11\copy11
\hfill } 
\vskip12pt plus4pt minus4pt
\noindent alors, pour que $\theta_1$ et $\theta_2$ soient tous
deux dans un intervalle pair, il faut que non seulement $\theta_1$ soit 
tomb\'e par hasard dans un intervalle pair, mais que $\alpha\,\theta_1$ 
soit lui aussi dans un intervalle pair: cela revient donc \`a d\'ecouper
l'intervalle de 0 \`a 180 degr\'es en
$10800 \times \alpha$ petits intervalles de longueur $1/\alpha$ minute
d'angle, alternativement pairs  et impairs, et \`a exiger que $\theta_1$
soit dans un intervalle pair de cette nouvelle s\'erie, qu'on peut
repr\'esenter graphiquement de la m\^eme fa\c con:
\vskip8pt plus4pt minus4pt
\line{\punkt\copy7\copy7\copy7\copy7\copy7\copy7\copy7 \copy7\copy7\hfill}
\vskip12pt plus4pt minus4pt
\noindent Ainsi la condition que $\theta_1$ et $\theta_2$ soient tous
deux dans un intervalle pair de la premi\`ere s\'erie, \'equivaut \`a dire 
que  $\theta_1$ soit \`a la fois dans un intervalle pair 
de la premi\`ere
s\'erie  et dans un intervalle pair de la deuxi\`eme s\'erie; autrement dit,
$\theta_1$ doit \^etre dans une zone qui est blanche \`a 
la fois sur la
premi\`ere et la deuxi\`eme s\'erie, ce qu'on peut visualiser en 
superposant les deux s\'eries: 
\vskip8pt plus4pt minus4pt
\line{\punkt\copy11\copy11\copy11\copy11\copy11\copy11 \hfill } \vskip-12.5pt
\line{\punkt\copy7\copy7\copy7\copy7\copy7\copy7\copy7 \copy7\copy7\hfill}
\line{\punkt\copy11\copy11\copy11\copy11\copy11\copy11 \hfill } \vskip-15pt
\line{\punkt\copy7\copy7\copy7\copy7\copy7\copy7\copy7 \copy7\copy7\hfill }
\vskip12pt plus4pt minus4pt
\noindent Sur la ligne du bas, les deux s\'eries sont compl\`etement
superpos\'ees, alors que sur les deux lignes rapproch\'ees du haut on les
voit encore s\'epar\'ees. La condition pour que $\theta_1$ et $\theta_2$
soient tous deux dans un intervalle pair de la premi\`ere s\'erie se traduit
par le fait que $\theta_1$ se trouve dans une r\'egion rest\'ee blanche
apr\`es la superposition. Puisque $\theta_1$ est choisi au hasard, la
probabilit\'e pour qu'il en soit ainsi est simplement le rapport entre la
somme des longueurs des r\'egions rest\'ees blanches et la longueur 
totale de 180 degr\'es. Ce rapport n'est pas facile \`a calculer car il varie
avec la finesse de la subdivision; mais il tend vers une limite quand la
subdivision devient de plus en plus fine (des intervalles d'une minute
d'angle sont en pratique assez fins pour \^etre assimil\'es au cas limite).
Voici par exemple ce que donnent graphiquement des subdivisions de
plus en plus fines:

\vskip10pt plus4pt minus4pt

\line{\punkt\copy11\copy11\copy11\copy11\copy11\copy11 \hfill } \vskip-12pt
\line{\punkt\copy7\copy7\copy7\copy7\copy7\copy7\copy7 \copy7\copy7\hfill }
\line{\punkt\copy11\copy11\copy11\copy11\copy11\copy11 \hfill } \vskip-15pt
\line{\punkt\copy7\copy7\copy7\copy7\copy7\copy7\copy7 \copy7\copy7\hfill }

\vskip15pt plus4pt minus4pt

\setbox11=\hbox to 11mm{\hfill\vrule height0.4pt depth0pt width5.5mm}
\setbox7=\hbox to 7mm{\hfill\vrule height0.4pt depth0pt width3.5mm}
\line{\punkt\copy11\copy11\copy11\copy11\copy11\copy11 \copy11\copy11\copy11\copy11\copy11\copy11\hfill } 
\vskip-12pt
\line{\punkt\copy7\copy7\copy7\copy7\copy7\copy7\copy7 \copy7\copy7\copy7\copy7\copy7\copy7\copy7
\copy7\copy7\copy7\copy7\copy7\hfill }
\line{\punkt\copy11\copy11\copy11\copy11\copy11\copy11 \copy11\copy11\copy11\copy11\copy11\box11\hfill }
\vskip-15pt
\line{\punkt\copy7\copy7\copy7\copy7\copy7\copy7\copy7 \copy7\copy7\copy7\copy7\copy7\copy7\copy7
\copy7\copy7\copy7\copy7\box7\hfill }

\vskip15pt plus4pt minus4pt

\setbox21=\hbox to 5.5mm{\hfill\vrule height0.4pt depth0pt
width2.75mm} 
\setbox17=\hbox to 3.5mm{\hfill\vrule height0.4pt
depth0pt width1.75mm}
\line{\punkt\copy21\copy21\copy21\copy21\copy21\copy21
\copy21\copy21\copy21\copy21\copy21\copy21\copy21\copy21
\copy21\copy21\copy21\copy21\copy21\copy21\copy21\copy21
\copy21
\copy21\hfill } \vskip-12pt
\line{\punkt\copy17\copy17\copy17\copy17\copy17\copy17
\copy17\copy17\copy17\copy17\copy17\copy17\copy17\copy17
\copy17\copy17\copy17\copy17\copy17\copy17\copy17\copy17
\copy17\copy17\copy17\copy17\copy17\copy17\copy17\copy17
\copy17\copy17\copy17\copy17\copy17\copy17\copy17\copy17
\hfill }
\line{\punkt\copy21\copy21\copy21\copy21\copy21\copy21
\copy21\copy21\copy21\copy21\copy21\copy21\copy21\copy21
\copy21\copy21\copy21\copy21\copy21\copy21\copy21\copy21
\copy21\box21\hfill } 
\vskip-15pt
\line{\punkt\copy17\copy17\copy17\copy17\copy17\copy17
\copy17\copy17\copy17\copy17\copy17\copy17\copy17\copy17
\copy17\copy17\copy17\copy17\copy17\copy17\copy17\copy17
\copy17\copy17\copy17\copy17\copy17\copy17\copy17\copy17
\copy17\copy17\copy17\copy17\copy17\copy17\copy17\box17
\hfill }
\vskip12pt plus4pt minus4pt

\setbox31=\hbox to 2.75mm{\hfill\vrule height0.4pt depth0pt
width1.375mm} 
\setbox27=\hbox to 1.75mm{\hfill\vrule height0.4pt
depth0pt width0.875mm}
\line{\punkt\copy31\copy31\copy31\copy31\copy31\copy31
\copy31\copy31\copy31\copy31\copy31\copy31\copy31\copy31
\copy31\copy31\copy31\copy31\copy31\copy31\copy31\copy31
\copy31\copy31\copy31\copy31\copy31\copy31\copy31\copy31
\copy31\copy31\copy31\copy31\copy31\copy31\copy31\copy31
\copy31\copy31\copy31\copy31\copy31\copy31\copy31\copy31
\copy31\copy31\copy31\hfill } 
\vskip-12pt
\line{\punkt\copy27\copy27\copy27\copy27\copy27\copy27
\copy27\copy27\copy27\copy27\copy27\copy27\copy27\copy27
\copy27\copy27\copy27\copy27\copy27\copy27\copy27\copy27
\copy27\copy27\copy27\copy27\copy27\copy27\copy27\copy27
\copy27\copy27\copy27\copy27\copy27\copy27\copy27\copy27
\copy27\copy27\copy27\copy27\copy27\copy27\copy27\copy27
\copy27\copy27\copy27\copy27\copy27\copy27\copy27\copy27
\copy27\copy27\copy27\copy27\copy27\copy27\copy27\copy27
\copy27\copy27\copy27\copy27\copy27\copy27\copy27\copy27
\copy27\copy27\copy27\copy27\copy27\copy27\copy27\hfill }
\line{\punkt\copy31\copy31\copy31\copy31\copy31\copy31
\copy31\copy31\copy31\copy31\copy31\copy31\copy31\copy31
\copy31\copy31\copy31\copy31\copy31\copy31\copy31\copy31
\copy31\copy31\copy31\copy31\copy31\copy31\copy31\copy31
\copy31\copy31\copy31\copy31\copy31\copy31\copy31\copy31
\copy31\copy31\copy31\copy31\copy31\copy31\copy31\copy31
\copy31\copy31\copy31\hfill } 
\vskip-15pt
\line{\punkt\copy27\copy27\copy27\copy27\copy27\copy27
\copy27\copy27\copy27\copy27\copy27\copy27\copy27\copy27
\copy27\copy27\copy27\copy27\copy27\copy27\copy27\copy27
\copy27\copy27\copy27\copy27\copy27\copy27\copy27\copy27
\copy27\copy27\copy27\copy27\copy27\copy27\copy27\copy27
\copy27\copy27\copy27\copy27\copy27\copy27\copy27\copy27
\copy27\copy27\copy27\copy27\copy27\copy27\copy27\copy27
\copy27\copy27\copy27\copy27\copy27\copy27\copy27\copy27
\copy27\copy27\copy27\copy27\copy27\copy27\copy27\copy27
\copy27\copy27\copy27\copy27\copy27\copy27\copy27\hfill }

\vskip12pt plus4pt minus4pt
\noindent On voit que les r\'egions rest\'ees blanches sont de plus en plus 
petites, mais en m\^eme temps de plus en plus nombreuses, de sorte que
la proportion de blanc sur la longueur totale (qui est la probabilit\'e pour
que $\theta_1$ et $\theta_2$ soient tous deux dans un intervalle pair) 
tend vers une limite.
\medskip
On peut calculer num\'eriquement cette limite (il n'y a pas de formule
analytique simple pour l'exprimer, mais on peut faire un petit programme
simple qui la calcule num\'eriquement); le tableau suivant en donne les
valeurs $x$ pour diff\'erentes valeurs de $\alpha$. \`A titre de
comparaison on a donn\'e entre parenth\`eses la valeur de $1/2\alpha$, 
qui \'etait la probabilit\'e pour que $\theta_1$ et $\theta_2$ soient 
tous deux dans le demi-intervalle $]0\, ,\, {\pi \over 2}[$.
$$\matrix{
&\alpha \quad &x  &\bigl( {1 \over 2\alpha�} \bigr) \qquad\qquad 
&\alpha \quad &x  &\bigl( {1 \over 2\alpha�} \bigr) \cr
\noalign{\medskip}
&3/2 \quad &0.25 &(0.333)\qquad\qquad  
&4/3 \quad &0.25 &(0.375)\cr
&5/3 \quad &0.2667 &(0.300)\qquad\qquad  
&5/4 \quad &0.25 &(0.400) \cr
&6/5 \quad &0.25 &(0.417)\qquad\qquad  
&7/5 \quad &0.2571 &(0.357) \cr
&7/6 \quad &0.25 &(0.428)\qquad\qquad  
&8/7 \quad &0.25 &(0.4375) \cr
&9/7 \quad &0.254 &(0.3889)\qquad\qquad  
&9/8 \quad &0.25 &(0.444) \cr
&10/7 \quad &0.25 &(0.350)\qquad\qquad  
&10/9 \quad &0.25 &(0.450) \cr
&11/8 \quad &0.25 &(0.3636)\qquad\qquad  
&11/9 \quad &0.2525 &(0.409)\cr
&11/10 \quad &0.25 &(0.454)\qquad\qquad  
&12/11 \quad &0.25 &(0.458)\cr
&13/10 \quad &0.25 &(0.3846)\qquad\qquad  
&13/11 \quad &0.2517 &(0.423)\cr 
&13/12 \quad &0.25 &(0.4615)\qquad\qquad 
&14/11 \quad &0.25 &(0.3928)\cr  
&14/13 \quad &0.25&(0.4643)\qquad\qquad  
&15/11 \quad &0.2515 &(0.3667) \cr
&15/13 \quad &0.2513 &(0.4333)\qquad\qquad  
&15/14 \quad &0.25 &(0.4667) \cr
&16/13 \quad &0.25 &(0.406)\qquad\qquad  
&16/15 \quad &0.25 &(0.469) \cr
&17/13 \quad &0.2511 &(0.382)\qquad\qquad  
&17/14 \quad &0.25 &(0.412) \cr
&17/15 \quad &0.2510 &(0.441)\qquad\qquad  
&17/16 \quad &0.25 &(0.471) \cr
&18/13 \quad &0.25 &(0.361)\qquad\qquad  
&18/17 \quad &0.25 &(0.472) \cr
&19/15 \quad &0.2509 &(0.395)\qquad\qquad  
&19/16 \quad &0.25 &(0.421) \cr
&19/17 \quad &0.2508 &(0.447)\qquad\qquad  
&19/18 \quad &0.25 &(0.4737) \cr
&129/113 \quad &0.25 &(0.438)\qquad\qquad  
&79/73 \quad &0.25 &(0.462) \cr
}$$
On voit que la valeur de $x$ est toujours proche de ${1 \over 4}$; le plus
souvent elle est m\^eme exactement ${1 \over 4}$ (sur le tableau les
valeurs marqu\'ees $0.25$ correspondent au cas o\`u $x$ est {\it
exactement} \'egal \`a ${1 \over 4}$; on aurait marqu\'e $0.2500$ si 
c'\'etait une valeur approch\'ee \`a $10^{-4}$ pr\`es). Le cas o\`u $x$ 
est diff\'erent de  ${1 \over 4}$ provient de contraintes li\'ees \`a la
divisibilit\'e des nombres entiers qui interviennent ici ($\alpha$ est
fractionnaire); on pourrait montrer math\'ematiquement que $x$ vaut
toujours ${1 \over 4}$ quand $\alpha$ est irrationnel, mais cela est sans
int\'er\^et pour la question qui nous pr\'eoccupe ici; il suffit de voir que
$x$ est presque toujours \'egal \`a ${1 \over 4}$, et, m\^eme quand il
n'est pas exactement \'egal \`a ${1 \over 4}$, en est toujours proche.
\medskip
Cela montre que l'ind\'ependance stochastique n'est pas li\'ee
\`a l'existence d'une ind\'ependance effective entre $\theta_1$ et
$\theta_2$,mais seulement \`a la finesse de la subdivision: avec la
m\^eme relation liant $\theta_1$ et $\theta_2$, on constate que la
propri\'et\'e du produit n'est pas v\'erifi\'ee lorsque l'intervalle de 0 \`a 180 
degr\'es est divis\'e en deux moiti\'es, et qu'il suffit de subdiviser
en un grand nombre d'intervalles pour qu'elle le devienne. Toutefois,
comme le montre le tableau pr\'ec\'edent, il y a des exceptions: pour
certaines valeurs de $\alpha$, la propri\'et\'e du produit n'est pas
exactement v\'erifi\'ee, ce qui refl\`ete qu'une certaine d\'ependance
stochastique subsiste; cela signifie simplement que le {\it brouillage} 
n'a pas \'et\'e suffisant pour cr\'eer un hasard parfait. Le m\'ecanisme
exact de ce r\'esidu de d\'ependance est une affaire d'arithm\'etique et 
de divisibilit\'e due \`a la p\'eriodicit\'e de la subdivision. Il suffirait
de
subdiviser en intervalles de longueur variable (mais comportant chacun
une moiti\'e paire et une moiti\'e impaire) pour  supprimer cette
p\'eriodicit\'e; et en effet, avec des subdivisions \`a pas variable, on
supprimerait les contraintes de divisibilit\'e et la valeur limite
$x$ serait alors toujours exactement ${1 \over 4}$. Le brouillage serait
alors devenu suffisant.  
\medskip 
Or, pour le point d'arriv\'ee de la bille, le brouillage est encore bien
plus consid\'erable: les 36 secteurs du disque ont une largeur de 10
degr\'es, ce qui est trop large pour que ce qui vient d'\^etre mis en
\'evidence se produise  avec l'angle  $\theta$, mais largement suffisant
pour que cela se  produise avec le point d'arriv\'ee de la bille, gr\^ace \`a
l'amplification  caract\'eristique du chaos d\'eterministe: si $\theta_2
= \alpha\theta_1$, le second point d'arriv\'ee de la bille sera une
fonction parfaitement d\'etermin\'ee du premier point d'arriv\'ee, mais
avec une variabilit\'e telle que cette fois le brouillage sera parfait:
toutes les contraintes de d\'ependance auront \'et\'e {\it effac\'ees} par
le tr\`es grand nombre de  rebondissements  de la bille sur le bord du
disque. La division du cercle en $36$ secteurs de $10$ degr\'es chacun
\'equivaut \`a une subdivision infiniment fine \`a cause de l'amplification; 
la transformation du d\'eterminisme en hasard est donc due \`a la finesse
effective de la subdivision. On peut ainsi deviner que le brouillage par
le chaos se ram\`ene toujours d'une mani\`ere ou d'une autre \`a 
raffiner des subdivisions.
\medskip 
Ce n'est bien s\^ur pas par hasard que la roulette, m\^eme
sousla forme d'un mod\`ele tr\`es simplifi\'e, illustre si bien tous 
les m\'ecanismes du hasard. Toutes nos discussions montrent que la
roulette est con\c cue de mani\`ere \`a accumuler la plus grande
vari\'et\'e possible de  {\it brouillages} dans un processus qui par
nature est d\'eterministe. On peut dire que la roulette est une
v\'eritable machine \`a effacer la causalit\'e. Un d\'e est aussi un
syst\`eme destin\'e \`a effacer la causalit\'e, mais moins sophistiqu\'e;
le roulement du d\'e, par exemple, a pour fonction de changer un grand
nombre de fois la face dirig\'ee vers le haut avant l'arr\^et, ce qui joue
le m\^eme r\^ole que les nombreuses r\'eflexions de la bille de la
roulette, \`a savoir cr\'eer l'amplification de la sensibilit\'e aux
conditions initiales. L'effacement des causes est d'autant plus radical
que les subdivisions sont plus fines compar\'ees \`a la variabilit\'e des
param\`etres; cette finesse relative peut \^etre renforc\'ee soit en
augmentant le nombre des subdivisions (c'est pourquoi la roulette
comporte 36 chiffres plut\^ot que 5, 7, ou 10; dans le cas du d\'e
cela reviendrait \`a faire des faces plus nombreuses et plus petites),
soit en augmentant l'amplification de la sensibilit\'e aux conditions
initiales (c'est pourquoi la roulette agence des mouvements tr\`es
chaotiques). Dans le cas d'une bille de roulette, la sensibilit\'e aux
conditions initiales du mouvement est telle que pour pouvoir pr\'edire
le point d'arriv\'ee \`a partir des lois de la M\'ecanique classique, il
faudrait contr\^oler la position et la vitesse initiales de la bille ainsi
que la forme exacte du plateau avec une pr\'ecision bien plus grande 
que la dimension des atomes, c'est-\`a-dire avec une pr\'ecision telle
que la M\'ecanique classique ne pourrait m\^eme plus \^etre tenue 
pour valide. 
\medskip 
L'ind\'ependance stochastique n'est donc pas l'expression exacte d'une
ind\'ependance causale r\'eelle et absolue;  elle r\'esulte d'un {\it
effacement} de la causalit\'e par des  proc\'ed\'es de brouillage 
ad\'equats (subdivisions fines et amplification). Ces proc\'ed\'es de
brouillage peuvent \^etre aussi  bien artificiels (roulette, d\'e, fonctions
{\bf random}, publication d'une information sur le Net, etc.) que  naturels
(agitation thermique des mol\'ecules,
combinaison des  g\`enes, dispersion des semences par le vent).

\vskip5mm plus3mm minus3mm

{\bf IV\ata .\aub 3. Probabilit\'es conditionnelles.}
\medskip

Lorsque deux \'ev\'enements $A$ et $B$ ne sont pas stochastiquement
in\-d\'e\-pen\-dants, c'est-\`a-dire lorsque ${\cal P}\, (AB) \neq 
{\cal P}\, (A)
\times {\cal P}\, (B)$, on peut introduire le rapport
$${\cal P}\, (A\mid B) = {{\cal P}\, (AB)\over {\cal P}\, (B)}\eqno(IV.1.)$$
\hskip12pt Si $A$ et $B$ \'etaient stochastiquement ind\'ependants, \hskip2pt 
on aurait bien s\^ur ${\cal P}\, (A \mid B) = {\cal P}\, (A)$. On dit que ce
rapport est la
{\bf probabilit\'e conditionnelle} de $A$ par rapport \`a $B$;dans le langage imag\'e, on dit aussi que c'est la probabilit\'e conditionnelle
de $A$ ``sachant que $B$ s'est produit''.  Le concept math\'e\-ma\-tique ainsi
introduit est \'evidemment cens\'e refl\'eter l'id\'ee sugg\'er\'ee par le
langage imag\'e,  \`a savoir que la probabilit\'e conditionnelle est la
pro\-ba\-bi\-lit\'e sur un ensemble restreint d'\'epreuves (celles qui 
v\'erifient une condition,  exprim\'ee par le langage imag\'e). 
\medskip
On peut aussi \'ecrire la relation $(IV.1.)$ sous la forme
$${\cal P}\, (A \mid B) = {\#(AB) \over \# B}$$
Si on la rapproche de $(I.2.)$ cela signifie que cette fois on consid\`ere 
l'\'ev\'enement $AB$ dans l'espace des \'epreuves $B$. En interpr\'etant, 
cela revient \`a dire qu'on traite le probl\`eme suivant: parmi l'ensemble
des \'epreuves \'equiprobables qui constituent l'\'ev\'enement $B$,
quelle est la proportion de celles qui satisfont en outre \`a
l'\'ev\'enement $A$? Autrement dit, il s'agit de la probabilit\'e pour que
$A$ se produise, mais sur l'espace restreint des \'epreuves qui
correspondent \`a la r\'ealisation de $B$. Cela exprime bien l'id\'ee de la
pro\-ba\-bi\-lit\'e conditionnelle de $A$, {\it sachant que} $B$ s'est produit.
L'ind\'ependance stochastique entre $A$ et une famille $B_i$ peut donc 
se traduire en disant que la probabilit\'e conditionnelle de $A$ sachant
que $B_i$ s'est produit, est la m\^eme pour tous les $B_i$.
\medskip
Du point de vue purement formel,  une probabilit\'e conditionnelle
est donc simplement une probabilit\'e sur un espace d'\'epreuves
restreint.  Dans n'importe quel probl\`eme de probabilit\'es,  on doit
construire un espace d'\'epreuves $\Omega$ adapt\'e \`a la r\'esolution 
du probl\`eme,  et nous voyons ici qu'on peut toujours consid\'erer
$\Omega$ comme une partie d'un espace plus gros,  les probabilit\'es
d\'efinies sur $\Omega$ \'etant alors des probabilit\'es conditionnelles
sur le plus grand espace.  Le meilleur choix de $\Omega$ est toujours le
plus petit,  mais il est peu commode de modifier $\Omega$ en cours de
probl\`eme.  Le meilleur $\Omega$ est donc le plus petit possible,  parmi
ceux qui contiennent tous les \'ev\'enements \'etudi\'es.
\medskip
Pour illustrer cela,  prenons une urne qui contient $p$ bulletins {\eightrm
OUI} et $q$ bulletins {\eightrm NON}, donc $n = p+q$ bulletins en tout. On
tire au hasard un bulletin; il est intuitivement \'evident que la 
probabilit\'e pour que ce bulletin porte {\eightrm OUI} est $p / n$, et la
probabilit\'e pour qu'il porte {\eightrm NON} est $q / n$; un calcul
rigoureusement d\'eriv\'e des principes aboutirait, heureusement, au 
m\^eme r\'esultat. Si on fait deux tirages successifs (sans 
remplacement), il est plus difficile de r\'epondre  imm\'ediatement avec
la seule perception intuitive, et le calcul rigoureusement d\'eriv\'e des
principes peut \^etre pr\'ef\'erable; il passe par le d\'enombrement
syst\'ematique: pour un seul tirage on pouvait prendre pour espace
$\Omega$ l'ensemble des bulletins de l'urne, pour deux tirages il faut
prendre l'ensemble des couples de bulletins (mais sans r\'ep\'etition).
D'apr\`es $II.2.$ son cardinal est $n(n-1)$. Le nombre de couples form\'es
de deux bulletins {\eightrm OUI} est $p(p-1)$, le nombre  de couples
form\'es de deux bulletins {\eightrm NON} est $q(q-1)$, et le nombre de
couples form\'es d'un bulletin {\eightrm OUI} et d'un bulletin  {\eightrm
NON}  est $pq$ (quel que soit l'ordre). Donc la probabilit\'e d'avoir deux
bulletins {\eightrm OUI} en deux tirages est $p(p-1) / n(n-1)$, celle
d'avoir  deux bulletins {\eightrm NON} est $q(q-1) / n(n-1)$, et celle
d'avoir un {\eightrm OUI} puis un {\eightrm NON} (ou l'inverse) est $pq /
n(n-1)$.  
\medskip
Il n'y a pas ind\'ependance stochastique entre les tirages (il y en aurait 
si on effectuait des tirages {\it avec} remplacement). Cela peut se
constater en introduisant les \'ev\'enements relatifs \`a chaque tirage,
tout comme  on avait introduit les \'ev\'enements relatifs \`a chaque
boule dans l'exemple discut\'e en $IV.1.$\looseness=-1
\medskip
On posera donc:
$$\eqalignno{
A_1 &: \hbox{``le premier bulletin est un {\eightrm OUI}''} \cr 
B_1 &: \hbox{``le premier bulletin est un {\eightrm NON}''} \cr 
A_2 &: \hbox{``le deuxi\`eme bulletin est un {\eightrm OUI}''} \cr 
B_2 &: \hbox{``le deuxi\`eme bulletin est un {\eightrm NON}''} \cr }$$
L'\'ev\'enement ``les deux bulletins sont des {\eightrm OUI} sera alors 
$A_1 A_2$, l'\'ev\'enement ``les deux bulletins sont des {\eightrm NON}
sera  $B_1 B_2$, l'\'ev\'enement ``le premier bulletin est un {\eightrm 
OUI}  et le deuxi\`eme est un {\eightrm NON}'' sera $A_1 B_2$, et
l'\'ev\'enement   ``le premier bulletin est un {\eightrm NON}  et le
deuxi\`eme est un {\eightrm OUI}'' sera $A_2 B_1$.
\medskip
On a ainsi, comme cela a \'et\'e dit plus haut:
$$\openup 2\jot\eqalignno{
{\cal P}\, (A_1) = {p \over n} \quad 
&\qquad {\cal P}\, (B_1)  = {q \over n} \cr  
\noalign{\medskip}
{\cal P}\, (A_1A_2)  &= {p(p-1) \over n(n-1)} \cr 
{\cal P}\, (A_1B_2)  &= {pq \over n(n-1)} \cr 
{\cal P}\, (B_1A_2)  &= {pq \over n(n-1)} \cr 
{\cal P}\, (B_1B_2)  &= {q(q-1) \over n(n-1)} \cr }$$
Les probabilit\'es ${\cal P}\, (A_2)$ et ${\cal P}\, (B_2)$ n'avaient pas
encore \'et\'e \'evoqu\'ees avant; mais on les obtient sans difficult\'e: 
$A_2 = A_2 A_1 \cup A_2 B_1$ et $A_1,\, B_1$ sont disjoints, donc 
$$\openup 2\jot\eqalignno{
{\cal P}\, (A_2)  &=  {\cal P}\, (A_2 A_1) + {\cal P}\, (A_2 B_1) = 
{p(p-1) \over n(n-1)} + {pq \over n(n-1)} = {p \over n} \cr  
{\cal P}\, (B_2)  &=  {\cal P}\, (B_2 A_1) + {\cal P}\, (B_2 B_1) = 
{pq \over n(n-1)} + {q(q-1) \over n(n-1)} = {q \over n} \cr  }$$
On peut voir que la r\`egle du produit n'est pas satisfaite:
$$\openup 2\jot\eqalignno{
{\cal P}\, (A_1) \times {\cal P}\, (A_2)  
&= {p \over n} \times {p \over n} \neq {p(p-1) \over n(n-1)} \cr 
{\cal P}\, (A_1) \times {\cal P}\, (B_2)  
&= {p \over n} \times {q \over n} \neq {pq \over n(n-1)} \cr 
{\cal P}\, (B_1) \times {\cal P}\, (A_2)  
&= {q \over n} \times {p \over n} \neq {pq \over n(n-1)} \cr 
{\cal P}\, (B_1) \times {\cal P}\, (B_2)  
&= {q \over n} \times {q \over n} \neq {q(q-1) \over n(n-1)} \cr }$$
\medskip
Calculons alors les probabilit\'es conditionnelles ${\cal P}\, (A_2 \mid
A_1)$, ${\cal P}\, (B_2 \mid A_1)$, ${\cal P}\, (A_2 \mid B_1)$, et 
${\cal P}\, (B_2 \mid B_1)$. On trouve 
$$\openup 2\jot\eqalignno{
{\cal P}\, (A_2 \mid A_1) &= {{\cal P}\, (A_2  A_1) \over {\cal P}\, 
(A_1)} = {p(p-1) / n(n-1) \over p / n} = {p-1 \over n-1} \cr 
{\cal P}\, (A_2 \mid B_1) &= {{\cal P}\, (A_2  B_1) \over {\cal P}\, 
(B_1)} = {pq / n(n-1) \over q / n} = {p \over n-1} \cr 
{\cal P}\, (B_2 \mid A_1) &= {{\cal P}\, (B_2  A_1) \over {\cal P}\, 
(A_1)} = {pq / n(n-1) \over q / n} = {q \over n-1} \cr 
{\cal P}\, (B_2 \mid B_1) &= {{\cal P}\, (B_2  B_1) \over {\cal P}\, 
(B_1)} = {q(q-1) / n(n-1) \over q / n} = {q-1 \over n-1} \cr }$$
Cette suite de calculs \'etait fastidieuse, quoique facile, mais 
maintenant nous pouvons {\it interpr\'eter}.
\medskip
Par exemple ${\cal P}\, (A_2 \mid A_1)$, la probabilit\'e conditionnelle
pour que $A_2$ se produise sachant que $A_1$ s'est produit, est dans le
langage imag\'e ``la probabilit\'e pour que le second bulletin tir\'e soit 
un  {\eightrm OUI}, sachant que le premier \'etait un {\eightrm OUI}''. Si 
on raisonne selon l'intuition premi\`ere, on dira que si le premier bulletin 
tir\'e  a \'et\'e un  {\eightrm OUI} il reste dans l'urne $p-1$ {\eightrm 
OUI} et $q$ {\eightrm NON}; donc la probabilit\'e de tirer un bulletin
{\eightrm OUI} la deuxi\`eme fois est toujours le rapport du nombre de
bulletins {\eightrm OUI} au nombre total de bulletins, soit $(p-1)\; /\;
(n-1)$.  De m\^eme, si le premier bulletin tir\'e a \'et\'e un {\eightrm
NON}, il reste dans l'urne $n-1$ bulletins, dont $p$ {\eightrm OUI} et
$q-1$ {\eightrm NON}, de sorte que la probabilit\'e de tirer un {\eightrm
OUI} est cette fois $p \; /\;  (n-1)$.
\medskip
L'intuition premi\`ere que tout le monde peut avoir a priori ne permet pas 
de voir imm\'ediatement et de mani\`ere \'evidente quelle doit \^etre la
valeur de probabilit\'es telles que ${\cal P}\, (A_1 A_2)$, ${\cal P}\, 
(A_1 B_2)$, ${\cal P}\, (B_1 A_2)$, et ${\cal P}\, (B_1 B_2)$; pour les
trouver, il a fallu d\'enombrer les couples de bulletins en utilisant $II.2.$
Par contre il \'etait imm\'ediatement \'evident que les probabilit\'es
${\cal P}\, (A_1)$ et ${\cal P}\, (B_1)$ valent respectivement $p/n$ et
$q/n$. Il  y a dans tous les domaines des choses qu'on peut percevoir par
intuition imm\'ediate et d'autres qu'on ne peut que trouver par un
raisonnement complexe, c'est-\`a-dire compos\'e de plusieurs
d\'eductions dont une seule \`a la fois peut \^etre reconnue comme
imm\'ediatement \'evidente, et il en est \'egalement ainsi en Calcul des
proba\-bi\-lit\'es.  Or il est aussi imm\'ediatement \'evident que la
probabilit\'e pour que le second bulletin tir\'e soit un {\eightrm OUI} {\it
sachant que le premier \'etait un} {\eightrm OUI} est $(p-1)\; / \; (n-1)$,
puisque cela r\'esulte directement, tout comme le calcul de  ${\cal P}\,
(A_1)$,  du fait que le nombre de bulletins {\eightrm OUI} pr\'esents dans
l'urne est $p-1$. Pour d\'enombrer tous les couples possibles de
bulletins,  il faut d\'ej\`a faire appel \`a  des notions math\'ematiques
telles que le produit cart\'esien de deux ensembles, qui sont abstraites 
et ne rel\`event pas de l'\'evidence imm\'ediate.  Il faut en effet {\it
d\'eduire} le nombre de couples possibles \`a partir du nombre donn\'e de
bulletins,  alors que relier la probabilit\'e et le quotient de $p$ par $n$
(ou de $p-1$ par $n-1$) rel\`eve de l'intuition premi\`ere.   
\medskip
Ci-dessus,  nous avons pr\'esent\'e le calcul de ${\cal P}\, (A_2 \mid 
A_1)$  de telle fa\c con qu'il se d\'eduise, par la d\'efinition
math\'ematique $(IV.1.)$, du calcul pr\'ealable, et non \'evident, de 
${\cal P}\, (A_2 A_1)$. En proc\'edant ainsi, nous retrouvons un r\'esultat 
qui aurait pu \^etre obtenu directement par l'intuition premi\`ere. Cette
fa\c con de proc\'eder est didactiquement int\'eressante, car elle montre 
que la th\'eorie marche. On fait souvent cela dans l'enseignement:
retrouver par une voie indirecte et compliqu\'ee un r\'esultat qui \'etait
\'evident a priori. C'est tr\`es mauvais pour calculer, mais tr\`es bon pour
comprendre. En effet, lorsqu'on expose une th\'eorie \'elabor\'ee, sous
forme axiomatique, les \'etudiants ont le sentiment d'un savoir qui tombe
du ciel. Mais les math\'ematiciens du pass\'e qui ont cr\'e\'e de toutes
pi\`eces ces concepts math\'ematiques (espaces d'\'epreuves,
ind\'ependance stochastique, proba\-bi\-lit\'es conditionnelles, etc.) sont
partis de l'intuition premi\`ere et ont construit peu \`a peu la th\'eorie
par t\^atonnements. Le crit\`ere essentiel de validit\'e qui guidait les
cr\'eateurs \'etait que la bonne th\'eorie serait celle qui, dans tous les
cas o\`u l'intuition premi\`ere permet de calculer un r\'esultat,
concorderait avec cette intuition premi\`ere. En Physique, il faut que la
th\'eorie soit confirm\'ee par l'exp\'erience: chaque fois que la th\'eorie
permet de calculer un r\'esultat et l'exp\'erience de le mesurer, les deux
r\'esultats doivent concorder. Il en va de m\^eme pour le Calcul des
proba\-bi\-lit\'es, sauf que ce qui joue le r\^ole de l'exp\'erience est
l'intuition premi\`ere.  En effet,  le Calcul des proba\-bi\-lit\'es n'est
pas une
science exp\'erimentale;  on ne {\it mesure} jamais lesproba\-bi\-lit\'es:
 c'est, comme pour la g\'eom\'etrie, \`a partir  desym\'etries
fondamentales qu'on postule l'\'equiprobabilit\'e d'unensemble
d'\'epreuves (voir chapitre $I$).  Si dans beaucoup
d'applications du Calcul des probabilit\'es on recoupe les faits 
exp\'erimentaux,  c'est parce que les sym\'etries fondamentales (de
l'espace-temps ou des particules quantiques,  par exemple) sont d'origine
exp\'erimentale.  Mais \`a l'int\'erieur du Calcul des proba\-bi\-lit\'es
ces sym\'etries fondamentales sont postul\'ees et on en d\'eduit des {\it
probabilit\'es a priori}.  C'est pourquoi le r\^ole jou\'e par l'intuition
premi\`ere est absolument essentiel et constitue le v\'eritable
fondementdu Calcul des proba\-bi\-lit\'es.  Si la d\'efinition $(IV.1.)$ permet
de retrouver,  par l'application rigoureuse et abstraite des principes de
base,  ce que l'intuition tient pour \'evident,  en l'occurrence que ${\cal
P}\, (A_2   \mid A_1) = (p-1)\; / \; (n-1)$,  c'est un signe que la th\'eorie
est bonne.  Mais  s'il en est ainsi, c'est bien entendu parce que tout avait
\'et\'e fabriqu\'e pour qu'il en soit ainsi:  le choix du mod\`ele $\Omega$ 
et l'\'equiprobabilit\'e des \'epreuves que ce choix sous-entend contient 
{\it en germe} le r\'esultat.  La th\'eorie a \'et\'e faite expr\`es pour 
qu'il en soit ainsi.   
\medskip  
Cela dit,  si retrouver par la th\'eorie un r\'esultat qui \'etait \'evident
a priori est conceptuellement instructif,  on peut aussi utiliser le chemin
inverse dans un but strictement pragmatique.  Supposons que nous
voulions calculer ${\cal P}\, (A_2 A_1)$ (dont la valeur ne se voit pas
imm\'ediatement \`a partir de l'intuition  premi\`ere).  Au lieu de
proc\'eder par d\'enombrement comme nous l'avons fait,  on pourrait
aussi suivre le chemin inverse:  il est {\it \'evident} que ${\cal P}\, (A_1)
= p/n$; il est {\it \'evident aussi} que ${\cal P}\, (A_2 \mid A_1) =
(p-1)\; / \; (n-1)$. En inversant la d\'efinition $(IV.1.)$ on obtient 
$${\cal P}\, (A_2 A_1) = {\cal P}\, (A_2 \mid A_1) \times {\cal
P}\, (A_1) \eqno (IV.2.)$$  
et on en d\'eduit alors que ${\cal P}\, (A_2 A_1) = {\struta p
\over\ldown{\scriptstyle n}} \cdot {\struta p-1 \over\lldown{
\scriptstyle n-1}}$, c'est-\`a-dire la m\^eme chose que ce qu'on  avait
obtenu  par d\'enombrement direct.  
\medskip 
Cet exemple montre l'int\'er\^et pratique des probabilit\'es 
conditionnelles. En pratique $(IV.2.)$ est plus int\'eressant et plus
fr\'equemment utilis\'e que $(IV.1.)$, car il arrive souvent, du moins 
dans les probl\`emes de niveau \'el\'ementaire, que les probabilit\'es
conditionnelles soient plus ``\'evidentes'' ou en tous cas plus faciles \`a
calculer que celles des intersections, et les probabilit\'es des
intersections d'\'ev\'enements s'obtiennent alors par produits successifs.   
\medskip
On remarquera aussi comment nous avons calcul\'e ${\cal P}\, (A_2)$ et 
${\cal P}\, (B_2)$. Contrairement aux probabilit\'es ${\cal P}\, (A_1)$ et 
${\cal P}\, (B_1)$, il n'y a aucune \'evidence imm\'ediate, car on ignore 
ce qui s'est pass\'e au premier tirage et par cons\'equent on ne conna{\^\i}t
pas le nombre de bulletins {\eightrm OUI} pr\'esents dans l'urne au
moment o\`u on effectue le second tirage.  Lorsqu'on avait \`a \'evaluer
${\cal P}\, (A_2 \mid A_1)$ ou ${\cal P}\, (A_2 \mid B_1)$ on pouvait
conna{\^\i}tre le nombre de bulletins {\eightrm OUI} pr\'esents dans 
l'urne car on se pla\c cait dans l'hypoth\`ese o\`u le premier tirage 
\'etait donn\'e; mais si on veut conna{\^\i}tre la probabilit\'e d'avoir
{\eightrm OUI} au second tirage sans savoir ce qui s'est produit au
premier,  il n'y a plus d'argument imm\'ediatement \'evident.  L'intuition
premi\`ere peut donner le sentiment vague que la probabilit\'e d'avoir
{\eightrm OUI} au deuxi\`eme tirage doit \^etre une sorte de moyenne sur
les deux possibilit\'es qui peuvent se produire au premier tirage. Mais
pour donner un contenu quantitatif clair et solide \`a ce sentiment vague,
il faut un raisonnement math\'ematique plus sophistiqu\'e, comme celui
que nous avons effectivement suivi: ce raisonnement consiste \`a
d\'ecomposer l'\'ev\'enement $A_2$ en deux  parties disjointes $A_2
A_1$ et $A_2 B_1$ dont on peut calculer s\'epar\'ement les
probabilit\'es; alors $\# A_2 = \# (A_2 A_1) + \#   (A_2 B_1)$, ou ${\cal
P}\, (A_2) =  {\cal P}\, (A_2 A_1) + {\cal P}\, (A_2 B_1)$. Mais si on
utilise $(IV.2.)$ on peut \'ecrire cela sous la forme 
$${\cal P}\, (A_2) = {\cal P}\, (A_2 \mid A_1) \cdot {\cal P}\, (A_1)
 + {\cal P}\, (A_2 \mid B_1) \cdot {\cal P}\, (B_1) \eqno (IV.3.)$$
formule qui exprime effectivement la probabilit\'e de $A_2$ comme une
moyenne  des probabilit\'es conditionnelles sur les deux possibilit\'es 
qui peuvent se produire au premier tirage; les coefficients qui
interviennent  dans ce calcul de moyenne sont les probabilit\'es
respectives de chacune  de  ces deux possibilit\'es.
\medskip
L'int\'er\^et de cette formule est qu'elle permet de calculer une 
probabilit\'e non \'evidente a priori, par composition de probabilit\'es
\'evidentes. Mais elle n'est rien d'autre qu'une mani\`ere un peu
sophistiqu\'ee de r\'ep\'eter que la probabilit\'e de la r\'eunion de deux
\'ev\'enements disjoints est la somme des probabilit\'es de chacun. 
\medskip
Cette formule $(IV.3.)$ peut d'ailleurs \^etre g\'en\'eralis\'ee comme suit. 
On d\'ecoupe $\Omega$ en $r$ morceaux disjoints $E_1, \, E_2, \, E_3, 
\ldots E_r$ (une telle famille d'\'ev\'enements disjoints et dont la 
r\'eunion  est $\Omega$ tout entier est appel\'ee une {\it famille
exhaustive d'\'ev\'enements}). Soit $A$ un \'ev\'enement quelconque. 
Alors on peut dire que les \'ev\'enements  $A E_1, \, A E_2, \, A E_3,
\ldots A E_r$ sont \'egalement disjoints et que leur r\'eunion est $A$.  
On en d\'eduit que ${\cal P}\, (A) = {\cal P}\, (A E_1) + {\cal P}\, (A E_2) 
+ {\cal P}\, (A E_3) +  \cdots  + {\cal P}\, (A E_r)$, ou encore
$$\eqalign{
{\cal P}\, (A) = {\cal P}\, (A \mid E_1) \cdot {\cal P}\, (E_1) &+
{\cal P}\, (A \mid E_2) \cdot {\cal P}\, (E_2) + \cr
&+ \cdots  + {\cal P}\, (A \mid E_r)  \cdot {\cal P}\, (E_r) \cr }
\eqno (IV.4.)$$
Ceci est vrai quel que soit l'\'ev\'enement $A$ et quelle que soit la
famille exhaustive $E_1, \, E_2, \, E_3, \ldots E_r$, mais pour en tirer 
un  r\'esultat int\'eressant dans un probl\`eme donn\'e il faut bien s\^ur 
choisir astucieusement les $E_j$.
\medskip
Nous verrons une application importante de cette formule au chapitre
{\bf VIII} (Processus en cascade): voir probl\`eme $N^\circ 2$. Nous en
discuterons une autre application (\`a la g\'en\'etique) dans la
section suivante ({\bf IV.4}). On peut pratiquer le Calcul des
probabilit\'es sans jamais y recourir, en d\'ecomposant l'\'ev\'enement
$A$ en la r\'eunion disjointe des $AE_j$.  Mais la discussion ci-dessus
montre que les probabilit\'es  conditionnelles sont souvent plus
commodes ou plus \'evidentes que celles des intersections: c'est ce qui
rend cette formule tr\`es pratique dans beaucoup de cas.
\medskip
Plus haut nous avons justifi\'e l'introduction des probabilit\'es
conditionnelles par un argument de commodit\'e: dans l'exemple des 
bulletins qu'on sort de l'urne, la valeur de ${\cal P}\, (A_2)$ n'\'etait pas
``\'evidente a priori'', tandis que ${\cal P}\, (A_1)$ et  ${\cal P}\, (A_2
\mid A_1)$ l'\'etaient.
\medskip
Pour {\it vraiment} comprendre les probabilit\'es conditionnelles il est
utile de revenir sur les raisons de cette \'evidence. D\`es le d\'ebut 
de cet ouvrage, nous avons appris que pour calculer une probabilit\'e il
fallait op\'erer de la mani\`ere suivante:
\smallskip
a) trouver l'ensemble $\Omega$ de toutes les \'epreuves \'equiprobables
possibles; 
\smallskip
b) traduire l'\'ev\'enement $A$ dont on cherche la probabilit\'e en terme 
de sous-ensemble de $\Omega$.
\smallskip
c) d\'enombrer $\Omega$ et $A$.
\medskip
Les probabilit\'es \'evidentes sont celles o\`u le quotient $\#A\; / 
\;\#\Omega$ est tout de suite \'evident, sans  qu'on ait besoin de
r\'efl\'echir pour  effectuer a), b), c). Il en est ainsi pour la probabilit\'e
de tirer  un bulletin {\eightrm OUI} dans une urne qui en contient $p$, 
sur $n$ bulletins en tout, car on voit tout de suite que $\#\Omega
=  n$ et $\#A = p$. Les cardinaux sont \'evidents avant m\^eme d'avoir
r\'efl\'echi \`a la nature des \'el\'ements. Lorsque les \'epreuves sont 
des mots de $n$ lettres, des permutations  d'un ensemble, ou des choses
encore plus compliqu\'ees, il faut d'abord  un raisonnement abstrait (les
\'etapes a et b ci-dessus) pour comprendre  que $\Omega$ est bien
l'ensemble des mots de $n$ lettres. Ensuite il faut appliquer une des
formules de d\'enombrement  du chapitre {\bf II}. Ces formules ne 
peuvent pas \^etre  devin\'ees en vitesse, elles font partie des
connaissances accumul\'ees par les math\'ematiciens au cours des
si\`ecles. En outre elles ne suffisent pas \`a elles seules et leur
application ne constitue que l'\'etape c): auparavant il faut avoir fait le
travail de mod\'elisation, qui constitue les \'etapes a) et b).  M\^eme 
dans le cas  tr\`es simple de $A_1A_2$ (tirer deux fois de suite un
{\eightrm OUI}), il faut r\'efl\'echir un peu, et abstraitement, pour
comprendre  que $\Omega$   est l'ensemble de tous  les tirages de deux
bulletins  distincts, et qu'il  rel\`eve du deuxi\`eme cas  de
d\'enombrement ({\bf II.2.} tirages sans remise) d'o\`u $\#\Omega =
n(n-1)$; il faut  encore recourir une seconde fois au m\^eme type de
raisonnement abstrait pour $\#(A_1A_2) = p(p-1)$. C'est pourquoi une
personne  n'ayant jamais \'etudi\'e le Calcul des probabilit\'es et \`a
peine fait de math\'ematiques dans sa vie  pourra  deviner la valeur de
${\cal P}\, (A_1)$, mais pas celle de ${\cal P}\, (A_1A_2)$; et encore
moins celle  de ${\cal P}\, (A_2)$, car pour calculer cette derni\`ere il
faut en outre  la d\'ecomposition ensembliste $A_2 =  A_1A_2 \cup
B_1A_2$.  
\medskip
Or la m\^eme personne n'ayant jamais \'etudi\'e le calcul des
probabilit\'es comprendra ais\'ement que si le premier bulletin tir\'e 
\'etait un {\eightrm OUI}, la probabilit\'e de tirer un {\eightrm OUI} une
seconde fois est $(p-1)/(n-1)$. Les \'etapes a) et b) sont ici aussi
inutiles, car il saute aux yeux que pour ce probl\`eme l'ensemble des
\'epreuves  est l'ensemble  des bulletins restants, et l'\'ev\'enement
l'ensemble des bulletins  {\eightrm OUI} restants. Si on consid\`ere ce
sous-probl\`eme: ``on a d\'ej\`a tir\'e un bulletin {\eightrm OUI} au
premier tirage, quelle est la probabilit\'e d'en tirer un au second tirage''
et qu'on veut malgr\'e tout passer par  a) et b), on prendra pour $\Omega$
l'ensemble des $n-1$ bulletins restant apr\`es le premier tirage, et pour
$A$ l'ensemble des $p-1$ bulletins {\eightrm OUI} restant apr\`es le 
premier tirage: c'est en tous cas la d\'emarche implicite dans le
raisonnement na\"\i f.  Toutefois, si on veut co\^ute que co\^ute
proc\'eder selon les r\`egles formelles  a), b), c), il  vaut mieux prendre
en compte pour ce premier  tirage tous les cas possibles, ce qui revient
\`a prendre pour $\Omega$ l'ensemble de tous les tirages de deux
bulletins {\it dont le premier} est {\eightrm OUI}, et pour $A$
l'ensemble de tous les tirages de deux bulletins qui sont tous deux des
{\eightrm OUI}; autrement dit, on prend pour espace des \'epreuves
$A_1$ et pour  \'ev\'enement $A = A_1A_2$. Compar\'es
(respectivement) \`a l'espace de tous les tirages de deux bulletins,  et
\`a l'\'ev\'enement ``le second est un {\eightrm OUI}'', ces deux
ensembles s'obtiennent en ne retenant que les \'epreuves qui satisfont
la condition ``le premier tirage est {\eightrm OUI}''.  
\medskip 
Dans ce probl\`eme ``r\'eduit'', o\`u l'espace des \'epreuves est $A_1$ et
l'\'ev\'e\-ne\-ment $A_1 A_2$, la probabilit\'e s'obtient toujours selon 
la formule {\it cardinal de l'\'ev\'enement} sur {\it cardinal de l'espace},
mais cela devient alors exactement ce que dit $IV.1$.
\medskip
Le but de ces remarques est le suivant: il n'y a absolument aucune
diff\'erence de nature entre une probabilit\'e conditionnelle et une
probabilit\'e tout court. On ne parle de probabilit\'e conditionnelle que
si on se situe dans un probl\`eme donn\'e, et qu'on y distingue un
sous-probl\`eme. N'importe quel probl\`eme de probabilit\'e peut
toujours \^etre consid\'er\'e comme un sous-probl\`eme d'un probl\`eme
plus vaste, de sorte qu'on puisse le traiter en termes de probabilit\'es
conditionnelles. Par  exemple, dans le probl\`eme des bulletins on peut
toujours supposer qu'avant le premier tirage avait eu lieu un tirage
$N^\circ 0$. La probabilit\'e $p/n$ de tirer un bulletin {\eightrm OUI} au
premier tirage est identique \`a la probabilit\'e conditionnelle de tirer 
ce bulletin  {\eightrm OUI} sachant qu'au tirage $N^\circ 0$ on avait tir\'e
un  {\eightrm NON}, et qu'il y avait alors $n+1$ bulletins en tout dont $p$
{\eightrm OUI}, ou \`a la probabilit\'e conditionnelle de tirer ce bulletin 
{\eightrm OUI} sachant qu'au tirage $N^\circ 0$ on avait tir\'e un 
{\eightrm OUI}, et qu'il y avait alors $n+1$ bulletins en tout dont $p+1$
{\eightrm OUI}.

\bigskip

{\bf IV. 4. Relativit\'e du hasard.}
\medskip
Ce constat concernant la nature essentiellement relative des
probabilit\'es conditionnelles a des cons\'equences pratiques. Jusqu'ici, 
nous avons toujours rencontr\'e des probl\`emes o\`u la premi\`ere 
chose \`a faire \'etait de chercher les invariances afin de d\'eterminer 
a priori ce qui est \'equiprobable. (en langage imag\'e ``le niveau o\`u
intervient le hasard pur''). Dans le probl\`eme longuement discut\'e
ci-dessus, o\`u on tirait un bulletin au hasard dans  une urne, il \'etait
ais\'e de postuler que tous les bulletins sont \'equiprobables, parce que
nous savions que nous tirions des bulletins individuels. Transformons un
peu l'exp\'erience: imaginons  qu'\`a l'int\'erieur de l'urne se trouvent, 
non des bulletins, mais des  d\'es: une partie des d\'es est blanche, les
autres  sont rouges; disons $p$ d\'es blancs et $q$ d\'es rouges, le 
nombre  total de d\'es \'etant $n = p + q$. Les d\'es rouges ont une face
gagnante:  si un d\'e rouge est tir\'e au sort,  et qu'en le faisant rouler 
il marque ``six'',  je gagne. Mais les d\'es blancs n'ont aucune face
gagnante. Il est ais\'e  de d\'eterminer l'espace des \'epreuves $\Omega$
correspondant, c'est l'ensemble de toutes les faces  de d\'es possibles,
dont le nombre est $6n$: $\#\Omega = 6n$. L'\'ev\'enement $A$ ``je 
gagne'' est l'ensemble  des faces gagnantes, c'est-\`a-dire l'ensemble 
des faces ``six'' de d\'es rouges, soit $q$ faces en tout. La probabilit\'e 
de gagner est donc $\#A\;  /\;\#\Omega = q / 6n$. 
\medskip
Nous avons ainsi
raisonn\'e une fois de plus  par \'equiprobabilit\'e a priori, comme nous le faisons depuis le d\'ebut. Ici, le ``niveau o\`u
intervenait le hasard pur'' \'etait le choix d'une face de d\'e parmi $6n$
faces \'equiprobables.
\medskip
Mais supposons que ce niveau  nous reste cach\'e; par exemple, supposons
que nous ignorions ce qui se trouve \`a l'int\'erieur de l'urne, que le 
tirage d'un d\'e se fasse \`a l'int\'erieur de l'urne et \`a l'abri des 
regards, soit  par un m\'ecanisme cach\'e, soit par un g\'enie qui habite
l'urne. Chaque fois que nous frottons l'urne, le g\'enie tire au hasard un
d\'e: si ce d\'e est rouge, le g\'enie le jette hors de l'urne puis nous (qui
sommes \`a l'ext\'erieur) le lan\c cons; si le d\'e est blanc, le g\'enie ne
se manifeste pas et rien ne sort de l'urne. 
\medskip 
Dans cette seconde version de l'exp\'erience il nous serait difficile de
deviner comment intervient le hasard pur. En effet, rien ne nous dit que 
la d\'ecision du g\'enie de lancer un d\'e hors de l'urne, provient d'un
tirage entre des d\'es \'equiprobables; nous ignorons qu'il y a aussi des 
d\'es blancs; le constat que les d\'es qui sortent sont toujours rouges  
ne  permet aucune d\'eduction. Certes, nous voyons bien comment
intervient le hasard dans la partie visible du processus: lorsque nous
lan\c cons le  d\'e que le g\'enie nous a jet\'e, nous sommes assur\'es 
que ses six faces sont \'equiprobables; mais la d\'ecision du g\'enie de
lancer un d\'e ou non,  peut \^etre attribu\'ee \`a n'importe quelle forme
de hasard.   Et nous n'avons aucun moyen de  savoir a priori quelle va
\^etre la probabilit\'e pour qu'il lance un d\'e hors de l'urne; nous ne
pouvons que  la mesurer statistiquement: en frottant un tr\`es grand
nombre de fois l'urne, on peut par exemple constater empiriquement 
que dans $30\%$  des cas, un d\'e est jet\'e hors de l'urne. 
\medskip
Le point essentiel est alors le suivant: malgr\'e notre ignorance de la 
nature du ph\'enom\`ene, nous pouvons calculer la probabilit\'e de gagner 
en utilisant la formule $IV.4$, \`a condition d'y introduire les
probabilit\'es empiriques mesur\'ees.  Soit en effet $S$ l'\'ev\'enement
``un d\'e sort de l'urne'', $R$ l'\'ev\'enement ``rien  ne sort'', et $G$
l'\'ev\'enement ``je gagne''. En utilisant $IV.4$, on peut dire que   
$${\cal P}\, (G) = {\cal P}\, (G\mid S) \cdot {\cal P}\, (S) + 
{\cal P}\, (G\mid R) \cdot {\cal P}\, (R) =
{\up{1} \over 6} \cdot 0.3 + 0 \cdot 0.7 = 0.05$$
Le raisonnement a priori aurait donn\'e $q/6n$, donc (pour $q/n = 0.3$),
le m\^eme r\'esultat.  Du moment que les probabilit\'es ${\cal P}\,
(S)$ et ${\cal P}\, (R)$ sont connues (peu importe que ce soit a priori
ou empiriquement),  la probabilit\'e ${\cal P}\, (G)$ est d\'etermin\'ee
{\it ind\'ependamment de la mani\`ere dont le hasard est intervenu dans
l'urne}.  Si par exemple le choix al\'eatoire effectu\'e \`a notre insu par 
le g\'enie dans l'urne \'etait d'une grande complexit\'e,  de sorte que
le ``v\'eritable'' espace $\Omega$ aurait un cardinal \'enorme,  que le 
quotient $p/n$ serait une fraction compliqu\'ee (c'est-\`a-dire que $p$
et $n$ seraient tous deux de tr\`es grands entiers),  mais proche de $0.3$,
le calcul ci-dessus, quoique approch\'e, serait tout aussi correct.
Math\'ematiquement,  la formule $IV.4$ se {\it d\'emontre} \`a partir de
l'existence d'un espace des \'epreuves.  Mais, une fois fix\'ees les
probabilit\'es  ${\cal P}\, (E_j)$, la probabilit\'e ${\cal P}\, (A)$ sera
d\'etermin\'ee ind\'e\-pen\-dam\-ment de la nature de l'espace 
$\Omega$. On peut m\^eme oublier qu'il existe quelque part un niveau
o\`u intervient le pur hasard.  Si toutefois ce niveau n'{\it existe 
r\'eellement pas},  alors la formule $IV.4$ peut perdre toute validit\'e.
C'est ce qui se produit dans la M\'ecanique quantique,  lorsque des
amplitudes interf\`erent (voir une discussion sur ce point \`a la section
{\bf XIII.3}). 
\medskip
Ainsi la formule $IV.4$ est utilisable dans les cas o\`u les
invariances ne sont pas int\'egralement connues,  de sorte qu'une partie
seulement des probabilit\'es peuvent \^etre calcul\'ees a priori et que 
les autres,  d'origine empirique,  ne peuvent \^etre d\'etermin\'ees 
qu'a posteriori par des mesures statistiques.
\medskip
Autrement dit, la formule $IV.4$ permet de remplacer une partie des
probabilit\'es a priori par des probabilit\'es empiriques.
Cet exp\'edientest in\'evitable lorsqu'on ne peut pas acc\'eder au niveau o\`u intervient
une invariance (cf. la discussion de la section {\bf I.2}).

\bigskip

Nous allons imm\'ediatement appliquer cela \`a un probl\`eme de 
g\'en\'etique, celui des mariages consanguins. Imaginons que dans une
population on isole un couple de parents, de sorte que les enfants de ces
parents se reproduisent ensuite entre eux. Nous nous int\'eressons
\`a la mani\`ere dont un g\`ene particulier, $G$, se combinera \`a ses
compagnons all\`eles du m\^eme locus (voir {\bf III.5} et {\bf III.6}).
Nous supposons d'abord que la combinaison homozygote $GG$ n'est pas
invalidante et que les individus $GG$ acc\`edent autant que les autres 
\`a la procr\'eation.
\medskip
Plusieurs cas de figure peuvent se pr\'esenter.  Chacun des deux parents
peut appartenir \`a l'une des quatre cat\'egories $XX$, $XG$, $GX$, $GG$.
La distinction entre $XG$ et $GX$ ne correspond \`a aucune diff\'erence
biologique entre les individus porteurs,  mais repr\'esente le choix qui
s'op\`ere entre les deux chromosomes au moment de la f\'econdation: 
dire que le p\`ere est $XG$ signifie que c'est le chromosome
porteur de $X$ qui sera retenu pour constituer une nouvelle paire: 
la
premi\`ere des deux lettres correspond au chromosome retenu.  Ainsi il 
y aura seize cat\'egories pour le couple.  Si les parents sont tous deux
$GX$,  les enfants seront tous $GG$,  et par cons\'equent si ces enfants 
se croisent ensuite entre eux les petits enfants le seront aussi;  si les
parents sont $XG$ et $GX$ (ou inversement),  aucun enfant ne sera $GG$
mais un enfant sur deux sera $GX$ (les autres seront alors $XG$).  Il y 
aura alors une chance sur quatre que les petits-enfants soient $GG$.  Si 
les parents sont $XX$ et $GX$ (ou $XX$ et $XG$),  les enfants seront aussi
pour moiti\'e $XG$ et pour l'autre moiti\'e $XG$,  et donc les
petits-enfants auront aussi une chance sur quatre d'\^etre $GG$.  En
regroupant syst\'ematiquement les diff\'erents cas qui se  pr\'esentent 
on peut distinguer les trois \'ev\'enements suivants,  selon la 
cat\'egorie des parents:  
\medskip
\centerline{\vbox{
\halign{#\quad &#\cr�
 $E_1$:  &$GG + GG$, $GG + GX$, $GX + GG$, ou $GX + GX$ \cr
\noalign{\smallskip}
 $E_2$:   &$GG + XX$, $GG + XG$, $XG + GG$, $XX + GG$, \cr
              &$GX + XX$, $GX + XG$, $XG + GX$, ou $XX + GX$ \cr
\noalign{\smallskip}
 $E_3$:  &$XG + XG$, $XG + XX$, $XX + XG$, ou $XX + XX$ \cr }   }   }
\medskip
Dans chacun de ces trois \'ev\'enements on aura une probabilit\'e 
sp\'ecifique pour que les petits-enfants soient $GG$:
\medskip
\centerline{\vbox{
\halign{ # &\hskip6mm #\hfill &\hskip6mm #\hfill \cr
 parents   &enfants   &petits-enfants \cr
\noalign{\medskip}
$E_1$  & $100\%\; GG$  & $100\%\; GG$ \cr
$E_2$  & $50\%\; GX$ et $50\%\; XG$ & $25\%\; GG$\cr
$E_3$ & $100\%\; XX$  & $0\%\; GG$ \cr }   }   }
\medskip
Si on appelle $E$ l'\'ev\'enement ``un petit-enfant est $GG$'', on
peut interpr\'eter le nombre qui figure dans la troisi\`eme colonne de ce
tableau comme la probabilit\'e conditionnelle de $E$ sachant que $E_1$,
$E_2$, ou $E_3$ s'est produit. Ainsi
$${\cal P}\,(E \mid E_1) = 1\quad ; \quad {\cal P}\,(E \mid E_2) = 0.25
\quad ; \quad {\cal P}\,(E \mid E_3) = 0\quad ; $$
Pour conna{\^\i}tre la probabilit\'e de $E$ lorsque les parents sont pris 
au hasard dans la population (donc on ignore s'ils sont dans $E_1$, dans
$E_2$, ou dans $E_3$), on utilise $IV.4$, ce qui donne:
$${\cal P}\,(E) = {\cal P}\,(E \mid E_1) \cdot {\cal P}\,(E_1) + 
{\cal P}\,(E \mid E_2) \cdot {\cal P}\,(E_2) + {\cal P}\,(E \mid E_3) 
\cdot {\cal P}\,(E_3)$$ 
Tout comme pour l'exp\'erience avec le g\'enie cach\'e dans l'urne, on ne 
peut  pas, dans ce probl\`eme, trouver ${\cal P}\,(E_1)$, ${\cal P}\,(E_2)$
et  ${\cal P}\,(E_3)$ en cherchant des \'epreuves \'equiprobables a priori
car les m\'ecanismes d'apparition des g\`enes $G$ et $X$ nous sont 
inconnus et certainement si complexes que nous serions bien incapables 
d'y  trouver des invariances. Conform\'ement aux remarques faites au 
d\'ebut de cette section, on peut cependant ignorer ces invariances, 
il n'est m\^eme pas n\'ecessaire qu'il existe un moyen de les mettre en 
\'evidence. Il
suffit de remplacer les coefficients ${\cal P}\,(E_j)$  parleurs
valeurs observ\'ees. Ainsi, l'espace $\Omega$ reste ici inconnu,  mais
nous pouvons nous en passer en utilisant des donn\'ees empiriques.
Celles-ci nous fournissent la valeur du nombre $\varepsilon$ (la
proportion d'individus $GG$ dans la population, d'o\`u nous d\'eduisons 
par la loi de Hardy-Weinberg que si on choisit au  hasard un individu
dans  la population, la probabilit\'e pour qu'il soit  $GG$ sera
$\varepsilon$, la probabilit\'e pour qu'il soit h\'et\'erozygote $GX/XG$ 
sera $2\, (\sqrt{\varepsilon } - \varepsilon )$, et la probabilit\'e pour 
qu'il  soit  $XX$ sera $\eta = 1 -  2\sqrt{\varepsilon } +\varepsilon = (1 -
\sqrt{\varepsilon })^2$. Si les couples de parents sont pris au hasard,
la probabilit\'e pour que par exemple les deux parents soient $GG$ sera
$\varepsilon^2$. L'\'ev\'enement $E_1$ du tableau ci-dessus aura donc 
la probabilit\'e  
$$\eqalignno{
{\cal P}\, (GG+GG) + {\cal P}\, (GG+GX) 
&+ {\cal P}\, (GX+GG)  + {\cal P}\, (GX+GX) \cr
\noalign{\smallskip}
&= \quad\varepsilon^2 + \varepsilon\cdot {\up{x}\over\ldown{2}} + 
{\up{x}\over\ldown{2}}\cdot \varepsilon + {\up{x}\over\ldown{2}}\cdot
{\up{x}\over\ldown{2}} \cr   
\noalign{\smallskip}
&=\quad\varepsilon^2 +
x\varepsilon + {\up{x^2}\over\ldown{4}}\quad = \quad\varepsilon\cr }$$
L'\'ev\'enement $E_2$ aura la probabilit\'e  
$$\eqalignno{
{\cal P}\, (GG &+ XG) + {\cal P}\, (GG+XX) + {\cal P}\, (GX+XG)  + 
{\cal P}\, (GX+XX) + \cr
 + {\cal P}\, (XG &+ GG) +   
{\cal P}\, (XG+GX) + {\cal P}\, (XX+GG)  + {\cal P}\, (XX+GX) \cr
\noalign{\smallskip}
&= \quad\varepsilon\cdot {\up{x}\over\ldown{2}} + \varepsilon\cdot\eta 
+ {\up{x}\over\ldown{2}}\cdot {\up{x}\over\ldown{2}} + 
{\up{x}\over\ldown{2}}\cdot\eta + {\up{x}\over\ldown{2}}\cdot
\varepsilon + {\up{x}\over\ldown{2}}\cdot {\up{x}\over\ldown{2}} +  
\eta\cdot\varepsilon  + \eta\cdot{\up{x}\over\ldown{2}} \cr 
\noalign{\smallskip}
&=\quad2\,\big(\sqrt\varepsilon -\varepsilon\big)\quad =\quad x\cr }$$
Enfin, l'\'ev\'enement $E_3$ aura la probabilit\'e   
$$\eqalignno{
{\cal P}\, (XG+XG) + {\cal P}\, (XG+XX) 
&+ {\cal P}\, (XX+XG)  + {\cal P}\, (XX+XX) \cr
\noalign{\smallskip}
&= \quad{\up{x}\over\ldown{2}}\cdot 
{\up{x}\over\ldown{2}} + {\up{x}\over\ldown{2}}\cdot\eta +\eta\cdot
{\up{x}\over\ldown{2}} + \eta^2 \cr 
\noalign{\smallskip}
&= \quad\big( 1-\sqrt\varepsilon\,\big)^2\quad = \quad\eta \cr }$$ 
Ces trois r\'esultats auraient pu \^etre obtenus sans calcul:
l'\'ev\'enement $E_1$ regroupe en effet les couples de parents qui
procr\'eeront des $GG$, l'\'ev\'enement $E_2$ les couples de parents qui
procr\'eeront des h\'et\'erozygotes, et l'\'ev\'enement $E_3$ les couples
de parents qui  procr\'eeront des $XX$; il est donc logique (puisqu'on
suppose que tous les couples ont la m\^eme f\'econdit\'e) que les poids
relatifs de ces \'ev\'enements soient \'egaux \`a celui de la
prog\'eniture correspondante. 
\medskip
Appliquons alors la r\`egle $IV.4$ des probabilit\'es conditionnelles:
${\cal P}\, (E) = \varepsilon + 0.25\, x + 0\, \eta = {1\over
2}\,(\sqrt\varepsilon + \varepsilon )$. 
\medskip
On peut donc \'enoncer le r\'esultat suivant: dans une population o\`u
la proportion d'individus $GG$ est $\varepsilon$, un couple de parents
pris au hasard a (pour chaque enfantement) la probabilit\'e $\varepsilon$ 
de mettre au monde un individu $GG$. Mais un couple form\'e de fr\`ere et 
soeur a la probabilit\'e ${1\over 2}\, (\sqrt\varepsilon + \varepsilon )$
de  mettre au monde un individu $GG$. Si par exemple il y a un $GG$ sur 
$10\, 000$ dans la population, la procr\'eation entre fr\`ere et soeur a 
une chance sur deux cent de produire un $GG$.
\medskip
La loi des accouplements consanguins peut \^etre g\'en\'eralis\'ee
mais nous ne donnons ici que les r\'esultats\ftn{1}{On trouvera un
expos\'e assez complet (mais rapide) des lois probabilistes  de
l'h\'er\'edit\'e dans le petit ouvrage de Gustave Mal\'ecot {\it Les
math\'ematiques de l'h\'er\'edit\'e} (Masson, Paris, {\oldstyle 1948}). 
Le lecteur qui ne craint pas d'effectuer tous les calculs lui-m\^eme pourra
trouver le th\`eme de la consanguinit\'e trait\'e en exercices dans l'ouvrage
de William  Feller  (chapitre V, probl\`emes $N^{\rm os}\, 33$ \`a $40$, 
pages 144 -- 145.}. Par le m\^eme proc\'ed\'e que celui utilis\'e plus 
haut (application de la r\`egle $IV.4$ des probabilit\'es conditionnelles) 
on peut analyser l'effet de croisements moins fortement consanguins;
dans le cas de croisements entre fr\`eres et soeurs, nous avons divis\'e 
les choix possibles de parents en  cat\'egories; pour des cousins 
germains, qui n'ont pas les m\^emes parents, mais les m\^emes
grands-parents, il faudra remonter \`a ces derniers.   Les lois obtenues
sont simples: pour les exprimer on introduit un nombre appel\'e {\it
facteur de consanguinit\'e}. Ce nombre vaut ${1\over 2}$ pour des 
fr\`eres et soeurs; il est \'egal \`a ${1\over 8}$ pour un couple form\'e
d'un demi-fr\`ere et sa demi-soeur ou de doubles cousins germains, \`a
${1\over 16}$ pour des cousins germains, etc. Si $f$ est le facteur de
consanguinit\'e d'un couple, la probabilit\'e pour  qu'il engendre un enfant
$GG$ est $f\sqrt{\varepsilon } + (1- f)\varepsilon$ (dans l'hypoth\`ese
o\`u les individus $GG$ ont  autant de chances de procr\'eer que les
autres). Pour les maladies g\'en\'etiques   tr\`es rares, $\varepsilon$ est
n\'egligeable devant $\sqrt\varepsilon$;  la probabilit\'e de transmettre
la maladie est donc beaucoup plus \'elev\'ee  en cas de mariage
consanguin, car le facteur $f$  est alors beaucoup plus grand que
$\sqrt\varepsilon$. Pour la mucoviscidose, qui n'est pas rarissime,
$\varepsilon = 1/2500$ et $\sqrt\varepsilon = 1/50$, donc un mariage
entre cousins germains donnera une probabilit\'e $1/615$, quatre fois
plus grande que pour une procr\'eation exogamique. 
\medskip
Il ne faut pas oublier qu'il s'agit l\`a des probabilit\'es correspondant 
\`a un choix au hasard de cousins germains parmi la population. Il est 
bien clair que si les cousins savent que l'un d'entre eux est homozygote
$XX$, la  probabilit\'e d'engendrer un $GG$ sera nulle, et s'ils savent
\^etre tous deux h\'et\'erozygotes, la probabilit\'e sera ${1\over 4}$.





\bye
