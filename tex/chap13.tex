\input twelvea4.tex
\input epsf.tex

\auteurcourant={\sl J. Harthong: probabilit\'es et statistique}
\titrecourant={\sl Les corr\'elations E.P.R.}

\pageno=377

\vskip10mm plus3mm minus3mm

\centerline{\tit XIII. CORR\'ELATIONS E.P.R.}
\vskip8pt
\centerline{\tit La Statistique et les myst\`eres de la causalit\'e.}

\vskip10mm plus3mm minus3mm

Pour couronner la discussion sur la causalit\'e amorc\'ee au chapitre 
{\bf I} et poursuivie en {\bf IV.2} puis {\bf XII.6}, nous allons 
\'etudier soigneusement un exemple issu  de la Physique, dans lequel 
la causalit\'e est fortement mise \`a l'\'epreuve: l'{\it exp\'erience 
d'Einstein, Podolski, Rosen} que nous abr\'egerons en {\it exp\'erience 
E.P.R.} Comme toute exp\'erience de M\'ecanique quantique, celle-ci  
fait appel \`a la Statistique, et par cons\'equent son interpr\'etation  
fait appel au Calcul des probabilit\'es. Le ph\'enom\`ene \'etudi\'e  dans 
cette exp\'erience est int\'eressant non (pour  le moment) par ses 
applications technologiques, mais parce qu'il fait  appara\^\i tre le  
monde quantique de mani\`ere si flagrante qu'aucune \'echappatoire  
n'est plus possible. Pour notre discussion sur la corr\'elation et la 
causalit\'e, il est int\'eressant parce  que, \'etant situ\'e \`a la 
fronti\`ere de ce qui est compris et ce qui ne l'est pas encore, n'\'etant 
donc pas encore soumis \`a des certitudes vraies ou fausses mais 
seulement \`a des doutes, il fournit un exemple de confrontation directe 
entre l'objectif et le subjectif: les corr\'elations observ\'ees dans 
l'exp\'erience  E.P.R. (appel\'ees de ce fait {\it corr\'elations E.P.R.}) 
sont d\'epourvues \`a la fois de l'ant\'eriorit\'e et de la relation de 
n\'ecessit\'e; la M\'ecanique  quantique les pr\'edit et les d\'ecrit, mais 
nul n'y per\c{c}oit une relation de n\'ecessit\'e causale. Enfin, 
l'absence d'ant\'eriorit\'e est exp\'erimentalement \'etablie.    
\medskip 
Par ailleurs, l'interpr\'etation de l'exp\'erience par le Calcul des  
probabilit\'es (effectu\'ee par J. S. Bell) r\'ev\`ele remarquablement bien
les \'evidences erron\'ees que nous pouvons v\'ehiculer inconsciemment  
en appliquant les r\`egles de ce Calcul. L'analyse de l'exp\'erience E.P.R.  
va donc nous servir \`a mieux comprendre ce qu'est {\it r\'eellement} un 
\'ev\'enement.  
 
\vskip6mm plus3mm minus3mm 

{\bf XIII. 1. Description du probl\`eme.}
\medskip
Einstein, Podolski, et Rosen avaient imagin\'e cette exp\'erience de 
pens\'ee en {\oldstyle 1935} \ftn{1}{{\it Can Quantum-Mechanical 
Description of Physical Reality Be Considered Complete ?} Physical 
Review, vol {\bf 47}, {\oldstyle 1935}, pp. 777 -- 780. L'exp\'erience  
que nous d\'ecrivons ici (fig. 58) n'est pas exactement celle de cet  
article, mais ---~conform\'ement \`a une tradition \'etablie~--- la 
version revue  et corrig\'ee par David Bohm.}. \`A cette  
\'epoque toutefois l'exp\'erience ne pouvait pas \^etre r\'ealis\'ee (la 
technologie ne  le permettait pas), et m\^eme si cela avait \'et\'e 
possible, on n'aurait pas pu en tirer des r\'esultats  concluants. La 
premi\`ere exp\'erience concr\`ete qui corresponde au mod\`ele  
imagin\'e tout en \'etant  absolument  concluante et incontestable, a 
\'et\'e r\'ealis\'ee par Alain Aspect \`a l'Institut d'Optique d'Orsay 
({\oldstyle 1982}). Il n'y a pas la place ici  pour faire un expos\'e 
historique de la question\ftn{2}{Un bon expos\'e de synth\`ese est {\it 
Bell's Theorem: experimental tests and implications} by John F. Clauser 
and Abner Shimony (Reports on Progress in Physics, vol. {\bf 41}, 
{\oldstyle 1978} pages 1181 -- 1927). Son seul d\'efaut est d'\^etre 
ant\'erieur \`a l'exp\'erience d'Aspect. Une autre r\'ef\'erence plus 
r\'ecente et plus accessible est le livre de Franco Selleri {\it Le grand
d\'ebat de la th\'eorie quantique}. Flammarion, Paris, {\oldstyle 1986}.}. 
Mais nous allons \'etudier la 
question dans  l'\'etat o\`u elle peut \^etre formul\'ee aujourd'hui, en 
ignorant les \'etapes historiques de clarification par lesquelles elle  
est pass\'ee. Voici donc le probl\`eme.    
\medskip 
\midinsert  
\epsfxsize=\hsize
\line{\epsfbox{../images/fig58.eps} } 
\centerline{\eightpoint figure 58} 
\endinsert 
On effectue l'exp\'erience repr\'esent\'ee sur la figure 58. Un faisceau 
de mol\'ecules diatomiques est pr\'epar\'e dans un \'etat ``singulet'',
c'est-\`a-dire de moment cin\'etique (spin) nul. Pour le lecteur qui
ne comprend pas ces termes, il suffit de savoir qu'on pr\'epare ces 
mol\'ecules de telle fa\c{c}on que lorsqu'elles se diviseront, leurs deux
composants se comporteront sym\'etriquement. Ce faisceau circule de 
$S$ vers $O$. En pratique le faisceau est produit en chauffant dans un 
four un m\'etal dont la vapeur est form\'ee de mol\'ecules diatomiques,
et en filtrant les vapeurs qui en sortent; on filtre la
direction de propagation avec des diaphragmes qui absorbent tout ce qui
ne part pas dans la bonne direction, et on filtre l'\'etat singulet avec des
champs magn\'etiques qui d\'evient tout ce qui poss\`ede un moment 
cin\'etique non nul. En $O$ ces mol\'ecules sont cass\'ees en deux et deux
atomes de  spins oppos\'es partent de $O$ dans des directions \'egalement
oppos\'ees (en pratique: on casse les mol\'ecules avec un laser, et on
filtre encore les directions de propagation avec des diaphragmes). Les
deux atomes issus de cette fission ont des spins oppos\'es d'apr\`es la
conservation du moment cin\'etique. On dispose sur leurs trajectoires 
respectives des aimants de Stern-Gerlach qui permettront de mesurer 
les spins des deux atomes. L\`a encore, il n'est pas demand\'e au lecteur 
de savoir ce qu'est un aimant de Stern-Gerlach; il suffit de comprendre 
qu'il s'agit d'un dispositif permettant de mesurer un param\`etre (le spin) 
qui ne peut prendre que les valeurs $+1$ ou $-1$. Ces d\'etails pratiques 
sont donn\'es pour fixer  les id\'ees, mais n'interviennent pas au niveau  
du principe. D'ailleurs dans l'exp\'erience d'Alain Aspect on utilisait des 
atomes de Calcium (et non  des mol\'ecules diatomiques), qui apr\`es
excitation par laser se d\'esexcitaient  spontan\'ement en \'emettant 
deux photons jumeaux de polarisations oppos\'ees, et on mesurait 
ensuite la polarisation des photons avec des polariseurs sp\'eciaux. La
d\'esexcitation des atomes de Calcium ne donnant des photons jumeaux 
que dans des conditions tr\`es sp\'ecifiques, on imagine  le soin
qu'exigeaient la r\'ealisation pratique de ces conditions; de m\^eme  pour
les diff\'erents filtrages. Nous passons  sur  ces probl\`emes techniques
complexes. Comme toujours en Physique, on  ne peut r\'ealiser
exp\'erimentalement une situation id\'ealement simple qu'au prix
d'appareillages complexes.  
\medskip  
Le dispositif est suppos\'e sym\'etrique; en particulier, les deux 
analyseurs (aimants de Stern-Gerlach ou polariseurs d'Aspect) sont 
suppos\'es rigoureusement identiques;
dans l'exp\'erience ils ne doivent diff\'erer que par leur orientation. 
\medskip 
Le trait essentiel du dispositif est le suivant: les deux aimants peuvent 
tourner autour de leur axe longitudinal, mat\'erialis\'e par la trajectoire 
de l'atome. On distinguera l'aimant de gauche ($G$) et celui de droite ($D$). 
Appelons  $a$ la direction selon laquelle est orient\'e l'aimant de gauche 
et $b$ celle de l'aimant de droite. Puisque les aimants tournent autour de 
leurs axes longitudinaux, qui sont fixes, les directions $a$ et $b$ peuvent  
\^etre rep\'er\'ees chacune par un seul param\`etre, un angle variant de  
$0$ \`a $2\pi$.  
\medskip 
Dans le langage imag\'e et souvent trompeur de la M\'ecanique quantique, 
on dit usuellement, 
lorsque l'aimant de Stern-Gerlach pointe dans la direction $a$, qu'il 
``mesure la composante selon $a$ du spin''. Le spin de l'atome se 
manifeste au cours de la travers\'ee de l'aimant par une d\'eviation dans  
la direction $a$, qui est soit positive (vers le ``haut'' de l'aimant), soit 
n\'egative (vers le ``bas'' de l'aimant). L'amplitude de la d\'eviation  
d\'epend des caract\'eristiques de l'aimant (longueur, champ magn\'etique) 
mais est en valeur absolue la m\^eme pour tous les atomes pr\'epar\'es; 
seul le sens (le signe) de d\'eviation varie avec l'atome individuel. 
Ce signe est al\'eatoire: il y a toujours une chance sur deux d'observer  
$+$ ou $-$.  La `mesure du spin'' consiste \`a constater si 
l'atome est d\'evi\'e positivement ou  n\'egativement. On dira que la 
composante selon $a$ du spin est $+{1  \over 2}\hbar$ dans le premier  
cas et $-{1 \over 2}\hbar$ dans le second cas. Pour simplifier, on  
omettra $\hbar$ et on indiquera seulement le signe $+1$ ou $-1$.  
Le lecteur qui n'est pas encore familier avec la M\'ecanique quantique 
ne doit pas chercher \`a comprendre ces phrases autrement que comme  
une mani\`ere tortueuse d'exprimer le constat de ces d\'eviations.  
\medskip 
La M\'ecanique quantique affirme qu'il est impossible de mesurer en 
m\^eme temps les composantes du spin d'un m\^eme atome dans deux 
directions orthogonales (principe de Heisenberg dit d'incertitude ou de 
compl\'ementarit\'e). Plus exactement:  l'\'etat de spin n'est pas une 
propri\'et\'e intrins\`eque de l'atome que  l'appareil mesure, mais une 
propri\'et\'e de  l'atome relative \`a l'appareil. Encore plus exact: ce
n'est pas une propri\'et\'e relative \`a l'appareil lui-m\^eme, mais \`a ses 
sym\'etries internes (un autre appareil compl\`etement diff\'erent, mais 
pr\'esentant les m\^emes sym\'etries internes, mesurerait aussi 
(\'eventuellement sous une forme autre qu'une d\'eviation) la ``composante 
selon $a$ du spin''. L'aimant de Stern-Gerlach que traverse l'atome lui 
fournit un environnement spatial form\'e d'un champ magn\'etique \`a fort 
gradient, le champ {\it et} son gradient \'etant dirig\'es selon $a$. On ne 
peut certes pas tourner l'aimant globalement autour de l'axe $a$ sans 
d\'er\'egler l'exp\'erience car l'atome doit traverser l'aimant 
longitudinalement; mais l'atome est localis\'e et sa trajectoire est 
rectiligne, de sorte qu'\`a  chaque instant, l'atome ``voit'' son 
environnement imm\'ediat (le champ magn\'etique local) comme invariant 
par les rotations autour de l'axe $a$. Ce qui se passe loin de l'atome 
(la forme particuli\`ere des fers) n'influe pas sur lui et la direction 
longitudinale sert uniquement \`a le maintenir dans l'entrefer tout au  
long de sa trajectoire. (On pourrait aussi exprimer cela en disant que le 
spin n'est modifi\'e ni par le mouvement rectiligne uniforme, ni par les 
valeurs du champ magn\'etique en dehors du voisinage imm\'ediat de 
l'atome).  
\vskip6pt plus5pt minus4pt 
Voici maintenant le but qu'Einstein, Podolski, et Rosen souhaitaient 
donner \`a cette exp\'erience. Si on mesure le spin d'un atome 
selon la direction $a$ dans l'aimant de gauche $G$ {\it et} le spin 
selon la direction $b$ de son jumeau (issu de la m\^eme 
mol\'ecule)  dans l'aimant de droite $D$, alors, si $a = b$, ils seront 
oppos\'es (les d\'eviations seront oppos\'ees). C'est bien ce que pr\'edit 
aussi la M\'ecanique quantique.  Par cons\'equent, si le spin selon $b$ du 
jumeau de droite est $+1$ (respectivement: $-1$), on est assur\'e qu'il
serait {\it forc\'ement} $-1$ (respectivement: $+1$) pour le jumeau de gauche
si les deux aimants pointaient dans la m\^eme direction. 
Sachant cela, on va prendre $b$ {\it orthogonal} \`a $a$ et mesurer le spin
du jumeau de droite selon $b$. Ainsi on peut d\'eduire le spin selon $b$ du
jumeau de gauche (c'est l'oppos\'e du spin mesur\'e \`a droite);  et comme
on a mesur\'e directement le spin selon $a$ du jumeau de gauche,  on
conna{\^\i}t,  pour ce jumeau,  \`a la fois le spin selon $a$ et selon $b$. 
Einstein,  Podolski,  et Rosen concluent alors que par cette exp\'erience
on peut conna{\^\i}tre \`a la fois le spin selon $a$ et le spin selon $b$
d'une particule (ici,  le jumeau de gauche).  
\medskip
Au d\'ebut de leur article de {\oldstyle  1935} ils d\'efinissent ainsi
ce qu'est un {\it \'el\'ement de r\'ealit\'e}:
\smallskip
{\cit Si,  sans perturber en aucune mani\`ere un syst\`eme, 
nous pouvons pr\'edire avec certitude (c'est-\`a-dire avec une
probabilit\'e \'egale \`a un) la valeur d'une grandeur physique, 
alors il existe un \'el\'ement de r\'ealit\'e physique correspondant
\`a cette grandeur physique. \par}
\medskip
Et peu avant ils affirment ceci:
\smallskip
{\cit Quel que soit le sens qu'on donne au terme {\it complet}, 
pour qu'une th\'eorie soit compl\`ete il semble que l'exigence
suivante soit n\'ecessaire: {\it chaque \'el\'ement de la r\'ealit\'e
physique doit avoir sa contrepartie dans la th\'eorie physique}. 
Nous appellerons cela la condition de compl\'etude. \par}
\medskip
La M\'ecanique quantique pr\'edit que si on sait que le spin
du jumeau de gauche selon $a$ est, disons, $+1$, alors il y a
exactement autant de chances a priori de trouver $+1$ que
$-1$ pour l'autre jumeau selon la direction orthogonale $b$, 
et cela est enti\`erement compatible avec les pr\'esuppos\'es
d'Einstein,  Podolski,  et Rosen:  cela veut dire que la mesure
indirecte du spin selon $b$ (c'est-\`a-dire la mesure effectu\'ee
sur le jumeau de droite afin de ne pas perturber celui de gauche)
donnera une fois sur deux $+1$ et une fois sur deux $-1$. 
\medskip
La M\'ecanique quantique de Bohr et Heisenberg affirme express\'ement
dans ses pr\'esuppos\'es que le spin ne peut \^etre un \'el\'ement de
r\'ealit\'e que dans une seule direction \`a la fois.  Toutefois
il s'agit l\`a d'un principe {\it qualitatif} de la M\'ecanique
quantique,  dont il est toujours possible de nier la validit\'e
car il ne peut \^etre v\'erifi\'e directement par l'exp\'erience. 
Cependant,  la M\'ecanique quantique propose aussi des pr\'edictions
quantitatives:  si $b$ fait un angle $\theta$ non n\'ecessairement
droit avec $a$,  les probabilit\'es  d'avoir $+1$ et $-1$ pour
l'autre jumeau seront respectivement $\cos^2{\theta \over 2}$ et
$\sin^2{\theta \over 2}$.  Nous verrons que cette pr\'ediction est
incompatible avec les pr\'esuppos\'es d'Einstein,  Podolski, 
et Rosen,  {\it m\^eme sans recourir au principe qualitatif
susmentionn\'e} (voir section {\bf XIII.4} ci-apr\`es et notamment la
figure 62),  sauf justement dans le cas de directions orthogonales
ou parall\`eles,  qui sont des cas particuliers trompeurs.  Si ces
cas particuliers trompeurs n'avaient pas exist\'e,  le d\'ebat entre
les {\og partisans d'Einstein\fg} et les {\og partisans de Bohr\fg}
aurait \'et\'e tranch\'e bien plus vite. 
\medskip 
S'il \'etait possible de constater d'abord la d\'eviation de l'atome dans  
un aimant selon $a$ puis de faire passer le m\^eme atome dans un second 
aimant pour en mesurer cette fois le spin selon $b$, cette 
exp\'erience compliqu\'ee ne serait pas n\'ecessaire. Mais tout le  
probl\`eme  provient de ce que la mesure effectu\'ee dans le premier 
aimant perturbe l'\'etat de l'atome: son spin est de l'ordre de $\hbar  
\sim 10^{-34}$ (il vaut $\pm {1\over 2}\hbar$), et la d\'eviation dans le 
champ magn\'etique est l'effet d'une action du champ sur la particule, 
dont l'intensit\'e est \'egalement de l'ordre de $\hbar$. Pour mesurer le 
spin on a donc exerc\'e une action physique aussi grande que la grandeur 
\`a mesurer. C'est comme si on voulait mesurer l'\'epaisseur d'un rouleau 
de p\^ate avec un pied \`a coulisse qui s'enfonce dedans. Einstein partait 
de ce constat, que la nature est ainsi faite que toute mesure utilise des 
champs ou de la lumi\`ere qui sont impalpables pour des corps 
macroscopiques, mais qui ne le sont plus pour des atomes. D'o\`u cette 
exp\'erience qui permet de mesurer le spin selon $b$ du jumeau de  
gauche en ne perturbant que le jumeau de droite, donc avec un proc\'ed\'e
impalpable pour le jumeau de gauche, dont on mesure alors indirectement et
``sans le perturber d'aucune mani\`ere'' le spin selon $b$.  
\medskip 
En \'etudiant bien les arguments des uns et des autres (ou pour  
simplifier:  de l'un -- Einstein --  et de l'autre -- Bohr), on peut voir que 
le d\'esaccord (ou la diff\'erence) entre les deux conceptions concernait 
la nature exacte de ce qui est mesur\'e; Einstein voulait que le spin f\^ut  
une caract\'eristique intrins\`eque de l'atome, ind\'ependante de  
l'appareil, et en particulier ind\'ependante de la position de l'appareil, 
tandis que Bohr, apparemment parce qu'il arrivait mieux \`a l'accepter, 
admettait que le spin est une caract\'eristique de l'atome {\it relative} 
\`a son environnement. Le titre de l'article d'Einstein, Podolski et Rosen 
``Can Quantum-Mechanical Description of Physical Reality Be Considered 
Complete ? '' sugg\`ere d\'ej\`a l'id\'ee sous-jacente que le spin serait 
fix\'e au moment o\`u la mol\'ecule se divise en deux au point $O$, de 
sorte que le signe que prendra la d\'eviation pendant la travers\'ee de 
l'aimant sera d\'etermin\'e \`a l'avance pour toute direction $a$ ou $b$. 
Admettant provisoirement que le spin attribu\'e \`a chacun des deux 
jumeaux au moment de la fission est totalement al\'eatoire et 
impr\'evisible comme le point o\`u vient s'arr\^eter une bille de roulette, 
la M\'ecanique quantique devrait au moins, pour pouvoir \^etre 
consid\'er\'ee  comme compl\`ete, per\-met\-tre de pr\'edire le  
r\'esultat de la  d\'eviation \`a partir de l'\'etat du jumeau 
imm\'ediatement  apr\`es la fission. C'est-\`a-dire que, pour \^etre 
compl\`ete, elle devrait inclure dans la th\'eorie des param\`etres 
d\'ecrivant cet \'etat, de telle sorte que, m\^eme s'il est techniquement 
difficile de les mesurer, du moins leur valeur d\'etermine le sens de la 
d\'eviation. Einstein, Podolski, et Rosen acceptent le constat qu'on ne 
peut pas mesurer ces param\`etres  avec de la  lumi\`ere ou des champs 
magn\'etiques sans  les perturber,  car ces ph\'enom\`enes sont trop 
brutaux; mais l'\'etat du jumeau est, selon eux, quand m\^eme 
d\'etermin\'e, et on pourrait le mesurer sans le perturber si on 
d\'ecouvrait des environnement physiques beaucoup plus t\'enus que les 
champs magn\'etiques, qui exerceraient sur la particule une action bien 
plus faible que la grandeur \`a mesurer. La M\'ecanique quantique est 
donc incompl\`ete car elle \'ecarte a priori de la th\'eorie  ce que des 
mesures moins brutales r\'ev\'eleraient.    
\medskip  
Bohr interpr\'etait cela diff\'eremment. Pour le comprendre on peut 
s'aider d'une comparaison: imaginons un sondage d'opinion avec des
questions du type ``\^etes-vous plut\^ot satisfait ou plut\^ot insatisfait
de la politique du gouvernement en mati\`ere de $X$?'' La r\'eponse du
sond\'e sera $+1$ s'il est ``plut\^ot satisfait'' et $-1$ s'il est ``plut\^ot
insatisfait''. Il y a bien les  sans opinion, mais cela existe aussi dans les
exp\'eriences du type E.P.R. car il y a toujours des atomes pour lesquels 
la mesure ne marche pas; on ne les comptabilise pas. Le $X$ dans la
question repr\'esente un th\`eme: \'education, d\'efense, diplomatie,
urbanisme, am\'enagement du territoire, recherche, agriculture, etc. Si 
on compare avec nos atomes, le th\`eme  $X$ joue le r\^ole de la direction
$a$. L'\'etat singulet correspondrait \`a  des couples o\`u chacun contredit
syst\'ematiquement son conjoint. Pr\'eparer les mol\'ecules dans l'\'etat
singulet correspond dans la comparaison \`a  s\'electionner ces couples
pour faire le sondage parmi eux seuls. Il ne semble  pas raisonnable de
croire qu'il existe  un ``\'etat objectif du sond\'e'', d\'ecrit par des
param\`etres bien pr\'ecis et tels que, si on les connaissait tous, on
pourrait pr\'edire sa r\'eponse quel que soit $X$. M\^eme si on fait
abstraction de d\'etails tels que l'influence du sondeur, l'humeur du jour,
etc, il semble assez plausible  que la question puisse prendre le sond\'e
au d\'epourvu, de telle sorte  qu'il improvise une r\'eponse qui  ne
correspond m\^eme pas \`a sa nature ou \`a ses pr\'ef\'erences politiques.
Cela n'emp\^echera pas le sondage d'\^etre statistiquement correct et de 
donner \`a la fourchette pr\`es   les m\^emes r\'esultats que n'importe
quel autre, effectu\'e dans la population fran\c{c}aise au m\^eme moment;
c'est l\`a le miracle des probabilit\'es. Cette comparaison ne doit
\'evidemment pas \^etre prise  au pied de la  lettre; son seul but ici est  
de montrer qu'il n'y a aucune {\it \'evidence a priori} pour que la r\'eponse
du sond\'e soit objectivement d\'etermin\'ee \`a l'avance par un ``\'etat''.
En vertu de quelle autre \'evidence pourrions-nous affirmer que la
r\'eponse des atomes  est objectivement  d\'etermin\'ee \`a l'avance par
son ``\'etat''? Parce que l'atome serait un  \^etre non vivant,
contrairement au sond\'e? Mais c'est l'exp\'erience et non  la raison a
priori qui nous a montr\'e que les \^etres vivants ne sont pas r\'eductibles
\`a un d\'eterminisme simple. Donc seule l'exp\'erience peut nous faire
savoir si les atomes sont ou non r\'eductibles \`a un d\'eterminisme
simple. Ou parce que les \^etres vivants seraient complexes, tandis que
les atomes seraient simples?  L\`a aussi, c'est l'exp\'erience et non la
raison a priori qui a montr\'e  que le mouvement des plan\`etes, ou les
ph\'enom\`enes \'electriques, ob\'eissent \`a des lois simples, et qu'il
n'en va pas ainsi pour le vivant. Le fait que les atomes soient ``les
composants \'el\'ementaires'' de la  mati\`ere ne prouve pas qu'ils sont
obligatoirement plus simples; justement, l'un des principaux
enseignements du Calcul des probabilit\'es est que des ph\'enom\`enes
extr\^emement complexes peuvent sembler simples \`a plus grande
\'echelle, soit par l'effet de la loi des grands nombres (cf. chapitre {\bf
VII}), soit par la transformation du chaos en hasard (sections {\bf I.4},
{\bf I.5}, {\bf II.6}, {\bf IV.2}).
\medskip
Ces remarques devraient aider \`a comprendre la position de Bohr, plus
bas\'ee sur le doute. D'un autre c\^ot\'e, la conviction exprim\'e  par
Einstein qu'il {\it doit} y avoir une explication (``Dieu ne joue pas aux
d\'es!'') est  n\'ecessaire pour chercher. La science serait impensable
sans  la conviction que ``Dieu est math\'ematicien''.
\medskip   
En ce qui concerne le spin des atomes, il faut se garder d'introduire
implicitement des  ``\'evidences'' issues d'un autre domaine d\'ej\`a connu.
L'{\it \'etat} dont parle la M\'ecanique quantique n'est pas l'\'etat de 
l'atome jumeau,  mais l'\'etat du syst\`eme atome + champ magn\'etique
\`a fort gradient, ou bien l'\'etat de l'atome relatif \`a l'environnement.
Ainsi le langage usuel de la M\'ecanique quantique (encore marqu\'e  par
les conceptions classiques) est trompeur; on ne devrait pas dire
``l'appareil orient\'e  selon $a$ mesure la composante du spin selon $a$''
comme s'il existait  un spin intrins\`eque  dont seule la composante
serait relative, mais ``on mesure le spin de l'atome dans   
l'environnement $a$''. 
\medskip  
Jusqu'ici cependant nous rencontrons des diff\'erences d'opinion 
ou des dif\-f\'e\-rences d'inter\-pr\'e\-ta\-tion, et nous ne pouvons pas  
trancher le  d\'ebat. Nous ne pouvons pas prouver que la position 
d\'efendue par Einstein est fausse. Ni -- m\^eme lorsque les  
pr\'edictions quantitatives de la M\'ecanique quantique sont v\'erifi\'ees 
-- prouver que Bohr a raison. Une chose est l'acte purement technique qui 
consiste \`a appliquer le formalisme de la M\'ecanique quantique et \`a 
effectuer une exp\'erience  pour constater l'accord entre la th\'eorie et 
l'exp\'erience, une toute autre chose est l'interpr\'etation. Est-il 
possible de trancher exp\'erimentalement ce d\'ebat  d'interpr\'etation ? 
En {\oldstyle 1964} parut un article de J. S. Bell\ftn{3}{J.S. Bell {\it On 
the Einstein Podolski Rosen Paradox} (Physics, Vol. {\bf 1}, $N^o\, 3$, pp. 
195 -- 200.)} qui r\'epondit positivement \`a cette question. Mais bien 
entendu, {\it ipso facto} les interpr\'etations se trouvent pr\'ecis\'ees 
-- par les conditions de l'exp\'erience -- et par cons\'equent rendues 
plus \'etroites. C'est donc l'analyse des conditions exactes de 
l'exp\'erience, o\`u la Statistique joue comme toujours un r\^ole cl\'e, 
que je voudrais pr\'esenter ici pour  conclure cet ouvrage, plut\^ot 
comme une question ouverte que comme un exemple p\'edagogique.  
\medskip 
Les arguments pr\'esent\'es par Bell dans son article sont repris ici, mais 
fortement modernis\'es, le sujet ayant b\'en\'efici\'e depuis trente ans  
d'une lente mais consid\'erable maturation. 
\medskip 
Nous prenons donc l'exp\'erience de principe d\'ecrite sur la figure 58, qui  
ne diff\`ere d'exp\'eriences r\'eelles que par des d\'etails inessentiels.  
Les mol\'ecules diatomiques peuvent \^etre produites en 
quantit\'e pratiquement illimit\'ee; les \'echantillons statistiques  
seront donc  aussi nombreux que n\'ecessaire. Nous allons effectuer des 
observations de  la fa\c{c}on suivante: les directions $a$ de l'aimant de 
gauche et $b$ de l'aimant de droite sont r\'eglables \`a volont\'e. 
Effectuons une premi\`ere  s\'erie de $n$ observations avec $a = a_1$ et 
$b = b_1$.  Remplissons un tableau comme le suivant: 
$$\matrix{ 
\quad\hbox{mol\'ecule}\quad &\quad\hbox{gauche}\quad 
&\quad\hbox{droite}\quad \cr  
N^o: & (a_1) & (b_1)\cr 
\noalign{\medskip} 
1 & +1 & +1 \cr 
2 & +1 & -1 \cr 
3 & +1 & -1 \cr 
4 & -1 & -1 \cr 
5 & +1 & +1 \cr 
6 & -1 & +1 \cr 
7 & -1 & -1 \cr 
\noalign{\smallskip} 
\cdots & \cdots & \cdots \cr 
\noalign{\smallskip} 
n & -1 & +1 \cr }$$ 
La premi\`ere colonne indique le num\'ero d'ordre des observations: pour 
chaque mol\'ecule arrivant en $O$ il y aura deux atomes jumeaux, l'un \`a 
gauche et l'autre \`a droite, qui seront analys\'es par l'aimant respectif 
et seront d\'evi\'es positivement ($+1$) ou n\'egativement ($-1$). 
\medskip 
Une premi\`ere propri\'et\'e d'invariance nous permet d'affirmer a priori  
que  pour chacun des deux jumeaux il y aura environ autant de $+1$ que de 
$-1$, les \'ecarts par rapport \`a l'\'equilibre \'etant distribu\'es selon la 
loi gaussienne. En effet, s'il y avait un d\'es\'equilibre \`a ce niveau, cela 
voudrait dire que la nature favorise une polarisation au d\'etriment de 
l'autre ou que l'on pourrait distinguer la gauche de la droite, car l'un des 
deux c\^ot\'es comporterait davantage de $+1$ que l'autre.  
Bien entendu, l'exp\'erience confirme cette invariance.  
\medskip 
Nous pouvons calculer \`a partir du tableau la corr\'elation empirique 
$R$: c'est la moyenne des produits des \'el\'ements de la deuxi\`eme 
colonne par ceux correspondants de la troisi\`eme; ou encore: le produit 
scalaire de la deuxi\`eme et de la troisi\`eme colonne du tableau,  
divis\'e par $n$.  
\medskip 
Le d\'ebat lanc\'e par Einstein, Podolski, et Rosen concerne la question  
suivante: s'il se cr\'ee un \'etat objectif des deux jumeaux (ou 
``\'el\'ement de r\'ealit\'e physique'', comme ils disent dans leur article) 
{\it au moment o\`u la mol\'ecule diatomique se casse en} $O$, et que le 
r\'esultat observ\'e des mesures est l'expression de cet \'etat (peu  
importe si l'\'etat cr\'e\'e est lui-m\^eme al\'eatoire), alors cet \'etat 
objectif est causalement ind\'ependant de l'orientation des aimants, 
d'autant plus que cette orientation peut \^etre mo\-di\-fi\'ee al\'eatoirement 
{\it apr\`es} la fission de la mol\'ecule,  comme  c'est effectivement le 
cas dans l'exp\'erience r\'eelle d'Aspect. Donc selon Einstein, Podolski, et 
Rosen l'exp\'erience {\it reproductible} est d\'ej\`a termin\'ee avec la 
fission de  la mol\'ecule: le passage \`a travers les aimants n'est plus 
qu'une mesure de ce qui a d\'ej\`a eu lieu.  
\medskip 
Lorsqu'on dit qu'une exp\'erience est reproductible, on entend par l\`a  
 que  ce qu'on appelle ``les conditions de l'exp\'erience'' sont identiques 
d'une fois sur l'autre. Lorsqu'on r\'ep\`ete le lancement d'une pi\`ece de 
monnaie, en limer le bord modifie ces conditions; mais que 
l'observateur siffle une fois la Marseillaise et l'autre fois  
l'Internationale ne change pas ces conditions  car il y a ind\'ependance 
causale entre l'air siffl\'e par l'observateur et le processus en cours:  
donc pour qu'une exp\'erience soit reproductible il n'est pas n\'ecessaire 
que d'une fois sur l'autre rien ne se modifie dans l'univers; il est  
seulement n\'ecessaire que d'une fois sur l'autre ne se modifient que  
des choses qui sont causalement  ind\'ependantes du processus. 
\medskip 
Dans la conception d'Einstein, Podolski, Rosen, le r\'esultat des mesures 
d\'epend certes des orientations des aimants, mais non l'\'etat objectif  
des atomes, de m\^eme que le diam\`etre apparent de la plan\`ete Mars 
d\'epend de sa distance \`a la Terre, mais non son diam\`etre r\'eel.  
Une meilleure comparaison serait plut\^ot la suivante: les composantes 
d'un vecteur d\'etermin\'e d\'ependent des axes de coordonn\'ees, mais  
non le vecteur lui-m\^eme.  
\medskip 
Le point essentiel est le suivant: on peut \`a juste titre \^etre convaincu 
que l'\'etat des jumeaux, qui est d\'etermin\'e au moment de la division, 
est ind\'ependant de l'orientation des aimants,  puisque de toute 
\'evidence les aimants sont loin du point $O$ et n'ont rien \`a voir avec  
le processus de division. En outre la division a lieu {\it avant} la mesure: 
admettre que l'\'etat des jumeaux d\'epend de l'orientation des aimants  
revient \`a admettre que l'avenir peut influer sur le pass\'e. 
\medskip 
Mais il reste la question: qu'entend-on par l'{\it \'etat des jumeaux}? 
Lorsque nous avons pr\'esent\'e le lancement d'un d\'e ou d'une pi\`ece de 
monnaie comme une exp\'erience reproductible, nous avons sous-entendu 
que les m\'ecanismes complexes qui d\'eterminent le mouvement de la 
pi\`ece ou du d\'e pouvaient \^etre {\it isol\'es} du reste du monde: il 
\'etait ``\'evident'' que l'air siffl\'e par l'observateur n'influait pas
sur le processus qu'il \'etait en train d'observer. Si l'air siffl\'e par  
l'observateur  changeait la loi de probabilit\'e, nous ne pourrions plus  
dire que le lancement du d\'e est une exp\'erience  reproductible: il 
faudrait consid\'erer une exp\'erience \'elargie, consistant \`a siffler 
toujours le m\^eme air chaque fois qu'on lance le d\'e.  
Ce qui rend une exp\'erience reproductible est son isolement par rapport  
au reste de l'univers. Cette notion d'exp\'erience reproductible  est le 
paradigme fondamental du Calcul des probabilit\'es, nous l'avons 
rencontr\'e d\`es le d\'ebut de cet ouvrage: en {\bf I.2}, ce qui distingue  
la distribution de trois boules  dans deux bo\^\i tes de la distribution  
de trois bosons entre deux \'etats quantiques de m\^eme \'energie peut 
s'exprimer ainsi: dans le cas des boules, on r\'ep\`ete bien trois fois la 
m\^eme exp\'erience, mais pas  dans le cas des bosons.  En effet, la 
mani\`ere  dont le premier boson remplit les deux \'etats modifie les 
conditions de l'exp\'erience pour  le deuxi\`eme boson, donc  
l'exp\'erience n'est pas r\'ep\'et\'ee \`a  l'identique. Avec les boules,  
bien que la premi\`ere  devait choisir entre deux bo\^\i tes vides alors  
que la seconde doit  choisir  entre deux bo\^\i tes dont l'une est d\'ej\`a 
occup\'ee, l'exp\'erience  se reproduit pourtant, car l'\'etat d'occupation 
des deux  bo\^\i tes n'a pas d'influence. Autrement dit, le fait que  
l'exp\'erience soit  reproductible est li\'e \`a la {\it s\'eparation  
causale} entre les boules;  \`a l'inverse il n'y a pas de s\'eparation  
causale entre les bosons.     
 
\vskip6mm plus3mm minus3mm 

{\bf XIII. 2. Interpr\'etation de J. S. Bell.}
\medskip 
Afin de d\'ecrire la situation imagin\'ee par Einstein, Podolski, et  
Rosen,  o\`u le processus de cr\'eation de la paire de jumeaux serait 
reproductible sans y inclure la position des  analyseurs, Bell propose de 
consid\'erer une variable al\'eatoire $\lambda$ dont on  ne pr\'ecise pas 
la nature (elle peut \^etre vectorielle, de dimension aussi \'elev\'ee 
qu'on voudra), qui d\'ecrit l'\'etat interne de chaque jumeau: comme les 
jumeaux sont suppos\'es identiques ou plus exactement parfaitement 
sym\'etriques l'un de  l'autre, on peut dire que la m\^eme valeur de 
$\lambda$ les caract\'erise tous les deux.  Cette approche est exactement
celle que nous avons discut\'ee plusieurs fois d\'ej\`a dans cet 
ouvrage: un processus est caract\'eris\'e par des probabilit\'es a priori, 
qui restent les m\^emes chaque fois qu'on le reproduit dans des conditions
identiques, et qui par cons\'equent co\"\i ncideront avec les probabilit\'es
empiriques mesur\'ees sur un grand nombre de r\'ep\'etitions ind\'ependantes 
du processus (cf. {\bf X.3}). Bell applique soigneusement le paradigme
du Calcul des probabilit\'es.
\medskip 
Si Bell ne pr\'ecise pas la nature de $\lambda$, c'est pour atteindre le 
maximum de g\'en\'eralit\'e. S'il la tient pour une variable al\'eatoire,  
c'est parce qu'il admet que la cr\'eation des jumeaux puisse ne pas   
\^etre  elle-m\^eme d\'eterministe.  L'id\'ee sous-jacente \'etait que  
la connaissance pr\'ealable d'un \'etat interne devait permettre de  
pr\'edire l'issue de la mesure (le processus qui se d\'eroule \`a 
l'int\'erieur d'un aimant analyseur pendant que la particule le traverse), 
comme la connaissance exacte de la trajectoire d'un d\'e devrait 
permettre de pr\'edire sur quelle face il s'arr\^etera. Mais on accepte   
que l'\'etat interne puisse  \^etre lui-m\^eme trop difficile \`a pr\'evoir  
\`a partir du m\'ecanisme d\'etaill\'e de la division, tout comme la 
trajectoire du d\'e est trop difficile \`a pr\'evoir \`a partir des  
conditions initiales. 
\medskip
Autrement dit,  Bell se propose de tester r\'eellement une hypoth\`ese
de principe.  Pour y parvenir,  il s'efforce de ne rien pr\'esupposer
en dehors de cette pr\'edictibilit\'e de principe. 
\medskip
Donc Bell suppose que si $\lambda$ est al\'eatoire, du moins sa valeur   
va d\'eterminer univoquement l'issue de la mesure; cela se traduit 
math\'ematiquement par le fait que le signe de la d\'eviation doit \^etre  
pour chacun des deux jumeaux une fonction de $\lambda$. Mais comme  
on admet aussi que la d\'eviation peut \^etre influenc\'ee par l'aimant 
travers\'e (c'est-\`a-dire que le r\'esultat de  la mesure n'est pas une 
propri\'et\'e de la particule isol\'ee et que la mesure perturbe celle-ci), 
on doit donc consid\'erer que le signe $X$ de  la d\'eviation est une  
fonction non seulement de $\lambda$, mais aussi de l'orientation $a$  
de l'analyseur, qu'on note $X(a,\lambda )$. Ainsi $\lambda$ est une  
variable al\'eatoire, mais $X$ est une fonction univoquement  
d\'etermin\'ee de $a$ et de $\lambda$. Les deux jumeaux \'etant  
sym\'etriques mais non identiques, la mani\`ere dont $X$ d\'epend de 
$\lambda$ n'est pas forc\'ement la m\^eme \`a  droite qu'\`a gauche;  
par exemple on pourrait avoir $X_D\big( a,\lambda\big) = X_G \big( 
a,-\lambda\big)$, o\`u plus g\'en\'eralement $X_D\big( a,\lambda \big)  
= X_G\big( a,\sigma(\lambda )\big)$ o\`u $\sigma$ d\'esigne une   
sym\'etrie de l'espace o\`u  $\lambda$ prend  ses valeurs. Le point 
essentiel est que  $X_G\big( a,\lambda \big)$ d\'epend de l'orientation  
$a$ de l'analyseur de gauche, mais pas de l'orientation de l'analyseur de 
droite;  de m\^eme $X_D\big( b,\lambda \big)$ d\'epend de l'orientation 
$b$ de l'analyseur de droite,  mais pas  de l'orientation de l'analyseur de 
gauche.    
\medskip 
Que signifient ces hypoth\`eses? Dire que $\lambda$ est une variable  
al\'eatoire, et que $X$ est une fonction des variables $a$ et $\lambda$, 
est du langage math\'ematique. Mais si nous voulons interpr\'eter 
correctement l'exp\'erience E.P.R. nous devons comprendre le sens 
concret de ces formulations math\'ematiques. Lorsqu'on dit que  
$X$ est une fonction des variables $a$ et $\lambda$, cela signifie  (c'est 
la {\it d\'efinition} math\'ematique d'une fonction) que si on donne \`a  
 $a$  et \`a $\lambda$ des valeurs d\'etermin\'ees, alors $X(a,\lambda )$ 
aura une valeur d\'etermin\'ee. En langage imag\'e, cela signifie que, une 
fois fix\'ees les valeurs de $a$ et $\lambda$, aucun hasard n'intervient 
plus pour la d\'etermination de la valeur de $X(a,\lambda )$. Mais 
$\lambda$ elle-m\^eme est, dans les hypoth\`eses de Bell, une variable 
al\'eatoire,  dont la valeur sera choisie par le hasard au cours du 
processus de cr\'eation de la paire de jumeaux. Ainsi, les hypoth\`eses 
exprim\'ees ci-dessus dans le langage math\'ematique se traduisent dans 
le langage imag\'e par: ``le hasard intervient {\it une seule fois} dans  
toute l'exp\'erience E.P.R. et ce moment est celui de la cr\'eation de la 
paire; il n'intervient plus ensuite; en particulier il n'intervient plus 
pendant la travers\'ee de l'analyseur''. 
\medskip 
On peut encore proposer une l\'eg\`ere variante: admettre que $a$ est  
aussi une variable al\'eatoire. Rappelons un point important  
de l'exp\'erience d'Aspect: elle a \'et\'e construite de telle sorte que  
l'orientation des polariseurs soit  modifi\'ee al\'eatoirement par 
des commutateurs ultra-rapides ($\sim 10^8$ commutations par  
seconde)  sous les conditions suivantes:    
\smallskip 
a) elle est modifi\'ee {\it apr\`es} la  cr\'eation de la paire par 
d\'esexcitation  de l'atome de  Calcium;   
\smallskip 
b) la commutation est effectu\'ee 
de telle sorte que le changement  de direction \`a gauche et la mesure  
\`a droite (ou vice-versa) soient des \'ev\'enements s\'epar\'es  par un 
intervalle du genre espace, autrement dit qu'un signal lumineux parti  du 
polariseur de gauche au moment de sa commutation al\'eatoire parvienne  
au  polariseur de  droite apr\`es la mesure du jumeau de droite.  
\smallskip 
Dans ce cas le hasard intervient aussi au moment du choix, pour chaque 
analyseur, entre deux orientations. Mais le point essentiel demeure: le 
hasard n'intervient pas pendant la travers\'ee des analyseurs; en un mot: 
les jeux sont faits avant.  
\medskip 
On peut donc r\'esumer la {\it signification} des hypoth\`eses de Bell en 
disant simplement:  
\smallskip 
1. La mesure effectu\'ee sur chaque jumeau est ind\'ependante de la  
position de l'autre analyseur.  
\smallskip 
2. Aucun hasard n'intervient plus pendant la travers\'ee des analyseurs. 
\medskip 
Enfin, une autre variante encore consiste \`a renoncer m\^eme \`a la
deuxi\`eme condition ci-dessus. Nous l'\'etudierons plus loin; mais
l\`a on renonce alors \`a la possibilit\'e d'une pr\'ediction \`a partir d'un
hypoth\'etique \'etat intrins\`eque, donc on s'\'ecarte du postulat E.P.R.
Aujourd'hui on se rend compte, comme nous verrons plus loin, que la
question n'est pas du tout li\'ee \`a la seconde condition ci-dessus; mais
elle \'etait essentielle dans la conception d'Einstein, Podolski, et Rosen.
\medskip       
Nous rapportons d'abord ci-apr\`es le raisonnement math\'ematique
que Bell a publi\'e en {\oldstyle 1964}. Ensuite nous passerons aux
variantes.
\medskip 
Afin d'obtenir un test effectif de son hypoth\`ese, Bell propose alors de 
consid\'erer la corr\'elation entre $X_G$ et $X_D$ pour des orientations 
quelconques $a$ \`a gauche et $b$ \`a droite.  Soit $\{ \lambda_j\; , \; 
(p_j) \}$ la loi de probabilit\'e de $\lambda$. La corr\'elation est 
$${\bf E}(a,b) = {\bf E}\big( X_G(a, \lambda ) \cdot X_D(b, \lambda )  
\big) = \sum_j p_j\, X_G(a, \lambda_j ) \cdot X_D(b, \lambda_j )
\eqno (XIII.1.)$$ 
Ces corr\'elations sont mesurables exp\'erimentalement,  
puisqu'il suffit de r\'ep\'eter un grand nombre $N$ de fois l'exp\'erience 
et d'\'etablir la corr\'elation empirique entre les r\'esultats de gauche  
et ceux de droite. Certes, leur mesure effective est difficile car dans 
une exp\'erience concr\`ete les r\'esultats significatifs sont m\^el\'es  
\`a de nombreux ph\'enom\`enes ind\'esirables et doivent \^etre 
soigneusement tri\'es; mais nous faisons confiance aux exp\'erimentateurs. 
Les r\'esultats de ces $N$ exp\'eriences peuvent \^etre enregistr\'es sous 
la forme d'un tableau comme celui pr\'esent\'e plus  haut (page 385): 
la corr\'elation empirique $R$ est la moyenne des produits 
des \'el\'ements de la deuxi\`eme colonne du tableau par les \'el\'ements 
correspondants de la troisi\`eme colonne. D'apr\`es la loi des grands 
nombres (puisque l'exp\'erience est reproductible) la corr\'elation 
empirique $R$ sera d'autant plus proche de la corr\'elation th\'eorique 
${\bf E}(a,b)$ que $N$ sera  plus grand, les \'ecarts sup\'erieurs \`a  
plusieurs fois $\sqrt{N}$ ne se produisant pratiquement jamais.    
\medskip 
Supposons qu'on effectue quatre s\'eries d'exp\'eriences: dans la  
premi\`ere on orientera l'analyseur selon $a_1$ \`a gauche et $b_1$ \`a 
droite; dans la seconde $a_1$ \`a gauche et $b_2$ \`a droite; dans la 
troisi\`eme $a_2$  \`a gauche et $b_1$ \`a droite; dans la quatri\`eme 
$a_2$ \`a gauche et $b_2$ \`a droite. On peut \'ecrire:  
$$\eqalignno{ 
{\bf E}&(a_1,b_1) - {\bf E}(a_1,b_2) = &(XIII.2.)\cr 
\noalign{\bigskip} 
&= \sum_j p_j\, \Big[ X_G(a_1,\lambda_j) \cdot X_D(b_1,\lambda_j) - 
X_G(a_1,\lambda_j) \cdot X_D(b_2,\lambda_j) \Big] \cr 
\noalign{\medskip} 
&= \sum_j p_j\, \Big[ X_G(a_1,\lambda_j) \cdot X_D(b_1,\lambda_j)  
\Big( 1\pm X_G(a_2,\lambda_j)\cdot X_D(b_2,\lambda_j)\Big)\Big]\cr 
&\hskip6mm - \sum_j p_j\, \Big[ X_G(a_1,\lambda_j) \cdot 
X_D(b_2,\lambda_j)  \Big( 1 \pm X_G(a_2,\lambda_j) \cdot 
X_D(b_1,\lambda_j) \Big) \Big] \cr }$$ 
On a simplement rajout\'e les termes pr\'ec\'ed\'es de $\pm$ qui  
s'annulent  mutuellement. \'Etant donn\'e que $X_G(a,\lambda_j)$ et 
$X_D(b,\lambda_j)$ ne prennent jamais d'autre valeur que $+1$ ou $-1$, 
les expressions de la forme $1\pm X_G(a,\lambda_j)\cdot 
X_D(b,\lambda_j)$ sont toutes positives et les expressions en facteur  
de la forme $X_G(a,\lambda_j) \cdot X_D(b,\lambda_j)$ sont \'egales \`a  
1 en valeur absolue, donc d'apr\`es l'in\'egalit\'e de la moyenne: 
$$\eqalignno{ 
|{\bf E}(a_1,b_1) - {\bf E}(a_1,b_2)|\quad 
&\leq\quad\sum_j p_j\,  
\Big( 1\pm X_G(a_2,\lambda_j)\cdot X_D(b_2,\lambda_j)\Big) + \cr 
&\hskip6mm + \sum_j p_j\, \Big( 1 \pm X_G(a_2,\lambda_j) \cdot 
X_D(b_1,\lambda_j) \Big) \cr  
\noalign{\medskip} 
&\leq\quad 2 \pm \Big( {\bf E}(a_2,b_2) + {\bf E}(a_2,b_1) \Big) 
&(XIII.3.)\cr }$$ 
d'o\`u on d\'eduit 
$$\eqalignno{ 
|{\bf E}(a_1,b_1) - {\bf E}(a_1,b_2) + {\bf E}(a_2,b_2) + {\bf 
E}(a_2,b_1)| \quad &\leq \quad 2 \hskip16mm &(XIII.4.)\cr 
|-{\bf E}(a_1,b_1) + {\bf E}(a_1,b_2) + {\bf E}(a_2,b_2) + {\bf 
E}(a_2,b_1)| \quad &\leq \quad 2 \hskip16mm &(XIII.5.)\cr }$$ 
Cela reste \'evidemment vrai si on permute $a_1$ avec $a_2$ ou $b_1$  
avec $b_2$, donc on a forc\'ement aussi 
$$\eqalignno{ 
|{\bf E}(a_1,b_1) + {\bf E}(a_1,b_2) - {\bf E}(a_2,b_2) + {\bf 
E}(a_2,b_1)| \quad &\leq \quad 2 \hskip16mm  &(XIII.6.)\cr 
|{\bf E}(a_1,b_1) + {\bf E}(a_1,b_2) + {\bf E}(a_2,b_2) - {\bf 
E}(a_2,b_1)| \quad &\leq \quad 2 \hskip16mm  &(XIII.7.)\cr }$$ 
N'importe laquelle de ces quatre in\'egalit\'es est appel\'ee {\it  
in\'egalit\'e de Bell}. Elles r\'esultent donc math\'ematiquement des 
hypoth\`eses  pr\'ealables, \`a savoir que $X_G$ et $X_D$ sont des {\it 
fonctions} de $\lambda$ et de l'orientation de l'analyseur respectif.  
\medskip 
Le probl\`eme est maintenant que ces in\'egalit\'es sont viol\'ees par la 
M\'ecanique quantique. Celle-ci pr\'evoit en effet que ${\bf E}(a,b) = 
-\cos\theta$, o\`u $\theta$ est l'angle entre les directions $a$ et $b$. 
D'apr\`es la M\'ecanique quantique on aurait donc 
$$\eqalignno{ 
Q &= |{\bf E}(a_1,b_1) + {\bf E}(a_1,b_2) + {\bf E}(a_2,b_2) - {\bf 
E}(a_2,b_1)|  &(XIII.8.)\cr 
&= |\cos\theta_1 + \cos\theta_2 + \cos\theta_3 - 
\cos\theta_4|  &(XIII.9.)\cr }$$ 
o\`u $\theta_1$ est l'angle entre $a_1$ et $b_1$, 
$\theta_2$ l'angle entre $a_1$ et $b_2$, $\theta_3$ l'angle entre $a_2$ 
et $b_1$, et $\theta_4$ l'angle entre $a_2$ et $b_2$.  
Choisissons alors ces quatre angles comme sur la figure 59.  
\medskip 
\midinsert 
\epsfxsize=\hsize
\line{\epsfbox{../images/fig59.eps} } 
\centerline{\eightpoint figure 59} 
\vskip8pt 
\endinsert 
Cette configuration particuli\`ere des quatre directions $a_1$, $a_2$, 
$b_1$, et $b_2$ correspond au maximum de la  
combinaison $XIII.9.$ des quatre cosinus. Dans cette configuration 
particuli\`ere on trouve que $Q = 2\sqrt{2}$. Or, d'apr\`es 
l'in\'egalit\'e $XIII.7$, on devrait avoir $Q \leq 2$. 
\medskip 
Nous ne pr\'esenterons pas ici une d\'emonstration d\'eductive du fait  
que  ${\bf E}(a,b) = -\cos\theta$  \`a partir du formalisme de la 
M\'ecanique quantique. Mais indiquons-en les raisons. Par des 
consid\'erations g\'en\'erales d'invariance, on peut conclure que  les 
corr\'elations ${\bf E}(a,b)$ ne doivent d\'ependre que de l'angle  
$\theta$ entre  les deux directions. Cela r\'esulte simplement de ce  
que les r\'esultats doivent \^etre les m\^emes si on tourne globalement  
tout le dispositif dans l'espace (sinon l'exp\'erience distinguerait une 
direction  privil\'egi\'ee). Par ailleurs, d'apr\`es un principe de la 
M\'ecanique  quantique appel\'e principe de correspondance, le spin de la 
particule  doit s'identifier \`a un moment cin\'etique lorsqu'on prend sa 
moyenne  sur un grand nombre de particules. C'est ce principe qui impose 
que la fonction de  $\theta$ mentionn\'ee soit obligatoirement  
$-\cos\theta$.  
\medskip 
La nature se comporte comme le dit la M\'ecanique quantique, et non 
selon les hypoth\`eses qui conduisent \`a l'in\'egalit\'e  de Bell. La  
discussion des conditions de l'exp\'erience devrait maintenant nous 
permettre de voir ce qui est faux dans ces hypoth\`eses. Puisque Bell les  
a exprim\'ees sous une forme math\'ematique rigoureuse, il devrait  
\^etre possible de trouver l'origine exacte de l'erreur. Deux conclusions 
possibles viennent imm\'ediatement \`a l'esprit: 
\smallskip 
{\bf Possibilit\'e\a\a N${}^\circ\!\!$ 1.} Nous avons suppos\'e que
$X_G(a,\lambda )$ ne d\'ependait pas de $b$,  ni $X_D(b,\lambda )$ de $a$; 
pour expliquer la violation de l'in\'egalit\'e   de Bell on pourrait donc
admettre que $X_G$ et $X_D$ sont des fonctions  de  $a$ {\it et} de $b$, 
$X_G(a,b,\lambda )$ et $X_D(b,a,\lambda )$.  On peut  en effet se convaincre
facilement que la d\'emonstration ci-dessus de  l'in\'egalit\'e ne marche
plus quand $X_G$ ou $X_D$ d\'epend \`a la fois  de $a$ et de $b$. 
\smallskip 
{\bf Possibilit\'e\a\a N${}^\circ\!\!$ 2.} Nous avons vu que l'hypoth\`ese
revenait \`a exclure toute action du hasard post\'erieure \`a la cr\'eation
de la paire ou \`a la fixation de l'orientation des analyseurs. On peut
donc \'egalement renoncer \`a cela.  Ainsi, si on exclut la possibilit\'e
$N^\circ 1$,  la M\'ecanique quantique impliquerait que le hasard intervient
{\it n\'ecessairement} pendant que la particule traverse l'aimant, et
l'exp\'erience d'Aspect prouverait  d\'efinitivement qu'il en est bien ainsi. 
\medskip 
Il s'agit l\`a des deux possibilit\'es qui viennent le plus naturellement
\`a l'esprit;  du moins d'apr\`es ce qu'on peut en juger en voyant
l'\'evolution historique du d\'ebat [je soutiens plut\^ot que l'erreur
est dans l'hypoth\`ese {\it non explicitement formul\'ee} que les
probabilit\'es $p_j$ qui interviennent dans $XIII.1$, $XIII.2$, et
$XIII.3$, sont ind\'ependantes  de $a$ et $b$;  cette hypoth\`ese fausse
est cependant logique si on part du principe que les jeux sont faits au
moment de la cr\'eation des jumeaux].  Toutes les sp\'eculations
sans fin sur {\og l'influence \fg} qu'exercerait chacun des deux
analyseurs sur l'autre ont \'et\'e inspir\'ees par la prise en compte
exclusive et {\it a priori},  sans effort critique,  de la possibilit\'e
$N^\circ 1$.  Certes la possibilit\'e $N^\circ 2$ aurait permis une
\'echappatoire,  mais elle ne r\'esoud rien.  Avant d'entreprendre
la discussion \`a la section suivante,  voyons tout de suite
pourquoi cela ne r\'esoud rien. 
\medskip 
C'est justement dans le but de tester la possibilit\'e $N^\circ 2$ que des
extensions des  hypoth\`eses de Bell ont \'et\'e \'etudi\'ees dans les
ann\'ees {\oldstyle 1970}. Nous les rapportons ci-dessous sous la forme la
plus g\'en\'erale possible. 
\medskip 
Il s'agit de voir si on peut pr\'eserver la description par un \'etat  
interne qui conserverait la corr\'elation entre les deux jumeaux sous la 
forme d'un param\`etre $\lambda$,  en renon\c{c}ant \`a ce que souhaitait 
Einstein, c'est-\`a-dire en renon\c{c}ant \`a l'id\'ee que cet \'etat interne 
d\'etermine la d\'eviation. Comme nous avons pu nous en convaincre, 
ce renoncement s'exprime par la possibilit\'e $N^\circ 2$, que le hasard 
intervienne une seconde fois pendant la travers\'ee des aimants. 
Autrement dit, nous n'exigeons plus que le param\`etre $\lambda$ 
d\'etermine le signe $+$ ou $-$ de la d\'eviation, nous exigeons  
seulement encore qu'il d\'etermine la loi de probabilit\'e de ce signe. 
Bien entendu,  la loi de probabilit\'e en question d\'epend aussi de 
l'orientation de l'analyseur, {\it mais pas de celui travers\'e par l'autre 
jumeau}. Cela se traduira  par l'ind\'ependance stochastique entre tout 
\'ev\'enement relatif \`a l'analyseur de gauche et tout \'ev\'enement 
relatif \`a l'analyseur de  droite. 
\medskip 
Math\'ematiquement,  on le formulera ainsi:  pour une valeur de $\lambda$ 
fix\'ee,  appelons $q(a,\lambda )$ la probabilit\'e pour que la d\'eviation
dans l'aimant de gauche soit $-\,$, et $1 -  q(a,\lambda )$ la probabilit\'e
pour que la d\'eviation soit $+\,$.  Autrement dit nous supposons maintenant
que la fonction $X_G(a,\lambda)$ est toujours une fonction de $a$ et
$\lambda$,  mais \`a valeurs al\'eatoires,  de loi $\{ +1, (1-q)\; ;\; 
-1, (q) \}$.  Si $\{ \lambda_j\, , \, (p_j) \}$ est la loi de probabilit\'e
de $\lambda$,  la r\`egle des probabilit\'es conditionnelles $IV.4$, 
dont nous avons vu qu'elle \'etait ind\'ependante de la mani\`ere dont
le hasard intervenait,  nous dit que la probabilit\'e d'avoir le m\^eme
signe de d\'eviation pour les deux jumeaux est   
$$\eqalignno{
&P_{+}(a,b) =  &(XIII.10.)\cr
&\hskip5mm = \sum_j \Big\{ \big[ 1 - q(a,\lambda_j)\big] \big[ 1 -
q(b,\lambda_j) \big] + q(a,\lambda_j)\, q(b,\lambda_j) \Big\}\; p_j\cr}$$  
En effet, cela est bien l'application de la r\`egle $IV.4$ en prenant  
comme famille exhaustive d'\'ev\'enements les $E_j$: ``la  variable 
$\lambda$ prend la valeur $\lambda_j$''. L'ind\'ependance causale
des \'ev\'enements relatifs \`a des analyseurs diff\'erents se traduit 
math\'ematiquement par l'ind\'ependance stochastique,  c'est-\`a-dire 
par les produits $\big[ 1 - q(a,\lambda_j)\big] \big[ 1 - 
q(b,\lambda_j)\big]$ (probabilit\'e d'avoir $+$ \`a gauche et $+$ \`a 
droite) et $q(a,\lambda_j)\, q(b,\lambda_j)$ (probabilit\'e d'avoir $-$ 
\`a gauche  et $-$ \`a droite). Enfin, la somme de ces produits exprime 
que les deux \'ev\'enements ``$+$ \`a gauche et $+$ \`a droite'' et ``$-$ 
\`a gauche et $-$ \`a droite'' sont disjoints. Nous avons ainsi appliqu\'e 
\`a la lettre les principes du Calcul des probabilit\'es.  
\medskip 
La loi $\{ \lambda_j\; , \; (p_j) \}$ qui d\'ecrit le hasard 
pendant la cr\'eation de la paire, et la loi des la variables maintenant
al\'eatoires $X_G(a,\lambda_j)$ et $X_D(b,\lambda_j)$ qui d\'ecrit  
une  nouvelle intervention du hasard pendant la travers\'ee de l'aimant, 
peuvent \^etre absolument quelconques: la seule contrainte qu'on leur 
impose est d'\^etre des lois de probabilit\'e, c'est-\`a-dire que 
$$\eqalign{ &(1)\qquad 
\sum_j p_j = 1\quad \hbox{et\quad pour tout $j$:}\quad p_j \geq 0\cr    
&(2)\qquad\hbox{pour tout $j$ et pour toute direction $a$:}\quad  
0 \leq q(a,\lambda_j) \leq 1\cr }$$ 
On peut alors montrer que {\it sous cette seule contrainte}, 
les corr\'elations possibles issues de ces lois,  $R(a,b) = P_{+}(a,b) -
P_{-}(a,b)$, satisferont aussi l'in\'egalit\'e de Bell.
La d\'emonstration est semblable \`a celle de Bell que nous avons donn\'ee 
plus haut, elle n'en diff\`ere que par une complication technique.  
Calculons en effet, comme nous l'avions fait en vue de l'in\'egalit\'e de 
Bell proprement dite (cf. $XIII.1$), la corr\'elation ${\bf E}(a,b)$ dans ce 
nouveau cas. On a cette fois d'apr\`es $XIII.10$:
$$\eqalignno{
{\bf E}(a,b) &=  P_{+}(a,b) - P_{-}(a,b) =   &(XIII.11) \cr
\noalign{\vskip12pt}
&= \sum_j \Big\{ 1 - 2\, q(a,\lambda_j) - 2\, q(b,\lambda_j)  
+ 4\, q(a,\lambda_j)\, q(a,\lambda_j)\Big\}\, p_j \cr
&= \sum_j \Big\{ \big[ 1 - 2\, q(a,\lambda_j)\big] \cdot
\big[ 1 - 2\, q(b,\lambda_j)\big] \Big\}\, p_j \cr }$$
Remarquons maintenant que pour obtenir l'in\'egalit\'e de Bell
nous pouvons proc\'eder exactement de la m\^eme mani\`ere que dans
$XIII.2$ et $XIII.3$, en rempla\c{c}ant formellement $X_G(a,\lambda_j)$
par $1 - 2\, q(a,\lambda_j)$ et $X_D(b,\lambda_j)$ par $1 - 2\,
q(b,\lambda_j)$. En effet, en dehors du simple jeu combinatoire des $a_1, 
a_2, b_1, b_2$ qui se reproduit ici, la seule propri\'et\'e  des variables
$X_G(a,\lambda_j)$  et $X_D(b,\lambda_j)$ qui intervenait \'etait 
d'\^etre dans l'intervalle $[ -1 ,+1]$. Or puisque $q(a,\lambda_j)$ et
$q(b,\lambda_j)$ sont des {\it probabilit\'es}, elles sont comprises entre
$0$ et $1$, donc les quantit\'es $1 - 2\, q(a,\lambda_j)$ et $1 - 2\,
q(b,\lambda_j)$  seront comprises entre  $-1$ et $+1$, ce qui suffit \`a
entra\^\i ner les diff\'erentes in\'egalit\'es de Bell. 
\medskip 
Autrement dit, la possibilit\'e $N^\circ 2$ est incompatible avec les faits. 
\medskip
On voit que l'in\'egalit\'e de Bell sera v\'erifi\'ee d\`es lors que la 
corr\'elation ${\bf E}(a,b)$ peut s'\'ecrire sous la forme $\sum_j f(a,j)\,
f(b,j)\, p_j$ o\`u les $p_j$ sont $\geq 0$ et tels que $\sum_j  p_j = 
1$,  et o\`u la fonction $(a,j) \mapsto f(a,j)$ prend ses valeurs dans
l'intervalle $[-1 , +1]$. Les quantit\'es $p_j$ peuvent \^etre n'importe
quoi,  de m\^eme que les $f(a,j)$: pourvu qu'elles v\'erifient ces
conditions, l'in\'egalit\'e de Bell aura lieu. Ceci est {\it ind\'ependant 
de toute interpr\'etation} des quantit\'es $p_j$ et $f(a,j)$.
\medskip      
Comme il s'agit d'une corr\'elation de signes, on peut toujours la
d\'ecom\-po\-ser en ${\bf E}(a,b) = P_{a,b}(++) + P_{a,b}(--) - P_{a,b}(+-) -
P_{a,b}(-+)$, o\`u $P_{a,b}(++)$ d\'esigne la probabilit\'e d'avoir le signe 
$+$ dans les deux analyeurs, $P_{a,b}(+-)$ celle d'avoir $+$ dans
l'analyseur de gauche et $-$ dans celui de droite, etc.  Ces probabilit\'es
d\'ependent comme il se doit des directions $a$ et $b$. Il est facile de
v\'erifier que si on a 
$$\eqalignno{
P_{a,b}(++) &= \sum_j P_{a}(+|\, j) \, P_{b}(+|\, j)\, p_j \cr
P_{a,b}(+-) &= \sum_j P_{a}(+|\, j) \, P_{b}(-|\, j)\, p_j \cr
P_{a,b}(-+) &= \sum_j P_{a}(-|\, j) \, P_{b}(+|\, j)\, p_j \cr
P_{a,b}(--) &= \sum_j P_{a}(-|\, j) \, P_{b}(-|\, j)\, p_j &(XIII.12.)\cr }$$ 
alors on aura
$$\eqalignno{
{\bf E}(a,b) &= \sum_j \Big\{ P_{a}(+|\, j) \, P_{b}(+|\, j) - 
P_{a}(+|\, j) \, P_{b}(-|\, j) - \null\cr
\noalign{\vskip-10pt}
&\hskip20mm - P_{a}(-|\, j) \, P_{b}(+|\, j) + 
P_{a}(-|\, j) \, P_{b}(-|\, j) \Big\}\, p_j \cr
\noalign{\vskip8pt}
&= \sum_j \big[ P_{a}(+|\, j)  - P_{a}(-|\, j) \big]\, 
\big[ P_{b}(+|\, j)  - P_{b}(-|\, j) \big]\, p_j \cr }$$
ce qui est bien de la forme $\sum_j f(a,j)\, f(b,j)\, p_j$ avec $f(a,j) = 
P_{a}(+|\, j) - P_{a}(-|\, j)$. Si les nombres $P_{a}(+|\, j)$, 
$P_{a}(-|\, j)$, $P_{b}(+|\, j)$, et $P_{b}(-|\, j)$ sont compris
entre $0$ et $1$, les nombres  $f(a,j) = P_{a}(+|\, j) - P_{a}(-|\, j)$
et $f(b,j) = P_{b}(+|\, j) - P_{b}(-|\, j)$ seront, eux, compris entre
$-1$ et $+1$; si en outre les $p_j$ sont tous compris entre $0$ et $1$ et
tels que $\sum_j p_j = 1$,   l'in\'egalit\'e de Bell sera {\it forc\'ement}
v\'erifi\'ee, ind\'ependamment de toute interpr\'etation des quantit\'es
$p_j$,  $P_{a}(\pm|\, j)$, $P_{b}(\pm|\, j)$.
\medskip
Il se trouve maintenant que $(XIII.12)$ peut s'interpr\'eter comme la
formule des probabilit\'es conditionnelles $IV.4$, avec la famille
exhaustive d'\'ev\'enements $E_j : \{ \lambda = \lambda_j\}$. Il y a
toutefois un d\'etail sp\'ecifique: les probabilit\'es conditionnelles
$P_{a,b}(++ |\, E_j)$,  $P_{a,b}(+- |\, E_j)$, etc, ont \'et\'e 
d\'ecompos\'ees sous la forme $P_{a,b}(++ |\, E_j) = P_{a}(+ |\, j)
\cdot P_{b}(+ |\, j)$, etc. Cela semble logique car c'est la propri\'et\'e
du produit qui  caract\'erise l'ind\'ependance stochastique, cens\'ee
refl\'eter ici l'ind\'ependance causale entre ce qui se passe dans chacun
des deux analyseurs.
\medskip
La relation purement formelle et ind\'ependante de toute interpr\'etation
qui conduit de $(XIII.12)$ \`a l'in\'egalit\'e de Bell montre que si cette
derni\`ere est fausse, c'est qu'il y a quelque chose de faux dans $(XIII.12)$.
Or il reste encore des hypoth\`eses suspectes. 
\smallskip
{\bf Possibilit\'e\a\a N${}^{\circ}\!\!\!$ 3.} Est-il l\'egitime
d'appliquer la propri\'et\'e du produit $P_{a,b}(++ |\, E_j) =
P_{a}(+ |\, j) \cdot P_{b}(+ |\, j)$, etc ?
\smallskip
{\bf Possibilit\'e\a\a N${}^{\circ}\!\!$ 4.} Est-il l\'egitime
d'appliquer la formule des probabilit\'es conditionnelles avec une
famille exhaustive $E_j$ ?
\smallskip
{\bf Possibilit\'e\a\a N${}^{\circ}\!\!$ 5.} Est-il l\'egitime de
supposer que les probabilit\'es $p_j$ sont {\it ind\'ependantes} des
directions $a$ et $b$?
 

\vskip6mm plus3mm minus3mm
 
{\bf XIII. 3. Discussion.}
\medskip
Dans cette discussion, nous laisserons de c\^ot\'e la question de
savoir si la formulation math\'ematique de Bell correspond bien \`a
l'id\'ee que se faisaient Einstein, Podolski, et Rosen. Cela est  
sans importance pour nous ici, puisque nous discutons le paradoxe
E.P.R.  uniquement pour mieux comprendre la vraie nature de la
causalit\'e et des probabilit\'es, et non dans un but de recherche
historique. Ce que  nous devons comprendre est la signification
exacte des hypoth\`eses de Bell, et par contrecoup, la signification
exacte des probabilit\'es qui interviennent, aussi bien dans les
hypoth\`eses fausses de Bell que dans les corr\'elations vraies
pr\'edites par la M\'ecanique quantique. 
\medskip
Il s'agit simplement de passer en revue les hypoth\`eses,  explicites
ou non,  qui permettent la d\'emonstration,  expos\'ee ci-dessus, 
des in\'egalit\'es de Bell.  Commen\c{c}ons donc par la possibilit\'e
$N^\circ 1$.  Elle est surprenante. Elle revient \`a admettre que
la travers\'ee d'un analyseur par un jumeau est influenc\'ee par la  
position de l'autre analyseur, ce qui est une violation flagrante de
la causalit\'e. En effet, dire que pour une valeur $\lambda_j$ de
$\lambda$, le signe $X_G$ de la d\'eviation d\'epend aussi de $b$
implique une influence. Or on ne peut \'echapper au raisonnement
suivant: ``les deux jumeaux sont rigoureusement sym\'etriques et
ignorent l'orientation que prendra l'analyseur de Stern-Gerlach ou
le polariseur d'Aspect lorsqu'ils le traverseront.  Comme ces
ana\-ly\-seurs sont tr\`es \'eloign\'es l'un de  l'autre ($14\, m.$
dans l'exp\'erience d'Aspect), la mesure sur un jumeau ne peut pas
influencer la mesure sur l'autre. Or, les corr\'elations en
$-\cos\theta$ prouvent le contraire, etc.'' Une telle influence \`a
distance contredit le principe de Relativit\'e selon lequel aucune
influence ne peut se propager plus vite que la lumi\`ere; c'est sans
aucun doute la raison pour laquelle Einstein a eu tant de mal \`a 
l'accepter. Dans l'une des versions de l'exp\'erience d'Aspect  les
polariseurs ont \'et\'e  sp\'ecialement \'equip\'es de commutateurs
optiques capables de  modifier l'orientation en $10^{-8}$ secondes, 
de sorte que si cette modification se produit \`a droite \`a un
instant $t$,  un signal se  d\'epla\c{c}ant \`a la vitesse de la
lumi\`ere ne pourra atteindre l'autre polariseur situ\'e \`a $14$
m\`etres de l\`a que $5 \cdot 10^{-8}$  secondes plus tard; par
cons\'equent, si la commutation d'orientation a eu  lieu \`a droite
apr\`es la cr\'eation de la  paire de photons jumeaux,  mais {\it
avant} que le jumeau de droite ait  travers\'e le polariseur,  le
signal apportant l'information du changement arrivera \`a gauche
{\it apr\`es} que le  jumeau de gauche ait travers\'e
le polariseur, c'est-\`a-dire trop tard. 
\medskip 
Reprenez un instant la d\'emonstration de l'in\'egalit\'e de Bell: 
vous constatez ais\'ement qu'elle ne marche plus si on suppose que la
fonction $X_G\big( a,\lambda \big)$ d\'epend aussi de $b$,  ou que la
fonction $X_D\big( b,\lambda \big)$
d\'epend aussi de $a$.
\medskip
Or dire que $X_G$ d\'epend de $b$ est \'evidemment l'expression
math\'ematique d'une influence de l'orientation \`a droite sur
ce qui se passe \`a gauche.  D'o\`u le surgissement,  comme l'\'ecrit
Fran\c{c}ois Lur\c{c}at (voir note 5 ci-apr\`es),  ``d'images
fantastiques:  les deux grains de poussi\`ere seraient coupl\'es
par une interaction physique de type nouveau,  inconnu. 
Certains physiciens sont all\'es jusqu'aux cons\'equences les plus
d\'eraisonnables:  par exemple de vouloir donner un fondement
physique \`a la parapsychologie,  avec la transmission de pens\'ee, 
etc...''
\medskip
Il y a pourtant d'autres fa\c{c}ons de saboter la d\'emonstration
des in\'egalit\'es de Bell:  la possibilit\'e $N^\circ 2$ ne le permettant
pas,  il reste les possibilit\'es $N^\circ 3$, $N^\circ 4$, et $N^\circ 5$
que nous soumettons maintenant \`a la critique. 
\medskip
La possibilit\'e $N^\circ 3$ n'est qu'une extension de la possibilit\'e
$N^\circ 1$.  En effet,  en supposant que les probabilit\'es conditonnelles
$P_{a,b}(++ |\, E_j)$ sont le produit $P_{a}(+ |\, j)
\cdot P_{b}(+ |\, j)$,  on exprime l'hypoth\`ese d'une
ind\'ependance stochastique entre les analyseurs,  on nie la
possibilit\'e d'une influence de l'un sur l'autre.  Sans cette
hypoth\`ese d'ind\'ependance,  la d\'emonstration des
in\'egalit\'es de Bell ne marche plus et la th\'eorie devient
donc compatible avec les faits quantiques;  le prix \`a payer
pour cela est l'acceptation d'une myst\'erieuse influence. 
C'est la m\^eme chose qu'avec la possibilit\'e $N^\circ 1$, 
sauf qu'on admet en plus une seconde intervention du hasard
pendant la travers\'ee des analyseurs.  
\medskip
Signalons au passage que cette expression: {\og pendant la
travers\'ee des analyseurs\fg} devrait \^etre proscrite,  car
ce n'est pas la M\'ecanique quantique,  mais les pr\'ejug\'es
classiques qui font imaginer une trajectoire dont une partie
comporterait la travers\'ee d'un analyseur.  Le ph\'enom\`ene
quantique qui se produit n'est pas cens\'e \^etre divisible
en morceaux.  Nous y reviendrons plus loin,  pour le moment
ne nous dispersons pas. 
\medskip
Pour quelqu'un qui reste accroch\'e aux id\'ees classiques et
refuse la philosophie quantique,  la possibilit\'e $N^\circ 3$ est
sans int\'er\^et;  en effet,  la possibilit\'e $N^\circ 1$ emp\^eche
tout aussi bien les in\'egalit\'es de Bell,  mais sans postuler une
seconde intervention du hasard.  Quant \`a l'inconv\'enient
---~admettre une influence~---,  il subsiste dans les deux cas. 
Autant choisir la possibilit\'e $N^\circ 1$. 
\medskip
J'ai gard\'e pour la fin l'examen des possibilit\'es $N^\circ 4$ et
$N^\circ 5$,  car elles ne permettent pas (contrairement \`a 1 et 3)
de sauver les apparences classiques. 
\medskip
Voyons ce que donne l'examen de la possibilit\'e $N^\circ 4$.  La formule des
probabilit\'es conditionnelles s'applique-t-elle vraiment?  Ce n'est pas
une \'evidence.  Dans tous les cours d'initiation \`a la M\'ecanique
quantique on discute l'exp\'erience des trous de Young\ftn{4}{La discussion
qui va suivre  suppose connue du lecteur au moins la le\c{c}on classique
d'introduction  \`a la M\'ecanique quantique, qui en principe fait partie
de la culture g\'en\'erale. Voir par exemple R. P. Feynman {\it 
M\'ecanique quantique}, chapitre {\bf I}.} et on montre que si on 
consid\`ere les deux \'ev\'enements (qui justement n'en sont pas) $E_j$: 
``la particule  est pass\'ee par le trou $N^\circ j\,$'' ($j=1$ ou $2$), et 
qu'on applique la formule des probabilit\'es conditionnelles, on obtient 
un r\'esultat faux.   
\medskip 
Dans cet ouvrage nous avons pr\'esent\'e un formalisme pour le Calcul  
des probabilit\'es, bas\'e sur l'espace des \'epreuves \'equiprobables 
et une notion d'\'ev\'enement repr\'esent\'e par un sous-ensemble. Cela 
s'applique fort bien \`a un grand nombre de probl\`emes vari\'es, m\^eme 
quantiques. Mais  dans l'exp\'erience des trous de Young, pour que les  
faux \'ev\'enements $E_j$: ``la particule est pass\'ee par le trou $N^\circ 
j\,$'' puissent en \^etre de vrais, il e\^ut fallu qu'ils soient deux  
sous-ensembles compl\'ementaires d'un espace des \'epreuves. Or on est 
absolument  incapable de dire ce que seraient dans ce cas les \'epreuves! 
On succombe simplement \`a l'habitude du langage courant. Nous avons 
observ\'e en {\bf IV.4} que la formule des probabilit\'es conditionnelles 
pouvait s'appliquer en ignorant tout de l'espace $\Omega$; nous l'avons 
alors appliqu\'ee \`a un probl\`eme de g\'en\'etique (les mariages  
consanguins) dans lequel nous \'etions \'egalement incapables de dire ce 
qu'\'etaient les \'epreuves. Mais le fait que {\it \c{c}a marche} justifiait  
a posteriori le proc\'ed\'e. Cela marche parce que la combinaison des 
chromosomes s'effectue dans l'espace et est donc semblable aux 
combinaisons de boules. Dans le cas des \'ev\'enements $E_j$: ``la 
particule est  pass\'ee par le trou $N^\circ j$'', cela ne marche pas. On 
aurait pu croire que l'hypoth\`ese d'un espace des \'epreuves n'\'etait 
n\'ecessaire que pour d\'emontrer la formule des probabilit\'es 
conditionnelles, et qu'une fois celle-ci obtenue on pourrait enlever 
l'hypoth\`ese comme les ma\c{c}ons enl\`event l'\'echafaudage une fois  
le mur construit. Toutefois l'exemple de la M\'ecanique quantique est l\`a 
pour montrer que la parabole de l'\'echafaudage n'est pas toujours juste: 
dans les m\'ecanismes de l'h\'er\'edit\'e, tout se passe comme s'il y avait 
un espace $\Omega$, bien qu'on soit incapable de le trouver; dans le cas 
des pseudo-\'ev\'enements $E_j$, cela {\it ne se passe pas comme si}.  
\medskip 
L'exp\'erience des trous de Young n'exclut pourtant pas toujours que les 
$E_j$  soient de v\'eritables \'ev\'enements auxquels on puisse appliquer 
la formule des probabilit\'es conditionnelles: il suffit pour cela de 
modifier le dispositif exp\'erimental de telle sorte que la particule  
puisse \^etre d\'etect\'ee dans les trous. La M\'ecanique quantique dit 
alors que les $E_j$, d\'efinis sous la forme  plus explicite ``la particule 
est d\'etect\'ee dans le trou $N^\circ j\,$'' sont de v\'eritables 
\'ev\'enements; dans ce cas on peut leur appliquer  la  formule des 
probabilit\'es conditionnelles.  
\medskip 
Quelle est la diff\'erence? Du point de vue de la Physique, les  
r\'esultats sont chang\'es, et cela n'a rien de paradoxal puisqu'on a 
modifi\'e le dispositif exp\'erimental. Mais nous voudrions savoir dans 
quels cas un \'ev\'enement d\'ecrit par une phrase entre guillemets du 
langage courant est r\'eellement repr\'esentable sous la forme d'un 
ensemble d'\'epreuves \'equiprobables et dans quels cas il ne l'est pas.  
Pour r\'epondre \`a cette question on ne  peut que r\'ep\'eter encore ce 
que Heisenberg et Bohr\ftn{5}{Les articles originaux de r\'ef\'erence 
sont Werner Heisenberg {\it \"Uber den anschaulichen Inhalt der 
quantentheoretischen Kinematik und Mechanik} (Zeitschrift f\" ur 
Physik, vol. {\bf 43}, {\oldstyle 1927}, pages 172 -- 198) et Niels Bohr 
{\it Das  Quantenpostulat und die neuere Entwicklung der Atomistik} 
(Naturwissenschaften, vol. {\bf 16}, {\oldstyle 1928}, pages 245 -- 
271).  Ensuite Bohr s'est plusieurs fois exprim\'e \`a nouveau;  avec le
recul du temps,  et jusqu'\`a sa mort il a clarifi\'e ses interpr\'etations, 
qu'on trouvera dans le recueil classique:  Niels Bohr,  {\it Physique atomique
et connaissance humaine},  Gauthier-Villars,  Paris,  {\oldstyle 1972}. 
Enfin,  les id\'ees qui sont d\'evelopp\'ees dans tous ces articles 
historiques sont tr\`es bien expliqu\'ees dans le livre de Fran\c{c}ois 
Lur\c{c}at {\it Niels  Bohr et la Physique quantique},  \'Ed. du Seuil, 
Paris,  {\oldstyle 2001},  coll {\it Points Sciences}. } se sont tu\'es \`a
dire (mais en le traduisant dans le langage du pr\'esent ouvrage): pour
qu'un \'ev\'enement comme ``la parti\-cule est pass\'ee par le trou $N^\circ 
j\,$'' soit un v\'eritable \'ev\'enement, il faut que le ``passage par le
trou'' soit un ph\'enom\`ene physiquement r\'eel. Or dans la nature (selon la
M\'ecanique quantique),  un ph\'enom\`ene est r\'eel s'il laisse une trace
objective, c'est-\`a-dire si quelque part un atome ou un \'electron a eu 
son \'etat modifi\'e par suite du ph\'enom\`ene. Si un atome situ\'e sur 
le bord du trou $N^\circ  j$ a eu son \'etat modifi\'e par suite du passage 
de la particule, alors l'\'ev\'enement $E_j$ est un  v\'eritable 
\'ev\'enement; mais dans un tel cas, si  on reproduit l'exp\'erience un 
grand  nombre de fois pour observer la distribution statistique des 
particules ayant ainsi travers\'e les trous en laissant une trace 
objective de leur passage, on ne verra aucune figure d'interf\'erence. 
La trace du passage doit exister {\it objectivement},  il n'est donc pas 
n\'ecessaire qu'un observateur l'ait remarqu\'ee: si on n'a rien 
d\'etect\'e \`a proximit\'e des trous, mais que la figure 
d'interf\'erence est absente, on peut {\it en d\'eduire} qu'une trace 
inconnue du passage a d\^u \^etre laiss\'ee quelque part.  Inversement,  
si la figure d'interf\'erence est pr\'esente, on peut en d\'eduire 
qu'aucune  trace objective du passage \`a travers l'un ou l'autre des 
trous n'a \'et\'e laiss\'ee, c'est-\`a-dire qu'il n'existe aucune {\it 
r\'ealit\'e} correspondant  \`a l'id\'ee ``la par\-ti\-cule est pass\'ee par 
l'un ou l'autre des trous''. Et si cette id\'ee ne correspond \`a aucune 
r\'ealit\'e, il ne peut pas non plus lui  correspondre une \'epreuve. Nous 
avons commenc\'e cet ouvrage en parlant (m\'etaphoriquement) du {\it 
niveau o\`u intervient le hasard} pour  d\'esigner la nature des 
\'epreuves parmi lesquelles la Fortune choisit. Nous pouvons ici ajouter 
une pr\'ecision:  pour qu'une \'epreuve puisse \^etre choisie par la 
Fortune  dans le monde  r\'eel, il faut que cette \'epreuve soit un 
ph\'enom\`ene r\'ealisable. Cette pr\'ecision allait sans dire pour des 
probl\`emes de boules, mais ne va plus du tout sans dire pour des 
ph\'enom\`enes quantiques. Toute la difficult\'e provient de ce qu'une 
perception  correcte de la r\'ealit\'e est re\-qui\-se. Quelqu'un qui 
voudrait d\'eterminer  un espace  des \'epreuves relatif \`a 
l'exp\'erience des  trous de Young  devrait chercher l'ensemble des 
mani\`eres de laisser  une trace et ne  pourrait donc pas ignorer le r\^ole 
jou\'e par les atomes  de l'\'ecran. Sans doute devrait-il, pour accomplir 
compl\`etement ce travail, aller au del\`a de la Physique connue 
actuellement. L'intuition h\'erit\'ee de la Physique classique nous 
pousse tout au contraire \`a  nous repr\'esenter les \'epreuves sous la 
forme des trajectoires  possibles pour une parti\-cule ponctuelle dans 
l'espace (trajectoires  qui dans le cas des  trous de Young  passeraient 
par l'un des trous). De l\`a vient l'illusion que la phrase ``la particule 
passe par le trou $N^\circ j\,$'' repr\'esente  un \'ev\'enement: cette 
phrase repr\'esente bien un ensemble  de trajectoires, mais pas un 
ensemble de traces pouvant \^etre laiss\'ees {\it r\'eellement}  par la 
particule dans son environnement mat\'eriel.  Elle est en r\'ealit\'e 
autant d\'epourvue de sens concret que la phrase  ``le d\'e marque 
$\pi\sqrt{2}$ points''.     
\medskip  
Revenons \`a l'exp\'erience $E.P.R.$ et \`a la discussion bas\'ee sur
l'in\'egalit\'e de Bell.  Cette in\'egalit\'e a \'et\'e d\'emontr\'ee en
utilisant la formule des probabilit\'es conditionnelles.  On a donc admis
implicitement que, m\^eme si l'espace $\Omega$ n'est pas connu (de 
sorte que la loi de la variable al\'eatoire $\lambda$ ne  pourrait \^etre
calcul\'ee a priori, mais seulement mesur\'ee par des statistiques), 
du moins tout se passe comme s'il en existait un. Nous avons discut\'e 
ce principe \`a la section {\bf 4.4} (``Relativit\'e du hasard''), 
et remarqu\'e qu'il s'appliquait parfaitement au probl\`eme des mariages 
consanguins.  Mais le fait qu'il s'applique n'est pas prouv\'e par nos
calculs:  il est prouv\'e par l'observation de la r\'ealit\'e.
\medskip
Lorsque nous disons que le signe de la d\'eviation par les 
aimants est une fonction de $a$ (l'orientation de l'aimant) et de 
$\lambda$ (la variable al\'eatoire qui repr\'esenterait l'\'etat de la 
particule), nous \'emettons implicitement une hypoth\`ese tr\`es forte, 
\`a savoir que cette r\`egle s'applique; or l'exp\'erience des trous de 
Young est l\`a pour montrer qu'elle ne s'applique pas toujours. La 
M\'ecanique quantique nous dit que l'\'ev\'enement $E_j$: $\{  \lambda = 
\lambda_j \}$ (``la variable al\'eatoire $\lambda$ prend la valeur 
particuli\`ere $\lambda_j$'') n'est un \'ev\'enement r\'eel que si le fait 
que $\lambda$ prenne la valeur particuli\`ere $\lambda_j$ se traduit 
quelque  part par une trace objective. On ne peut appliquer la r\`egle des  
probabilit\'es conditionnelles qu'\`a des \'ev\'enements r\'ealisables.  
De quel droit en effet pourrions nous  affirmer que si une phrase 
exprim\'ee en langage imag\'e  semble avoir un sens (pour des points 
imaginaires que nous nous repr\'esentons dans l'espace en fermant les 
yeux), il y aura n\'ecessairement des \'epreuves r\'eelles qui lui 
correspondent?  \`A  quelles \'epreuves r\'eelles peut correspondre la 
phrase ``le d\'e marque  un nombre de points compris entre $\sqrt{2}$  et 
$\sqrt{3}$''? Que  donnerait un calcul utilisant la r\`egle des 
probabilit\'es conditionnelles, si l'on consid\`ere la famille exhaustive 
d'\'ev\'enements $E_j$: ``le d\'e marque un nombre de points compris 
entre $\sqrt{\sdown{8.6} j}$ et $\sqrt{\sdown{8.6} j+1}\,$'' ($j=0,1,2 
\ldots  35$) et en posant ${\cal P}\, (E_j) = {1\over 
6}\,\big(\sqrt{\sdown{8.6} j+1} - \sqrt{\sdown{8.6} j}\,\big)$ ?  
\medskip 
Certains auteurs avides de sensation ont proclam\'e (\`a la suite des 
discussions sur ces ph\'enom\`enes \'etranges) que ``la r\'ealit\'e  
n'existe donc pas''. Mais conclure que la r\'ealit\'e n'existe pas repose  
sur le raisonnement par l'absurde ``si la r\'ealit\'e existait, elle serait  
d\'ecrite par la variable $\lambda$, or $\ldots$''. On pourrait par le 
m\^eme raisonnement  conclure \`a l'inexistence de la r\'ealit\'e   
chaque fois que celle-ci ne se plierait pas enti\`erement \`a une 
repr\'esentation math\'ematique pr\'ed\'efinie (et ce raisonnement  
idiot a \'et\'e tenu plus d'une fois, d\'ej\`a bien avant la M\'ecanique 
quantique). Une telle affirmation n'a strictement aucun sens en  
l'absence  d'une d\'efinition pr\'ealable, pr\'ecise et op\'eratoire, du mot  
r\'ealit\'e\ftn{6}{Notons qu'Einstein, Podolski, et Rosen ont, justement, 
propos\'e dans leur article une telle d\'efinition pr\'ecise, op\'eratoire, 
et r\'efutable; et on peut d\'eduire de l'exp\'erience que cette 
d\'efinition a \'et\'e r\'efut\'ee.}. Au contraire la M\'ecanique quantique 
rend plus forte et plus rigoureuse la notion de r\'ealit\'e: elle montre 
qu'un \'ev\'enement r\'eel  est ce qui laisse quelque part une trace 
mat\'erielle, m\^eme si celle-ci \'echappe \`a l'observateur humain. Elle  
fournit m\^eme un moyen concret de  savoir si un \'ev\'enement se  
produit r\'eellement (par exemple si des atomes situ\'es  a proximit\'e  
de l'un des trous de Young subissent un changement d'\'etat au passage  
de  la particule): il suffit de constater statistiquement, dans un montage 
ad\'equat,  qu'une figure d'interf\'erence s'estompe. De  toute  fa\c{c}on 
nous sommes ici dans un ouvrage s\'erieux et n'avons  pas \`a discuter 
les sophismes d'auteurs \`a sensation.  La question qui nous int\'eresse 
est de savoir quelles sont les conditions de validit\'e  du paradigme de 
l'espace des \'epreuves. On voit que les ph\'enom\`enes quantiques sont 
une magnifique illustration du mauvais usage qu'on peut en faire; non que 
la r\`egle ${\cal P}\, (A) =  \#A\; / \;\#\Omega$ puisse dans certains 
cas \^etre fausse, mais les \'epreuves doivent correspondre \`a des 
possibilit\'es r\'eelles et non \`a des extrapolations purement  
imaginaires de nos perceptions.    
\medskip  
Prenons encore un exemple. Tout au d\'ebut de l'ouvrage, nous 
avons exa\-min\'e (cf. {\bf I.2}, premier exemple) les r\'epartitions 
possibles de trois boules dans deux bo\^\i tes. Selon la statistique de 
Bose, il n'y avait que quatre \'epreuves possibles (les modes 
d'occupation) et si on comparait aux huit distributions classiques, on 
pouvait concevoir les \'ev\'enements 
\smallskip 
\line{$E$: ``particule $N^\circ 1$ dans bo\^\i te  
${\cal A}$, particules $N^\circ 2$ et $N^\circ 3$ dans bo\^\i te  
${\cal B}$''} 
\smallskip 
\line{$F$: ``particule $N^\circ 2$ dans bo\^\i te  
${\cal A}$, particules $N^\circ 1$ et $N^\circ 3$ dans bo\^\i te  
${\cal B}$''} 
\smallskip 
\line{$G$: ``particule $N^\circ 3$ dans bo\^\i te  
${\cal A}$, particules $N^\circ 1$ et $N^\circ 2$ dans bo\^\i te  
${\cal B}$''} 
\smallskip\noindent 
dont la r\'eunion constituerait le mode d'occupation ``une particule dans  
la bo\^\i te ${\cal A}$, deux particules dans la bo\^\i te  ${\cal B}$'' 
Mais ces \'ev\'enements ne sont pas r\'eels; nous les avons r\^ev\'es.  
Pour  qu'ils soient r\'eels, il faudrait qu'il existe objectivement un  
moyen de  distinguer les trois particules. Un tel moyen existe pour les 
boules, m\^eme si elles sont absolument identiques: ce sont leurs 
trajectoires dans l'espace. Nous avons insist\'e sur la possibilit\'e de  
calculer des probabilit\'es sans conna\^\i tre l'espace des \'epreuves, en 
recourant  \`a la r\`egle des  probabilit\'es conditionnelles, et illustr\'e 
cela sur un exemple issu de la g\'en\'etique. Mais si nous voulions 
appliquer cette r\`egle avec les \'ev\'enements $E,F,G$ ci-dessus, en 
\'ecrivant  
$${\cal P}\, (X) = {\cal P}\, (X\mid E\, ) \cdot {\cal 
P}\, (E) +  {\cal P}\, (X\mid F\, ) \cdot {\cal P}\, (F) + {\cal P}\, (X\mid 
G\, )   \cdot {\cal P}\, (G)\; ,$$  
et en attribuant \`a chacun des \'ev\'enements $E,F,G$ une probabilit\'e 
\'egale par  exemple au tiers de celle du mode d'occupation, nous 
aboutirions facilement \`a des r\'esultats faux (c'est-\`a-dire  
contredits par l'exp\'erience). Pourtant il a \'et\'e possible de  
d\'eterminer un espace des \'epreuves r\'eel, celui des modes 
d'occupation. Nous avons vu en {\bf II.5} que les modes d'occupation 
peuvent \^etre  d\'ecrits par des repr\'esentations spatiales (par 
exemple les graphiques form\'es de $\bigcirc$ et de $\mid$), alors que 
l'{\it individualit\'e} des  particules de Bose ne  correspond \`a aucune 
repr\'esentation spatiale. Un ensemble fini comme $\Omega$ peut 
toujours \^etre repr\'esent\'e par des ensembles de points sur du papier  
ou \`a d\'efaut, si son cardinal est trop grand, \^etre imagin\'e avec 
notre intuition de l'espace.  Cela ne semble pas \^etre toujours le cas 
pour la r\'ealit\'e quantique.  
\vskip6pt plus5pt minus4pt 
L'exp\'erience d'Aspect montre que la formulation math\'ematique qui 
conduit \`a l'in\'egalit\'e de Bell ne correspond pas \`a la r\'ealit\'e.  
C'est tout ce qu'elle montre r\'eellement. Ce n'est pas tant le Calcul des  
probabilit\'es qui trouve l\`a ses limites, que la mani\`ere toujours  
spatiale de concevoir les \'epreuves. Certes, on pourrait conserver 
une description math\'ematique qui tiendrait pour r\'eels des  
\'ev\'enements tels que $E_j$: $\{ \lambda = \lambda_j \}$, en recourant
\`a la possibilit\'e $N^\circ 1$ pour compenser la diff\'erence entre  
la r\'ealit\'e quantique  et les illusions de nos repr\'esentations: 
la variable al\'eatoire $X_G(a,\lambda )$ d\'ependrait aussi de $b$ et 
deviendrait $X_G(a,b,\lambda )$. Mais  il semble clair que ce sch\'ema  
ne serait  qu'un artifice, car l'erreur est plus profonde. Tout 
comme les \'ev\'enements ``la particule est pass\'ee par le trou 
$N^\circ j$'', les \'ev\'enements ``$\lambda = \lambda_j$'' ne 
correspondent \`a aucun ph\'enom\`ene r\'eel; ce qui est r\'eel est le
r\'esultat observ\'e {\it apr\`es} la travers\'ee des deux analyseurs, 
et il est logique que cela d\'epende des directions $a$ et $b$ (en fait 
de $\theta$). Un \'ev\'enement ``$\lambda = \lambda_j$'' qui se serait 
produit avant que $a$ et $b$ n'aient \'et\'e fix\'ees est un fantasme et 
non un ph\'enom\`ene. Tout d'abord, l'id\'ee d'une influence non causale
(remontant le temps ou voyageant  plus vite que la lumi\`ere) est 
difficilement acceptable. A priori et dans l'absolu, on ne  peut toutefois 
pas plus \'ecarter cette \'eventualit\'e  que le fluide de Kepler (cf. la 
fin du chapitre {\bf XII}), la force gravitationnelle de Newton, ou l'\'ether
de Maxwell. Mais s'agit-il vraiment d'une influence?  
\vskip6pt plus5pt minus4pt 
Newton a lui-m\^eme insist\'e sur le fait que la force n'est qu'une 
repr\'e\-sen\-ta\-tion intellectuelle et que seule la loi quantitative en  
$1/r^2$ \'etait objective. Maxwell a \'ecrit cela de la m\^eme fa\c{c}on  
\`a propos  de l'\'ether. On pourrait dire ici aussi que cela ne co\^ute  
rien d'imaginer intellectuellement une influence qui se propage d'un 
aimant  \`a l'autre pour favoriser l'un des sens de d\'eviation au 
d\'etriment de l'autre: seule serait objective, de toute fa\c{c}on, la loi 
quantitative des corr\'elations en $-\cos\theta$. Cela nous ram\`ene 
encore \`a la discussion sur la nature de la causalit\'e \`a la fin du 
chapitre {\bf XII}. La phrase c\'el\`ebre de Galil\'ee ``Le monde est un 
livre \'ecrit par le Cr\'eateur, et la langue dans laquelle il est \'ecrit 
est la G\'eom\'etrie'' peut \'egalement s'interpr\'eter dans le  m\^eme 
sens:  pour dire ce qu'est un fluide ou une force il faut recourir \`a 
l'exp\'erience subjective,  par exemple la sensation qu'on \'eprouve en 
tirant sur une corde pour hisser un objet lourd;  le langage humain 
exprime ces sensations.  Mais la langue  du Grand Livre ne les exprime 
pas,  d'autant moins que ce livre \'etait d\'ej\`a \'ecrit avant qu'il
y ait des \^etres vivants pour \'eprouver lesdites sensations. Donc
$m\vec\gamma = -K\vec r / r^3$ ou  $R = -\cos\theta$ sont pour 
Galil\'ee des phrases du langage de l'univers, mais les repr\'esentations 
imag\'ees d'un ange qui pousserait les plan\`etes pour  les rapprocher  
du Soleil ou d'un signal \'emis par l'un  des aimants pour aller informer 
l'autre sur son orientation n'en sont pas. 
\medskip
J'ai gard\'e pour la fin la discussion de la possibilit\'e
$N^\circ 5$.  D'un point de vue purement math\'ematique, 
on voit bien que la d\'emonstration de l'in\'egalit\'e de
Bell ne fonctionne plus si les probabilit\'es $p_j$ d\'ependent
de $a$ et $b$.  Notons tout de suite que les $p_j$ ne peuvent
pas \^etre des fonctions de $a$ seul,  ni de $b$ seul: 
cela contredirait la sym\'etrie du probl\`eme.  Quand il
s'agissait de $X_G$ ou $X_D$,  postuler que $X_G$ est une
fonction de $a$ seul \'etait coh\'erent,  puisque la valeur de
$X_G$ \'etait cens\'ee ne concerner que le jumeau de gauche, 
qui ne traverse pas l'analyseur de droite.  Quant aux $p_j$, 
soit ils concernent seulement la cr\'eation de la paire de
jumeaux {\it avant} toute approche des analyseurs (et alors il
est raisonnable de les supposer ind\'ependants de $a$ et $b$), 
soit ils concernent l'ensemble du processus (et alors ils doivent
d\'ependre,  et sym\'etriquement,  de $a$ et $b$).  En fait, 
comme le r\'esultat de l'exp\'erience ne saurait d\'ependre non
plus de l'orientation globale du dispositif,  les $p_j$ doivent
\^etre des fonctions de l'angle $\theta$ entre $a$ et $b$. 
\medskip
Mais que signifie concr\`etement cette hypoth\`ese purement
math\'ematique que les $p_j$ sont des fonctions de l'angle
$\theta$?  Que traduit-elle? 
\medskip
En pr\'ecisant les hypoth\`eses du th\'eor\`eme de Bell,  nous
avons insist\'e sur le fait que la variable $\lambda$ re\c{c}oit
ses valeurs du hasard {\it au moment de la cr\'eation de la
paire},  et que ce choix du hasard d\'etermine enti\`erement la
suite des \'ev\'enements.  C'est ce qui avait \'et\'e retenu par
la possibilit\'e $N^\circ 1$.  En admettant maintenant que la loi
$p_j$ de cette variable $\lambda$ d\'epend de l'angle $\theta$, 
on jette cette hypoth\`ese aux orties.  Consid\'erer les $p_j$
comme des fonctions de $\theta$,  c'est admettre que la variable
$\lambda$ est d\'etermin\'ee d\`es le d\'epart par l'ensemble du
dispositif.  C'est donc admettre que le ph\'enom\`ene observ\'e
ne se d\'ecompose pas en plusieurs parties,  dont la premi\`ere, 
disons la partie $P1$,  serait la cr\'eation de la paire dans un
environnement spatio-temporel isotrope (le voisinage imm\'ediat
du point $O$),  la partie $P2$ la propagation du jumeau de gauche
et la partie $P3$ celle du jumeau de droite,  puis enfin les
parties $P4$ et $P5$ qui seraient respectivement la travers\'ee
de l'analyseur de gauche par le jumeau de gauche et la travers\'ee
de l'analyseur de droite par le jumeau de droite. 
On pourrait ajouter les parties $P6$ et $P7$ correspondant aux
processus qui sont \`a l'oeuvre dans les analyseurs et qui les
conduisent \`a d\'evier les particules qui les traversent. 
C'est cette d\'ecomposition du ph\'enom\`ene en parties qui
exprime le mieux les pr\'esuppos\'es de la M\'ecanique classique: 
on imagine les jumeaux comme des {\og grains de poussi\`ere\fg}
qui \`a chaque instant peuvent \^etre localis\'es quelque part
dans l'espace,  et alors on peut parler du {\og moment o\`u ils
sont encore au voisinage du point $O$\fg},  du {\og moment o\`u
ils sont en route vers l'analyseur\fg},  du {\og moment o\`u ils
traversent l'analyseur\fg} \dots \looseness=-1
\medskip
Or tout cela a \'et\'e invent\'e de toutes pi\`eces par l'homme, 
et rajout\'e \`a ce que dit la nature.  La le\c{c}on tir\'ee par
Niels Bohr de l'aventure quantique,  c'est que les ph\'enom\`enes
comme celui que nous discutons ne sont pas divisibles. 
\medskip
Personne n'a jamais vu la cr\'eation de la paire,  ni vu les
particules en route vers un analyseur ou un d\'etecteur;  ce qu'on
a vu,  c'est un signal \'electrique envoy\'e par un d\'etecteur
vers un galvanom\`etre,  aujourd'hui vers une carte de conversion
analogique-digital;  ou bien la d\'ecomposition d'une mol\'ecule
de bromure d'argent en argent m\'etallique dans une \'emulsion
photographique,  aujourd'hui la saturation d'un pixel C.C.D. 
\medskip
Il existe sans doute une v\'erit\'e cach\'ee qui explique {\og ce
qu'on voit\fg}.  Mais nous n'avons pas encore les moyens de
l'appr\'ehender,  ni observationnellement,  ni conceptuellement. 
Cela exige une certaine modestie:  il ne s'agit pas de
{\og positivisme\fg},  il s'agit de critiquer des pr\'ejug\'es.
\medskip
Revenons encore \`a l'examen de la possibilit\'e $N^\circ 5$. 
En postulant que la loi $p_j$ d\'epend de l'angle $\theta$, 
on exclut la divisibilit\'e du ph\'enom\`ene.  C'est justement l\`a
un des enseignements de Niels Bohr.  Dans un texte de {\oldstyle
1954} ({\it Unit\'e de la connaissance})\ftn{7}{Voir {\it Physique
atomique et connaissance humaine} (d\'ej\`a cit\'e \`a la note 5), 
page 110.},  Bohr \'ecrivit ceci: 
\smallskip
{\cit (\dots) il est en effet plus correct,  dans une description
objective,  de ne se servir du mot de ph\'enom\`ene que pour
rapporter des observations obtenues dans des conditions
parfaitement d\'efinies,  dont la description implique celle de
tout le dispositif exp\'erimental.  Avec cette terminologie le
probl\`eme de l'observation en physique quantique perd toute
complexit\'e particuli\`ere.  De plus,  elle nous rappelle
directement que tout ph\'enom\`ene atomique est {\og
d\'efinitivement clos\fg},  en ce sens que son observation est
fond\'ee sur des enregistrements obtenus au moyen de dispositifs
d'amplification au fonctionnement irr\'eversible,  tels que
les traces permanentes laiss\'ees sur une plaque photographique
par des \'electrons p\'en\'etrant dans l'\'emulsion. \par}
\medskip
John Archibald Wheeler,  qui eut des discussions approfondies
avec Bohr dans les ann\'ees 1950,  en garda une sentence qu'il
r\'ep\'eta d'innombrables fois:
\medskip
{\narrower \sl ``No elementary quantum phenomenon is yet a
phenomenon until it is a registred phenomenon,  brought to a
close by an irreversible act of amplification.'' \par}
\medskip
{\narrower \sl ``Un ph\'enom\`ene quantique \'el\'ementaire n'est
pas encore un ph\'e\-no\-m\`ene tant qu'il n'est pas un ph\'enom\`ene
enregistr\'e,  amen\'e \`a bonne fin par un processus
d'amplification irr\'eversible.\par}
\medskip
Comment le langage abstrait du Calcul des probabilit\'es, 
d\'evelopp\'e dans le pr\'esent ouvrage,  peut-il traduire une
assertion aussi mat\'erialiste?  \`A mon avis,  en posant que
la loi $p_j$ d\'epend d'embl\'ee de l'ensemble du dispositif
exp\'erimental,  ou plus exactement des param\`etres qui en
d\'ecrivent le mod\`ele id\'ealis\'e;  ici c'est le param\`etre
angulaire $\theta$. 
\medskip
Bien entendu,  c'est la d\'ependance de $p_j$ par rapport \`a
$a$ et $b$ qui sabote la d\'emonstration du th\'eor\`eme de Bell, 
car dans la combinaison ${\bf E}(a_1,b_1) - {\bf E}(a_1,b_2) +
{\bf E}(a_2,b_2) + {\bf E}(a_2,b_1)$ les coefficients $p_j$ sont
diff\'erents dans chacun des quatre termes.  Que devait signifier
l'hypoth\`ese que $p_j$ est ind\'ependant de $a$ et $b$?  Tout
simplement que la premi\`ere partie du ph\'enom\`ene,  que nous
avions d\'esign\'ee plus haut par $P1$,  est reproductible entre
les quatre mesures $(a_1,b_1)$,  $(a_1,b_2)$,  $(a_2,b_1)$,  $(a_2, 
b_2)$.  L'in\'egalit\'e de Bell exprime ainsi l'hypoth\`ese que 
\smallskip
a) le ph\'enom\`ene est divisible en parties;
\smallskip
b) l'une de ces parties est reproduite \`a l'identique
bien qu'on change les orientations $a$ et $b$. 
\medskip
Or la M\'ecanique quantique dit que le ph\'enom\`ene n'est pas
divisible,  qu'en refaisant les mesures avec d'autres
orientations,  on ne reproduit pas le ph\'enom\`ene \`a
l'identique,  et qu'il n'existe pas une partie du ph\'enom\`ene
qui,  elle,  serait reproduite \`a l'identique. 
\medskip
Voil\`a l'erreur. 
\medskip
Un dernier point pour que la discussion soit compl\`ete: 
supposer que les $p_j$ d\'ependent de $\theta$ suffit pour
saboter la d\'emonstration des in\'egalit\'es de Bell; 
il n'y a pas besoin en outre de supposer que $X_G(a, \lambda)$
est aussi fonction de $b$ (et $X_D(b, \lambda)$ fonction de $a$). 
On conserve alors l'expression math\'ematique du fait qu'il n'y
a pas d'influence exerc\'ee entre les analyseurs.  Ce \`a quoi
on renonce,  c'est la division du ph\'enom\`ene.  On n'a donc
pas besoin de l'image fantastique de {\og deux grains de
poussi\`ere qui seraient coupl\'es par une interaction physique
de type nouveau,  inconnu\fg}.  Postuler des fonctions
$p_j(\theta)$ n'exprime pas une influence d'un analyseur sur
l'autre,  mais que la variable $\lambda$,  pour peu qu'elle
ait un sens,  n'est pas fix\'ee (par le hasard ou par ce qu'on
voudra) au d\'ebut du ph\'enom\`ene ---~il n'y a pas de d\'ebut
du ph\'enom\`ene~--- mais pendant tout son d\'eroulement. 

\vskip6mm plus3mm minus3mm 

{\bf XIII. 4. Un mod\`ele g\'eom\'etrique.}
\medskip
Peut-\^etre la discussion a-t-elle \'et\'e jusqu'ici trop
math\'e\-ma\-ti\-que pour montrer \`a quel point la possibilit\'e
$N^\circ 1$ est conditionn\'ee par de simples habitudes de pens\'ee. 
La d\'emonstration  de l'in\'egalit\'e  de Bell est en effet
math\'e\-ma\-tique et abstraite;  cependant elle refl\`ete fid\`element
le mode de pens\'ee enseign\'e dans cet ouvrage;  tous les st\'er\'eotypes
que je me suis efforc\'e d'inculquer au lecteur y sont mis en oeuvre. 
Les faits exp\'erimentaux  sont l\`a pour montrer que leur validit\'e
n'est pas universelle, mais l'exc\`es d'abstraction math\'ematique
peut encore servir de tampon ou d'\'ecran et prot\'eger un certain
nombre d'id\'ees fausses. Pour mieux percevoir  le sens de l'in\'egalit\'e
de Bell, nous  allons construire un mod\`ele g\'eom\'etrique. 
\medskip 
Reprenons une \`a une les caract\'eristiques vraiment essentielles  
de l'exp\'e\-rience E.P.R.  Chacun des deux aimants ne se caract\'erise que  
par son orientation:  seule intervient la direction du champ magn\'etique 
et de  son gradient, tout le reste n'est que du d\'ecor. Par cons\'equent  
la repr\'esentation sch\'ematique la plus simple de ces aimants est une 
simple fl\`eche indiquant cette direction. Pour rendre la repr\'esentation 
plus imag\'ee on  imaginera un disque divis\'e en deux par un diam\`etre 
dont on peut distinguer les deux extr\'emit\'es: par exemple on marquera 
$N$ (nord)  l'une des extr\'emit\'es et $S$ (sud) l'autre:  voir figure 60. 
\medskip 
\midinsert 
\epsfxsize=\hsize
\line{\epsfbox{../images/fig60.eps} } 
\vskip3mm
\centerline{\eightpoint figure 60} 
\endinsert 
Supposons que le spin de chaque particule soit d\'etermin\'e  
par un \'etat interne qui serait une direction dans l'espace;  
le param\`etre d\'ecrivant un tel \'etat serait donc un angle $\varphi$, 
compris entre $0$ et $2\pi$. Cet angle serait ``choisi au hasard'' au 
moment de la cr\'eation des jumeaux. Dans ce mod\`ele la variable 
$\lambda$ de  Bell est donc cet angle, et sa loi de probabilit\'e est 
uniforme sur $[0, 2\pi]$, sinon le processus de cr\'eation des jumeaux 
violerait l'invariance de l'espace par rotation; n'oublions pas, en effet, 
que dans le sch\'ema de Bell, le hasard intervient au moment de la 
cr\'eation de la paire: l'orientation des analyseurs n'a aucune influence 
sur ce processus.  La corr\'elation entre les deux jumeaux s'exprimera 
alors comme une sym\'etrie entre les deux angles:  si  $\varphi_G$ est 
l'angle du jumeau de gauche et $\varphi_D$ est l'angle du jumeau de droite, 
la corr\'elation se traduira par une \'egalit\'e telle que $\varphi_G +
\varphi_D = 2\pi$.  Ce n'est que l'expression math\'ematique de la 
sym\'etrie entre les deux jumeaux.
\medskip 
L'ensemble des valeurs prises par $\varphi$ est donc repr\'esentable  
par une circonf\'erence, sur laquelle $\varphi_G$ et $\varphi_D$ sont 
diam\'etralement oppos\'es.  Les aimants \'etant repr\'esent\'es par la 
fl\`eche $NS$, on se servira du cercle dont $NS$ est le diam\`etre pour  
repr\'esenter l'\'etat du jumeau correspondant. 
\medskip  
Pour repr\'esenter le signe de la d\'eviation des particules (autrement dit 
le spin) dans  l'exp\'erience, nous divisons le cercle en deux  moiti\'es,
l'une \`a droite du diam\`etre $NS$, (r\'egion blanche sur la figure 60),
l'autre \`a gauche du diam\`etre $NS$, (r\'egion gris\'ee).  Si la particule
s'arr\^ete  dans la r\'egion blanche on attribue le signe $+$,  si elle
s'arr\^ete dans  la  r\'egion gris\'ee, le signe $-$. Si les deux 
disques (celui qui repr\'esente l'aimant de gauche et celui qui repr\'esente  
l'aimant de droite) ont la m\^eme orientation, il est clair que les signes 
seront toujours oppos\'es, puisque les points d'arriv\'ee des  particules 
sont diam\'etralement  oppos\'es  (si  l'une s'arr\^ete dans la  r\'egion 
blanche, l'autre s'arr\^etera  dans la r\'egion gris\'ee, et vice versa).  
Donc la corr\'elation des signes sera $-1$. Jusque l\`a, tout se passe 
comme dans l'exp\'erience  E.P.R.  
\medskip 
\midinsert 
\epsfxsize=\hsize
\line{\epsfbox{../images/fig61.eps} } 
\vskip3mm
\centerline{\eightpoint figure 61} 
\endinsert 
Que se passe-t-il si les deux disques (celui ``de gauche'' et celui ``de 
droite'') ont des orientations diff\'erentes, formant un angle $\theta$ 
entre elles? Sur la figure 61 on peut voir les deux disques \`a gauche et 
\`a droite orient\'es diff\'eremment et au milieu la superposition des 
deux. La zone noire dans le disque du milieu est l'intersection des zones 
gris\'ees des deux disques de gauche et de droite, la zone blanche 
l'intersection des deux zones blanches. La zone claire dans le disque  
du milieu est l'intersection d'une zone gris\'ee  avec une zone blanche. 
Comme les deux particules sont  toujours diam\'etralement oppos\'ees,  
si l'une aboutit dans la zone  blanche, l'autre aboutira dans la zone noire 
et inversement puisque ces zones sont sym\'etriques l'une de l'autre;  
dans ce cas leurs signes seront oppos\'es. Mais si l'une des particules 
aboutit dans  la partie claire, l'autre y aboutira aussi puisque les deux 
zones claires sont \'egalement sym\'etriques l'une de l'autre; les deux 
particules se verront alors attribuer le m\^eme signe. On peut donc dire 
que la partie du disque form\'ee de la r\'eunion de la zone blanche et de  
la zone noire est celle  qui donnera des signes oppos\'es ($+\, -$ ou$-\, 
+$), et la partie claire  est celle qui correspond \`a des signes \'egaux 
($+\, +$ ou $-\, -$).    
\medskip 
Bien entendu les particules de ce mod\`ele g\'eom\'etrique n'ont,
intentionnellement, rien de quantique.  
\medskip 
Posons-nous maintenant la question de la {\it probabilit\'e} d'avoir 
le m\^eme signe ou des signes oppos\'es. Puisque tous les angles  
$\theta$ sont \'equiprobables, et que la  zone claire repr\'esente une 
proportion $\theta / \pi$ de la circonf\'erence compl\`ete, la 
probabilit\'e d'avoir des signes \'egaux pour les deux particules jumelles 
est $\theta / \pi$. Si $\theta$ avait \'et\'e n\'egatif (c'est-\`a-dire si 
sur la figure 61 on avait tourn\'e le disque de droite dans le sens 
oppos\'e), cette proportion serait $|\theta | / \pi$. Cela   n'est 
\'evidemment valide que pour $-\pi < \theta < +\pi$, car si $\theta$ 
devenait plus grand que $\pi$ le rapport deviendrait sup\'erieur \`a  
$1$  et ne serait plus une probabilit\'e; dans ce cas il faudrait ajouter  
ou retrancher \`a $\theta$ un multiple de $2\pi$ pour le ramener dans  
cet intervalle. Pour des rotations d'angles plus grands il faut donc 
compter modulo $2\pi$,  autrement dit la probabilit\'e est la fonction  
p\'eriodique $\Phi (\theta )$, de p\'eriode $2\pi$, qui est \'egale \`a 
$|\theta | / \pi$ sur l'intervalle $[-\pi , +\pi ]$. Elle est repr\'esent\'ee 
sur la figure $62\, a$, superpos\'ee \`a la fonction pr\'edite par la 
M\'ecanique quantique. 
\medskip 
Ce prolongement de la fonction $\Phi (\theta )$
pour des angles $\theta$ hors de l'intervalle  $[-\pi , +\pi ]$ n'est qu'une
commodit\'e purement math\'ematique, et ne joue aucun r\^ole concret: 
on veut simplement dire par l\`a que si par exemple on faisait faire dix 
tours complets \`a l'un des analyseurs, sa position effective dans
l'espace ne d\'ependrait que modulo $2\pi$ de l'angle dont il aurait 
tourn\'e,  et par cons\'equent il doit en \^etre de m\^eme de la
probabilit\'e $\Phi (\theta )$ correspondante.
\medskip 
\midinsert 
\epsfxsize=\hsize
\line{\epsfbox{../images/fig62a.eps} } 
\vskip3mm
\centerline{\vbox{\hsize=11cm 
\eightpoint a) probabilit\'e $\Phi (\theta )$ pour que les 
particules jumelles soient de m\^eme signe (fonction 
affine par morceaux), compar\'ee  \`a la loi correspondante dans 
l'exp\'erience E.P.R. quantique (fonction sinuso\"\i dale).}  } 
\vskip5mm
\epsfxsize=\hsize
\line{\epsfbox{../images/fig62b.eps} } 
\vskip3mm
\centerline{\vbox{\hsize=11cm 
\eightpoint b) corr\'elation des signes entre les particules 
jumelles (fonction affine par morceaux), compar\'ee  \`a la corr\'elation  
E.P.R. (fonction sinuso\"\i dale).}  } 
\vskip3mm 
\centerline{\eightpoint figure 62} 
\vskip3mm 
\endinsert 
La probabilit\'e que les deux particules obtiennent des signes oppos\'es 
est alors $1 - \Phi (\theta )$. Quant \`a la corr\'elation entre les 
signes, elle  est \'egale \`a $\rho = -\big[ 1 - \Phi (\theta ) \big] + 
\Phi (\theta )  = 2\Phi (\theta ) - 1$. Le graphique de cette 
corr\'elation est repr\'esent\'e sur la figure $62\, b$, o\`u il est 
compar\'e \`a la  corr\'elation  $-\cos\theta$ de la M\'ecanique  
quantique. 
\medskip 
Il est facile de v\'erifier que la corr\'elation $2\Phi (\theta ) - 1$ 
satisfait l'in\'egalit\'e de Bell: en particulier si on prend les quatre  
angles de la figure 59, on trouve $\rho = -2$. Cela correspond bien \`a  
ce qu'on a toujours fait dire \`a l'in\'egalit\'e de Bell: que la particule  
est  une particule classique.   
\medskip 
Mais voici maintenant le point essentiel. Cette corr\'elation $\rho = 
2\Phi (\theta ) - 1$ est la cons\'equence directe de notre hypoth\`ese 
d'\'equiprobabilit\'e: que tous les points de la 
circonf\'erence sont \'equivalents. La sym\'etrie circulaire est une 
cons\'equence n\'ecessaire des hypoth\`eses de Bell, 
puisque ces hypoth\`eses consistent \`a admettre que le choix du hasard 
a \'et\'e effectu\'e avant que les jumeaux ne traversent les analyseurs, 
donc ind\'ependamment des orientations que prendront ces derniers. Par
contre cette invariance circulaire n'existe plus dans les analyseurs, 
puisque ceux-ci privil\'egient une direction particuli\`ere.  Cela 
implique que dans le mod\`ele,  on pourrait obtenir les corr\'elations de   
la M\'ecanique quantique en renoncant \`a cette invariance. Et en effet,  
il suffit de remplacer la fonction $2\Phi (\theta ) - 1$ par 
$-\cos\theta$, ce qui \'equivaut \`a remplacer $\Phi (\theta )$ par 
${1\over 2}\big[ 1 - 2\cos\theta \big] = \sin^2{\theta\over 2}$. Si on 
interpr\`ete cela g\'eom\'etriquement (figure 63),  on arrive \`a la 
conclusion que ce ne  sont pas les points de la circonf\'erence qui sont 
\'equivalents, mais les points du diam\`etre sur lesquels se projette 
orthogonalement le point  de la circonf\'erence qui rep\`ere le secteur 
d'arriv\'ee de la particule.   
\medskip 
\midinsert 
\epsfxsize=\hsize
\line{\epsfbox{../images/fig63.eps} } 
\vskip3mm
\centerline{\eightpoint figure 63} 
\vskip4pt 
\centerline{\vbox{\hsize=11cm \eightpoint  
On n'obtient pas les m\^emes lois de probabilit\'es (et par cons\'equent  
pas les m\^emes corr\'elations) selon que les \'epreuves \'equiprobables  
sont les points $P$ de la circonf\'erence ou leurs projections $Q$ sur un 
diam\`etre donn\'e, comme cela avait d\'ej\`a \'et\'e vu au chapitre  
{\bf  I} \`a propos des cordes sur un cercle.}  } 
\vskip3mm 
\endinsert  
Posons donc que la probabilit\'e pour que l'angle $\varphi$ d'une 
particule quantique tombe dans un secteur angulaire  compris entre les 
angles  $\alpha$ et $\beta$ serait, non pas proportionnelle \`a $\beta - 
\alpha$,  mais \`a $\cos\alpha -\cos\beta$. Cela exige qu'on choisisse  
une origine particuli\`ere sur la circonf\'erence pour  compter les angles: 
en effet  $\beta - \alpha$ est ind\'ependant de l'origine choisie, mais 
pas $\cos\alpha -\cos\beta$. Autrement dit cela exige qu'on distingue  
une  direction particuli\`ere dans le disque. {\bf Mais cela est justement 
le cas dans l'exp\'erience que nous discutons: le champ magn\'etique de 
l'aimant de Stern-Gerlach et  son gradient d\'efinissent  bien une  
direction particuli\`ere}.   
\medskip 
Voyons cela de plus pr\`es. Avant de discuter des jumeaux, consid\'erons 
une seule particule et un seul disque qui, dans notre mod\`ele,  
est la repr\'esentation  symbolique d'un analyseur de Stern-Gerlach.  
L'analyseur a par nature  une orientation  bien d\'efinie (celle du champ 
magn\'etique  et de son gradient). Sur le disque nous la repr\'esentons  
par le  diam\`etre $NS$ (cf. figure 60). Ce diam\`etre  d\'etermine une 
origine des angles (le point $N$), \`a partir de laquelle on  compte les 
angles  pour rep\'erer un point de la circonf\'erence. Et maintenant, au 
lieu d'admettre que tous les angles sont \'equiprobables, on va admettre 
que ce sont leurs cosinus qui sont \'equiprobables. Pour \^etre plus 
pr\'ecis: on rep\`ere le secteur d'arriv\'ee de la particule  par un point  
$P$ de la circonf\'erence (figure 63); ce point se projette sur le 
diam\`etre $NS$ en $Q$,  et comme deux points $P$ de la circonf\'erence 
peuvent se projeter au m\^eme point $Q$ on d\'edouble ce diam\`etre  
pour distinguer son  c\^ot\'e gauche de son c\^ot\'e droit (donc sa  
longueur totale est quatre fois le rayon). On pose que la loi de 
probabilit\'e \`a laquelle  ob\'eit le processus est {\bf 
l'\'equiprobabilit\'e des points $Q$ le long  du diam\`etre}.  Cette
loi pour la variable $\varphi$, qui est ici la $\lambda$ de Bell, a une 
densit\'e qui d\'epend de $a$ et de $b$ (en fait de $\theta$), et c'est 
pr\'ecis\'ement ce qui a pour effet d'invalider les in\'egalit\'es de Bell.
Ainsi la loi de $\varphi$ , pour \^etre compatible avec les faits,  
d\'epend des directions $a$ et $b$, ce qui est naturel puisque 
l'appareil de la figure 58 privil\'egie ces deux directions. C'est l\`a 
la conclusion \`a laquelle conduit l'analyse du mod\`ele g\'eom\'etrique. 
\medskip  
\`A partir de cette loi il est ais\'e de calculer la probabilit\'e pour que  
deux particules sym\'etriques (c'est-\`a-dire telles que $\varphi_G + 
\varphi_D = 2\pi$) aboutissent toutes deux dans la r\'egion claire 
correspondant au cas o\`u les signes sont oppos\'es (cf. figure 61): elle 
est \'egale \`a $(1 - \cos\theta ) / 2$. En effet, les deux particules 
\'etant diam\'etralement oppos\'ees, l'\'ev\'enement ``elles arrivent 
toutes deux dans  la zone claire'' est identique \`a l'\'ev\'enement ``la 
jumelle de gauche arrive dans la zone claire'', ou \`a l'\'ev\'enement ``la 
jumelle  de droite arrive dans la zone claire''. La probabilit\'e de cet 
\'ev\'enement, qui \'etait $|\theta | / \pi$ lorsqu'on partait de 
l'\'equiprobabilit\'e des points de la circonf\'erence, est maintenant 
${1 \over 2}(1  - \cos\theta )$, car les  points situ\'es sur les deux arcs  
de cercle correspondants se projettent sur le diam\`etre $NS$ du disque  
de gauche (vertical sur la figure 61) selon  deux segments de longueur  
$(1 - \cos\theta) \times {\sl rayon}$ chacun; la longueur totale du  
diam\`etre d\'edoubl\'e \'etant $4 \times {\sl rayon}$, le rapport est  
bien ${1\over 2}(1 - \cos\theta )$. Le calcul a \'et\'e effectu\'e \`a  
partir de la jumelle de gauche, mais cela ne  contredit nullement la 
sym\'etrie entre les deux jumelles: si on avait  bas\'e le calcul sur la  
jumelle de droite, on aurait tout projet\'e sur le diam\`etre $NS$ du 
disque de droite (oblique sur la figure 61), et \'evidemment trouv\'e le 
m\^eme r\'esultat.  
\vskip6pt plus5pt minus4pt 
Les probabilit\'es et les corr\'elations en $-\cos\theta$ de la M\'ecanique 
quantique peuvent donc se calculer a priori en partant non pas de 
l'\'equiprobabilit\'e des points de la circonf\'erence, mais de  
l'\'equiprobabilit\'e de leurs projections sur le diam\`etre $NS$  
repr\'esentant l'orientation de l'analyseur respectif. Cette nouvelle 
invariance ne se ram\`ene \`a aucune des invariances macroscopiques de 
l'espace. L'apparente d\'ependance par rapport \`a l'orientation de {\it  
l'autre} analyseur ne provient que du comptage des co\"\i ncidences 
de signe: les points sur la circonf\'erence qui repr\'esentent les jumeaux 
de m\^eme spin sont situ\'es dans la zone claire de la figure 61, qui 
est d\'elimit\'ee par les orientations des {\it deux} aimants. Autrement dit, 
il faut \'evidemment conna\^\i tre l'orientation de l'{\it autre} analyseur 
pour d\'elimiter cette zone, mais tant qu'on \'etait dans le domaine 
classique, cela ne choquait personne car il \'etait tenu pour \'evident 
qu'il ne s'agissait pas d'une influence \`a distance: pour d\'ecompter les  
particules qui ont le  m\^eme spin que leur jumeau \'eloign\'e, on est bien   
oblig\'e d'aller voir le spin de l'autre jumeau. Dans le cas de la loi 
quantique, on pourrait cependant penser \`a  premi\`ere vue que projeter 
les deux particules sur le m\^eme diam\`etre  contredit la localit\'e car 
l'une des particules serait ainsi projet\'ee selon l'orientation de l'autre 
analyseur; mais si  on projette chaque particule sur le diam\`etre de son 
analyseur on trouve la m\^eme  loi de  probabilit\'e.  
\vskip6pt plus5pt minus4pt 
\`A  premi\`ere vue la nouvelle r\`egle d'\'equiprobabilit\'e est 
\'etrange  et le mod\`ele \'etudi\'e ici tr\`es artificiel;  mais ce qui est 
artificiel est beaucoup moins  l'\'equiprobabilit\'e des  points  du
diam\`etre, que le parti pris de vouloir d\'ecrire l'\'etat interne par 
l'angle $\varphi$. Si nous vivions dans le monde quantique, c'est
l'introduction de cet angle $\varphi$ qui nous semblerait absurde.
Cependant sa prise en compte dispense d'imaginer des actions physiques
\`a  distance. Ce que montre essentiellement cette r\`egle, c'est que le
ph\'enom\`ene quantique ne subit pas les invariances euclidiennes, mais
celles qui sont cr\'e\'ees par ``l'appareil de mesure'', o\`u les deux
directions $a$ et $b$ sont privil\'egi\'ees; ce qui au fond est 
parfaitement logique.
\vskip6pt plus5pt minus4pt 
La moralit\'e de cette histoire est la suivante: les corr\'elations E.P.R.  
ne r\'esultent pas d'actions \`a distance, mais simplement du fait que 
dans le monde quantique (donc \`a l'\'echelle atomique) il y a d'autres 
sym\'etries que celles de l'environnement spatial local et instantan\'e,
comme nous l'avons d\'ej\`a vu en comparant les boules aux particules 
de Bose. \`A  la  section {\bf I.2}, \`a propos des cordes sur un cercle,
 nous avions fait remarquer que quelqu'un qui observe leur distribution
peut en d\'eduire des  renseignements sur la mani\`ere dont elles ont
\'et\'e choisies (le niveau o\`u le pur hasard est intervenu). De
l'exp\'erience E.P.R. cette personne peut \'egalement d\'eduire que ce
n'est certainement pas une orientation particuli\`ere (un vecteur de
moment cin\'etique) des atomes jumeaux par rapport  \`a l'espace vide
que la Fortune a choisi. \'Egalement \`a la section  {\bf I.2}, nous avions
vu que les particules de Bose diff\'eraient  des particules classiques (les
boules) par la nature de l'invariance. Nous disions 
que  ``le hasard pur n'intervenait pas au m\^eme niveau'': pour les boules 
(classiques) il intervenait dans le choix des bo\^\i tes, pour les 
particules de Bose (quantiques) il intervenait dans le choix des modes 
d'occupation. Nous retrouvons ici la m\^eme diff\'erence entre le 
classique et le quantique: les choix \'equiprobables ne sont pas 
effectu\'es sur le m\^eme type d'\'epreuves. L'explication des 
corr\'elations E.P.R. est que ``le hasard pur'' ou ``la Fortune'' choisit 
les points sur l'axe $NS$ de l'analyseur et non sur la circonf\'erence,  
tout comme l'explication de la loi de Planck est que ``le hasard pur'' 
choisit les modes collectifs d'occupation et non les \'etats 
individuellement accessibles \`a chaque photon.  Il est bien clair que 
si on mod\'elise cette situation en prenant pour espace des \'epreuves
l'ensemble des points du diam\`etre (ou plus artificiellement l'ensemble 
des points de la circonf\'erence, mais avec la densit\'e compensatoire 
${1\over 2} \sin\varphi$) le Calcul des probabilit\'es sera applicable  
aux \'ev\'enements qui en sont des sous-ensembles; la formule des 
probabilit\'es conditionnelles sera vraie pour de tels \'ev\'enements,
alors que cela n'\'etait pas \'evident pour la mani\`ere dont elle \'etait
appliqu\'ee dans le raisonnement conduisant \`a l'in\'egalit\'e de Bell. 
Mais cela signifie que ``le hasard pur'' choisit en tenant compte du
r\^ole particulier jou\'e par le diam\`etre $NS$.  On retrouve ainsi notre
conclusion que la loi $p_j$ de la variable $\lambda$ (qui,  dans notre
petit mod\`ele,  est l'angle $\varphi$),  d\'epend de l'angle $\theta$. 
\medskip 
C'est donc bien ici que se trouve l'erreur: l'exp\'erience d'Aspect prouve 
qu'{\it il n'existe pas} de variable $\lambda$ dont la loi est 
ind\'ependante des directions $a$ et $b$; autrement dit: 
\smallskip 
a) le hasard n'intervient pas avant l'interaction avec les aimants,  auquel
cas il pourrait ignorer leurs directions,  mais sur l'ensemble du processus;
\smallskip 
b) lorsqu'il intervient, l'invariance qui le caract\'erise n'est pas
conditionn\'ee par l'espace vide, mais par les deux directions caract\'eristiques. 
\medskip 
Dans $(13.10)$ par exemple, ce n'est pas l'ind\'ependance stochastique 
entre les deux analyseurs qui est en cause, mais plus fondamentalement le
fait que la loi $p_j$ elle-m\^eme d\'epend des directions.  On l'a vu tr\`es
clairement avec le mod\`ele g\'eom\'etrique. 
\medskip
C'est la seule interpr\'etation raisonnable.  L'appel \`a une {\og interaction
de type nouveau,  inconnu\fg} n'est pas raisonnable:  elle consiste \`a donner
plus de valeur \`a des visions purement imaginaires h\'erit\'ees d'habitudes
de pens\'ees s\'eculaires,  conditionn\'ees par le monde macroscopique. 
On n'a jamais pu diviser le processus (le {\it ph\'enom\`ene}),  r\'eellement, 
concr\`etement,  exp\'erimentalement;  il est donc ill\'egitime d'inventer des
subdivisions de ph\'enom\`enes.  De telles inventions sont certes une pente
naturelle de l'esprit humain:  les peuples antiques,  \'eduqu\'es depuis des
mill\'enaires dans les cosmologies religieuses et mythiques,  \'etaient tout
naturellement enclins \`a imaginer des dieux invisibles derri\`ere chaque
nouveau myst\`ere de la nature.  \^Etre rationnel,  c'est savoir se m\'efier
de l'imagination et s'incliner devant les r\'ev\'elations de la nature. 
\medskip 
L'id\'ee d'une action \`a distance est li\'ee \`a l'abandon de  
l'ind\'ependance stochastique car,  comme on l'a vu au chapitre {\bf IV}, 
celle-ci exprime l'ind\'e\-pen\-dance causale;  l'abandonner revient \`a 
admettre une influence.  Mais on vient de voir que ce n'est pas 
l'ind\'ependance stochastique  qui est source de l'erreur.  Alors qu'il est 
logique que la loi $p_j$ refl\`ete le type de sym\'etrie cr\'e\'e par la 
pr\'esence des deux directions $a$ et $b$,  rien dans les principes du 
Calcul des probabilit\'es n'impose une relation causale entre les deux 
analyseurs.  Seule la persistance des conceptions classiques, 
l'incapacit\'e fonci\`ere de renoncer \`a l'image de la petite bille,  
conduit \`a une telle conclusion. 

\vskip6mm plus3mm minus3mm

{\bf XIII. 5. Conclusion.}
\medskip
Il ne faut pas chercher dans le mod\`ele pr\'esent\'e une explication plus 
profonde de la M\'ecanique quantique. Il s'agissait seulement d'illustrer 
les hypoth\`eses qui conduisent \`a l'in\'egalit\'e de Bell sous une forme 
g\'eom\'etrique moins abstraite que la d\'emonstration la plus  
g\'en\'erale. La figure 62 montre tr\`es concr\`etement la diff\'erence   
entre le comportement classique et le comportement quantique. L'angle 
$\varphi$ (ou plus g\'en\'eralement la variable al\'eatoire $\lambda$ de  
Bell) exprime le pr\'ejug\'e classique d'une bille {\it orient\'ee dans 
l'espace}, qui conduit in\'evitablement \`a la fonction affine par 
morceaux de la figure 62, et qui v\'erifie l'in\'egalit\'e de Bell. La  
courbe qui repr\'esente  les ph\'enom\`enes r\'eels est au contraire 
sinuso\"\i dale, et viole l'in\'egalit\'e de Bell. 
\medskip 
On pourrait ajouter que le ph\'enom\`ene naturel dont nous discutons  
et qui ne consiste au fond en rien d'autre que la corr\'elation en 
$-\cos\theta$ est bien plus qu'un simple constat exp\'erimental 
et est l'expression des n\'ecessit\'es absolument fondamentales  
suivantes, que nous avons d\'ej\`a mentionn\'ees:     
\smallskip  
a) l'exp\'erience repr\'esent\'ee sur la figure 58 est invariante 
par  rotation globale dans l'espace;   
\smallskip 
b) macroscopiquement on retrouve la M\'ecanique classique  
o\`u le moment cin\'etique est un vecteur\ftn{8}{Le fait que la corr\'elation
soit $-\cos\theta$ et non une autre fonction de $\theta$ est une 
cons\'equence n\'ecessaire de a) et b). Cela a \'et\'e \'etabli: voir 
Claude Comte {\it Symmetry, Relativity, and Quantum Mechanics}, 
Il Nuovo Cimento B, vol {\bf 111} ({\oldstyle 1996}.}.  
\medskip 
La condition b) exprime le fait que si on mesure,  dans l'aimant de
Stern-Gerlach,  le {\og spin selon $Ox$\fg} pour une grosse particule
(non quantique),  form\'ee par exemple de $N \geq 1$ quantons de spin
${1\over 2}\hbar$ chacun,  on doit retrouver les relations de la
M\'ecanique classique et notamment que la composante $M_x$ selon
$Ox$ du moment cin\'etique ${\vec M}$ est \'egale \`a $M \cos\alpha$, 
$\alpha$ \'etant l'angle entre l'axe $Ox$ et le vecteur ${\vec M}$. 
Le cosinus qui appara{\^\i}t dans $M_x = M\cos\alpha$ est li\'e \`a
celui de la corr\'elation ${\bf E}(a,b) = -\cos\theta$,  c'est-\`a-dire
que si ${\bf E}(a,b)$ \'etait une fonction de $\theta$ autre que $-\cos\theta$, 
on ne retrouverait plus,  \`a la limite des gros syst\`emes,  la relation
$M_x = M \cos\alpha$,  $M_x$ \'etant la somme des {\og spins selon $Ox$\fg}
de tous les quantons et $M = N \times {1\over 2}\hbar$. 
\medskip
En d'autres termes:  si l'exp\'erience avait donn\'e tort \`a la M\'ecanique
quantique et raison \`a l'in\'egalit\'e de Bell, on ne comprendrait plus
pourquoi le monde macroscopique ob\'eit \`a la M\'ecanique classique!  
\medskip 
S'il est assez clair apr\`es la discussion pr\'ec\'edente que les signaux 
qui partiraient pr\'evenir chaque analyseur de ce qui est arriv\'e dans 
l'autre est plut\^ot de l'ordre des na\"\i vet\'es comme l'ange qui 
pousserait les plan\`etes ou les fantasmes d'Archibald de la Cruz,  et 
n'est en tous cas pas n\'ecessaire puisque l'expression math\'ematique 
$R = -\cos\theta$ exprime toute l'information que nous poss\'edons,  il  
n'est par contre pas \'evident du tout que {\it toute} id\'ee  d'\'etat   
interne doive \^etre abandonn\'ee.  Toutefois, pour que le recours \`a une 
notion d'\'etat interne plus profonde que ne le permet la M\'ecanique 
quantique soit justifi\'e, il faut que ce soit une hypoth\`ese cr\'eatrice 
et non un artifice uniquement destin\'e \`a sauver des pr\'ejug\'es. 
\medskip 
En tant qu'auteur de cet ouvrage,  je dois \`a ce propos pr\'evenir les  
mauvais proc\`es qui pourraient m'\^etre intent\'es.  Dans les discussions 
ou les pol\'emiques qui ont entour\'e le probl\`eme de l'existence d'un 
\'etat interne,  la position qui consiste \`a dire ``l'expression 
math\'ematique $R = -\cos\theta$ exprime toute l'information que nous 
poss\'edons'' a souvent \'et\'e d\'enonc\'ee comme positiviste (un terme  
\'evidemment p\'ejoratif).  Le positivisme, du moins dans le sens 
p\'ejoratif du mot,  est une attitude dogmatique,  qui d\'evalorise ou 
r\'ecuse a priori la recherche d'une v\'erit\'e plus profonde.  
\medskip 
L'un des exemples les plus fameux d'un tel comportement dogmatique a
\'et\'e le refus de l'hypoth\`ese atomique. \`A partir de {\it 1860}\ftn{9}
{R. Clausius: {\it \"Uber die Art der Bewegung, die wir W\"arme nennen.}
Poggendorffer Annalen, vol. {\bf 100} ({\oldstyle 1857}). 
\smallskip
J. C. Maxwell: {\it Illustrations of the Dynamical Theory of Gases}
Philosophical Magazine ({\oldstyle 1860}). 
\smallskip
L. Boltzmann: {\it \"Uber die Beziehung zwischen dem zweiten Hauptsatze
der mechanischen W\"armetheorie und der Wahrscheinlichkeitsrechnung
respektive den S\"atzen \"uber das W\"armegleichgewicht}.  Wiener
Berichte vol. {\bf 76} ({\oldstyle 1877}).},  R. Clausius,  J. C. Maxwell, 
puis plus tard L. Boltzmann (mais l'id\'ee  remonte \`a Daniel Bernoulli au
d\'ebut du $XVIII^{\rm e}$ si\`ecle) d\'evelopp\`erent une interpr\'etation
statistique de la Thermodynamique  en postulant  que les propri\'et\'es
li\'ees \`a la chaleur, notamment pour les gaz, s'expliquent
math\'ematiquement par le mouvement al\'eatoire des mol\'ecules.  C'est
l'origine de ce qui s'appelle aujourd'hui la {\it Physique statistique} et 
le plus grand succ\`es du Calcul des probabilit\'es.  Jusque l\`a, la 
Thermodynamique avait \'et\'e une th\'eorie purement axiomatique et 
Boltzmann proposait de l'expliquer \`a partir d'une v\'erit\'e plus 
profonde,  celle des mouvements d\'esordonn\'es,  chaotiques (et donc 
stochastiques) des atomes et des mol\'ecules.  Beaucoup de physiciens 
refus\`erent une telle interpr\'etation en disant que la r\'ealit\'e des 
atomes n'est pas prouv\'ee par l'exp\'erience et serait donc une 
hypoth\`ese purement m\'etaphysique, inutile car ``la Thermodynamique 
axiomatique d'avant {\oldstyle 1860} exprime toute l'information que nous 
poss\'edons''.  
\medskip 
L'Histoire a donn\'e raison \`a Clausius,  Maxwell,  et Boltzmann non
pas simplement parce que la r\'ealit\'e des atomes a fini par \^etre 
prouv\'ee,  mais surtout parce que l'hypoth\`ese statistique est
cr\'eatrice;  par exemple Planck n'aurait pas pu trouver l'explication
du rayonnement du corps noir s'il n'avait pas suivi un raisonnement de
Physique statistique.  Si on poss\`ede une interpr\'etation de lois
existantes \`a partir d'une v\'erit\'e plus profonde,  on peut deviner
des lois nouvelles. 
\medskip
Pour donner un autre exemple plus proche du sens commun: 
conna\^\i tre tous les sympt\^omes cliniques des maladies infectieuses
apporte peu de moyens pour les combattre;  mais si on sait qu'elles
sont produites par des micro-organismes invisibles,  on peut chercher
un moyen de d\'etruire ceux-ci (asepsie).  Dans le but peut-\^etre 
louable en soi d'\'ecarter la m\'etaphysique,  doit-on refuser 
l'hypoth\`ese des micro-organismes parce qu'elle n'est pas n\'ecessaire 
pour conna\^\i tre les sympt\^omes cliniques?  C'est en cela que le 
posi\-ti\-visme est dogmatique;  les m\'esaventures de Boltzmann ont 
fortement discr\'edit\'e cette position philosophique.     
\medskip  
Avec l'exp\'erience $E.P.R.$ a \'et\'e soulev\'ee (notamment par Einstein) 
la question de savoir s'il n'y aurait pas une {\it v\'erit\'e plus profonde} 
qui expliquerait les lois quantiques comme la Physique statistique 
explique la Thermodynamique,  comme le code g\'en\'etique explique la  
transmission de la mucoviscidose, etc.  Mais la recherche de cette 
v\'erit\'e plus profonde doit partir de l'exp\'erience et non de nos 
pr\'ejug\'es philosophiques sur la causalit\'e.  
\medskip 
Je ne voudrais donc pas \^etre mal compris. En paraphrasant une  
d\'ecla\-ra\-tion c\'el\`ebre de Heisenberg,  j'ai bien \'ecrit ``l'expression  
math\'e\-ma\-ti\-que $R = -\cos\theta$ exprime toute l'information que
nous poss\'edons''.  Mais je n'ai pas ajout\'e ``inutile de chercher plus 
loin''. Il faut bien voir ceci:  l'exp\'erience $E.P.R.$ est un d\'efi pour
la mani\`ere  dont nous concevons la causalit\'e; toutefois l'exp\'erience 
d'Aspect est une {\it exp\'erience} et est donc plus s\^urement vraie que  
les pr\'ejug\'es.  Chercher une v\'erit\'e plus profonde doit \^etre le but, 
mais ce n'est pas \^etre positiviste que de penser que lorsque celle-ci 
sera connue,  elle sera tr\`es diff\'erente des mod\`eles na\"\i fs que  
nous inspirent encore une vision de la causalit\'e h\'erit\'ee de la 
M\'ecanique classique.  L'assertion ``l'expression math\'ematique $R = 
-\cos\theta$ exprime toute l'information que nous poss\'edons'' est 
parfaitement exacte:  nous ne poss\'edons aujourd'hui aucune autre
{\it  information}.  D'ailleurs ---~pour prendre un autre exemple
digne d'\^etre suivi~--- Einstein,  confront\'e \`a la th\'eorie
\'electromagn\'etique de Maxwell dans laquelle la vitesse de la lumi\`ere
est la m\^eme pour tous les observateurs quelles que soient leurs
vitesses relatives (confirm\'ee sur ce point par les exp\'eriences
de Michelson et Morley en {\oldstyle 1887}),  n'a pas cherch\'e des
mod\`eles na\"\i fs qui auraient pr\'eserv\'e un temps absolu:  il a
d\'eduit la Relativit\'e de l'{\it information} apport\'ee par
l'exp\'erience de Michelson-Morley.  On a souvent accus\'e Bohr et
Heisenberg d'\^etre des positivistes,  c'est-\`a-dire de se conduire
comme les adversaires de l'hypoth\`ese mol\'eculaire.  Je pense que ce 
jugement est erron\'e,  voire grossier,  et j'esp\`ere que ce chapitre 
qui apporte la rigueur du Calcul des probabilit\'es,  aidera \`a 
corriger ce jugement.  
\medskip 
Mais cet objectif est secondaire:  j'esp\`ere surtout que cette discussion 
sur l'exp\'erience E.P.R. aura montr\'e qu'un formalisme math\'ematique  
ne constitue jamais une v\'erit\'e universelle et que la conscience de sa 
signification r\'eelle doit toujours rester pr\'esente.  


\bye 



