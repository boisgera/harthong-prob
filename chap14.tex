\input /home/harthong/tex/formats/twelvea4.tex
\input /home/harthong/tex/formats/epsf.tex

\auteurcourant={\sl J. Harthong: probabilit\'es et statistique}
\titrecourant={\sl L'irr\'eversibilit\'e}

\pageno=418

\font\bigsl=cmti17

\newdimen\blocksize   \blocksize=\vsize  \advance\blocksize by -11pt
\newdimen\imgsize   \imgsize=60mm  

\null\vskip10mm plus4mm minus3mm

\centerline{\tit XIV\hskip-2.5pt . \hskip2.5pt L'IRR\'EVERSIBILIT\'E.}

\vskip8mm plus4mm minus3mm

{\sl \`A cause de l'abondance de la litt\'erature sur le sujet, 
ce chapitre comporte une bibliographie s\'epar\'ee, plac\'ee \`a la fin 
de l'annexe, page 458. Les chiffres entre crochets, par exemple {\rm [9]},
r\'ef\`erent \`a cette bibliographie.}

\vskip6mm plus3mm minus3mm

{ \bf 1. Introduction.}
\medskip
Le probl\`eme de l'irr\'eversibilit\'e est celui de l'\'evolution des 
syst\`emes {\it macroscopiques}, c'est-\`a-dire constitu\'es d'un nombre 
immense de mol\'ecules en perp\'etuelle agitation. L'exemple typique est le 
gaz; mais n'importe quel objet mat\'eriel dont la masse est de l'ordre du 
gramme ou plus, est un tel corps, puisque les mol\'ecules qui le composent, 
m\^eme si elles sont tr\`es grosses, ont des masses infinit\'esimales si 
on les compare au gramme. Nous avons d\'ej\`a abord\'e l'\'etude probabiliste 
de tels corps, notamment en {\bf II.6} (probl\`eme du corps noir). 
L'irr\'eversibilit\'e est le fait que par exemple un gaz enferm\'e au 
d\'epart dans un r\'ecipient va se r\'epandre dans l'espace si on ouvre 
le r\'ecipient, alors qu'il n'arrivera jamais que du gaz r\'epandu dans 
l'espace se r\'etracte jusqu'\`a revenir de lui-m\^eme s'enfermer dans un
r\'ecipient. 
L'irr\'eversibilit\'e est \`a la base de tous les ph\'enom\`enes naturels: 
les cadavres se d\'ecomposent et ne redeviennent jamais vivants, les vieux 
tuyaux rouill\'es ne redeviennent jamais neufs, des livres br\^ulent et 
partent en fum\'ee mais les cendres et la fum\'ee ne redonneront plus jamais 
les livres, etc. etc. 
\medskip
Le pr\'esent chapitre est consacr\'e aux fondements probabilistes de 
l'irr\'e\-ver\-si\-bi\-lit\'e. Nous avons d\'ej\`a vu au chapitre 
{\bf I} comment le chaos d\'eterministe cr\'ee du hasard;  au 
chapitre {\bf IV} comment il efface la causalit\'e;  au chapitre 
{\bf VII} comment le hasard se transforme en d\'eterminisme. 
Ici il s'agit de comprendre comment le chaos d\'eterministe, dont
l'\'evolution est en principe r\'eversible, efface la r\'eversibilit\'e. 
Cela implique que nous ne discuterons pas les fondements quantiques de 
l'irr\'eversibilit\'e, car cela sortirait du cadre du pr\'esent ouvrage 
(consulter le {\it Cours de Physique de Berkeley}, tome 5, Physique 
statistique). Pour comprendre il nous faut des mod\`eles simples et la 
M\'ecanique classique y pourvoit; le principe de base de l'{\it effacement 
de la r\'eversibilit\'e} est essentiellement le m\^eme en quantique qu'en
classique; le premier est juste techniquement plus difficile.
\medskip
Les corps macroscopiques se modifient spontan\'ement du fait de l'agitation
perp\'etuelle des mol\'ecules qui les composent. Toutefois les objets 
{\it isol\'es}, c'est-\`a-dire \'ecart\'es de tout contact ou \'echange 
avec le reste de l'univers, y compris l'\'emission ou l'absorption de
rayonnement, \'evoluent vers un \'etat 
asymptotique stable, appel\'e {\it \'etat d'\'equilibre}. Le simple sens 
commun suffit \`a le comprendre: si un corps m\'etallique est maintenu \`a 
l'\'ecart de tout \'echange, il ne pourra s'oxyder puisqu'il faut pour 
cela une action de l'oxyg\`ene sur le corps. De m\^eme un liquide au
repos dans un r\'eservoir ne se mettra \`a couler que si on bascule ou
perce le r\'eservoir, l'air calme ne peut commencer \`a \^etre agit\'e par 
le vent que s'il est expos\'e \`a des masses d'air plus chaudes ou plus 
froides, etc. Si au lieu de mettre le corps m\'etallique fra{\^\i}chement 
poli \`a l'abri de l'oxyg\`ene et du monde ext\'erieur,  on
isole ce corps {\it avec une certaine quantit\'e d'oxyg\`ene} de tout 
le reste, il va s'oxyder progressivement jusqu'\`a ce qu'il n'y ait 
plus assez d'oxyg\`ene pour que la corrosion se poursuive, et on 
atteint aussi un \'etat d'\'equilibre. Pour le sens commun, forg\'e 
par l'exp\'erience quotidienne, il est bien \'evident qu'une fois la 
surface m\'etallique corrod\'ee, il n'arrivera plus jamais que l'oxyg\`ene 
se retire spontan\'ement du m\'etal pour retrouver sa forme gazeuse, en 
rendant ainsi tout son brillant \`a la surface m\'etallique. C'est 
pourquoi on dit que la corrosion est une \'evolution {\it irr\'eversible}. 
Bien entendu, il est possible d'appliquer au m\'etal un traitement chimique 
qui s\'epare \`a nouveau l'oxyg\`ene et le m\'etal, mais cela brise alors
l'isolement du syst\`eme {\it corps m\'etallique plus oxyg\`ene}.
\medskip
Cette notion d'{\it \'etat asymptotique stable} est essentielle pour 
comprendre l'irr\'e\-ver\-si\-bi\-lit\'e. Plus pr\'ecis\'ement, ce qui est 
essentiel est que ces ``\'etats stables'' ne sont stables que dans leur
apparence macroscopique; l'\'etat \hbox{exact} du mouvement des mol\'ecules,
lui, n'est \'evidemment pas stable puisque l'agitation mol\'eculaire ne 
s'arr\^ete pas.  Il s'agit d'une stabilit\'e de {\it l'apparence 
macroscopique} qui est maintenue par le mouvement incessant des mol\'ecules. 
Pour clarifier la question on introduira un langage pr\'ecis:  
l'\'etat exact du mouvement des mol\'ecules sera appel\'e l'\'etat 
{\it m\kern0.45pt{\bigsl i\kern0.6pt}croscopique} du syst\`eme, tandis que
l'ensemble des param\`etres caract\'erisant l'apparence macroscopique du
syst\`eme sera son \'etat {\it m{\bigsl a\kern0.3pt}croscopique}.
L'\'etat asymptotique stable, ou \'etat d'\'equilibre, est donc un \'etat 
{\it macroscopique} du syst\`eme. Les \'etats microscopiques du 
syst\`eme \`a l'\'equilibre sont \'evidemment instables et en perp\'etuelle 
transformation. Ceci sera examin\'e en d\'etail \`a la section {\bf 4}.
\medskip
\`A la suite des travaux de Sadi Carnot ({\it R\'eflexions sur la puissance 
motrice du feu}, {\oldstyle 1824}) Rudolph Clausius a d\'egag\'e le concept
de l'{\it entropie} d'un tel syst\`eme isol\'e. L'entropie est une grandeur 
associ\'ee \`a l'\'etat macroscopique du syst\`eme; en langage math\'ematique 
on dirait que c'est une fonction de l'\'etat macroscopique, \`a valeurs 
r\'eelles.  L'entropie permet d'exprimer l'irr\'eversibilit\'e des 
syst\`emes macroscopiques sous une forme quantitative: c'est le 
{\it second principe de la Thermodynamique}, qui affirme que l'entropie 
d'un syst\`eme isol\'e ne peut jamais diminuer. Carnot analysait le principe 
des machines thermiques, qui produisent du mouvement \`a partir de la 
chaleur, en injectant de la vapeur ou de l'air sous pression dans un 
cylindre pour d\'eplacer un piston. Il a constat\'e que la vapeur devait 
n\'ecessairement se refroidir, et qu'avec une quantit\'e de charbon 
donn\'ee, l'\'energie m\'ecanique qu'on peut r\'ecup\'erer est d'autant 
plus \'elev\'ee que la vapeur a plus refroidi. Plus pr\'ecis\'ement il 
\'etablit la loi quantitative suivante: Si $T_1$ est la temp\'erature 
(absolue, en degr\'es Kelvin) \`a laquelle on a chauff\'e la vapeur et 
$T_0$ la temp\'erature \`a laquelle se refroidit cette vapeur apr\`es le 
passage dans le cylindre ou la turbine, l'\'energie m\'ecanique obtenue 
pour une quantit\'e de chaleur fournie $Q$ sera proportionnelle \`a 
$(1 - T_0/T_1) \times Q$ et non \`a $Q$ seul. Cela veut dire que si 
par exemple on chauffe de l'air \`a 273 degr\'es Celsius (546 Kelvin) 
dans un cylindre pour qu'il pousse un piston et d\'eplace ainsi un objet 
lourd, puis qu'on le refroidit \`a 0 degr\'es Celsius (273 Kelvin) pour que 
le piston se r\'etracte, le rapport $1 - T_0/T_1$ sera $0.5$ et le travail 
m\'ecanique de d\'eplacement de l'objet lourd aura \'et\'e la moiti\'e de 
l'\'energie calorifique d\'epens\'ee pour chauffer l'air dans le cylindre. 
L'autre moiti\'e se sera perdue dans le refroidissement de l'air. (N. B. 
cette perte par refroidissement est n\'ecessaire, car sinon le piston ne se
r\'etracte pas tout seul; il faudrait le pousser et donc perdre le
travail m\'ecanique qu'on vient de gagner).
\medskip
Le processus {\it inverse} de celui de la machine thermique consisterait 
\`a produire la chaleur \`a partir du mouvement m\'ecanique au lieu de 
l'obtenir en br\^ulant du  charbon. On peut produire de la chaleur \`a 
partir du mouvement par frottement; on peut m\^eme convertir enti\`erement 
l'\'energie m\'ecanique en chaleur: dans ce cas le mouvement est 
compl\`etement arr\^et\'e par l'effet des frottements. Or la loi de Carnot 
montre que, sauf si $T_0 = 0$ ou $T_1 = \infty$, la chaleur ne peut jamais 
\^etre enti\`erement convertie en mouvement. De toute fa\c con la condition 
$T_0=0$ est irr\'ealisable, car pour que la vapeur puisse \^etre refroidie
\`a $T_0=0$ il faut maintenir un syst\`eme de refroidissement bien plus
co\^uteux que l'\'energie produite par la machine. Ce constat fait par 
Carnot marque l'origine du probl\`eme th\'eorique de l'irr\'eversibilit\'e: 
la transformation d'\'energie m\'ecanique en chaleur par les frottements 
n'est pas r\'eversible, en ce sens qu'aucune machine thermique ne 
pourra retransformer int\'egralement la chaleur en le mouvement. 
La th\'eorie (la Thermodynamique) interpr\`ete cela en disant que la
transformation d'\'energie m\'ecanique en chaleur par les frottements
augmente l'entropie; pour retransformer toute la chaleur en \'energie
m\'ecanique il faudrait alors diminuer l'entropie. Quanti\-ta\-ti\-ve\-ment, 
si nous reprenons l'exemple ci-dessus avec la vapeur refroidie 
de $273$ degr\'es Celsius \`a $0$ degr\'es Celsius, on peut dire
que $4184$ joules de travail m\'ecanique permettent d'\'echauffer par 
frottement $1$ kilogramme d'eau de $1$ degr\'e, mais inversement, avec 
une machine thermique fonctionnant entre $273$ et $0$ degr\'es Celsius, 
cette m\^eme quantit\'e de chaleur ne permettrait de r\'ecup\'erer que 
$2092$ joules de travail m\'ecanique. Bien entendu dans une 
machine r\'eelle une grande partie de la chaleur fournie se perd aussi
par les d\'efauts d'isolation, en sorte qu'on r\'ecup\'ererait encore bien 
moins que ces $2092$ joules; la loi de Carnot concerne le cas id\'eal o\`u 
on aurait enti\`erement \'elimin\'e ces pertes. Elle dit que m\^eme si ces 
pertes sont rendues infinit\'esimales, il restera toujours une
irr\'eversibilit\'e de principe, car le fonctionnement m\^eme de la machine
exige qu'une partie de la chaleur soit perdue par le refroidissement.
\medskip
L'explication fondamentale du comportement des corps macroscopiques 
tels que la dilatation des gaz chauff\'es dans les machines thermiques, 
mais aussi l'\'ecoulement des liquides, l'\'evaporation, la fusion ou la 
solidification, les \'echanges de chaleur, etc, a \'et\'e trouv\'ee dans 
le comportement {\it al\'eatoire} des mouvements mol\'eculaires. C'est en 
appliquant la {\it loi des grands nombres} au mouvement chaotique d'un 
nombre immense de mol\'ecules qu'on retrouve le comportement des corps 
macroscopiques. La loi de Carnot mentionn\'ee plus haut peut \^etre 
d\'eduite ainsi, de m\^eme que toutes les lois gouvernant les 
\'echanges de chaleur, l'agitation des fluides, etc. Quoique cette 
explication statistique ait d\'ej\`a \'et\'e propos\'ee comme hypoth\`ese 
par Daniel Bernoulli (Hydrodynamica, 1731), elle n'a commenc\'e 
\`a devenir pleinement op\'eratoire que dans la seconde moiti\'e du 
$XIX^{\rm e}$ si\`ecle. Les travaux fondateurs de cette {\it M\'ecanique 
statistique} ont \'et\'e effectu\'es pour l'essentiel par J. C. Maxwell 
([2], {\oldstyle 1860}) et Ludwig Boltzmann ([1], {\oldstyle 1872}). 
L'irr\'eversibilit\'e mentionn\'ee 
pr\'ec\'edemment n'est qu'un aspect du comportement  des corps 
macroscopiques, et au fond, elle ne joue qu'un r\^ole tr\`es marginal dans 
les pr\'eoccupations des physiciens car elle ne vaut que comme principe 
g\'en\'eral et abstrait. Pour celui qui doit {\it calculer} ou d\'ecrire 
des ph\'enom\`enes pr\'ecis et particuliers, la {\it M\'ecanique 
statistique} est une science tr\`es technique dont le quotidien est 
bien \'eloign\'e des grands principes. Par contre, l'irr\'eversibilit\'e 
est le genre de probl\`eme qui a toujours fascin\'e les philosophes, 
ainsi que tous les amateurs passionn\'es de science, qui connaissent 
cette derni\`ere bien plus par les ouvrages de vulgarisation que par 
l'\'etude approfondie et patiente de probl\`emes concrets mais ardus. De 
ce fait, le th\`eme de l'irr\'eversibilit\'e inspire depuis Boltzmann toute 
une litt\'erature pseudo- ou para-scientifique, pleine de confusion, de 
r\^eve, et de visions inexactes ou m\^eme carr\'ement fausses. 
\medskip
Le point crucial de cette litt\'erature est le {\it paradoxe de Loschmidt}. 
Apr\`es la publication par Ludwig Boltzmann de l'{\it interpr\'etation 
mol\'eculaire de l'entropie} ([1], {\oldstyle 1872}), quoique ce travail n'ait 
pas eu imm\'ediatement un grand \'echo (probablement parce qu'il \'etait 
math\'ematiquement tr\`es ardu), d'autres publi\`erent des critiques. Il 
s'agissait l\`a d'une tradition de la Physique allemande, encourag\'ee par 
les \'editeurs des revues savantes: loin d'\^etre pol\'emiques ou li\'ees 
\`a des rivalit\'es, ces critiques favorisaient la discussion et
l'approfondissement des id\'ees [des pol\'emiques, il y en eut aussi, 
car l'hypoth\`ese mol\'eculaire avait des adversaires]. 
Parmi celles faites \`a Boltzmann, deux sont devenues c\'el\`ebres. 
\medskip 
Le premier article [3]
de ce type fut publi\'e en {\oldstyle 1876} par Joseph Loschmidt et 
contenait une vingtaine de critiques ou d'objections exigeant de la 
part de Boltzmann des explications plus d\'etaill\'ees; une seule de ces 
critiques est entr\'ee dans l'Histoire sous le nom de {\it paradoxe de
Loschmidt}; il s'agit du fait apparemment paradoxal que le syst\`eme 
dynamique ait globalement un comportement irr\'eversible alors que 
la M\'ecanique est enti\`erement r\'eversible: pour tout mouvement d'un
syst\`eme de points mat\'eriels tels que les mol\'ecules,
le mouvement inverse, c'est-\`a-dire celui qu'on verrait dans un film 
projet\'e en marche arri\`ere, est \'egalement possible et tout aussi 
pro\-bable. Boltzmann avait r\'epondu ([12] et [13]) \`a la question de
Loschmidt, et sa r\'eponse est essentiellement 
correcte. Elle peut certes \^etre affin\'ee par des connaissances plus 
r\'ecentes, mais rien ne change {\it sur le fond}. Ainsi Boltzmann 
postulait pour les mol\'ecules un mouvement newtonien, alors que la 
M\'ecanique statistique moderne postule un mouvement quantique, ce qui 
induit de grandes diff\'erences (satistiques de Fermi-Dirac et de 
Bose-Einstein). Mais l'argument de Loschmidt et la r\'eponse de Boltzmann 
\`a cet argument ne s'en trouvent pas affect\'es de mani\`ere vraiment
essentielle: les mouvements microscopiques quantiques sont, tout comme 
les classiques, parfaitement r\'eversibles, et la propri\'et\'e statistique
universelle qui explique l'irr\'eversibilit\'e est la m\^eme. La 
r\'eponse de Boltzmann repose sur l'observation suivante: les \'etats 
initiaux du syst\`eme qui feraient diminuer l'entropie existent, certes, 
mais sont si prodigieusement rares qu'il est pratiquement impossible de 
les rencontrer ou de les produire. Toutefois sous cette forme succinte 
l'argument laisse planer trop de malentendus [et c'est ce qui explique 
la persistance des incompr\'ehensions]. C'est bien pourquoi il faut ce 
chapitre entier pour discuter la question. 
\medskip
Un second article [11]
de ce type fut publi\'e en {\oldstyle 1896} par Ernst Zermelo et reprenait 
pour la communaut\'e de langue allemande des critiques de Henri Poincar\'e.
L'argumentation de Poincar\'e est bas\'ee sur son {\it th\'eor\`eme du
retour}: {\sl ``un syst\`eme dynamique de $N$ points mat\'eriels qui 
\'evolue au cours du temps en restant born\'e repassera au bout d'un 
temps fini aussi pr\`es qu'on voudra de son \'etat initial''}.
Autrement dit, {\sl  ``pour tout $\varepsilon$, il existe un temps
$T_{\varepsilon}$ au bout duquel le syst\`eme repassera \`a une distance
inf\'erieure \`a $\varepsilon$ de son \'etat initial.'' }
\medskip
L'argument de Zermelo repose sur ce th\'eor\`eme: si on part d'un \'etat de 
faible entropie et que cette entropie cro{\^\i}t comme le dit le second
principe de la Thermodynamique, alors elle devra forc\'ement d\'ecro{\^\i}tre 
\`a nouveau pour revenir au bout d'un temps $T_\varepsilon$ pr\`es de 
sa valeur initiale. On trouvera dans les sections {\bf 5} et {\bf 6} 
de l'annexe quelques textes de Poincar\'e et des extraits de la r\'eponse 
de Boltzmann.
\medskip
L'explication de cet apparent paradoxe est plus simple que pour celui 
de Loschmidt: le th\'eor\`eme de Poincar\'e est juste; mais le temps
$T_{\varepsilon}$ au bout duquel le syst\`eme repassera pr\`es de son 
\'etat initial et o\`u par cons\'equent l'entropie diminuera \`a nouveau 
est de l'ordre de $10^N$ pour un syst\`eme de $N$ particules. Si $N$ est 
le nombre de mol\'ecules d'un litre de gaz $\simeq$ nombre d'Avogadro
$\simeq 10^{23}$, cela veut dire que $T_{\varepsilon}$ serait de l'ordre 
de $10^{100\, 000\, 000\, 000\, 000\, 000\, 000\, 000}$ ann\'ees. La loi de 
croissance de l'entropie n'est vraie que pour des dur\'ees {\it pratiques}
et ne doit pas \^etre extrapol\'ee ainsi \`a des dur\'ees d\'epourvues de 
sens physique. C'est ce que Boltzmann r\'epondit \`a Zermelo (voir annexe, 
section {\bf 6}).

\vskip6mm plus3mm minus3mm
\penalty-600

{ \bf 2. La nature microscopique des gaz.}

\penalty600
\medskip
Imaginons un gaz maintenu dans un r\'ecipient herm\'etique comme un nuage
de poussi\`eres dont les grains sont les mol\'ecules. On va consid\'erer 
un mouvement parfaitement newtonien pour le syst\`eme de point mat\'eriels 
auquel on assimile l'ensemble des mol\'ecules du gaz. Les substitutions 
fr\'equentes des vitesses, chaque fois que la mol\'ecule frappe une 
paroi du r\'ecipient ou entre en collision avec une autre, cr\'ee un 
{\it brouillage} qui, au bout d'un certain temps (apr\`es plusieurs 
collisions) rend la distribution des mol\'ecules en apparence compl\`etement 
al\'eatoire; c'est ce qu'on appelle le {\it chaos d\'eterministe}. 
L'analyse math\'ematique d\'etaill\'ee de ce mouvement de points qui 
entrent mutuellement en collision, incluant le calcul de l'\'evolution 
des positions et des vitesses a \'et\'e effectu\'e pour la premi\`ere 
fois en {\oldstyle 1860},  par J. C. Maxwell [2]. Ce texte de Maxwell
est aujourd'hui encore l'expos\'e le plus clair, le plus rigoureux 
(malgr\'e un raisonnement faux devenu c\'el\`ebre, et corrig\'e six ans 
plus tard), et le plus p\'en\'etrant jamais \'ecrit sur le sujet. 
\medskip
Nous avons d\'ej\`a rencontr\'e cette notion de brouillage en {\bf I.5}, 
{\bf II.5}, {\bf IV.2}. Elle est essentielle pour la r\'esolution du 
paradoxe de Loschmidt.  En effet, le mouvement exact des mol\'ecules,
c'est-\`a-dire leur mouvement newtonien math\'ematique, est 
r\'eversible: en retournant toutes les vitesses (mais en conservant les 
positions), le syst\`eme revient en arri\`ere, en d\'ecrivant le mouvement 
exactement inverse de celui suivi jusque l\`a; de sorte que, si le syst\`eme 
\'etait dans une configuration $X$ \`a l'instant $0$, qu'on le laisse
\'evoluer jusqu'\`a l'instant $T$ o\`u l'on inverse toutes les vitesses, 
il reviendra en parcourant dans l'ordre inverse toutes les positions 
pr\'ec\'edentes, et se retrouvera \`a l'instant $2T$ dans la 
configuration $X$ invers\'ee. Pour exprimer cela de mani\`ere pr\'ecise il 
faut un vocabulaire pr\'ecis. Ce qu'on entend par configuration est en fait 
la {\it configuration en phase}: c'est la donn\'ee des positions 
{\it et des vitesses} de toutes les mol\'ecules. Cette donn\'ee 
caract\'erise l'\'etat {\it m\kern0.45pt{\bigsl i\kern0.6pt}croscopique} 
du syst\`eme; en fait l'\'etat microscopique \`a un instant donn\'e 
{\it est} la configuration en phase. Inverser la {\it configuration en phase}, 
c'est simplement retourner les vitesses (sans modifier les positions). 
Le syst\`eme revient alors en arri\`ere en retrouvant dans l'ordre 
inverse les positions ant\'erieures, mais avec des vitesses oppos\'ees.
\medskip
Dans ces conditions, comment se fait-il que l'on observe
l'irr\'eversibilit\'e?  C'est justement la question pos\'ee par 
Joseph Loschmidt. Si on prend un gaz, initialement (c'est-\`a-dire 
\`a l'instant $0$) comprim\'e dans un vase, il va se r\'epandre tout 
autour et tendre \`a remplir tout l'espace disponible, mais on ne verra 
{\it jamais} un gaz r\'epandu dans une grande pi\`ece se comprimer 
progressivement et venir se concentrer dans un vase en faisant le vide 
alentour. Or, c'est bien ce qui devrait se produire si, une fois le gaz 
uniform\'ement r\'epandu dans la grande pi\`ece, on inversait exactement 
la vitesse de chacune des $N \sim 10^{23}$ mol\'ecules qui le composent. 
Mais il faut inverser {\it exactement} les $N$ vitesses. Si une seule de
ces $N \sim 10^{23}$ vitesses est mal invers\'ee, le mouvement de retour 
commencera effectivement comme l'inverse du mouvement pr\'ec\'edent 
(c'est-\`a-dire que le gaz commencera \`a se recomprimer apr\`es 
l'inversion des vitesses), mais cela ne durera pas: l'unique vitesse 
mal invers\'ee modifiera peu \`a peu (et en fait assez rapidement)
les vitesses des autres mol\'ecules 
\`a cause des innombrables chocs, jusqu'\`a ce que la totalit\'e du 
syst\`eme soit brouill\'ee (par le ph\'enom\`ene du chaos d\'eterministe) 
et ne ressemble plus du tout au mouvement inverse. M\^eme si l'unique 
vitesse mal invers\'ee diff\`ere tr\`es peu de l'inversion exacte, 
cela suffira \`a cr\'eer le chaos au bout d'un temps tr\`es court; 
si la diff\'erence entre la vitesse mal invers\'ee et l'inverse exact 
est $\varepsilon$, et si le nombre de mol\'ecules est $N$, 
ce temps sera de l'ordre de $(1 / \varepsilon )\, 10^{-N}$. 
Il faudrait donc prendre $\varepsilon
\sim 10^{-N}$ pour que ce temps soit de l'ordre de la seconde. Cela 
signifie que l'erreur dans le retournement de la vitesse devrait porter 
sur la $N$i\`eme d\'ecimale. Si $N$ est de l'ordre du nombre d'Avogadro, 
soit $N \sim 10^{23}$, on voit ce que cela signifie!
\medskip
On voit apparara{\^\i}tre ici une des raisons pour lesquelles la parfaite 
r\'eversibilit\'e du mouvement microscopique des mol\'ecules ne se refl\`ete 
pas au niveau des apparences macroscopiques: c'est parce qu'il est 
{\it essentiellement} impossible d'inverser les vitesses avec une telle 
pr\'ecision. Cependant cette raison n'est pas la seule. Une autre est qu'il 
est tout aussi essentiellement impossible d'inverser (m\^eme 
approximativement) les vitesses de toutes les $N$ mol\'ecules; ce serait 
possible s'il n'y avait que cinq ou dix mol\'ecules, mais la difficult\'e 
qui intervient ici cro{\^\i}t exponentiellement avec leur nombre. 
\medskip
Ces deux raisons ont en commun qu'elles ne sont pas li\'ees \`a la nature
physique du gaz, mais aux limites humaines. On pourrait en faire abstraction
pour se concentrer sur l'objet (le gaz) en tant qu'existant ind\'ependamment
de l'homme et de ses limites. Par exemple en tenant un raisonnement comme
celui-ci (d\'ej\`a cit\'e au chapitre {\bf I}):
\medskip
{\cit ``Une intelligence qui, pour un instant donn\'e, conna{\^\i}trait
toutes les forces dont la nature est anim\'ee, et la situation respective 
des \^etres qui la composent, si d'ailleurs elle \'etait assez vaste pour 
soumettre ces donn\'ees \`a l'analyse, embrasserait dans la m\^eme formule 
les mouvements des plus grands corps de l'univers et ceux du plus l\'eger 
atome: rien ne serait incertain pour elle, et l'avenir comme le pass\'e 
serait pr\'esent \`a ses yeux. 
\smallskip
\line{\hfill \vbox{\hbox{Pierre-Simon Laplace}
\hbox{{\sl Essai philosophique sur les probabilit\'es} ({\oldstyle 1819})}} }
\par}

\vskip6mm plus3mm minus3mm

{ \bf 3. Un mod\`ele simple}
\medskip
Pour comprendre exactement le m\'ecanisme probabiliste qui transforme la
r\'eversibilit\'e des mouvements microscopiques en irr\'eversibilit\'e, le 
mieux est comme toujours de commencer par \'etudier un mod\`ele simple, 
qui soit suffisamment analogue aux gaz pour avoir en commun avec ceux-ci 
la propri\'et\'e qui fonde l'irr\'eversibilit\'e. Dans un \'echantillon
macroscopique de gaz, il y a de l'ordre de $10^{23}$ mol\'ecules qui se
d\'eplacent en tous sens selon trois dimensions et entrent sans cesse en
collision mutuelle. C'\'etait du moins le mod\`ele envisag\'e par James 
Maxwell en {\oldstyle 1860}. Contentons nous de deux dimensions et de 
$200$ mol\'ecules qui, au lieu de se heurter entre elles, ne font que 
rebondir sur la paroi du r\'ecipient. Ce r\'ecipient sera (puisque nous 
sommes en deux dimensions) une courbe ferm\'ee ovale. 
\medskip
Un tel mod\`ele, consid\'erablement simplifi\'e, peut-il avoir assez de 
ressemblance avec le gaz de Maxwell pour justifier son emploi? 
\medskip
--- La diff\'erence la plus flagrante semble \^etre la disparit\'e entre le 
nombre d'Avogadro ($\sim 10^{23}$) et $200$. En fait ce sera la moins lourde 
de cons\'equences, car ici les mol\'ecules n'interagissent pas entre elles, 
mais seulement avec la cloison; en sorte que si on en augmente le nombre, 
cela ne changera rien au mouvement des $200$ premi\`eres. On peut donc se
faire une id\'ee juste de l'\'evolution de $N$ particules en extrapolant 
ce qu'on voit advenir pour $200$. 
\medskip
--- Un d\'efaut plus significatif est la disparition des collisions 
entre particules. Nous verrons cependant, en faisant fonctionner notre 
mod\`ele (tout comme nous l'avons vu pour le mod\`ele de roulette au 
chapitre {\bf I} section {\bf 5}), que ce qui produit la 
{\it stochasticit\'e} de l'\'evolution est le ph\'enom\`ene 
du chaos d\'eterministe: des mol\'ecules qui au d\'epart 
sont parfaitement ordonn\'ees (en rang d'oignon) seront au bout d'un 
certain temps {\it en apparence} compl\`etement d\'esordonn\'ees et leur 
disposition semblera al\'eatoire. Or, supprimer les innombrables 
collisions mutuelles ne fait que ralentir ce brouillage par le chaos, 
mais ne le fait pas dispara{\^\i}tre. Au contraire, l'avoir 
ralenti va le rendre plus visible, tout comme un film au ralenti 
rend plus visibles les d\'etails fins d'un mouvement. 
\medskip
Ainsi ce second d\'efaut est en r\'ealit\'e, pour la pr\'esente \'etude, 
une qualit\'e! Il faut cependant garder \`a l'esprit que la disparition 
des collisions n'est
sans cons\'equences {\it que pour la discussion de  l'ir\'eversibilit\'e}: 
notre petit mod\`ele permettra de comprendre la notion d'entropie, mais 
serait tr\`es mauvais, et m\^eme absolument insuffisant, pour illustrer 
le r\^ole des autres grandeurs thermodynamiques telles que la pression, 
la temp\'erature, l'\'energie, la densit\'e. Les relations math\'ematiques 
entre ces grandeurs et l'entropie, qui sont l'essentiel de la 
Thermodynamique puisqu'elles en expriment les lois, disparaissent
compl\`etement dans notre mod\`ele simpliste.
\medskip
--- La seule diff\'erence majeure et essentielle (pour ce qui concerne 
la pr\'esente discussion) entre les vraies mol\'ecules d'un gaz et 
les points mat\'eriels de notre mod\`ele est que le vrai mouvement 
mol\'eculaire est quantique et non classique. Mais ce d\'efaut existait
d\'ej\`a dans les th\'eories de Maxwell et Boltzmann. 
C'est pourquoi nous excluons ici la discussion sur la nature quantique de l'entropie ou 
les fondements quantiques de l'irr\'eversibilit\'e; cela m\`enerait 
beaucoup trop loin! Il s'agira essentiellement de comprendre l'origine 
du paradoxe de Loschmidt, formul\'e en {\oldstyle 1876}, enti\`erement dans 
le cadre de la M\'ecanique classique. Pour cela le petit mod\`ele convient 
parfaitement. D'ailleurs le paradoxe de Loschmidt s'explique par un 
m\'ecanisme stochastique qui intervient de la m\^eme fa\c{c}on dans 
le cas quantique.
\medskip

--- Enfin, les diff\'erences li\'ees \`a la dimension sont sans 
cons\'equences.

\bigskip


Voici alors des simulations num\'eriques. Les positions et vitesses des
mol\'ecules sont repr\'esent\'ees par des symboles tels que:
\medskip


\centerline{\epsfbox{../imgEPS/ch14eps/particul.eps
}}\centerline{\eightrm figure 64}
\medskip


\noindent de sorte que le pied du T pointe vers l'avant, le point
d'intersection des deux segments du T repr\'esentant la position de la
mol\'ecule et le pied du T le vecteur vitesse (ce vecteur reste constant
en norme, les r\'eflexions successives sur le bord du domaine ne modifiant
que la direction du mouvement).
\medskip

Au d\'epart (instant $t=0$) les mol\'ecules sont parfaitement align\'ees
et \'equidistantes, comme ceci (les d\'efauts de r\'egularit\'e qui
apparaissent
sont d\^us aux arrondis):
\medskip
\epsfxsize=\hsize
\centerline{\epsfbox{../imgEPS/ch14eps/ordonne.eps
}}\centerline{\eightrm figure 65}
\medskip

\midinsert
\vbox to \blocksize{\vskip-2mm \eightpoint 
\line{\epsfxsize=\imgsize \epsfbox{../imgEPS/ch14eps/images3/el000,00.eps}
\hfill 
\epsfxsize=\imgsize \epsfbox{../imgEPS/ch14eps/images3/el000,25.eps}}
\line{\hbox to \imgsize{\hfill figure $66\, a$: \  $t=0$.\hfill} \hfill 
\hbox to \imgsize{\hfill figure $66\, b$: \  $t=0.25$.\hfill}}
\vfill\vfill
\line{\epsfxsize=\imgsize \epsfbox{../imgEPS/ch14eps/images3/el001,00.eps}
\hfill 
\epsfxsize=\imgsize \epsfbox{../imgEPS/ch14eps/images3/el003,00.eps}}
\line{\hbox to \imgsize{\hfill figure $66\, c$: \  $t=1$.\hfill} \hfill 
\hbox to \imgsize{\hfill figure $66\, d$: \  $t=3$.\hfill}}
\vfill
\line{\epsfxsize=\imgsize \epsfbox{../imgEPS/ch14eps/images3/el008,00.eps}
\hfill 
\epsfxsize=\imgsize \epsfbox{../imgEPS/ch14eps/images3/el016,00.eps}}
\line{\hbox to \imgsize{\hfill figure $66\, e$: \  $t=8.00$.\hfill} \hfill 
\hbox to \imgsize{\hfill figure $66\, f$: \  $t=16$.\hfill}} 
\vfill
\line{\epsfxsize=\imgsize \epsfbox{../imgEPS/ch14eps/images3/el050,00.eps}
\hfill 
\epsfxsize=\imgsize \epsfbox{../imgEPS/ch14eps/images3/el200,00.eps}}
\line{\hbox to \imgsize{\hfill figure $66\, g$: \  $t=50$.\hfill} \hfill 
\hbox to \imgsize{\hfill figure $66\, h$: \  $t=200$.\hfill}}   }
\endinsert
\noindent mais peu \`a peu les rebondissements successifs sur la
paroi vont avoir pour effet de {\it brouiller} cet ordonnancement
jusqu'\`a aboutir \`a un aspect enti\`erement d\'esordonn\'e. La 
rapidit\'e
de cette \'evolution d\'epend de la forme du r\'ecipient.
Dans les figures $66\, a$ \`a $66\, j$ ci-apr\`es on voit l'\'etat du
mouvement \`a diff\'erents instants de $t=0$ ($66\, a$) \`a $t=2000$ 
($66\, j$),
lorsque le domaine est une ellipse. \medskip

On constate qu'un certain ordre partiel subsiste jusqu'\`a
l'instant $t=50$ ($66\, g$), et dans une moindre mesure \`a l'instant
$t=200$ ($66\, h$). Cette appr\'eciation de ``l'ordre'' reste pour le
moment mal d\'efinie et subjective, mais si cet ``ordre''
subsiste pendant un temps aussi long, c'est parce que l'ellipse est une
courbe tr\`es particuli\`ere: le mouvement d'une particule qui rebondit
sur une ellipse est un {\it mouvement non chaotique} (le syst\`eme
dynamique correspondant \`a ce mouvement est int\'egrable). En attendant
plus longtemps encore ($t \sim 1000$) on a \`a premi\`ere vue une
{\it apparence} de d\'esordonn\'e, mais en regardant de plus pr\`es
on verrait que certaines r\'egions sont moins fr\'equent\'ees que d'autres.
On voit d\'ej\`a \`a l'oeil nu sur la figure $66\, j$ qu'il y a moins de 
points
dans les r\'egions proches des extr\'emit\'es droite et gauche del'ellipse.
C'est que la plupart des trajectoires ne passent pas par ces points; en
effet, selon que le point initial est situ\'e entre les deux foyers
ou \`a l'ext\'erieur des
foyers, les trajectoires seront comme sur les deux figures que voici (fig 67):
\medskip
\line{\epsfxsize=\imgsize \epsfbox{../imgEPS/ch14eps/images3/ellpath2.eps}
\hfill 
\epsfxsize=\imgsize \epsfbox{../imgEPS/ch14eps/images3/ellpath1.eps}}
\centerline{\eightrm figure 67: trajectoires dans une ellipse.}
\medskip
\midinsert
\line{\epsfxsize=\imgsize \epsfbox{../imgEPS/ch14eps/images3/el1000,0.eps}
\hfill 
\epsfxsize=\imgsize \epsfbox{../imgEPS/ch14eps/images3/el2000,0.eps}}
\line{\eightpoint \hbox to \imgsize{\hfill figure $66\, i$: \  $t=1000$.
\hfill} \hfill 
\hbox to \imgsize{\hfill figure $66\, j$: \  $t=2000$.\hfill}}
\vskip4mm
\endinsert
\medskip
Or la majorit\'e des points initiaux repr\'esent\'es
sur la figure $66\, a$ sont entre les deux foyers, en sorte que les
trajectoires correspondantes sont comme sur la figure 67 \`a gauche. 
\medskip
Par cons\'equent, si on avait plac\'e plus \`a gauche la rang\'ee de points
initiaux, en sorte que la majorit\'e soit ext\'erieure aux foyers,
on verrait un vide dans la r\'egion centrale de l'ellipse.
\medskip
Le ph\'enom\`ene illustr\'e sur la figure 67 est semblable
\`a ce qui se produit \`a l'int\'erieur d'un cercle et qui a 
\'et\'e \'etudi\'e en {\bf I.5} (voir la figure 4).
\medskip
Pour avoir une v\'eritable \'evolution vers le
d\'esordre, et une similitude suffisante avec les gaz pour que le mod\`ele
discut\'e soit pertinent, il est essentiel d'avoir un syst\`eme
chaotique, qui garantit un brouillage suffisant. Pour cela, ni le cercle ni
l'ellipse ne conviennent; par contre l'{\it ovale de Cassini}\  convient
tr\`es bien. Si on fait la m\^eme chose avec cette courbe, on voit en effet
que la disparition de l'ordre est bien plus rapide. La diff\'erence est
d'ailleurs flagrante en comparant les trajectoires obtenues pour
l'ellipse (fig 67) \`a la figure 68,
dont le caract\`ere chaotique saute aux yeux. L'\'evolution
vers un d\'esordre apparent est alors bien plus rapide dans l'ovale
de Cassini que dans l'ellipse, comme on le voit en comparant les figures
$69\, a$ \`a $69\, p$ (pages suivantes) avec les figures $66\, a$ \`a 
$66\, j$ 
relatives \`a l'ellipse.
\midinsert
\centerline{\epsfxsize=\hsize
\epsfbox{../imgEPS/ch14eps/images3/casspath.eps}}
\centerline{\eightrm figure 68 : trajectoires dans l'ov{\"\i}de de Cassini.}
\vskip3mm
\endinsert

\midinsert
\vbox to \blocksize{\eightpoint 
\line{\epsfxsize=\imgsize \epsfbox{../imgEPS/ch14eps/images2/cg000,00.eps}
\hfill 
\epsfxsize=\imgsize \epsfbox{../imgEPS/ch14eps/images2/cg000,16.eps}}
\line{\hbox to \imgsize{\hfill figure $69\, a$: \  $t=0$.\hfill} \hfill 
\hbox to \imgsize{\hfill figure $69\, b$: \  $t=0.16$.\hfill}}
\vfill\vfill
\line{\epsfxsize=\imgsize \epsfbox{../imgEPS/ch14eps/images2/cg000,21.eps}
\hfill 
\epsfxsize=\imgsize \epsfbox{../imgEPS/ch14eps/images2/cg000,25.eps}}
\line{\hbox to \imgsize{\hfill figure $69\, c$: \  $t=0.21$.\hfill} \hfill 
\hbox to \imgsize{\hfill figure $69\, d$: \  $t=0.25$.\hfill}}
\vfill
\line{\epsfxsize=\imgsize \epsfbox{../imgEPS/ch14eps/images2/cg000,32.eps}
\hfill 
\epsfxsize=\imgsize \epsfbox{../imgEPS/ch14eps/images2/cg000,50.eps}}
\line{\hbox to \imgsize{\hfill figure $69\, e$: \  $t=0.32$.\hfill} \hfill 
\hbox to \imgsize{\hfill figure $69\, f$: \  $t=0.50$.\hfill}} 
\vfill
\line{\epsfxsize=\imgsize \epsfbox{../imgEPS/ch14eps/images2/cg001,00.eps}
\hfill 
\epsfxsize=\imgsize \epsfbox{../imgEPS/ch14eps/images2/cg002,00.eps}}
\line{\hbox to \imgsize{\hfill figure $69\, g$: \  $t=1$.\hfill} \hfill 
\hbox to \imgsize{\hfill figure $69\, h$: \  $t=2$.\hfill}}   }
\endinsert

\midinsert
\vbox to \blocksize{\eightpoint 
\line{\epsfxsize=\imgsize \epsfbox{../imgEPS/ch14eps/images2/cg003,00.eps}
\hfill 
\epsfxsize=\imgsize \epsfbox{../imgEPS/ch14eps/images2/cg008,00.eps}}
\line{\hbox to \imgsize{\hfill figure $69\, i$: \  $t=3$.\hfill} \hfill 
\hbox to \imgsize{\hfill figure $69\, j$: \  $t=8$.\hfill}}
\vfill\vfill
\line{\epsfxsize=\imgsize \epsfbox{../imgEPS/ch14eps/images2/cg016,00.eps}
\hfill 
\epsfxsize=\imgsize \epsfbox{../imgEPS/ch14eps/images2/cg050,00.eps}}
\line{\hbox to \imgsize{\hfill figure $69\, k$: \  $t=16$.\hfill} \hfill 
\hbox to \imgsize{\hfill figure $69\, l$: \  $t=50$.\hfill}}
\vfill
\line{\epsfxsize=\imgsize \epsfbox{../imgEPS/ch14eps/images2/cg200,00.eps}
\hfill 
\epsfxsize=\imgsize \epsfbox{../imgEPS/ch14eps/images2/cg900,00.eps}}
\line{\hbox to \imgsize{\hfill figure $69\, m$: \  $t=200$.\hfill} \hfill 
\hbox to \imgsize{\hfill figure $69\, n$: \  $t=900$.\hfill}} 
\vfill
\line{\epsfxsize=\imgsize \epsfbox{../imgEPS/ch14eps/images2/cg2000,0.eps}
\hfill 
\epsfxsize=\imgsize \epsfbox{../imgEPS/ch14eps/images2/cg2012,0.eps}}
\line{\hbox to \imgsize{\hfill figure $69\, o$: \  $t=2000$.\hfill} \hfill 
\hbox to \imgsize{\hfill figure $69\, p$: \  $t=2012$.\hfill}}   }
\endinsert

Cette fois, gr\^ace \`a la chaoticit\'e du mouvement, on arrive d\`es
$\, t=50$ \`a une distribution apparemment faite au hasard. On peut s'en
rendre compte visuellement en comparant avec une troisi\`eme s\'erie de
figures ($70\, a$, $70\, b$, $70\, c$, et $70\, d$),
qui ont \'et\'e fabriqu\'ees \`a l'aide d'une fonction {\bf random}.

\midinsert
{\eightpoint 
\line{\epsfxsize=\imgsize \epsfbox{../imgEPS/ch14eps/images2/random1.eps}
\hfill 
\epsfxsize=\imgsize \epsfbox{../imgEPS/ch14eps/images2/random2.eps}}
\line{\hbox to \imgsize{\hfill figure $70\, a$\hfill} \hfill 
\hbox to \imgsize{\hfill figure $70\, b$\hfill}}
\vfill\medskip
\line{\epsfxsize=\imgsize \epsfbox{../imgEPS/ch14eps/images2/random3.eps}
\hfill 
\epsfxsize=\imgsize \epsfbox{../imgEPS/ch14eps/images2/random4.eps}}
\line{\hbox to \imgsize{\hfill figure $70\, c$\hfill} \hfill 
\hbox to \imgsize{\hfill figure $70\, d$\hfill}}
\vskip5mm
\centerline{\vbox{\hsize=11cm 
\centerline{figure 70: configuration ``au hasard''.}
\medskip
La position du point \'etant rep\'er\'ee par deux coordonn\'ees
$x,y$, et la direction par un angle $\theta$, on a simplement pris
$x = {\bf random}$, $y = {\bf random}$, $\theta = 2 \pi \cdot\, {\bf random}$
(la fonction {\bf random} retournant un flottant compris entre $0$ et $1$)
et on n'a retenu que les points situ\'es \`a l'int\'erieur de l'ovale.} } }
\vskip4mm
\endinsert
\medskip
Comme on sait, les fonctions {\bf random} ne produisent pas un hasard plus
authentique que celui des figures $69$, puisqu'elles exploitent elles
aussi la chaoticit\'e et sont en r\'ealit\'e absolument d\'eterministes.
\medskip
Toutefois le point crucial pour la discussion qui va suivre n'est pas 
la similitude de nature entre les fonctions {\bf random} et les trajectoires 
chaotiques; mais que  l'aspect d\'esordonn\'e de la distribution des 
mol\'ecules est purement subjectif: la ressemblance entre la distribution 
des figures $69\, m$, $69\, n$, $69\, o$, ou $69\, p$, et celles des 
figures $70\, a$, $70\, b$, $70\, c$, $70\, d$ n'est qu'une apparence 
due \`a la non perception du fait pourtant essentiel que voici: si dans la 
figure $69\, m$ on retourne chacune des $200$ mol\'ecules, c'est-\`a-dire si 
on inverse la vitesse sans toucher \`a la position, on obtient une 
configuration qui aura l'extraordinaire propri\'et\'e qu'apr\`es un temps 
$t=200$, elle aboutit \`a un parfait ordonnancement (le rang d'oignon 
initial, avec vitesses pointant vers le bas), ce qui ne sera certainement 
pas le cas pour les configurations $70\, a$, $70\, b$, $70\, c$, $70\, d$. 
Autrement dit, la configuration $69\, m$ n'est absolument pas 
une distribution ``au hasard'', c'est au contraire une distribution
tout \`a fait exceptionnelle. On s'en doutait, puisqu'elle est 
l'effet d'un mouvement rigoureusement d\'eterministe. C'est ce caract\`ere
exceptionnel des configurations comme $69\, m$ (ou sa conjugu\'ee obtenue en
inversant les vitesses) qui est la cl\'e du paradoxe de Loschmidt; mais 
pour en comprendre toutes les implications, il est indispensable de fixer 
un langage ad\'equat. Dans la section qui suit, on va donner un sens 
pr\'ecis aux notions d'{\it \'etat microscopique}, d'{\it \'etat 
macroscopique},  et d'{\it entropie}, qui jusqu'ici n'avaient \'et\'e
qu'\'evoqu\'ees.

\vskip4mm plus4mm minus3mm
\penalty-600

{ \bf 4. Discussion du mod\`ele.}

\penalty600
\medskip
On appellera {\it \'etat m\kern0.45pt{\bigsl i\kern0.6pt}croscopique} 
la donn\'ee compl\`ete des positions et des vitesses de chacune des $N$ 
mol\'ecules; c'est la {\it configuration en phase} (les deux termes 
sont \'equivalents).  Dans le petit 
mod\`ele $N$ est \'egal \`a $200$, et il y a trois coordonn\'ees pour 
chaque mol\'ecule ($x$, $y$, et $\theta$); par cons\'equent un \'etat 
microscopique est la donn\'ee des $200 \times 3 = 600$ coordonn\'ees 
correspondantes [En langage math\'ematique: c'est un \'el\'ement
de $(U \times [0, 2\pi [)^{200}$, $U$ \'etant le domaine bidimensionnel 
d\'elimit\'e par la courbe. Dans un vrai gaz $N \sim 10^{23}$,  et il
y a six coordonn\'ees pour chacune (trois coordonn\'ees de position dans 
l'espace, et les trois composantes du vecteur vitesse); un \'etat 
microscopique est alors la donn\'ee des $10^{23} \times 6$ coordonn\'ees 
correspondantes. Pour pouvoir d\'efinir l'entropie, il 
est cependant n\'ecessaire de {\it discr\'etiser} ces donn\'ees, ce 
que nous allons faire dans le cadre du petit mod\`ele. L'\'etendre au 
cas d'un vrai gaz ne changera conceptuellement rien. 
\medskip 
Une remarque au passage:  la discr\'etisation est consubstantielle 
\`a la notion d'entropie; cette derni\`ere ne peut prendre aucun sens 
sur la base d'un continuum d'\'etats.  Dans la r\'ealit\'e c'est 
la nature quantique des \'etats microscopiques 
qui veut qu'ils soient naturellement discrets, sans qu'il soit 
n\'ecessaire d'introduire une discr\'et{\it isation} artificielle. 
Le pas d'une telle discr\'etisation artificielle est arbitraire 
(changer sa valeur ne fait qu'ajouter une constante conventionnelle 
\`a l'entropie); il faut seulement qu'il soit petit (mais pas trop).
Dans le cas quantique par contre, il n'y a pas de
telle convention: le nombre d'\'etats microscopiques est ce 
qu'il est et l'entropie est alors une grandeur absolue et non
une grandeur d\'efinie \`a une constante additive pr\`es.
\medskip 
Pour discr\'etiser le petit mod\`ele on supposera le domaine int\'erieur
\`a la courbe (le r\'ecipient) d\'ecoup\'e en petites cellules carr\'ees,
en sorte que les coordonn\'ees $x,y$ sont arrondies \`a la cellule la 
plus proche. Les figures ayant \'et\'e r\'ealis\'ees num\'eriquement 
il est naturel de consid\'erer les pixels comme \'etant les cellules.
L'\'echelle de ces figures est de $880$ pixels par unit\'e de longueur,
ce qui fait que l'aire d\'elimit\'ee par la courbe ovo{\"\i}de contient
exactement $177\, 883$ pixels. Alors le nombre de valeurs possibles pour
$(x,y)$ sera $177\, 883$. Il faut aussi discr\'etiser l'angle. On supposera 
que celui-ci est simplement arrondi au {\it degr\'e}; cette discr\'etisation 
qui d\'ecoupe l'espace en pixels et l'angle en degr\'es, revient \`a
d\'ecouper l'espace des phases en pixels $\times$ degr\'es, ce qu'on appelle
des {\it cellules de phase}. Il y a en tout $Z = 360 \times 177\, 883 = 
64\, 037\, 880$ cellules de phase, en sorte que le nombre de possibilit\'es 
de distribuer les $200$ mol\'ecules sera $Z^{200} \simeq  1.94 \cdot
10^{1561}$. Autrement dit, il y a en tout environ $1.94 \cdot 10^{1561}$
\'etats microscopiques possibles.
\medskip
Un \'etat  {\it \'etat m{\bigsl a\kern0.3pt}croscopique}, 
comme son nom l'indique, n'est d\'efini que par des param\`etres 
{\it m{\bigsl a\kern0.3pt}croscopiques}, donc ind\'ependants
de la connaissance d\'etaill\'ee des positions et vitesses des 
mol\'ecules. Par exemple l'information suivante:
$$\vcenter{\hsize=90mm ``les mol\'ecules se r\'epartissent uniform\'ement 
dans le r\'ecipient.''} \eqno (1)$$
caract\'erise un \'etat macroscopique; de m\^eme
$$\vcenter{\hsize=90mm ``les mol\'ecules sont concentr\'ees uniform\'ement
sur le segment $AB$ de l'axe horizontal, avec la m\^eme vitesse.''} 
\eqno (2)$$
\vskip2mm plus4mm minus2mm
\penalty-800

Plus g\'en\'eralement, \'etant donn\'ee une r\'egion ${\cal R}$ du 
r\'ecipient, 
$$\vcenter{\hsize=90mm ``les mol\'ecules se r\'epartissent uniform\'ement 
dans la r\'egion ${\cal R}$, aucune ne se trouvant en dehors de cette 
r\'egion.''} \eqno (3)$$
ou encore, \'etant donn\'e un segment de courbe ${\cal S}$ contenu dans 
le r\'ecipient:
$$\vcenter{\hsize=90mm ``les mol\'ecules se r\'epartissent uniform\'ement 
le long de ${\cal S}$, avec leurs vitesses dirig\'ees selon la normale 
au segment.''} \eqno (4)$$

Ces sp\'ecifications ne prennent pas en compte les coordonn\'ees et 
vitesses {\it individuelles} de chaque mol\'ecule: elles ne d\'ecrivent 
que des situations collectives. C'est cela qu'on appelle un {\it \'etat 
m{\bigsl a\kern0.3pt}croscopique}. Le point crucial de cette d\'efinition 
est la notion de param\`etre collectif.  Par exemple la sp\'ecification (4) 
ne fait appel qu'aux para\-m\`etres qui d\'ecrivent le segment de courbe. 
Si c'est un segment de parabole, on donnera son \'equation 
$y=ax^2 + bx + c$, et les deux abscisses extr\^emes du segment, $x_0$ 
et $x_1$, en sorte que le segment sera l'ensemble des points $x,y$ 
tels que $y=ax^2 + bx + c$ et $x_0 < x < x_1$. Les cinq param\`etres 
$a,b,c,x_0, x_1$ sont alors les param\`etres {\it macroscopiques} qui
d\'efinissent l'{\it \'etat macroscopique}. Ces param\`etres ne
contiennent d'information sur les coordonn\'ees ou la vitesse d'aucune 
mol\'ecule particuli\`ere. Le fait que la vitesse est le vecteur normal 
\`a la courbe ne donne pas de renseignement sur les vitesses individuelles 
des mol\'ecules car on ne sait pas laquelle des mol\'ecules est \`a la 
base d'une normale particuli\`ere. Toutefois il peut arriver que la 
sp\'ecification macroscopique fournisse ce renseignement: par exemple 
la position initiale des mol\'ecules dans les 
figures $66\, a$ ou $69\, a$ correspond \`a un \'etat macroscopique, 
caract\'eris\'e par une sp\'ecification du type $(2)$ ci-dessus. Il s'agit 
d'un \'etat macroscopique tr\`es particulier, car toutes les mol\'ecules ont 
exactement le m\^eme vecteur vitesse. Celui-ci devient alors un param\`etre 
macroscopique dont la connaissance fournit \'evidemment aussi la vitesse de 
chaque mol\'ecule individuelle. Mais pas sa position. De la m\^eme fa\c{c}on, 
si toutes les mol\'ecules \'etaient au m\^eme endroit, les coordonn\'ees de 
cet endroit deviendraient des param\`etres macroscopiques, mais pas les 
vitesses (\`a moins qu'elles ne soient toutes \'egales, elles aussi).
\medskip
Un m\^eme \'etat macroscopique peut en g\'en\'eral \^etre r\'ealis\'e par 
un grand nombre de configurations en phase, c'est-\`a-dire d'\'etats
microscopiques 
diff\'erents. On le comprend rien qu'en remarquant qu'une permutation 
entre les mol\'ecules ne change pas l'\'etat macroscopique: par exemple si on 
prend la sp\'ecification $(3)$ ci-dessus, on voit qu'elle demeure respect\'ee 
si on ne fait qu'\'echanger les mol\'ecules entre elles. Par contre l'\'etat 
microscopique n'est \'evidemment plus le m\^eme apr\`es permutation. 
\medskip
L'{\it entropie} d'un \'etat macroscopique ${\cal E}$ du syst\`eme est alors 
d\'efinie comme \'etant le logarithme du nombre d'\'etats microscopiques 
qui r\'ealisent l'\'etat macroscopique ${\cal E}$. Voyons cela sur les 
exemples $(1)$,  $(2)$,  $(3)$,  $(4)$. 
\medskip
$(1)$. Dans l'\'etat $(1)$, les mol\'ecules se r\'epartissent uniform\'ement 
sur l'ensemble du domaine ovo{\"\i}de. Cela veut dire que la distance entre 
deux mol\'ecules voisines ne peut pas \^etre trop petite, elle doit \^etre 
du m\^eme ordre que la maille d'un quadrillage r\'egulier de 200 mailles 
recouvrant l'ovo{\"\i}de. Si $Q$ est l'aire d'une telle maille, 
compt\'ee en prenant pour unit\'e la taille d'un pixel, et $G$ le nombre
de pixels contenus dans l'ovo{\"\i}de (ici $G = 177\, 883$), on doit avoir
$Q = G / 200 \simeq 889.415$.
N'importe quelle r\'epartition \`a peu pr\`es uniforme des mol\'ecules sur 
la surface de l'ovo{\"\i}de doit avoir la propri\'et\'e que chacune de 
ces mol\'ecules est seule dans une maille dont l'aire est \`a peu pr\`es 
$889$ pixels. Tout se passe donc en gros comme si on devait placer les 
$200$ mol\'ecules dans deux cents mailles 
ayant chacune une aire \'egale \`a $889$ ou $890$ pixels, de 
mani\`ere \`a n'avoir qu'une seule mol\'ecule par maille. Le nombre de 
possibilit\'es est $Q^{200} \times 200! = G^{200} \times 200! / 
200^{200} \simeq 5.22 \cdot 10^{964}\,$.  Comme
les vitesses sont indiff\'erentes puisque seule la place des mol\'ecules
est prise en compte, et qu'il y a $360$ vitesses en tout pour {\it chaque}
mol\'ecule, on voit que le nombre d'\'etats microscopiques est
$$\nu_1 = \Big[360 \times Q \Big]^{200} \times 200! \simeq 
0.95 \times 10^{1476}$$
dont le logarithme (n\'ep\'erien) est environ $3398.56$.
\medskip
$(2)$. Pour simplifier supposons que la longueur du segment soit 200 
pixels. Le nombre de possibilit\'es de ranger les 200
mol\'ecules est alors $200!\,$. La vitesse est impos\'ee pour chaque 
mol\'ecule, il n'y a donc pas d'autre possibilit\'e, en sorte que 
$$\nu_2 = 200!$$
dont l'entropie est  environ $863.23$. Le calcul de l'entropie pour le 
cas o\`u la longueur du segment n'est pas 200 est laiss\'e en exercice
(indication: c'est un probl\`eme de d\'enombrement du type trait\'e 
au chapitre {\bf II}, section {\bf 4}). 
\medskip
$(3)$. On peut appliquer le raisonnement de $(1)$ en rempla\c{c}ant 
l'ovo{\"\i}de par la r\'egion ${\cal R}$; si $G_{\cal R}$ est le   
nombre de pixels couverts par cette r\'egion on aura
$$\nu_3 = \Big[360 \times G_{\cal R} / 200 \Big]^{200} \times 200!$$
En d\'esignant par $\sigma$ le rapport $G_{\cal R} / G$ (qui est aussi 
l'aire de ${\cal R}$ en proportion de l'aire totale de l'ovo{\"\i}de)
on voit que $\nu_3 = \sigma^{200} \times \nu_1$ et l'entropie 
sera $3398.56 + 200\,\ln\sigma$ ($\sigma < 1$ donc $\ln\sigma < 0$).
\medskip
$(4)$. La r\'eponse est la m\^eme que pour $(2)$: le fait d'avoir une 
courbe au lieu d'une droite ne change combinatoirement rien.
\medskip
On voit que l'entropie est bien la ``mesure du d\'esordre'' comme on le 
dit dans la science populaire: les structures tr\`es ordonn\'ees telles 
que $(2)$ ou $(4)$ ont une entropie bien plus petite que $(1)$, tandis que 
$(3)$ qui est un peu ordonn\'e a une entropie un peu plus petite que 
$(1)$. La structure la plus ordonn\'ee possible est celle o\`u toutes les 
mol\'ecules sont au m\^eme endroit avec la m\^eme vitesse; comme on l'a 
signal\'e plus haut, cet endroit et cette vitesse uniques constituent les 
param\`etres macroscopiques d'un tel \'etat, et d\'eterminent aussi les 
positions et vitesses individuelles de chaque mol\'ecule. Il n'y a que 
dans ce cas extr\^eme que les param\`etres macroscopiques d\'eterminent 
les positions et vitesses individuelles de {\it chaque} mol\'ecule. Cela
entra{\^\i}ne qu'il n'y a qu'un seul \'etat microscopique sous-jacent 
possible, et par cons\'equent l'entropie de cet \'etat macroscopique 
extr\^eme est $0$. Cela correspond \`a l'ordre maximal: on ne peut pas 
imaginer un \'etat macroscopique plus ordonn\'e que celui o\`u la position 
et la vitesse de chaque mol\'ecule individuelle sont fix\'ees. 
\medskip 
C'est la notion d'\'etat macroscopique, caract\'eris\'e par des param\`etres 
macroscopiques, qui est la cl\'e du paradoxe de Loschmidt. Ce paradoxe 
cesse de para{\^\i}tre paradoxal d\`es lors qu'on a compris ce qu'est 
un \'etat macroscopique. Et pour bien comprendre ce qu'est un \'etat 
macroscopique, le mieux est peut-\^etre de voir ce qui {\it n'est pas}
un \'etat macroscopique. Je propose donc d'examiner la r\'eversibilit\'e 
microscopique en montrant qu'elle fait intervenir des \'etats qui {\it ne 
sont pas} des \'etats macroscopiques. 
\medskip
Consid\'erons un syst\`eme de mol\'ecules qui \'evolue d'un \'etat 
${\cal B}$ (relativement ordonn\'e) vers un \'etat ${\cal A}$ d'entropie 
maximale. L'argument de Loschmidt consistait \`a dire que, si on inverse 
le sens du temps, le sys\-t\`eme va revenir \`a l'\'etat ${\cal B}$, donc 
pour chaque \'evolution qui fait cro{\^\i}tre l'entropie, il y en a une 
autre, sym\'etrique, qui la fait d\'ecro{\^\i}tre. Il y a donc ``autant'' 
d'\'evolutions microscopiques qui font cro{\^\i}tre l'entropie que 
d'\'evolutions microscopiques qui la font d\'ecro{\^\i}tre et par 
cons\'equent on ne comprend plus pourquoi le premier cas est plus 
probable que le second (et encore moins pourquoi c'est {\it toujours} 
le premier cas qui se produit dans la nature).
\medskip
Reprenons cet argument de Loschmidt dans le cadre de notre petit mod\`ele. 
Supposons qu'\`a l'instant $0$ on dispose les mol\'ecules dans  l'\'etat 
macroscopique de la figure $69\, a$, en rang d'oignon sur l'axe horizontal,
les vitesses, toutes identiques, point\'ees vers le haut. Il s'agit d'un 
\'etat macroscopique du type $(2)$, que nous pouvons appeler l'\'etat 
${\cal B}$. Lorsque le sys\-t\`eme \'evolue, il se trouvera \`a l'instant 
$t=200$ dans la configuration de la fi\-gu\-re $69\, m$. Il s'agit d'une
configuration en phase puisque les vitesses sont marqu\'ees, donc 
cette figure repr\'esente 
un \'etat microscopique: chaque mol\'ecule est repr\'esent\'ee par 
un $T$ qui indique sa position et son vecteur vitesse (il manque 
la num\'erotation des mol\'ecules, mais leur trajectoire per\-met\-trait 
de les identifier). L'\'etat macroscopique qui correspond \`a la 
con\-fi\-gu\-ra\-tion de la figure $69\, m$ est un \'etat du type $(1)$ (les 
mol\'ecules sont r\'e\-par\-ties au hasard, \`a peu pr\`es uniform\'ement), 
qui a l'entropie maximale, et qu'on ap\-pel\-lera donc l'\'etat ${\cal A}$.
Aux instants ult\'erieurs ($t > 200$) l'\'etat macroscopique reste le 
m\^eme (figures $69\, n$, $69\, o$, et $69\, p$), mais \'evidemment 
l'\'etat microscopique change \`a chaque instant. On 
a vu qu'il y avait $\nu_1 \simeq 10^{1476}$ \'etats
microscopiques diff\'erents qui donnent tous la m\^eme apparence 
macroscopique, celle de l'\'etat ${\cal A}$. Lorsque le syst\`eme continue 
d'\'evoluer apr\`es l'instant $t=200$, il passe successivement par d'autres 
\'etats microscopiques qui font tous partie de cet ensemble de $\nu_1$ 
\'etats microscopiques. Ce nombre $\nu_1$ est \'enorme: au bout d'un 
temps $t$ de l'ordre de un milliard, le syst\`eme n'aura parcouru que 
quelques centaines de milliards d'\'etats microscopiques, sur $\nu_1$ en 
tout. Pour avoir une chance qu'il ait parcouru une proportion notable 
de ces $\nu_1$ \'etats microscopiques, il faudrait 
attendre que $t \sim 10^{1476}$. Et ici on consid\`ere un syst\`eme
de $200$ mol\'ecules seulement; avec $10^{23}$ ($\sim$ nombre d'Avogadro) 
mol\'ecules, le nombre $\nu_1$ serait de l'ordre de 
$10^{756\, 916\, 605\, 020\, 625\, 000\, 000\, 000}$, et il faudrait 
donc attendre un temps du m\^eme ordre que ce nombre.
\medskip
N'importe quel \'etat macroscopique plus ordonn\'e que ${\cal A}$ a une 
entropie plus faible, mais l'entropie n'\'etant que le logarithme du 
nombre d'\'etats microscopiques sous-jacents, le nombre de ces \'etats 
sera \'enorm\'ement plus faible. On a vu que l'entropie de ${\cal A}$ est 
environ $3398$; un \'etat ${\cal A}'$ d'entropie $3300$ aura $e^{98}
\simeq 10^{43}$ fois moins d'\'etats microscopiques sous-jacents.
On comprend donc pourquoi il est extr\^emement peu probable que le syst\`eme
tombe pr\'ecis\'ement sur un de ces tr\`es rares \'etats macroscopiques 
de plus faible entropie. C'est bien ce qu'on observe: si on laisse le 
programme qui simule l'\'evolution repr\'esent\'ee sur les figures $69\, a$ 
\`a $69\, p$ tourner, on devra attendre vraiment tr\`es longtemps avant de 
voir appara{\^\i}tre ``spontan\'ement'' un \'etat macroscopique de plus 
faible entropie.  Le temps qu'il faudra attendre sera \'evidemment 
inconcevablement plus long pour $10^{23}$ mol\'ecules que pour $200$. 
Si on d\'esigne par $\delta$ la dur\'ee moyenne d'un \'etat microscopique, 
c'est-\`a-dire le temps pendant lequel le syst\`eme reste dans une cellule 
de phase ($\delta$ est de l'ordre de $10^{-12}$ secondes pour un gaz, de
$10^{-2}$ pour le petit mod\`ele), alors on peut dire que le temps
moyen au bout duquel se produira spontan\'ement une baisse d'entropie de
$\Delta S$ est en gros $T = \delta \times \exp\{ \Delta S \}$. Apr\`es un
temps petit devant $T$ une diminution de $\Delta S$ est pratiquement
impossible, mais au bout d'un temps qui serait grand devant $T$ (ne serait-ce
que $100 \times T$), une diminution d'entropie de $\Delta S$ devient par 
contre presque certaine. Il s'agit \'evidemment du m\^eme ph\'enom\`ene que
celui qui a \'et\'e \'etudi\'e \`a la section {\bf V\kern-2pt .\kern2pt 4}. 
Le paradoxe de Loschmidt n'existerait pas si l'esp\'erance de vie des humains 
\'etait de l'ordre de $10^{756\, 916\, 605\, 020\, 625\, 000\, 000\, 000}$
ann\'ees. Mais la loi de croissance de l'entropie serait alors fausse.
\medskip 
En revenant \`a l'inversion du temps, on arrive \`a la 
conclusion suivante: si on retourne en arri\`ere, c'est-\`a-dire si 
\`a l'instant $t=200$ on inverse brusquement toutes les vitesses dans 
la configuration de la figure $69\, m$, le syst\`eme reviendra effectivement 
\`a la configuration de la figure $69\, a$ (avec vitesses invers\'ees) \`a 
$t = 400$. Mais ce que dit le second principe de la Thermodynamique, c'est 
justement qu'on ne peut pas effectuer une telle inversion {\it dans les 
syst\`emes physiques}. Aucun 
physicien ne peut en laboratoire produire une telle inversion. 
Pour r\'eussir une telle exp\'erience, il faudrait, apr\`es avoir laiss\'e 
le syst\`eme \'evoluer pendant un certain temps, retourner les vitesses 
de chaque mol\'ecule individuelle {\it sans rien perturber d'autre}. Il 
faudrait aussi que ce retournement soit extr\^emement pr\'ecis: la moindre 
erreur ferait qu'au retour on retrouverait une configuration enti\`erement 
diff\'erente de celle qui est attendue, c'est-\`a-dire la configuration 
initiale invers\'ee. Il n'existe aucun moyen 
physique d'effectuer une telle inversion. En r\'ealit\'e, ce que dit
le second principe de la Thermodynamique n'est pas tant que l'entropie 
est toujours croissante (ceci en est une formulation math\'ematique et 
abstraite); le sens concret de ce principe est:
\smallskip
{\sl Il est impossible d'effectuer, sur un syst\`eme mat\'eriel, une
inversion exacte des vitesses individuelles des mol\'ecules. \par}
\medskip
Essayons d'imaginer concr\`etement ce que serait une exp\'erience
de ce type, c'est-\`a-dire une exp\'erience mettant en d\'efaut le second 
principe. On voudrait pr\'eparer un syst\`eme (par exemple un gaz) de 
mani\`ere que:
\smallskip
a) L'\'etat macroscopique imm\'ediatement apr\`es la pr\'eparation est 
d'en\-tro\-pie maximale;
\smallskip
b) Si apr\`es la pr\'eparation on laisse \'evoluer le syst\`eme \`a l'abri 
de toute influence ext\'erieure, il \'evoluera vers un \'etat macroscopique 
de moindre entropie. 
\medskip
Pour mieux comprendre revenons \`a notre petit mod\`ele; on voudrait 
qu'au d\'epart la configuration soit analogue \`a celle  de la figure 
$69\, m$, et qu'apr\`es \'evolution elle devienne analogue \`a celle  de la 
figure $69\, a$. Pour que cela ait une chance de marcher il faut 
\'evidemment prendre les configurations avec les vitesses invers\'ees.
On voit ais\'ement sur le petit mod\`ele que la pr\'eparation doit 
obligatoirement consister \`a r\'ealiser la configuration de la figure 
$69\, m$ (avec vitesses invers\'ees). Si on se trompe dans cette 
pr\'eparation, on ne retrouvera pas la configuration de la figure $69\, a$ 
mais une autre, qui avec une quasi certitude sera d'entropie maximale. 
Pour retrouver la configuration de $69\, a$ (avec vitesses invers\'ees) 
il faudra viser extr\^emement juste.  
\smallskip
Une petite remarque en passant: dans ce petit mod\`ele, 
si on se trompe sur une seule des $200$ vitesses, les $199$ autres \'etant 
correctes, il n'y aura qu'une seule mol\'ecule qui ne reviendra pas sur 
sa position de la figure $69\, a$. Mais cela est uniquement d\^u au fait 
que ce mod\`ele trop simpliste ne comporte pas de collision entre mol\'ecules:
l'\'evolution de chaque mol\'ecule \'etant ind\'ependante des autres, une 
erreur sur une seule mol\'ecule n'aura aucune cons\'equence sur les $199$ 
autres. Dans un vrai gaz, les mol\'ecules entrent en collision des 
milliards de milliards de fois par seconde. Alors la moindre erreur sur une 
mol\'ecule se r\'epercute rapidement sur l'\'evolution de toutes les autres.
D'autre part le petit mod\`ele est num\'erique et non physique: il est
possible de cr\'eer {\it num\'eriquement} un \'etat initial qui 
\'evoluera en faisant diminuer l'entropie, il suffit pour cela de calculer 
d'abord l'\'evolution de $69\, a$ vers $69\, m$, avec les valeurs des 
positions et des vitesses dans un tableau, puis d'effectuer num\'eriquement 
l'inversion. Le second principe dit qu'il n'existe pas de proc\'edure 
analogue pour un syst\`eme physique.
\medskip
Et voil\`a la solution du paradoxe de Loschmidt. Le nombre d'\'etats 
microscopiques susceptibles de donner l'apparence macroscopique de la figure 
$69\, m$ est $\nu_1 \sim 10^{1476}$.
Mais parmi ces $\nu_1$ \'etats microscopiques, il n'y en a que 
$\nu_2 = 200! \sim 10^{375}$ qui correspondent \`a la configuration 
pr\'ecise de la figure $69\, m$, celle qui, apr\`es retournement, est en 
mesure de revenir \`a la configuration en rang d'oignons. Cela 
r\'esulte de la sym\'etrie par retournement du temps.
\medskip
En effet, la sym\'etrie par retournement du temps des \'equations de la 
M\'ecanique, qui est invoqu\'ee dans l'argument de Loschmidt, se traduit 
ainsi: \`a chaque \'etat microscopique $C_0$, correspond un {\it unique} 
\'etat microscopique $C_t$ qui est le r\'esultat de l'\'evolution de $C_0$ 
apr\`es un temps $t$, et vice-versa. En termes math\'ematiques:  
\medskip
{\bf Th\'eor\`eme:} {\sl la transformation 
$$f\; : \quad C_0 \mapsto C_t$$
est une bijection.} 
\medskip
{\bf D\'emonstration:} Il suffit de montrer que $f$ poss\`ede un inverse 
$f^{-1}$. Or, si on appelle $\overline{C}$ 
l'\'etat obtenu en inversant toutes les vitesses de l'\'etat $C$, la 
sym\'etrie par retournement du temps peut s'\'enoncer sous la forme suivante: 
\smallskip
{\sl
``Si $C_t$ est le r\'esultat de l'\'evolution m\'ecanique de $C_0$ apr\`es 
le temps $t$, alors $\overline{C_0}$ est le r\'esultat de l'\'evolution 
m\'ecanique de $\overline{C_t}$ apr\`es le temps $t$.''}
\medskip
Traduit en langage math\'ematique cela se dit, en d\'esignant par $Q$ la
transformation $C \mapsto \overline{C}$:
$$f^{-1} = Q \circ f \circ Q$$
c'est-\`a-dire que $f$ a un inverse, qui est la composition 
$$f^{-1} : C_t \mapsto \overline{C_t} \mapsto \overline{C_0} \mapsto C_0$$

\line{\hfill C.Q.F.D. \hskip7mm}

\bigskip

C'est l\`a la seule traduction logique de la sym\'etrie par retournement 
du temps. Le fait que la transformation $f$ soit inversible, donc 
bijective, entra{\^\i}ne \'evidemment que 
\smallskip
{\sl
\noindent le nombre d'\'etats microscopiques distincts r\'esultant de 
l'\'evolution pendant un temps donn\'e d'un ensemble 
de $n$ \'etats microscopiques initiaux, est exactement \'egal \`a $n$.\par }
\medskip 
\noindent ou, de mani\`ere \'equivalente en passant aux conjugu\'es:
\smallskip
{\sl
\noindent le nombre d'\'etats microscopiques distincts qui, apr\`es 
\'evolution r\'etrograde pendant un temps donn\'e, redonneront un ensemble 
de $n$ \'etats microscopiques initiaux, est exactement \'egal \`a $n$. \par }
\medskip
C'est donc la sym\'etrie par retournement du temps qui permet de dire que
\smallskip
{\sl \noindent
parmi les $\nu_1 \sim 10^{1476}$ \'etats microscopiques susceptibles de
donner l'apparence macroscopique de la figure $69\, m$, il y en a exactement 
$\nu_2 = 200! \sim 10^{375}$ qui redonneront par retour la configuration 
en rang d'oignons initiale.''\par }
\medskip  
R\'esumons tout cela sous la forme d'un argument g\'en\'eral, non li\'e 
au petit mod\`ele: 
si on veut pr\'eparer un syst\`eme qui au d\'epart se trouve dans un 
\'etat macroscopique ${\cal A}$ d'entropie maximale $S_{\cal A}$ afin 
qu'il \'evolue vers un \'etat macroscopique ${\cal B}$ d'entropie 
$S_{\cal B} < S_{\cal A}$, il faut le pr\'eparer dans un \'etat 
microscopique extr\^emement particulier: parmi les $\nu_{\cal A} = 
\exp\{ S_{\cal A} \}$ \'etats microscopiques susceptibles de donner 
l'apparence macroscopique ${\cal A}$, il faut en r\'ealiser un parmi 
ceux qui \'evolueront vers ${\cal B}$; d'apr\`es ce qu'on vient de voir, 
le nombre de ces \'etats microscopiques est $\nu_{\cal B} = 
\exp\{ S_{\cal B} \}$, puisqu'ils sont en bijection avec les \'etats 
microscopiques qui produisent l'apparence ${\cal B}$.\looseness=-1
\medskip
Il en r\'esulte que la proportion de ces \'etats microscopiques 
exceptionnels parmi tous ceux qui produisent l'apparence macroscopique 
${\cal A}$ est \'egale \`a $\nu_2 / \nu_1 = \exp\{ S_{\cal B} - 
S_{\cal A}\}$. Dans le cas du petit mod\`ele et de l'\'evolution entre 
les figures $69\, a$ et $69\, m$, cette proportion 
est  $\nu_2 / \nu_1 \simeq 10^{-590}$
\medskip
C'est le probl\`eme de cette pr\'eparation qui est omis dans 
le raisonnement de Loschmidt. Le second principe de la Thermodynamique 
ne dit pas qu'{\it aucun} des \'etats microscopiques sous-jacents \`a un 
\'etat macroscopique ${\cal A}$ ne peut conduire \`a un \'etat macroscopique 
${\cal B}$ de moindre entropie; il dit que:
\smallskip
a) de tels \'etats microscopiques sont extr\^emement peu nombreux (au 
regard de tous ceux qui fournissent l'apparence macroscopique de ${\cal A}$);
\smallskip
b) il est impossible d'en am\'enager un (par des manipulations sur le corps).
\medskip
On peut pr\'eciser davantage en \'enon\c{c}ant au lieu de b) :
\smallskip
c) il n'est possible de pr\'eparer que des \'etats macroscopiques, sans 
contr\^ole aucun de l'\'etat microscopique pr\'ecis qui le r\'ealisera.
\medskip
La partie a) du principe est la cons\'equence rigoureuse de la sym\'etrie 
par retournement du temps, comme on l'a vu ci-dessus. La partie 
b) ou c) est un fait empirique du m\^eme type que 
\smallskip
{\sl
``Il est impossible de r\'ealiser un mouvement perp\'etuel''}
\medskip
\noindent dont on sait que c'est une formulation alternative de la 
conservation de l'\'energie, ou que
\smallskip
{\sl ``Tous les hommes sont mortels'}
\medskip
Bien s\^ur, dans le programme num\'erique qui calcule l'\'evolution des 
$200$ mol\'ecules du petit mod\`ele, il est possible de ``pr\'eparer''
n'importe quel \'etat microscopique; si on donne aux variables les 
valeurs initiales correspondant \`a la configuration de la figure $69\, m$, 
avec les vitesses en sens oppos\'e, et qu'on laisse le programme calculer, 
les $200$ mol\'ecules viendront bien sagement se ranger en rang d'oignon 
apr\`es un temps $t=200$. Mais il est impossible de r\'ealiser une 
telle initialisation de mani\`ere {\it physique}, avec un gaz ou 
avec tout autre corps mat\'eriel. C'est cela 
que dit le second principe. Le second principe ne s'applique pas 
aux simulations num\'eriques, car ces derni\`eres permettent de 
contr\^oler les \'etats microscopiques. 
\medskip 
Prenons un exemple fr\'equemment invoqu\'e dans les manuels de Physique, 
celui d'un gaz qui au d\'epart est concentr\'e dans la moiti\'e gauche d'un 
r\'eservoir (imm\'ediatement apr\`es que l'exp\'erimentateur ait retir\'e 
une cloison de s\'eparation), et qui ensuite se r\'epand dans la 
partie laiss\'ee vide. L'\'etat macroscopique 
initial est extr\^emement peu probable, en ce sens que, si on voulait 
attendre qu'il se produise spontan\'ement par le seul fait du hasard, 
il faudrait attendre pendant un temps fantastique: la probabilit\'e \'etant 
$2^{-N}$ (pour $N$ mol\'ecules), ce temps serait de l'ordre de $2^{N}$ (si 
$N = 10^{23}$, $2^N \sim 10^{301\, 030\, 000\, 000\, 000\, 000\, 000\, 000}$)
Si l'exp\'erimentateur pouvait pr\'eparer un gaz dans l'un des \'etats 
microscopiques exceptionnels o\`u les mol\'ecules, quoique distribu\'ees 
uniform\'ement entre les deux moiti\'es du r\'ecipient, le seraient 
de mani\`ere \`a \'evoluer vers l'\'etat o\`u toutes les mol\'ecules 
seront dans la moiti\'e gauche du r\'ecipient, il obtiendrait \'evidemment 
une vio\-la\-tion de la loi de croissance de l'entropie.  Mais parmi tous
les \'etats pour lesquels la distribution est uniforme, il n'y en a qu'une  
proportion $2^{-N}$ qui soit un tel \'etat exceptionnel.  Ces \'etats 
exceptionnels ne forment pas un \'etat macroscopique, ou plus exactement 
celui qu'ils forment (la distribution uniforme) est r\'ealis\'e {\it aussi} 
par d'autres \'etats microscopique non exceptionnels, qui sont $2^N$ fois 
plus nombreux. 
L'exp\'erimentateur ne peut pas, en agissant sur des param\`etres 
macroscopiques (les seuls qui soient sous son contr\^ole), pr\'eparer le 
gaz dans un de ces \'etats exceptionnels. C'est ce que dit la partie 
b) du second principe. 
\medskip
Ainsi, ce qui n'a pas \'et\'e pris en compte dans le paradoxe de Loschmidt,
et c'est justement cela qui cr\'ee le paradoxe, c'est que les \'etats 
microscopiques susceptibles de conduire \`a une diminution d'entropie 
sont des \'etats qu'il est impossible de pr\'eparer. La sym\'etrie de la 
M\'ecanique par rapport \`a l'inversion du temps, qui \'etait \`a la 
base du raisonnement,  exis\-te bien.  Mais elle s'applique de la mani\`ere 
suivante:  puisque les \'etats microscopiques susceptibles de conduire \`a 
une diminution d'entropie sont des \'etats absolument exceptionnels, 
alors par inversion du temps on peut dire de mani\`ere \'equivalente que 
{\sl les \'etats microscopiques r\'ealis\'es au cours de l'\'evolution 
m\'ecanique \`a partir d'un \'etat initial d'entropie non maximale sont 
aussi des \'etats absolument exceptionnels}.  Ces \'etats restent, 
tout au long de l'\'evolution,  inclus dans un ensemble de 
$\exp\{ S_{\cal B}\}$ \'etats microscopiques exceptionnels
(${\cal B}$ \'etant l'\'etat macroscopique initial), mais leur apparence 
macroscopique change, et devient celle d'un \'etat macroscopique qu'une 
infinit\'e d'autres \'etats microscopiques {\it non exceptionnels} 
r\'ealiseraient tout aussi bien.  Or, pour cette infinit\'e d'autres 
\'etats microscopiques, l'inversion du temps ne conduit pas du tout \`a 
une diminution de l'entropie. 
\medskip
Une fois admis l'\'enonc\'e b) ou c), qui interdit l'acc\`es au contr\^ole 
microscopique des param\`etres, il ne reste plus qu'une possibilit\'e de 
r\'ealiser un de ces \'etats microscopiques exceptionnels qui \'evoluerait 
vers une entropie moindre: compter sur la chance.  Puisqu'il n'y a aucun 
contr\^ole possible, on peut essayer de r\'ealiser un grand nombre de 
fois un \'etat macroscopique d'entropie maximale,  en esp\'erant qu'une fois, 
par hasard, ce soit un ``bon'' \'etat microscopique qui le sous-tende. 
Mais l'\'enormit\'e des nombres tels que $\nu_1$ rend cette entreprise 
absolument utopique.  On a vu plus haut,  dans le cadre du petit mod\`ele, 
qu'il faudrait r\'ealiser $10^{1476} / 10^{375} = 10^{1101}$ fois un \'etat
macroscopique d'entropie maximale pour avoir une chance de le voir 
sous-tendu par un ``bon'' \'etat microscopique. Pour un corps form\'e 
de $10^{23}$ mol\'ecules il faudrait r\'ep\'eter l'essai environ 
$10^{756\, 916\, 605\, 020\, 625\, 000\, 000\, 000}$ fois.

\vskip6mm plus3mm minus2mm

{ \bf 5. Conclusion.}

\medskip
L'exp\'erience de pens\'ee qui consisterait \`a pr\'eparer un \'etat 
microscopique exceptionnel a \'et\'e imagin\'ee par Maxwell en 
{\oldstyle 1867} (voir [7]) sous la 
forme d'un tr\`es petit syst\`eme capable d'observer instantan\'ement 
la vitesse de chaque mol\'ecule et d'effectuer une s\'election ou 
correction de trajectoire.  Un tel syst\`eme ---~appel\'e depuis 
{\it d\'emon de Maxwell}~--- permettrait, en s\'electionnant les positions 
et vitesses individuelles de chaque mol\'ecule, de pr\'eparer n'importe 
quel \'etat microscopique voulu \`a l'avance, en particulier ces \'etats 
exceptionnels qui \'evoluent en faisant d\'ecro{\^\i}tre l'entropie.
\medskip
Dans beaucoup d'ouvrages de Physique statistique, on \'evoque l'id\'ee 
du d\'emon de Maxwell,  puis on montre par des arguments plus ou moins 
quantitatifs selon les auteurs, que le second principe {\it interdit} 
l'existence d'un tel d\'emon. L'argumentation consiste \`a montrer que 
la mise en oeuvre du d\'emon de Maxwell co\^ute plus cher en entropie 
que ce qu'elle est cens\'ee en \'economiser. Cela revient \`a poser la 
croissance de l'entropie comme axiome, et \`a en d\'eduire l'impossibilit\'e 
d'un d\'emon de Maxwell. Mais entre deux assertions logiquement 
\'equivalentes, poser l'une comme axiome et l'autre comme cons\'equence 
est affaire de pure convention. Il serait tout aussi logique de 
postuler l'impossibilit\'e d'un d\'emon de Maxwell pour en d\'eduire 
la croissance de l'entropie. Comme on sait, on peut de mani\`ere tout 
\`a fait analogue d\'emontrer l'impossibilit\'e du mouvement perp\'etuel 
\`a partir de l'axiome de la conservation de l'\'energie, mais beaucoup 
d'auteurs du $XIX^{\rm e}$ si\`ecle ont pr\'ef\'er\'e poser 
l'impossibilit\'e du {\it perpetuum mobile} comme axiome premier,  
d'o\`u se d\'eduit la conservation de l'\'energie. 
\medskip 
Du point de vue de la seule logique, il est donc parfaitement 
indiff\'erent de poser la croissance de l'entropie comme principe premier, 
ou de formuler le second principe par les \'enonc\'es a) et b) ci-dessus.
Je soutiens cependant que la premi\`ere formulation est au moins en 
partie responsable de la confusion qui entoure depuis toujours le 
probl\`eme de l'irr\'eversibilit\'e, et que je mentionnais \`a la fin de 
l'introduction. La formulation du second principe par les \'enonc\'es 
a) et b) a en effet deux avantages. Le premier avantage est de montrer 
tr\`es clairement la signification concr\`ete, pratique, exp\'erimentale, 
du second principe de la Thermodynamique, alors que la formulation par la 
croissance de l'entropie, plus \'eloign\'ee des conditions pratiques, 
entretient une confusion, notamment par le fait que l'entropie n'est 
croissante que pendant des temps physiques, et non au sens math\'ematique. 
Cela permet aussi de s\'eparer deux aspects compl\'ementaires du second 
principe: l'\'enonc\'e a) exprime l'aspect purement probabiliste (celui 
qui est mis en avant dans l'interpr\'etation statistique de l'entropie);
l'\'enonc\'e b) exprime l'aspect empirique et ph\'enom\'enologique (celui 
qui provient des limites humaines).
Le second avantage est de rendre moins myst\'erieuse la mani\`ere dont
la r\'eversibilit\'e des mouvements microscopiques 
se traduit par l'irr\'eversibilit\'e des \'evolutions macroscopiques et 
donc d'\'eclaircir ais\'ement le paradoxe de Loschmidt. 
\medskip
Les lois de la Physique sont des lois objectives du monde, mais elles 
concernent le monde ph\'enom\'enal, tel qu'il est per\c{c}u par des \^etres 
tels que nous. Les lois de la Thermodynamique d\'ecrivent le comportement 
des corps macroscopiques \`a l'aide de param\`etres que nous pouvons 
manipuler et mesurer. Pour des \^etres ayant la taille des mol\'ecules 
la physique serait tr\`es diff\'erente, elle montrerait un autre aspect 
du monde, objectif lui aussi, mais compl\'ementaire. Le second principe 
contient cette dualit\'e: les lois physiques expriment des propri\'et\'es 
de l'objet (le monde), mais aussi des propri\'et\'es du sujet qui observe 
le monde et \'etablit ces lois. Le but de la 
Physique est certes de transcender la perception humaine pour atteindre 
une v\'erit\'e plus profonde et inhumaine, et c'est bien ce que fait 
l'interpr\'etation mol\'eculaire de la Thermodynamique. Mais les limites 
humaines qui conditionnent la nature des lois physiques ne sont pas tant 
celles de la perception que celles de la complexit\'e: la 
description exacte et compl\`ete du mouvement microscopique (peu 
importe ici que ce soit la classique ou la quantique), c'est-\`a-dire 
du mouvement individuel de toutes les mol\'ecules, ne nous est pas 
seulement interdite par nos facult\'es de perception, mais aussi par nos 
facult\'es de traitement de donn\'ees. Or la complexit\'e est bien plus 
difficile \`a transcender que la perception. C'est bien pourquoi il ne 
peut pas y avoir une Physique des mouvements mol\'eculaires d\'etaill\'es 
et c'est aussi pourquoi les lois de la Physique sont des lois que nous 
pouvons \'ecrire et utiliser, et qui concernent des objets que nous 
pouvons observer et manipuler. Une loi dont la formulation exigerait 
$10^{639201738}$ symboles math\'ematiques ne peut pas \^etre une loi de 
la Physique (ne serait-ce que parce jamais personne ne la d\'ecouvrira).
Pour \'echapper \`a ce second principe, Laplace devait invoquer ``une
intelligence qui, pour un instant donn\'e, conna{\^\i}trait toutes les 
forces dont la nature est anim\'ee''. 
\medskip 
L'\'enonc\'e b) exprime un fait empirique 
incontestable: il est impossible de pr\'eparer un \'etat microscopique, 
on ne peut pr\'eparer que des \'etats macroscopiques. Cet \'enonc\'e est 
consubstantiel au second principe. 
\medskip 
Beaucoup d'ouvrages (en g\'en\'eral des ouvrages non techniques, mais 
plut\^ot orient\'es vers les implications philosophiques) font \'etat 
du fait qu'on n'a jamais pu prouver l'irr\'eversibilit\'e {\it 
math\'ematiquement}. On veut signifier par l\`a qu'il n'y a aucun 
{\it th\'eor\`eme} qui fonde l'irr\'eversibilit\'e des \'evolutions 
macroscopiques sur la r\'eversibilit\'e des mouvements microscopiques. 
La tradition pseudo-scientifique que je mentionnais dans l'introduction 
se r\'eclame de ce constat, qui est juste, et qui est d'ailleurs souvent
rapport\'e par des auteurs \`a la comp\'etence d'ailleurs incontestable, 
pour se justifier: ``on n'a jamais pu le 
d\'emontrer, donc vous voyez bien que le probl\`eme est encore ouvert, 
et par cons\'equent susceptible de nouvelles recherches que justement 
nous nous proposons de mener''. Or il est \'evident qu'un tel th\'eor\`eme 
ne peut pas exister, puisque les d\'emonstrations math\'ematiques ne 
peuvent pas prendre en compte des faits empiriques tels que b). Et 
quand on d\'emontre math\'ematiquement la seule partie a) du second 
principe, il est parfaitement logique qu'on ne puisse pas r\'esoudre 
le paradoxe de Loschmidt! Mais cela ne signifie pas que le paradoxe de 
Loschmidt demeure un probl\`eme ouvert. 
Il est parfaitement r\'esolu par la prise en compte de b) et le 
raisonnement pr\'esent\'e \`a la section pr\'ec\'edente.
\medskip 
En conclusion, je propose d'\'enoncer les principes de la Thermodynamique 
sous la forme que voici, qui me para{\^\i}t beaucoup plus claire. Elle 
se pr\^ete en tous cas beaucoup moins aux malentendus que j'ai d\'ecrits 
dans cet article. Il ne s'agit pas de {\it remplacer} les \'enonc\'es usuels 
par ceux ci-dessous, mais de les compl\'eter ou de les commenter par 
ceux ci-dessous.

\vfill
\centerline{\vbox{\hsize=11cm 
{\bf Premier principe de la Thermodynamique:}
\smallskip
{\sl
``Il est impossible de r\'ealiser un mouvement perp\'etuel''.} } }
\vfill
\centerline{\vbox{\hsize=11cm 
{\bf Deuxi\`eme principe de la Thermodynamique:}
\smallskip
{\sl
``Il est impossible de r\'ealiser un d\'emon de Maxwell''.}
\medskip
(i.e. il n'est possible de pr\'eparer que des \'etats macroscopiques, sans 
contr\^ole aucun de l'\'etat microscopique pr\'ecis qui le r\'ealisera.) } }
\vfill \vfill \break

\null\vskip10mm plus4mm minus3mm

\centerline{\tit L'IRR\'EVERSIBILIT\'E : Annexes}

\vskip10mm plus4mm minus3mm

On regroup\'e ici un choix de textes historiques. Le lecteur qui aura fait 
l'effort de les lire attentivement se rendra compte que ce qu'on appelle 
``{\sl le probl\`eme de l'irr\'eversibilit\'e}'' n'a pratiquement pas 
\'evolu\'e au cours des \^ages. Chacun des auteurs cit\'es est oblig\'e 
\`a son \'epoque de r\'ep\'eter ce que les pr\'ec\'edents ont dit, 
parce qu'apparemment cela n'a pas \'et\'e compris. 
Ce ne le sera probablement jamais.  Ce constat a d\'ej\`a 
\'et\'e fait par d'autres auteurs. Par exemple un chapitre du livre de Rudolph
Peierls, {\it Surprises in Theoretical Physics} [7], commence ainsi:
\medskip
{\cit  We turn next to one of the most fundamental questions of
statistical Mechanics, to which the answer has been known to some for 
a long time, but does not appear to be known very widely even today. 
The question is about the precise origin of the irreversibility in 
statistical mechanics. 
\smallskip
\line{\hfill [7], page 73.}\par  }
\medskip
Je suis bien conscient que la r\'esolution du {\og paradoxe de Loschmidt\fg}
demande une argumentation beaucoup trop longue pour retenir l'attention de
lecteurs press\'es;  il m'a fallu un chapitre entier pour y parvenir. 
Il faut donc s'attendre \`a la persistance d'id\'ees fausses,  mais plus aptes
\`a franchir la barri\`ere de la communication.  Je n'ai aucune illusion
quant \`a l'impact de ce chapitre,  qui ne sera lu en profondeur que par
quelques esprits vraiment curieux et passionn\'es.  Pour eux,  j'ajoute
cette annexe,  qui rassemble quelques textes \'ecrits par les plus grands, 
et \`a laquelle j'assigne un double objectif:
\smallskip
a) ces textes historiques sont un bon compl\'ement au chapitre, 
car chaque auteur apporte un \'eclairage diff\'erent; 
\smallskip
b) ces textes montrent que leurs auteurs,  en leur temps,  avaient d\'ej\`a
tout expliqu\'e et tout dit. 

\bigskip

{\bf 1.}\hskip3mm Extrait de [4], (pages 18 -- 20 ):
\medskip
\def\ttv{\vrule height 56pt}

\line{\hskip29pt \raise-2pt
\vbox{\hrule
\hbox{\ttv\kern6pt
\vbox{\hsize=116mm \eightpoint
Si un syst\`eme isol\'e est dans une situation sensiblement non uniforme, 
il \'evoluera en fonction du temps pour se rapprocher de la situation 
ultime la plus uniforme o\`u il est en \'equilibre (\`a l'exception de
fluctuations qui ont peu de chances d'\^etre importantes). \par
\vskip4pt}\kern6pt \ttv\hfill}
\hrule}\hskip10pt\hfill}
\medskip
 \centerline{\bf Irr\'eversibilit\'e}
\smallskip
{\cit La conclusion [encadr\'ee ci-dessus] affirme que quand un syst\`eme
macroscopique isol\'e \'evolue en fonction du temps, il tend \`a le faire 
dans une direction bien d\'efinie: depuis un \'etat de moindre d\'esordre 
vers une situation de plus grand d\'esordre. Nous pourrions observer le 
processus du changement en filmant le syst\`eme. Supposons maintenant que 
nous projetions le film \`a l'envers (c'est-\`a-dire que nous passions le 
film dans le projecteur en marche arri\`ere) nous observerions alors sur 
l'\'ecran le m\^eme {\it processus remontant le temps} c'est-\`a-dire le 
processus qui appara{\^\i}trait si l'on imaginait que la direction du temps 
a \'et\'e renvers\'ee. Le film sur l'\'ecran serait vraiment tr\`es curieux 
en ce sens qu'il pr\'esenterait un processus par lequel un syst\`eme \'evolue 
depuis un \'etat de grand d\'esordre vers une situation moins d\'esordonn\'ee, 
chose que l'on n'observe jamais en r\'ealit\'e. En regardant simplement le 
film sur l'\'ecran, nous pourrions conclure, avec une compl\`ete certitude, 
que le film est projet\'e \`a l'envers. 
\smallskip
Un processus est dit irr\'eversible si le processus obtenu en changeant le 
signe du temps (celui qu'on observerait en projetant le film \`a l'envers) 
est tel qu'il n'appara{\^\i}t pratiquement jamais en r\'ealit\'e. Mais tous 
les syst\`emes macroscopiques hors \'equilibre \'evoluent vers l'\'equilibre, 
c'est-\`a-dire vers une situation de plus grand d\'esordre. ( . . . )
\smallskip
Notons bien qu'il n'y a rien dans les lois de la m\'ecanique r\'egissant le 
mouvement des particules du syst\`eme qui indique un sens privil\'egi\'e pour 
l'\'ecoulement du temps. En effet, imaginons que l'on prenne un film du gaz 
isol\'e en \'equilibre ( . . . )\par  }
\medskip
{\bf Commentaire:} Ici il est fait r\'ef\'erence \`a un ``film'', en fait 
une simulation num\'erique du mouvement de 40 mol\'ecules dans une bo{\^\i}te 
rectangulaire. Cette simulation est une des grandes innovations didactiques 
du {\it Berkeley Physics Course} (pages 9 et 24 -- 25), dont la force 
visuelle ne peut \^etre reproduite en citation; c'est pourquoi 
j'introduis ce commentaire. On peut mesurer le degr\'e de d\'esordre en 
donnant simplement en fonction du temps le {\it nombre de mol\'ecules 
situ\'ees dans la moiti\'e gauche de la bo{\^\i}te}. La relation entre 
ce nombre et l'entropie n'est pas clairement d\'efinie, 
mais pour l'argumentation il suffit que les deux quantit\'es aient le 
m\^eme type de comportement, qu'elles soient croissantes ou 
d\'ecroissantes en m\^eme temps et maximales ou minimales en 
m\^eme temps. Il s'agit donc de comprendre pourquoi on aboutit \`a 
l'irr\'eversibilit\'e alors que ce film est parfaitement r\'eversible: 
\medskip
{\cit  En regardant le film projet\'e sur l'\'ecran, nous n'aurions aucun
moyen de dire si le projecteur fonctionne dans le sens normal ou \`a 
l'envers. La notion de sens privil\'egi\'e pour l'\'ecoulement du temps 
n'appara{\^\i}t que lorsque l'on consid\`ere un syst\`eme macroscopique 
isol\'e dont nous avons de bonnes raisons de {\it penser} qu'il est dans 
une situation tr\`es sp\'eciale non d\'esordonn\'ee \`a un certain temps 
$t_1$. Si le syst\`eme n'a pas \'et\'e perturb\'e pendant longtemps et s'il 
atteint cette situation par le jeu des rares fluctuations \`a l'\'equilibre, 
il n'y a, en fait, rien qui indique le sens du temps. ( . . . ) \par }
\medskip
{\bf Suite du commentaire:} Cette derni\`ere phrase est capitale: 
supposons que le syst\`eme ne subit aucune rupture de son mouvement 
normal (mouvement newtonien avec collisions mutuelles ou avec la paroi 
de la bo{\^\i}te, mais surtout pas avec autre chose, comme par exemple 
une nouvelle paroi s\'eparant la bo{\^\i}te en deux). Cela exprime le 
fait que le syst\`eme est isol\'e. Il peut alors arriver que ``par hasard'' 
\`a un instant $t_1$ toutes les mol\'ecules se trouvent dans la moiti\'e 
droite de la bo{\^\i}te. Cela n'arrive pas souvent: avec quarante 
mol\'ecules, en admettant qu'entre deux vues successives du ``film'' les 
mol\'ecules se soient d\'eplac\'ees en moyenne sur une distance de l'ordre 
du dixi\`eme de la largeur de la bo{\^\i}te, il faut laisser passer au 
moins $10^{13}$ vues instantan\'ees pour avoir une chance d'observer cela. 
Avec $8$ mol\'ecules, il suffirait de 2500 images, et avec $10^{24}$ 
mol\'ecules il faudrait quelque 
$10^{300\, 000\, 000\, 000\, 000\, 000\, 000\, 000}$ images. Pour un 
film au format $16\, mm$, cela correspond \`a une dur\'ee de projection 
de l'ordre de $100$ secondes pour $8$ mol\'ecules, de $10\, 000$ ans 
pour $40$ mol\'ecules, $10^{300\, 000\, 000\, 000\, 000\, 000\, 000\, 000}$ 
ann\'ees pour $10^{24}$ mol\'ecules. Si vous regardez le film de $40$ 
mol\'ecules pendant $10\, 000$ ans, ne ratez pas l'instant o\`u toutes 
les mol\'ecules seront dans la moi\-ti\'e gauche de la bo{\^\i}te 
(attention, l'\'ev\'enement ne dure qu'une fraction de se\-con\-de), car il 
serait dommage d'avoir attendu cet instant pendant $6000$ ans et de le 
rater. Il n'aurait en effet gu\`ere de chances de se reproduire avant 
$10\, 000$ nouvelles ann\'ees. Lorsque cet \'ev\'enement se sera produit, 
d\'ecoupez le morceau de film qui commence une minute avant et se termine 
une minute apr\`es et projetez le \`a l'endroit ou \`a l'envers. Il vous 
sera effectivement impossible de savoir lequel des deux sens de projection 
est plus r\'ealiste que l'autre. 
\medskip
Mais dans aucune situation concr\`ete de la vie r\'eelle vous ne pourrez 
attendre $10^{300\, 000\, 000\, 000\, 000\, 000\, 000\, 000}$ 
ann\'ees pour voir un gaz se concentrer spontan\'ement dans une moiti\'e de 
r\'ecipient. Si vous voulez mettre un gaz dans une bouteille vous le ferez 
passer par un tuyau, pouss\'e par une pompe. D'o\`u la conclusion: 
\medskip
{\cit  Le syst\`eme \'evolue toujours vers une situation de plus grand
d\'esordre, que le temps se d\'eroule en avant ou en arri\`ere. 
La seule autre possibilit\'e pour amener le syst\`eme dans une situation
particuli\`ere non d\'esordonn\'ee \`a un instant $t_1$, c'est une 
interaction avec un autre syst\`eme \`a un instant ant\'erieur \`a $t_1$
[c'est-\`a-dire une {\it pr\'eparation}]. Mais dans ce cas, le sens du 
temps est indiqu\'e par la connaissance de cette interaction avec un autre
syst\`eme \`a un autre instant pr\'ec\'edant $t_1$. \par }

\bigskip

Les textes suivants parlent de la m\^eme chose, avec seulement des 
diff\'erences de style.

\bigskip

{\bf 2.}\hskip3mm Extrait de [5], (vers la fin):
\medskip
{\cit  Ce n'est en aucune fa\c con le signe avec lequel on compte les temps
qui constitue la diff\'erence caract\'eristique entre un \'etat organis\'e 
et un \'etat d\'enu\'e d'organisation. Si, dans l'\'etat que l'on a adopt\'e 
comme \'etat initial de la repr\'esentation m\'ecanique de l'univers, on 
venait \`a inverser exactement les directions de toutes les vitesses sans 
changer ni leurs grandeurs ni les positions des parties du syst\`eme; si 
l'on parcourait, pour ainsi dire, \`a reculons, les diff\'erents \'etats du 
syst\`eme, ce serait encore un \'etat non probable par lequel on d\'ebuterait 
et un \'etat plus probable qu'on atteindrait par la suite. C'est seulement 
pendant le laps de temps qui conduit d'un \'etat initial tr\`es peu probable 
\`a un \'etat ult\'erieur beaucoup plus probable, que les \'etats se 
transformemt d'une fa\c con diff\'erente dans la direction positive des 
temps et dans la direction n\'egative. \par }
\medskip
Et un peu plus loin 
\medskip
{\cit
Pour l'univers tout entier, les deux directions du temps sont donc 
impossibles \`a distinguer, de m\^eme que dans l'espace, il n'y a ni dessus 
ni dessous. Mais, de m\^eme qu'en une r\'egion d\'etermin\'ee de la surface 
de notre plan\`ete, nous consid\'erons comme le dessous la direction qui 
va vers le centre de la Terre, de m\^eme un \^etre vivant dans une phase 
d\'etermin\'ee du temps et habitant un tel monde individuel, d\'esignera la 
direction de la dur\'ee qui va vers les \'etats les moins probables autrement 
que la direction contraire: la premi\`ere sera pour lui le pass\'e ou le 
commencement, et la seconde l'avenir ou la fin. \par  }

\bigskip

{\bf 3.}\hskip3mm Extraits de [6] pages 75 -- 81.
\medskip
Peierls reprend d'abord le probl\`eme des mol\'ecules enferm\'ees dans une 
bo{\^\i}te divis\'ee par la pens\'ee en deux moiti\'es (``the two chambers 
problem''):
\medskip
{\cit  Some textbooks explain this paradox [Loschmidt's paradox] by saying
that, whereas particle mechanics makes predictions about the motion of 
individual particles, statistical mechanics makes probability statements 
about large ensembles of particles. This is true, but it does not explain 
why the use of probabilities and statistics should create a difference 
between past and future where none existed before. 
\smallskip
The real answer is quite different. Suppose from $t=0$ when we assumed the 
particles distributed at random within each container and to move in random 
directions, we follow the particle trajectories, not for positive times, but 
negative $t$, i.e., into the past. This will give a curve for the entropy 
looking like the broken curve in figure [hereafter], and it will be the 
mirror image of the solid curve. \par }
\vskip2mm plus7mm minus1mm
\epsfxsize=100mm
\centerline{\epsfbox{../imgEPS/ch14eps/peierls1.eps}}
\vskip2mm plus7mm minus1mm
{\cit  We see therefore that the symmetry in time is preserved fully in these
two calculations. However, the solid curve to the right describes a situation 
which occurs in practice, and therefore provides the answer to a realistic 
question, whereas the broken curve to the left does not. 
\smallskip
The situation to which the broken, left-hand curve would be applicable 
would be the following: Arrange for particles at $t=0$ to be distributed 
in given numbers over the two chambers [the two parts of the box], their 
positions being random in each chamber, and their velocites having a 
Maxwell-Boltzmann distribution. Ensure that prior to $t=0$, at least after 
some finite $-T$, there was no external interference, and observe the state 
of affairs at $t = -T$. This is evidently impossible; the only way in which 
we can influence the distribution of molecules at $t=0$ is by taking action 
prior to that time. \par }
\medskip
On reconna{\^\i}t dans ce passage essentiellement le m\^eme argument que 
dans [4], [5], [6] cit\'es ci-dessus. Mais Peierls aborde encore le 
probl\`eme par un autre c\^ot\'e (le ``Sto{\ss}zahl Ansatz'' de Boltzmann, 
``l'argument du nombre de collision''). Consid\'erons un flux de mol\'ecules 
en mouvement uniforme de vitesse $\vec{v_a}$; cela correspond \`a un \'etat 
macroscopique d'entropie non maximale. Dans ce flux, d\'ecoupons par la 
pens\'ee un cylindre \'etroit parall\`ele \`a la direction de ce flux, le 
cylindre $a$ comme sur la figure ci-apr\`es:
\vskip1mm plus7mm minus1mm
\epsfxsize=100mm
\centerline{\epsfbox{../imgEPS/ch14eps/peierls2.eps}}
\vskip1mm plus7mm minus1mm
Les mol\'ecules du cylindre $a$, qui ont toutes la m\^eme vitesse 
$\vec{v_a}$, rebondissent sur l'obstacle diffuseur (``the scatterer'', 
hachur\'e sur la figure), en sorte que leurs vitesses apr\`es cette 
collision sont diverses puisque le diffuseur est suppos\'e courbe. 
Par cons\'equent dans le cylindre $b$ de la 
figure, il ne reste plus qu'une partie des mol\'ecules qui avant la 
collision \'etaient dans le cylindre $a$, mais elles s'ajoutent \`a 
celles qui \'etaient en dehors du cylindre $a$ et qui ont poursuivi leur 
trajectoire \`a la vitesse $\vec{v_a}$ sans rencontrer de diffuseur. 
\medskip
{\cit  The Stosszahl Ansatz of Boltzmann now consists in the seemingly
innocuous assumption that $\rho_a$ [the density in cylinder $a$] equals 
the average densisty of molecules of this type anywhere in the gas, 
i.e., that there is nothing exceptional about the particular cylinder 
we have defined. 
\smallskip
This assumption is the origin of irreversibility, because if it is true, 
the corresponding statement about the cylinder labeled $b$ in the figure 
is {\it not} true. The only special thing about cylinder $a$ is that it 
contains the molecules which are going to collide with the scatterer; 
cylinder $b$ contains those which have just collided. In non-equilibrium 
conditions, for example in the presence of a drift motion in the $a$ 
direction, there will be more molecules in the gas as a whole moving 
in the $a$ direction than in the $b$ direction. Scattering by the center 
will therefore tend to increase the number in the $b$ direction. 
If $\rho_a$ is the same as elsewhere in the gas, $\rho_b$ must then be 
greater than the average. 
\smallskip
If the scattering is compared to the time-reversed situation, we see a 
difference. To reverse the direction of time, we have to replace each  
molecule in $b$ by one of the opposite velocity, and have them scattered 
by the target to travel in the direction opposite to that of $a$. The 
number would not be changed, and if $\rho_b$ in the cylinder $b$ differs 
from the average over the whole gas, it will also differ from what, with 
Boltzmann, we should assume about the inverse process. 
\smallskip
It seems intuitively obvious that the molecules should not be influenced 
by the fact that they are going to collide, and very natural that they should 
be affected by the fact that they have just collided. But these assumptions, 
which cause the irreversibility, are not self-evident. If we assume, however, 
that the state of the gas was prepared in some manner in the past, and that 
we are watching its subsequent time development, then it follows that 
correlations between molecules and scattering centers will arise only from 
past, but not from future, collisions. This shows that the situation is, 
in principle, still the same as in our two-chamber problem. \par  }
\medskip
L'argument n'est peut-\^etre pas d\'evelopp\'e avec la clart\'e maximale, 
donc j'ajoute une petite explication suppl\'ementaire. L'id\'ee est ici 
la suivante: si au d\'epart les mol\'ecules ont toutes la m\^eme vitesse 
$\vec{v_a}$, tout le monde comprend que, {\it \`a cause} du diffuseur, 
les vitesses apr\`es collision seront d\'esordonn\'ees. En retournant le 
temps, l'intuition sera choqu\'ee que des vitesses d\'esordonn\'ees 
aboutissent \`a un flux ordonn\'e, parce que ce processus inverse donnera 
l'impression que les mol\'ecules du cylindre $b$ devaient {\it savoir} 
comment elles allaient rebondir sur le diffuseur, et devaient ajuster leur 
vitesse de telle mani\`ere qu'apr\`es collision elle devienne \'egale 
\`a $-\vec{v_a}$. Elles devaient donc se d\'eterminer d'apr\`es leur futur.
Peierls veut ainsi montrer que l'inversion est contraire \`a la causalit\'e. 
\medskip
{\cit We have recognized the origin of the irreversibility in the question
we ask of statistical mechanics, and we have seen that their lack of symmetry 
originates in the limitations of the experiments we can perform. ( . . . )  As
long as we have no clear explanation for this limitation, we might speculate 
whether the time direction is necessarily universal, or whether we could 
imagine intelligent beings whose time runs opposite to ours ( . . . ) \par }
\medskip
En attendant que l'on d\'ecouvre l'explication de cette limitation, je 
propose de l'int\'egrer sans explication parmi les principes fondamentaux:
{\sl il est impossible de r\'ealiser un d\'emon de Maxwell}, tout comme on 
a fait pour l'inertie en attendant d'en trouver l'explication.  

\bigskip

{\bf 4.}\hskip3mm Extraits de [7] (chapitre 12). Ce passage de 
{\it Theory of Heat} se trouve quelques pages avant la fin. La partie qui
d\'ecrit ce qui sera plus tard appel\'e {\it d\'emon de Maxwell} ---~par
Lord Kelvin,  et avec la post\'erit\'e que l'on sait~--- est extr\^emement
c\'el\`ebre (``Imaginons cependant un \^etre dont les facult\'es seraient
si p\'en\'etrantes  . . . '').  Cependant la citation ci-dessous commence
un peu avant ce passage c\'el\`ebre et finit un peu au-del\`a afin de
montrer qu'en 1871 Maxwell avait parfaitement compris que le point
crucial du second principe n'est pas tant la croissance math\'ematique 
de l'entropie, que l'impossibilit\'e de r\'ealiser un \'etat microscopique 
pr\'ed\'efini. Cette lucidit\'e pourra \^etre confront\'ee \`a la confusion 
du d\'ebat qui perdure depuis $130$ ans. 
\medskip
{\cit  Un des faits les plus solidement \'etablis de la Thermodynamique
est que, dans un syst\`eme qui est enferm\'e \`a l'int\'erieur d'une 
cloison ne permettant ni variation de volume ni \'echange de chaleur, 
et dont la temp\'erature et la pression ont partout la m\^eme valeur,
il est impossible de produire un \'ecart de temp\'erature sans fournir 
du travail. C'est l\`a tout le sens du second principe de la 
thermodynamique; ce dernier est sans aucun doute v\'erifi\'e tant que 
nous ne manipulons les corps que par grandes masses et que nous ne 
disposons pas du pouvoir d'identifier et de manipuler les mol\'ecules 
individuelles qui composent ces masses. Imaginons cependant un \^etre 
dont les facult\'es seraient si aig\"ues qu'il serait en mesure de 
suivre chaque mol\'ecule dans son mouvement, tout en \'etant comme nous 
m\^emes de  conformation essentiellement finie; alors il lui serait 
possible de r\'ealiser ce qui nous est impossible. Car nous avons vu 
que les mol\'ecules d'un gaz de temp\'erature uniforme contenu dans  
un r\'ecipient ne sont nullement anim\'ees de vitesses uniform\'ement 
distribu\'ees, bien que la vitesse moyenne soit pratiquement 
la m\^eme sur n'importe quel sous-ensemble suffisamment gros d'entre elles. 
Imaginons donc qu'un tel r\'ecipient soit divis\'e en deux parties $A$ et 
$B$ par une cloison s\'eparatrice, dans laquelle serait pratiqu\'ee une 
petite ouverture et qu'un tel \^etre capable de voir les mol\'ecules 
individuelles ouvre ou ferme cette ouverture de mani\`ere \`a ne 
laisser passer de $A$ vers $B$ que les seules mol\'ecules rapides, et 
de $B$ vers $A$ les seules mol\'ecules lentes. Cet \^etre est ainsi 
en mesure de relever la temp\'erature de la partie $B$ au d\'etriment 
de la partie $A$ sans d\'epense de travail, ce qui est en contradiction 
avec le second principe. 
\smallskip
Ce n'est l\`a qu'un exemple parmi d'autres, dans lequel les conclusions 
que nous avons tir\'ees de notre exp\'erience avec les corps compos\'ees 
d'un grand nombre de mol\'ecules pourraient cesser d'\^etre applicables 
\`a des m\'ethodes d'observation et d'investigation plus fines telles 
que pourraient les mettre en oeuvre des \^etres capables de percevoir et
manipuler individuellement ces mol\'ecules que nous ne pouvons manipuler 
que par grandes masses. 
\smallskip
Et puisqu'en les manipulant par masses nous n'avons aucun acc\`es aux 
mol\'ecules individuelles, nous sommes bien oblig\'es de recourir au 
calcul statistique; ce pas accompli, nous abandonnons la m\'ethode 
dynamique rigoureuse, par laquelle nous calculons le d\'etail de chaque 
mouvement individuel. \par }
\medskip
{\bf N. B.} Le passage ci-dessus est l'origine historique du {\it d\'emon 
de Maxwell}; c'est en effet dans {\it Theory of Heat} de {  1871}
que cette id\'ee est publi\'ee pour la premi\`ere fois. 
Elle \'etait cependant reprise d'une lettre de Maxwell \`a Peter Guthrie 
Tait en {  1867}.

\bigskip

{\bf 5.}\hskip3mm Voici maintenant deux extraits de H. Poincar\'e. 
Le principal argument avanc\'e par Poincar\'e est celui du 
n\'ecessaire retour de n'importe quel 
syst\`eme dynamique \`a des \'etats d\'ej\`a occup\'es dans le pass\'e. 
Il s'agit de la propri\'et\'e des syst\`emes dynamiques que, ou bien les 
trajectoires sont p\'eriodiques, ou bien elles remplissent de mani\`ere 
dense l'hypersurface d'\'energie. Si le syst\`eme a occup\'e \`a l'instant 
$t=0$ un \'etat microscopique d\'efini 
par les valeurs de toutes les coordonn\'ees et impulsions, alors au bout 
d'un temps fini $T$ il repassera {\it aussi pr\`es qu'on voudra} de cet 
\'etat initial apr\`es s'en \^etre \'ecart\'e. Ainsi, si l'entropie 
avait une valeur non maximale $S_0$ \`a $t=0$, elle redescendra 
in\'evitablement \`a cette valeur \`a l'instant $T$, apr\`es avoir 
\'et\'e maximale entretemps. Cet argument a \'et\'e repris notamment par 
E. Zermelo dans [11] (voir extraits de [12] et [13] ci-dessous). On ne
reproduira 
pas ici les travaux de Poincar\'e sur ce point, ils sont bien trop 
techniques et de toute fa\c{c}on sont fort connus. On les trouvera 
dans [9], mais aussi dans n'importe quel ouvrage actuel sur le chaos.
\medskip
L'extrait qui suit concerne un autre th\'eor\`eme qui affirme qu'une 
fonction des coordonn\'ees et des vitesses d'un syst\`eme dynamique 
ne peut en aucun cas \^etre monotone. 

\bigskip

Extraits de [9]: 
\medskip
{\cit  Parmi les tentatives qui ont \'et\'e faites pour rattacher aux
th\'eor\`emes g\'en\'eraux de la M\'ecanique les principes fondamentaux de 
la Thermodynamique, la plus int\'eressante est, sans contredit, celle que 
M. Helmholtz a d\'evelopp\'ee dans son M\'emoire sur la statique des 
syst\`emes monocycliques (Journal de Crelle, t. {\bf 97}) et dans son 
M\'emoire sur le principe de la moindre action (Journal de Crelle, t. {\bf 
100}). L'explication propos\'ee dans ces deux M\'emoires me para{\^\i}t 
satisfaisante en ce qui concerne les ph\'enom\`enes r\'eversibles. 
\smallskip
Les ph\'enom\`enes irr\'eversibles se pr\^etent-ils de la m\^eme mani\`ere 
\`a une explication m\'ecanique; peut-on, par exemple, en se repr\'esentant 
le monde comme form\'e d'atomes, et ces atomes comme soumis \`a des 
attractions d\'ependant des seules distances, expliquer pourquoi la chaleur 
ne peut jamais passer d'un corps froid sur un corps chaud? Je ne le crois 
pas, et je vais expliquer pourquoi la th\'eorie de l'illustre physicien ne 
me semble pas s'appliquer \`a ce genre de ph\'enom\`enes. \par }
\medskip
Poincar\'e expose alors sa d\'emonstration d'un th\'eor\`eme qui sera 
fr\'e\-quem\-ment invoqu\'e dans la suite (voir plus bas les extraits de 
[14]). En voici le principe.
Le syst\`eme \'etant un syst\`eme dynamique, on peut avoir les \'equations 
du mouvement exact de toutes les mol\'ecules sous la forme hamiltonienne; 
si les $x_j$ sont les coordonn\'ees et les $p_j$ les impulsions des 
mol\'ecules, on aura [$H(x,p)$ \'etant l'hamiltonien du syst\`eme]:
$$\dot{x_j} = {\partial H \over \partial p_j}\quad ; \hskip8mm 
\dot{p_j} = -{\partial H \over \partial x_j}\quad .$$ 
Si $S(t)$ est l'entropie comme fonction du temps on peut \'ecrire
$$\eqalignno{
{dS \over dt} &= \sum_j \Big( {\partial S \over \partial x_j}\,\dot{x_j} + 
{\partial S \over \partial p_j}\,\dot{p_j}\Big)\; =  \cr 
&= \sum_j \Big( {\partial S \over \partial x_j}\, {\partial H \over 
\partial p_j} - {\partial S \over \partial p_j}\, {\partial H \over 
\partial x_j}\Big)\cr }$$ 
et ceci doit \^etre positif. 
Si un \'etat quelconque du syst\`eme correspond \`a l'\'equilibre, 
appelons $p_j^{(0)}$ et $x_j^{(0)}$ les coordonn\'ees correspondantes 
et consid\'erons le d\'eveloppement de Taylor des fonctions $S$ et $H$ 
en puissances de $p_j - p_j^{(0)}$ et $x_j - x_j^{(0)}$. Le terme 
lin\'eaire est nul \`a cause du choix de l'origine. Poincar\'e \'ecrit
[j'ai modifi\'e ses notations pour respecter les usages actuels]: 
\medskip
{\cit  Pour ces valeurs ($p_j^{(0)}$ et $x_j^{(0)}$), les d\'eriv\'ees du
premier ordre de $S$ s'annulent, puisque $S$ doit atteindre son maximum. 
Les d\'eriv\'ees de $H$ s'annulent \'egalement, puique ce maximum est une 
position d'\'equilibre et que $dx_j / dt$ et $dp_j / dt$ doivent s'annuler. 
\smallskip
Si donc nous d\'eveloppons $S$ et $H$ suivant les puissances croissantes 
des $p_j - p_j^{(0)}$ et $x_j - x_j^{(0)}$, les premiers termes qui ne 
s'annuleront pas seront ceux du deuxi\`eme degr\'e. Si, de plus, on 
consid\`ere les valeurs de $p_j$ et de $x_j$ assez voisines de $p_j^{(0)}$ 
et $x_j^{(0)}$ pour que les termes du troisi\`eme degr\'e soient 
n\'egligeables, $S$ et $H$ se r\'eduiront \`a deux formes quadratiques en 
$p_j - p_j^{(0)}$ et $x_j - x_j^{(0)}$.
\smallskip
La forme $H$ pourra \^etre d\'efinie ou ind\'efinie. L'expression 
$$\sum_j \Big( {\partial S \over \partial x_j}\, {\partial H \over 
\partial p_j} - {\partial S \over \partial p_j}\, {\partial H \over 
\partial x_j}\Big)$$
sera encore une forme quadratique par rapport aux $p_j - p_j^{(0)}$ 
et aux $x_j - x_j^{(0)}$. 
\smallskip
Pour que l'in\'egalit\'e $dS / dt > 0$ soit satisfaite, il faudrait que 
cette forme f\^ut d\'efinie et positive; or il est ais\'e de s'assurer 
que cela est impossible si l'une des deux formes $S$ et $H$ est d\'efinie, 
ce qui a lieu ici. 
\smallskip
Nous devons donc conclure que les deux principes de l'augmentation de 
l'entropie et de la moindre action (entendu au sens hamiltonien) sont 
inconciliables. Si donc M. von Helmholtz a montr\'e, avec une admirable 
clart\'e, que les lois des ph\'enom\`enes r\'eversibles d\'ecoulent des 
\'equations ordinaires de la Dynamique, il semble probable qu'il faudra 
chercher ailleurs l'explication des ph\'enom\`enes irr\'eversibles et 
renoncer pour cela aux hypoth\`eses famili\`eres de la M\'ecanique 
rationnelle d'o\`u l'on a tir\'e les \'equations de Lagrange et de 
Hamilton. \par  }

\bigskip

Extraits de [10]:  
\medskip
{\cit  Maxwell admet que, quelle que soit la situation initiale du syst\`eme,
il passera toujours une infinit\'e de fois, je ne dis pas par toutes les 
situations {\it compatibles avec l'existence des int\'egrales, mais aussi 
pr\`es qu'on voudra} d'une quelconque de ces situations. 
\smallskip
C'est ce qu'on appelle le {\it postulat de Maxwell}. Nous le discuterons 
plus loin. ( . . .  ) \par  }
\medskip
Et plus loin: 
\medskip
{\cit  Tous les probl\`emes de M\'ecanique admettent certaines solutions
remarquables que j'ai appel\'ees p\'eriodiques et asymptotiques et dont 
j'ai parl\'e ici m\^eme dans un pr\'ec\'edent article [9]. 
\smallskip
Pour ces solutions, le postulat de Maxwell est {\it certainement faux}. 
\smallskip
Ces solutions, il est vrai, sont tr\`es particuli\`eres, elles ne peuvent se 
rencontrer que si la situation initiale est tout \`a fait exceptionnelle.
\smallskip
Il faudrait donc au moins ajouter \`a l'\'enonc\'e du postulat cette 
restriction, d\'ej\`a bien propre \`a provoquer nos doutes: {\it sauf pour 
certaines situations initiales exceptionnelles}. 
\smallskip
Ce n'est pas tout: si le postulat \'etait vrai, le syst\`eme solaire 
serait instable; s'il est stable, en effet, il ne peut passer que par 
des situations peu diff\'erentes de sa situation initiale. C'est l\`a 
la d\'efinition m\^eme de la stabilit\'e. 
\smallskip
Or, si la stabilit\'e du syst\`eme solaire n'est pas d\'emontr\'ee, 
l'instabilit\'e l'est moins encore et est m\^eme peu probable. 
\smallskip
Il est possible et m\^eme vraisemblable que le postulat de Maxwell est 
vrai pour certains syst\`emes et faux pour d'autres, sans qu'on ait aucun 
moyen certain de discerner les uns des autres. 
\smallskip
Il est permis de supposer {\it provisoirement} qu'il s'applique aux gaz 
tels que la th\'eorie cin\'etique les con\c{c}oit; mais cette th\'eorie 
ne sera solidement assise que quand on aura justifi\'e cette supposition 
mieux qu'on ne l'a fait jusqu'ici. \par }
\smallskip
On comprendra mieux l'ampleur du malentendu entre Poincar\'e (\'eminent 
repr\'esentant de la {\it Physique math\'ematique}) et la Physique 
r\'eelle en \'evaluant {\it quantitativement} les grandeurs dont seules 
l'``{existence}'', ou la ``{finitude}'', sont ici \'evoqu\'ees.
En effet, le th\'eor\`eme de Poincar\'e sur l'\'eternel retour d'un 
syst\`eme dynamique au voisinage de son \'etat initial est un th\'eor\`eme 
qui s'\'enonce sous la forme:
\smallskip
{\sl  ``pour tout $\varepsilon$, il existe un temps $T_{\varepsilon}$ au
bout duquel le syst\`eme repassera \`a une distance inf\'erieure \`a 
$\varepsilon$ de son \'etat initial.'' \par }
\medskip
Poincar\'e interpr\`ete le second principe d'une mani\`ere analogue: pour 
lui, affirmer la croissance de l'entropie, c'est affirmer que pour tout 
$t' > t$ on doit avoir $S(t') \geq S(t)$. 
Or le principe {\it physique} est tr\`es diff\'erent; il dit que pour 
toute dur\'ee {\it physique} l'entropie ne peut diminuer que d'une valeur 
infime, et pendant un temps tr\`es bref. Le th\'eor\`eme de Poincar\'e 
affirme qu'il existe un temps $T_{\varepsilon}$ au bout duquel l'entropie 
reprendra sa valeur initiale, mais il ne dit pas que ce temps est 
de l'ordre de $10^{300\, 000\, 000\, 000\, 000\, 000\, 000\, 000}$ 
ann\'ees, ni que la dur\'ee du retour \`a la valeur initiale est de l'ordre 
d'une fraction de seconde. Le vrai second principe ne dit pas sans autre 
pr\'ecision que l'entropie est une fonction croissante du temps. Si on veut 
l'\'enoncer sous une forme vraiment compl\`ete, cela donne ceci: 
\medskip
{\sl a) Pour tout \'etat initial du syst\`eme sauf un nombre infime, et
pendant des dur\'ees ayant un sens physique [donc incomparablement plus 
courtes que $10^N$, $N$ \'etant le nombre de mol\'ecules], l'entropie du 
syst\`eme ne s'\'ecarte pas notablement d'une fonction croissante. 
\medskip
b) Pendant chaque seconde de la dur\'ee d'existence physique du 
syst\`eme isol\'e, l'entropie ne cesse de cro{\^\i}tre et d\'ecro{\^\i}tre 
des millions de fois, en effectuant des oscillations qui sont toujours 
imperceptibles, car il est absolument impossible qu'un \'ecart notable se 
produise spontan\'ement et ``{par hasard}'' avant des temps bien
sup\'erieurs \`a $10^{\sqrt{N}}$.  }

\bigskip

{\bf 6.}\hskip3mm Et voici la r\'eponse de Boltzmann aux 
arguments de Poincar\'e. 
Ces derniers ont \'et\'e rapport\'es aux physiciens de langue allemande 
par E. Zermelo (Wiedemanns Annalen, {  1896}, vol. {\bf 57},
p. 485 et vol. {\bf 59}, p. 793).
\medskip
Extraits de [12]:   
\medskip
{\cit  Le m\'emoire de M. Zermelo ``{\it\"Uber einen Satz der Dynamik und
die mechanische W\"armetheorie}'' montre que mes travaux sur le sujet n'ont 
malgr\'e tout pas \'et\'e compris; en d\'epit de cela, je dois cependant 
me r\'ejouir de cette publication comme \'etant la premi\`ere manifestation 
de l'int\'er\^et suscit\'e par ces travaux en Allemagne.
\smallskip
Le th\'eor\`eme de Poincar\'e discut\'e au d\'epart par M. Zermelo est bien 
entendu juste, mais son application \`a la th\'eorie de la chaleur ne 
l'est pas. J'ai d\'eduit la loi de r\'epartition des vitesses de Maxwell 
du th\'eor\`eme probabiliste qu'une certaine grandeur ${\cal H}$ (en 
quelque sorte la mesure de l'\'ecart de l'\'etat du syst\`eme par rapport 
\`a l'\'etat d'\'equilibre) ne peut, pour un gaz en repos dans un 
r\'ecipient, que diminuer. La meilleure fa\c{c}on d'illustrer le mode de 
d\'ecroissance de cette grandeur sera d'en repr\'esenter la courbe de 
variation, en portant le temps $t$ en abscisse et la quantit\'e 
${\cal H}(t) - {\cal H}_{\rm min}$ en ordonn\'ee; on obtiendra ainsi ce 
que j'appelle la courbe ${\cal H}$. \hskip6mm ( . . .  )
\smallskip
La courbe reste alors la plupart du temps tout pr\`es de l'axe des 
abscisses. Ce n'est qu'\`a des instants extr\^emement rares qu'elle s'en 
\'ecarte, en formant ainsi une bosse, et il est clair que la probabilit\'e  
d'une telle bosse d\'ecro{\^\i}t rapidement avec sa hauteur. \`A 
chacun des instants pour lesquels l'ordonn\'ee de la courbe est tr\`es 
petite, r\`egne la distribution des vitesses de Maxwell; on s'en \'ecarte 
notablement l\`a o\`u il y a une grosse bosse. M. Zermelo croit alors 
pouvoir d\'eduire du th\'er\`eme de Poincar\'e que le gaz ne peut se 
rapprocher constamment de la distribution de Maxwell que pour certaines 
conditions initiales tr\`es particuli\`eres, en nombre infime compar\'e 
\`a celui de toutes les configurations possibles, tandis que pour la 
plupart des conditions initiales il ne s'en rapprocherait pas. 
Ce raisonnement ne me semble pas correct.  \hskip3mm ( . . .  )
\smallskip
Si l'\'etat [microscopique] initial du gaz correspond \`a une tr\`es 
grosse bosse, c'est-\`a-dire s'il s'\'ecarte compl\`etement de la 
distribution des vitesses de Maxwell, alors il s'en rapprochera avec 
une \'enorme probabilit\'e, apr\`es quoi il ne s'en \'ecartera plus 
qu'infinit\'esimalement pendant un temps gigantesque. Toutefois, si 
on attend encore plus longtemps, une nouvelle bosse notable de la 
courbe ${\cal H}$ finira par se produire \`a nouveau et si ce temps est 
suffisamment prolong\'e on verra m\^eme se reproduire l'\'etat initial. 
On peut dire que, si le temps d'attente est infiniment long au sens 
math\'ematique, le syst\`eme reviendra infiniment souvent \`a l'\'etat 
initial. 
\smallskip
Ainsi M. Zermelo a enti\`erement raison quand il affirme que le mouvement 
est, au sens math\'ematique, p\'eriodique [ou quasi-p\'eriodique]; 
mais loin de contredire mes th\'eor\`emes, cette p\'eriodicit\'e est au 
contraire en parfaite harmonie avec eux.
\smallskip
\line{\hfill (Vienne, le 20 mars 1896)} \par }
\medskip
Cette argumentation magistrale se poursuit, mais je l'interromps ici 
avec regret pour \'eviter de rendre cette anthologie trop longue. 

\bigskip

Extraits de [13],  suite de la r\'eponse de Boltzmann aux objections 
de Zermelo. 
\medskip
{\cit  Imaginons que nous retirions brusquement une cloison qui s\'eparait
deux gaz de nature diff\'erente enferm\'es dans un r\'ecipient [par 
exemple azote d'un c\^ot\'e et oxyg\`ene de l'autre]. On aurait du mal \`a 
trouver une autre situation o\`u il y aurait davantage de variables aussi 
ind\'ependantes les unes des autres, et o\`u par cons\'equent  
l'intervention du Calcul des probabilit\'es serait plus justifi\'ee. 
Admettre que dans un tel cas le Calcul des probabilit\'es ne s'applique 
pas, que la plupart des mol\'ecules ne s'entrem\^elent pas, qu'au contraire 
des parties notables du r\'ecipient contiendraient nettement plus 
d'oxyg\`ene, d'autres plus d'azote, et ce pendant longtemps, est une 
th\`ese que je suis bien incapable de r\'efuter en calculant dans le 
d\'etail le mouvement exact de trillions [$10^{12}$] de mol\'ecules, dans 
des millions de cas particuliers diff\'erents, et d'ailleurs je ne veux 
pas le faire; une telle vision ne serait pas assez fond\'ee pour remettre 
en question l'usage du Calcul des probabilit\'es, et les cons\'equences 
logiques qui s'ensuivent. 
\smallskip
D'ailleurs le th\'eor\`eme de Poincar\'e ne contredit pas l'usage du Calcul 
des probabilit\'es, au contraire il parle en sa faveur, puisque ce 
dernier enseigne lui aussi que sur des dur\'ees fantastiques surviendront 
toujours de brefs instants pendant lesquels on sera dans un \'etat de 
faible probabilit\'e et de faible entropie, o\`u par 
cons\'equent se produiront \`a nouveau des \'etats plus ordonn\'es et m\^eme 
des \'etats tr\`es proches de l'\'etat initial. En ces temps prodigieusement 
\'eloign\'es dans le futur, n'importe quel \'ecart notable de l'entropie par 
rapport \`a sa valeur maximale demeurera \'evidemment toujours hautement 
improbable, mais l'existence d'un tr\`es bref \'ecart sera lui aussi 
[sur une telle dur\'ee prodigieusement longue] toujours hautement 
probable. \par}
\medskip
En effet, le Calcul des probabilit\'es enseigne bien que si par exemple 
on jette une pi\`ece mille fois il est tr\`es peu probable d'avoir mille 
fois face (la probabilit\'e en est $2^{-1000} \simeq 10^{-301}$); mais si 
on la jette $10^{302}$ fois, alors on n'a qu'une chance sur $45\, 000$ de 
ne {\it jamais} avoir une s\'erie de mille faces cons\'ecutives. 
\medskip
{\cit \noindent (reprise de la citation)\hskip12pt Il est clair aussi,
d'apr\`es cet exemple 
[celui de l'oxyg\`ene et de l'azote], que si le processus se d\'eroule 
de fa\c{c}on irr\'eversible pendant un temps observable, c'est parce 
qu'on est parti d\'elib\'er\'ement d'un \'etat improbable. ( . . . )
\smallskip
\line{\hfill (Vienne, le 16 d\'ecembre 1896)} \par }

\bigskip

{\bf 7.}\hskip3mm Extraits de [14] pages 205 -- 207. 
Cet extrait de {\it la nouvelle alliance} est particuli\`erement 
lumineux. Cependant on le comparera aux textes de Maxwell et 
Boltzmann ci-dessus pour constater que ce qui est expliqu\'e l\`a 
en 1979 \'etait d\'ej\`a bien compris par les p\`eres fondateurs. Il est 
cependant pr\'evisible que l'effort d'explication tent\'e par Prigogine et 
Stengers restera aussi vain que les efforts de Boltzmann, et que d'autres 
auteurs devront le r\'ep\'eter \`a nouveau en 2079. 
\medskip
{\cit D\`es la publication du travail de Boltzmann en {  1872},
des objections furent oppos\'ees \`a l'id\'ee que le mod\`ele propos\'e 
ramenait l'irr\'eversibilit\'e \`a la dynamique. Retenons ici deux d'entre 
elles, l'une de Poincar\'e, l'autre de Loschmidt. 
\smallskip
L'objection de Poincar\'e porte sur la question de la sym\'etrie de 
l'\'equation de Boltzmann. \par  }
\medskip
Pour \'eviter de rendre la citation trop longue ou d'avoir \`a donner trop 
d'explications, je signale simplement qu'il s'agit ici de l'\'equation 
\'etablie par Boltzmann pour la fonction de distribution $f(r,v,t)$ qui 
repr\'esente le nombre de mol\'ecules du syst\`eme ayant, \`a l'instant $t$, 
la vitesse $v$ et la position $r$. Boltzmann a montr\'e que la fonction 
${\cal H} = \int f\, \log f \, dv$ ne peut que diminuer, et a postul\'e 
que l'entropie du syst\`eme est la m\^eme chose que $-k{\cal H}$ ($k$ 
constante de Boltzmann). 
\medskip
{\cit  Un raisonnement correct [\'ecrit Poincar\'e] ne peut mener \`a des
conclusions en contradiction avec les pr\'emisses. Or, comme nous l'avons 
vu, les propri\'et\'es de sym\'etrie de l'\'equation d'\'evolution obtenue 
par Boltzmann pour la fonction de distribution contredisent celles de la 
dynamique. Boltzmann ne peut donc pas avoir d\'eduit l'entropie de la 
dynamique, il a introduit quelque chose, un \'el\'ement \'etranger \`a 
la dynamique. Son r\'esultat ne peut donc \^etre qu'un mod\`ele 
ph\'enom\'enologique, sans rapport direct avec le dynamique. 
\smallskip
Poincar\'e \'etait d'autant plus ferme dans sa position qu'il avait 
\'etudi\'e dans une br\`eve note s'il \'etait possible de construire une 
fonction $M$ des positions et des moments, $M(p,q)$, qui aurait les 
propri\'et\'es de l'entropie (ou plut\^ot de la fonction ${\cal H}$): 
alors qu'elle m\^eme serait positive ou nulle, sa variation au cours du 
temps ne pourrait que la faire d\'ecro{\^\i}tre ou la maintenir \`a une 
valeur constante. Sa conclusion fut n\'egative ---~dans le cadre de la 
dynamique {\it hamiltonienne} une telle fonction n'existe pas. Comment, 
d'ailleurs s'en \'etonner? Comment les lois r\'eversibles de la 
dynamique pourraient-elles engendrer, de quelque mani\`ere que ce 
soit, une \'evolution irr\'eversible? C'est sur une note d\'ecourag\'ee 
que Poincar\'e termine ses c\'el\`ebres {\it Le\c{c}ons de Thermodynamique}: 
il faudra sans doute faire appel \`a d'autres consid\'erations, au calcul 
des probabilit\'es. Mais comment justifier cet appel \`a des notions 
\'etrang\`eres \`a la dynamique? \par }
\medskip
{\bf Remarque 1:} Ce passage [voir aussi les citations directes de 
Poincar\'e ci-dessus] met l'accent sur une des 
sources possibles de confusion. Le r\'esultat de Poincar\'e est un 
th\'eor\`eme {\it math\'ematique}: ``il ne peut pas exister de fonction 
$M(p,q)$ qui soit \`a la fois d\'ecroissante et toujours positive''. 
Or l'entropie, ou toute fonction qui en tient lieu (comme ${\cal H}$), 
ou toute autre fonction caract\'erisant un \'etat {\it macroscopique} 
(comme ``le nombre de mol\'ecules situ\'ees dans la partie gauche du 
r\'ecipient'', etc.) n'est pas une fonction monotone, \`a cause des 
fluctuations. Lorsqu'on dit que le syst\`eme est parvenu \`a 
l'\'equilibre et y reste, c'est-\`a-dire que l'entropie est devenue 
maximale, cela signifie qu'elle continue presque \'eternellement 
\`a osciller autour de son maximum th\'eorique et non qu'elle 
reste math\'ematiquement \'egale \`a ce maximum ou continue de s'en 
approcher sans cesse davantage et {\it en croissant}. Ces oscillations 
sont tr\`es petites si le nombre $N$ de mol\'ecules est grand (leur 
\'ecart-type est de l'ordre de $1 / \sqrt{N}$), mais il peut s'en 
produire d'importantes si on attend pendant un temps de l'ordre de 
$10^N$. Il est donc essentiel de bien comprendre ceci: l'entropie n'est 
pas une fonction croissante dans le sens math\'ematique du terme; c'est 
seulement une fonction croissante dans un sens pratique. On peut 
l'exprimer en disant que {\it sur des dur\'ees raisonnables}, et 
{\it \`a de petites fluctuations pr\`es} elle ne peut d\'ecro{\^\i}tre. 
La v\'eritable entropie d'un syst\`eme physique r\'eel n'est donc pas 
concern\'ee par le th\'eor\`eme de Poincar\'e. C'est ce que Boltzmann 
s'est efforc\'e d'expliquer dans [12] et [13].
\medskip
Suite de la citation (page 206):
\medskip
{\cit L'objection de Loschmidt permet, quant \`a elle, de mesurer
les {\it limites de validit\'e} du mod\`ele cin\'etique de Boltzmann. 
Il note en effet que ce mod\`ele ne peut rester valable apr\`es un 
renversement du sens des vitesses $v \mapsto -v$. {\it Du point de 
vue de la dynamique}, il n'y a pas d'\'echappatoire: les collisions, 
se produisant en sens inverse, ``{d\'eferont}'' ce qu'elles ont fait,
le syst\`eme retournera vers son \'etat initial. Et la fonction 
${\cal H}$, qui d\'epend de la distribution des vitesses, devra bien 
cro{\^\i}tre elle aussi jusqu'\`a sa valeur initiale. Le renversement 
des vitesses impose donc une \'evolution {\it antithermodynamique}. 
Et en effet, la simulation sur ordinateur confirme bien une 
{\it croissance de} ${\cal H}$ apr\`es l'inversion des vitesses sur 
un syst\`eme dont les trajectoires sont calcul\'ees de mani\`ere exacte.
\smallskip
Il faut donc admettre que la tentative de Boltzmann n'a rencontr\'e 
qu'un succ\`es partiel: certaines conditions initiales, notamment 
celles qui r\'esultent de l'op\'eration d'inversion des vitesses, 
peuvent engendrer, en contradiction avec le mod\`ele cin\'etique, 
une \'evolution {\it dynamique} \`a ${\cal H}$ croissant. Mais 
comment distinguer les syst\`emes auxquels le raisonnement de 
Boltzmann s'applique de ceux auxquels il ne s'applique pas? 
\smallskip
Ce probl\`eme une fois pos\'e, il est facile de reconna{\^\i}tre 
la nature de la limitation impos\'ee au mod\`ele de Boltzmann. Ce 
mod\`ele repose en fait sur une hypoth\`ese statistique qui permet 
l'\'evaluation du nombre moyen de collisions ---~``{le chaos
mol\'eculaire}'' . \par }
\medskip
{\bf Remarque 2:} le terme ``chaos'' n'est pas employ\'e ici 
dans le sens pr\'ecis qu'il a acquis depuis, et devrait \^etre remplac\'e 
---~ici et dans la suite~--- par ``stochasticit\'e''. En effet c'est 
le mouvement dynamique exact des mol\'ecules qui est chaotique (``chaos 
d\'eterministe'') et l'hypoth\`ese statistique qu'il est question 
d'introduire consiste \`a \'eliminer l'exactitude d\'eterministe des 
conditions initiales et de les supposer simplement al\'eatoires. 
\medskip
{\cit  \noindent (reprise de la citation) Cette hypoth\`ese suppose
qu'{\it avant} la collision, les mol\'ecules ont des comportements 
ind\'ependants les uns des autres, ce qui revient \`a dire qu'il n'y a 
aucune corr\'elation entre leurs vitesses. Or, si on impose au syst\`eme 
de ``{remonter le temps}'' , on cr\'ee une situation tout \`a fait
anormale: certaines mol\'ecules sont d\'esormais ``{destin\'ees}''\
\`a se rencontrer en un instant d\'eterminable \`a l'avance et \`a subir 
\`a cette occasion un changement de vitesse pr\'ed\'etermin\'e. Aussi 
\'eloign\'ees qu'elles soient les unes des autres au moment de l'inversion 
des vitesses, cette op\'eration cr\'ee donc entre elles des corr\'elations, 
elles ne sont plus ind\'ependantes. L'hypoth\`ese du chaos 
[stochasticit\'e] mol\'eculaire ne peut \^etre faite \`a propos 
d'un syst\`eme qui a subi l'op\'eration d'inversion des vitesses.
\smallskip
L'inversion des vitesses est donc une op\'eration qui cr\'ee un syst\`eme 
{\it hautement organis\'e}, au comportement apparemment finalis\'e: 
l'effet des diverses collisions produit, comme par harmonie pr\'e\'etablie, 
une \'evolution globale ``{antithermodynamique}'' (par exemple la
s\'egr\'egation spontan\'ee entre mol\'ecules lentes et rapides si, 
\`a l'instant initial, le syst\`eme avait \'et\'e pr\'epar\'e par la mise
en contact de deux gaz de temp\'eratures diff\'erentes). Mais accepter la 
possibilit\'e de telles \'evolutions antithermodynamiques, m\^eme rares, 
m\^eme exceptionnelles (aussi exceptionnelles que la condition initiale 
issue de l'inversion des vitesses), c'est mettre en cause la formulation 
du second principe: il existe des cas o\`u par exem\-ple une diff\'erence 
de temp\'erature pourrait se produire ``{spontan\'ement}'' . Nous
devons alors pr\'eciser les circonstances dans lesquelles un processus 
irr\'eversible pourrait devenir r\'eversible, voire m\^eme annuler un 
processus irr\'eversible qui s'est produit dans le pass\'e. Le principe 
cesse d'\^etre un principe pour devenir une g\'en\'eralisation de port\'ee 
limit\'ee. \par  }
\medskip
{\bf Remarque 3:} Prigogine et Stengers parlent donc ici d'une {\it mise 
en cause} du second principe, et d'une limitation de sa port\'ee. 
La limitation \'etant que, pour un syst\`eme dynamique {\it chaotique} 
(au sens actuel de ce terme: chaotique = rigoureusement d\'eterministe, 
mais avec extr\^eme sensibilit\'e aux conditions initiales), l'entropie 
n'est pas une fonction croissante dans {\it absolument} tous les cas. En 
r\'ealit\'e c'est plut\^ot un probl\`eme d'interpr\'etation de l'\'enonc\'e 
du second principe. Il y a un ``second principe pour math\'ematiciens'' qui 
stipule que l'entropie est une fonction du temps $t$ qui tend en croissant 
vers une limite lorsque $t$ tend vers l'infini. Ce principe est faux car 
il existe des \'etats microscopiques du syst\`eme qui le mettent en d\'efaut 
(les \'etats ``{\it hautement organis\'es}, au comportement apparemment 
finalis\'e''). M\^eme si on \'ecarte ces \'etats exceptionnels, la 
d\'emonstration de Poincar\'e prouve en outre que l'entropie n'est 
jamais rigoureusement croissante au sens math\'ematique, mais on pourrait 
ais\'ement corriger ce dernier d\'efaut en \'enon\c{c}ant par exemple: 
``l'entropie ne s'\'ecarte jamais notablement d'une fonction croissante''. 
La difficult\'e qui demeurera cependant toujours est que, m\^eme 
ainsi \'enonc\'e, on ne pourra pas garantir avec l'absolue certitude 
math\'ematique que la fonction reste croissante pendant des dur\'ees 
aussi grandes qu'on veut. Pourtant, pour la quasi totalit\'e des \'etats, 
la fonction restera croissante pendant des dur\'ees si longues qu'elles 
en perdent tout sens physique. Ainsi. en tant que ``g\'en\'eralisation de 
port\'ee limit\'ee'' (et non principe digne de ce nom) le second principe 
pourrait s'\'enoncer: 
\medskip
{\sl Pour tout \'etat microscopique initial du syst\`eme sauf un nombre infime,
et pendant des dur\'ees courtes devant $10^N$ ($N$ \'etant le nombre de 
mol\'ecules), l'entropie du syst\`eme ne s'\'ecarte pas notablement d'une 
fonction croissante. }
\medskip
Voir plus haut les commentaires \`a propos de [11]. 
\medskip
Cela dit, le fait de juger cet \'enonc\'e comme trop r\'eduit ou trop 
limit\'e pour m\'eriter le nom de principe est une affaire de convention. 
Car les dur\'ees (non courtes devant $10^N$) pour lesquelles il ne 
s'applique pas n'ont aucune existence pratique, et les fluctuations 
qui produisent les oscillations non monotones de l'entropie sont bien plus 
petites que ce qu'on a l'habitude, dans les th\'eories physiques, de 
consid\'erer comme nul. 
\medskip
Les \'etats ``{\it hautement organis\'es}, au comportement apparemment 
finalis\'e'' ont une probabilit\'e si inconcevablement petite de se produire 
spontan\'ement qu'ils ne se produisent jamais (\'Emile Borel), et la seule 
possibilit\'e de les rencontrer en Physique serait de les pr\'eparer. Pour 
avoir un principe {\it physique} et non un principe pour purs 
math\'ematiciens, cens\'e s'appliquer dans le ciel des 
id\'ees, il suffit de dire qu'on ne peut pas pr\'eparer de tels \'etats 
et d'inclure cette impossibilit\'e dans l'\'enonc\'e du principe. Cela 
ne le fait pas tomber d'un pi\'edestal, mais a au contraire l'avantage 
d'en d\'egager le v\'eritable sens, celui d'une propri\'et\'e de la nature 
et non d'une vision de l'esprit. 
\medskip
Pages 211 -- 213, est pr\'esent\'ee l'``{interpr\'etation subjectiviste}''\
de l'irr\'e\-ver\-si\-bi\-lit\'e [que les auteurs vont ensuite critiquer]: 
\medskip
{\cit  M\'elangeons [proposait Gibbs], une goutte d'encre noire \`a de
l'eau pure. Bient\^ot l'eau devient grise en une \'evolution qui, pour nous, 
est l'irr\'eversibilit\'e m\^eme; cependant, pour l'observateur aux sens 
assez aigus pour observer non pas le liquide macroscopique mais chacune 
des mol\'ecules qui constituent la population, le liquide ne deviendra 
jamais gris; l'observateur pourra suivre les trajectoires de plus en plus 
d\'elocalis\'ees des ``{mol\'ecules d'encre}'' d'abord rassembl\'ees
dans une petite r\'egion du syst\`eme, mais l'id\'ee que le milieu 
d'h\'et\'erog\`ene est irr\'eversiblement devenu homog\`ene, que l'eau 
est ``{devenue grise}'' sera, de son point de vue, une illusion
d\'etermin\'ee par la grossi\`eret\'e de nos moyens d'observation, une 
illusion subjective. Lui-m\^eme n'a vu que des mouvements, r\'eversibles, 
et ne voit rien de gris, mais du ``{noir}'' {\it et} du ``{blanc}'' .
( . . .  ) Selon cette interpr\'etation, la croissance de l'entropie ne
d\'ecrit pas le syst\`eme lui-m\^eme, mais seulement notre connaissance 
du syst\`eme. Ce qui ne cesse de cro{\^\i}tre c'est l'ignorance o\`u nous 
sommes de l'\'etat o\`u se trouve le syst\`eme, de la r\'egion de l'espace 
des phases o\`u le point qui le repr\'esente a des chances de se trouver. 
\`A l'instant initial, nous pouvons avoir beaucoup d'informations sur un 
syst\`eme, et le localiser assez pr\'ecis\'ement dans une r\'egion 
restreinte de l'espace des phases, mais, \`a mesure que le temps passe, 
les points compatibles avec les conditions initiales pourront donner 
naissance \`a des trajectoires qui s'\'eloignent de plus en plus de la 
r\'egion de d\'epart. L'information li\'ee \`a la pr\'eparation initiale 
perd ainsi irr\'eversiblement sa pertinence jusqu'au stade ultime o\`u 
on ne conna{\^\i}t plus du syst\`eme que les grandeurs que l'\'evolution 
dynamique laisse invariantes. Le syst\`eme est alors \`a l'\'equilibre 
( . . .  ) La croissance de l'entropie repr\'esente donc la d\'egradation
de l'information disponible; le syst\`eme est initialement d'autant 
plus loin de l'\'equilibre que nous le connaissons mieux, que nous pouvons 
le d\'efinir plus pr\'ecis\'ement, le situer dans une r\'egion plus petite 
de l'espace des phases. 
\smallskip
Cette interpr\'etation subjectiviste de l'irr\'e\-ver\-si\-bi\-lit\'e comme 
croissance de l'ignorance (encore renforc\'ee par l'analogie ambig\"ue 
avec la th\'eorie de l'infor\-ma\-tion) fait de l'observateur le vrai 
responsable de l'asym\'etrie temporelle qui caract\'erise le devenir 
du syst\`eme. Puisque l'observateur ne peut embrasser d'un seul coup 
d'oeil les positions et les vitesses des particules qui constituent
un syst\`eme complexe, il n'a pas acc\`es \`a la v\'erit\'e fondamentale 
de ce syst\`eme: il ne peut conna{\^\i}tre l'\'etat instantan\'e qui en 
contient \`a la fois le pass\'e et le futur, il ne peut saisir la loi 
r\'eversible qui, d'instant en instant, lui permettrait d'en d\'eployer 
l'\'evolution. Et il ne peut pas non plus manipuler le syst\`eme comme le 
fait le d\'emon de Maxwell, capable de s\'eparer les particules rapides 
et les particules lentes, et d'imposer ainsi \`a un syst\`eme une 
\'evolution antithermodynamique vers une distribution de temp\'erature 
de moins en moins uniforme. 
\smallskip
La thermodynamique est certes la science des syst\`emes complexes, 
mais, selon cette interpr\'etation, la seule sp\'ecificit\'e des 
syst\`emes complexes, c'est que la connaissance qu'on a d'eux est 
toujours approximative et que l'incertitude d\'etermin\'ee par cette 
approximation va croissant au cours du temps. ( . . .  )
\smallskip
Cependant, l'objection est imm\'ediate: dans ce cas, la thermodynamique 
devrait \^etre aussi universelle que notre ignorance. C'est l\`a la pierre 
d'achoppement de l'ensemble des interpr\'etations ``{simples}'' de
l'entropie, en termes d'incertitude sur les conditions initiales ou sur 
les conditions aux limites. Car, l'irr\'eversibilit\'e {\it n'est pas 
une propri\'et\'e universelle}; articuler dynamique et thermodynamique 
n\'ecessite donc la d\'efinition d'un crit\`ere {\it physique} de 
diff\'erentiation entre les syst\`emes, selon qu'ils peuvent ou non 
\^etre d\'ecrits thermodynamiquement, n\'ecessite une d\'efinition de 
la complexit\'e en termes physiques et non en termes de manque de 
connaissance. \par }
\medskip
\`A partir de l\`a les auteurs insistent sur le caract\`ere objectif 
de l'irr\'e\-ver\-si\-bi\-lit\'e ou plut\^ot de la complexit\'e (page 213 
et apr\`es): le comportement des corps macroscopiques est bien r\'eel et 
physique, la  complexit\'e est une qualit\'e r\'eelle et physique qui 
d\'ecidera si le corps aura un comportement thermodynamique ou un 
mouvement m\'ecanique, etc. Ils ont bien raison, mais cela nous 
\'eloignerait de notre sujet.  

\vskip8mm plus3mm minus3mm

\centerline{\hbox to 40mm{\hrulefill}}

\vskip12mm plus3mm minus3mm

\centerline{\bf R\'EF\'ERENCES.} 

\vskip8mm plus3mm minus3mm

[1] {\bf Ludwig Boltzmann} \hskip4mm {\it Weitere Studien \"uber 
W\"armegleichgewicht unter Gasmolek\"ulen}.  Wiener Berichte {\bf 66} 
({\oldstyle 1872}), p. 275.
\medskip
\filbreak

[2]   {\bf James Clerk Maxwell} \hskip4mm {\it Illustrations of the 
Dynamical Theory of Gases.} {Phil. Mag.} {\bf 19} ({\oldstyle 1860}), pp. 19.
\medskip
\filbreak

[3] {\bf Joseph Loschmidt} \hskip4mm {\it \"Uber das 
W\"arme\-gleich\-gewicht eines Systems von K\"orpern mit R\"ucksicht 
auf die Schwere.}  Wiener Berichte  {\bf 73}, 
({\oldstyle 1876}), pp. 139.
\medskip
\filbreak

[4] {\bf Frederik Reif} \hskip4mm {\it Cours de Physique de Berkeley: 
tome 5, Physique statistique.} Armand Colin, Paris ({\oldstyle 1972}),
pour l'\'edition fran\c{c}aise. 
\medskip
\filbreak

[5] {\bf Ludwig Boltzmann} \hskip4mm {\it Le\c{c}ons sur la th\'eorie 
des gaz.} R\'e\'edition Jacques Gabay, Paris, ({\oldstyle 1987}).
\medskip
\filbreak

[6] {\bf Rudolf Peierls} \hskip4mm {\it Surprises in Theoretical Physics.} 
Princeton University Press (coll. Princeton series in Physics), 
Princeton, New Jersey ({\oldstyle 1979}).
\medskip
\filbreak

[7] {\bf James Clerk Maxwell} \hskip4mm {\it Theory of Heat.} 
Longmans {\&} Green, London ({\oldstyle 1871}).
\medskip
\filbreak

[8] {\bf Henri Poincar\'e} \hskip4mm {\it Sur le probl\`eme des trois 
corps.} Revue g\'en\'erale des Sciences pures et appliqu\'ees II, 
vol {\bf 8} (15 janvier {\oldstyle 1891}), page 529.
\medskip
\filbreak

[9] {\bf Henri Poincar\'e} \hskip4mm {\it Sur les tentatives d'explication 
m\'ecanique des principes de la Thermodynamique.}  Comptes-rendus de 
l'Acad\'emie des Sciences, vol {\bf 108} (18 mars {\oldstyle 1889}),
pages 550 -- 553. 
\medskip
\filbreak

[10] {\bf Henri Poincar\'e} \hskip4mm {\it Sur la th\'eorie cin\'etique 
des gaz.} Revue g\'en\'erale des Sciences pures et appliqu\'ees, 
vol {\bf 5} ({\oldstyle 1894}), pages 513 -- 521.
\medskip
\filbreak

[11] {\bf Ernst Zermelo} \hskip4mm {\it\"Uber einen Satz der Dynamik und
die mechanische W\"armetheorie}  Wiedemanns Annalen, vol {\bf 57} p.485 
et vol {\bf 59} p.793 ({\oldstyle 1896}).
\medskip
\filbreak 

[12] {\bf Ludwig Boltzmann} \hskip4mm {\it  Entgegnung auf die 
W\"armetheoretischen Betrachtungen des Hrn. E. Zermelo.}  Wiedemanns 
Annalen, vol {\bf 57} ({\oldstyle 1896}), pages 773 -- 784.
\medskip
\filbreak

[13] {\bf Ludwig Boltzmann} \hskip4mm {\it  Zu  Hrn. Zermelos Abhandlung 
``\"Uber die mechanische Erkl\"arung irreversibler Vorg\"ange''.}  Wiedemanns 
Annalen, vol {\bf 60} ({\oldstyle 1897}), pages 392 -- 398.
\medskip
\filbreak

[14] {\bf Ilya Prigogine, Isabelle Stengers} \hskip4mm {\it La nouvelle 
alliance.} NRF Gallimard, Paris ({\oldstyle 1979}).



\end
