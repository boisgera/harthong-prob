\input /home/harthong/tex/formats/twelvea4.tex
\input /home/harthong/tex/formats/epsf.tex

\auteurcourant={\sl J. Harthong: probabilit\'es et statistique}
\titrecourant={\sl Variables al\'eatoires}

\pageno=130

\newdimen\blocksize  \blocksize=\vsize \advance\blocksize by -8pt
\def\struta{\vrule depth1.8pt width0pt}
\def\strutb{\vrule height7.5pt width0pt}
\def\ata{\hskip-2.5pt}
\def\aub{\hskip1pt}

\null\vskip10mm plus5mm minus4mm

\centerline{\tit VI.  VARIABLES AL\'EATOIRES.}

\vskip10mm plus3mm minus2mm

{\bf VI.\aub 1. Le concept de variable al\'eatoire.}
\medskip
Dans les livres sur le Calcul des probabilit\'es, on d\'efinit usuellement 
une variable al\'eatoire en disant que c'est une application d\'efinie sur
l'espace $\Omega$ des \'epreuves, et \`a valeurs r\'eelles.
\medskip
Ceci est une d\'efinition math\'ematique; l'id\'ee sous-jacente est celle 
d'une grandeur qui prend des valeurs ``au hasard''. Voyons cela sur des
exemples. Imaginons une exp\'erience de physique avec un d\'etecteur, 
par exemple on veut d\'etecter les impacts d'\'electrons sur un \'ecran
fluorescent. La position de l'\'electron sur l'\'ecran est impr\'evisible 
(peu importe que ce soit pour des raisons quantiques ou thermiques ou
autres). On peut rep\'erer cette position par deux coordonn\'ees $x$ et
$y$. Mais ni $x$, ni $y$, n'ont une valeur d\'etermin\'ee. Elles peuvent
prendre n'importe quelle valeur (entre les limites de l'\'ecran), et d'un
impact \`a l'autre ces valeurs varient au hasard: elles sont variables et
al\'eatoires. C'est pourquoi on dit que $x$ et $y$ sont des {\it variables
al\'eatoires}. Il s'agit maintenant de comprendre le rapport entre cette
id\'ee et la d\'efinition math\'ematique. 
\medskip
Au chapitre $I$ nous avons vu que la structure de l'espace des \'epreuves
$\Omega$ contenait implicitement  toute l'information sur la mani\`ere 
dont intervient le hasard pur dans le probl\`eme consid\'er\'e. On parvient
en quelque sorte au {\it hasard pur} lorsqu'on a identifi\'e le niveau 
auquel se situent les choix \'equiprobables (nous avions insist\'e alors
sur l'analogie entre le hasard {\it pur} et les rep\`eres galil\'eens). Ainsi,
dans l'exemple des trois boules qui remplissent deux cases (figure 1), on
a vu qu'il y a  huit distributions diff\'erentes, dont l'\'equiprobabilit\'e
refl\`ete les sym\'etries spatio-temporelles: chaque boule a une chance 
sur deux exactement de choisir l'une ou l'autre case (sym\'etrie spatiale)
et l'exp\'erience avec une boule est rigoureusement reproductible
(invariance temporelle), de sorte que la deuxi\`eme ou la troisi\`eme
boule n'est pas influenc\'ee par ce qui est arriv\'e aux pr\'ec\'edentes. Si
les deux cases ont la m\^eme probabilit\'e $1 \over 2$ d'\^etre choisies,
on consid\`ere que c'est le hasard pur qui a pr\'esid\'e (si l'on peut dire)
au choix; si l'une des deux cases avait plus de chances que l'autre, on
penserait qu'une cause, peut-\^etre inconnue, modifie les choix purs du
hasard. Comme nous l'avons vu \`a la fin du chapitre {\bf I}, quand il y a
\'equiprobabilit\'e quelque  part, la qu\^ete d'une cause s'arr\^ete, car on
a alors atteint en quelque sorte le niveau de connaissance maximale sur 
la partie compr\'ehensible du ph\'enom\`ene. Cela est tout
particuli\`erement convaincant dans le cas des quantons\ftn
1{particules soumises \`a la M\'ecanique quantique; cf. l'ouvrage de J. M.
LEVY-LEBLOND et F. BALIBAR, {\it Quantique, rudiments}.}: la propri\'et\'e
de non-s\'eparabilit\'e quantique fait que pour des bosons par exemple il
n'y a pas huit distributions,  mais quatre modes d'occupation
\'equiprobables. Une cons\'equence de cela est que (si on lance les
particules l'une apr\`es l'autre) le deuxi\`eme boson a une probabilit\'e
$2 \over 3$ d'aller dans l'\'etat d\'ej\`a occup\'e par le premier et  une
probabilit\'e $1 \over 3$ d'aller dans l'\'etat encore vide. On dit alors
qu'il y a une cause qui influence le second boson afin qu'il pr\'ef\`ere
aller dans l'\'etat d\'ej\`a occup\'e et on donne des noms \`a cette cause:
non-s\'eparabilit\'e, condensation de Bose, etc. Mais cela signifie
simplement que le hasard pur n'intervient pas au niveau du choix,
s\'epar\'e pour chaque quanton, de l'\'etat qu'il va occuper. Si on veut
chercher \`a quel niveau intervient le hasard pur, il faut chercher ce qui
est \'equiprobable et on trouve que ce sont les modes d'occupation (figure
1, colonne de droite). 
\medskip  
Dans l'exp\'erience avec un d\'etecteur, nous mesurons les coordonn\'ees
$x$ et $y$ des impacts d'\'electrons sur un \'ecran; si nous constatons 
une certaine r\'epartition, qui favorise certains points et en d\'efavorise
d'autres (par exemple si les impacts s'accumulent le long de certaines
lignes et forment ainsi une figure d'interf\'erence), nous dirons que le
hasard pur n'agit pas directement au niveau du choix du point sur
l'\'ecran, et qu'il y a une cause qui favorise les cr\^etes des franges
d'interf\'erence. L'espace $\Omega$ devra \^etre recherch\'e ``en amont''.
Par contre si les impacts se r\'epartissent uniform\'ement sur l'\'ecran,
on pourra prendre pour $\Omega$ l'ensemble des pixels de l'\'ecran. Dans
le cas o\`u $\Omega$ devra \^etre  recherch\'e en amont, chaque
\'el\'ement de $\Omega$ devra \'evidemment correspondre \`a un
r\'esultat possible; autrement dit, \`a {\it chaque} \'epreuve $\omega$,
\'el\'ement de $\Omega$, doit correspondre un impact {\it d\'etermin\'e}
sur l'\'ecran. Ici, attention! Le mot {\it d\'etermin\'e} ne doit  pas \^etre
compris comme signifiant qu'il y a du d\'eterminisme. Pour chaque
\'electron, il y a un choix {\it non d\'eterministe}, en ce sens que s'il y a
eu un d\'eterminisme quelconque dans ce choix, il a \'et\'e effac\'e
par un brouillage; mais le choix ``par le pur hasard'' ne se situe pas au
niveau de la position sur l'\'ecran, puisqu'il n'y a pas \'equiprobabilit\'e
\`a ce niveau, et que donc l'action du hasard est m\'elang\'ee avec celle
d'une cause qui favorise les cr\^etes de franges. L'action du pur hasard se
situe ``en amont'', \`a un niveau o\`u il y a une \'equiprobabilit\'e des
choix, un niveau qu'il s'agit de d\'ecouvrir et qui sera celui de l'espace
des \'epreuves $\Omega$. Alors, pour chaque choix non-d\'eterministe
fait par le hasard {\it \`a ce niveau}, il correspondra un impact {\it
d\'etermin\'e} sur l'\'ecran, de coordonn\'ees $x$ et $y$. Ainsi, $x$ et $y$
sont des fonctions de ce choix.   
\medskip  
D'o\`u la d\'efinition math\'ematique ``une variable al\'eatoire est une 
fonction d\'efinie sur l'espace des \'epreuves, \`a valeurs r\'eelles''. Cela
signifie tout simplement qu'\`a chaque choix \'equiprobable $\omega$ 
(une \'epreuve, un \'el\'ement de $\Omega$), la fonction fait correspondre
un nombre (par exemple la coordonn\'ee de l'impact sur l'\'ecran. Le 
hasard pur choisit aveugl\'ement, au  niveau o\`u il intervient, une
\'epreuve parmi toutes les \'epreuves \'equiprobables possibles, sans en
favoriser aucune, et on observe une valeur num\'erique $x$, qui est  un
effet de ce choix, mais qui n'a aucune raison d'\^etre r\'epartie
uniform\'ement sur l'ensemble des valeurs possibles. C'est cette valeur
num\'erique qui constitue la variable al\'eatoire. 

\bigskip

{\bf VI.\aub 2. \vtop{\hsize=9cm  
\hbox{La loi d'une variable al\'eatoire. Esp\'erance, variance.} 
\hbox{Fonction g\'en\'eratrice. Fonction caract\'eristique.} } }  
\medskip
On peut donc r\'esumer ce qui pr\'ec\`ede de la fa\c{c}on suivante: une 
variable al\'eatoire prend une valeur num\'erique d\'etermin\'ee pour
chaque \'epreuve $\omega$ (en langage math\'ematique: c'est une 
fonction de $\Omega$ dans \R ). Ce n'est pas la valeur num\'erique prise
par la variable al\'eatoire qui est ``choisie au hasard'', mais l'\'epreuve
$\omega$;  le choix de $\omega$ parmi toutes les \'epreuves
\'equiprobables d\'etermine alors la valeur que prendra la variable
al\'eatoire. Il est clair que les valeurs num\'eriques prises par la 
variable al\'eatoire ne sont pas, sauf exception, \'equiprobables. Par
cons\'equent chacune des valeurs num\'eriques possibles a une certaine
probabilit\'e.  On appelle alors la donn\'ee de ces probabilit\'es la {\bf 
loi de la variable al\'eatoire}. En termes plus math\'ematiques: 
\medskip
{\narrower Soit $X$ une variable al\'eatoire. L'ensemble $X(\Omega )$ 
de toutes les valeurs prises par $X$ est n\'ecessairement fini (et de
cardinal au plus \'egal \`a celui de $\Omega$), appelons $x_1, \, x_2, \,
x_3, \ldots x_N$ ses \'el\'ements. Pour chaque $x_k$ soit $A_k$
l'ensemble des  \'epreuves $\omega$ telles que $X(\omega ) = x_k$ ($A_k
= \{ \omega  \mid X(\omega ) = x_k \}$). $A_k$ est donc un \'ev\'enement
qu'on peut d\'esigner dans le langage imag\'e par ``la variable al\'eatoire
$X$ prend  la valeur $x_k$''. Chaque $A_k$ a une probabilit\'e $p_k = \#
A_k / \#\Omega$. La loi de $X$ est la donn\'ee des $x_k$ et des $p_k$:
on la notera $\{ x_k\,  ,\, (p_k)\}$.}
\medskip 
{\bf Exemple a.}\hskip5pt Des cordes \'etant distribu\'ees au hasard sur 
un cercle  de rayon $R$ (selon l'un des trois proc\'ed\'es que nous avons
vu au chapitre $I$), la  {\it longueur}  d'une corde est une variable
al\'eatoire. Appelons-la  $\ell$; l'\'ev\'enement $A$ dont nous avions
calcul\'e la probabilit\'e au chapitre $I$ \'etait ``$\ell > R\sqrt{3}$''.  
Une
autre variable al\'eatoire est $d$, la distance au centre du cercle.
L'\'ev\'enement $A$ est aussi ``$d < {1 \over 2}R$''. Pour fixer les id\'ees,
mettons que les cordes sont distribu\'ees selon le cas 2 (cf. figure 2), et
que leur distribution est discr\'etis\'ee en 360 degr\'es d'angle pour
rep\'erer (\`a 1 degr\'e pr\`es, donc) la direction de la corde, et que le
rayon $R = 1\, m.$ est divis\'e en $100\, cm$ (ainsi $\#\Omega = 36000$).
Alors la variable al\'eatoire $d$ prend les valeurs num\'eriques (en
$cm.$): $x_0 = 0, \,  x_1 = 1, \, x_2 = 2, \, x_3 = 3, \ldots  x_{100} =
100$, avec les probabilit\'es $p_0 = 0.01, \, p_1 = 0.01, \, p_2 = 0.01, \,
p_3 = 0.01, \ldots p_{99} = 0.01$ (elles sont toutes \'egales, on peut dire
que la loi est {\it uniforme}).  
\smallskip
Si on s'int\'eresse \`a la variable $\ell$, on discr\'etisera les valeurs 
prises par la longueur. Celle-ci variant entre $0\, cm$ et $200\, cm$, il 
est naturel de discr\'etiser en $cm.$ On trouve alors que la probabilit\'e
pour que  $\ell = k\, cm.$ (ou de fa\c{c}on plus pr\'ecise  $k\, cm. \leq 
\ell  < k+1 \, cm.$) est $k\; / \; 2\sqrt{40000 - k^2}$ (cette fois la loi
n'est plus uniforme). 
\medskip
{\bf Exemple b.}\hskip5pt On consid\`ere les marches al\'eatoires \`a 
$2n$ pas. L'abscisse $X$ atteinte  par la marche au $2n^{\rm e}$ pas est
une variable al\'eatoire  qui prend les valeurs $-2n, \, -2n+2, \, -2n+4,
\ldots +2n-4, \, +2n-2, \, +2n$ avec les probabilit\'es suivantes (cf.
chapitre $III\;$): 
$$\eqalign{
{\cal P}\, (X=-2n) &= 2^{-2n} \cr
{\cal P}\, (X=-2n+2) &= 2^{-2n}{2n \choose 1} \cr
{\cal P}\, (X=-2n+4) &= 2^{-2n}{2n \choose 2} \cr
\noalign{\smallskip}
&\ldots \cr
\noalign{\smallskip}
{\cal P}\, (X=-2n+2k) &= 2^{-2n}{2n \choose k}  \cr
\noalign{\smallskip}
&\ldots \cr
\noalign{\smallskip}
{\cal P}\, (X=2n-4) &= 2^{-2n}{2n \choose 2}  \cr
{\cal P}\, (X=2n-2) &= 2^{-2n}{2n \choose 1}  \cr
{\cal P}\, (X=2n) &= 2^{-2n}  \cr }$$
Nous avons vu au chapitre {\bf II} que les coefficients bin\^omiaux 
${2n \choose k}$ atteignaient leur maximum pour $k = n$, que ce 
maximum, pour $n$ grand, pouvait \^etre approch\'e par $2^{2n} /
\sqrt{\pi n}$, et que au voisinage du maximum on avait (cf. $II.6.$)
$${2n \choose n + j} \simeq {2n \choose n}\cdot
\e^{-{\; j^2 \struta\over n}} \simeq { 2^{2n} \over \sqrt{\pi n}}
\cdot  \e^{-{\;  j^2\struta \over n}}$$
On en d\'eduit que {\it si $n$ est grand}, ces probabilit\'es 
(c'est-\`a-dire la loi de $X$) sont 
$${\cal P}\, (X = 2j) \simeq { 1 \over \sqrt{\pi n}}\cdot 
\e^{-{\; j^2\struta \over n}}$$
Pour $\; j^2 \gg n\;$ ces probabilit\'es sont infinit\'esimales; elles ne
sont  notables que lorsque $j$ est du m\^eme ordre de grandeur que
$\sqrt{n}$ ou plus petit. Pour se faire une id\'ee plus concr\`ete, soit
$n=1000$; alors  $\sqrt{n} \simeq 32$; si $j > 100$, $\e^{-j^2/n} <
\e^{-10} \simeq 0.0000454$, si $j > 150$, $\e^{-j^2/n} < \e^{-22.5} \simeq
1.7 \cdot 10^{-10}$. On voit qu'il est pratiquement impossible que $X$
prenne  des valeurs sup\'erieures \`a  $300$ ou inf\'erieures \`a $-300$.
La probabilit\'e pour que $X = -2000$ ou $X = +2000$ est $2^{-2000}
\simeq 10^{-602}$, alors que la probabilit\'e pour que $X=0$ est
$2^{-2000} \cdot {2000 \choose 1000} \simeq 1/\sqrt{1000\pi}
\simeq 1/56$. Ainsi, alors que toutes les marches possibles sont
\'equiprobables, on observe une \'enorme disparit\'e pour les
probabilit\'es des valeurs prises par la variable $X$. Sur l'ensemble
$\Omega$ de tous les parcours possibles, le hasard n'en favorise aucun; 
mais pr\'ecis\'ement de ce fait il en r\'esulte, {\it uniquement par
l'interm\'ediaire des m\'ecanismes de combinaisons}, que le hasard
favorise \'enorm\'ement les valeurs de $X$ comprises entre $-300$ et
$+300$, au d\'etriment des valeurs extr\^emes. 
\medskip
Si on avait pris $n=1\, 000\, 000$, ce ph\'enom\`ene serait encore plus
marqu\'e; en effet, on aurait alors ${\cal P}\, (X=\pm 2\, 000\, 000) =
2^{-2\, 000\, 000} \simeq 10^{-602\, 060}$, et ${\cal P}\, (X=0) \simeq
1/1772$. La probabilit\'e pour que $X$ soit compris entre $-5000$ et 
$+5000$ est alors sup\'erieure \`a $0.9996$. Si on compte en valeur
relative, c'est-\`a-dire \`a l'\'echelle de $n$, en posant $x = X/n$, cela
signifie que la probabilit\'e pour que $x$ soit compris entre $-0.005$ 
et $+0.005$ est sup\'erieure \`a $0.9996$. Pour $n=1\, 000\, 000\,
000$, $x$ serait compris entre  $-0.0002$ et $+0.0002$ avec une
probabilit\'e sup\'erieure \`a $0.9999$. Autrement dit, {\it par le seul 
m\'ecanisme des combinaisons}, un choix al\'eatoire entre des parcours
\'equiprobables se traduit, lorsque  le nombre de pas $n$ est tr\`es
grand, par un quasi-d\'eterminisme au niveau de l'ordonn\'ee atteinte:
on est pratiquement certain que celle-ci est proche  de 0. En g\'en\'eral,
la loi est la suivante:
\medskip
{\narrower La probabilit\'e pour que $X/\sqrt{n}$ s'\'ecarte de 0 de plus 
de  5 est inf\'erieure \`a $0.0005$; ou encore: la probabilit\'e pour que
$x = X/n$ s'\'ecarte de 0 de plus de  $5/\sqrt{n}$ est inf\'erieure \`a
$0.0005$. \par }
\medskip
On est donc pratiquement certain que $x$ sera dans un intervalle de
largeur $10/\sqrt{n}$ autour de 0. La largeur de cet intervalle diminue 
proportionnellement \`a $1/\sqrt{n}$ quand $n$ augmente, elle devient
donc minuscule quand $n$ est tr\`es grand. Le facteur 10 au 
num\'erateur de la fourchette $10/\sqrt{n}$ est li\'e au degr\'e de
certitude (d\'etermin\'e ici par la probabillit\'e $0.0005$); si on prend
une fourchette  moins large, le degr\'e de certitude baisse: ainsi la
probabilit\'e pour que $x$ s'\'ecarte de 0 de plus de $4/\sqrt{n}$ est
$0.0046$; pour  $3/\sqrt{n}$ elle est de $0.034$, pour $2/\sqrt{n}$ elle
est de $0.157$, et pour  $1/\sqrt{n}$ elle est de $0.48$. On voit donc
que si la fourchette est cinq fois plus \'etroite, il n'y a pratiquement
plus de certitude. Tout cela r\'esulte de la variation en $\e^{-t^2}$:  cette
fonction garde une valeur notablement diff\'erente de z\'ero tant que
$t$ est inf\'erieur \`a 2 environ; pour $t > 3$, elle devient tr\`es vite
extr\^emement petite.
\medskip
La propri\'et\'e qui se manifeste ainsi est une propri\'et\'e g\'en\'erale 
du hasard, que nous voyons ici \`a l'oeuvre dans un cas particulier. Cette
propri\'et\'e est {\bf la loi des grands nombres}; elle sera \'etudi\'ee 
sous sa forme g\'en\'erale au chapitre suivant. La loi des grands
nombres a pour effet de transformer le hasard en d\'eterminisme.
L'exemple ci-dessus de la variable al\'eatoire $x = X/n$ le montre. Pour
de petites valeurs de  $n$, par exemple $n=2$ ou $n=5$, la variable $x$
est r\'eellement {\it al\'eatoire}, c'est-\`a-dire que les diff\'erentes
valeurs qu'elle peut prendre ($-2\,$, $-1\,$, $0\,$, $+1\,$, $+2\,$ si 
$n=2$ et $-2\,$, $-1.6\,$, $-1.2\,$,  $-0.8\,$, $-0.4\,$, $0\,$, $+0.4\,$,
$+0.8\,$, $+1.2\,$, $+1.6\,$, $+2\,$ pour $n=5$) sont impr\'evisibles et 
ont seulement une certaine probabilit\'e. Mais si $n$ est tr\`es grand,
et qu'on mesure les valeurs prises par la variable avec une pr\'ecision
de l'ordre de  $5\over \sqrt{n}$, alors la variable $x$ prend la valeur 0
avec certitude. Dans un tel processus, le hasard n'a pas cess\'e d'agir:
c'est bien toujours ``le hasard pur'' qui effectue pour la marche
al\'eatoire des choix \'equiprobables {\it sur l'ensemble de tous les
parcours possibles}; mais \`a un autre niveau d'observation, qui est
celui des valeurs prises par la variable $x$, cela se traduit par du
d\'eterminisme.  
\medskip 
Cette loi des grands nombres par laquelle le hasard se transforme en
d\'eterminisme est en quelque sorte l'inverse  du chaos, qui transforme
le d\'eterminisme en hasard. Ce ph\'enom\`ene n'est pas diff\'erent de
celui que nous avons plusieurs  fois d\'ej\`a soulign\'e, \`a savoir que
des choix \'equiprobables du hasard \`a un certain niveau se traduisent
par des r\'esultats non \'equiprobables \`a un autre niveau (que par
exemple un choix \'equiprobable des deux extr\'emit\'es d'une corde sur
un cercle se traduisait par une r\'epartition non \'equiprobable de la
distance au centre); la loi des grands nombres est un effet de ce type,
mais amplifi\'e au point que la non-\'equiprobabilit\'e devient extr\^eme.
\medskip
Nous avions \'etudi\'e \`a la fin du chapitre {\bf II} un exemple typique 
o\`u intervient cette loi des grands nombres: la loi de Planck. En termes 
de variables al\'eatoires, on peut reformuler le probl\`eme \'etudi\'e 
alors (cf. {\bf II.6.}) en disant que les {\it nombres d'occupation} $n_i$ 
de   chaque intervalle d'\'energie sont des variables al\'eatoires. La loi de
Planck r\'esultait de la loi des grands nombres, car la probabilit\'e pour
que les $n_i$ soient proches de $m_i \bigl/ \bigl(\exp(\hbar\omega_i /
kT)  - 1\bigr)$ est pratiquement \'egale \`a 1 pour la m\^eme raison que 
ci-dessus pour la variable al\'eatoire $x=X/n$. Lorsqu'on constate
exp\'erimentalement la distribution selon les fr\'equences du 
rayonnement \'emis par un corps chauff\'e, la loi de Planck appara{\^\i}t
comme une loi d\'eterministe. En r\'ealit\'e elle est un effet du hasard.
\medskip
La Physique a montr\'e que le monde macroscopique est
enti\`erement gouvern\'e par la loi des grands nombres\ftn2{Pour en 
savoir plus sur le  r\^ole jou\'e par la loi des grands nombres dans le
comportement des corps macroscopiques, on pourra consulter : {\it Cours
de Physique de Berkeley}, vol. {\bf 5}, {\it Physique statistique} (Armand
Colin, \'ed., coll. U), chapitre 1: propri\'et\'es caract\'eristiques des
syst\`emes macroscopiques.}. Le monde des atomes et des particules est
gouvern\'e par le hasard, qui effectue des choix \'equiprobables sur
l'ensemble de tous les \'etats quantiques possibles (par exemple sur
l'ensemble des modes d'occupations en statistique de  Bose-Einstein). 
Mais au niveau des observations macroscopiques, cela se traduit par le
d\'eterminisme des lois de la thermodynamique ou de  la m\'ecanique
classique. Il n'y a pas d'autre secret dans cette transformation du hasard 
en d\'eterminisme, que la loi des grands nombres.
\medskip
{\bf Exemple c.}\hskip5pt On consid\`ere toujours les marches
al\'eatoires, et on  appelle $T$ l'instant du premier retour \`a $0$. Nous
avons vu au  chapitre $III$ que 
$${\cal P}\, (T=k) = \cases{
2^{-2\ell}{2\ell \choose \ell}{1 \over 2\ell - 1} & si $k = 2\ell$ et
$\ell\neq 0$ \cr     \noalign{\medskip}
0 & si $k$ est impair ou nul \cr }$$
Ainsi $T$ est une variable al\'eatoire qui prend les valeurs enti\`eres 
paires, avec les probabilit\'es indiqu\'ees ci-dessus. Mais si on 
consid\`ere les marches al\'eatoires \`a $2n$ pas, il peut arriver aussi
que le retour \`a  z\'ero ne se produise jamais; la variable $T$ peut donc
prendre aussi  la valeur ``jamais'', qui n'est \'evidemment pas num\'erique
si on prend la d\'efinition au pied de la lettre. On peut montrer que la
probabilit\'e de  cette valeur ``jamais'' est 
$$\sum_{j = n+1}^{\infty} 2^{-2j}{2j \choose j} {1 \over 2j-1} \simeq 
{1 \over \sqrt {\pi n}}$$
Ainsi la variable $T$ prend les valeurs $2,\, 4,\, 6, \ldots 2n-2,\, 2n, \,
\hbox{et ``jamais''}$ avec les probabilit\'es respectives donn\'ees
ci-dessus. 
\smallskip
{\eightpoint {\bf Remarque:} En Calcul des probabilit\'es on dit qu'une
variable al\'eatoire qui peut ainsi n'avoir aucune valeur d\'efinie pour
certaines \'epreuves est {\it incompl\`ete}. Toutefois on voit que si $n$ 
tend vers l'infini, la probabilit\'e de la valeur ``jamais'' tend vers 0.  
Ceci
donne envie de consid\'erer parfois des espaces $\Omega$ infinis. Par
exemple dans le cas pr\'esent, on aurait pour $\Omega$ l'ensemble de
toutes les marches de dur\'ee infinie (donc sans limitation du nombre de
pas). C'est la raison pour laquelle beaucoup (et m\^eme la plupart) des
livres sur le Calcul des probabilit\'es pr\'esentent une th\'eorie pour des
espaces $\Omega$ infinis. Le prix \`a payer pour cela est toutefois
exorbitant: la plupart des propri\'et\'es que nous avons rencontr\'ees dans
ce cours sont extr\^emement faciles \`a \'etablir tant que $\Omega$ est un
ensemble fini. Elles exigent des math\'ematiques tr\`es sophistiqu\'ees
d\`es lors que $\Omega$ est infini (int\'egrale de Lebesgue, analyse
fonctionnelle, etc. En outre, elles ne permettent pas de calculer davantage
de choses: comme cela a \'et\'e montr\'e dans un ouvrage
r\'ecent\ftn3{Edward NELSON {\it Radically Elementary Probability
Theory}, Princeton University Press, {\oldstyle 1989}.}, tout  ce qui peut
\^etre effectivement calcul\'e dans le cadre de ces th\'eories
sophistiqu\'ees peut aussi \^etre calcul\'e en traitant d'abord le probl\`eme
avec un espace $\Omega$ fini, et en faisant tendre {\it ensuite}, dans les
r\'esultats obtenus, le cardinal de $\Omega$ vers l'infini.}
\medskip
On peut faire la convention que ``jamais'' soit repr\'esent\'ee par la valeur
num\'erique $0$. Cette valeur ne risque pas, en effet, d'\^etre d\'ej\`a prise 
par la variable $T$, puisque l'instant z\'ero est celui du d\'epart et ne peut
donc \^etre celui d'un {\it retour}. Cela \'elimine l'in\'el\'egance d'une 
valeur non num\'erique sans recourir pour autant \`a l'infini. On peut
pratiquement toujours se d\'ebrouiller avec de tels exp\'edients.

\bigskip

\'Etant donn\'ee une variable al\'eatoire $X$ qui prend les valeurs $x_1$, 
$x_2$, $x_3$, $\ldots$ $x_{N-1}$, $x_N$ avec les probabilit\'es
respectives  $p_1, \,  p_2, \, p_3, \ldots p_{N-1}, \, p_N$, on appelle {\it
moyenne} ou {\it esp\'erance math\'ematique} de X la grandeur
$${\bf E} (X) = \sum_{k=1}^{N} x_k p_k   \eqno (VI.1.)$$ 
On appelle {\it variance} de $X$ la grandeur
$${\bf Var} (X) = \sum_{k=1}^{N} (x_k - m)^2 p_k   \eqno (VI.2.)$$ 
avec $m = {\bf E} (X)$.
En d\'eveloppant le carr\'e $(x_k - m)^2 = x_k^2 - 2mx_k + m^2$ on peut
\'ecrire aussi:
$$\eqalign{
{\bf Var}(X) &= \sum_{k=1}^N x_k^2 p_k - 2m\sum_{k=1}^N x_k p_k + 
m^2\sum_{k=1}^N p_k \cr
&= {\bf E}(X^2) - 2 m^2 + m^2 = {\bf E}(X^2) - m^2\cr } 
\eqno (VI.2\, a.)$$

La moyenne ne n\'ecessite aucune explication, elle rel\`eve de l'intuition
premi\`ere. La variance m\'erite par contre un commentaire. Comme on le 
voit dans la d\'efinition $VI.2$, c'est une somme de termes positifs (des
carr\'es); la variance est donc d'autant plus grande que les nombres $x_k -
m$ sont plus grands, mais la somme \'etant pond\'er\'ee par les
probabilit\'es  $p_k$, une grande valeur pour $x_k - m$ va peser d'autant
plus que la probabilit\'e correspondante, $p_k$, est plus grande. $x_k - m$
est en fait l'\'ecart de $x_k$ par rapport \`a la moyenne $m$. La variance
est la moyenne du carr\'e de l'\'ecart. Elle est d'autant plus grande que la
variable $X$ s'\'ecarte davantage de sa moyenne, et d'autant plus grande
aussi que ces \'ecarts se produisent avec une forte probabilit\'e.
Autrement dit, la variance est une estimation de la tendance de la variable
al\'eatoire \`a s'\'ecarter de la moyenne, \`a se disperser. On aurait pu
aussi mesurer cette tendance \`a  la dispersion par la moyenne de la valeur
absolue $\sum p_k\, | x_k - m |$ (il faut \'evidemment prendre la valeur
absolue, car si on calcule la moyenne de $x_k - m$ en valeur alg\'ebrique,
on trouve 0), ou par la moyenne de la puissance quatri\`eme $\sum p_k (x_k
- m)^4$. Compar\'e \`a la valeur absolue, le carr\'e fait payer plus cher les
grands \'ecarts. La puissance quatri\`eme les ferait payer encore plus cher.
Il n'y a pas d'estimation absolue ou universelle de la dispersion des valeurs
d'une variable al\'eatoire. Le choix du carr\'e est purement conventionnel et
s'il sert de r\'ef\'erence c'est simplement parce qu'on n'a pas trop \`a s'en
plaindre; entre la valeur absolue qui sous-estime les grands \'ecarts et la
puissance quatri\`eme qui les surestime, c'est peut-\^etre un bon
compromis; il se peut aussi que l'analogie avec le moment d'inertie ait
contribu\'e \`a en faire un {\it standard mondial}. Pour voir les choses plus
concr\`etement, on peut examiner un exemple tr\`es simple. 
\medskip 
Soient les variables al\'eatoires:
\smallskip
$X$, prenant les valeurs $-2, \, -1, \, 0, \, +1,  \, +2$ avec les 
probabilit\'es respectives ${1 \over 20}, \, {1 \over 10}, \, {7 \over 10}, 
\, {1 \over  10}, \, {1 \over 20}$
\smallskip
$Y$, prenant les valeurs $-2, \, -1, \, 0, \, +1,  \, +2$ avec les 
probabilit\'es respectives ${1 \over 5}, \, {1 \over 5}, \, {1 \over 5}, \, {1
\over  5}, \, {1 \over 5}$
\smallskip
$Z$, prenant les valeurs $-2, \, -1, \, 0, \, +1,  \, +2$ avec les 
probabilit\'es respectives ${4 \over 10}, \, {1 \over 10}, \, {0}, \, {1 \over 
10}, \, {4 \over 10}$.
\medskip
Pour les trois variables, la dispersion des valeurs est la m\^eme, mais la
probabilit\'e d'une grande dispersion est \'elev\'ee pour $Z$ et faible pour 
$X$. Leurs variances sont:
$${\bf Var} (X) = 0.6 \quad {\bf Var} (Y) = 2.0 \quad 
{\bf var} (Z) = 3.4$$
ce qui refl\`ete bien les diff\'erents types de dispersion (la variable $Z$ 
\`a forte dispersion a une variance nettement plus \'elev\'ee que la
variable $X$ \`a faible dispersion). Si on avait pris la moyenne des valeurs
absolues pour mesurer la dispersion on aurait obtenu les valeurs:
$$\hbox{pour} \; X \; : 0.4 \quad \hbox{pour} \; Y \; : 1.2 \quad 
\hbox{pour} \; Z \; : 1.8$$
Comme pr\'evu on voit que les trois cas sont moins nettement 
d\'epartag\'es (les grands \'ecarts ne sont pas renforc\'es). Enfin, si on
prend la moyenne des quatri\`emes puissances:
$$\hbox{pour} \; X \; : 0.9 \quad \hbox{pour} \; Y \; : 6.8 \quad 
\hbox{pour} \; Z \; : 13.0$$
On voit l'effet de la surestimation des grands \'ecarts.
La variance est une convention tr\`es r\'epandue, mais peut aussi devenir 
un pont-aux-\^anes; dans tel ou tel cas particulier on peut avoir de bonnes
raisons de pr\'ef\'erer estimer la dispersion par la quatri\`eme puissance 
que par la variance, et il faut donner la priorit\'e aux bonnes raisons sur 
le conformisme.
\medskip
Un d\'efaut de la variance est de ne pas mesurer l'\'ecart dans les m\^emes
unit\'es que la variable: si les $x_k$ sont des $cm.$ la variance sera en 
$cm^2$. C'est pourquoi on introduit une autre estimation de l'\'ecart qui est 
la racine carr\'ee de la variance ou {\it \'ecart-type}. Pour les variables
$X,\,  Y,\, Z$ les \'ecarts-types sont
$$\hbox{pour} \; X \; : 0.77 \quad \hbox{pour} \; Y \; : 1.41 \quad 
\hbox{pour} \; Z \; : 1.8$$
Pour l'estimation par les puissances quatri\`emes, on pourra \'egalement
prendre la racine quatri\`eme du r\'esultat, et on obtient ainsi une autre
estimation,  un {\og \'ecart-type d'ordre 4\fg}:
$$\hbox{pour} \; X \; : 0.97 \quad \hbox{pour} \; Y \; : 1.61 \quad 
\hbox{pour} \; Z \; : 1.90$$
On voit que la surestimation des grands \'ecarts est r\'eduite. En r\`egle
g\'en\'erale, on peut dire ceci: avec un \'ecart-type d'ordre $n$  (racine
$n^{\rm e}$ de la moyenne des valeurs absolues des puissances  $n^{\rm
e}$), plus $n$ est \'elev\'e, plus on p\'enalise les grands \'ecarts, m\^eme
s'ils ont une probabilit\'e faible.

\medskip

{\eightpoint Nous avons ainsi vu qu'on pouvait analyser la loi d'une variable
al\'eatoire \`a l'aide de param\`etres quantitatifs tels que la variance,
l'\'ecart-type, etc. Nous avons introduit la moyenne des puissances
quatri\`emes des \'ecarts, qui comme les carr\'es, ou comme toute autre
puissance paire, sont toujours positives. Si on voulait prendre des
puissances impaires, il faudrait bien s\^ur prendre la moyenne {\it de
leurs valeurs absolues} pour avoir une estimation significative de la
dispersion. Toutefois, la moyenne d'une puissance impaire, sans valeur
absolue, contient aussi de l'information sur la loi de la variable al\'eatoire;
quoique elle n'exprime pas la dispersion, une telle grandeur joue un r\^ole
dans l'analyse de Fourier des lois de probabilit\'e. On appelle {\bf moments}
ce type de grandeurs. De fa\c{c}on pr\'ecise:
\medskip
{\narrower On appelle {\bf moment d'ordre} $n$ d'une variable al\'eatoire 
$X$ la moyenne $\sum p_k x_k^n$ des $n^{\rm e}$ puissances des valeurs de
$X$. La moyenne de $X$ est ainsi le moment d'ordre 1 de $X$ et la variance
de $X$ est le moment d'ordre 2 de $X - m$.}  }

\bigskip

Les param\`etres quantitatifs tels que les moments ou la moyenne des
valeurs absolues, les \'ecarts-types, etc. contiennent une information 
sur  la loi de la variable al\'eatoire, mais ne contiennent pas toute 
l'information. On ne peut pas retrouver la loi d'une variable al\'eatoire 
si on n'en conna{\^\i}t que la moyenne et l'\'ecart-type. Sous certaines
conditions assez compliqu\'ees mais pas trop restrictives, la donn\'ee de
tous les moments d\'etermine la loi de la variable, mais nous laissons
cette question de c\^ot\'e. Par contre il y a deux mani\`eres tr\`es
efficaces de condenser la totalit\'e de l'information sur la loi d'une
variable al\'eatoire, c'est d'introduire la {\it fonction g\'en\'eratrice} et 
la {\it fonction caract\'eristique}. 
\medskip
\'Etant donn\'ee une variable al\'eatoire de loi $\{ x_k, \, (p_k) \}$, on 
appelle {\it fonction caract\'eristique} de $X$ la fonction
$$\Phi_X(t) = \sum_{k} p_k \e^{itx_k}   \eqno (VI.3.)$$
En fait c'est simplement la transform\'ee de Fourier de la loi. La fonction
g\'en\'eratrice est
$$G_X(z) = \sum_{k} p_k z^{x_k}   \eqno (VI.4.)$$
La fonction g\'en\'eratrice est plut\^ot recommand\'ee lorsque les $x_k$ 
sont  des nombres entiers; dans ce cas c'est un polyn\^ome, ou, si on fait
tendre le nombre des $x_k$ vers l'infini, une fonction analytique de $z$. 
On b\'en\'eficie alors de tous les outils math\'ematiques li\'es aux
polyn\^omes ou aux fonctions d'une variable complexe. 
Rien n'interdit dans le principe de la prendre en consid\'eration
m\^eme lorsque les $x_k$ sont non entiers, mais dans un tel cas elle est
beaucoup moins commode et on lui pr\'ef\'erera alors la fonction
caract\'eristique.  
\medskip
De toute fa\c{c}on, fonction caract\'eristique et fonction g\'en\'eratrice 
sont li\'ees par la relation 
$$\Phi_X (t) = G_X (\e^{it})   \eqno (VI.5.)$$

Pour une variable al\'eatoire $X$ \`a valeurs enti\`eres, on obtient les
probabilit\'es de chaque valeur (c'est-\`a-dire la loi de la variable
al\'eatoire $X$) en d\'eveloppant la fonction g\'en\'eratrice $G_X(z)$ en 
s\'erie  enti\`ere, ou en s\'erie de Laurent s'il y a des valeurs n\'egatives.
\medskip
On peut aussi d\'eduire directement de la fonction g\'en\'eratrice d'autres
grandeurs li\'ees \`a la variable al\'eatoire telles que la moyenne et la
variance. Ainsi, la moyenne n'est autre que $G_X'(1)$ (la d\'eriv\'ee de
$G_X(z)$  au point $z=1$). En effet
$$G_X'(z) = \sum_n n \, p_n\, z^{n-1}$$
donc pour $z=1$ cela donne 
$$G_X'(1) = \sum_n n p_n = {\bf E}(X)  \eqno (VI.6.)$$ 
La d\'eriv\'ee seconde fournira la variance: 
$$G_X''(z) = \sum_n n(n-1) \, p_n\, z^{n-2}$$
ce qui pour $z=1$ donne $G_X''(1) = \sum_n n(n-1) p_n = \sum_n n^2 p_n -
\sum_n n p_n = {\bf E}(X^2) - {\bf E}(X)$. En utilisant $VI.2\, a$ on peut
avoir la variance sous la forme:
$${\bf Var}(X) = G_X''(1) + G_X'(1) -  G_X'(1)^2  \eqno (VI.7.)$$

Des expressions analogues pour la moyenne ou la variance peuvent \^etre
obtenues \`a partir des fonctions caract\'eristiques. Ces expressions
seront utilis\'ees lorsque la variable al\'eatoire n'est pas \`a valeurs
enti\`eres. On obtient en effet par d\'erivation:
$$\eqalign{
\Phi_X'(t) &=\sum_{k} i x_k p_k \e^{ix_kt} \cr
\Phi_X''(t) &=\sum_{k} - x_k^2 p_k \e^{ix_kt} \cr }$$
donc pour $t=0$
$$\eqalign{
\Phi_X'(0) &= i\sum_{k} x_k p_k = i {\bf E}(X) \cr
\Phi_X''(0) &=-\sum_{k} x_k^2 p_k = - {\bf E}(X^2) \cr }$$

Cela conduit (compte tenu aussi de $VI.2\, a$) aux expressions suivantes
de la moyenne et de la variance:
$$\eqalign{
{\bf E}(X) &= -i \Phi_X'(0)  \cr
{\bf Var}(X) &= \Phi_X'(0)^2 - \Phi_X''(0)  \cr } \eqno (VI.8.)$$
\medskip
Comme nous connaissons d\'ej\`a quelques variables al\'eatoires, nous
pouvons \`a titre d'exemple illustratif calculer leurs fonctions
g\'en\'eratrices ou leurs fonctions caract\'eristiques. Reprenons les 
 exemples $b$ et $c$ ci-dessus: il s'agissait de $(a)$ l'abscisse atteinte 
par une marche al\'eatoire au bout de $2n$ pas; $(b)$ l'instant du premier
retour en z\'ero.  Les fonctions g\'en\'eratrices correspondantes sont 
$$\eqalign{
G_a (z) &= \sum_{j=-n}^{+n} 2^{-2n} {2n \choose n+j}\; z^{2j} = \biggl[ {1
\over 2} \Bigl( z + {1 \over z}\Bigr)\biggr]^{2n} \cr
G_b (z) &= \sum_{j=1}^{n} 2^{-2n} {2j \choose j} {1 \over 2j - 1}\; z^{2j} +
\hbox{reste} \cr }$$
Le reste n'est pas vraiment d\'efini puisqu'il correspond \`a la valeur 
``{\it jamais}'' du premier retour mais il tend vers z\'ero lorsque $n$ tend
vers l'infini, et on peut alors montrer que la limite est $G_b (z)  = 1 -
\sqrt{1-z^2}$. Si on convient, comme nous en avons \'evoqu\'e la 
possibilit\'e plus haut, de r\'eserver la valeur $0$ pour ``jamais'', le reste
sera tout simplement le terme constant
$$\sum_{j = n+1}^{\infty} 2^{-2j}{2j \choose j} {1 \over 2j-1}$$
On voit que l'exp\'edient qui consiste \`a prendre la valeur z\'ero pour
``jamais'' n'apporte aucune simplification dans les calculs: les
fonctions g\'en\'eratrices sont en g\'en\'eral un puissant outil de calcul.
Dans ce
cas particulier, elles le restent quand $n$ est grand, car $G_b(z)$ est alors
une fonction dont l'expression analytique est simple.
\medskip
Quant aux fonctions caract\'eristiques,  elles s'obtiennent en rempla\c{c}ant
$z$ par $\e^{it}$ dans les expressions de la fonction g\'en\'eratrice,
 soit:
$$\eqalign{
\Phi_a (t) &= \biggl[ {1\over 2}\Bigl( \e^{it} + \e^{-it}\Bigr)\biggr]^{2n} 
= \; \bigl[ \cos (t) \bigr]^{2n}  \cr
\Phi_b (t) &\simeq 1 - \sqrt{1 - \e^{2it}} \cr }$$
(ici $\simeq$ signifie que l'\'egalit\'e est approch\'ee pour $n$ grand).

\bigskip

{\bf IV\ata .\aub 3. \vtop{\hsize=9cm  \hbox{
L'ind\'ependance stochastique de deux (ou plusieurs)}
\hbox{variables al\'eatoires.} }  }
\medskip
Au chapitre $IV$ nous avons \'etudi\'e l'ind\'ependance stochastique de 
deux \'ev\'enements ou de deux familles d'\'ev\'enements. Nous avons vu
que l'ind\'ependance stochastique \'etait toujours li\'ee \`a une
factorisation de l'espace $\Omega$: deux \'ev\'enements $A$ et $B$
\'etaient alors stochastiquement ind\'ependants si on pouvait factoriser
l'ensemble $\Omega$ sous la forme d'un tableau (une matrice)
rectangulaire, dans lequel l'\'ev\'enement $A$ \'etait un ensemble de 
 lignes (plus exactement une r\'eunion de lignes compl\`etes) et
l'\'ev\'enement $B$ un ensemble de colonnes. De m\^eme une famille
d'\'ev\'enements $\{ A_i \}_{i=1,2, \ldots m}$ est stochastiquement
ind\'ependante d'une famille d'\'ev\'enements $\{ B_j \}_{j=1,2, \ldots n}$,
si on peut factoriser $\Omega$ de telle mani\`ere que les $A_i$ soient  
des ensembles de lignes et les $B_j$ des ensembles de colonnes. On a  
alors la propri\'et\'e ${\cal P}\, (A_i B_j) = {\cal P}\, (A_i) \times {\cal
P}\, (B_j)$.  
\medskip
On dit que deux variables al\'eatoires $X$ et $Y$ de lois respectives 
$\{ x_i \; (p_i) \}$ et $\{ y_j \; (q_j) \}$ sont stochastiquement
ind\'ependantes si la famille d'\'ev\'enements $A_i \; : \; X = x_i$ est
stochastiquement ind\'ependante de la famille $B_j \; : \; Y = y_j$. Il y a
alors une factorisation de $\Omega$ dans laquelle les \'ev\'enements 
$A_i$ sont form\'es de lignes compl\`etes et les $B_j$ de colonnes
compl\`etes; cela signifie que la variable al\'eatoire $X$ est constante le
long des lignes et  $Y$ est constante le long des colonnes. Une \'epreuve
$\omega$ dans le tableau est d\'efinie par la donn\'ee de son num\'ero de
ligne $\ell (\omega )$ et son num\'ero de colonne ${\it col} (\omega )$. On
peut alors exprimer l'ind\'ependance stochastique de $X$ et $Y$ en disant
que $X (\omega )$ ne d\'epend que de $\ell  (\omega )$, et $Y  (\omega )$ ne
d\'epend que de ${\it col} (\omega )$. En tous cas, on a la propri\'et\'e de
factorisation  ${\cal P}\, (X = x_i \; {\rm et } \;  Y = y_j) = {\cal P}\, (X =
x_i) \times {\cal P}\, (Y = y_j) = p_i q_j$, qui \'evidemment n'aurait pas
lieu si $X$ et $Y$ n'\'etaient pas stochastiquement ind\'ependantes.
\medskip
Lorsque deux variables al\'eatoires sont stochastiquement ind\'ependantes, 
il y a des r\`egles simples pour calculer leur somme. En effet,
l'\'ev\'enement $X + Y = z_k$ est la r\'eunion des \'ev\'enements disjoints
$X = x_i$ et $Y = y_j$ pour lesquels $x_i + y_j = z_k$; or d'apr\`es
l'ind\'ependance stochastique on peut dire que ${\cal P}\, (X = x_i \; {\rm
et } \;  Y = y_j) = p_i q_j$, donc 
$${\cal P}\, (X + Y = z_k) = \sum_{x_i + y_j = z_k} p_i q_j$$
(la sommation sur $x_i + y_j = z_k$ signifie qu'on ne retient pour la
somme que les indices $i$ et $j$ pour lesquels $x_i + y_j = z_k$).
Si on applique cela \`a la fonction caract\'eristique de la variable 
al\'eatoire $X + Y$, on obtient 
$$\eqalignno{
\Phi_{X+Y} (t) &= \sum_{k} {\cal P}\, (X + Y = z_k) \e^{itz_k} \cr
&= \sum_k \Bigl[ \sum_{x_i + y_j = z_k} p_i q_j \Bigr]  \e^{itz_k} \cr
&= \sum_k \Bigl[ \sum_{x_i + y_j = z_k} p_i q_j \Bigr]  \e^{it(x_i + y_j)}\cr
&= \sum_k \Bigl[ \sum_{x_i + y_j = z_k} p_i\e^{itx_i}  q_j\e^{ity_j}  \Bigr] 
\cr
&= \sum_{i,j} p_i\e^{itx_i}  q_j\e^{ity_j}\cr }$$
qui n'est autre que le produit $\Phi_X(t) \times \Phi_Y(t)$ des fonctions
caract\'eristiques de $X$ et de $Y$. 
\medskip
On peut donc \'enoncer:
\medskip
{\bf Th\'eor\`eme:} Si $X$ et $Y$ sont des variables al\'eatoires
stochastiquement ind\'ependantes, alors la fonction caract\'eristique de 
leur somme est le produit des fonctions caract\'eristiques de chacune:
$$\Phi_{X+Y} (t) = \Phi_X(t) \times \Phi_Y(t)   \eqno (VI.9.)$$
\medskip
Bien entendu, puisque $\Phi_X(t) = G_X(\e^{it})$, il est clair qu'on a la
m\^eme propri\'et\'e pour les fonctions g\'en\'eratrices:
$$G_{X+Y} (z) = G_X(z) \times G_Y(z)   \eqno (VI.10.)$$
mais comme on le verra dans les exemples et exercices, cette relation
pour les fonctions g\'en\'eratrices, quoique toujours vraie, n'est d'un 
emploi commode que si les $x_i$ et $y_j$ sont entiers.
\medskip
On peut interpr\'eter l'abscisse atteinte au bout de $2n$ pas par une 
marche al\'eatoire comme la somme de $2n$ variables al\'eatoires $X_j$: 
$$X_j = \cases{+1 \; (\hbox{avec prob. $1/2$}) &si le $j^{\rm e}$ pas est 
un pas en avant \cr
               -1 \; (\hbox{avec prob. $1/2$}) &si le $j^{\rm e}$ pas 
est un  pas en arri\`ere \cr }$$ 
l'abscisse atteinte apr\`es $2n$ pas est alors \'egale \`a
$S = X_1 + X_2 + X_3 + \cdots + X_{2n}$. Ces variables al\'eatoires sont
stochastiquement ind\'ependantes, puisque chaque pas d'une marche
al\'eatoire est stochastiquement ind\'e\-pen\-dant des autres. La fonction
g\'en\'eratrice de chacune des $X_j$ est $G_j(z) = {1 \over 2}(z + 1/z)$ et
d'apr\`es le th\'eor\`eme ci-dessus on doit avoir $G_S (z) = \bigl[ {1 \over
2}(z  + 1/z)\bigr]^{2n}$, ce qui effectivement recoupe le calcul direct
pr\'ec\'edent. On pouvait donc utiliser la multiplication des fonctions
g\'en\'eratrices pour calculer la loi de $S$, au cas o\`u on ne l'aurait pas
d\'ej\`a calcul\'ee par un autre proc\'ed\'e.

\bigskip

{\bf VI.\aub 4.  La loi conjointe de deux (ou plus) variables
al\'eatoires.} 
\medskip
Lorsque deux variables al\'eatoires sont ind\'ependantes, on peut calculer 
la probabilit\'e d'\'ev\'enements tels que $\{ X = x_j \; ;\; Y = y_k \}$ \`a
partir de la seule connaissance des lois respectives de $X$ et $Y$. Il suffit
pour cela d'appliquer la r\`egle du produit. En effet, l'\'ev\'enement $\{ X =
x_j \; ;\;  Y = y_k \}$ (``$X = x_j$ et $Y = y_k$'') est l'intersection $\{ X =
x_j \} \cap \{ Y = y_k \}$; donc sa probabilit\'e est, d'apr\`es la r\`egle du
produit, \'egale \`a  ${\cal P}\, ( X = x_j ) \cdot {\cal P}\, ( Y = y_k )$.
\medskip
Par contre, si $X$ et $Y$ ne sont pas stochastiquement ind\'ependantes, on
ne peut pas d\'eduire les probabilit\'es ${\cal P}\, (  X = x_j \; ;\; Y = y_k 
)$
des probabilit\'es ${\cal P}\, (  X = x_j )$ et ${\cal P}\, ( Y = y_k )$. Il 
faut
disposer d'une information suppl\'ementaire, qui caract\'erise la 
d\'ependance  entre $X$ et $Y$. 
\medskip
Supposons que $Y$ soit une fonction de $X$, par exemple $Y = X^2$. Dans 
ce cas on peut calculer les probabilit\'es ${\cal P}\, ( X = x_j \; ;\; Y = 
y_k )$ \`a partir des probabilit\'es ${\cal P}\, (  X = x_j )$ et  ${\cal P}\,  
( Y = y_k )$.  
\medskip
On peut m\^eme faire plus fort: les calculer \`a partir des seules ${\cal P}
\, (  X = x_j )$. En effet, les $y_k$ sont alors n\'ecessairement les carr\'es
des $x_j$; si deux $x_j$ distincts, disons $x_{j_1}$ et $x_{j_2}$, ont le
m\^eme carr\'e $y_k$ (cela \'equivaut \`a dire que $x_{j_2} = -x_{j_1}$),
alors ${\cal P}\, (Y = y_k) = {\cal P}\, (X = x_{j_1}) + {\cal P}\, (X =
x_{j_2})$; si un seul des $x_j$, disons $x_{j_0}$, a pour carr\'e $y_k$,
alors ${\cal P}\, (Y = y_k) = {\cal P}\, (X = x_{j_0})$. On voit que dans ce
cas la loi de $Y$ se d\'eduit directement de celle de $X$. Quant aux
probabilit\'es ${\cal P}\, (  X = x_j \; ; \; Y = y_k )$, elles sont
\'evidemment nulles si $y_k$ n'est le carr\'e d'aucun $x_j$, et \'egales \`a
${\cal P}\, ( X = x_j )$ sinon. On peut dire  que non seulement il n'y a pas 
plus d'information dans la donn\'ee des probabilit\'es des \'ev\'enements
faisant intervenir les deux variables al\'eatoires $X$ et $Y$ que dans la
donn\'ee des probabilit\'es des \'ev\'enements  faisant intervenir
s\'epar\'ement chacune des deux variables al\'eatoires $X$ et $Y$, mais
m\^eme qu'il n'y en a pas plus que dans la donn\'ee des probabilit\'es des
\'ev\'enements  faisant intervenir  la seule variable al\'eatoire $X$.  
\medskip
L'exemple que nous venons de voir est en fait un cas extr\^eme de
d\'ependance. Il s'agit du cas o\`u $X$ et $Y$ sont fonction l'une de l'autre.
$X$ est  bien al\'eatoire, c'est-\`a-dire que les valeurs prises par $X$
d\'ependent d'un choix (fait sur $\Omega$) du hasard; mais la valeur prise
par $X$ d\'etermine alors univoquement la valeur prise par $Y$, de sorte
qu'il n'y a aucune intervention {\it suppl\'ementaire} du hasard pour 
d\'eterminer $Y$. Entre les deux extr\^emes, il peut y avoir des formes de
d\'ependance moins radicales qu'une telle relation fonctionnelle: que par
exemple, la valeur de $X$ \'etant fix\'ee, plusieurs valeurs puissent \^etre
prises par $Y$, chacune avec une certaine probabilit\'e, mais que ces
probabilit\'es d\'ependent de la valeur de $X$. L'ind\'ependance
stochastique signifie que les probabilit\'es pour que $Y$ prenne une valeur
$y_k$ ne sont pas influenc\'ees par la valeur prise par $X$. Si ces
probabilit\'es d\'ependent de la valeur de $X$, il y a une d\'ependance {\it
stochastique} de $Y$ par rapport \`a $X$, mais $Y$  n'est pas pour autant
une {\it fonction} de $X$. Pour rendre cela plus concret, prenons les trois
exemples que voici.  
\medskip

\def\hfq{\hfill\quad}
\def\cc#1{\hfill\quad #1\quad\hfill}
\def\tv{\vrule height 12pt depth 5pt width0.4pt}

\midinsert
\vbox to \blocksize{\null\vfill
\vbox{\offinterlineskip
\def\trhor{\noalign{\hrule }}
\halign{#&#\hfq&\tv#&#\hfq&\tv#&#\hfq&
\tv#&\hfq#&\tv#&#\hfq&\tv#&#\hfq&\tv#\cr
&\omit\cc {} &&\omit\cc {$X=-2$}&&\omit\cc {$X=-1$}&&\omit\cc 
{$X=0$}& &\omit\cc {$X=+1$}&&\omit\cc {$X=+2$}&\cr 
\trhor
&\omit\cc {$Y=0$}&&\omit\cc {$1/25$}&&\omit\cc {$1/25$}
&&\omit\cc {$1/25$}& &\omit\cc {$1/25$}&&\omit\cc {$1/25$} &\cr 
\trhor 
&\omit\cc {$Y=+1$}&&\omit\cc {$2/25$}&&\omit\cc {$2/25$}
&&\omit\cc {$2/25$}& &\omit\cc {$2/25$}&&\omit\cc {$2/25$}&\cr 
\trhor
&\omit\cc {$Y=+4$}&&\omit\cc {$2/25$}&&\omit\cc {$2/25$}
&&\omit\cc {$2/25$}& &\omit\cc {$2/25$}&&\omit\cc {$2/25$}&\cr
\trhor
}}
\vskip9mm
\centerline{\vbox{\hsize=116mm
\centerline{\bf tableau 1}
\vskip6pt
{\eightpoint Ce tableau repr\'esente la loi de probabilit\'e conjointe de 
deux variables al\'eatoires $X$ et $Y$ {\it ind\'ependantes}. On trouve la
probabilit\'e pour que $X = j$ et $Y = k$ (avec $j = -2,\, -1,\, 0,\, +1,\, 
+2$ et $k = 0,\, 1,\, 4$) \`a l'intersection de la ligne $Y = k$ et de la
colonne $X = j$. On peut constater que la probabilit\'e conditionnelle pour 
que $Y$ prenne l'une de ses trois valeurs est la m\^eme quelle que soit la
valeur de $X$ (les rapports des nombres figurant dans une m\^eme ligne
\`a la somme des \'el\'ements de la ligne sont les m\^emes pour chaque
ligne; de m\^eme pour les colonnes). } 
}}

\vfill

\vbox{\offinterlineskip
\def\trhor{\noalign{\hrule }}
\halign{#&#\hfq&\tv#&#\hfq&\tv#&#\hfq&\tv#&\hfq#&\tv#&#\hfq&\tv#&#\hfq&\tv#\cr
&\omit\cc {} &&\omit\cc {$X=-2$}&&\omit\cc {$X=-1$}&&\omit\cc {$X=0$}& 
&\omit\cc {$X=+1$}&&\omit\cc {$X=+2$}&\cr 
\trhor
&\omit\cc {$Y=0$}&&\omit\cc {$0$}&&\omit\cc {$0$}&&\omit
\cc {$1/5$}& &\omit\cc {$0$}&&\omit\cc {$0$}&\cr
\trhor
&\omit\cc {$Y=+1$}&&\omit\cc {$0$}&&\omit\cc
{$1/5$} &&\omit\cc {$0$}& &\omit\cc {$1/5$}&&\omit\cc {$0$}&\cr
\trhor
&\omit\cc {$Y=+4$}&&\omit\cc {$1/5$}&&\omit\cc
{$0$}&&\omit\cc {$0$}& &\omit\cc {$0$}&&\omit\cc {$1/5$}&\cr
\trhor
}}
\vskip9mm
\centerline{\vbox{\hsize=116mm
\centerline{\bf tableau 2}
\vskip6pt
{\eightpoint Ce tableau repr\'esente la loi de probabilit\'e conjointe de 
deux variables al\'eatoires $X$ et $Y$ {\it li\'ees} par la relation $Y = X^2$.
On voit que les probabilit\'es sont nulles lorsque $Y \neq X^2$. Mais la loi
de $X$ (somme des probabilit\'es figurant sur une m\^eme colonne) et la 
loi de $Y$  (somme des probabilit\'es figurant sur une m\^eme ligne) sont
les m\^emes que dans le tableau 1.}
}}

\vfill
} 
\endinsert

\midinsert
\null\vskip1pt
\vbox{\offinterlineskip
\def\trhor{\noalign{\hrule }}
\halign{#&#\hfq&\tv#&#\hfq&\tv#&#\hfq&\tv#&\hfq#&\tv#&#\hfq&\tv#&#\hfq&\tv#\cr
&\omit\cc {} &&\omit\cc {$X=-2$}&&\omit\cc {$X=-1$}&&\omit\cc {$X=0$}& 
&\omit\cc {$X=+1$}&&\omit\cc {$X=+2$}&\cr 
\trhor
&\omit\cc {$Y=0$}&&\omit\cc {$1/60$}&&\omit\cc {$1/30$}
&&\omit\cc {$1/10$}& &\omit\cc {$1/30$}&&\omit\cc {$1/60$}&\cr
\trhor 
&\omit\cc {$Y=+1$}&&\omit\cc {$1/15$}&&\omit\cc {$1/10$}
&&\omit\cc {$1/15$}& &\omit\cc {$1/10$}&&\omit\cc {$1/15$}&\cr 
\trhor
&\omit\cc {$Y=+4$}&&\omit\cc {$7/60$}&&\omit\cc {$1/15$}
&&\omit\cc {$1/30$}& &\omit\cc {$1/15$}&&\omit\cc {$7/60$}&\cr
\trhor
}}
\vskip8mm
\centerline{\vbox{\hsize=116mm
\centerline{\bf tableau 3}
\vskip6pt
{\eightpoint Ce tableau repr\'esente la loi de probabilit\'e conjointe de 
deux variables al\'eatoires $X$ et $Y$ qui ne sont ni ind\'ependantes, ni
fonction l'une de l'autre. La probabilit\'e conditionnelle pour que $Y=k$
sachant que $X=j$ d\'epend de $j$, mais par exemple pour $X=0$ il y a
comme on voit  trois valeurs possibles pour $Y$, avec probabilit\'es non
nulles $1/10$,  $1/15$, $1/30$. La probabilit\'e conditionnelle pour que 
$Y=0$ sachant que $X=0$ est ${1 \over 2}$; mais la probabilit\'e 
conditionnelle pour que  $Y=0$ sachant que $X=1$ est ${1 \over 6}$ et  la
probabilit\'e conditionnelle pour que  $Y=0$ sachant que $X=2$ est ${1 
\over 12}$: elle d\'epend fortement de la valeur de $X$, mais n'est pas
\'egale \`a $0$ ou $1$ comme c'\'etait le cas pour le tableau 2.} 
}}
\vskip5mm plus2mm minus2mm \null
\endinsert

$X$ est une variable al\'eatoire qui prend les valeurs $-2,\, -1,\, 0,\, +1,\, 
+2$. $Y$ est une autre variable al\'eatoire qui prend les valeurs $0,\, +1,\, 
+4$. Chacun des trois tableaux ci-contre repr\'esente une r\'epartition
diff\'erente des probabilit\'es ${\cal P}\, (  X = x_j \; ; \; Y = y_k )$.
Dans les trois tableaux, les lois de probabilit\'es de $X$ et de $Y$ sont les
m\^emes, c'est-\`a-dire que les probabilit\'es
$$p_j = {\cal P}\, (X = j) = \sum_{k=0,1,4} {\cal P}\, (X = j\; ;\; Y = k)$$
pour $j = -2,\,   -1,\,   0,\,   +1,\,   +2$, ainsi que  les probabilit\'es
$$q_k = {\cal P}\, (Y = k) = \sum_{j=0,\pm 1,\pm 2} {\cal P}\, (X = j\; ; \; 
Y = k)$$ 
pour $k = 0,\,  1,\,  4$, sont les m\^emes dans les trois tableaux (on
obtient les $p_j$ en faisant la somme des \'el\'ements de la colonne $j$ 
et les $q_k$ en faisant la somme des \'el\'ements de la ligne $j$). Par
contre ces trois exemples diff\`erent par la r\'epartition bidimensionnelle
des probabilit\'es, qui refl\`ete la d\'ependance entre les deux variables.
\medskip
Dans le premier tableau la r\'epartition est faite de fa\c{c}on \`a assurer
l'ind\'ependance stochastique entre $X$ et $Y$. En effet, si on calcule les
probabilit\'es conditionnelles 
$${\cal P}\, (X = j \mid Y = k) = {{\cal P}\, (X = j \; ; \; Y = k) \over  {\cal
P}\, (Y = k)} = {{\cal P}\, (X = j \; ; \; Y = k) \over \sum_j {\cal
P}\, (X = j \; ; \; Y = k)}$$
qui sont donc pour chaque ligne le quotient des \'el\'ements de la ligne
par la somme des \'el\'ements de la ligne, on constate que ces 
probabilit\'es conditionnelles ne d\'ependent pas de $k$, ce qui est bien   
la marque de l'ind\'ependance stochastique; au demeurant ${\cal P}\, (X =
j  \mid Y = k) = {\cal P}\, (X = j)$. De m\^eme les probabilit\'es
conditionnelles   
$${\cal P}\, (Y = k \mid X = j) = {{\cal P}\, (X = j \; ; \; Y =
k) \over  {\cal P}\, (X = j)} = {{\cal P}\, (X = j \; ; \; Y = k) \over \sum_k
{\cal P}\, (X = j \; ; \; Y = k)}$$
qui sont donc pour chaque colonne le quotient des \'el\'ements de la 
colonne par la somme des \'el\'ements de la colonne, ne d\'ependent pas 
de $j$. (V\'erifiez cela sur le tableau avec un crayon). 
\medskip
Le second tableau repr\'esente la r\'epartition des probabilit\'es lorsque 
$Y = X^2$. Cette d\'ependance absolue fait que $Y$ ne peut pas prendre une
valeur qui ne soit pas le carr\'e de $X$. Si $X=2$, $Y$ ne peut pas \^etre
\'egal \`a $0$ ou \`a $1$, ce qui se traduit sur le tableau par le fait que la
probabilit\'e pour que $X=2$ et $Y=0$, ou pour que $X=2$ et $Y=1$, est
nulle.  
\medskip
Enfin, dans le troisi\`eme tableau, on a un exemple de d\'ependance
{\it stochastique}, c'est-\`a-dire que la variable $Y$ n'est pas une
fonction de $X$, mais n'est pas non plus stochastiquement ind\'ependante  
de $X$: lorsque $X$ prend la valeur $+2$, $Y$ ne prend pas {\it forc\'ement}
la valeur $+4$, et peut prendre n'importe laquelle des trois valeurs
$0$, $1$, ou $4$. Mais cette fois, contrairement \`a ce qui se passe sur le
tableau 1, les probabilit\'es conditionnelles ${\cal P}\, (Y = k \mid X = j)$
d\'ependent de $j$, et aussi les probabilit\'es conditionnelles ${\cal P}\,  
(X = j \mid Y = k)$ d\'ependent de $k$. Par exemple, la probabilit\'e
(conditionnelle) pour que $Y=0$ sachant que $X=0$ est ${1 \over 2}$, mais
la probabilit\'e conditionnelle pour que $Y=0$ sachant que $X=+1$ est ${1
\over 6}$ et la probabilit\'e conditionnelle pour que $Y=0$ sachant que
$X=+2$ est ${1 \over 12}$. Dans le tableau 2, si $X=0$ , il est {\it
certain} que $Y=0$; dans le tableau 1,  la probabilit\'e conditionnelle pour 
que $Y=0$ sachant que $X=j$ est {\it toujours} ${1 \over 5}$ quelle que
soit la valeur $j$ de $X$.
\bigskip
Tout cela montre que, si deux variables al\'eatoires $X$ et $Y$ ne sont 
ni stochastiquement ind\'ependantes, ni li\'ees par une relation 
fonctionnelle, on ne peut d\'eduire {\it \`a partir de leurs deux lois} les
probabilit\'es de toutes les combinaisons de valeurs qu'elles peuvent
prendre conjointement: il faut disposer d'un tableau qui donne  
directement l'information compl\`ete. On appelle {\it loi conjointe} 
des deux variables al\'eatoires $X$ et $Y$ la donn\'ee de ce tableau,
c'est-\`a-dire la donn\'ee de   
\smallskip
a) toutes les valeurs $x_j$ que peut prendre $X$ et toutes les valeurs
$y_k$ que peut prendre $Y$;
\smallskip
b) toutes les probabilit\'es $r_{j,k} = {\cal P}\, (X=x_j \; ; \; Y=y_k)$
\medskip
Pour d\'esigner commod\'ement une loi conjointe on \'ecrira $\{ x_j, y_k
\; ; \;  (r_{j,k})\}$. Cela signifiera que la probabilit\'e pour que $X=x_j$ 
et $Y=y_k$ est $r_{j,k}$. Cette loi conjointe contient donc une information
qui {\it n'est pas}  d\'ej\`a contenue dans les deux lois $\{ x_j \; ; \; (p_j)
\}$ et $\{ y_k \; ;  \; (q_k) \}$, sauf, comme nous l'avons d\'ej\`a dit, si 
$X$ et $Y$ sont stochastiquement ind\'ependantes (alors $r_{j,k} = p_j
q_k$) ou si $Y = f(X)$, $f$ \'etant une fonction non al\'eatoire (alors
$r_{j,k} = 0$ si $y_k \neq f(x_j)$ et $r_{j,k} = p_j$ si $y_k = f(x_j)$,
comme sur le  tableau 2). 
\medskip
Les deux lois $\{ x_j \; ; \; (p_j)\}$ et $\{ y_k \; ;  \; (q_k) \}$,
c'est-\`a-dire les lois de $X$ et de $Y$ sont appel\'ees {\it lois 
marginales}, par opposition \`a la loi conjointe du couple $X,Y$. Lorsque 
la loi conjointe de deux variables est repr\'esent\'ee sur un tableau \`a 
double entr\'ee comme les tableaux 1, 2, et 3, on obtient les lois
marginales en effectuant les sommes sur les lignes et sur les colonnes.
\medskip
Cela peut se g\'en\'eraliser \`a un nombre quelconque de
variables al\'eatoires: au lieu d'un tableau \`a deux indices (ou matrice), 
on aurait un tableau \`a $n$ indices ou dimensions.
\medskip
Tout cela n'est bien s\^ur qu'une question de langage; si d\`es le d\'epart 
(et certains auteurs proc\`edent ainsi) on avait annonc\'e que les 
variables al\'eatoires sont des grandeurs vectorielles, que ce sont bien 
des fonctions d\'efinies sur l'espace $\Omega$, mais \`a valeurs dans
\R${}^n$ et non dans \R, on n'aurait pas eu a distinguer le cas de plusieurs
variables al\'eatoires du cas d'une seule; la loi d'une variable al\'eatoire
vectorielle $X$ serait la donn\'ee des valeurs vectorielles $x_j$ et de 
leurs probabilit\'es, qui serait \'evidemment la m\^eme chose que la loi
conjointe des $n$ composantes de $X$. Dans ce langage vectoriel on ne
parlerait pas de l'ind\'ependance stochastique entre $n$ variables
al\'eatoires, mais entre les $n$  composantes d'une variable al\'eatoire.
\medskip
Ce qui en tous cas reste certain, c'est que (dit dans le langage non
vectoriel) si deux variables al\'eatoires ne sont ni stochastiquement
ind\'ependantes, ni li\'ees par une relation fonctionnelle, on ne peut 
calculer les probabilit\'es qui leur sont relatives que si on conna{\^\i}t 
leur loi conjointe.
\medskip
Pr\'ec\'edemment (\S {\bf VI.\aub 3}) nous avons associ\'e \`a la loi
d'une  variable al\'eatoire des  grandeurs comme les moments (la variance)
ou des fonctions comme les fonctions g\'en\'eratrices ou caract\'eristiques.
On peut en faire autant avec les lois conjointes.  Ainsi on appelle {\it
covariance} des deux variables al\'eatoires $X$ et $Y$ (de loi conjointe 
$\{ x_j, y_k\; ;\; (r_{j,k})\}$) la grandeur 
$${\bf Cov}\, (X,Y) = \sum_{j,k} r_{j,k}\, \bigl[ x_j-{\bf E}(X)\bigr] \bigl[ 
y_k-{\bf E}(Y)\bigr]   \eqno (VI.11.)$$ 
De m\^eme on appellera {\it fonction g\'en\'eratrice conjointe} de $X$ et 
$Y$  la fonction de {\it deux} variables complexes 
$$G_{X,Y}(z,w) = \sum_{j,k} r_{j,k}\; z^{x_j} w^{y_k}   \eqno (VI.12.)$$ 
(peu pratique si les nombres $x_j$ et $y_k$ ne
sont pas entiers) et {\it fonction caract\'eristique conjointe} de $X$ et 
$Y$ la fonction de {\it deux} variables r\'eelles 
$$\Phi_{X,Y}(s,t) = \sum_{j,k} r_{j,k}\; \e^{isx_j} \e^{ity_k}\eqno (VI.13.)$$
Bien entendu si $X$ et $Y$ sont stochastiquement ind\'ependantes on aura
$$G_{X,Y}(z,w) = G_X(z) \cdot G_Y(w)  \eqno (VI.14.)$$
et 
$$\Phi_{X,Y}(s,t) = \Phi_X(s) \cdot \Phi_Y(t)   \eqno (VI.15.)$$
Si $X$ et $Y$ sont stochastiquement ind\'ependantes, leur covariance est
n\'ecessairement nulle: en effet, dans ce cas on aura $r_{j,k}  = p_j q_k$,
et par cons\'equent l'expression $(VI.11.)$ de la covariance se factorise
$$\eqalign{
{\bf Cov}\, (X,Y) &= \sum_{j,k} r_{j,k} \bigl[ x_j-{\bf E}(X)\bigr] \bigl[ 
y_k-{\bf E}(Y)\bigr]   \cr
&= \sum_{j} p_{j} \bigl[ x_j-{\bf E}(X)\bigr] \; \times \;  \sum_{k} q_{k}
\bigl[ y_k-{\bf E}(Y)\bigr]  \cr
&= 0 \cr }$$ 
Une erreur tentante chez les d\'ebutants est de croire que la r\'eciproque 
a lieu: que si la covariance de deux variables al\'eatoires est nulle, alors 
elles sont ind\'ependantes. Cela est faux et on en a deux exemples avec 
les tableaux 2 et 3: si on calcule la covariance de $X$ et  $Y$ avec la loi 
conjointe du tableau 2 ou du du tableau 3, on trouve qu'elle est nulle, 
alors que $X$ et $Y$ ne sont ind\'ependantes dans aucun de ces deux cas.
\medskip
Nous avions vu au \S {\bf VI.\aub 2.} que si deux variables al\'eatoires
$X$ et  $Y$ sont ind\'ependantes, on peut facilement calculer la loi de leur
somme  en faisant le produit des fonctions g\'en\'eratrices ou des 
fonctions caract\'eristiques. Lorsque deux variables ne sont pas
ind\'ependantes,  mais qu'on conna{\^\i}t leur loi conjointe, et par
cons\'equent leur fonction  g\'en\'eratrice ou caract\'eristique conjointe,
on obtient la fonction  g\'en\'eratrice ou  caract\'eristique de leur somme
par les formules  suivantes:
$$\eqalign{
\Phi_{X+Y}(t) &= \Phi_{X,Y}(t,t) \cr
G_{X+Y}(z) &= G_{X,Y}(z,z) \cr } \eqno (VI.16.)$$
Si on combine cela avec $(VI.15.)$ ou $(VI.14.)$ on retrouve $(VI.9.)$ ou
$(VI.10.)$ 
\bigskip
Cela s'\'etend comme on le devine au cas de $2,\, 3,\, 4 \ldots$
variables al\'eatoires.





\bye
