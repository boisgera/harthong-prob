\input /home/harthong/tex/formats/twelvea4.tex
\input /home/harthong/tex/formats/epsf.tex
\vsize=219mm

\auteurcourant={\sl J. Harthong: probabilit\'es et statistique}
\titrecourant={\sl Lois de probabilit\'e asymptotiques}

\pageno=226 
 
\font\narrowtit=cmb17

\def\hfq{\hfill\quad} 
\def\cc#1{\hfill#1\quad\hfill} 
\def\tv{\vrule height 30pt depth 6pt width0.4pt} 
\def\punkt{\vrule height0.4pt depth0pt width0.4pt} 

\newdimen\blocksize  \blocksize=\vsize  \advance\blocksize by -9pt 

\def\ara{\hskip-1.5pt} 
\def\asa{\hskip-2pt} 
\def\ata{\hskip-2.5pt} 
\def\aua{\hskip-1pt} 
\def\arb{\hskip1.5pt} 
\def\asb{\hskip2pt} 
\def\atb{\hskip2.5pt} 
\def\aub{\hskip1pt} 
 
\null\vskip10mm plus4mm minus3mm 
 
\centerline{{\tit IX.}\hskip5pt {\narrowtit LOIS DE PROBABILIT\'E
ASYMPTOTIQUES.}} 
 
\vskip10mm 
 
Nous avons rencontr\'e aux chapitres {\bf VII} et {\bf VIII} des  
situations o\`u  apparaissent des lois de probabilit\'e complexes,  
qu'il est non seulement tr\`es difficile, voire impossible d'exprimer 
analytiquement, mais  qu'il est en outre {\it inutile} ou m\^eme nuisible 
de conna{\^\i}tre exactement.  
\medskip 
Au chapitre {\bf VII} il s'agissait de la loi d'une somme d'un 
grand nombre de variables al\'eatoires ind\'ependantes. Une telle loi  
peut bien s\^ur s'exprimer de mani\`ere math\'ematiquement exacte  
dans  certains cas exceptionnels o\`u les r\'egularit\'es sont tr\`es 
fortes (par exemple la loi bin\^omiale),  mais en g\'en\'eral elle est  
dans son d\'etail prodigieusement complexe, chaotique, et irr\'eductible 
\`a une formule analytique simple (voir figure 17 par exemple). 
Cependant en distinguant dans la loi exacte un signal (la densit\'e) et du 
bruit (le ``bruit discret''), on peut mettre en \'evidence une r\'egularit\'e  
\`a grande \'echelle (la forme gaussienne de la densit\'e).   
\medskip 
Au chapitre {\bf VIII} nous avons \'etudi\'e la loi d'un {\it processus en  
cascade} (du type r\'eaction de fission nucl\'eaire en cha{\^\i}ne) apr\`es  
un grand nombre de g\'en\'erations. L\`a aussi, on obtient une densit\'e 
limite, caract\'eris\'ee explicitement par sa transform\'ee de Fourier  
ou fonction caract\'eristique. Nous avons vu que, si la loi exacte et 
d\'etaill\'ee est en g\'en\'eral prodigieusement complexe et 
irr\'eductible \`a une formule analytique, on peut cependant calculer  
\`a l'aide d'algorithmes simples toute l'information {\it utile},  qui 
consiste en    
\smallskip 
a) la probabilit\'e de l'extinction ultime; 
 
b) la densit\'e de probabilit\'e du nombre de particules-filles de 
la $n^{\rm e}$ g\'en\'eration.  
\medskip 
Ces deux situations ont en commun que les lois exactes sont complexes, 
que les valeurs possibles de la variable consid\'er\'ee sont tr\`es  
nombreuses et qu'il importe peu de conna{\^\i}tre la probabilit\'e 
individuelle de chaque valeur possible; ce qui est utile est de pouvoir 
calculer, pour n'importe quel intervalle grand par rapport aux distances 
s\'eparant les valeurs individuelles, la probabilit\'e pour que ces valeurs  
se situent dans cet intervalle. Cette information est fournie par la 
densit\'e limite.  
\medskip 
On appelle {\it lois de probabilit\'es asymptotiques} ces lois limites, qui 
sont souvent, mais pas toujours, des densit\'es continues. Parfois elles 
comportent \`a la fois une partie discr\`ete et une partie continue: par 
exemple dans le cas du processus en cascade \'etudi\'e au chapitre {\bf  
VIII}, la loi asymptotique \'etait compos\'ee d'une partie discr\`ete (la 
probabilit\'e que $Z_n=0$) et d'une partie continue (pour les valeurs  
autres que z\'ero).  
\medskip 
Dans ce chapitre nous nous proposons d'\'etudier quelques lois de 
probabilit\'e asymptotiques, qu'on rencontre souvent et qu'il faut 
conna{\^\i}tre. 
\medskip 
La loi normale est un cas 
typique. Un autre type classique de loi limite, enti\`erement  
discr\`ete, est la loi de Poisson. Les lois limite qu'on rencontre dans la 
pratique se ram\`enent souvent \`a ces deux cas. Par  exemple  
les lois asymptotiques du $\chi^2$ ou de Student (voir chapitre {\bf XI})  
ne sont que des {\it d\'eguisements} (par changement de variable) de la  
loi asymptotique normale. Dans la section {\bf 1} nous verrons un cas 
particulier de la loi de Student, la {\it  loi de Cauchy}. Dans la section  
{\bf 2} nous \'etudierons la {\it loi de Poisson}, qui est une loi limite  
tr\`es fr\'equente, enti\`erement discr\`ete: elle r\'egit  g\'en\'eralement  
les \'ev\'enements caract\'eris\'es par comptage, tels que le nombre de 
v\'ehicules qui se pr\'esentent chaque minute \`a un p\'eage d'autoroute. 
\medskip  
Il y a une diff\'erence de nature importante entre d'une part les 
lois asymptotiques telles que la loi normale et la loi de Poisson, et  
d'autre  part les lois limite de processus en cascade. En effet, les 
premi\`eres  ont un caract\`ere universel: ainsi la somme d'un grand 
nombre de variables al\'eatoires ind\'ependantes aura toujours une 
densit\'e  gaussienne, quelle que soit la loi particuli\`ere des variables 
individuelles. De m\^eme  le nombre de v\'ehicules qui se pr\'esentent  
par minute \`a un p\'eage sera toujours une loi de Poisson, quelle que 
soit la loi de probabilit\'e particuli\`ere qui r\'egit l'heure ou le lieu de 
d\'epart  de chaque v\'ehicule.  Par contre les secondes ---~comme le 
montre la diversit\'e des  figures \`a la fin du chapitre {\bf VIII}~---  
sont tr\`es vari\'ees et d\'ependent de la loi g\'en\'eratrice $Z_1$. Mais  
les premi\`eres sont des sommes  de variables ind\'ependantes (ce  
qui  a  pour effet de brouiller l'information contenue dans les lois 
particuli\`eres), tandis  que les secondes ne sont  pas des sommes de 
termes ind\'ependants, et amplifient au contraire certaines 
caract\'eristiques de la loi originelle. 
 
\bigskip 
 
{\bf IX.\aub 1. Loi de Cauchy.} 
\medskip 
Soient $X$ et $Y$ deux variables al\'eatoires de lois approximativement 
gaussiennes. Par exemple $X$ et $Y$ peuvent \^etre bin\^omiales ou  
\^etre  la somme d'un grand nombre de termes ind\'ependants, mais on 
supposera que la probabilit\'e pour que $Y = 0$ est nulle.  
\medskip 
Si $X$ et $Y$ sont  stochastiquement ind\'ependantes, quelle est la loi 
appro\-ximative (la densit\'e) de $Z = X / |Y|$?  
\medskip 
Le probl\`eme est de trouver la densit\'e de la loi de $Z$ sans  
passer  par les lois exactes, \`a partir de la seule connaissance des 
densit\'es de $X$ et de $Y$.  Ces densit\'es sont caract\'eris\'ees par 
$$\eqalign{ 
{\cal P}\, (a < X < b)\;  &\simeq\; \int_a^b\;\sqrt{\up\alpha\sdown{4}  
\over\down\pi}\; \e^{-\alpha x^2}\; dx  \cr 
{\cal P}\, (a < Y < b)\;  &\simeq\; \int_a^b \;\sqrt{\beta\over\ldown\pi}\;  
\e^{-\beta y^2}\; dy  \cr } \eqno (IX.1.)$$ 
\medskip 
Supposons un instant qu'on connaisse les lois discr\`etes exactes de $X$  
et de $Y$: ${\cal P}\, (X=x_j) = p_j$ et ${\cal P}\, (Y=y_k) = q_k$. Alors 
$${\cal P}\, (a < Z < b)\; = \,\sum_{j,k \in E} p_j q_k \eqno (IX.2.)$$ 
o\`u $E$ est l'ensemble des couples d'indices $j,k$ tels que $a|y_k| < x_j  
< b|y_k|$. Bien entendu, que les termes de la somme ci-dessus soient  
les produits $p_j q_k$ est d\^u \`a l'ind\'ependance stochastique des 
variables $X$ et $Y$: sinon on aurait une loi conjointe $r_{j,k}$ non 
factorisable. 
\medskip 
Quoique \'etant \`a deux dimensions, cette loi conjointe peut \^etre  
liss\'ee par convolution: apr\`es lissage, la loi discr\`ete $\{ x_j , p_j  
\}$ donnera une densit\'e  ${\scriptstyle \sqrt{ \alpha / \pi }}\,  
\e^{-\alpha x^2}$ et la loi $\{ y_k , q_k \}$ une densit\'e ${\scriptstyle 
\sqrt{\beta / \pi }}\, \e^{-\beta  y^2}$, de sorte que la loi conjointe $\{ 
x_j, y_k , p_j q_k \}$ donnera  une  densit\'e ${\scriptstyle 
\big(\sqrt{\alpha\beta} \big/ \pi\big)}\, \e^{-\alpha x^2 - \beta y^2}$. 
Par ailleurs, l'\'ev\'enement $a < X/|Y| <b$ correspond,  dans le plan des 
coordonn\'ees $x,y$, au domaine $a|y| < x < b|y|$, donc l'approximation   
des lois discr\`etes par des densit\'es  donnera  
$${\cal P}\, (a < Z < b)\quad \simeq \quad\int\int_{\lower6pt 
\hbox{\hskip-29pt $\scriptstyle a|y| < x < b|y|$}}\! {\sqrt{ 
\sdown{8}\alpha\beta}\over \ldown{\pi }}\; \e^{-\alpha x^2  -   
\beta y^2}\;  dx\, dy \eqno (IX.3.)$$   
Il ne reste plus qu'\`a calculer cette int\'egrale double. 
\medskip 
 
\midinsert 
\vskip3pt
\centerline{\epsfbox{../imgEPS/ch09eps/fig34.eps}} 
\vskip2mm
\centerline{\eightpoint figure 34} 
\vskip4mm
\endinsert 

Pour cela il suffit d'effectuer de bons changements de variable; la  
figure 34 montre quel est le domaine d'int\'egration $a|y| < x < b|y|$ 
(secteur hachur\'e, d\'elimit\'e par les droites $x = a|y|$ et $x = b|y|$).  
La figure montre le cas o\`u $a$ et $b$ sont positifs. Introduisons les 
angles  $u$ et $v$ tels que $\tan u = a$ et $\tan v = b$. En coordonn\'ees 
polaires $r, \theta$ telles que $x = r\sin\theta$ et $y = r\cos\theta$ le 
secteur  hachur\'e correspond \`a $0 \leq r < \infty$, $u < \theta < v$.  
Par cons\'equent, dans ces coordonn\'ees polaires l'int\'egrale 
pr\'ec\'edente  devient:  
$$2\; {\sqrt{\sdown{8}\alpha\beta } \over \ldown{\pi }}\int_0^\infty 
\!\!\!\!\int_u^v \e^{-[\alpha\sin^2\theta  + \beta\cos^2\theta ]\, r^2}\;  
r\, dr\, d\theta$$ 
Le facteur 2 devant provient de ce que dans $IX.3$ on int\`egre aussi  
sur le secteur $-ay < x < -by$ correspondant \`a $y < 0$ alors que 
ci-dessus (voir figure 34) on n'a retenu que le cas $y>0$. L'int\'egration 
par rapport \`a la variable $r$ est imm\'ediate, et on obtient 
$${\sqrt{\sdown{8}\alpha\beta } \over \ldown{\pi }}\int_u^v 
{1 \over \alpha\sin^2\theta  + \beta\cos^2\theta }\; d\theta$$ 
On effectue alors un second changement de variable: $t = \tan\theta$; 
l'int\'egrale devient maintenant 
$${\sqrt{\sdown{8}\alpha\beta } \over \ldown{\pi }}\int_a^b 
{1 \over \alpha\, t^2  + \beta }\; dt$$ 
On peut donc \'ecrire en conclusion: 
$${\cal P}\, (a < Z < b)\; \simeq\; {\sqrt{\sdown{8}\alpha\beta } \over  
\ldown{\pi }} \int_a^b {1 \over \alpha\, t^2  + \beta }\; dt \eqno (IX.4.)$$ 
ce qui signifie simplement que la variable $Z$ a pour densit\'e la  
fonction  
$${\sqrt{\sdown{8}\alpha\beta }\over\ldown{\pi }}\, 
{1\over\alpha\, t^2  + \beta }$$   
On pourra remarquer que  
$${\sqrt{\sdown{8}\alpha\beta } \over \ldown{\pi }}\int_{-\infty}^{+ 
\infty} {1 \over \alpha\, t^2  + \beta }\; dt\; = \; 1$$ 
comme il se doit. 
\medskip 
Cette densit\'e (ou loi asymptotique) est appel\'ee {\it densit\'e de  
Cauchy}. 
\medskip 
La densit\'e de Cauchy d\'ecro{\^\i}t lentement \`a l'infini (en $t^{-2}$).  
Cela entra{\^\i}ne que la moyenne ou esp\'erance math\'ematique est 
donn\'ee par une int\'egrale divergente, $\int \big[ t / (\alpha\, t^2 
+\beta^2)\big]\, dt$; \`a plus forte raison encore la variance. Mais bien 
s\^ur la loi discr\`ete exacte de $Z$, n'ayant qu'un nombre fini de 
valeurs possibles, a bien une moyenne, qui est d'ailleurs pratiquement 
nulle, et une variance (tr\`es grande).  
\medskip 
Les variables $X$ et $Y$ avaient chacune une loi de probabilit\'e  
gaussienne (de variances $1/ 2\alpha$ et $1/ 2\beta$ respectivement). 
Pour donner une illustration concr\`ete \`a cela, imaginons que deux 
personnes ${\cal A}$ et ${\cal B}$ jouent chacune \`a pile ou face avec 
des partenaires respectifs ${\cal A}'$ et ${\cal B}'$,  en payant $1\,$  
euro au partenaire chaque fois que pile sort,  et en recevant $1\,$  
euro du partenaire chaque fois que face sort.  Sur un grand nombre 
$N$ de lancers,  il est tr\`es peu probable que les gains ou les pertes 
de ${\cal A}$ soient sup\'erieures \`a trois ou quatre fois $\sqrt{N}$  
(voir chapitre {\bf VII}):  c'est la loi gaussienne.  Mais supposons 
que le joueur ${\cal A}$,  au lieu de compter son gain en euros,  le rapporte 
au gain en valeur absolue de ${\cal B}$,  qu'il dise par exemple:  ``j'ai 
gagn\'e le double de ${\cal B}$'' ($Z = 2$),  ou bien ``j'ai gagn\'e $3.7$  
fois ce que ${\cal B}$ a perdu'' ($Z = 3.7$),  ou encore ``j'ai perdu $1.75$ 
fois plus que ${\cal B}$'' ($Z = -1.75$),  etc.  Afin qu'il n'y ait pas de 
division par z\'ero,  on supposera $N$ impair,  car on sait (voir chapitre  
{\bf III}) que dans ce cas les gains sont toujours un nombre impair
d'euros et ne peuvent donc pas \^etre nuls.  Alors ce gain relatif est 
distribu\'e selon la densit\'e de Cauchy.  Or la densit\'e de Cauchy, 
au contraire de la densit\'e gaussienne,  ne d\'ecro{\^\i}t pas tr\`es 
rapidement lorsque $t$ tend vers l'infini.  Les rapports $X / |Y|$ \'elev\'es 
ont certes des probabilit\'es plus faibles que les rapports proches de  
$1$,  mais pas infinit\'esimales.  Cela n'est pas paradoxal:  $X / |Y|$ est 
\'elev\'e si $X$ est grand (ce qui est  tr\`es peu probable),  mais aussi si 
$Y$ est petit (ce qui par contre est probable). 
\medskip 
La variable al\'eatoire $Z$,  c'est-\`a-dire le rapport des gains de  
${\cal A}$ \`a ceux (en valeur absolue) de ${\cal B}$ est bien s\^ur une 
variable al\'eatoire discr\`ete,  qui ne prend jamais qu'un nombre fini de 
valeurs;  ces valeurs sont des fractions et sont d'autant plus nombreuses 
et serr\'ees que $N$ (le nombre de lancers) est plus grand.  Pour $N$  
grand ces valeurs sont alors bien trop nombreuses et proches les unes  
des autres pour que la connaissance math\'ematiquement exacte de la 
probabilit\'e de chacune soit utile;  en outre son expression exacte est 
compliqu\'ee.  C'est pourquoi la densit\'e de $Z$ est \`a nouveau la seule 
information int\'eressante.  
 
\bigskip 
 
{\bf IX.\aub 2. Pourquoi la densit\'e gaussienne est-elle si r\'epandue ?} 
\medskip 
Le calcul par lequel nous avons obtenu la densit\'e de Cauchy 
n'\'etait qu'une affaire de changements de variables. Il laisse donc le 
sentiment que n'importe quelle fonction positive et dont l'int\'egrale  
vaut 1 peut \^etre la densit\'e d'une variable al\'eatoire gaussienne,  
apr\`es avoir effectu\'e sur celle-ci une transformation non lin\'eaire 
ad\'equate. Nous retrouvons ici encore le ph\'enom\`ene qui \'etait \`a  
l'origine du paradoxe des cordes prises au hasard sur un cercle: si une 
certaine  variable al\'eatoire a une densit\'e uniforme, son carr\'e ou sa 
racine carr\'ee  n'auront pas une densit\'e uniforme. Ici, au lieu de 
densit\'es uniformes, nous consid\'erons ce qu'il advient des densit\'es 
gaussiennes apr\`es une transformation non lin\'eaire. N'importe quoi 
(pourvu que ce soit positif et d'int\'egrale 1) pourrait apr\`es une 
transformation non lin\'eaire ad\'equate se ramener \`a une densit\'e 
gaussienne. D'un point de vue purement math\'ematique, cela est vrai; 
donnons-nous en effet une densit\'e arbitraire $f(x)$. Tout ce que nous  
lui imposons est que 
$$f(x) \;\geq\; 0 \qquad\hbox{et}\qquad\int f(x)\, dx\; = \; 1$$
Prenons alors une variable al\'eatoire $X$ de densit\'e gaussienne 
$\sqrt{\sdown{8}\smash{\eightpoint \alpha / \pi }}\,\e^{-\alpha\, x^2}$ et
posons $Y = h(X)$,  $h$ \'etant une transformation non lin\'eaire que nous 
supposerons inversible (disons croissante pour fixer les id\'ees), avec  
la fonction inverse $h^{-1} = g$: $X = g(Y)$.  On peut alors \'ecrire: 
$${\cal P}\, (a < Y < b)\quad = \quad {\cal P}\, (a < h(X) < b) \quad\simeq 
\quad\int_{\lower6pt \hbox{\hskip-20pt $\scriptstyle a < h(x) < b$}} 
\sqrt{\sdown{10.5}\strup{5}\smash{\up{\eightpoint \alpha}\sdown{4}
\over\down\pi}} \,\e^{-\alpha\, x^2}\; dx$$ 
En effectuant le changement de variable $x = g(y)$ dans cette int\'egrale 
on obtient 
$${\cal P}\, (a < Y < b) \;\simeq\; \int_a^b \sqrt{\sdown{10.5}\strup{5}
\smash{\up{\eightpoint \alpha}\sdown{4} \over\down\pi}} \,
\e^{-\alpha\, g(y)^2}\; g'(y)\; dy \eqno (IX.5.)$$  
Cette \'egalit\'e met bien en \'evidence que la densit\'e de $Y = h(X)$
est la fonction $\sqrt{\sdown{8}\smash{\eightpoint \alpha / \pi }}
\,\exp\{-\alpha\, g(y)^2\}\; g'(y)$,  donc pour que cette densit\'e soit la
densit\'e prescrite $f(y)$ il suffit d'avoir  
$$\sqrt{\sdown{10.5}\strup{5} \smash{\up{\eightpoint \alpha}\sdown{4} \over
\down\pi}} \,\e^{-\alpha\, g(y)^2}\; g'(y) \; = \; f(y)  \eqno (IX.6.)$$  
ce qu'on peut interpr\'eter comme une \'equation diff\'erentielle dont la 
fonction inconnue est $g$. Elle est facile \`a r\'esoudre en  introduisant  
la fonction  
$${\cal N}\, (x) \; = \; \int_{-\infty }^x \sqrt{\sdown{10.5}\strup{5} \smash{
\up{\eightpoint \alpha}\sdown{4} \over\down\pi}} \,\e^{-\alpha\, t^2}\; dt$$  
car alors l'\'equation diff\'erentielle peut s'\'ecrire sous la forme  
\'equivalente  
$${d \over dy}\, {\cal N}\, \big( g(y) \big) \; = \; f(y)$$ 
En introduisant la primitive $F(y)$ de $f(y)$ (celle qui s'annule pour 
$y = -\infty$) l'\'equation se r\'esoud par quadrature: 
$$g(y) \; = \; {\cal N}^{-1}\big( F(y)\big) \eqno (IX.7.)$$  
La fonction ${\cal N}$ est en effet croissante, donc a un inverse 
${\cal N}^{-1}$.  
\medskip 
En conclusion, si $Y$ est une variable al\'eatoire densit\'e arbitraire  
$f(y)$, il suffit de choisir $g(y)$ selon $IX.7$ pour que la variable $X = 
g(Y)$  soit de densit\'e gaussienne. Cela est un r\'esultat purement 
math\'ematique, qui ne signifie  \'evidemment pas que n'importe quelle 
densit\'e se rencontre dans des probl\`emes concrets. L'exp\'erience 
montre que les densit\'es gaussiennes sont fr\'equentes dans la nature, 
alors que la plupart des autres qu'on pourrait imaginer 
math\'ematiquement ne se rencontrent  pas. Or, s'il suffit d'observer le 
carr\'e, la racine carr\'ee, ou le  logarithme, pour que la densit\'e cesse 
d'\^etre gaussienne, on peut se demander pourquoi on rencontre si souvent la
densit\'e gaussienne. Cette question m\'erite une discussion quelque peu
approfondie.  Lorsqu'on rencontre des distributions de probabilit\'es ``dans
la nature'',  c'est sous la forme d'une distribution statistique;  en effet, 
on peut faire beaucoup d'exp\'eriences d\'eterministes,  par exemple
si on mesure dans un conducteur donn\'e le rapport de l'intensit\'e
\`a la tension entre les bornes,  on trouvera toujours pratiquement
la m\^eme valeur.  Dans un tel cas on ne dit pas que le r\'esultat
est al\'eatoire.  Par contre si on ne retient pas la valeur mesur\'ee
de l'intensit\'e,  mais seulement le  petit \'ecart,  de l'ordre des
erreurs de mesure,  par rapport \`a cette  valeur,  on constate une 
grande variabilit\'e et on dit que cet \'ecart est al\'eatoire.  De m\^eme, 
si on enregistre le nombre de naissances qui ont lieu chaque jour dans 
une maternit\'e,  on observera une certaine variabilit\'e:  certains jours 
il y aura eu cinq naissances,  d'autres jours dix;  toutefois les jours sans 
aucune naissance,  ou les jours avec vingt naissances,  seront rarissimes. 
Dans ce cas, on dira que le nombre de naissances est al\'eatoire,  mais  
bien s\^ur tous les nombres ne sont pas \'equiprobables et certains sont 
plus fr\'equents que d'autres.  On peut donc repr\'esenter le nombre de 
naissances par jour par une {\it variable al\'eatoire}.  Si au lieu de 
prendre dix valeurs possibles,  la variable al\'eatoire en prend un 
nombre tr\`es grand,  par exemple le nombre de naissances par semaine  
et non par jour,  on en repr\'esentera la r\'epartition par une densit\'e.  
On observera alors que le nombre de naissances par semaine,  qui varie 
d'une semaine \`a l'autre,  suit une densit\'e gaussienne,  avec un 
\'ecart-type qu'on pourra mesurer en faisant une \'etude statistique sur 
dix ans (bien s\^ur il faudra la corriger par rapport aux \'evolutions 
saisonni\`eres).  Il semblerait farfelu de vouloir plut\^ot \'etudier
la r\'epartition statistique des carr\'es ou des logarithmes de 
ces nombres,  qui auraient alors une densit\'e de r\'epartition autre,  
mais on constate que c'est justement avec la variable qui semble la  
plus naturelle qu'on observe la densit\'e gaussienne.   
\medskip 
D'autre part,  si le nombre qu'on mesure est petit (par exemple si dans la  
maternit\'e on compte le nombre de naissances {\it par jour} et non par  
an),  on ne peut obtenir une densit\'e gaussienne pour la simple raison
que les valeurs possibles ne sont pas assez nombreuses:  une densit\'e  
n'appara{\^\i}t que si les valeurs de la variable al\'eatoire sont  
nombreuses.  Mais on observera presque toujours dans de tels cas une loi  
de Poisson (voir section suivante).  Que ces deux lois,  la loi normale pour  
les variables qui prennent beaucoup de valeurs,  et la loi de Poisson pour  
les variables qui prennent des valeurs enti\`eres et petites,  soient aussi  
r\'epandues dans la nature,  doit pouvoir se comprendre. 
\medskip 
Les conclusions du chapitre {\bf VII} conduisent \`a penser que la 
distribution gaussienne autour d'une valeur moyenne est due \`a  
l'addition d'un grand nombre d'effets ind\'ependants les uns des autres 
(et nous verrons plus loin que c'est le cas aussi pour la loi de Poisson).  
Par ailleurs,  le constat purement math\'ematique que nous avons fait 
ci-dessus nous conduit \`a penser que de telles grandeurs \`a fluctuations 
gaussiennes devraient \^etre la somme d'un grand nombre d'effets 
ind\'ependants,  et non le carr\'e ou le logarithme d'une telle somme.   
\vskip 6pt plus5pt minus4pt 
Lorsqu'on d\'efinit une grandeur physique,  on ne la choisit pas 
arbitrairement.  L'intensit\'e d'un courant \'electrique pourrait,  
d'un point de vue purement m\'etrologique,  \^etre tout aussi bien  
mesur\'ee par sa racine carr\'ee ou son carr\'e: au lieu de la d\'efinir 
comme proportionnelle \`a la tension ($I = U/R$),  on la d\'efinirait  
comme proportionnelle \`a la puissance ($I^2 = P/R$).  Mais en agissant  
de la sorte \`a tort et \`a travers,  on changerait compl\`etement la  
nature des lois physiques et on perdrait les invariances 
spatio-temporelles (voir au chapitre {\bf I} les remarques 
---~emprunt\'ees \`a H. Poincar\'e~---  \`a propos du principe de 
relativit\'e de Galil\'ee et la mesure du temps).  Les grandeurs  
physiques ne sont pas d\'efinies \`a la l\'eg\`ere et sont choisies de 
mani\`ere \`a  refl\'eter les invariances fondamentales,  sans lesquelles  
la physique perdrait son sens;  plus exactement:  elles sont choisies de 
mani\`ere \`a avoir les lois les plus simples possibles,  et les 
invariances fondamentales sont l'expression de ce choix.  Lorsque des 
variations al\'eatoires {\it ind\'ependantes} s'ajoutent,  elles expriment 
implicitement ces invariances:  soit il s'agit d'une accumulation au cours 
du temps,  soit d'une addition de d\'eplacements dans l'espace.  En toute 
derni\`ere instance,  une grandeur se ram\`ene toujours \`a cela puisque, 
quel que soit l'instrument de mesure,  on lira toujours la valeur 
mesur\'ee sur une graduation,  un \'ecran, etc.  L'ind\'ependance 
stochastique de ces varia\-bles qui s'ajoutent spatialement les unes aux 
autres est une expression de la causalit\'e et de la s\'eparabilit\'e  
dans le temps et l'espace:  que dans n'importe quel  ph\'enom\`ene 
complexe on observe  des fluctuations gaus\-siennes (erreurs  de 
mesures, dispersion al\'eatoire, etc.) refl\`ete d'une part que de 
nombreux effets s'ajoutent spatialement,  et d'autre part que ces  
nombreux effets sont s\'eparables dans le temps ou dans l'espace, 
de sorte qu'ils ne s'influencent pas mutuellement (comme les boules  
qui tombent dans les bo{\^\i}tes sans s'influencer). 
\medskip 
Nous avons cependant \'etudi\'e des situations quantiques o\`u cette 
s\'epa\-ra\-bi\-lit\'e n'existe pas (par exemple {\bf II. 6}  
{\it la loi de Planck}) et pourtant on y rencontre aussi des fluctuations  
gaussiennes;  l'analyse math\'ematique d\'etaill\'ee (voir {\bf II.6.})  
faisait appara{\^\i}tre ces fluctuations gaussiennes comme une 
propri\'et\'e des factorielles.  Or les factorielles interviennent par 
l'inter\-m\'e\-diaire des formules de d\'enombrement (coefficients 
bin\^omiaux, etc.),  qui s'appli\-quent d\`es lors qu'une {\it 
\'equiprobabilit\'e} a \'et\'e postul\'ee.  Mais l'\'equiprobabilit\'e 
pr\'esuppose toujours une invariance,  comme cela a \'et\'e largement 
discut\'e au chapitre {\bf I}.  Dans les statistiques quantiques,  il n'y a  
pas de s\'eparabilit\'e,  mais des invariances de nature diff\'erente (et 
aujourd'hui encore myst\'erieuses).  Dans tous ces cas,  aussi bien lorsque 
des causes tr\`es nombreuses et causalement s\'epar\'ees (au sens de la 
M\'ecanique classique) s'ajoutent parce que leur effet est spatial,  que 
lorsqu'on analyse les statistiques quantiques o\`u une invariance  
d'origine autre que spatiale conduit \`a des factorielles,  on rencontrera 
la forme gaussienne.  On pourrait dire que la forme gaussienne des 
fluctuations statistiques est le sympt\^ome d'une invariance 
sous-jacente. 
\medskip 
Il n'est donc pas surprenant que les fluctuations al\'eatoires qu'on  
observe dans la nature soient si souvent gaussiennes.    
\medskip 
La discussion serait bien s\^ur plus concr\`ete sur un exemple;  en voici  
un.  Pour mesurer la quantit\'e d'information qu'on peut stocker sur une 
m\'emoire on utilise une unit\'e qui est l'octet (ou le bit);  dans un octet 
on peut ranger 256 signes diff\'erents,  alors pourquoi ne dit-on pas que  
la quantit\'e d'information est 256?  Sur $1\, Mo$ on peut ranger environ 
$4.26 \cdot 10^{2525222}$ textes diff\'erents,  mais on pr\'ef\`ere 
mesurer la quantit\'e d'information par le logarithme en base 2 de ce 
nombre;  c'est certes plus commode,  mais est-ce seulement une question 
de commodit\'e?    
\medskip 
En analysant les choses de plus pr\`es,  on peut remarquer que la  
m\'emoire mesur\'ee en octets est proportionnelle au volume que la 
m\'emoire physique occupe dans l'espace (\`a sa longueur si la m\'emoire 
est stock\'ee sur une ligne,  \`a la superficie si elle est stock\'ee sur un 
disque, \`a un volume si la m\'emoire est stock\'ee en trois dimensions). 
Ainsi la quantit\'e d'information mesur\'ee en octets se ram\`ene \`a de 
l'espace,  alors que la quantit\'e mesur\'ee par le nombre de textes 
possibles ne s'y ram\`ene pas lin\'eairement.  Si on remplit une m\'emoire 
progressivement en y vidant une centaine de fois un buffer,  l'espace 
disponible ne sera pas rempli compl\`etement,  il restera toujours des  
trous d\^us au fait que l'adressage des buffers successifs ne co{\"\i}ncide 
pas avec un pavage g\'eom\'etriquement parfait de l'espace disponible. 
Ces trous auront donc un volume assez petit,  mais qu'on peut  
avec raison consid\'erer comme al\'eatoire;  il n'est peut-\^etre pas 
absolument \'evident que les trous successifs soient stochastiquement 
ind\'ependants,  mais si le remplissage de l'espace-disque est 
suffisamment chaotique,  c'est une hypoth\`ese qui se tient,  parce qu'on 
peut admettre ---~en vertu justement du principe d'invariance~--- qu'il 
n'y a pas de r\'egion privil\'egi\'ee de l'espace (voir \`a ce sujet {\bf IV. 
6} {\it l'effacement de la  causalit\'e}).  
\medskip 
Apr\`es avoir vid\'e quelques centaines de fois le buffer,  le volume  
total de ces trous sera une variable al\'eatoire de densit\'e gaussienne 
(somme des volumes des trous accumul\'es),  dont l'\'ecart-type 
repr\'esente alors une incertitude sur la m\'emoire disponible.  Si on 
avait compt\'e en nombre de textes possibles,  on n'observerait pas une 
fluctuation gaussienne,  puisque la relation entre les deux grandeurs est 
logarithmique et non lin\'eaire.  Mais le niveau o\`u se produit une 
{\it addition} de variables ind\'ependantes est celui du remplissage  
de l'espace,  et non celui du nombre de textes possibles,  qui serait 
multiplicatif.  Ainsi, la grandeur qui a \'et\'e choisie pour mesurer 
l'information est celle qui refl\`ete le remplissage spatial et par 
cons\'equent aussi celle qui pr\'esentera des fluctuations gaussiennes.  
\medskip 
On pourra faire le m\^eme constat pour l'intensit\'e \'electrique (nombre 
d'\'electrons passant {\it par seconde} \`a travers  une {\it section} de 
conducteur), etc.  
\medskip 
C'est la v\'eritable raison de l'universalit\'e de la densit\'e gaussienne. 
Il s'y ajoute encore une raison secondaire,  mais tr\`es importante: 
m\^eme lorsqu'une grandeur ne se ram\`ene pas lin\'eairement \`a des 
d\'eplacements dans l'espace,  ou plus g\'en\'eralement \`a des formules  
de d\'enombrement contenant des factorielles,  les fluctuations peuvent 
rester gaussiennes tout simplement parce qu'elles sont petites.  En 
effet,  soit $x$ une grandeur subissant des fluctuations gaussiennes 
$\varepsilon$:  on mesure donc $x$,  mais avec un bruit $\varepsilon$  
qui s'ajoute \`a $x$.  Si on consid\`ere maintenant la grandeur $y = f(x)$  
qui d\'epend non lin\'eairement de $x$,  celle-ci fluctuera selon 
$f(x+\varepsilon )$.  Si $\varepsilon$ est grand,  la non lin\'earit\'e de  
$f$ d\'eformera les fluctuations,  de telle sorte que celles-ci ne seront 
plus gaussiennes,  comme l'\'etude pr\'ec\'edente l'a montr\'e.  Mais si 
$\varepsilon$ est petit et $f$ diff\'erentiable,  on aura $f(x+\varepsilon ) 
\simeq f(x) + f'(x)\, \varepsilon$,  donc la fluctuation sera n\'eanmoins 
transform\'ee lin\'eairement.  En fin de compte,  il est m\^eme plut\^ot 
difficile de trouver des grandeurs qui \'echappent \`a toutes ces bonnes 
raisons. 
\medskip 
Malgr\'e tout,  nous avons pu constater que les processus en cascade 
engendraient une bien plus grande vari\'et\'e de densit\'es 
asymptotiques;  mais la  variable $Z_n$ du chapitre {\bf VIII} n'est pas 
une somme de variables ind\'ependantes,  le principe de composition est 
diff\'erent et n'efface pas enti\`erement l'information sur la loi  
initiale de $Z_1$. 
 
\bigskip 
 
{\bf IX.\aub 3. Loi de Poisson.} 
\medskip 
La loi de Poisson est la plus simple de toutes les lois de probabilit\'e  
asymptotiques; elle est enti\`erement discr\`ete. 
\medskip 
Consid\'erons le probl\`eme tr\`es simple (et purement math\'ematique)  
sui\-vant:  on a $n$ variables al\'eatoires $X_1,\;  X_2, \;\ldots\; X_n$, 
stochastiquement ind\'e\-pen\-dan\-tes, et ne prenant que les deux valeurs $0$ 
et $1$ avec les probabilit\'es respectives $p$ et $q$ ($p + q = 1$). La loi  
de leur somme  $S = X_1 + X_2  + \cdots + X_n$ est  
$${\cal P}\, (S = k) = {n \choose k} p^{n-k} q^k$$ 
(loi de Bernoulli). Mais supposons que $n$ soit tr\`es grand et $q$ tr\`es  
petit, de l'ordre de $1/n$. On pourra poser $q = \lambda / n$ avec 
$\lambda$ ni grand ni petit. Alors, pour $k \ll n$, $p^{n-k} = (1 - \lambda 
/n)^{n-k} \simeq e^{-\lambda }$ et ${n \choose k} \simeq n^k/k!$ De sorte 
que  
$${\cal P}\, (S = k) \simeq {\lambda^k \over k!}\, \e^{-\lambda }  \eqno 
(IX.8.)$$ 
On constate d'abord que pour $k \gg \lambda$ ces probabilit\'es sont 
pratiquement nulles (aussi bien sous leur forme approch\'ee que sous leur 
forme exacte). Seules sont significatives les probabilit\'es correspondant 
aux petites valeurs de $k$. 
\medskip 
Plus g\'en\'eralement, supposons que les $X_j$ n'aient pas toutes 
exactement la m\^eme loi, mais que toutes, comme ci-dessus,  
prennent la valeur 1 avec une probabilit\'e tr\`es petite $\varepsilon_j$, 
et la valeur z\'ero avec une probabilit\'e $1 - \varepsilon_j$ presque  
\'egale \`a 1. 
\medskip 
La fonction g\'en\'eratrice de chaque $X_j$ \'etant $G_j(x) = 1 +  
\varepsilon_j\, (x-1)$, la somme $S = X_1 + X_2 + \cdots + X_n$ 
aura pour fonction g\'en\'eratrice (si les $X_j$ sont ind\'ependantes): 
$$G(x) = \prod_{j=1}^{j=n}  \bigl[ 1 + \varepsilon_j\, (x-1) \bigr]$$ 
Pour approcher ce type d'expression on prend bien s\^ur le logarithme; 
or  
$$\ln\big( 1 + \varepsilon_j\, (x-1) \big) \simeq 
\varepsilon_j\, (x-1) - {1\over 2} \varepsilon_j^2\, (x-1)^2$$ 
o\`u on a conserv\'e le terme d'ordre deux pour une \'evaluation de  
l'erreur. On obtient donc en revenant aux exponentielles: 
$$G(x) \simeq \exp\Big\{\sum_{j=1}^{j=n} \varepsilon_j\, (x-1) 
 - {1\over 2} \sum_{j=1}^{j=n}\varepsilon_j^2\, (x-1)^2\Big\}$$ 
Introduisons les param\`etres 
$$\lambda = \sum_{j=1}^{j=n} \varepsilon_j \qquad\hbox{et}\qquad 
\mu = \sum_{j=1}^{j=n} \varepsilon_j^2 \eqno (IX.9.)$$ 
Le premier, $\lambda$, est la somme des moyennes des $X_j$, donc 
la moyenne de $S$. On a suppos\'e les $\varepsilon_j$ tr\`es petits, 
supposons maintenant que l'ordre de grandeur de $n$ est tel que la  
somme $\lambda$ ait une valeur appr\'eciable. Alors, les carr\'es 
$\varepsilon_j^2$ \'etant bien plus petits que les $\varepsilon_j$, leur 
somme sera tr\`es petite par rapport \`a $\lambda$. Ainsi la fonction 
g\'en\'eratrice $G(x)$ sera  approximativement donn\'ee par 
$$G(x) \simeq \exp\big\{\lambda\, (x-1)\big\} \eqno (IX.10.)$$ 
avec une erreur relative de l'ordre de $- {1\over 2} \mu\, (x-1)^2$. 
Les probabilit\'es pour que $S = 0,1,2,3, \ldots$ sont donc \`a nouveau 
donn\'ees par la loi de Poisson, avec une erreur relative de l'ordre de 
$\mu$. Toutes ces approximations ne sont justifi\'ees que pour les  
petites valeurs de $S$ et non pour $S=1\, 000$ ou $S=10\, 000$, mais 
pour les grandes valeurs, bien que l'erreur {\it relative} ne puisse plus 
\^etre consid\'er\'ee comme petite, les probabilit\'es aussi bien exactes 
qu'approch\'ees sont de toute fa\c{c}on si prodigieusement petites (ce que 
montrent les factorielles au d\'enominateur de la loi de Poisson) que  
cela n'a pas d'importance. 
\medskip 
Nous voyons donc appara{\^\i}tre le r\'esultat suivant: lorsqu'on  
consid\`ere une somme d'un grand nombre $n$ de variables 
ind\'ependantes qui ne prennent que les deux valeurs $0$ et $1$, avec  
une tr\`es faible  probabilit\'e (de l'ordre de $1/n$) pour $1$, alors leur 
somme ob\'eit \`a la loi de Poisson. 
\medskip 
A premi\`ere vue,  on pourrait penser que cela contredit les conclusions  
du chapitre {\bf VII}:  on devrait en effet avoir une loi gaussienne. 
Mais en \'etudiant le passage \`a la limite, nous avons remarqu\'e que, 
surtout si la loi des $X_j$ est fortement dissym\'etrique ---~ce qui est
le cas ici~--- il fallait que le nombre $N$ de variables \`a sommer soit 
vraiment tr\`es grand pour arriver \`a la loi  gaussienne;  nous allons 
voir qu'il doit \^etre beaucoup plus grand que le nombre $n$ qui 
intervient ici.  Nous en avons donn\'e une estimation explicite ($VII.5$):  
il fallait que $N$ soit grand par rapport \`a $400\, M_3^2 / M_2^3$,  
$M_2$  et $M_3$ \'etant les moments d'ordre deux et trois de la  loi de 
$X$.  Ces moments ne sont pas les m\^emes pour toutes les $X_j$; mais
pour chaque $X_j$,  ils sont tous trois \'egaux \`a $\varepsilon_j$
($M_1 = M_2 = M_3 = \varepsilon_j$),  et sous nos hypoth\`eses ils ont
donc tous le m\^eme ordre de grandeur $1/n$.  De sorte que $M_3^2/M_2^3
\sim  n^3/n^2 = n$.  D'apr\`es $VII.5$ il faudrait donc, pour arriver
\`a la densit\'e gaussienne,  que $N \gg 400\, n$;  autrement dit, 
$n$ n'est pas assez grand et c'est pour cela qu'on obtient la loi
de Poisson au lieu de la densit\'e gaussienne.   
\medskip 
Un tel ph\'enom\`ene est tr\`es courant pour les lois asymptotiques:  il y  
en a qui sont vraiment des limites lorsque $n$ tend vers l'infini:  c'est le 
cas pour la loi normale du chapitre {\bf VII},  ainsi que pour la loi limite 
de $Z_n/E(Z_n)$ du chapitre {\bf VIII}.  Mais il y en a aussi,  comme ici
la loi de Poisson,  qui sont une \'etape interm\'ediaire sur la route vers 
l'infini.  Tant que $n$ est de l'ordre de grandeur des $1/\varepsilon_j$ on 
a la loi de Poisson,  puis,  lorsque $n$ poursuit son voyage vers l'infini et 
atteint un ordre de grandeur nettement sup\'erieur aux nombres 
$400/\varepsilon_j$,  on arrive \`a la loi gaussienne. 
\medskip 
On remarquera cependant que la loi de Poisson est aussi la loi d'une 
somme de variables al\'eatoires ind\'ependantes,  mais de variables qui  
ne prennent que les deux valeurs $0$ et $1$,  la seconde avec une faible 
probabilit\'e:  la somme de ces variables sera donc un nombre entier 
petit (comme le nombre de naissances par jour ou par heure dans une 
maternit\'e).  Les remarques pr\'esent\'ees \`a la section {\bf 2} \`a
propos de la loi normale expliquent donc aussi pourquoi la loi de Poisson
s'observe si fr\'equemment dans la nature,  d\`es lors que les nombres
qu'on mesure sont entiers et petits. 
 
\bigskip 
 
Jusqu'ici nous sommes rest\'es dans l'abstraction:  nous avons  
consid\'er\'e des sommes de variables al\'eatoires sans nous  
pr\'eoccuper de ce qu'elles peuvent repr\'esenter. 
\medskip 
Voici maintenant un probl\`eme concret:  les queues qui peuvent se 
former \`a un p\'eage d'autoroute.  Mettons qu'il faut une minute pour 
effectuer les paiements au guichet.  On supposera qu'il n'y a qu'un guichet; 
s'il y en a plusieurs,  le probl\`eme n'est pas essentiellement diff\'erent;  
la dur\'ee de paiement de une minute n'a ici aucune signification pratique,  
n'importe quelle unit\'e de temps e\^ut convenu.  On voudrait conna{\^\i}tre  
la loi de proba\-bilit\'e du nombre de voitures qui se pr\'esentent par 
minute,  sachant qu'en moyenne il s'en pr\'esente $0.76$.  L'int\'er\^et
de ce probl\`eme est le suivant:  cette moyenne est inf\'erieure \`a $1$, 
donc ``en moyenne'' il ne se forme pas de queue:  au bout de cent minutes, 
environ $76$ v\'ehicules sont pass\'es,  alors que le guichet aurait pu en
absorber cent.  Il est bien clair que si la moyenne avait \'et\'e $1.22$, 
il y aurait un r\'esidu d'environ $22$ v\'ehicules au bout de cent minutes, 
et on serait donc s\^ur qu'une queue se formerait et s'allongerait en
moyenne de $22$ v\'ehicules par cent minutes.  Toutefois,  lorsque la
moyenne est inf\'erieure \`a $1$,  la probabilit\'e qu'il se forme des
queues m\^eme longues n'est pas nulle;  si cette probabilit\'e est
sup\'erieure \`a $5\%$ pour une dur\'ee de trois heures ($180$ minutes)
par exemple,  la compagnie peut avoir int\'er\^et \`a am\'eliorer le p\'eage. 
En effet une probabilit\'e de $5\%$ pour trois heures signifie que toutes
les soixante heures environ se forme une queue longue,  soit tous les trois
jours.  Cela peut dissuader bon nombre d'usagers d'emprunter l'autoroute
et faire perdre de l'argent \`a la compagnie.  Celle-ci peut am\'eliorer
le p\'eage de deux fa\c{c}ons:  soit construire un second guichet,  ce qui
aura pour effet de faire tomber la moyenne de $0.76$ \`a $0.38$,  soit
am\'eliorer son fonctionnement pour que la dur\'ee de paiement soit
raccourcie:  si celle-ci est abaiss\'ee \`a une demi-minute,  la moyenne
tombera aussi \`a $0.38$.  Il est assez \'evident a priori que la
probabilit\'e de formation de queues de longueur fix\'ee doit \^etre
une fonction d\'ecroissante du nombre moyen de v\'ehicules par minute. 
Mais cette intuition a priori ne permet pas de conna{\^\i}tre la loi
de cette d\'ecroissance.  C'est cette loi que nous nous proposons de
d\'ecouvrir par le Calcul des probabilit\'es. 
\medskip 
Qu'il puisse se former des queues m\^eme lorsque le nombre moyen de 
v\'ehicules par minute est plus petit que $1$ n'est pas paradoxal, il suffit 
d'imaginer que si la moyenne  est $0.76$,  il peut parfois arriver deux, 
trois,  ou m\^eme quatre voitures en une seule minute.  Cela va alors faire  
durer les formalit\'es plus d'une minute,  puisqu'il y aura plus d'une  
voiture au guichet;  or pendant ces deux ou trois minutes vont 
arriver en moyenne encore deux ou trois fois $0.76$ voitures,  qui \`a
leur tour bloqueront les guichets deux minutes au lieu d'une,  pendant
lesquelles arriveront encore des voitures, etc.  
\medskip 
Pour aborder ce probl\`eme il faut commencer par trouver la loi de 
probabilit\'e pour qu'il arrive z\'ero, un, deux, trois, $\ldots$ v\'ehicules. 
C'est seulement avec la connaissance quantitative de cette loi qu'on 
pourra ensuite \'etudier quantitativement la loi de formation des queues. 
\medskip 
 Or il est facile de montrer que cette loi doit \^etre une loi de Poisson, du 
moins si on est assur\'e que le nombre de voitures par minute est petit 
(qu'il n'est pas de l'ordre de mille ou un million, auquel cas il faut 
proc\'eder autrement).  
\medskip 
Supposons d'abord que le flux est constant au cours de la journ\'ee; cela 
est toujours faux en pratique, o\`u il y a des heures de pointe 
(aux environs de huit heures ou de dix-huit heures) et des heures  
creuses, mais cette hypoth\`ese permettra d\'ej\`a de 
trouver la loi de probabilit\'e pour une moyenne constante; on agit de 
m\^eme lorsqu'on veut d\'efinir la vitesse en cin\'ematique: on commence 
par la d\'efinir  dans le cas o\`u le mouvement est uniforme, puis on 
\'etend la d\'efinition en partant du principe que tout mouvement  peut 
\^etre consid\'er\'e comme uniforme pendant un instant assez court. 
\medskip  
Sous cette hypoth\`ese, les automobilistes quittent leur 
domicile et prennent l'autoroute \`a des instants al\'eatoires,  mais 
uniform\'ement distribu\'es au long de la journ\'ee (car si ces instants 
n'\'etaient pas uniform\'ement distribu\'es on ne pourrait avoir un flux 
constant).  Chaque automobiliste agit ind\'ependamment des autres, donc 
les d\'eparts du domicile (ou les arriv\'ees au p\'eages) de tous ces  
usagers sont stochastiquement ind\'ependants.  
\medskip 
Lorsqu'on dit que le flux moyen au p\'eage est de $\lambda$ voitures par 
minute, cela signifie que si dans la journ\'ee $N$ voitures en tout sont 
pass\'ees, $\lambda = N / 1\, 440$, puisqu'il y a $1\, 440$ minutes dans 
une journ\'ee. Si on prend {\it au hasard} une voiture parmi les $N$, 
la probabilit\'e pour que cette voiture arrive au p\'eage au cours d'une  
minute donn\'ee parmi les $1\, 440$ que compte la journ\'ee est  
$\varepsilon = 1 / 1\, 440$. L'hypoth\`ese d'un flux constant se traduit  
en effet par l'\'equiprobabilit\'e a priori de chacune des minutes de la 
journ\'ee. Inversement, la probabilit\'e pour que cette voiture arrive en   
dehors de la minute donn\'ee, est  $1 - \varepsilon = 1\, 439 / 1\, 440$. 
On peut donc associer \`a chaque automobiliste $j$ ($j= 1,2,3, \ldots N$) 
une variable al\'eatoire $X_j$ qui vaut $1$ si l'automobiliste $j$ arrive  
au p\'eage au cours de la minute donn\'ee, et $0$ si l'automobiliste  
arrive au p\'eage en dehors de cette minute donn\'ee. Le fait que chaque 
automobiliste agit ind\'ependamment se traduit par l'ind\'ependance 
stochastique des $X_j$, et les $X_j$ ont toutes la m\^eme loi: 
$$X_j = \cases{1 &avec probabilit\'e $\varepsilon$; \cr 
                        0 &avec probabilit\'e $1 - \varepsilon$; \cr } \eqno 
(IX.11.)$$  
La somme des $X_j$, $S = \sum_{j=1}^N X_j$, est donc le nombre 
de voitures qui arrivent pendant la minute donn\'ee. Nous avons vu 
pr\'ec\'edemment que, si $N$ est du m\^eme ordre de grandeur que $1 / 
\varepsilon$, la loi de la somme $S$ est approximativement une loi de 
Poisson: 
$${\cal P}\, (S = k) \quad = \quad {\up{\lambda^k} \over k!}\,  
\e^{-\lambda } \eqno (IX.12.)$$  
Puisque $\lambda = N/1\, 440 = N\varepsilon$, la condition que $N$ 
soit du m\^eme ordre de grandeur que $1 / \varepsilon$ signifie 
simplement que $\lambda$ ne doit \^etre ni trop grand, ni trop petit. Ceci 
est la loi de probabilit\'e pour une minute donn\'ee, mais bien entendu 
toutes les minutes de la journ\'ee sont \'equivalentes et la loi sera la 
m\^eme pour n'importe quelle minute de la journ\'ee.  
\medskip 
Revenons alors \`a l'exemple qui a introduit la discussion, o\`u   
$\lambda$ valait $0.76$. Les valeurs num\'eriques de la loi de Poisson  
sont donn\'ees par le tableau suivant:  
$$\matrix{ 
\qquad k \qquad  &\qquad {\cal P}\, (S=k) \qquad \cr 
\noalign{\medskip} 
0  &0.4677 \cr 
1  &0.3554 \cr  
2  &0.1351 \cr 
3  &0.0342 \cr 
4  &0.0065 \cr 
5  &0.0010 \cr 
6  &0.0001 \cr 
7  &0.0000 \cr } \eqno (IX.13.)$$ 
Au-del\`a de $6$ les probabilit\'es sont inf\'erieures au dix-milli\`eme.  
En  faisant la somme des six derni\`eres, on voit que la probabilit\'e pour  
qu'il se pr\'esente deux v\'ehicules ou plus au p\'eage est $0.177$; la 
probabilit\'e pour qu'il se pr\'esente trois v\'ehicules ou plus est 
$0.0418$. Avec les capacit\'es d'absorption que nous avons pos\'ees, il  
suffit qu'une seule fois se pr\'esentent plus de deux v\'ehicules pour que 
l'engorgement momentan\'e ne puisse \^etre r\'esorb\'e par la suite que si 
on a la  chance d'avoir moins que deux v\'ehicules dans les minutes qui 
suivent; or la probabilit\'e pour qu'il se pr\'esente plus de deux v\'ehicules  
\'etant $0.177$, cela va se  produire en moyenne toutes les six 
minutes. Par cette estimation purement qualitative, on comprend que  
des bouchons  puissent se former bien que le flux moyen soit inf\'erieur  
\`a un v\'ehicule par minute. 
\medskip 
Nous nous proposons maintenant d'analyser cela quantitativement.  
 
\bigskip 
 
{\bf IX.\aub 4. La loi des queues.} 
 \medskip 
\`A chaque instant $n$ de la journ\'ee (les instants \'etant les minutes  
successives, $n = 1,2,3, \;\ldots\; 1\, 440$), la longueur de la queue  est 
\'egale au nombre de v\'ehicules  d\'ej\`a arriv\'es moins le nombre  de 
v\'ehicules d\'ej\`a autoris\'es. Mais le nombre de v\'ehicules d\'ej\`a 
autoris\'es n'est en g\'en\'eral  pas $n$, car si par exemple aucun 
v\'ehicule ne s'est pr\'esent\'e pendant les trois premi\`eres minutes, puis 
qu'il s'en pr\'esente trois d'un coup pendant la quatri\`eme minute, la 
longueur de la queue sera de deux v\'ehicules, et non de $0 + 0 + 0 + 3 - 4$; 
\`a la fin de la quatri\`eme minute, le nombre de v\'ehicules autoris\'es 
aura \'et\'e $1$ et non $n = 4$. En effet, si pendant une p\'eriode il se 
pr\'esente moins de v\'ehicules que pr\'evu, il y aura des temps morts au 
guichet, mais ces temps morts sont perdus et ne peuvent plus \^etre 
recompt\'es plus tard pour absorber un trafic plus dense. Il faut donc  
tenir compte de ces temps morts. 
\medskip 
Appelons $Q_n$ la variable al\'eatoire ``longueur de la queue \`a  
l'instant $n$'', et $X_n$ le nombre de v\'ehicules arriv\'es {\it pendant} 
la $n^{\rm e}$ minute (les $X_n$ ont toutes la loi de Poisson 
mentionn\'ee ci-dessus et sont ind\'ependantes entre elles). Si par 
exemple $Q_{n-1} = 1$ et qu'il  arrive z\'ero v\'ehicules pendant la 
minute suivante, $Q_n$ sera \'egale \`a $0$; dans ce cas on obtient 
$Q_n$ en ajoutant $X_n - 1$ \`a $Q_{n-1}$. De m\^eme si $Q_{n-1} = 0$  
et $X_n = 1$, $Q_n$ sera nulle: l\`a aussi $Q_n = Q_{n-1} + X_n  
- 1$. Par contre si $Q_{n-1} = 0$ et $X_n = 0$, on n'aura pas $Q_n = 
Q_{n-1} + X_n - 1 = -1$, mais $Q_n = 0$ (la queue ne peut jamais avoir 
une longueur n\'egative). Pour trouver la r\'ecurrence qui fait passer de  
$Q_{n-1}$ \`a $Q_n$, le mieux est de d\'ecomposer les \'ev\'enements  
en r\'eunions disjointes ad\'equates. Dans de tels probl\`emes {\it il ne 
faut jamais essayer de deviner},  mais toujours suivre un proc\'ed\'e 
syst\'ematique.   
\medskip 
L'\'ev\'enement $\{ Q_n = 0 \}$ est la r\'eunion de $\{ Q_{n-1} = 0 \}  
\cap  \{ X_n = 0 \}$, de $\{ Q_{n-1} = 0 \} \cap \{ X_n = 1 \}$, et de $\{ 
Q_{n-1} = 1 \} \cap \{ X_n = 0 \}$. Ces trois \'ev\'enements sont 
disjoints: on ne peut \'evidemment pas avoir \`a la fois $Q_{n-1} = 0$ et 
$Q_{n-1} = 1$, ni $X_n = 0$ et $X_n = 1$. D'autre part, les variables 
al\'eatoires $Q_{n-1}$ et $X_n$ sont stochastiquement ind\'ependantes 
(les v\'ehicules qui arrivent au cours de la $n^{\rm e}$ minute 
``ignoraient''  ce qui s'\'etait produit jusque l\`a). Donc  
$$\eqalign{ 
{\cal P}\, (Q_n = 0)\quad  
&= \quad {\cal P}\, (Q_{n-1} = 0\; \hbox{et}\; X_n = 0)\; +\cr  
&\hskip2cm + {\cal P}\, (Q_{n-1} = 0\; \hbox{et}\; X_n = 1)\; + \cr 
&\hskip2cm + {\cal P}\, (Q_{n-1} = 1\; \hbox{et}\; X_n = 0) \cr 
\noalign{\medskip} 
&= \quad {\cal P}\, (Q_{n-1} = 0) \times {\cal P}\, (X_n = 0)\; + \cr 
&\hskip2cm + {\cal P}\, (Q_{n-1} = 0) \times {\cal P}\, (X_n = 1)\; + \cr 
&\hskip2cm + {\cal P}\, (Q_{n-1} = 1)\times {\cal P}\, (X_n = 0)\cr } 
\eqno (IX.14.)$$ 
Cette expression de ${\cal P}\, (Q_n = 0)$ diff\`ere de ce qu'elle serait  
si $Q_n$ \'etait simplement la somme $Q_{n-1} + X_n - 1$: le premier  
des trois termes serait alors absent. 
\medskip 
En revanche, pour les probabilit\'es ${\cal P}\, (Q_n = k)$ avec $k \geq  
1$, tout se passe comme si $Q_n$ \'etait bien la somme $Q_{n-1} + X_n  
- 1$; $Q_n$ ne peut en effet \^etre \'egal \`a $k \geq 1$ que si $Q_{n-1}  
+ X_n =  k + 1$, car il n'y a pas dans ce cas de temps mort au guichet: il  
y a $Q_{n-1}$ v\'ehicules qui attendent ($Q_{n-1}$ pouvant \^etre  
z\'ero), il  en arrive $X_n$ de plus, et un passera. C'est seulement  
lorsque $Q_{n-1}$  et $X_n$ sont tous deux nuls qu'il aura un temps  
mort  et que $Q_n$ sera \'egal \`a $0$ au lieu de $-1$. Ainsi  
l'\'ev\'enement $\{ Q_n = k \}$ pour $k \geq 1$ sera la r\'eunion pour $j$ 
variant de $0$ \`a $k+1$ des \'ev\'enements disjoints $\{ Q_{n-1} = j \} 
\cap \{ X_n = k+1-j \}$, de sorte que $${\cal P}\, (Q_n = k) = 
\sum_{j=0}^{k+1} {\cal P}\, (Q_{n-1} = j) \times {\cal P}\, (X_n = k+1-j) 
\eqno (IX.15.)$$ tout comme si $Q_n$ \'etait simplement la somme de 
$Q_{n-1}$ et de $X_n - 1$. 
En regroupant $(IX.14.)$ et $(IX.15.)$ on obtient la r\'ecurrence qui fait 
passer de la loi de $Q_{n-1}$ \`a celle de $Q_n$. Cette r\'ecurrence  
permet d'\'ecrire un programme qui calculera r\'ecursivement les lois  
de $Q_n$ \`a partir de celle de $Q_1$, qui est \'evidemment la suivante: 
$$\eqalign{ 
{\cal P}\, (Q_1 = 0) &= {\cal P}\, (X_1 = 0) + {\cal P}\, (X_1 = 1)  
\quad\hbox{pour $k = 0$}\cr 
{\cal P}\, (Q_1 = k) &= {\cal P}\, (X_1 = k+1)\quad \hbox{pour $k\geq 1$} 
\cr }$$  
Si on ex\'ecute ce programme, on s'aper\c{c}oit que la loi de $Q_n$ se 
stabilise au bout d'une centaine d'it\'erations (voir figures 36, 37, 38),  
ce qui correspond au fait que la loi de $Q_n$ tend vers une limite  
lorsque $n$ tend vers l'infini. Le nombre d'it\'erations avant  
stabilisation d\'epend \'evidemment de la pr\'ecision:  plus on
prend en compte de d\'ecimales,  plus grand est ce nombre.  La centaine  
correspond \`a trois ou quatre d\'ecimales, et la limite n'est,  au sens 
math\'ematique, jamais atteinte.  Ainsi,  il s'\'etablit au bout d'un  
certain temps un r\'egime stable.  Mais ceci n'est vrai que si le flux 
moyen est inf\'erieur \`a un v\'ehicule par minute,  sinon \'evidemment 
les queues vont s'allonger ind\'efiniment (c'est-\`a-dire que la 
probabilit\'e d'avoir une queue inf\'erieure \`a n'importe quelle longueur 
fix\'ee tendra vers z\'ero). L'\'etablissement d'un r\'egime stable peut  
se comprendre a priori: dans le ph\'enom\`ene de formation des queues,  
il n'y a aucune raison de penser que la loi de probabilit\'e de $Q_n$ 
puisse d\'ependre de l'instant $n$, puisque les conditions (le flux moyen, 
la dur\'ee du paiement, etc.) sont suppos\'ees constantes; seul le d\'ebut 
du processus pr\'esente des conditions sp\'eciales, car on a impos\'e 
artificiellement que la queue soit de longueur nulle \`a l'instant z\'ero.  
La longueur de la queue (c'est-\`a-dire la variable al\'eatoire $Q_n$)  
redevient nulle ind\'efiniment \`a des instants al\'eatoires; mais les  
distributions possibles de ces instants de retour \`a z\'ero sont toutes  
\'equiprobables, il n'y a pas de distribution privil\'egi\'ee. En imposant  
que l'instant $n=0$ soit obligatoirement l'un de ces instants on 
d\'etruit l'\'equivalence des instants. C'est pourquoi pendant une 
p\'eriode de l'ordre de la centaine de minutes la loi de $Q_n$ \'evolue 
selon un r\'egime transitoire avant de retrouver la loi constante  
comme limite.    
\medskip 
On peut exprimer commod\'ement la r\'ecurrence des lois des $Q_n$ et  
leur limite en consid\'erant les fonctions g\'en\'eratrices. En effet, les  
variables $Q_n$ prennent des valeurs enti\`eres non n\'egatives, donc 
l'usage des fonctions g\'en\'eratrices est tout \`a fait recommand\'e. 
\medskip 
Appelons $G_n(z)$ la fonction g\'en\'eratrice inconnue de $Q_n$ et  
$F(z)$ celle, connue, des  $X_n$. La loi commune des $X_n$ \'etant la loi 
de Poisson, $F(z)$ est la fonction $\exp\big(\lambda [z-1]\big)$. Les 
relations $(IX.14.)$ et $(IX.15.)$ se traduisent en termes de fonctions 
g\'en\'eratrices comme  suit: 
$$z\, G_n(z) - F(z)\, G_{n-1}(z) \; = \; (z-1)\, F(0)\, G_{n-1}(0) 
\eqno (IX.16.)$$ 
On peut montrer math\'ematiquement que,  pour $0 <\lambda < 1$, $G_n(z)$
doit tendre vers une limite,  mais on se contentera de l'admettre. 
La d\'emonstration est assez longue et peu instructive,  j'estime donc
qu'elle ne vaut pas la peine de rallonger excessivement cette section. 
De toutes fa\c{c}ons le programme qui calcule la r\'ecurrence $IX.14$ --
$IX.15$ montre une stabilisation de la loi de $Q_n$ (voir figures
$36$ \`a $39$). 
\medskip
Sachant que $G_n(z)$ tend bien vers une limite finie $G(z)$,  on peut
faire tendre $n$ vers l'infini dans la relation ci-dessus,  ce qui donne 
$$\big[\, z - F(z)\, \big]\, G(z) \; = \; (z-1)\, F(0)\, G(0)
\eqno (IX.17\, a.)$$ 
d'o\`u on d\'eduit ce que doit \^etre la fonction g\'en\'eratrice de  
la  loi limite si cette limite existe: 
$$G(z) \; = \; F(0)\, G(0)\, {z-1 \over z-F(z)} \eqno (IX.17\, b.)$$ 
La fonction ${(z-1) / \big( z-F(z)\big)}$ se prolonge analytiquement en  
$z=1$, o\`u elle vaut $1/\big( 1-F'(0) \big)$; en prenant $z=1$ dans  
l'expression ci-dessus, on obtient $G(1) = F(0)\, G(0)/\big( 1-F'(0) \big)$, 
et bien s\^ur $G(1) = 1$ puisque $G(1)$ est la limite des $G_n(1)$ qui  
sont tous \'egaux \`a $1$. Donc $F(0)\, G(0) = 1 - F'(0)$. On remarquera en 
passant que la relation $IX.16$ est valable quelle que soit la loi de la 
variable  $X_n$, c'est-\`a-dire pour des fonctions g\'en\'eratrices $F(z)$  
arbitraires; de sorte que la loi limite aura pour fonction g\'en\'eratrice  
$$G(z) \; = \; \big[1 - F'(0) \big]\, {z-1 \over z-F(z)} \eqno (IX.17\, c.)$$ 
r\'esultat obtenu presque sans calculs. Dans le cas qui nous concerne la 
loi des $X_n$ est la loi de Poisson, pour laquelle $F(z) = \exp \big( 
\lambda [z-1]\big)$; dans ce cas $1 - F'(0) = 1 - \lambda$. Mais 
les expressions $IX.17\, b$ ou $IX.17\, c$ sont valables pour n'importe 
quelle loi initiale.  
\medskip 
Nous voyons ici, avec la formation de queues, un nouveau ph\'enom\`ene 
conduisant \`a une loi de probabilit\'e asymptotique discr\`ete: pour une  
loi donn\'ee des arriv\'ees (qui est g\'en\'eralement, dans les cas ayant 
un sens pratique, une loi de Poisson), la loi de la longueur des queues 
apr\`es stabilisation est d\'etermin\'ee par $IX.17\, a,b,c$. Notez bien  
que les $Q_n$ ne sont pas des sommes de variables ind\'ependantes.  
\medskip 
Si on veut calculer la loi limite, il suffit en principe de calculer les 
coefficients de Taylor de cette fonction analytique $G(z)$, ce qui est  
ais\'e pour les premiers, mais de plus en plus difficile pour les suivants  
(il n'y a pas de formule simple). Toutefois la relation $(IX.17\, c.)$  
contient toute l'information.  
\medskip 
Le probl\`eme des queues \`a un guichet a \'et\'e abord\'e par \'Emile 
Borel\ftn{1}{\'Emile Borel {\it Sur l'emploi du th\'eor\`eme de Bernoulli, 
pour le calcul d'une infinit\'e de coefficients. Application au probl\`eme 
d'attente \`a un guichet.} Comptes-rendus de l'Acad\'emie des Sciences, 
mars {\oldstyle 1942}.}. Son approche est diff\'erente: il ne   
prend pas en consid\'eration la variable al\'eatoire ``longueur de la 
queue'', dont  la loi est certes exprimable analytiquement (formule 
$IX.17\, c.$ ci-dessus), mais ne permet pas une expression simple pour 
chaque probabilit\'e. Il faut dire qu'\`a l'\'epoque on ne disposait pas de  
{\it personal computers} sur lesquels on pouvait \'ecrire en vitesse un 
petit programme qui calcule tout.  
\medskip 
\midinsert 
\vskip3pt
\centerline{\epsfbox{../imgEPS/ch09eps/fig35.eps}} 
\vskip-1mm
\centerline{\eightpoint figure 35} 
\vskip6pt 
\centerline{\vbox{\hsize=12cm\eightpoint  
Le graphique ci-dessus illustre la loi de Borel pour la longueur des 
s\'eries d'usagers; ici pour la valeur $\lambda = 0.76$ (celle de notre 
exemple). Le premier trait vertical correspond \`a la probabilit\'e 
$e^{-\lambda } \simeq 0.46$ d'avoir une s\'erie de longueur 1, 
c'est-\`a-dire un usager qui se pr\'esente au guichet et qui n'est pas 
imm\'ediatement suivi d'un autre; le second trait correspond \`a la  
probabilit\'e $\simeq 0.17$ d'avoir une s\'erie de longueur 2: un usager  
se pr\'esente, puis un second arrive avant que le premier n'ait termin\'e, 
apr\`es quoi le guichet reste inoccup\'e au  moins pendant la  minute 
suivante. Si le flux $\lambda$ reste constant toute la journ\'ee, on peut 
conclure d'apr\`es la loi des grands nombres qu'au cours de la journ\'ee 
environ $46\%$ des s\'eries seront r\'eduites \`a un seul usager, $17\%$ 
\`a deux usagers cons\'ecutifs, etc. Bien entendu les queues longues ne 
peuvent se former que dans des s\'eries au moins aussi longues.}}   
\endinsert 
Borel a donc trouv\'e le biais suivant: 
au lieu de consid\'erer les queues, il consid\`ere ce qu'il appelle une 
s\'erie: c'est une suite d'usagers qui se succ\`edent au guichet 
cons\'ecutivement sans que le guichet se lib\`ere; d\`es que le guichet 
redevient libre (m\^eme une seule minute) la s\'erie se termine, mais  
une nouvelle s\'erie se produira avec le premier usager qui se pr\'esente 
apr\`es le temps mort. Une s\'erie peut se r\'eduire \`a un seul usager. 
Les queues longues ne peuvent se former qu'\`a l'int\'erieur d'une 
s\'erie longue, mais il n'y a aucune relation math\'ematique pr\'ecise  
entre la longueur d'une s\'erie et le maximum de la longueur de la queue 
pendant la s\'erie: une s\'erie peut \^etre tr\`es longue sans que jamais 
plus de deux usagers ne fassent la queue (si les usagers se pr\'esentent 
de fa\c{c}on assez r\'eguli\`erement espac\'ee),  et inversement une 
queue longue peut se former dans une s\'erie relativement courte (si 
beaucoup d'usagers arrivent presque en m\^eme temps). Il y a toutefois 
une corr\'elation assez forte entre la longueur des queues et la longueur 
des s\'eries dans lesquelles elles se forment. On peut donc utiliser la 
probabilit\'e $\alpha_n$ d'avoir des s\'eries de longueur sup\'erieure \`a 
un seuil donn\'e $n$ comme un crit\`ere quantitatif pour d\'ecider si le 
nombre de guichets est suffisant, comme nous l'avons propos\'e pour la  
probabilit\'e pour que se forme une queue de longueur sup\'erieure \`a $n$. 
\medskip  
Borel a  obtenu la formule explicite suivante pour la probabilit\'e $p_n$ 
pour qu'une  s\'erie ``prise au hasard'' soit form\'ee de  $n$ usagers: 
$$p_n = \e^{-n\lambda } \lambda^{n-1} {n^{n-2}\over (n-1)!}$$ 
o\`u $\lambda$ est le nombre moyen d'usagers par minute 
($\lambda = 0.76$ dans notre exemple). Cette formule n'est valable que 
si $\lambda <1$. Elle permet de calculer les dur\'ees d'attente moyennes 
pour un usager, mais ne permet pas de calculer la loi de la longueur des 
queues $IX.17\, c.$ que nous avons obtenue par une autre m\'ethode. 
\medskip 
\midinsert 
\null\vbox to \blocksize{\eightpoint  
\centerline{\epsfbox{../imgEPS/ch09eps/fig36.eps}} 
\vskip8mm plus2mm minus1mm
\centerline{ figure 36} 
\vskip5mm plus2mm minus1mm 
\centerline{\vbox{\hsize=12.3cm Les neuf graphiques ci-dessus  
repr\'esentent les lois de probabilit\'e de la longueur des queues pour 
$\lambda = 0.76$ apr\`es 1, 2, 3, 5, 10, 20, 30, 50, et 100 minutes:  
comme d'habitude, la hauteur des barres est proportionnelle \`a la 
probabilit\'e de l'abcisse correspondante. L'\'echelle des abcisses est 
visualis\'ee par une  graduation (chaque trait de graduation correspond 
\`a une valeur -- enti\`ere -- de la variable al\'eatoire ``longueur de la 
queue''). L'\'echelle des ordonn\'ees est visualis\'ee \'egalement: comme 
il s'agit de probabilit\'es, celle-ci va de $0.0$ \`a $1.0$. On peut observer 
l'\'evolution vers une loi limite: pour $n$ petit (donc peu apr\`es le 
d\'ebut), la barre d'abscisse z\'ero est encore haute, puis elle diminue  
de plus en plus en  plus au profit des autres; le graphique n'\'evolue 
pratiquement plus au-del\`a de $n=100$. } } \vfill } \vfill 
\endinsert 
 
\midinsert 
\vbox to \blocksize{ 
\vskip10mm plus2mm minus1mm
\centerline{\epsfbox{../imgEPS/ch09eps/fig37.eps}} 
\vskip10mm plus2mm minus1mm 
\centerline{\eightpoint figure 37} 
\vskip7mm plus2mm minus1mm 
\centerline{\vbox{\hsize = 12cm \eightpoint  
Afin de visualiser l'\'evolution de la loi de $Q_n$ vers la loi limite, on  
peut aussi repr\'esenter la probabilit\'e que la queue ait une longueur 
sup\'erieure ou \'egale \`a un seuil fix\'e, en fonction du temps  
\'ecoul\'e. La courbe ci-dessus correspond \`a un seuil de 12 (avec 
toujours $\lambda = 0.76$): l'abscisse repr\'esente le temps \'ecoul\'e 
en minutes et l'ordonn\'ee la probabilit\'e que la queue comporte au 
moins douze v\'ehicules. Bien s\^ur la courbe n'est qu'une interpolation 
des valeurs successives de ces probabilit\'es, puisque le temps est un 
nombre entier de minutes.  
\smallskip 
On voit qu'au d\'ebut cette probabilit\'e est pratiquement nulle: elle est 
donn\'ee par la loi de Poisson: pour que la queue soit d'au moins douze 
v\'ehicules d\`es le premier instant, il faut que douze v\'ehicules ou plus  
se pr\'esentent pendant la premi\`ere minute, \'ev\'enement dont la 
probabilit\'e est   
$$\sum_{k \geq 12} {\lambda^k \over k!}\e^{-\lambda}$$ 
Mais apr\`es cette probabilit\'e augmente et tend vers une limite,  
\'egale  \`a $0.0015$ environ. Ce qui est minuscule et ne justifie pas 
qu'on construise un second guichet. Toutefois pour $\lambda = 0.9$, la 
probabilit\'e limite d'avoir une queue d'au moins douze v\'ehicules est  
de $7\%$ environ. 
\vskip6pt plus3pt minus1pt
Nous n'avons pas {\it d\'emontr\'e} que les probabilit\'es ${\cal P}\,
(Q_n = k)$ tendaient vers une limite,  mais le calcul num\'erique montre
bien une stabilisation des valeurs.} } 
\vfill }
\endinsert 
 
\midinsert 
\null\vbox to \blocksize{\eightpoint  
\vskip3pt
\centerline{\epsfbox{../imgEPS/ch09eps/fig38.eps}} 
\vskip10mm plus2mm minus1mm 
\centerline{ figure 38 } 
\vskip7mm plus2mm minus1mm 
\centerline{\vbox{\hsize=12cm  
Ici on repr\'esente la m\^eme chose que sur la figure 36, mais  
pour  $\lambda = 0.9$ au lieu de $0.76$. En comparant les deux figures, 
on peut observer que la barre d'abscisse z\'ero -- qui repr\'esente la 
probabilit\'e de n'avoir aucune queue -- diminue ici plus vite, et les  
autres barres augmentent plus vite, ce qui est bien conforme \`a 
l'\'evidence.  Mais on peut calculer quantitativement: par exemple la 
probabilit\'e limite pour que la queue d\'epasse douze v\'ehicules 
(obligeant ainsi les usagers \`a attendre leur tour plus de douze 
minutes) est ici de $7\%$, alors qu'elle \'etait insignifiante pour 
$\lambda = 0.76$ .} } \vfill } \vfill  
\endinsert 
 
\midinsert 
\null\vbox to \blocksize{\eightpoint  
\vskip3pt
\centerline{\epsfbox{../imgEPS/ch09eps/fig39.eps}} 
\vskip10mm plus2mm minus1mm 
\centerline{ figure 39 } 
\vskip7mm plus2mm minus1mm 
\centerline{\vbox{\hsize=12cm  
M\^eme chose que dans les figures 36 et 38, mais avec cette fois 
$\lambda = 0.99$.} } \vfill } 
\vfill 
\endinsert 
 
Revenons \`a la loi des queues et \`a la fonction g\'en\'eratrice $G(z)$ 
de la formule $(IX.17\, c)$. 
\medskip 
Pour calculer num\'eriquement les coefficients de Taylor de $G(z)$  
(c'est-\`a-dire  les probabilit\'es limite de la loi de la longueur des 
queues) d'ordre \'elev\'e l'expression $(IX.17\, c.)$ n'est gu\`ere  
commode. Pour le calcul num\'erique effectif de la loi limite il est 
pr\'ef\'erable d'ex\'ecuter le programme it\'eratif bas\'e sur la 
r\'ecurrence $(IX. 14, 15)$. Par contre  l'expression analytique $(IX.17\, 
c)$ de la fonction g\'en\'eratrice est tr\`es pratique pour calculer  
la moyenne et la variance de la loi limite qui sont donn\'ees par  
$G'(1)$ et $G''(1)$, ainsi que les premiers coefficients de Taylor qui  
sont donn\'ees par  $G'(0)$ et $G''(0)$. Notons $Q$ sans indice pour  
la variable al\'eatoire limite. Rappelons que ${\bf E}(Q) = G'(1)$ et  
${\bf Var}(Q) = G''(1) + {\bf E}(Q) - {\bf E}(Q)^2$ (cf $VI. 6$ et $VI. 7$).  
\medskip 
D\'erivant formellement $(IX.17\, c.)$ on obtient 
$$\eqalign{ 
G'(z) &= \big[ 1 - F'(1) \big]\; {1 - F(z) + (z-1)F'(z) \over \sdown{11} 
\big[ z - F(z) \big]^2 } \cr 
\noalign{\medskip} 
G''(z) &= \big[ 1 - F'(1) \big] \times \hskip62mm  
\hbox{\raise6pt\hbox{$(IX.18.)$}}\cr 
\times\; &\hbox to 
101mm{\lower19.5pt\hbox{\vbox{\eightpoint\hsize=100mm 
$$ {\big[ 1  - F'(1) \big] (z-1) F''(z) -2 \big[ 1 - F'(z) \big] \big[ 1 - F(z)  
+ 
(z-1) F'(z) \big] \strup{6} 
\over \sdown{11} \big[ z - F(z) \big]^3 }$$}}\hfill } \cr }$$ 
Pour $z=0$ ces expressions prennent les valeurs particuli\`eres 
suivantes: 
$$\eqalignno{ 
G(0) = {\cal P}\, (Q=0) &= {1 - F'(0) \over F'(0)} = (1 - \lambda )\, 
\e^{\lambda } \cr 
\noalign{\medskip} 
G'(0)={\cal P}\, (Q=1) &=\big[1-F'(0)\big]{ 1-F(0)-F'(0)\over F'(0)^2}\cr  
&= (1 - \lambda )\,\e^\lambda\, ( \e^\lambda - 1 - \lambda) \cr 
\noalign{\bigskip} 
G''(0) = {\cal P}\, (Q=2) &= (1-\lambda )\, \big[ 2\, (1-\lambda )\, 
\e^{2\lambda}\, (\e^{\lambda}-1-\lambda )\big] \cr }$$ 
On voit que $G''(0)$ est d\'ej\`a relativement compliqu\'ee et cela  
permet d'imaginer \`a quoi pourrait ressembler par exemple ${\cal P}\, 
(Q=12)$ si au lieu de le calculer num\'eriquement par it\'erations, on en 
voulait  une expression analytique formelle en fonction de $\lambda$. 
\medskip 
Pour les valeurs en $z=1$ on obtient des rapports 
ind\'etermin\'es en  $0/0$; on peut lever l'ind\'etermination en  
recourant \`a la r\`egle de l'Hospital, mais il est plus commode de 
revenir \`a $IX.16.$ qu'on d\'erive trois fois; ainsi: 
$$\displaylines{ 
\big[ z - F(z) \big] G(z) = (z-1) F(0) G(0) \cr 
\big[ 1 - F'(z) \big] G(z) + \big[ z - F(z) \big] G'(z) = F(0) G(0) \cr 
- F''(z) G(z) + 2 \big[ 1 - F'(z) \big] G'(z) + 
\big[ z - F(z) \big] G''(z) = 0 \cr 
- F'''(z) G(z) - 3 F''(z) G'(z) +  3 \big[ 1 - F'(z) \big] G''(z) + 
\big[ z - F(z) \big] G'''(z) = 0 \cr }$$ 
rempla\c{c}ant apr\`es coup $z$ par $1$, on obtient $0 = 0$ 
pour la premi\`ere ligne et $1-\lambda = F(0) G(0)$ pour la seconde 
(ce qui n'apporte rien de nouveau), mais $-F''(1) + 2\, (1-\lambda )\,  
G'(1) = 0$ pour la troisi\`eme et $-F'''(1) - 3 F''(1) G'(1) + 3\, (1 - 
\lambda )\,  G''(1) = 0$ pour la quatri\`eme, ce qui donne ais\'ement ce 
qu'on cherche:  
$$\eqalignno{ 
G'(1) &= {\bf E} (Q) = {F''(1) \over 2\, (1 - \lambda )} =  {\lambda^2  
\over   2\, (1 -  \lambda )}\cr 
G''(1) &= {F'''(1) + 3 F''(1) G'(1) \over 3\, (1 - \lambda )} =  
{\lambda^3 (\lambda + 2) \over 6\, (1 - \lambda )^2} \cr }$$ 
d'o\`u la variance 
$${\bf Var} (Q) = G''(1) + {\bf E} (Q) - {\bf E} (Q)^2 =  
{6\lambda^2 - 2\lambda^3 - \lambda^4 \over 12\, (1 - \lambda )^2}$$ 
\medskip 
Le moment est maintenant venu de r\'esumer et de critiquer notre  
approche de ce probl\`eme des queues aux guichets. Pour le r\'esoudre  
nous avons  copieusement recouru \`a des approximations, qui nous 
\'etaient  offertes sous la forme de lois asymptotiques, en l'occurrence 
la loi de Poisson et un passage \`a la limite. 
\medskip 
La loi de Poisson est d'abord apparue comme celle (\'evidemment 
appro\-ch\'ee) du nombre de v\'ehicules qui arrivent pendant une minute, 
\`a condition de supposer,  ce qui est essentiel,  que la densit\'e
du trafic,  d\'efinie par le flux moyen instantan\'e $\lambda = 0.76$
v\'ehicules par minute, est stable. En admettant que le flux
instantan\'e reste stable pendant une heure,  on peut le mesurer
en comptant le nombre de v\'ehicules qui sont pass\'es pendant une
heure,  puis en divisant ce nombre par soixante;  si le flux instantan\'e
\'etait rest\'e stable pendant deux  heures, on aurait compt\'e le nombre
de v\'ehicules pass\'es pendant ces  deux heures, et on aurait divis\'e
ce nombre par cent-vingt.  S'il n'\'etait  rest\'e stable que pendant
une demie-heure,  on aurait divis\'e le nombre de v\'ehicules pass\'es
pendant la demie-heure par trente.  Plus haut, nous avons compar\'e
cette id\'ee de flux instantan\'e \`a celui de vitesse instantan\'ee
en cin\'ematique.  On rencontre cependant ici une diff\'erence
essentielle avec la cin\'ematique.  
\medskip
Lorsqu'on veut d\'efinir la vitesse instantan\'ee d'un mobile, 
on commence,  comme nous avons fait ici,  par d\'efinir la vitesse
moyenne au cours d'un intervalle de temps;  disons pour fixer les
id\'ees qu'on la mesure en m\`etres par seconde.  Sur une heure, 
c'est la distance parcourue en m\`etres, divis\'ee par $3\, 600$. 
Sur une minute,  c'est la distance parcourue divis\'ee par $60$. 
On admet que lorsque l'intervalle de temps devient petit (par exemple
une seconde),  cette vitesse moyenne tend vers une limite,  au point
que si on mesurait la distance parcourue pendant un milliardi\`eme
de seconde,  et qu'on la divisait par $10^{-9}$,  on trouverait la
m\^eme limite (ou une valeur plus exacte de cette limite, d'autant
plus exacte que l'intervalle serait plus court).  Avec le flux il n'en
va plus de m\^eme, car le flux moyen calcul\'e sur un intervalle de
temps long (deux heures, une heure, une demie-heure)  a des chances de
rester \`a peu pr\`es le m\^eme ($\simeq 0.76$) \`a cause de la loi
des grands nombres;  par contre,  si on compte le nombre de v\'ehicules
qui passent pendant une minute,  on n'obtiendra pas $0.76$,  mais des
valeurs enti\`eres al\'eatoires $0,\, 1,\, 2,\, 3,\, \ldots$ distribu\'ees
justement  selon la loi de Poisson.  Loin de se rapprocher d'une limite, 
on obtiendra des valeurs de plus en plus divergentes lorsque la dur\'ee
de l'intervalle sera rendue plus courte:  par exemple,  le nombre de
v\'ehicules arrivant pendant une seconde sera presque toujours $0$, 
rarement $1$,  et pratiquement jamais $2$ (il sera donn\'e par la loi
de Poisson de param\`etre $\lambda/60 = 0.76 / 60 \simeq 0.01267$, 
qui donne $0.987$ pour la probabilit\'e d'avoir z\'ero v\'ehicule, 
mais seulement $0.0125$ pour la probabilit\'e d'avoir un seul v\'ehicule, 
et $0.000079$ pour la probabilit\'e d'avoir deux v\'ehicules).      
\medskip 
Le flux instantan\'e ne peut pas \^etre mesur\'e s'il varie sans  
cesse, car la loi des grands nombres est indispensable pour sa 
d\'etermination; il faut donc s'assurer de sa stabilit\'e par des mesures 
pr\'ealables. Dans le cas du p\'eage d'autoroute, il se peut que le flux 
instantan\'e varie beaucoup au cours de la journ\'ee, mais se r\'ep\`ete 
d'une journ\'ee \`a l'autre (en excluant \'evidemment les dimanches et 
jours f\'eri\'es, les d\'eparts en vacances, etc.). Par exemple le flux 
peut \^etre constamment variable, \^etre diff\'erent \`a $8$ heures,  
$8$  heures trente, $9$ heures, $9$  heures trente, $\ldots$, mais \^etre 
identique tous les jours \`a $8$  heures, ou tous les jours \`a $8$  
heures  trente. On pourra donc mesurer le flux moyen en additionnant 
sur vingt jours le nombre de v\'ehicules qui arrivent entre $8\; h.$ et 
$8\; h.\; 05$, puis entre $8\; h.\; 05$ et $8\; h.\; 10$, etc. Tous les 
proc\'ed\'es statistiques utilisables auront en commun  de reposer sur 
une  hypoth\`ese d'invariance: soit stabilit\'e sur une  dur\'ee 
suffisamment longue, soit similitude des diff\'erentes  journ\'ees entre 
elles. Comme cela avait \'et\'e longuement expliqu\'e au chapitre {\bf I}, 
la  notion de probabilit\'e exige une hypoth\`ese d'invariance; nous 
voyons  sur l'exemple pr\'esent qu'il en va de m\^eme lorsqu'on veut {\it 
mesurer}.  
\medskip 
L'invariance permet de calculer des probabilit\'es a priori, que nous  
avons oppos\'ees aux probabilit\'es empiriques; mais lorsqu'on veut 
mesurer de telles probabilit\'es empiriques il faut aussi s'assurer qu'il  
y a une  invariance quelque part. Ici, c'est l'invariance au 
cours du temps ou {\it reproductibilit\'e}. 
\medskip 
Au chapitre {\bf I} nous avions signal\'e une analogie entre 
l'\'equiprobabilit\'e qui r\'esulte d'une invariance, et le temps physique 
qui r\'esulte des invariances galil\'eennes.  Poursuivant cette 
comparaison,  nous pouvons dire que la mesure du temps (c'est-\`a-dire 
la r\'ealisation d'appareils mesurant le temps) exige un principe 
physique pr\'ealable d'invariance:  la mesure du temps pr\'esuppose une 
conviction (que seule une perception th\'eorique peut donner) quant \`a 
l'homog\'en\'eit\'e de l'\'ecoulement du temps.  De  m\^eme on ne peut 
pr\'etendre mesurer de probabilit\'es empiriques sans postuler la loi 
des grands nombres,  et pour que ce postulat soit valide,  il faut une 
conviction que les conditions se reproduisent,  que par exemple le trafic 
se reproduit \`a l'identique d'un jour \`a l'autre.  
\medskip 
Il va de soi que la pr\'ecision de mesures ainsi bas\'ees sur une  
hypoth\`ese de reproductibilt\'e tr\`es approximative et sur la loi des  
grands nombres ne peut en aucun cas atteindre la seconde d\'ecimale; 
ainsi la valeur suppos\'ee de $0.76$ pour le flux instantan\'e suppos\'e 
constant pourrait aussi bien avoir \'et\'e de $0.7$ ou $0.8$ v\'ehicules  
par minute.  Il est tr\`es rare que des moyennes statistiques ou des 
probabilit\'es empiriques soient connues avec une pr\'ecision plus 
grande.  En effet,  la loi des grands nombres,  avec les fluctuations 
gaussiennes \'etudi\'ees au chapitre {\bf VII},  nous a montr\'e que  
l'\'ecart-type des fluctuations \'etait de l'ordre de la racine carr\'ee  
du grand nombre.  Cela signifie que si on d\'etermine la moyenne 
$\lambda$ en comptant le nombre de v\'ehicules pass\'es pendant une 
heure (qui est donc $0.76 \times 60 \simeq105$),  l'incertitude relative 
sera de l'ordre de $1 / \sqrt{105}$,  c'est-\`a-dire du dixi\`eme. 
\medskip  
Il faut donc partir du principe que dans des probl\`emes comme celui du 
p\'eage d'autoroute,  ainsi que dans tous les probl\`emes de statistiques 
humaines ou industrielles,  les probabilit\'es sont d\'etermin\'ees au 
dixi\`eme,  ou \`a la rigueur dans des cas exceptionnels,  au centi\`eme 
pr\`es.  Il n'y a gu\`ere que la physique statistique o\`u on peut aller au 
del\`a,  le grand nombre \'etant celui d'Avogadro ($\sim 10^{24}$);  les 
grandeurs sont alors en principe d\'efinies (mais rarement {\it 
effectivement connues}) \`a $10^{-12}$ pr\`es.   
\medskip 
On peut conclure que dans le calcul des probabilit\'es une approximation 
m\^eme grossi\`ere est {\it toujours} meilleure qu'une expression 
math\'ematique compliqu\'ee.  Il ne faut donc pas h\'esiter \`a utiliser 
l'approximation gaus\-sienne,  m\^eme pour la somme de dix ou vingt 
variables ind\'ependantes,  ou la formule de Stirling pour des factorielles: 
pour $4! = 24$ la formule de Stirling donne $23.506$,  pour $3! = 6$ elle
donne $5.836$,  pour $2! = 2$ elle donne $1.919$,  et m\^eme pour $1! = 1$
elle donne $0.922$:  \`a $10\%$ pr\`es,  la formule de Stirling est correcte
pour {\it n'importe quelle factorielle},  $0!$ except\'e. 
\medskip
On peut \'egalement utiliser l'approximation plus pr\'ecise
$$n! \simeq n^n\, \e^{-n}\, \sqrt{2\pi\, (n + 1/6)}$$
qui donne $1.075\, 827$ pour $1!$,  $2.078\, 921$ pour $2!$, 
et $6.160\, 443$ pour $3!$;  m\^eme $0!$ est alors approch\'e par
$1.023\, 327$.  Une pr\'ecision relative de $2\%$ d\`es $n=0$, 
et qui s'am\'eliore encore quand $n$ augmente: \hskip7pt $1\%$ pour
$n=8$,\hskip10pt $0.5\%$ pour $n=16$,\hskip10pt $0.2\%$ pour $n=40$, \ \dots

\vskip14pt plus12pt minus6pt
 
{\bf IX.\aub 5. Autres lois asymptotiques.}
\medskip 
En discutant l'exemple de la densit\'e de Cauchy, nous avons remarqu\'e 
que celle-ci apparaissait plut\^ot comme un d\'eguisement de la loi 
normale, par l'effet d'une d\'eformation non lin\'eaire. Nous avons  
m\^eme  pu d\'emontrer math\'ematiquement que n'importe quelle 
densit\'e fix\'ee \`a l'avance pouvait \^etre obtenue comme une 
d\'eformation non lin\'eaire de  la densit\'e gaussienne. Il en va de 
m\^eme pour la loi de Poisson: on pourrait d\'emontrer que 
n'importe quelle loi discr\`ete  peut \^etre obtenue artificiellement 
comme une d\'eformation non  lin\'eaire de la loi de Poisson. Mais il ne 
s'agirait que de constructions artificielles qui ne r\'epondraient pas \`a 
la v\'eritable question:  ``quelles lois limites (asymptotiques) 
rencontre-t-on dans des probl\`emes ayant une pertinence r\'eelle ?'' 
(cette question a \'et\'e discut\'ee \`a la fin de la section {\bf 2}). 
\medskip 
Bien que ce chapitre soit sp\'ecialement consacr\'e aux lois  
asymptotiques,  la plupart de ces lois ont \'et\'e ou seront rencontr\'ees 
dans d'autres chapitres \`a propos de probl\`emes particuliers.  C'est 
pourquoi nous conclurons par un petit catalogue des lois asymptotiques 
restantes,  avec r\'ef\'erence au chapitre o\`u elles sont trait\'ees. 
 
\vskip14pt plus30pt minus3pt
 
{\bf La loi du dernier retour} 
\medskip 
Elle a \'et\'e rencontr\'ee au chapitre {\bf III}. Nous avions calcul\'e  
alors que pour une marche al\'eatoire de $2n$ pas, la probabilit\'e $R_k$ 
pour que  la marche soit pass\'ee {\it pour la derni\`ere fois} par z\'ero 
\`a l'instant  $2k$  est  
$$R_k = 2^{-2n} \cdot {2k \choose k} \cdot {2n -2k \choose n-k} 
\eqno (IX.18.)$$ 
On en d\'eduisait que la probabilit\'e pour que le dernier retour avant 
l'instant $2n$ se soit produit {\it avant} l'instant $2\ell$ est 
$$S_\ell = \sum_{k=1}^{k=\ell } R_k$$ 
Lorsque $n$ est grand on peut approcher $R_k$ par $1/[\pi\sqrt{k(n-k)}]$ 
en appro\-chant les factorielles des coefficients bin\^omiaux par 
la formule de Stirling, puis remarquer que la somme des $R_k$ est la 
somme de Riemann de l'int\'egrale 
$$\int_0^{\ell / n} {1\over\sdown{14}\pi\sqrt{\, t\, (1-t)}}\; dt$$ 
d'o\`u on d\'eduit que 
$$S_\ell \simeq {\up{\eightpoint 2}\over\down{\pi}} \arcsin\Bigg( 
\sqrt{{\ell\over n}}\;\Bigg)$$ 
Ces r\'esultats font donc appara{\^\i}tre une densit\'e asymptotique: 
lorsque $n$ est grand, la loi de ${1\over\textdown{n}}\,R_k$ a pour 
densit\'e la fonction 
$$f(t) = \cases{ {\hbox{$1$}\over\sdown{10.5} 
\hbox{$\pi$}\sqrt{t\, (1-t)} } &si $0<t<1$;\cr 
\noalign{\vskip9pt} 
\hskip24pt   0    & si $t \leq 0$ ou $t \geq 1$.\cr } \eqno (IX.19.)$$  
La variable $R_k$ ne pouvant prendre des valeurs en dehors de $0 
< k < n$, il est logique que la densit\'e $f$ soit nulle (ou non d\'efinie)  
en dehors de $0<t<1$. 
\medskip 
 
\midinsert 
\vbox to \blocksize{ 
\vskip-11mm plus3mm minus3mm
\centerline{\epsfbox{../imgEPS/ch09eps/fig40a.eps}} 
\vskip8pt 
\centerline{\eightpoint figure 40a} 
\vskip2mm 
\centerline{\vbox{\hsize=12cm\eightpoint 
Ce graphique compare les probabilit\'es de dernier retour $R_k$ pour 
$k=0 \ldots 7$ ($n=7$), repr\'esent\'ees par les hauteurs des huit 
rectangles, \`a la densit\'e $f(x) = 1/\pi\sqrt{x(1-x)}$.} } 
\vskip6mm 
\centerline{\epsfbox{../imgEPS/ch09eps/fig40b.eps}} 
\vskip8pt 
\centerline{\eightpoint figure 40b} 
\vskip2mm 
\centerline{\vbox{\hsize=12cm\eightpoint 
Ici $n=15$, les probabilit\'es $R_k$ sont repr\'esent\'ees par les  
hauteurs  des seize rectangles, la courbe de la densit\'e $f(x)$ est 
inchang\'ee.} }  
\vfill } 
\endinsert 
 
\midinsert 
\vbox to \blocksize{ 
\null 
\vskip-2mm plus2mm minus1mm
\centerline{\epsfbox{../imgEPS/ch09eps/fig40c.eps}} 
\vskip7pt 
\centerline{\eightpoint figure 40c} 
\vskip2mm 
\centerline{\eightpoint M\^eme chose pour $n=31$.} 
\vskip6mm 
\centerline{\epsfbox{../imgEPS/ch09eps/fig40d.eps}} 
\vskip7pt 
\centerline{\eightpoint figure 40d} 
\vskip2mm 
\centerline{\eightpoint $n=95$.} 
\vfill  } 
\endinsert 
 
\midinsert 
\vskip3pt
\centerline{\epsfbox{../imgEPS/ch09eps/fig40e.eps}} 
\vskip2mm 
\centerline{\eightpoint figure 40e} 
\vskip2mm 
\centerline{\vbox{\hsize=11.6cm\eightpoint \noindent $n=191$. Il est devenu  
impossible de distinguer les sommets des rectangles de la courbe. } } 
\vskip3mm 
\endinsert 

L'approximation qui conduisait \`a la densit\'e $(IX.19)$ 
reposait sur la for-mule de Stirling  $n! \simeq n^n\e^{-n}\sqrt{2\pi n}$.  
Celle-ci n'est en principe correcte que pour $n$ grand; elle a \'et\'e 
appliqu\'ee \`a l'expression $(IX.18)$ de la probabilit\'e $R_k$, qui  
contient les factorielles $(2k)!\ ,\ k!\ ,\ (2n-2k)!\ ,$ et $(n-k)!$ donc il 
faut que $k$ et $n-k$ soient tous deux assez grands. Voyons ce que cela 
donne en pratique: pour $n=1000$ et $k=1$ on obtient $R_1=0.0089$ 
(valeur exacte) et $R_1\simeq 0.0101$ (valeur approch\'ee),  soit une  
diff\'erence relative de $12\%$;  pour $k=2$,  $R_2=0.0067$ (valeur 
exacte) et $R_2\simeq 0.0071$ (valeur approch\'ee),  soit une  
diff\'erence relative  de $6.3\%$.  Pour $k=10$,  $R_{10}=0.00316$ 
(valeur exacte) et $R_{10}\simeq 0.00320$ (valeur approch\'ee),  soit  
une diff\'erence relative de $1.2\%$.  La seule erreur vraiment grosse 
concerne les extr\'emit\'es,  $k=0$ ou $k=n$,  car alors la valeur 
approch\'ee est infinie et la valeur exacte $R_0=0.0178$.  On peut 
constater sur les valeurs num\'eriques donn\'ees ci-dessus que la valeur 
approch\'ee est toujours plus grande que la valeur exacte calcul\'ee par 
les factorielles;  pour de petites valeurs de $k$,  la diff\'erence ne  
diminue pas lorsque $n$ tend vers l'infini,  mais ces valeurs  
repr\'esentent par contre une proportion de plus en plus petite de 
l'ensemble de toutes les valeurs (qui vont de z\'ero \`a $n$). 
\medskip  
On peut voir sur les figures 40 (a, b, c, d, e) comment 
la loi discr\`ete se rapproche de la densit\'e asymptotique.  Ces figures 
se passent de commentaire. 
 
\vskip12pt plus10pt minus5pt
 
{\bf La loi de Bernoulli.} 
\medskip 
En statistique,  un proc\'ed\'e classique est le {\it sondage par  
\'echantillon}.  On veut \'etudier un certain caract\`ere dans une 
population (par exemple combien d'\'electeurs vont voter pour le Parti  
du Progr\`es). Au lieu d'interroger tout le monde, on prend au hasard un 
\'echantillon de deux mille personnes. Le principe est que la probabilit\'e 
pour que la proportion  des partisans du Progr\`es dans l'\'echantillon 
diff\`ere sensiblement de  la proportion r\'eelle dans la population  
totale, est faible. Mais cela  n'est vrai que pour des \'echantillons  
assez gros.   
\medskip 
On peut utiliser les m\'ethodes qui ont \'et\'e  pr\'esent\'ees dans ce 
livre pour calculer cette probabilit\'e: on cherche une invariance, puis on 
calcule par d\'enombrement, ce qui conduira \`a une formule pleine de 
factorielles; ensuite, on cherche une approximation simple qui s'applique 
aux  grands entiers, ce qui donnera une loi asymptotique.  
\medskip 
L'invariance r\'esulte du choix {\it au hasard} de l'\'echantillon: 
``au hasard'' signifie que tous les \'echantillons sont \'equiprobables. 
Dans une population de $N$ personnes, il y a $N \choose n$ \'echantillons
de taille $n$. Le probl\`eme sera \'etudi\'e en d\'etail au chapitre {\bf X},  
section {\bf 2} ({\sl la th\'eorie des \'echantillons de Bernoulli}), 
de sorte que le lecteur y est renvoy\'e pour les calculs.  Si $p$ est
le nombre de partisans du Progr\`es dans la population totale, 
la probabilit\'e pour que l'\'echantillon de $n$ personnes tir\'e
au hasard en contienne $k$ est  
$$p_k = { \vrule height0pt depth6pt width0pt {n \choose k} \cdot  {N-n  
\choose p-k}  \over \vrule height12pt depth0pt width0pt  {N \choose p} }  
= {\vrule height0pt depth13pt width0pt {\displaystyle {n!  
\over k! \,  (n-k)!} \cdot {(N-n)! \over (p-k)! \, (N-p-n+k)!}} \over  \vrule 
height18pt depth0pt width0pt{\displaystyle {N! \over p! \, (N-p)!}} } $$  
Ceci est la loi ``math\'ematiquement exacte'', qui est valable quels que 
soient les nombres $N$, $p$, $n$, et $k$. On l'appelle {\it la loi 
hyperg\'eom\'etrique}. 
\medskip 
Cette loi gouverne les probl\`emes de tirage du type: {\sl Une urne  
contient $N$ boules, dont $p$ blanches et $N-p$ noires. On tire au   
hasard  et sans remise $n$ boules dans l'urne; quelle est la probabilit\'e 
pour que l'\'echantillon tir\'e contienne $k$ boules blanches et $n-k$  
boules noires ?} 
\medskip 
Si $n$ est beaucoup plus petit que $N$, on peut faire les approximations 
suivantes:   
$$\eqalignno{ 
(N-n)! \quad\quad &=\quad {N! \over N(N-1)(N-2)\cdots (N-n+1)}  
\quad\simeq\quad  {N! \over N^n} \cr   
\noalign{\medskip} 
(p-k)! \quad\quad &=\quad {p! \over p(p-1)(p-2) \cdots (p-k+1)} 
\quad\simeq\quad {p! \over p^k} \cr  
\noalign{\medskip} 
(N-p-n+k)! \quad &=\quad {(N-p)! \over (N-p)(N-p-1) \cdots (N-p-n+k+1)} 
\cr &\simeq\quad {(N-p)! \over (N-p)^{n-k}} \cr }$$  
(ces approximations supposent \'evidemment que $n \ll N$, $k \ll p$, et 
$n-k \ll N-p$). Par cons\'equent, si on pose $\alpha = p/N$ et $\beta = 
(N-p)/N = 1 - \alpha$, on aura: 
$$p_k = {n \choose k} \cdot { \vrule height0pt depth6pt width0pt 
{N-n \choose p-k} \over  
\vrule height0pt height12pt width0pt {N \choose p} } 
\simeq   {n \choose k} \, \alpha^{k} \, \beta^{n-k}$$ 
Cette approximation, valable pour $N$ grand, est donc une loi 
asymptotique, appel\'ee {\it loi de Bernoulli}. On peut aussi la rencontrer 
sous une forme non asymptotique dans le probl\`eme de l'urne cit\'e plus 
haut, mais pour des tirages {\it avec} remise: si on tire successivement 
et avec remise $n$ boules dans l'urne, \`a chaque tirage la probabilit\'e 
de tirer une boule blanche est $\alpha = p/N$; la fonction g\'en\'eratrice 
pour chaque tirage est donc $G(z) = \alpha + \beta\, z$; les tirages  
avec remise sont ind\'ependants, donc la fonction g\'en\'eratrice pour  
$n$ tirages est $(\alpha + \beta\, z)^n$; en d\'eveloppant selon la  
formule du bin\^ome, on retrouve bien la loi de Bernoulli.  
\medskip 
Il n'est \'evidemment pas surprenant que la loi sans remise donne 
asymptotiquement pour $N$ grand la m\^eme loi qu'avec remise: en effet, 
si $n \ll N$ , le tirage de $n$ boules sans remise modifie \`a peine les 
proportions de boules noires ou blanches, de sorte qu'on doit obtenir \`a 
peu pr\`es le m\^eme r\'esultat qu'avec remise.  
\medskip 
Nous ne mentionnons cette loi ici que pour insister sur le fait qu'elle  
peut aussi se rencontrer sous forme asymptotique. De toute fa\c{c}on   
elle sera \'etudi\'ee en d\'etail au chapitre {\bf X}. 
\medskip 
Cette loi d\'ecrit le sondage par \'echantillon lorsque $n \ll N$; mais 
signalons ---~ce sera \'etudi\'e au chapitre {\bf X}~--- que si $n$  
devient lui-m\^eme grand, ainsi que $k$ et $n-k$, la loi de Bernoulli 
devient gaussienne: comme la loi de Poisson, la loi de Bernoulli  
est une loi asymptotique de l'\'echelle interm\'ediaire; si $n$  est grand,  
mais que $k$ reste petit et $\beta\exp\{\alpha / \beta\} \simeq 1$ (par  
exemple $k \leq 10$, et $10 \ll n \ll N$) alors la loi de Bernoulli devient  
une loi de Poisson. Cela montre que la m\^eme loi hyperg\'eom\'etrique  
peut se transformer en  plusieurs sortes de lois asymptotiques selon 
l'\'echelle, c'est-\`a-dire selon les ordres de grandeur relatifs des  
nombres entiers $N$, $p$, $n$, et $k$. Mais on constate que l'on retrouve 
toujours la m\^eme famille: loi gaussienne ou loi de Poisson. Cela 
provient de ce que la loi de Bernoulli est la loi d'une somme de variables 
ind\'ependantes.  
  
\vskip12pt plus10pt minus5pt
 
{\bf Les lois du khi-deux et de Student.} 
\medskip 
Enfin, deux lois asymptotiques \`a densit\'e continue que nous signalons  
sans insister: la loi du khi-deux ou $\chi^2$ est la densit\'e d'une somme 
de {\it carr\'es} de variables gaussiennes ind\'ependantes. Si $X_1$,  
$X_2$, $\ldots$ $X_n$ sont $n$ variables al\'eatoires ind\'ependantes, 
ayant chacune une loi de densit\'e gaussienne de variance $1$ et de 
moyenne  $0$, alors la somme des carr\'es $S = X_1^2 + X_2^2 + \cdots + 
X_n^2$ aura la densit\'e $\chi^2$ \`a $n$ degr\'es de libert\'e. Cette 
densit\'e sera \'etudi\'ee en d\'etail au chapitre {\bf XI}. Elle est \`a la 
base d'un test statistique simple et bien connu.  
\medskip 
La loi de Student est la loi du quotient d'une variable gaussienne par la 
racine carr\'ee d'une somme de carr\'es de gaussiennes: si $Y$, $X_1$,  
$X_2$, $\ldots$ $X_n$ sont $n+1$ variables al\'eatoires gaussiennes 
ind\'ependantes, de variance $1$ et de moyenne $0$, alors la variable 
$$T = {Y \over \sdown{15}\sqrt{X_1^2 + X_2^2 + \cdots + X_n^2} }$$ 
aura une loi de densit\'e de Student. Cette densit\'e aussi est, tout  
comme celle du $\chi^2$, \`a la base d'un test statistique bien connu, et 
sera \'etudi\'ee dans cet esprit au chapitre {\bf XI}. 
\medskip 
Comme le montrent leurs d\'efinitions m\^eme, ces lois ne sont, comme  
la loi de Cauchy, qu'un d\'eguisement de la loi normale. Leur fonction  est 
de simplifier le travail des statisticiens en fournissant des tests {\it 
cl\'es en main}. Ces densit\'es ne se rencontrent pas directement ``dans 
la nature''; elles apparaissent parce qu'on choisit d\'elib\'er\'ement, 
pour la commodit\'e du test, de calculer les grandeurs de mani\`ere 
non lin\'eaire (somme des carr\'es des \'ecarts). 
\medskip 
On peut noter que la loi de Cauchy est le cas particulier 
de la loi de Student pour $n=1$. 
 
 
 
 
 
\bye 
 
