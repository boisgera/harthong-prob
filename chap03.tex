\input/home/harthong/tex/formats/twelvea4.tex
\input/home/harthong/tex/formats/epsf.tex
\vsize=218mm

\auteurcourant={\sl J. Harthong: probabilit\'es et statistique}
\titrecourant={\sl Marches al\'eatoires et lois de l'h\'er\'edit\'e}

\pageno=61 
 
\null\vskip10mm plus4mm minus3mm 
 
\centerline{\tit III. DEUX APPLICATIONS CLASSIQUES:} 
\vskip7pt 
\centerline{\tit les marches al\'eatoires et les lois de l'h\'er\'edit\'e.} 

\vskip10mm plus4mm minus3mm  
 
Une marche al\'eatoire \`a une dimension est le mouvement d'un point 
mat\'eriel qui fait des pas vers l'avant ou vers l'arri\`ere sur un axe,  
chacune de ces deux possibilit\'es \'etant choisie au hasard. On peut 
d\'esigner par $t_0,\, t_1,\, t_2,\, \ldots$ les instants successifs o\`u  
ces pas sont effectu\'es, et par $\varepsilon_0 = t_1-t_0,\, 
\varepsilon_1 = t_2-t_1,\, \varepsilon_2 = t_3-t_2,\, \ldots$ les 
intervalles de temps entre deux pas successifs. Alors $x_0,\, x_1,\, 
x_2,\, \ldots$ seront les abscisses successives du point; les amplitudes 
de chaque pas seront $\alpha_0 = x_1-x_0,\, \alpha_1 = x_2-x_1,\, 
\alpha_2 = x_3-x_2,\, \ldots$. On ne se pr\'eoccupera pas de la 
cin\'ematique int\'erieure aux intervalles (en fait  on fera comme si la 
vitesse entre deux pas cons\'ecutifs \'etait constante).   
\medskip 
\`A chaque pas, le sens (avant ou arri\`ere) peut \^etre choisi par un  
tirage \`a pile ou face, par un algorithme de nombres au hasard (par 
exemple la fonction {\bf random} des langages de programmation usuels), 
par des collisions avec les mol\'ecules d'un liquide, ou tout autre 
proc\'ed\'e.  
\medskip 
Nous allons \'etudier les marches al\'eatoires {\it uniformes},  pour 
lesquelles les $\varepsilon_j$ et les valeurs absolues des $\alpha_j$  
sont toutes \'egales: $\varepsilon_j = \varepsilon$ et $\alpha_j = 
\pm\alpha$. L'\'etude des marches al\'eatoires non uniformes n'est pas 
plus difficile dans le principe, mais les calculs \`a faire sont alourdis 
par la variation des $\varepsilon_j$ et $\alpha_j$: ces deux param\`etres 
deviennent des {\it variables al\'eatoires} (voir chapitre {\bf VI}). 
L'\'etudiant qui a compris le cas uniforme et assimil\'e le chapitre {\bf  
VI} sur les variables al\'eatoires saura traiter lui-m\^eme le cas non 
uniforme  (en fait les propri\'et\'es sont les m\^emes). 
\medskip 
Nous ne donnons ici que des rudiments sur les marches al\'eatoires, en
traitant au moins par exemple les marches uniformes  \`a une dimension. 
Les cas les plus int\'eressants de marches al\'eatoires sont en dimension 
deux ou trois, et non uniformes. Alors, \`a chaque pas, au  lieu de 
{\it deux} choix possibles pour le sens, il y a toutes 
les directions possibles: cela correspond \`a une distribution de 
probabilit\'e sur le  cercle en dimension deux, sur la sph\`ere en dimension 
trois (une telle distribution pouvant bien s\^ur \^etre discr\'etis\'ee), 
avec en outre une distribution de probabilit\'e pour la longueur du pas.   
Une \'etude des marches al\'eatoires de dimensions sup\'erieures \`a un  
est math\'ematiquement assez p\'enible mais semblable dans son 
principe au cas de la dimension 1. 
\medskip 
Une marche al\'eatoire en dimension trois est un mod\`ele  
math\'ematique  pour le mouvement brownien.  Mais beaucoup de 
propri\'et\'es du mouvement brownien se voient d\'ej\`a sur les  
marches al\'eatoires uniformes de dimension un,  d'o\`u l'int\'er\^et 
scolaire d'une telle \'etude.  
\medskip 
Il faut ajouter \`a cela que les marches al\'eatoires, surtout en dimension  
1, peuvent aussi servir de mod\`ele \`a des probl\`emes de probabilit\'es 
tr\`es simples, tels que la partie de {\it pile ou face}. Le principal  
avantage de la mod\'elisation par marche al\'eatoire est la possibilit\'e  
de raisonner g\'eom\'etriquement: on peut alors utiliser toutes les 
ressources de la g\'eom\'etrie euclidienne (voir {\bf III. 3}
le principe de sym\'etrie de D\'esir\'e Andr\'e).
\bigskip
Un domaine d'application important et significatif du Calcul des
pro\-ba\-bi\-lit\'es  est aussi la g\'en\'etique. Depuis qu'on conna{\^\i}t
les m\'ecanismes mol\'eculaires de l'h\'er\'edit\'e on a pu en expliquer
certaines lois empiriques par le Calcul des probabilit\'es. Cela r\'esulte
de ce que ces m\'ecanismes mol\'eculaires sont des combinaisons de
g\`enes ob\'eissant \`a une causalit\'e spatio-temporelle, et sont par
cons\'equent soumis,  comme les boules qu'on dispose dans des bo\^\i
tes, aux diff\'erentes r\`egles de d\'enombrement du chapitre {\bf II}. 
Nous en donnerons un bref aper\c cu en pr\'esentant deux exemples
simples et c\'el\`ebres de lois de la g\'en\'etique, celles de  Mendel
(section {\bf III.5}) et de Hardy-Weinberg (section {\bf III.6}).  On
verra que le Calcul des probabilit\'es joue un r\^ole capital en
g\'en\'etique, car c'est lui qui en fait une science exacte (quantitative). 
\bigskip 
Le contenu de ce chapitre ne pr\'etend \`a aucune originalit\'e: on pourra
en trouver un expos\'e pratiquement identique dans d'autres ouvrages, 
notamment celui de William Feller. Mais les applications trait\'ees ici
sont tellement typiques que nous les utiliserons fr\'equemment dans la
suite comme exemples pour rendre plus concr\`ete l'introduction de
tel ou tel nouveau concept. Il serait g\^enant pour la coh\'erence du
pr\'esent ouvrage de renvoyer \`a chaque fois le lecteur \`a d'autres
sources. 

\vskip7mm plus4mm minus3mm

{\bf III. 1. Graphe d'une marche al\'eatoire.}
\medskip 
\midinsert
\epsfxsize=\larg
\epsfbox{../imgEPS/ch03eps/fig6a.eps}
\endinsert
\midinsert
\epsfxsize=\larg
\epsfbox{../imgEPS/ch03eps/fig6b.eps}
\endinsert
\midinsert
\epsfxsize=\larg
\epsfbox{../imgEPS/ch03eps/fig6c.eps}
\endinsert
\midinsert
\epsfxsize=\larg
\epsfbox{../imgEPS/ch03eps/fig6d.eps}
\endinsert
On suppose donc \`a partir de maintenant que les marches al\'eatoires sont 
toujours uniformes. Une marche al\'eatoire \'etant avant tout un {\it 
mouvement}, on peut repr\'esenter l'\'evolution du point mat\'eriel sur un 
graphique, avec le temps en abscisse et la position sur l'axe en ordonn\'ee. 
Chaque instant $t_j = j\varepsilon$ peut correspondre \`a un changement  
du sens de parcours, ce qui sur un tel graphique se traduit par un 
changement de pente (voir figures pages suivantes). Dans le graphique on fait 
comme si la vitesse entre deux changements de direction cons\'ecutifs 
\'etait constante, quoique ce d\'etail soit insignifiant; par ailleurs, 
on choisit les unit\'es sur les axes $t$ et $x$ de sorte que la vitesse soit 
toujours 1 en valeur absolue.    
\medskip 
L'espace des \'epreuves pour une marche al\'eatoire \`a $n$ pas est  
l'ensemble de {\it tous} les mouvements possibles; mais chaque 
mouvement possible est d\'etermin\'e par la liste des $n$ choix de sens: 
on peut le repr\'esenter par une liste de $n$ signes $+$ ou $-$. L'espace 
des \'epreuves $\Omega$ est donc isomorphe \`a celui des parties de pile 
ou face \`a $n$ lancers, en particulier son cardinal est $2^n$: il y a $2^n$ 
marches al\'eatoires diff\'erentes \`a $n$ pas, qui correspondent \`a 
$2^n$ graphes diff\'erents, ou \`a $2^n$ suites diff\'erentes de $+$ et de 
$-$. Le hasard pur se traduit ici par le fait que toutes les marches 
possibles sont \'equiprobables.  
\medskip 
Cette repr\'esentation g\'eom\'etrique sous forme de graphes est bien  
s\^ur {\it logiquement} \'equivalente \`a la repr\'esentation sous forme 
de suites form\'ees de $+$ et de $-$, mais le fait qu'elle soit 
g\'eom\'etrique permet de poser des probl\`emes de suites sous une 
forme imag\'ee qui, comme nous le verrons, peut fournir des m\'ethodes 
astucieuses de r\'esolution.  
\medskip 
Ainsi une question qui se formule tr\`es ais\'ement dans le langage 
g\'eo\-m\'e\-tri\-que est ``quelle est la probabilit\'e pour qu'une marche  
al\'eatoire partie de $0$ \`a l'instant $0$ aboutisse \`a $x$ \`a l'instant 
$n$''. Essayez donc de formuler la question \'equivalente dans le langage 
des suites de $+$ et de $-$. Pourtant, si on {\it pose} mieux la question 
dans le langage g\'eom\'etrique, on la {\it r\'esoud} mieux en 
consid\'erant les suites. En effet, si une suite est faite de $p$ signes $+$ 
et de $q$ signes $-$ (avec \'evidemment $p+q=n$), la marche 
correspondante parviendra \`a l'ordonn\'ee $x=p-q$. Cette valeur ne 
d\'epend pas de l'ordre des $+$ et des $-$, mais seulement de leur 
nombre. Autrement dit, toutes les marches qui ont effectu\'e $p$ pas en 
avant et $q$ pas en arri\`ere aboutissent au m\^eme point $x$. Leur 
nombre est le nombre de mani\`eres diff\'erentes de placer $p$ signes 
$+$ et $q$ signes $-$, soit $n! / p!\, q!$ (chapitre {\bf II}, 
section {\bf 3}).  Si on veut calculer $p$ et $q$ en fonction de $x$, 
il suffit de r\'esoudre le syst\`eme  
$$\eqalign{ 
p+q &= n \cr 
p-q &= x \cr }$$ 
si $n$ et $x$ ont la m\^eme parit\'e, cela donne $p = {1 \over 2} (n + x),\;  
q = {1 \over 2} (n - x)$ (si $n$ et $x$ n'ont pas la m\^eme parit\'e il n'y a 
aucune solution). Ainsi l'\'ev\'enement $A$: ``la marche, partie de 0 \`a 
l'instant 0 aboutit \`a $x$ \`a l'instant $n$'' est de cardinal $n! / p!\, q!$ 
avec $p = {1 \over 2} (n + x),\;  q = {1 \over 2} (n - x)$ si $n$ et $x$ ont  
la m\^eme parit\'e, et de cardinal 0 si $n$ et $x$ n'ont pas la m\^eme 
parit\'e. D'o\`u la probabilit\'e:  
$${\cal P}\, (A)\;\;  = \cases{  
2^{-n}\, {n! \over p!\, q!} & si $n$ et $x$ ont m\^eme parit\'e \cr  
\noalign{\medskip} 
\quad  0 & si $n$ et $x$ ont des parit\'es diff\'erentes. \cr }$$ 
 
\vskip7mm plus4mm minus3mm  
 
{\bf III. 2. Le probl\`eme du retour \`a z\'ero d'une marche al\'eatoire.} 
\medskip 
Dans le paragraphe pr\'ec\'edent nous avons cherch\'e la probabilit\'e  
pour  qu'une marche de $n$ pas aboutisse (au $n^{\rm e}$, c'est-\`a-dire 
au {\it dernier} pas) au point d'ordonn\'ee $x$. Pour $m < n$ on peut aussi 
chercher la probabilit\'e pour qu'une marche (toujours de $n$ pas) passe 
en un point donn\'e $x$ au $m^{\rm e}$ pas: soit donc $A_{m,x}$ 
l'\'ev\'enement: ``la  marche passe au point $x$ au $m^{\rm e}$ pas''. Ce 
qui a \'et\'e vu au paragraphe pr\'ec\'edent s'applique \'egalement. Pour 
une marche appartenant \`a $A_{m,x}$ il faut que parmi les $m$ premiers 
pas il y ait $p$ pas en avant et $q$ pas en arri\`ere avec $p+q = m$ et 
$p-q = x$; par contre ce qui arrive apr\`es le $m^{\rm e}$ pas est 
indiff\'erent. Il y a donc (du moins si $m$ et $x$ sont de m\^eme parit\'e) 
$m! / p! q!$ possibilit\'es avec les $m$ premiers pas et $2^{n-m}$ 
possibilit\'es pour les pas suivants (de $m+1$ \`a $n$) de sorte que $\# 
A_{m,x} = 2^{n-m}\, {m \choose p}$ avec $p = {1 \over 2}(m+x)$ si $x$ a  
la m\^eme parit\'e que $m$, et 0 sinon. On voit qu'on obtient pour la 
probabilit\'e  $$ {\cal P}\, (A_{m,x}) = {\# A_{m,x} \over \#\Omega } = 
{2^{n-m}\, {m \choose p} \over 2^n} = 2^{-m}\, {m \choose p}$$ 
c'est-\`a dire la m\^eme chose que si l'espace des \'epreuves $\Omega$  
avait \'et\'e l'ensemble de toutes les marches \`a $m$ pas au lieu d'\^etre 
l'ensemble de toutes les marches \`a $n$ pas. On voit que le r\'esultat ne 
d\'epend pas de la mod\'elisation choisie. Si donc on s'int\'eresse \`a ce  
qui se produit \`a l'instant $m$, ou avant, mais que ce qui arrive apr\`es 
est indiff\'erent, il est inutile de consid\'erer des marches ayant plus de 
$m$  pas.  
\medskip 
Un cas particulier int\'eressant est celui o\`u $x = 0$. Si l'\'ev\'enement 
$A_{m,0}$ correspondant se produit, on dira aussi: ``il y a un retour \`a 
z\'ero \`a l'instant $m$''. Cet \'ev\'enement est vide si $m$ est impair, et 
de  probabilit\'e  $2^{-2\ell }\, {2\ell \choose \ell}$ si $m = 2\ell$.  
Ce r\'esultat se d\'eduit par une application imm\'ediate de $(II.5.)$  
\medskip 
Pour les retours \`a z\'ero il se pose cependant un autre probl\`eme, celui  
du {\it premier} retour \`a z\'ero. La question est cette fois de trouver la 
probabilit\'e pour qu'\`a l'instant $m = 2\ell$ la marche soit revenue \`a 
z\'ero, {\it et qu'en outre} il n'y ait eu aucun autre retour \`a z\'ero avant.  
Il est clair que la probabilit\'e pour qu'\`a l'instant $m$ se produise {\it  
le premier} retour \`a z\'ero est inf\'erieure \`a la probabilit\'e pour 
qu'\`a l'instant $m$ se produise {\it un} retour \`a z\'ero.  Mais quelle
est sa valeur exacte ? C'est ce que nous nous proposons de calculer 
maintenant. On voit que pour tous ces probl\`emes (retour \`a z\'ero ou 
premier retour \`a z\'ero \`a l'instant $m$) il est inutile de consid\'erer 
ce qui se passe apr\`es l'instant $m$, et donc nous prenons pour 
$\Omega$ l'ensemble des marches de $m$ pas, qui contient $2^m$ \'el\'ements.  
\medskip 
L'\'ev\'enement ``\`a l'instant $m$ se produit un retour \`a 0'' \'etant 
vide si $m$ est impair nous posons $m = 2\ell$; les marches appartenant \`a  
cet \'ev\'enement ont en commun que $x_0 = 0$ et $x_{2\ell} = 0$. Par 
contre  les marches appartenant \`a l'\'ev\'enement $P_\ell$ : ``\`a 
l'instant $2\ell$ se produit le {\it premier} retour \`a 0'' v\'erifient en 
outre $x_1 \neq 0, \, x_2 \neq 0, \,  x_3 \neq 0, \ldots x_{2\ell - 1} \neq 
0$. On ne peut d\'enombrer  $P_\ell$ par application imm\'ediate d'une 
formule du chapitre $II$. Ce type de probl\`eme est plus d\'elicat, mais 
nous allons voir qu'en exploitant la g\'eom\'etrie on peut trouver des 
m\'ethodes efficaces; en effet, le grand avantage des marches 
al\'eatoires est leur sens g\'eom\'etrique: les graphes se situent dans le 
plan euclidien et par cons\'equent toute la g\'eom\'etrie euclidienne peut 
\^etre mise \`a profit. Or beaucoup de probl\`emes de probabilit\'es qui au 
d\'epart n'ont rien \`a voir avec la g\'eom\'etrie peuvent se mod\'eliser 
en termes de marches al\'eatoires: par exemple un jeu de pile ou face 
avec mises entre deux joueurs $P$ et $F$, o\`u $P$ re\c coit $1$ Euro de 
$F$ chaque fois que sort {\it pile}, tandis que $P$ donne $1$ Euro \`a $F$ 
chaque fois que sort {\it face} . Une marche al\'eatoire est en effet 
caract\'eris\'ee de fa\c con biunivoque par la suite des signes $+$ et $-$ 
de chacun de ses pas. Si donc on traduit $+$ par {\it pile} et $-$ par {\it 
face} on voit qu'il y a une correspondance bijective entre l'ensemble de 
toutes les marches possibles et l'ensemble de toutes les parties de pile 
ou face possibles. L'ordonn\'ee  $x_j$ est alors le gain de $P$ \`a 
l'instant $j$.  L'\'ev\'enement $P_\ell$ signifie dans ce cas que le gain de 
$P$ est rest\'e constamment positif de l'instant $1$ jusqu'\`a l'instant 
$2\ell - 1$.   
\medskip 
Une premi\`ere \'etape pour le d\'enombrement de $P_\ell$ fait d\'ej\`a  
appel \`a la g\'eom\'etrie: elle consiste \`a remarquer que par sym\'etrie, 
\`a tout graphe tel que $x_0 = 0, \, x_1 > 0, \, x_2 > 0, \,  x_3 > 0, \ldots 
x_{2\ell - 1} > 0$ correspond bijectivement un graphe tel que $x_0 = 0, \, 
x_1 < 0, \, x_2 < 0, \,  x_3 < 0, \ldots x_{2\ell - 1} < 0$ (figure 7). 
Mais d'autre part un graphe tel que $x_0 = 0, \, x_1 \neq 0, \, x_2 
\neq 0, \, x_3 \neq 0, \ldots x_{2\ell - 1} \neq 0$ ne peut \^etre que
de l'un ou l'autre des deux types pr\'ec\'edents, en vertu du fait bien
connu qu'on ne peut passer du n\'egatif au positif (ou vice-versa) sans
passer par 0 (on pourrait passer par exemple de $-\alpha$ \`a $+\alpha$
avec un pas \'egal \`a $2\alpha$,  mais la marche al\'eatoire ne  fait
que des pas \'egaux \`a $\pm \alpha$). 
Par cons\'equent l'\'ev\'enement $P_{2\ell}$: $x_0 = 0, \, x_1 \neq 0, \, 
x_2 \neq 0, \,  x_3 \neq 0, \ldots x_{2\ell - 1} \neq 0,\, x_{2\ell} = 0$ 
doit contenir exactement {\it deux} fois le nombre d'\'el\'ements de 
$P_{2\ell}^{(+)}$: $x_0 = 0, \, x_1 > 0, \, x_2 > 0, \,  x_3 > 0, \ldots 
x_{2\ell - 1} > 0,\, x_{2\ell} = 0$ ou de $P_{2\ell}^{(-)}$: $x_0 = 0, \, x_1  
< 0, \, x_2 < 0, \,  x_3 < 0, \ldots x_{2\ell - 1} < 0,\, x_{2\ell} = 0$. 
 
\vskip7mm plus4mm minus3mm  
\penalty-800
 
{\bf III. 3. Le principe de sym\'etrie de D\'esir\'e Andr\'e.} 

\penalty800
\medskip 
Cette utilisation des sym\'etries g\'eom\'etriques pour le  
d\'enombrement  est attribu\'ee historiquement \`a D\'esir\'e Andr\'e 
({\oldstyle 1887}). Elle consiste \`a \'etablir des correspondances 
biunivoques entre \'ev\'enements par sym\'etrie ou translation dans le 
plan (ou dans l'espace). Pour cela il faut bien s\^ur avoir trouv\'e 
auparavant une mod\'elisation g\'eom\'etrique du probl\`eme. 
\medskip 
Nous allons appliquer une seconde fois ce principe pour calculer le 
cardinal de $P_{2\ell}^{(+)}$ (et par voie de cons\'equence, celui de 
$P_{2\ell}$).  On remarquera tout d'abord qu'une marche appartenant \`a  
$P_{2\ell}^{(+)}$ v\'erifie n\'ecessairement $x_{2\ell - 1} = +1$ et  
$x_{2\ell  - 2} = +2$; en effet on ne peut arriver \`a $x_{2\ell} = 0$ que  
par $x_{2\ell  - 1} = +1$ ou $x_{2\ell - 1} = -1$, la seconde possibilit\'e 
\'etant exclue dans   $P_{2\ell}^{(+)}$. De m\^eme, on ne peut arriver \`a 
$x_{2\ell - 1} = +1$ que par $x_{2\ell - 2} = +2$ ou $x_{2\ell - 2} = 0$, la 
seconde possibilit\'e \'etant elle aussi exclue dans $P_{2\ell}^{(+)}$. Le 
nombre de marches v\'erifiant $x_0 = 0, \, x_1 > 0, \, x_2 > 0, \,  x_3 > 0, 
\ldots x_{2\ell - 1} > 0,\, x_{2\ell} = 0$ est donc \'egal au nombre de 
marches v\'erifiant $x_0 = 0, \, x_1 > 0, \, x_2 > 0, \,  x_3 > 0, \ldots 
x_{2\ell - 3} > 0,\, x_{2\ell - 2} = 2$. 
\medskip

\midinsert 
\epsfxsize=\larg
\centerline{\epsfbox{../imgEPS/ch03eps/fig7.eps}} 
\vskip2pt 
\centerline{\eightpoint figure 7} 
\vskip6pt 
\endinsert 

Ainsi nous sommes ramen\'es \`a d\'enombrer l'\'ev\'enement $C_{2\ell  
-  2}^2$: $x_0 = 0, \, x_1 > 0, \, x_2 > 0, \,  x_3 > 0, \ldots x_{2\ell - 3} > 
0,\, x_{2\ell - 2} = 2$. Appelons plus g\'en\'eralement pour $n$ et $r$ 
quelconques $C_{2n}^{2r}$  l'\'ev\'enement  $x_0 = 0, \, x_1 > 0, \, x_2 > 
0, \,  x_3 > 0, \ldots x_{2n -1} > 0,\, x_{2n} = 2r$. $C_{2n}^{2r}$ 
est ainsi l'ensemble des marches al\'eatoires issues de $0$ \`a l'instant  
$0$ et aboutissant \`a $2r$ \`a l'instant $2n$, et qui restent constamment 
positives. Or une marche  al\'eatoire issue de $0$ \`a l'instant $0$ et 
aboutissant \`a $2r$  (avec $2r > 0$) \`a l'instant $2n$ est ou bien 
toujours positive, ou bien peut s'annuler au moins une fois entre $0$ et 
$2n$ (ce {\it ou bien} \'etant exclusif). C'est pourquoi nous allons 
consid\'erer les marches al\'eatoires telles que $x_0 = 0, \, x_1 = +1, \,  
x_{2n} = 2r$ et les d\'ecomposer en deux parties disjointes, celles qui 
restent constamment positives, et celles qui repassent par 0 entre les 
instants 1 et $2n$.  
\medskip 
Introduisons donc les \'ev\'enements: 
$$\eqalign{ 
&A_{2n}^{2r}: x_0=0,\, x_1=+1, \;\hbox{\eightrm et}\; x_{2n}=2r\, ;\cr 
&B_{2n}^{2r}: x_0=0,\, x_1=+1, \,  x_{2n}=2r \; \hbox{\eightpoint  
et il existe un $j$,  $1<j<2n$, tel que $x_j = 0$}\, ; \cr 
&D_{2n}^{2r}: x_0=0,\, x_1=-1, \;\hbox{\eightrm et}\; \, x_{2n}=2r\, ;\cr  
&E_{2n}^{2r}: x_0 = 0, \, x_{2n} = 2r\, . \cr}$$ 
L'\'ev\'enement $A_{2n}^{2r}$ est donc l'ensemble de toutes les marches 
telles que $x_0 = 0, \, x_1 = +1, \,  x_{2n} = 2r$, positives ou non;  
$B_{2n}^{2r}$ est le sous-ensemble de $A_{2n}^{2r}$ des marches qui 
repassent par $0$, et, rappelons-le, $C_{2n}^{2r}$ le sous-ensemble des 
marches toujours positives; de sorte que $A_{2n}^{2r} = B_{2n}^{2r} \cup 
C_{2n}^{2r}$, r\'eunion de deux ensembles disjoints, d'o\`u  
$$\# A_{2n}^{2r} = \# B_{2n}^{2r} + \# C_{2n}^{2r}$$  
Mais comme le montre la figure 7, on peut associer de mani\`ere  
biunivoque \`a tout graphe de $B_{2n}^{2r}$ un graphe de $D_{2n}^{2r}$,  
en vertu de la sym\'etrie par rapport \`a l'axe des abscisses: pour 
n'importe quelle marche appartenant \`a $B_{2n}^{2r}$, on prend la 
branche positive entre l'instant $0$ et celui du premier retour \`a  
z\'ero (qui a forc\'ement lieu puisqu'on est dans $B_{2n}^{2r}$), et on  
la remplace par sa sym\'etrique, en laissant intact ce qui arrive apr\`es 
le premier retour \`a z\'ero.    
\medskip 
Par cons\'equent:  
$$\# B_{2n}^{2r} = \# D_{2n}^{2r}$$ 
(ceci constitue le principe de sym\'etrie de D\'esir\'e Andr\'e).  
\medskip 
On en d\'eduit alors un moyen de calculer  $\# C_{2n}^{2r} = \# 
A_{2n}^{2r} - \# D_{2n}^{2r}$: en effet les cardinaux de $A_{2n 
}^{2r}$ et $D_{2n}^{2r}$ peuvent, eux (contrairement \`a celui de  
$C_{2n}^{2r}$), \^etre calcul\'es par application imm\'ediate de $(II.5.)$: 
$A_{2n}^{2r}$ \'equivaut \`a l'ensemble de toutes les marches issues de 
$+1$ \`a l'instant $1$ et aboutissant \`a $2r$ \`a l'instant $2n$, qui  
doivent donc faire $p$ pas en avant et $q$ pas en arri\`ere avec $p+q = 
2n -1$ et $p-q = 2r-1$, il  y en a $(2n-1)! / (n+r-1)!\, (n-r)!$; 
$C_{2n}^{2r}$ \'equivaut \`a l'ensemble de toutes les marches issues de 
$-1$ \`a l'instant $1$ et aboutissant \`a $2r$ \`a l'instant $2n$, qui  
doivent donc faire $p$ pas en avant et $q$ pas en arri\`ere avec $p+q = 
2n-1$ et $p-q = 2r+1$, il y en a $(2n-1)! / (n+r)!\, (n-r-1)!$ 
D'o\`u    
$$\openup 2\jot \eqalignno{  
\# C_{2n}^{2r} &= \# A_{2n}^{2r} - \# D_{2n}^{2r} \cr  
& = {2n-1 \choose n+r-1} - {2n-1 \choose n+r} \cr 
&= {(2n-1)! \over (n+r-1)!\, (n-r)!} - {(2n-1)! \over (n+r)!\, 
(n-r-1)!}  &(III.1.)\cr  
&= {(2n-1)! \over (n+r)!\, (n-r)!} \cdot 2r  \cr 
&= {(2n)! \over (n+r)!\, (n-r)!} \cdot {r \over n} \cr }$$  
Il ne reste plus qu'\`a voir que dans le cas particulier qui nous int\'eresse 
($r = 1$ et $2n = 2\ell - 2$) on a  $\# C_{2\ell - 2}^2 = \bigl( (2n)! / n!^2 
\bigr)\cdot \bigl( 1 / 2(2n-1)\bigr)$. D'apr\`es ce qui avait \'et\'e dit  
plus haut l'\'ev\'enement $P_{2\ell}$ a un cardinal exactement double, de 
sorte que la probabilit\'e que le {\it premier} retour se produise \`a 
l'instant  $2\ell$ est   
$${\cal P}\, (P_{2\ell}) = 2^{-2\ell}\; {(2\ell )! \over \ell !^2} \; 
{1 \over 2\ell-1} \eqno (III.2.)$$  
On voit qu'elle est $2\ell - 1$ fois plus petite que la probabilit\'e pour  
qu'il y ait un retour \`a z\'ero (non n\'ecessairement le premier) \`a 
l'instant $2\ell$. 

\vskip7mm plus4mm minus3mm  
\penalty-800 
{\bf III. 4. La loi du dernier retour.} 

\penalty800 
\medskip 
Un autre probl\`eme qui m\'erite une \'etude est le suivant: consid\'erons  
une marche al\'eatoire de $2n$ pas\ftn 1{Dans le cas impair $2n-1$ le 
calcul est un peu diff\'erent mais de fa\c con absolument inessentielle; 
c'est pourquoi on peut le laisser de c\^ot\'e.}. Quelle est la probabilit\'e 
pour que le pr\'ec\'edent retour \`a z\'ero ait eu lieu \`a l'instant $2k$ ? 
(c'est-\`a-dire que la marche soit pass\'ee par z\'ero \`a l'instant $2k$ 
mais plus jamais ensuite entre $2k$ et $2n$).  
\medskip 
Peut-\^etre cette fa\c con de formuler le probl\`eme n'est-elle pas assez 
concr\`ete. Alors imaginons les choses ainsi: supposons qu'on reproduise  
des milliards de fois une marche al\'eatoire (par exemple un ordinateur 
dessine des milliards de graphes du type de la figure 6 \`a partir d'un 
programme qui appelle la fonction {\bf random}). Pour chacun de ces 
graphes on note le dernier instant avant la fin o\`u le graphe est pass\'e  
par z\'ero, c'est-\`a-dire l'instant du dernier retour. Comment se 
r\'epartissent ces instants ? Il se trouve que leur r\'epartition est 
inattendue: il est fortement probable que ce dernier retour ait eu lieu  
\`a un instant soit tout r\'ecent (c'est-\`a-dire proche de $2n$), soit 
tr\`es ancien (c'est-\`a-dire proche de $0$), et faiblement probable qu'il 
ait eu lieu \`a mi-chemin. Pour donner une id\'ee plus quantitative, on 
peut dire que la probabilit\'e pour que le dernier retour ait eu lieu 
pendant le premier dixi\`eme de la dur\'ee totale du parcours est $1 
\over 5$, la probabilit\'e pour qu'il ait eu lieu pendant le dernier 
dixi\`eme est aussi $1 \over 5$, et la probabilit\'e pour qu'il ait eu lieu 
pendant les huit dixi\`emes interm\'ediaires est $3 \over 5$.   
\medskip   
Avec cette r\'epartition des probabilit\'es on doit s'attendre \`a ce que  
pour un \'echantillon de graphes pris au hasard, un nombre important 
d'entre eux aient  fluctu\'e tout au d\'ebut autour de z\'ero, pour ensuite 
rester toujours loin de l'axe; par exemple d'apr\`es les chiffres ci-dessus 
une marche sur cinq environ devrait fluctuer autour de z\'ero pendant le 
premier dixi\`eme de son parcours, puis demeurer constamment positive  
ou constamment n\'egative sur les neuf dixi\`emes restants: un coup 
d'oeil \`a la figure 6 montre que tel est bien le cas (il n'y a eu aucun 
trucage). Une marche sur {\it dix} devrait ne fluctuer autour de z\'ero que 
sur le premier {\it cinquanti\`eme} de son parcours. Ces longs s\'ejours 
de la marche al\'eatoire  loin de 0 ne sont donc pas l'effet  d'une cause 
inconnue qui favoriserait de tels \'ecarts (par exemple si la fonction {\bf 
random} du logiciel de calcul \'etait incorrecte), mais bien celui du {\it 
pur hasard}. 
 
\bigskip 
 
Pour poser math\'ematiquement le probl\`eme, \'ecrivons l'\'ev\'enement  
$A_k$ qui, en tant que sous-ensemble de l'ensemble de toutes les  
marches possibles, exprime le fait que le dernier retour en 0 a eu lieu \`a 
l'instant $2k$. Une marche appartenant \`a $A_k$ v\'erifie donc $x_0 = 0$ 
et $x_{2k} = 0$, mais aucune condition n'est requise pour les $x_j$ avec 
$0 < j < 2k$. Pour une marche al\'eatoire allant de z\'ero \`a $2n$ il y a 
donc $2k \choose k$ possibilit\'es diff\'erentes pour la partie ayant lieu 
entre $0$ et $2k$. En ce qui concerne la partie entre $2k$ et $2n$, on 
commencera  par remarquer que l'on doit avoir  $x_{2k+1} \neq 0, \, 
x_{2k+2} \neq 0, \, x_{2k+3}  \neq 0, \ldots  x_{2n-1} \neq 0$, et aussi, 
si on suppose qu'il n'y a pas retour \`a z\'ero \`a l'instant $2n$ 
lui-m\^eme (car alors on aurait $k = n$), $x_{2n} \neq 0$. \`A cause de 
l'invariance par translation, il y a exactement autant de possibilit\'es 
pour cette partie de la marche situ\'ee entre $2k$ et $2n$ que pour une 
marche de $2n - 2k$ pas v\'erifiant $x_{0} = 0, \, x_{1} \neq 0, \, x_{2} 
\neq 0, \, x_{3} \neq 0, \ldots  x_{2n-2k-1} \neq 0, \, x_{2n-2k} \neq 0$. 
L'ensemble de toutes les possibilit\'es pour les $2n$ pas est alors le {\it 
produit} du nombre de possibilit\'es pour les $2k$ premiers pas par le 
nombre de possibilit\'es  pour les $2n-2k$ pas suivants, puisque \`a 
chaque possibilit\'e pour les $2k$ premiers pas on peut ajouter n'importe 
laquelle des possibilit\'es pour les pas suivants\ftn 2{Cette propri\'et\'e 
est en fait l'ind\'ependance stochastique, cf. chapitre $IV$.}. 
\medskip 
Le probl\`eme est donc r\'eduit \`a d\'enombrer les marches telles que  
$x_{0} = 0, \, x_{1} \neq 0, \, x_{2} \neq 0, \, x_{3} \neq 0, \ldots  
x_{2m-1} \neq 0, \, x_{2m} \neq 0$ pour $m$ quelconque, apr\`es quoi on 
prendra $m = n-k$. Par sym\'etrie leur nombre est exactement le double  
de celles qui v\'erifient $x_{0} = 0, \, x_{1} > 0, \, x_{2} >0, \, x_{3} > 0, 
\ldots  x_{2m-1} > 0, \, x_{2m} > 0$. Or une marche qui v\'erifie ces 
conditions appartient \`a un et un seul des \'ev\'enements $A_{m}^{r}\; :  
\; x_{0} = 0, \, x_{1} > 0, \, x_{2} >0, \, x_{3} > 0, \ldots  x_{2m-1} > 0, \, 
x_{2m} = 2r$ avec $r=1,2,3, \ldots m$. Leur nombre est donc 
$\sum_{r=1}^{r=m}\# A_{m}^{r}$ 
\medskip 
Enfin, pour calculer chacun des $\# A_{m}^{r}$, on utilise le principe  
de sym\'etrie de D\'esir\'e Andr\'e; il suffit en effet de r\'eutiliser  
$(III.1.)$ qui nous dit que  
$$ \# A_{m}^{r} = {2m-1 \choose m+r-1} - {2m-1 \choose m+r}$$ 
On doit sommer cela de $r=1$ \`a $r=m$, ce qui fait que les termes 
s'annulent mutuellement : 
$$\eqalignno{ 
\sum_{r=1}^{r=m} \# A_{m}^{r} &= {2m-1 \choose m} - {2m-1  \choose 
m+1} + {2m-1 \choose m+1} - {2m-1 \choose m+2} + \cr 
&+ {2m-1 \choose m+2} - {2m-1 \choose m+3} + {2m-1 \choose m+3} -  
\cdots \cr  
\noalign{\bigskip} 
&= \; {2m-1 \choose m}\; = \; {1 \over 2} \, {2m \choose m} \cr }$$ 
Le nombre de toutes les marches de $2m$ pas ne revenant jamais \`a  
z\'ero est donc le double, soit ${2m \choose m}$; enfin, le nombre de 
marches de $2n$ pas, revenant \`a z\'ero \`a l'instant $2k$ et n'y revenant 
plus entre $2k+1$ et $2n$ inclus est alors (comme nous l'avions d\'ej\`a 
dit plus haut) le produit ${2k \choose k} \times {2(n-k) \choose n-k}$.  
\bigskip 
De sorte que la probabilit\'e pour que le dernier retour avant l'instant $2n$
se soit produit \`a l'instant $2k$ est (l'approximation vient de $II.7$):
$$ 2^{-2n} \cdot {2k \choose k} \cdot {2(n-k) 
\choose n-k} \simeq {1 \over \pi \sqrt{k(n-k)}}$$ 
Si on cherche par exemple la probabilit\'e pour que le dernier retour ait  
eu lieu {\it avant} l'instant $2\ell$, il suffit de faire la somme de ces 
valeurs pour $k$ allant de $1$ \`a $\ell$ (le dernier retour ne peut 
avoir lieu \`a deux instants diff\'erents \`a la fois, donc les \'ev\'enements 
correspondants sont disjoints et les probabilit\'es s'additionnent). Ainsi 
$$\eqalign{ 
{\cal P}\, (\hbox{dernier retour avant $2\ell$}) &=\sum_{k=1}^{k=\ell} 
2^{-2n} \cdot {2k \choose k} \cdot {2(n-k) \choose n-k} \cr 
&\simeq \sum_{k=1}^{k=\ell} {1 \over \pi \sqrt{k(n-k)}} \cr }$$  
Cette somme discr\`ete peut \^etre interpr\'et\'ee comme la somme de  
Riemann de l'int\'egrale $\int_0^x \bigl[ 1 / \pi \sqrt{t(1-t)}\bigr] \;  
dt$ avec $x = \ell / n$.  Or cette int\'egrale peut \^etre calcul\'ee par 
primitives, elle est \'egale \`a ${2 \over \pi} \arcsin(\sqrt{x})$, de sorte 
que finalement    
$${\cal P}\, (\hbox{dernier retour avant $2\ell$}) \simeq {\up{2} \over  
\down{\pi}} \arcsin\Bigg\{\sqrt{\up{\ell} \over \down{n}}\,\Bigg\}
\eqno (III.3.)$$  
Les estimations donn\'ees au d\'ebut du paragraphe ont \'et\'e tir\'ees  
de  cette approximation. Bien s\^ur il faut \^etre dans le domaine de 
validit\'e  de cette approximation: elle est en principe incorrecte si 
$\ell$ ou $n-\ell$ est trop petit, mais elle donne d\'ej\`a un r\'esultat 
correct \`a $4\%$  pr\`es pour $k$ ou $n-k$ sup\'erieur \`a 3, et \`a  
$1\%$ pr\`es pour $k$ ou $n-k$ sup\'erieur \`a 10. 

\vskip7mm plus4mm minus3mm  

{\bf III. 5. Loi de l'h\'er\'edit\'e de Mendel} 
\medskip
La loi de Mendel concerne la transmission de caract\`eres par  
l'h\'er\'edit\'e. On appelle {\it ph\'enotypes} ces caract\`eres. Un  
exemple de ph\'enotype (\'etudi\'e par Mendel) est la forme, lisse ou 
rid\'ee, des grains de petits pois. D'autres ph\'enotypes connus sont les 
groupes sanguins, le facteur Rh\'esus, la couleur des yeux ou de 
certaines fleurs, le type albinos, des maladies g\'en\'etiques telles  
que  la mucoviscidose. Consid\'erons l'exp\'erience suivante, qu'on peut 
effectuer par exemple sur le  {\it mirabilis},  une plante africaine dont  
les fleurs s'ouvrent la nuit (appel\'ee pour cela  ``belle de nuit''): on a 
s\'electionn\'e sur un grand nombre de  g\'en\'erations deux vari\'et\'es 
de mirabilis: l'une \`a fleurs rouges, l'autre \`a fleurs blanches. Tout 
croisement entre deux fleurs rouges donne un descendant \`a fleurs 
rouges et tout croisement entre deux  fleurs blanches donne un 
descendant \`a fleurs blanches.  On effectue alors des croisements  
entre une fleur rouge et une fleur blanche, qui  donnent toujours des 
descendants \`a fleurs roses. Puis on effectue \`a nouveau des 
croisements entre deux de ces descendants \`a fleurs roses, et on  
observe parfois une fleur rouge, parfois une blanche, parfois une  rose.  
Si on renouvelle un grand nombre de fois les croisements de fleurs  
roses, afin de disposer d'un \'echantillon statistiquement significatif,  
on observe qu'on obtient une fois sur quatre une fleur rouge, une fois  
sur quatre une blanche, et une fois sur deux une fleur rose. Ces 
exp\'eriences avaient \'et\'e effectu\'ees par Gregor Mendel au milieu du 
$XIX^{\rm e}$ si\`ecle sur des souches de petits pois (pois rid\'es et pois 
lisses).  
\medskip 
Les conclusions que Gregor Mendel a tir\'ees de ces exp\'eriences sont  
les suivantes. Le premier croisement entre une fleur rouge et une fleur 
blanche, qui donne une fleur rose, pourrait s'expliquer par le m\'elange  
de deux substances. Mais le second croisement, o\`u l'on peut retrouver  
une fois sur deux une couleur pure, montre que le m\'elange n'a pas 
r\'eellement  eu lieu, et que les \'el\'ements mat\'eriels qui ont servi de 
vecteur aux caract\`eres h\'er\'editaires ont d\^u se transmettre sous 
forme pure  jusqu'\`a la g\'en\'eration suivante. Gregor Mendel a d\'eduit 
de ses exp\'eriences en {\oldstyle 1865} que ces \'el\'ements 
d'h\'er\'edit\'e  restaient s\'epar\'es: les diff\'erents ph\'enotypes 
r\'esultaient bien  de leur combinaison, mais les \'el\'ements devaient  
se combiner en conservant leur int\'egrit\'e au cours des g\'en\'erations. 
La biologie  du  $XX^{\rm e}$ si\`ecle a identifi\'e ces \'el\'ements 
d'h\'er\'edit\'e, qu'on appelle aujourd'hui les g\`enes.  
\medskip 
Le principe de la transmission des caract\`eres est donc le suivant: 
un g\`ene $R\,$ est responsable de la couleur rouge, un autre $B\,$ de la 
couleur blanche; mais chacun des deux parents apporte un g\`ene, et  
c'est  la combinaison $RR\,$ et non la pr\'esence d'un seul g\`ene $R\,$  
qui  produit la couleur rouge. De m\^eme c'est la combinaison $BB\,$ qui   
produit la   couleur blanche. Si l'un des parents est rouge et l'autre   
blanc, la combinaison sera $RB\,$ ou $BR\,$ et produira la couleur rose.  
On appelle {\it homozygote} une combinaison de deux g\`enes 
identiques, et {\it h\'et\'erozygote} une combinaison de deux g\`enes 
diff\'erents. Ainsi $RR\,$ et $BB\,$ sont des combinaisons homozygotes, 
$RB\,$ et $BR\,$ sont  des combinaisons h\'et\'erozygotes. La th\'eorie  
de Mendel  postule que les m\'ecanismes de la reproduction s\'eparent  
\`a nouveau les combinaisons en deux g\`enes intacts, qui se recombinent 
autrement  pour former la g\'en\'eration suivante. Si les deux parents  
sont rouges (combinaison $RR\,$   et $RR$) il n'y a aucun g\`ene $B\,$ qui 
peut appara\^\i  tre et  les enfants seront \'egalement rouges. Si les 
parents  sont tous les deux roses il y  a quatre possibilit\'es:   
\medskip   
\centerline{\vbox{\hsize=10.5cm 
1. Les parents sont $RB$ et $RB$; alors l'enfant est $RR$. 
\smallskip   
2. Les parents sont $RB$ et $BR$; alors l'enfant est $RB$. 
\smallskip   
3. Les parents sont $BR$ et $RB$; alors l'enfant est $BR$. 
\smallskip   
4. Les parents sont $BR$ et $BR$; alors l'enfant est $BB$. } } 
\medskip   
On voit que la combinaison fille a \'et\'e obtenue en retenant la  
premi\`ere lettre  de chacune des deux combinaisons parentales, mais 
bien entendu ceci n'est qu'une convention d'\'ecriture; les parents ne  
``sont'' pas $RB$ ou $BR$; ils sont h\'et\'erozygotes, et ce 
sont les hasards de la recombinaison qui d\'ecident ce qui sera ---~dans 
cette convention~--- la premi\`ere lettre. Le m\'ecanisme 
mol\'eculaire r\'eel de ce processus a \'et\'e \'elucid\'e au milieu du  
$XX^{\rm e}$ si\`ecle, c'est la division des chromosomes. La reproduction 
sexu\'ee fonctionne de la mani\`ere suivante. Tous les g\`enes sont 
regroup\'es en cha\^\i nes ordonn\'ees appel\'ees chromosomes, qui sont  
toujours coupl\'es par paires homologues: ces paires sont des cha\^\i nes  
doubles comportant deux cha\^\i nons sym\'etriques  qui se font face  
(les chromosomes homologues), de telle sorte qu'\`a chaque g\`ene de  
l'un des  cha\^\i nons correspond un compagnon sym\'etrique sur l'autre   
cha\^\i non appel\'e all\`ele. Les combinaisons de deux g\`enes qui  
interviennent dans la  th\'eorie de Mendel concernent ces couples  
d'all\`eles. Lors de la reproduction les deux chromosomes homologues  
se s\'eparent, et chaque chromosome ainsi isol\'e se recombine {\it au 
hasard} avec un chromosome isol\'e de l'autre parent. Chaque parent  
fournit un seul des deux chromosomes de chaque paire, le choix \'etant  
fait au hasard. Ainsi une nouvelle paire sera form\'ee \`a partir d'un 
chromosome du p\`ere et d'un chromosome de la m\`ere. Il y a donc quatre 
recombinaisons \'equiprobables possibles. 
\medskip 
L'ensemble des g\`enes possibles, ainsi que leur place sur les cha\^\i 
nons, est une caract\'eristique invariable de l'esp\`ece. C'est pourquoi  
\`a un endroit donn\'e du chromosome (on appelle cela un {\it locus}) on 
trouvera toujours les m\^emes g\`enes (les all\`eles du locus): les g\`enes  
pouvant occuper un locus donn\'e sont g\'en\'eralement peu nombreux. 
Dans les cas les plus simples, il n'y a qu'un seul all\`ele sur le locus (tout 
le monde est alors homozygote) ou deux, par exemple $R$ et $B$, et alors  
les quatre combinaisons $RR$, $RB$,  $BR$, et $BB$ sont possibles. Mais 
bien entendu ce n'est pas la r\`egle g\'en\'erale: il y a souvent plus que  
deux all\`eles.  Dans une esp\`ece donn\'ee, on trouvera presque 
toujours l'une des combinaisons possibles  d'all\`eles au locus 
correspondant (les exceptions sont les {\it mutations});  mais en un 
autre locus, on trouvera  des combinaisons d'all\`eles diff\'erents. Si les 
deux parents sont h\'et\'erozygotes  et s'il n'y a que deux all\`eles (cas 
o\`u on croise deux fleurs roses), les quatre recombinaisons possibles de 
chromosomes  donneront {\it au locus consid\'er\'e} les  quatre 
combinaisons $BB$,  $BR$, $RB$, et $RR$.  
\medskip 
Dans une population naturelle, il n'y a aucune raison que les all\`eles 
soient tous exactement aussi r\'epandus: g\'en\'eralement, les uns sont 
plus rares que les autres (par exemple les mirabilis \`a fleurs rouges  
sont plus rares que ceux \`a fleurs blanches). Toutefois, dans les 
exp\'eriences comme celle d\'ecrite plus  haut, on  a pr\'ealablement 
s\'electionn\'e la vari\'et\'e rouge et la vari\'et\'e blanche, de sorte  
qu'en croisant les fleurs rouges avec les blanches on a cr\'e\'e 
artificiellement une population comportant exactement autant de   
g\`enes $R\,$ que de g\`enes $B$. La r\`egle statistique observ\'ee par 
Mendel s'explique ais\'ement par l'\'equiprobabilit\'e des quatre 
recombinaisons de chromosomes: au locus consid\'er\'e, si les deux  
parents sont h\'et\'erozygotes, on obtiendra (avec probabilit\'e ${1\over 
4}$ pour chacune) les combinaisons $RR$, $RB$, $BR$, et $BB$. On appelle 
{\it g\'enotype} la combinaison de g\`enes, le ph\'enotype \'etant le  
caract\`ere observable. La couleur rose de  la fleur est un ph\'enotype 
qui correspond indistinctement \`a l'un ou l'autre des g\'enotypes $RB$ ou 
$BR$, donc sa probabilit\'e sera ${1\over 4} + {1\over 4} = {1\over 2}$. 
Le ph\'enotype rouge correspond au g\'enotype  $RR$. Mais une 
telle correspondance entre g\'enotype et ph\'enotype est exceptionnelle; 
l'immense majorit\'e des g\'enotypes n'ont aucun effet ponctuellement 
observable sous  forme de ph\'enotype; non que le  g\'enotype soit sans 
effet, mais cet effet est tellement  dilu\'e qu'il ne  lui correspond aucun  
caract\`ere manifeste.   
\medskip   
\'Etant donn\'ee l'\'equiprobabilit\'e de ces combinaisons,  on peut   
\'etendre les raisonnements probabilistes \`a des situations plus 
g\'en\'erales,  o\`u il n'y a pas exactement autant de g\`enes $R$ que de 
g\`enes $B$ dans la population,  ce qui conduit \`a la loi de 
Hardy-Weinberg.  Pour \'etablir la loi de Mendel on n'a crois\'e que des 
fleurs roses,  parmi lesquelles il y a autant de $BR$ que de $RB$.  On 
retrouve alors forc\'ement autant de g\`enes $B$ que de g\`enes $R$  
dans la population fille.  Mais si on croise des fleurs au hasard,  \`a partir 
d'une population naturelle,  sans tenir compte de la couleur,  et qu'il y a 
moins de rouges que de blanches,  on n'aura plus cette sym\'etrie.     
 
\vskip7mm plus4mm minus3mm  
 
{\bf III. 6. Loi de Hardy-Weinberg.} 
\medskip 
On \'etablit la loi de Hardy-Weinberg par le Calcul des probabilit\'es; 
c'est ce qu'a fait le math\'ematicien Hardy ({\oldstyle 1900}) dont le nom 
est associ\'e. Prenons par exemple le cas de la mucoviscidose (chez 
l'homme). Il s'agit d'une maladie g\'en\'etique, c'est-\`a-dire due \`a la 
pr\'esence d'un g\`ene sp\'ecifique dans le code g\'en\'etique de l'individu 
qui en est  atteint. Cette maladie est fortement invalidante: l'un des 
sympt\^omes, particuli\`erement p\'enible, est l'obstruction permanente 
des bronches par du mucus, entra\^\i nant difficult\'es respiratoires et 
toux permanentes. Il en va de m\^eme dans les canaux excr\'eteurs du 
pancr\'eas, ce qui bloque le passage de l'insuline. La cause en est la  
viscosit\'e excessive des s\'ecr\'etions muqueuses, d'o\`u le nom de la  
maladie. Parmi les individus poss\'edant le g\`ene responsable, seuls ceux 
qui portent la combinaison homozygote sont frapp\'es par la maladie; ceux  
qui poss\`edent le g\`ene sous la forme h\'et\'erozygote sont  sains et de  
ce fait appel\'es {\it porteurs sains}.  
\medskip 
Appelons $M$ le g\`ene fatidique.  Celui-ci peut se trouver combin\'e \`a 
son all\`ele $X$ (cas h\'et\'erozygote $XM$ ou $MX$), ou \`a lui-m\^eme 
(cas homozygote $MM$).  Il importe peu ici de savoir s'il n'y a que deux 
all\`eles ($X$ repr\'esentant alors un g\`ene unique) ou plus ($X$ 
repr\'esentant alors n'importe lequel des autres all\`eles),  puisque les 
g\`enes autres que $M$ n'interviennent pas dans la mucoviscidose.  La 
fr\'equence de la maladie dans la population europ\'eenne est d'environ 
$1/2500$ (un enfant sur $2\, 500$ qui naissent en est atteint).  Comme 
pour les croisements entre deux fleurs rouges de mirabilis, deux parents 
atteints tous les deux de mucoviscidose auraient forc\'ement des   
enfants \'egalement atteints, mais ce cas est \'evidemment rarissime et 
peut \^etre n\'eglig\'e.  Le cas o\`u l'un des deux parents serait atteint, 
quoique moins improbable,  peut aussi \^etre n\'eglig\'e,  d'autant plus que  
la maladie d\'etourne du mariage et de la procr\'eation.  Le cas normal est
celui de parents qui sont tous deux h\'et\'erozygotes.  
\medskip 
Supposons que tous les parents sains s'accouplent au hasard;  cette 
hypoth\`ese signifie concr\`etement que les parents ignorent s'ils sont 
porteurs ou non,  ou que s'ils le savent cela n'a aucune incidence sur leur 
f\'econdit\'e.  Cette hypoth\`ese est appel\'ee la {\it panmixie};  le terme 
anglo-saxon {\it random mating} est cependant plus courant.  Parmi  
ces parents ``pris au hasard'',  il y a autant de p\`eres que de m\`eres; 
soit donc $N$ le nombre de  p\`eres (ou de m\`eres).  Si parmi les $N$ 
p\`eres,  il y a $k$ porteurs sains, la proportion de porteurs sains est  
$x = k/N$.  Le g\`ene $M$ \'etant distribu\'e ind\'ependamment du sexe,   
on aura la m\^eme proportion chez les m\`eres, donc $k$ m\`eres 
h\'et\'erozygotes (environ, car la proportion pr\'esente toujours des 
fluctuations statistiques). La proportion $x$ est  l'inconnue que justement 
nous d\'esirons calculer \`a partir de la fr\'equence connue de la maladie. 
Dans la loi de Mendel le hasard intervenait dans la recombinaison des 
chromosomes.  Ici le hasard intervient en outre dans le choix des parents 
possibles:  on admet que dans la population,  {\it tous les couples de 
parents sont \'equiprobables}.  Le nombre de tous les couples possibles est 
$N \times N$.  Parmi ces couples, ceux dont les membres sont tous deux 
h\'et\'erozygotes $MX$ ou $XM$ sont au nombre de $k \times k$.  La proportion
de couples dont les membres sont tous deux porteurs sains est donc $k^2/n^2 
= x^2$.  Mais  nous devons tenir compte des deux interventions du hasard; 
l'espace des \'epreuves correspondant n'est pas l'ensemble des couples de 
parents,  mais l'ensemble des combinaisons de chromosomes r\'esultant de 
la division des paires parentales:  chaque parent fournit un chromosome de  
la paire,  donc chaque couple peut produire quatre combinaisons de 
chromosomes.  Par cons\'equent l'espace $\Omega$ des \'epreuves est de 
cardinal $4 N^2$ (quatre combinaisons de chromosomes pour chaque couple 
de parents).  Si les parents ne sont pas tous les deux porteurs du g\`ene 
$M$,  {\it aucune} combinaison ne pourra donner le g\'enotype $MM$;  mais 
si les parents sont tous les deux porteurs sains,  une seule des quatre 
combinaisons qu'ils peuvent produire donnera le g\'enotype $MM$:  par 
cons\'equent le nombre de possibilit\'es d'avoir le g\'enotype $MM$ est 
$k^2$ (une seule combinaison pour chaque couple d'h\'et\'erozygotes,  plus 
z\'ero combinaisons pour les autres couples).  La probabilit\'e d'avoir un 
enfant de g\'enotype $MM$ sera donc ${k^2 \over 4N^2} = {x^2 \over 4}$.    
\medskip 
Or on sait par les donn\'ees cliniques que cette  proportion est $1/2500$,  
c'est-\`a-dire que ${x^2 \over 4} = {1 \over 2500}$, d'o\`u $x = 
1/25 = 0.04$. On peut donc d\'eduire de la proportion $1/2500$ de cas 
homozygotes que la proportion de cas h\'et\'erozygotes est $1/25$. Un 
individu sur vingt  cinq est porteur sain.   
\medskip  
Il est ais\'e de trouver la formule g\'en\'erale: si pour un g\`ene 
$G$  quelconque $\varepsilon$ est la proportion d'homozygotes $GG$, la 
proportion d'h\'et\'erozygotes $GX$ ou $XG$ est $x = 2\sqrt{\varepsilon}$, 
car on doit avoir ${x^2 \over 4} = \varepsilon$. Cela r\'esulte directement 
du d\'enombrement: il y a $N$ p\`eres et $N$ m\`eres donc $N^2$ couples 
possibles, qui peuvent donner  chacun quatre combinaisons possibles de 
chromosomes, donc $\#\Omega = 4 N^2$; parmi ceux-ci il y a $k^2$ 
couples qui peuvent donner chacun une combinaison $GG$, donc  
l'\'ev\'enement $A$: ``l'enfant est de g\'enotype $GG$'' a pour cardinal  
$k^2$. La probabilit\'e pour que tous ces couples form\'es au hasard 
engendrent un enfant $GG$ est donc  
$$\varepsilon = {\#A \over \#\Omega�} = {\up{k^2} \over 4 N^2} =  
{\up{x^2} \over 4} \eqno (III.4.)$$ 
Cette formule extr\^emement simple permet de calculer la 
fr\'equence  $\varepsilon$ de la maladie \`a partir de la proportion 
statistique $x$ de porteurs sains.  On en d\'eduit que si par exemple on 
dissuade la moiti\'e des porteurs sains de procr\'eer, on divise par quatre  
le nombre de cas de mucoviscidose. Inversement, on obtiendra la proportion 
inconnue $x$ de porteurs sains \`a partir de la fr\'equence $\varepsilon$  
de la maladie  ($x = 2\sqrt{\varepsilon}$). Rappelons encore qu'il s'agit  
de la fr\'equence dans une population o\`u les parents se choisissent 
``au hasard''. Il est bien \'evident qu'une personne qui parmi ses proches  
(fr\`eres ou cousins) compte d\'ej\`a un cas de mucoviscidose a une  
probabilit\'e bien plus forte que $x$ d'\^etre porteur sain. Deux parents qui 
ont d\'ej\`a mis au monde un enfant atteint sont assur\'ement tous les  
deux porteurs sains et ont donc une chance sur quatre que le prochain 
enfant ait aussi la maladie. 
\medskip 
La fr\'equence moyenne $\varepsilon$ est variable selon les populations: 
$1/2500$ \'etait sa valeur sur l'ensemble de  l'Europe occidentale. Mais  
elle  diff\`ere d'un pays \`a l'autre; \`a l'int\'erieur d'un m\^eme pays elle 
varie selon les r\'egions. En outre, la mucoviscidose est beaucoup  plus 
rare en Asie, par exemple.  Le tableau suivant\ftn{3}{D'apr\`es G.  Lenoir 
{\it La mucoviscidose} \'Ed Doin.} en montre quelques exemples:  
\medskip 
\centerline{\vbox{ 
\halign{#&\quad#\hfill \cr 
France:\quad\dotfill\quad   global &$1/2\, 000$ \cr 
\hfill                                   Finist\`ere Nord &$1/1\, 650$ \cr 
\hfill                                   Morbihan &$1/3\, 500$ \cr 
\noalign{\medskip} 
Royaume Uni:\quad\dotfill\quad   Angleterre &$1/2\, 350$�\cr 
\hfill                                            Pays de Galles &$1/1\,  
650$�\cr 
\noalign{\medskip} 
Italie:\quad\dotfill\quad global &$1/2\, 700$ \cr 
\noalign{\medskip} 
Su\`ede:\quad\dotfill\quad global &$1/5\, 700$ \cr  
\noalign{\medskip} 
Hawaii:\quad\dotfill\quad   
              population de souche europ\'eenne &$1/3\, 800$�\cr 
\hfill     population de souche polyn\'esienne &$1/90\, 000$�\cr }  }  } 
\medskip 
On d\'eduit  alors imm\'ediatement la proportion de porteurs sains dans  
ces  populations: France $1/22$, Su\`ede $1/34$, hawaiiens de souche 
polyn\'esienne $1/150$, etc. Bien entendu le raisonnement suivi pour  
obtenir la formule $III.4$ suppose une population  g\'en\'etiquement  
stable et isol\'ee (endo\-gamique), ainsi que  l'\'equiprobabilit\'e des  
choix de parents (random mating). Ces  hypoth\`eses ne sont que tr\`es 
approximativement v\'erifi\'ees dans les  populations humaines r\'eelles. 
Mais il est possible  de les r\'ealiser en  laboratoire sur des cultures de 
fleurs ou de bact\'eries. En observant la  distribution statistique de 
ph\'enotypes dans de telles cultures artificielles, et en la comparant \`a 
des probabilit\'es a priori calcul\'ees  \`a partir de {\it mod\`eles de 
g\'enotypes}, il devient possible  d'\'etudier scientifiquement l'influence 
des g\`enes sur les ph\'enotypes. Cette m\'ethode est \`a la base de la 
g\'en\'etique.   
\medskip  
Dans le raisonnement suivi plus haut, nous avons cependant n\'eglig\'e   
que la naissance d'un $GG$ peut aussi provenir du croisement d'un $GX$  
avec un $GG$, ou du croisement de deux $GG$.  Cette n\'egligence  
volontaire se justifie, soit parce que le g\'enotype $GG$ est tr\`es 
invalidant (donc rend l'acc\`es \`a la procr\'eation difficile ou  
impossible), soit  tout simplement parce que le g\'enotype $GG$ est si  
rare qu'on peut  statistiquement le n\'egliger. Ces deux conditions  
\'etaient v\'erifi\'ees  pour  la mucoviscidose.  
\medskip 
Si le g\'enotype $GG$ n'est ni rare ni invalidant, il faut tenir compte des 
mariages $GX + GG$ ou $GG + GG$. Supposons toujours que les couples de 
parents sont choisis au hasard. L'\'ev\'enement $A$: ``naissance d'un $GG$'' 
est alors la r\'eunion  des quatre \'ev\'ene\-ments suivants:   
\medskip  
\centerline{\vbox{\hsize=8cm 
$A_1$:\hskip2mm $GX + GX \rightarrow 
GG$\hskip4mm (probabilit\'e  ${x^2\over 4}$); 
\smallskip 
$A_2$:\hskip2mm $GG + GX \rightarrow GG$\hskip4mm (probabilit\'e  
${\varepsilon x\over 2}$);  
\smallskip 
$A_3$:\hskip2mm $GX + GG \rightarrow GG$\hskip4mm (probabilit\'e  
${x\varepsilon\over 2}$);  
\smallskip 
$A_4$:\hskip2mm $GG + GG \rightarrow GG$\hskip4mm (probabilit\'e  
$\varepsilon^2$); } } 
\medskip  
Les probabilit\'es de $A_2$, $A_3$, et $A_4$ se calculent par 
d\'enombrement exactement de la m\^eme fa\c con que pour $A_1$.  
Ces quatre \'ev\'enements \'etant disjoints, on obtient donc  
$${x^2\over 4} + {\varepsilon x\over 2} + {\varepsilon x\over 2} + 
\varepsilon^2 = \varepsilon$$  
de sorte que $x$ est solution de l'\'equation du second degr\'e 
${x^2\over 4} + x\varepsilon + \varepsilon^2 - \varepsilon = 0$, dont la 
seule solution positive est  
$$x = 2\, (\sqrt{\varepsilon} - \varepsilon ) \eqno (III.5.)$$ 
Cette relation est la loi de Hardy-Weinberg. Le raisonnement simplifi\'e 
pr\'ec\'edent donnait $x = 2\sqrt{\varepsilon}$, ce qui est \`a peu pr\`es 
la m\^eme chose si $\varepsilon$ est petit (nous avions pass\'e sous 
silence les \'ev\'enements $A_2$, $A_3$, et $A_4$, qui pour $\varepsilon$ 
petit sont en effet  beaucoup moins probables que $A_1$). 
\medskip  
La loi de Hardy-Weinberg n'est \'evidemment applicable que si les 
hypoth\`eses d'\'equiprobabilit\'e sont satisfaites. Pour d\'enombrer les  
quatre \'ev\'enements $A_1$, $A_2$, $A_3$, et $A_4$, nous avons admis 
que les g\`enes se combinaient uniform\'ement. Cela suppose que les 
parents qui procr\'eent se choisissent ``au hasard'', et qu'en outre, la 
recombinaison des chromosomes est parfaitement uniforme. Il est bien 
clair que les couples ne se forment jamais au hasard, mais le ``hasard 
pur'' n'est exig\'e que pour ce qui concerne les combinaisons des g\`enes 
$X\,$et $G$. Si les ph\'enotypes corres\-pondant  \`a ces combinaisons 
n'ont aucun effet susceptible d'augmenter ou de diminuer l'attirance 
sexuelle, la f\'econdit\'e, etc. tout se passera selon le hasard pur,  
m\^eme  si on peut trouver des d\'eterminismes sociaux ou psychologiques   
\`a la formation des couples. Il est cependant assez \'evident que  
si la combinaison homozygote $GG$ produit des sympt\^omes 
invalidants, ceux-ci auront  une r\'eelle influence (n\'egative) sur la 
f\'econdit\'e, ou aboutiront \`a la mort  de l'individu avant la pubert\'e. 
Seuls les individus porteurs de $XX$ ou de $GX$ seront d\'epourvus du 
caract\`ere invalidant et s'accoupleront au hasard. Cela exclut les 
\'ev\'enements $A_2$, $A_3$,  et $A_4$, et dans ce cas notre 
raisonnement approch\'e devient correct m\^eme si $\varepsilon$ n'est  
pas petit. Toutefois supposer $\varepsilon$ grand signifie que le 
carac\-t\`ere invalidant est fr\'equent, et aucune  esp\`ece ayant subi la 
lutte pour la vie pendant des mill\'enaires ne peut correspondre \`a une 
telle hypoth\`ese.  D'autre part on peut critiquer l'application de ces 
hypoth\`eses \`a l'homme civilis\'e: celui-ci n'est  pas soumis \`a la lutte 
pour la vie; des caract\`eres invalidants peuvent  \^etre suffisamment 
att\'enu\'es par la m\'edecine  pour ne plus d\'etourner de la  
procr\'eation; un caract\`ere qui augmente la f\'econdit\'e peut \^etre  
compens\'e par l'usage de contraceptifs, un caract\`ere qui la diminue  
peut  \^etre combattu par des traitements hormonaux, etc. 
\medskip 
Le Calcul des probabilit\'es joue un r\^ole essentiel en g\'en\'etique; les 
m\'ethodes que nous avons mises en oeuvre pour aboutir aux lois de  
Mendel et de Hardy-Weinberg peuvent \^etre g\'en\'eralis\'ees \`a des 
situations  moins simples. Pour \'etudier en laboratoire les m\'ecanismes 
mol\'eculaires de  l'h\'er\'edit\'e on analyse les prot\'eines: celles-ci  
sont form\'ees d'acides amin\'es qu'on peut isoler par des m\'ethodes 
ad\'equates (chromatographie, etc.) Mais la bio\-logie mol\'eculaire ne 
permet  pas de conna\^\i tre la r\'epartition des g\`enes dans une 
population, ni l'influence d'un g\`ene sur le ph\'enotype d'une plante.  
C'est pourquoi  les g\'en\'eticiens cultivent en laboratoire des 
po\-pu\-la\-tions  enti\`erement artificielles (comme le fit Mendel),   
dans lesquelles  l'observation empirique permet de mesurer la  
r\'epartition statistique  de certains ph\'enotypes. L'\'equiprobabilit\'e  
des recombinaisons chromosomiques et des croisements de semences 
permet d'autre part de calculer a \hbox{priori} des probabilit\'es de  
r\'epartition de g\'enotypes: toute hypoth\`ese sur un  g\'enotype peut  
ainsi \^etre confront\'ee \`a l'observation statistique.    
\medskip 
En ce qui concerne la m\'ethodologie, le point essentiel est le caract\`ere 
quantitatif des pr\'edictions:  toute hypoth\`ese sur un g\'enotype peut 
\^etre test\'ee quantitativement \`a l'aide d'observations statistiques 
sur les ph\'enotypes correspondants. Cette m\'ethode rigoureuse, initi\'ee 
par Gregor Mendel dans le cas le plus simple o\`u le ph\'enotype \'etudi\'e 
est d\'etermin\'e par deux all\`eles seulement, et o\`u le Calcul des 
probabilit\'es joue un r\^ole cl\'e, est  donc le fondement de la 
g\'en\'etique. Pour  que les exp\'eriences soient statistiquement 
pr\'ecises, il faut disposer en laboratoire de populations nombreuses et 
pouvoir les croiser sur un grand nombre de g\'en\'erations; donc les 
esp\`eces \'etudi\'ees doivent \^etre  de petite  taille et se reproduire 
vite; d'o\`u la pr\'ef\'erence pour des esp\`eces particuli\`eres ayant ces 
propri\'et\'es: bact\'erie {\it Escherichia coli}, mouche drosophile, fleur 
{\it Arabidopsis thaliana} (arabette des dames).   
\medskip 
 
 
 
 
 
\bye 
